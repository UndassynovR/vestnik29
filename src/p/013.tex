\id{МРНТИ 68.35.31}{}

\begin{header}
\swa{}{ВЛИЯНИЕ ФЕРМЕНТАТИВНОЙ АКТИВНОСТИ ЗАРОДЫШЕЙ НА ПОСЕВНЫЕ КАЧЕСТВА СЕМЯН НУТА И ГОРОХА, ВОЗДЕЛЫВАЕМЫХ В УСЛОВИЯХ КАЗАХСТАНА}

\tsp{1}М.У. Жонысова\envelope,
\tsp{1}Н. Онгарбаева,
\tsp{2}Б.О. Кулжанова,
\tsp{1}М. Байгайыпкызы
\end{header}

\begin{affil}
Алматинский Технологический университет, Алматы, Казахстан,

ТОО «Научно-Образовательный центр «Qazyna»,Алматы, Казахстан

\corrauthor{Корреспондент- автор: mira\_86u@mail.ru}
\end{affil}

В статье представлены результаты исследования ферментативной активности
зародышей и посевных качеств семян нута (Cicer arietinum L.) и гороха
(Pisum sativum L.), возделываемых в условиях Республики Казахстан.
Изучены показатели активности основных дыхательных ферментов --
дегидрогеназ, каталазы и пероксидазы, а также их влияние на энергию
прорастания и всхожесть семян. Выявлены значительные межвидовые
различия: семена гороха отличались более высокой ферментативной
активностью (98,5 \%), что коррелировало с увеличением энергии
прорастания (r = 0,89). Разработана модель прогнозирования всхожести на
основе ферментативных показателей: Всхожесть = 0,95 × Активность + 5,2
(R² = 0,92).

Полученные результаты подтверждают возможность использования
тетразолиевого теста как экспресс-метода оценки жизнеспособности семян и
представляют практическую ценность для семеноводства и переработки
зернобобовых культур.

{\bfseries Ключевые слова}: нут, горох, ферментативная активность,
дегидрогеназы, энергия прорастания, тетразолиевый тест, посевные
качества, корреляция.

\begin{header}
INFLUENCE OF EMBRYONIC ENZYMATIC ACTIVITY ON THE SEED QUALITY OF CHICKPEA AND PEA GROWN UNDER THE CONDITIONS OF KAZAKHSTAN

\tsp{1}M.U. Zhonyssova\envelope,
\tsp{1}N.Ongarbayeva,
\tsp{2}B.O. Kulzhanova,
\tsp{1}M. Baigaiypkyzy
\end{header}

\begin{affil}
Almaty Technological University, Kazakhstan,

LLP «Scientific and Educational Center «Qazyna», Kazakhstan,

e-mail: mira\_86u@mail.ru
\end{affil}

The article presents the results of a study on the enzymatic activity of
embryos and seed germination qualities of chickpea (Cicer arietinum L.)
and pea (Pisum sativum L.) grown under the agroclimatic conditions of
the Republic of Kazakhstan. The activity of key respiratory enzymes --
dehydrogenases, catalase, and peroxidase -- and their relationship with
germination energy and seed viability were analyzed. Significant
interspecific differences were found: pea seeds exhibited higher
enzymatic activity (98.5\%), which positively correlated with
germination energy (r = 0.89). A predictive model of germination based
on enzymatic indicators was developed: Germination = 0.95 × Activity +
5.2 (R² = 0.92).

The findings confirm the feasibility of using the tetrazolium test as a
rapid method for assessing seed viability and have practical
significance for seed production and processing of legume crops.

{\bfseries Keywords:} chickpea, pea, enzymatic activity, dehydrogenases,
germination energy, tetrazolium test, seed quality, correlation.

\begin{header}
ҚАЗАҚСТАН ЖАҒДАЙЫНДА ӨСІРІЛЕТІН НОҚАТ ПЕН БҰРШАҚ ТҰҚЫМДАРЫНЫҢ ЕГІСТІК ҚАСИЕТТЕРІНЕ ҰРЫҚ ФЕРМЕНТТЕРІНІҢ БЕЛСЕНДІЛІГІ ӘСЕРІ

\tsp{1}М.У. Жонысова\envelope,
\tsp{1}Н. Онгарбаева,
\tsp{2}Б.О. Құлжанова,
\tsp{1}М. Байғайыпқызы
\end{header}

\begin{affil}
Алматы Технологиялық Университеті, Қазақстан,

«Qazyna» Ғылыми-білім беру Орталығы, ЖШС, Қазақстан,

e-mail: mira\_86u@mail.ru
\end{affil}

Мақалада Қазақстан Республикасының жағдайында өсірілетін ноқат (Cicer
arietinum L.) пен бұршақтың (Pisum sativum L.) тұқым ұрықтарының
ферментативтік белсенділігі мен егіндік сапасы зерттеу нәтижелері
келтірілген. Негізгі тыныс алу ферменттерінің -- дегидрогеназа, каталаза
және пероксидазаның белсенділігі мен олардың тұқымның өнгіштік
энергиясына және өну қабілетіне әсері анықталды. Зерттеу нәтижесінде
бұршақ тұқымында ферментативтік белсенділіктің жоғары екені (98,5 \%)
және оның өнгіштік энергиясымен тығыз байланысы бар екені (r = 0,89)
дәлелденді. Ферментативтік көрсеткіштер негізінде өнгіштікті болжау
моделі ұсынылды:

Өнгіштік = 0,95 × Белсенділік + 5,2 (R² = 0,92).

Зерттеу нәтижелері тетразолий сынамасын тұқымның тіршілік қабілетін
жедел бағалау әдісі ретінде қолдануға болатынын көрсетіп, бұршақ
тұқымдас дақылдарды тұқым шаруашылығында және өңдеу кәсіпорындарында
пайдалануға мүмкіндік береді.

{\bfseries Түйін сөздер}: ноқат, бұршақ, ферментативтік белсенділік,
дегидрогеназа, өнгіштік энергиясы, тетразолий сынамасы, егіндік сапа,
корреляция.

\begin{multicols}{2}
{\bfseries Введение.} Зернобобовые культуры играют важную роль в
продовольственном обеспечении населения, обладая высокой питательной
ценностью и способностью к обогащению почвы азотом за счёт
симбиотической фиксации {[}1{]}. Нут (Cicer arietinum L.) и горох (Pisum
sativum L.) являются одними из наиболее перспективных зернобобовых
культур, возделываемых в Казахстане, благодаря своей адаптированности к
засушливым условиям, биохимическому составу и спросу на внутреннем и
внешнем рынках {[}2{]}.

Особый интерес в последние годы представляет использование пророщенных
семян нута и гороха в пищевой промышленности. В результате прорастания в
семенах активизируются ферментативные процессы, повышается содержание
биологически активных веществ, улучшаются органолептические
характеристики и усвояемость питательных компонентов {[}3,4{]}.
Пророщенные семена применяются в производстве продуктов быстрого
приготовления - таких как хлопья, крупы, салаты и функциональные смеси
{[}5{]}. Таким образом, расширение ассортимента на основе проращиваемого
сырья требует научно обоснованного подхода к оценке его качества.

Одним из важнейших показателей жизнеспособности и пищевой ценности
пророщенных семян является ферментативная активность зародышей, в
частности активность дегидрогеназ, участвующих в дыхании и
энергетическом обмене {[}6{]}. Определение этих показателей на ранних
этапах позволяет прогнозировать как посевные качества семян, так и их
пригодность для пищевого использования {[}7{]}. Тем не менее, в условиях
Казахстана комплексные исследования, сочетающие агротехнологическую и
пищевую направленность, остаются ограниченными.

Актуальность исследования обусловлена необходимостью разработки
комплексной методики оценки качества семян нута и гороха, сочетающей
биохимические и физиологические подходы. Это особенно важно в контексте
растущего спроса на высококачественное сырьё для пищевой промышленности,
основанное на использовании пророщенных зернобобовых. Современные
методы, такие как тетразолиевый тест, позволяют оценивать
жизнеспособность и физиологическую полноценность семян более точно и
оперативно, чем традиционные агротехнические показатели.

Целью исследования является комплексная оценка ферментативной активности
зародышей и посевных качеств семян нута и гороха, возделываемых в
условиях Казахстана, с применением современных физиологических и
аналитических методов, включая тетразолиевое тестирование.

В связи с поставленной целью исследования были поставлены следующие
задачи:

- определение активности дегидрогеназ в зародышах семян нута и гороха
методом тетразолиевого теста;

- оценка энергии прорастания и лабораторной всхожести семян;

- сравнительный анализ физиологических показателей семян нута и гороха;

- выявление корреляционных связей между ферментативной активностью
зародышей и посевными качествами семян.

Впервые проведена комплексная оценка ферментативной активности зародышей
нута и гороха, возделываемых в условиях Казахстана, с использованием
тетразолиевого метода в качестве диагностического инструмента. На основе
полученных данных установлены достоверные корреляционные связи между
показателями ферментативной активности и основными посевными
характеристиками, что позволяет прогнозировать посевные качества и
технологический потенциал семян при их использовании как в
сельскохозяйственных, так и в пищевых целях.

Разработанный подход может быть использован в практике семеноводства для
предварительной оценки качества семян перед посевом.

Результаты исследования могут быть применены при отборе сырья для
пищевой промышленности с учётом пригодности семян к проращиванию и
последующему использованию в производстве продуктов функционального
питания.

Повышается эффективность использования местных сортов нута и гороха для
расширения ассортимента продуктов быстрого приготовления и здорового
питания.

\emph{{\bfseries Объект исследования.}} В качестве объектов исследования
использовались семена нута сорта Нұрлы-80 и гороха сорта Ақсары,
выведенные специалистами ТОО «КазНИИЗиР» (Казахский
научно-исследовательский институт земледелия и растениеводства). Посевы
нута размещались в южных регионах Казахстана -- Туркестанской,
Алматинской и Жамбылской областях, тогда как горох возделывался как в
южной, так и в северной зонах страны.

Для проведения экспериментов применялись образцы семян урожая 2023--2024
гг.

Отбор проб и их подготовка осуществлялись в соответствии с требованиями
ГОСТ 13586.3--2015 «Зерно. Правила приёмки и методы отбора проб»
{[}8{]}. Из каждой партии формировали три параллельные навески, на
основании которых готовили средний образец для последующего анализа.

\emph{{\bfseries Материалы и методы. Методы исследования ферментативной
активности}.}

Ферментативную активность зародышей семян нута и гороха оценивали
комплексно, по показателям, характеризующим интенсивность метаболических
процессов.

Активность каталазы, пероксидазы и полифенолоксидазы определяли
стандартными химико-аналитическими методами:

- каталазы - по количеству разложенного пероксида водорода титрованием
0,1 н раствором перманганата калия;

- пероксидазы - спектрофотометрически по интенсивности окрашивания
гуайаколового раствора при 470 нм;

- полифенолоксидазы - колориметрически с пирокатехином при 410 нм.

Дополнительно дыхательную активность зародышей определяли по активности
дегидрогеназ методом тетразолиевого теста с использованием анализатора
\emph{GERMPRO}. Реакцию восстановления 1\% раствора хлорида
трифенилтетразолия регистрировали фотометрически по интенсивности
образования формазана.

Все измерения проводили в трёхкратной повторности, с последующей
статистической обработкой данных методами вариационного анализа.
Полученные результаты использовали для оценки взаимосвязи ферментативной
активности зародышей и посевных качеств семян.

\emph{{\bfseries Методы определения посевных качеств семян.}} Определение
всхожести проводили по методике ГОСТ 12038-84 «Семена
сельскохозяйственных культур. Методы определения всхожести» {[}9{]}. Для
анализа отбирали 100 семян каждого образца, проращивание осуществляли в
лабораторных условиях на фильтровальной бумаге при температуре 20-22 °С.
Учет всхожести проводили на 8-е сутки, результат выражали в процентах от
общего количества посеянных семян. Определение посевных качеств семян
проводили по методике ГОСТ Р 52325--2005. Семена сельскохозяйственных
растений. Сортовые и посевные качества {[}10{]}.

Энергию прорастания устанавливали по количеству нормально проросших
семян через 4 суток после закладки на проращивание, также в процентах.

Жизнеспособность семян определяли по активности дегидрогеназ методом
окрашивания семян раствором 2,3,5-трифенилтетразолия хлорида (ТТХ) в
концентрации 1 \%. После выдерживания в тёмном месте при температуре 30
°С в течение 3 ч оценивали интенсивность окрашивания зародышей, что
позволило судить о метаболической активности и физиологическом состоянии
семян.

Полученные данные использовали для оценки исходных свойств семян и
установления взаимосвязи между посевными качествами и ферментативной
активностью зародышей. Все эксперименты проводились в трёхкратной
повторности с последующей статистической обработкой результатов методами
вариационного анализа.

\emph{{\bfseries Методы статистической обработки данных}.} Обработку
экспериментальных данных проводили с использованием методов вариационной
статистики. Для каждого показателя вычисляли среднее арифметическое
значение (𝑥̄), стандартное отклонение (σ) и стандартную ошибку среднего
(±𝑆𝑥). Достоверность различий между вариантами определяли с применением
критерия Стьюдента (t) при уровне значимости p ≤ 0,05.

Статистическую обработку полученных данных выполняли с использованием
программных пакетов \emph{Microsoft Excel 2019} и \emph{Statistica
10. }0, что обеспечивало высокую точность вычислений и наглядное
представление результатов. Графическое отображение данных (диаграммы,
гистограммы) осуществлялось с помощью программы \emph{OriginPro 2022}.

Все измерения проводились не менее чем в трёхкратной повторности,
результаты представлены в виде среднего значения ± стандартная ошибка
среднего (x̄ ± Sx). Для оценки взаимосвязей между изучаемыми параметрами
использовали корреляционный анализ с расчётом коэффициента корреляции
Пирсона (r), что позволило установить статистически значимые зависимости
между ферментативной активностью зародышей и посевными качествами семян.

\emph{{\bfseries Результаты и обсуждение}. {\bfseries Ферментативная
активность зародышей.}}

Для характеристики интенсивности метаболических процессов в зародыше
была проведена оценка активности дегидрогеназ методом тетразолиевого
теста. Определение выполняли на анализаторе \emph{GERMPRO} с применением
1\%-го раствора хлорида трифенилтетразолия (ТТХ).
\end{multicols}

\begin{figs}[Рис.1- Определение энзиматической активности зародышей семян: \emph{а- нут, б - горох}]
  \fig[0.45\textwidth]{p2/image46}[а]
  \fig[0.45\textwidth]{p2/image47}[б]
\end{figs}

\begin{multicols}{2}
Результаты показали наличие выраженных межвидовых различий (рисунок 1).
У гороха отмечалась высокая активность дегидрогеназ - 98,5 ± 0,5 \%,
тогда как у нута данный показатель составил 85,3 ± 1,2 \%.

Интенсивность окрашивания формазана коррелировала с дыхательной
активностью зародышей. У гороха наблюдалось равномерное и насыщенное
красное окрашивание тканей, соответствующее высокой активности
окислительно-восстановительных ферментов (оптическая плотность -- 0,85 ±
0,03). У нута, напротив, зафиксировано неравномерное окрашивание с менее
выраженной пигментацией (оптическая плотность - 0,63 ± 0,05), что
свидетельствует о более низком уровне ферментативной активности (рисунок
1).

\emph{{\bfseries Посевные качества семян}.} Посевные качества определялись
по показателям энергии прорастания, лабораторной всхожести, количеству
аномальных проростков и загнивших семян. Эти параметры отражают
физиологическое состояние семенного материала и его способность
формировать полноценные проростки. Результаты стандартного тестирования
семян нута и гороха представлены в таблицах 1 и 2, рисунок 2.
\end{multicols}

\tcap{Таблица 1 - Посевные качества семян нута}
\begin{longtblr}[
  label = none,
  entry = none,
]{
  cells = {c},
  cells = {font = \small},
  hlines,
  vlines,
}
\textbf{Показатель}      & \textbf{Образец 1} & \textbf{Образец 2} & \textbf{Образец 3} & \textbf{Среднее} \\
Энергия прорастания, \%  & 91,8               & 93,5               & 92,7               & 92,7±0,9         \\
Всхожесть, \%            & 90,3               & 93,1               & 91,8               & 91,7±1,4         \\
Аномальные проростки, \% & 3,2                & 2,7                & 3,5                & 3,1±0,4          \\
Загнившие семена, \%     & 6,5                & 4,2                & 4,7                & 5,1±1,2          
\end{longtblr}

\tcap{Таблица 2 - Посевные качества семян гороха}
\begin{longtblr}[
  label = none,
  entry = none,
]{
  cells = {c},
  cells = {font = \small},
  hlines,
  vlines,
}
\textbf{Показатель}      & \textbf{Образец 1} & \textbf{Образец 2} & \textbf{Образец 3} & \textbf{Среднее} \\
Энергия прорастания, \%  & 97,5               & 96,8               & 97,2               & 97,2±0,4         \\
Всхожесть, \%            & 96,8               & 95,7               & 97,1               & 96,5±0,7         \\
Аномальные проростки, \% & 1,2                & 2,1                & 1,5                & 1,6±0,5          \\
Загнившие семена, \%     & 1,7                & 2,2                & 1,3                & 1,7±0,5          
\end{longtblr}

\begin{multicols}{2}
Сравнительный анализ показал, что горох характеризуется более высокой
энергией прорастания (97,2 \%) и всхожестью (96,5 \%) по сравнению с
нутом (92,7 \% и 91,7 \%, соответственно). Кроме того, у гороха отмечено
меньшее количество аномальных проростков (1,6 \%) и загнивших семян (1,7
\%) по сравнению с нутом (3,1 \% и 5,1 \% соответственно).

Полученные данные свидетельствуют о более высоком физиологическом
потенциале семян гороха, что, вероятно, связано с большей ферментативной
активностью зародышей и более интенсивным обменом веществ на ранних
этапах прорастания.
\end{multicols}

\begin{figs}
  \fig[0.45\textwidth][0.5\textwidth]{p2/image49}[а]
  \fig[0.45\textwidth][0.5\textwidth]{p2/image48}[б]
\end{figs}

\begin{figs}[Рис.2 - Энергия прорастания \emph{(а - нут, б - горох) и всхожесть семян (в - нут, г- горох)}]
  \fig[0.45\textwidth]{p2/image51}[в]
  \fig[0.45\textwidth]{p2/image50}[г]
\end{figs}

\begin{multicols}{2}
Выявленные различия между нутом и горохом подтверждают наличие тесной
взаимосвязи между ферментативной активностью зародышей и посевными
качествами семян. У гороха более высокая активность дегидрогеназ
коррелирует с повышенной энергией прорастания и всхожестью, что
свидетельствует о физиологической зрелости и высокой жизнеспособности
семян. У нута, напротив, относительно низкий уровень ферментативной
активности, что, вероятно, связано с особенностями строения зародыша и
меньшей интенсивностью дыхательных процессов на ранних стадиях
прорастания.

Таким образом, установлена прямая корреляция между ферментативной
активностью и посевными качествами (r = 0,84, p ≤ 0,05), что позволяет
рассматривать тетразолиевый тест как экспресс-метод предварительной
оценки жизнеспособности семян зернобобовых культур.

Полученные результаты подтверждают наличие прямой зависимости между
активностью дегидрогеназ и посевными качествами семян. Установлено, что
повышение ферментативной активности способствует увеличению энергии
прорастания и лабораторной всхожести. Горох, характеризующийся
наибольшей активностью дегидрогеназ (98,5 \%), демонстрировал также
высокие значения энергии прорастания (98,3 ± 0,4 \%) и всхожести (97,8 ±
0,3 \%). У нута, несмотря на более низкую активность ферментов (85,3
\%), показатели всхожести оставались на удовлетворительном уровне (91,7
\%), что может быть связано с особенностями использования резервных
веществ эндосперма на ранних этапах прорастания.

Характер зависимости между активностью дегидрогеназ и энергией
прорастания описывается линейным уравнением регрессии:

\begin{equation}
y = 12,45 + 0,86x (R^2 = 0,79)
\end{equation}

где \(y\) - энергия прорастания (\%), \(x\) - активность дегидрогеназ (\%).

Высокий коэффициент детерминации (\(R^2\) = 0,79) и коэффициент корреляции (r
= 0,87, p ≤ 0,05) указывают на тесную связь между интенсивностью
ферментативных процессов и физиологическим состоянием семян. Полученные
результаты позволяют рассматривать активность дегидрогеназ как надёжный
биохимический индикатор посевных качеств и физиологической зрелости
зернобобовых культур.

\emph{{\bfseries Выводы.}} Установлены достоверные межвидовые различия в
активности дыхательных ферментов: семена гороха характеризуются более
высокой активностью дегидрогеназ, каталазы и пероксидазы по сравнению с
нутом, что отражает интенсивность их метаболических процессов.

Определена высокая корреляционная зависимость (r = 0,89, p ≤ 0,05) между
интенсивностью окрашивания в тетразолиевом тесте и энергией прорастания
семян, что подтверждает диагностическую значимость данного метода.

Все исследованные образцы нута и гороха продемонстрировали высокие
посевные качества, соответствующие требованиям ГОСТ Р 52325-2005.

Разработана математическая модель прогнозирования всхожести на основе
ферментативной активности: всхожесть = 0,95 × активность + 5,2 (R² =
0,92), которая позволяет оценивать потенциальную энергию прорастания
семян с высокой степенью достоверности.

Результаты исследования подтверждают возможность применения
тетразолиевого теста для экспресс-оценки физиологического состояния
семян и рекомендуются к использованию в семеноводческих хозяйствах и на
предприятиях по переработке зернобобовых культур.

\emph{{\bfseries Финансирования.}~Данная работа была выполнена в рамках
программы грантового финансирования молодых ученых «Жас ғалым»
AP25796105 «Разработка технологии хлопьев из пророщенного зерна гороха и
нута районированных в Казахстане» на 2025--2027 годы при поддержке
Министерства науки и высшего образования Республики Казахстан.}
\end{multicols}

\begin{center}
{\bfseries Литература}
\end{center}

\begin{refs}
1. Khattab R.Y., Arntfield S.D., Nyachoti C.M. Nutritional quality of
legume seeds as affected by some physical treatments. Part 1: Protein
quality evaluation // LWT - Food Science and
Technology.-2009.-Vol.42(6).-P.1107-1112.
\href{https://doi.org/10.1016/j.lwt.2009.02.008}{DOI
10.1016/j.lwt.2009.02.008}.

2. Zatybekov A., Yeshengaliyeva A., Anuarbek S., Kudaibergenov M.,
Turuspekov Y., Abugalieva S. Field Evaluation and Diversity of 238
Global Chickpea (Cicer arietinum L.) Genotypes Grown in South‑East
Kazakhstan//Fundamental and Experimental Biology.-2025.-Vol.3(119).-
P.48-58. DOI 10.31489/2025feb3/48-58.

3. Xu, M., Rao, J., and Chen, B. (2020). Phenolic compounds in
germinated cereal and pulse seeds: classification, transformation, and
metabolic process~//Crit. Rev. Food Sci. Nutr.-2020.-Vol.60 (5).-P
740-759. \href{https://doi.org/10.1080/10408398.2018.1550051}{DOI
10.1080/10408398.2018.1550051}.

4. \href{https://www.semanticscholar.org/author/Ding-Tao-Wu/2152162843}{Ding-Tao
Wu},~\href{https://www.semanticscholar.org/author/Wen-Xing-Li/2221729379}{Wen-Xing
Li}. A Comprehensive Review of Pea (Pisum sativum L.): Chemical
Composition, Processing, Health Benefits, and Food Applications//
Foods.-2023.-Vol.12(13).-
P.1-40\href{https://doi.org/10.3390/foods12132527}{. DOI
10.3390/foods12132527}.

5. Wu Z., Song L., Feng S.,Liu Y.,He G.,YioeY.et al. Germination
dramatically increases

isoflavonoid content and diversity in chickpea (Cicer arietinum
L.)//Agricultural and Food. -2012.-Vol.60(35) - P.8606-8615. DOI
\href{https://pubs.acs.org/doi/10.1021/jf3021514}{10.1021/jf3021514}.

6. \href{https://pubmed.ncbi.nlm.nih.gov/?term=\%22Ray\%20S\%22\%5BAuthor\%5D}{Soham
Ray},~\href{https://pubmed.ncbi.nlm.nih.gov/?term=\%22Vijayan\%20J\%22\%5BAuthor\%5D}{Joshitha
Vijayan},~\href{https://pubmed.ncbi.nlm.nih.gov/?term=\%22Sarkar\%20RK\%22\%5BAuthor\%5D}{Ramani
K Sarkar}. Germination Stage Oxygen Deficiency (GSOD): An Emerging
Stress in the Era of Changing Trends in Climate and Rice Cultivation
Practice//Front. Plant Sci.-2016.-Vol.7:671 DOI
\href{https://doi.org/10.3389/fpls.2016.00671}{10.3389/fpls.2016.00671}.

7. Zhang W., Zhang G., Liang W., Tian J., Sun S., et al. Structure,
functional properties, and applications of foxtail millet prolamin: A
review // Biomolecules.-2024.-Vol.14(8).-P.1-17
\href{https://doi.org/10.3390/biom14080913}{DOI 10.3390/biom14080913}.

8. ГОСТ 13586.3-2015. Зерно. Правила приёмки и методы отбора проб.
//Стандартинформ.- 2016.- Дата обращения: 15.09.2025 г.

9. ГОСТ 12038-84. Семена сельскохозяйственных культур. Методы
определения всхожести//Стандартинформ.-2011.- Дата обращения: 15.09.2025
г.

10. ГОСТ Р 52325-2005. Семена сельскохозяйственных растений. Сортовые и
посевные качества. Общие технические условия//Стандартинформ.-2006.ОКП
9710. - Дата обращения: 15.09.2025 г.
\end{refs}

\begin{center}
{\bfseries References}
\end{center}

\begin{refs}
1. Khattab R.Y., Arntfield S.D., Nyachoti C.M. Nutritional quality of
legume seeds as affected by some physical treatments. Part 1: Protein
quality evaluation // LWT - Food Science and
Technology.-2009.-Vol.42(6).-P.1107-1112.
\href{https://doi.org/10.1016/j.lwt.2009.02.008}{DOI
10.1016/j.lwt.2009.02.008}.

2. Zatybekov A., Yeshengaliyeva A., Anuarbek S., Kudaibergenov M.,
Turuspekov Y., Abugalieva S. Field Evaluation and Diversity of 238
Global Chickpea (Cicer arietinum L.) Genotypes Grown in South‑East
Kazakhstan//Fundamental and Experimental Biology.-2025.-Vol.3(119).-
P.48-58. DOI 10.31489/2025feb3/48-58.

3. Xu, M., Rao, J., and Chen, B. (2020). Phenolic compounds in
germinated cereal and pulse seeds: classification, transformation, and
metabolic process~//Crit. Rev. Food Sci. Nutr.-2020.-Vol.60 (5).-P
740-759. \href{https://doi.org/10.1080/10408398.2018.1550051}{DOI
10.1080/10408398.2018.1550051}.

4. \href{https://www.semanticscholar.org/author/Ding-Tao-Wu/2152162843}{Ding-Tao
Wu},~\href{https://www.semanticscholar.org/author/Wen-Xing-Li/2221729379}{Wen-Xing
Li}. A Comprehensive Review of Pea (Pisum sativum L.): Chemical
Composition, Processing, Health Benefits, and Food Applications//
Foods.-2023.-Vol.12(13).-
P.1-40\href{https://doi.org/10.3390/foods12132527}{. DOI
10.3390/foods12132527}.

5. Wu Z., Song L., Feng S.,Liu Y.,He G.,YioeY.et al. Germination
dramatically increases

isoflavonoid content and diversity in chickpea (Cicer arietinum
L.)//Agricultural and Food. -2012.-Vol.60(35) - P.8606-8615. DOI
\href{https://pubs.acs.org/doi/10.1021/jf3021514}{10.1021/jf3021514}.

6. \href{https://pubmed.ncbi.nlm.nih.gov/?term=\%22Ray\%20S\%22\%5BAuthor\%5D}{Soham
Ray},~\href{https://pubmed.ncbi.nlm.nih.gov/?term=\%22Vijayan\%20J\%22\%5BAuthor\%5D}{Joshitha
Vijayan},~\href{https://pubmed.ncbi.nlm.nih.gov/?term=\%22Sarkar\%20RK\%22\%5BAuthor\%5D}{Ramani
K Sarkar}. Germination Stage Oxygen Deficiency (GSOD): An Emerging
Stress in the Era of Changing Trends in Climate and Rice Cultivation
Practice//Front. Plant Sci.-2016.-Vol.7:671 DOI
\href{https://doi.org/10.3389/fpls.2016.00671}{10.3389/fpls.2016.00671}.

7. Zhang W., Zhang G., Liang W., Tian J., Sun S., et al. Structure,
functional properties, and applications of foxtail millet prolamin: A
review // Biomolecules.-2024.-Vol.14(8).-P.1-17
\href{https://doi.org/10.3390/biom14080913}{DOI 10.3390/biom14080913}

9. GOST 12038-84. Semena sel' skohozjajstvennyh
kul' tur. Metody opredelenija vshozhesti//
Standartinform.-2011.- Data obrashhenija: 15.09.2025 g. {[}in Russian{]}

10. GOST R 52325-2005. Semena sel' skohozjajstvennyh
rastenij. Sortovye i posevnye kachestva. Obshhie tehnicheskie
uslovija//Standartinform.-2006.OKP 9710.- Data obrashhenija: 15.09.2025
g. {[}in Russian{]}
\end{refs}

\begin{info}
\hspace{1em}\emph{{\bfseries Сведения об авторах}}

Жонысова М.У.- Алматинский Технологический университет, докторант,
Алматы, Казахстан,e-mail: mira\_86u@mail.ru;

Онгарбаева Н.- д.т.н., профессор, АО «Алматинский технологический
университет», Алматы, Казахстан, e-mail:o.nurlaim@mail.ru;

Құлжанова Б.О.- PhD, ТОО «Научно-Образовательный центр «Qazyna»,
Алматы, Казахстан, e-mail: botagoz-89@mail.ru;

Байғайыпқызы М.- докторант, АО «Алматинский технологический
университет», Алматы, Казахстан, e-mail: makpal\_atyhanova@mail.ru.

\hspace{1em}\emph{{\bfseries Information about the authors}}

Zhonyssova M.U.- Almaty Technological University, doctoral student,
Almaty, Kazakhstan, e-mail: mira\_86u@mail.ru;

Ongarbayeva N.- doctor of technical sciences, professor, Almaty
Technological University, Almaty, Kazakhstan, e-mail:
e-mail:o.nurlaim@mail.ru;

Kulzhanova B.O.- Phd,LLP «Scientific and Educational Center «Qazyna»,
Almaty Kazakhstan e-mail: botagoz-89@mail.ru;

Baigaiypkyzy M.- doctoral student, Almaty, Kazakhstan, Almaty
Technological University, Kazakhstan, e-mail: e-mail:
makpal\_atyhanova@mail.ru.
\end{info}
