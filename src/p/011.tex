\id{МРНТИ 65.09.01}{}

{\bfseries APPLICATION OF MOLECULAR HYDROGEN IN FOOD PRODUCTION AND
PRESERVATION: CURRENT TRENDS AND FUTURE PROSPECTS (REVIEW)}

{\bfseries A.A.Melissova}{\bfseries ,
A.D.Daiyrbekova}
{\bfseries ,
M.Zh.Nurbekova}{\bfseries ,
A.R.Okassov},

{\bfseries B.T.Bolkenov}
{\bfseries ,
K.S.Bekbayev}
\corrauthor{}
\emph{Shakarim University, Semey, Kazakhstan}

\corrauthor{Corresponding-authors:k\_bekbaev@mail.ru}

The modern food industry faces many challenges, including accelerated
processes of oxidative degradation of lipids and proteins, active
reproduction of pathogenic microflora, limited shelf life and
accumulation of toxic compounds, which necessitates the introduction of
innovative and environmentally friendly technologies to ensure the
quality and safety of products. In this context, molecular hydrogen (H₂)
is considered a promising functional agent due to its high diffusion
capacity and selective antioxidant activity, which allows it to
effectively neutralize the most reactive radicals without interfering
with natural biochemical processes. Studies show that the use of H₂ in
the meat and dairy industries helps to slow down peroxidation processes,
preserve nutritional value and organoleptic characteristics, reduce the
formation of biogenic amines and inhibit the development of microbial
flora, and its role in storage and freezing technologies is expressed in
stabilizing tissue structures, preventing darkening of vegetables and
fruits and reducing vitamin losses. The use of hydrogen in modified gas
environments and biopolymer coatings opens up new opportunities for
extending shelf life and increasing microbiological safety, while
integration into plant-based raw material processing and beverage
production processes helps preserve antioxidant potential, optimize
fermentation, and reduce the formation of undesirable by-products. The
totality of the data obtained confirms that molecular hydrogen is an
environmentally friendly and effective tool for the sustainable
development of food technologies.

{\bfseries Keywords:} molecular hydrogen, food preservation, antioxidant
activity, antimicrobial properties, biogenic amines, meat and dairy
products, hydrogen-rich water (HRW), food safety.

{\bfseries ТАМАҚ ӨНІМДЕРІН ӨНДІРУДЕ ЖӘНЕ САҚТАУДА МОЛЕКУЛАЛЫҚ СУТЕКТІ
ҚОЛДАНУ: ҚАЗІРГІ ЗАМАНҒЫ ҮРДІСТЕР МЕН ПЕРСПЕКТИВАЛАР (ШОЛУ)}

{\bfseries А.А.Мелісова, А.Д.Дайырбекова, М.Ж.Нұрбекова, А.Р.Окасов,
Б.Т.Болкенов, Қ.С.Бекбаев\envelope }

\emph{Шәкәрім университеті, Семей, Қазақстан,}

e-mail: k\_bekbaev@mail.ru

Қазіргі заманғы тамақ өнеркәсібі липидтер мен ақуыздардың тотығу
тозуының жеделдетілген процестерін, патогенді микрофлораның белсенді
көбеюін, сақтау мерзімінің шектелуін және уытты қосылыстардың
жинақталуын қоса алғанда, көптеген сын-қатерлерге тап болады, бұл
өнімнің сапасы мен қауіпсіздігін қамтамасыз ету үшін инновациялық және
экологиялық таза технологияларды енгізу қажеттілігін негіздейді. Бұл
контексте молекулалық сутек (H₂) өзінің жоғары диффузиялық қабілеті мен
селективті антиоксиданттық белсенділігінің арқасында перспективалы
функционалдық агент болып саналады, бұл оған табиғи биохимиялық
процестерге араласпай, ең реакцияға қабілетті радикалдарды тиімді
бейтараптандыруға мүмкіндік береді. Зерттеулер H₂ ет және сүт
өнеркәсібінде пайдалану тотығу процестерін баяулатуға, қоректік
құндылығы мен органолептикалық сипаттамаларын сақтауға, биогендік
аминдердің пайда болуын азайтуға және микробтық флораның дамуын тежеуге
көмектесетінін және оның сақтау және мұздату технологияларындағы рөлі
тіндердің құрылымын тұрақтандырудан, көкөністер мен жемістердің
қараңғылануын болдырмаудан және ысырапты азайтудан көрінеді дәрумендер.
Түрлендірілген газ ортасы мен биополимерлік жабындарда сутекті пайдалану
жарамдылық мерзімін ұзарту және микробиологиялық қауіпсіздікті арттыру
үшін жаңа мүмкіндіктер ашады, ал өсімдік шикізатын қайта өңдеу және
сусындарды өндіру процестеріне интеграциялау антиоксиданттық әлеуетті
сақтауға, ферменттеуді оңтайландыруға және жағымсыз жанама өнімдердің
пайда болуын азайтуға көмектеседі. Алынған деректердің жиынтығы
молекулярлық сутегі тамақ технологияларын тұрақты дамытудың экологиялық
таза және тиімді құралы болып табылатынын растайды.

{\bfseries Түйін сөздер:} молекулалық сутек, тамақ өнімдерін
консервациялау, антиоксиданттық белсенділік, микробқа қарсы қасиеттер,
биогендік аминдер, ет және сүт өнімдері, сутегіге байытылған су (HRW),
тамақ өнімдерінің қауіпсіздігі

{\bfseries ПРИМЕНЕНИЕ МОЛЕКУЛЯРНОГО ВОДОРОДА В ПРОИЗВОДСТВЕ И СОХРАНЕНИИ
ПРОДУКТОВ ПИТАНИЯ: СОВРЕМЕННЫЕ ТЕНДЕНЦИИ И ПЕРСПЕКТИВЫ (ОБЗОР)}

{\bfseries А.А.Мелісова, А.Д.Дайырбекова, М.Ж.Нурбекова, А.Р.Окасов,
Б.Т.Болкенов, К.С.Бекбаев\envelope }

\emph{Шәкәрім Университет, Семей, Казахстан,}

e-mail: k\_bekbaev@mail.ru

Современная пищевая промышленность сталкивается со многими вызовами,
включая ускоренные процессы окислительной деградации липидов и белков,
активное размножение патогенной микрофлоры, ограниченный срок хранения и
накопление токсичных соединений, что обуславливает необходимость
внедрения инновационных и экологически чистых технологий для обеспечения
качества и безопасности продукции. В этом контексте молекулярный водород
(H₂) считается перспективным функциональным агентом благодаря своей
высокой диффузионной способности и селективной антиоксидантной
активности, что позволяет ему эффективно нейтрализовать наиболее
реакционноспособные радикалы, не вмешиваясь в естественные биохимические
процессы. Исследования показывают, что использование H₂ в мясной и
молочной промышленности помогает замедлить процессы перекисного
окисления, сохраняют питательную ценность и органолептические
характеристики, уменьшают образование биогенных аминов и тормозят
развитие микробной флоры, и его роль в технологиях хранения и
замораживания выражается в стабилизации структур тканей, предотвращение
потемнения овощей и фруктов и снижение потерь витаминов. Использование
водорода в модифицированных газовых средах и биополимерных покрытиях
открывает новые возможности для продления срока годности и повышения
микробиологической безопасности, а интеграция в процессы переработки
растительного сырья и производства напитков помогает сохранить
антиоксидантный потенциал, оптимизировать ферментацию и уменьшить
образование нежелательных побочных продуктов. Совокупность полученных
данных подтверждает, что молекулярный водород является экологически
чистым и эффективным инструментом устойчивого развития пищевых
технологий.

{\bfseries Ключевые слова:} молекулярный водород, консервация пищевых
продуктов, антиоксидантная активность, антимикробные свойства, биогенные
амины, мясные и молочные продукты, вода обогащенная водородом (HRW),
безопасность пищевых продуктов.

{\bfseries Introduction.} Modern food production faces many challenges that
directly affect their quality and safety: accelerated oxidation of fats
and proteins, growth of pathogenic microflora, reduction of shelf life
and formation of toxic compounds. All this prompts the search for new,
effective and environmentally friendly preservation methods. In this
context, molecular hydrogen (H₂) is considered as an extremely promising
tool that can extend shelf life, stabilize food matrices and improve
food safety {[}1{]}. Research interest in the use of H\tsb{2}
in food technology is due to its physicochemical properties - small
molecule size and high diffusion capacity, which ensures free
penetration through cell membranes and uniform distribution in tissues,
which allows hydrogen to interfere with the oxidation-reduction balance
and prevent deterioration of the organoleptic and nutritional
characteristics of food {[}2{]}. One of the key features of hydrogen
application is its antioxidant activity. It is highly selective: it
effectively neutralizes only the most reactive and toxic radicals, such
as •OH (hydroxyl radical) and ONOO⁻ (peroxynitrite), without affecting
normal cellular signaling and metabolic processes {[}3{]}.

\fig{p/image38}{}

{\bfseries Fig.1 -- Applications of Molecular Hydrogen in Food Production
and Preservation}

In the meat industry, saturation of products with molecular hydrogen or
the use of hydrogen-containing films helps reduce lipid and protein
peroxidation, slow down the growth of microbial flora and increase
product freshness. In the dairy and fat industry, H₂ helps preserve
organoleptic properties and prevent rancidity, which is confirmed by
studies using butter and margarines as an example {[}4{]}.

Food packaging plays a vital role in food industry by maintaining
product quality, extending shelf life and ensuring food safety. However,
conventional food packaging systems are limited in their ability to
inhibit bacterial growth, prevent oxidative spoilage, and preserve
product freshness {[}5,6{]}. The use of H\tsb{2} in packaging
and storage technologies is also promising. The introduction of
molecular hydrogen into modified gas environments and the development of
hydrogen-containing biopolymer films enhance the barrier properties of
packaging, reducing oxidation and microbial contamination, thereby
increasing the shelf life and safety of products {[}7{]}.

Another important area is the reduction of the formation of biogenic
amines and carcinogenic compounds. For example, saturation of meat raw
materials with hydrogen reduces the accumulation of histamine, tyramine
and other biogenic amines - markers of microbiological spoilage and
potential danger to humans. In addition, the use of hydrogen as a fuel
when frying meat reduces the formation of polycyclic aromatic
hydrocarbons and volatile organic compounds, reducing the carcinogenic
load {[}8{]}.

The topic is actively developing at the international level: in Japan,
South Korea and China, fundamental and applied research is being
conducted on the implementation of hydrogen technologies in food
production, while in Europe and the USA, attention is focused on the
safety and technological integration of H₂ into existing lines {[}9{]}.
Thus, the existing studies allow us to consider molecular hydrogen as a
promising technological factor that simultaneously ensures an increase
in shelf life, microbiological safety and preservation of the sensory
properties of products. Generalization and in-depth analysis of the
areas of application of H₂ in the food industry is an important task of
modern science {[}10{]}.

The objective of this review is to assess the existing and recent
advances in the use of molecular hydrogen in food production and safety,
provide a summary of the main research directions, and determine the
feasibility of its practical implementation in the food production
industry.

1. \emph{Antioxidant properties of molecular hydrogen}

The problem of oxidative processes in food is fundamental: they are the
ones that cause loss of freshness, formation of rancid taste, darkening
of color, appearance of foreign odors and destruction of nutrients. The
most vulnerable are fats and proteins, which undergo peroxidation,
resulting in the formation of toxic by-products. These reactions not
only reduce shelf life, but also directly affect the safety of finished
products. Therefore, the search for new antioxidant solutions that could
be both effective and safe remains one of the priority tasks of modern
nutrition science {[}11{]}.

Antioxidant properties of molecular hydrogen (H₂) are now considered one
of the key factors determining the prospects of its use in the food
industry. The key feature of molecular hydrogen is its selective
antioxidant activity. Unlike vitamins (ascorbic acid, tocopherols) or
synthetic antioxidants (butylhydroxyanisole, butylhydroxytoluene), which
act non-specifically and can interfere with beneficial metabolic
processes, H₂ reacts exclusively with the most aggressive radicals.
These include the hydroxyl radical (•OH) and peroxynitrite (ONOO⁻) -
highly reactive molecules that damage lipids, proteins, nucleic acids
and initiate avalanche-like chain reactions of oxidation. At the same
time, H₂ does not affect the superoxide anion or hydrogen peroxide,
which perform important physiological functions, such as participation
in cellular signaling. This selectivity distinguishes hydrogen favorably
from all known antioxidants and makes it extremely attractive for
applied use {[}12{]}.

The use of H2 in food technology has demonstrated its effectiveness in a
wide range of products. For example, in the meat industry, hydrogen
treatment of raw materials reduced the level of malondialdehyde, a
marker of lipid peroxidation, thereby preserving the taste and smell of
products for longer. In the dairy industry, saturation of butter with
hydrogen prevented rancidity, slowed down texture changes, and preserved
the characteristic aroma. In studies with grain products, it was shown
that treatment of rice with hydrogen nanobubble water improved storage,
reduced the accumulation of volatile substances, and prevented the
formation of unpleasant odors. In vegetables and fruits, the use of H2
allowed the preservation of bright color, reduced darkening, and
preserved the content of vitamins C and E, which are usually quickly
destroyed by oxygen {[}4{]}.

The antioxidant properties of molecular hydrogen also have a pronounced
sanitary and hygienic significance. Lipid oxidation products, such as
aldehydes and ketones, can accumulate in the body and provoke the
development of inflammatory processes and chronic diseases. For the food
industry, this is not only a quality problem, but also an important
safety issue: products with high concentrations of secondary oxidation
products are potentially dangerous for consumers. The use of H₂ can
significantly reduce the level of these compounds in finished products,
minimizing health risks and increasing confidence in products on the
market {[}7{]}.

An important area of research is the study of
the stability of the antioxidant effect of H₂ in various matrices. It
has been shown that in fat systems (oils, margarines), H₂ effectively
prevents rancidity and preserves the texture of the product, and in
protein systems (meat, dairy products), it inhibits the formation of
carbon derivatives and increases resistance to microbiological spoilage.
In beverages, especially functional ones (juices, energy and sports
drinks), the use of hydrogen-rich water (HRW) has shown high efficiency
in preserving the antioxidant potential and prolonging shelf life. These
data expand the boundaries of hydrogen application and make it a
universal tool for different segments of the food industry {[}8{]}.

The complex of antioxidant properties of molecular hydrogen covers
various levels of protection of food matrices: it prevents rancidity of
lipid components, stabilizes the color and texture characteristics of
protein structures and thereby maintains the organoleptic quality of
products {[}13{]}. Considering H₂ as an innovative preservative, we can
talk about its potential to partially replace or supplement existing
antioxidant systems. In the future, this will contribute to the gradual
rejection of excessive use of synthetic additives and the formation of
"clean" and environmentally sustainable food production technologies.

2. \emph{Antimicrobial action of molecular hydrogen}

Microbiological safety of food products remains one of the most pressing
issues in the food industry, since it is the growth of pathogenic and
opportunistic microflora that is the main reason for the reduction of
shelf life, changes in organoleptic characteristics and the emergence of
risks to human health {[}14{]}. The presence of microorganisms such as
Salmonella spp., Listeria monocytogenes, Escherichia coli O157:H7 and
Staphylococcus aureus leads to serious product losses and threatens the
epidemiological situation. Traditional control methods are based on heat
treatment, the addition of preservatives or the use of a modified gas
environment, but their effectiveness is limited: heating destroys
nutrients and vitamins, chemical additives reduce the environmental
friendliness of the product and raise safety concerns, and gas mixtures
do not always prevent the development of resistant forms of
microorganisms. Against this background, molecular hydrogen (H₂) has
emerged as a promising agent capable of providing an antimicrobial
effect without negative consequences for food quality {[}15{]}.

The antimicrobial effect of hydrogen is not associated with direct cell
destruction, but with the regulation of the redox balance and a decrease
in the level of oxidative stress in the environment. This leads to the
suppression of the activity of membrane enzymes of microorganisms and
the inhibition of cell division processes. Unlike antibiotics or
traditional preservatives, H₂ does not disrupt beneficial microbiota and
does not cause the formation of resistant strains, which makes its use
safer and more environmentally friendly {[}16{]}. Experiments have shown
that the use of hydrogen during the storage of meat products slows the
growth of psychrotrophic bacteria and allows to extend their shelf life
by several days. In addition, hydrogen reduces the formation of biogenic
amines, such as histamine, tyramine and cadaverine, which are formed
during the microbiological decomposition of proteins and can cause
serious poisoning in humans {[}4{]}.

Positive results have also been recorded in the dairy industry:
saturation of dairy products with hydrogen inhibits the rapid
development of acid-forming bacteria, which prevents premature
fermentation and souring. When processing butter and margarines,
hydrogen suppressed the growth of microorganisms and thereby slowed down
the spoilage process, preserving the aroma and consistency of fatty
products longer than the standard shelf life {[}8{]}. In seafood,
treatment with hydrogen water reduced the rate of formation of amines
and sulfur compounds, which are the result of protein degradation and
determine the characteristic smell of spoilage. Similar effects were
observed in the processing of vegetables and fruits, where hydrogen
reduced the development of fungal microflora, helped to preserve the
appearance of the fruit and prevented darkening of the surface {[}17{]}.

An additional significance of the antimicrobial action of H2 is the
reduction of the toxicological load of products. When hydrogen was used
as a fuel in the heat treatment of meat, a reduction in the formation of
polycyclic aromatic hydrocarbons and volatile organic compounds with
carcinogenic properties was recorded {[}9{]}. Thus, hydrogen not only
inhibits the growth of microorganisms, but also reduces the formation of
hazardous chemical compounds, which makes it especially valuable for
comprehensive safety.

The totality of available data allows us to conclude that molecular
hydrogen is an effective factor capable of significantly increasing the
microbiological safety of food products. It simultaneously slows down
the growth of pathogenic microflora, prevents the formation of toxic
metabolites and helps preserve the freshness and quality of products.
These properties make H₂ a promising tool for the development of new
storage and processing technologies focused on sustainability,
environmental friendliness and consumer health protection {[}10{]}.

3\emph{. Use of hydrogen in the meat and dairy industries}

The meat and dairy industries are traditionally considered the most
vulnerable segments of the food industry, as their products are prone to
rapid oxidation and microbiological spoilage. The high content of
proteins, lipids and moisture creates a favorable environment for the
development of pathogenic flora and also accelerates oxidation
processes. This is why meat, sausages, milk and dairy products have a
limited shelf life and require the use of preservation technologies.
Traditional methods include heat treatment, pasteurization, vacuum
packaging and the use of preservatives, but they do not always ensure
the preservation of organoleptic properties and often lead to the
destruction of valuable nutrients. Against this background, molecular
hydrogen (H₂) is considered an innovative and environmentally friendly
tool that can provide comprehensive protection against oxidative and
microbiological changes without compromising product quality {[}4{]}.

The use of H2 in the meat industry demonstrates a number of positive
effects. One of the main ones is the suppression of lipid peroxidation,
which is the cause of rancidity and the formation of an unpleasant odor.
Studies have shown that saturating meat raw materials with hydrogen or
storing it in an atmosphere containing H2 reduces the level of
malondialdehyde, a marker of lipid peroxidation, and slows down the
formation of secondary oxidation products. As a result, meat retains its
fresh color, characteristic taste, and elastic texture longer. In
addition, H2 treatment prevents the destruction of proteins and amino
acids, which helps preserve the nutritional value of the product. It is
especially important that the effect is achieved without changing the
organoleptic profile and without the need to use synthetic antioxidants
such as butylhydroxyanisole or butylhydroxytoluene, the use of which is
restricted by law {[}8{]}.

Of additional importance is the effect of H2 on the microbiological
stability of meat products. Experiments have shown that treating chilled
meat with a hydrogen environment slows the growth of psychrotrophic
bacteria, which increases the shelf life by 2-3 days. In sausages, the
introduction of hydrogen reduced the formation of biogenic amines -
histamine, tyramine, and cadaverine - which are formed during the
microbial decomposition of proteins. These compounds not only worsen
organoleptic characteristics, but also pose a serious risk to human
health, causing allergic reactions and food poisoning. Thus, hydrogen
ensures both sensory and sanitary-hygienic stability of meat products
{[}4{]}.

No less impressive results have been obtained in the dairy industry.
Butter and margarine are particularly susceptible to rancidity due to
the high concentration of unsaturated fatty acids, which are easily
oxidized. Saturation of such products with H₂ significantly slows down
their spoilage, prevents the formation of an unpleasant odor and bitter
taste, and also maintains a soft consistency. In milk, hydrogen reduces
the activity of acid-forming bacteria, which prevents rapid souring and
extends shelf life without the use of antibiotics or preservatives.
Moreover, experiments have shown that hydrogen treatment prevents the
destruction of vitamins A and E, which are usually lost as a result of
oxidative processes, which increases the nutritional value of dairy
products {[}9{]}.

The possibility of using H2 in modified gas mixtures for food packaging
attracts particular attention from researchers. The introduction of
hydrogen into the gas composition together with carbon dioxide and
nitrogen provides an additional level of protection against oxidative
processes and microbial contamination. This approach allows for
simultaneously extending the shelf life and preserving the organoleptic
qualities of meat and dairy products. Moreover, the development of
active packaging technologies using hydrogen-containing biopolymer films
opens up new prospects: such materials not only create a barrier for
oxygen and microorganisms but also act as an active preservative
{[}10{]}.

An interesting area is the use of hydrogen as a fuel in the thermal
processing of meat products. It has been shown that the use of hydrogen
for frying meat reduces the formation of polycyclic aromatic
hydrocarbons (PAH) and volatile organic compounds with carcinogenic
properties. At the same time, the product retains its usual sensory
characteristics, such as taste and aroma. This makes the technology
especially valuable for meat processing plants seeking to improve
product safety and reduce health risks for consumers {[}8{]}.

Taken together, the results of numerous studies confirm that molecular
hydrogen is an effective tool for extending shelf life and improving the
quality of meat and dairy products. Its use allows for the simultaneous
solution of problems of antioxidant protection, microbiological
stability and toxicological safety. Unlike traditional methods, H₂
provides a comprehensive effect and is environmentally safe, which makes
its implementation particularly relevant in the context of the global
trend towards ``clean label'' and sustainable production technologies
{[}9{]}.

4. \emph{Use of Hydrogen in Freezing and Storage Technologies}

Freezing and refrigerated storage are basic methods of preserving food
products, but their effectiveness is limited by a number of serious
problems {[}18{]}. At low temperatures, the oxidation of fats and
proteins slows down, but does not stop completely, which leads to a
gradual deterioration in organoleptic characteristics. Meat and fish
darken, lose elasticity and aroma, dairy products acquire a rancid
taste, vegetables and fruits lose color and freshness. An additional
problem is the formation of ice crystals in tissues: when frozen, they
damage cell membranes, disrupt water-holding capacity and worsen the
texture of products after defrosting. These changes not only reduce
consumer properties, but also increase the volume of food waste. Against
this background, molecular hydrogen (H₂) is increasingly considered as
an innovative factor that can improve storage efficiency and minimize
damage that occurs during the freezing process {[}4{]}.

The use of H₂ in storage technologies has demonstrated high efficiency
in the meat and fish industries. Studies have shown that saturation of
meat raw materials with hydrogen before freezing reduces the intensity
of lipid peroxidation, which helps preserve the natural color and aroma
of the meat. Similar results were obtained in the storage of seafood.
Experiments with dried shrimp showed that the use of a modified
atmosphere with H₂ effectively slows down quality deterioration during
accelerated storage, extending shelf life without the use of synthetic
preservatives. At the same time, the products retained their
characteristic taste and texture, which is especially important for
premium products aimed at export {[}19{]}.

The effectiveness of hydrogen has also been confirmed in the storage of
plant products. Fruits and vegetables are very sensitive to oxidative
processes and enzymatic darkening, which is associated with the activity
of polyphenol oxidase. When processing strawberries with a
hydrogen-containing packaging atmosphere, it was possible to
significantly slow down the loss of vitamins, prevent darkening, and
preserve the fresh aroma of the berries throughout the entire storage
period. Such results demonstrate the potential of H₂ as an
environmentally friendly tool for the transportation and sale of
perishable plant products {[}20{]}.

Additional studies on fresh-cut fruits showed that hydrogen can protect
products even under conditions of high tissue metabolism. Thus, when
storing apples in an atmosphere with H₂, there was a slowdown in
browning, an improvement in texture, and the preservation of antioxidant
activity. Hydrogen not only extended the shelf life, but also preserved
the functional properties of the product, making it more useful for
consumers. These results are especially important for the fresh-cut
sector, where the shelf life is usually limited to a few days {[}21{]}.

The use of H₂ in the dairy industry has also yielded convincing results.
In experiments with butter and margarine, it was shown that hydrogen
saturation slows the rate of rancidity, reduces the accumulation of
biogenic amines, and allows for an extension of shelf life to 30 days
without changing sensory characteristics. The products retained a soft
consistency and natural taste, making H₂ an effective tool for
increasing the stability of fatty products. In fresh milk and fermented
milk drinks, H₂ slowed the development of acid-forming bacteria,
prevented premature souring, and increased stability during storage
under refrigerated conditions {[}22{]}.

Particular attention is paid to the combined use of H₂ with modern
methods of active packaging. Hydrogen can be included in modified gas
mixtures (MAP), together with nitrogen and carbon dioxide, providing
multi-level protection: from oxidative processes, microbial
contamination and enzymatic changes. Experiments have shown that MAP
with H₂ allows preserving the freshness of vegetables, fruits and
seafood much longer than traditional gas mixtures. In addition, the
development of hydrogen-containing biopolymer films opens up new
possibilities in the field of "smart packaging", where the material not
only isolates the product from the external environment, but also
actively maintains its quality during storage {[}23{]}.

The body of evidence suggests that hydrogen could become a key factor in
storage and freezing technologies. It reduces the rate of oxidation
processes, stabilizes tissue structure, prolongs freshness and increases
product safety. In the future, hydrogen storage technologies could play
an important role in the sustainable development strategy of the food
industry, reducing food losses and minimizing the use of chemical
additives.

5. \emph{Hydrogen and Packaging Technologies}

Modern food packaging technologies are rapidly evolving, moving from a
simple barrier approach to the concept of active and ``smart''
packaging. Today, packaging materials must not only protect the product
from oxygen, moisture, light and microorganisms, but also actively
influence the preservation of its quality, nutritional and sensory
characteristics. In this context, molecular hydrogen (H₂) is a promising
tool that can be integrated into various packaging solutions due to its
unique antioxidant and antimicrobial properties, as well as high
diffusion capacity {[}23{]}.

One of the most studied areas is modified atmosphere packaging (MAP),
where hydrogen is used in combination with traditional gases -- nitrogen
and carbon dioxide. Adding H₂ to MAP can significantly extend the shelf
life of perishable products. Studies on strawberries have shown that
hydrogen reduces the activity of enzymes responsible for browning, slows
down the loss of antioxidants and preserves the freshness, taste and
aroma of berries {[}20{]}. Similar results were obtained for fresh-cut
apples, where H₂ prevented oxidative browning, preserved vitamin C and
improved the texture of the product throughout the entire storage period
{[}21{]}.

Promising data have also been obtained for seafood. Experiments with
dried shrimp demonstrated that MAP with hydrogen reduced the rate of
lipid oxidation and rancid odor formation, significantly increasing
shelf life under accelerated testing conditions {[}19{]}. These results
highlight the versatility of H₂, applicable not only to fruits and
vegetables, but also to protein products with a high degree of
susceptibility to oxidative spoilage.

Much attention is paid to the development of active biopolymer films
enriched with molecular hydrogen. Such materials work as ``smart
packaging'': they create a physical barrier and simultaneously ensure
product stabilization due to antioxidant and antimicrobial action.
Biopolymer coatings with H₂ are especially in demand for meat and dairy
products, where maximum slowdown of microbiological activity and
reduction of the rate of oxidative processes are required. This approach
opens up opportunities for abandoning synthetic preservatives and meets
modern trends for ``clean label'' {[}19{]}.

The use of micro- and nanobubble water enriched with hydrogen (HRW)
deserves special attention. Nanobubbles have high stability and a large
specific surface area, which ensures the gradual release of H₂ in the
packaging environment. This allows for a long-term maintenance of the
antioxidant and antimicrobial effect, reducing microbial contamination
and preserving the nutritional value of products. Studies have shown
that HRW slows down the loss of texture and nutrients in fruits and
vegetables during storage {[}2{]}. Moreover, HRW can be used in
combination with MAP and active coatings, creating a multi-level
protection system.

Importantly, hydrogen packaging technologies combine efficiency and
environmental friendliness. They reduce the use of synthetic additives,
extend the shelf life of products, reduce food losses and increase
consumer confidence in natural products. In addition, due to the
flexibility of application, H₂ can be used for different product
categories - from fresh fruits and vegetables to meat, dairy products
and seafood. All this makes molecular hydrogen a powerful tool for the
transition to sustainable and innovative storage technologies {[}10{]}.

6. \emph{Hydrogen in the processing of plant foods and beverages}

Processing of plant raw materials and beverages is one of the most
dynamically developing sectors of the food industry. The key tasks here
are the preservation of nutrients, aromatic compounds, color and texture
of products, as well as ensuring their microbiological stability during
long-term storage. In recent years, molecular hydrogen (H₂) has been
actively considered as an innovative agent capable of significantly
improving the quality of plant foods and beverages due to its
antioxidant and antimicrobial action, as well as participation in
metabolic stabilization {[}10{]}.

One of the most significant areas is the use of H2 to stabilize juices,
nectars, and functional beverages. Experiments have shown that enriching
fruit juices with molecular hydrogen slows down the degradation of
vitamin C and polyphenols, which are important biologically active
components. At the same time, beverages retain their fresh taste and
aroma longer, and their antioxidant potential remains high even after
long-term storage {[}22{]}. Such results allow us to consider hydrogen
as a promising alternative or supplement to traditional preservation
methods.

Much attention is also paid to preserving the quality of freshly cut
vegetables and greens. Hydrogen-rich water (HRW) treatment reduces the
rate of darkening of leafy greens, cucumbers and tomatoes, reduces
moisture loss and prevents the accumulation of toxic metabolites. At the
same time, the cellular structure of tissues is preserved, which is
directly related to the prolongation of freshness and an increase in the
shelf life of products. In the conditions of the fresh-cut market, where
visual appeal and nutritional value are critically important, such
approaches have high practical potential {[}9{]}.

Promising research is also being conducted in the field of storing
grapes, berries and citrus fruits. Hydrogen exhibits the ability to slow
down the processes of glycolysis and enzymatic darkening, reduce the
activity of polyphenol oxidase and catalases, which has a positive
effect on the preservation of color and aroma. In particular, when
storing grapes, the use of HRW and modified media with H₂ made it
possible to reduce weight loss and preserve antioxidant properties
throughout the entire storage period {[}2{]}.

An equally important area is winemaking and the production of fermented
beverages. Hydrogen enrichment of wort or intermediate semi-finished
products helps reduce oxidative reactions that often worsen the aromatic
profile of wine and beer. It has also been shown that H₂ can modulate
yeast activity, increase fermentation efficiency while reducing the
accumulation of unwanted by-products {[}17{]}. These data open up new
prospects for the use of H₂ in the high-value-added beverage industry.

The totality of available studies confirms that the use of molecular
hydrogen in the processing of plant-based raw materials and beverages
can become a breakthrough direction, combining the tasks of maintaining
quality, increasing shelf life and forming functional value. In the long
term, such technologies will reduce food losses, increase the
competitiveness of products and bring to market new categories of
beverages and plant-based products with improved characteristics
{[}10{]}.

{\bfseries Conclusion.} Molecular hydrogen (H₂) has evolved in recent years
from an object of fundamental medical research into a promising tool for
the food industry. The unique properties of H₂, including selective
antioxidant activity, the ability to reduce microbiological load and a
positive effect on the preservation of organoleptic characteristics,
make it a universal agent applicable in a wide range of food processing
and storage segments.

The data reviewed show that hydrogen can be effectively used to extend
the shelf life of meat, dairy and plant products, as well as to improve
the stability of beverages. It demonstrates advantages in modified
atmosphere packaging (MAP), in active biopolymer films and in
technologies using hydrogen micro- and nanobubble water. Such approaches
can significantly reduce the use of chemical preservatives, improve
product quality and safety, and increase consumer confidence in clean
label products.

Particular attention should be paid to the potential of hydrogen in the
beverage and fresh-produce industries: the use of H₂ helps slow down
browning, preserve texture, taste and antioxidant properties of
products. In the long term, these technologies can make a significant
contribution to reducing food losses and increasing the sustainability
of the agro-industrial sector.

Despite the positive results, it should be noted that the field of
application of molecular hydrogen in the food industry is at the stage
of active development. For full implementation, large-scale research
into the mechanisms of its action, standardization of technologies and
assessment of economic feasibility are required. An important area will
also be the development of a regulatory framework defining the
permissible conditions and methods of using hydrogen in food
technologies.

Molecular hydrogen is thus an innovative and environmentally friendly
tool that can change the approach to ensuring the quality and safety of
food products. Its implementation in the practice of processing, storage
and packaging opens up new horizons for the creation of sustainable and
functional solutions in the food industry of the future.

\emph{{\bfseries Funding:} This research was funded by the Committee of
Science of the Ministry of Science and Higher Education of the Republic
of Kazakhstan (grant No. BR24992914).}

{\bfseries References}

1. Alwazeer D. et al. Molecular hydrogen: a sustainable strategy for
agricultural and food production challenges // Frontiers in Food Science
and Technology. - 2024. - Vol.4:1448148. DOI 10.3389/frfst.2024.1448148.

2. Cai C. et al. Molecular Hydrogen Improves Rice Storage Quality via
Alleviating Lipid Deterioration and Maintaining Nutritional
Values//Plants.-2022.-Vol.11(19):2588. DOI 10.3390/plants11192588.

3. Yıldız F., LeBaron T.W., Alwazeer D. A comprehensive review of
molecular hydrogen as a novel nutrition therapy in relieving oxidative
stress and diseases: Mechanisms and perspectives // Biochemistry and
Biophysics Reports.- 2025.-Vol.41:101933. DOI
10.1016/j.bbrep.2025.101933.

4. Alwazeer D., Engin T. Moleküler Hidrojenin Gıda Teknolojilerinde
Kullanımı // Turkish Journal of Agriculture - Food Science and
Technology.- 2022.- Vol.10(7).-P.1205-1213. DOI
10.24925/turjaf.v10i7.1205-1213.5100.

5. Zhang Z., Xu G., Hu S. A comprehensive review on the recent
technological advancements in the processing, safety, and quality
control of Ready-to-Eat meals // Processes.- 2025.-Vol.13(3): 901. DOI
10.3390/pr13030901.

6. Yan M.R., Hsieh S., Ricacho N. Innovative Food Packaging, Food
Quality and Safety, and Consumer Perspectives //
Processes.-2022.-Vol.10(4):747 DOI 10.3390/pr10040664.

7. Akbar U. et al. Hydrogen assisted technologies in food processing,
preservation and safety: A comprehensive review // Food Control.-
2025. -Vol.178:111535. DOI 10.1016/j.foodcont.2025.111535.

8. Russell G., Nenov A., Hancock J.T. How hydrogen (H₂) can support food
security: from farm to fork // Applied Sciences.- 2024. - Vol.14(7):
2877. DOI 10.3390/app14072877.

9. Hancock J.T., Russell G., Stratakos A.C. Molecular Hydrogen: The
Postharvest Use in Fruits, Vegetables and the Floriculture Industry //
Applied Sciences. - 2022. - Vol.12(20):10448. DOI
\href{https://doi.org/10.3390/app122010448?urlappend=\%3Futm_source\%3Dresearchgate.net\%26medium\%3Darticle}{10.3390/app122010448}

10. Alwazeer D. et al. Editorial:Recent knowledge on the applications of
molecular hydrogen in plant physiology, crop production, and food
processing // Frontiers in Food Science and Technology. - 2024. - Vol.
4: 1501046. DOI 10.3389/frfst.2024.1501046.

11. Ohsawa I. et al. Hydrogen acts as a therapeutic antioxidant by
selectively reducing cytotoxic oxygen radicals // Nature Medicine. -
2007. - Vol.13.- P.688-694. DOI 10.1038/nm1577.

12. Hirayama M. et al. Inhalation of hydrogen gas elevates urinary
8-hydroxy-2′-deoxyguanine in Parkinson's disease // Medical Gas
Research. - 2019.-Vol.8(4).- P.152-158. DOI 10.4103/2045-9912.248264.

13. Akbar U. et al. Hydrogen assisted technologies in food processing,
preservation and safety: A comprehensive review//Food
Control.-2025.-Vol.178:111535. DOI 10.1016/j.foodcont.2025.111535.

14. Alegbeleye O. et al. Microbial spoilage of vegetables, fruits and
cereals // Applied Food Research// - 2022. - Vol.2(1): 100122. DOI
10.1016/j.afres.2022.100022.

15. Ohta S. Molecular hydrogen as a preventive and therapeutic medical
gas: Initiation, development and potential of hydrogen medicine //
Pharmacology and Therapeutics.- 2014. -Vol.144(1).- P.1-11. DOI
10.1016/j.pharmthera.2014.04.006.

16. Ichihara M. et al. Beneficial biological effects and the underlying
mechanisms of molecular hydrogen: Comprehensive review of 321 original
articles // Medical Gas Research. - 2015. - Vol.5(1): 12. DOI
10.1186/s13618-015-0035-1.

17. Alwazeer D. et al. Molecular hydrogen: a sustainable strategy for
agricultural and food production challenges // Frontiers in Food Science
and Technology.-2024.-Vol.4:1448148. DOI 10.3389/frfst.2024.1448148.

18. Lisboa H.M., Pasquali M.B., Dos Anjos A.I. Innovative and
sustainable food preservation techniques: enhancing food quality,
safety, and environmental sustainability // Sustainability. - 2024. --
Vol.16(18):8223. DOI 10.3390/su16188223.

19. Jiang K. et al. Hydrogen-based modified atmosphere packaging delays
the deterioration of dried shrimp (Fenneropenaeus chinensis) during
accelerated storage // Food Control.-2023.-Vol.152: 109897. DOI
10.1016/j.foodcont.2023.109897.

20. Lei T., Qian J., Yin C. Equilibrium modified atmosphere packaging on
postharvest quality and antioxidant activity of strawberry //
International Journal of Food Science \& Technology. - 2022. - Vol.
57(11): 6981-6990. DOI 10.1111/ijfs.16052.

21. Jin S. et al. Fresh apple slice preservation driven by molecular
hydrogen // Food Chemistry. - 2025. - Vol.488: 144886. DOI
10.1016/j.foodchem.2025.144886.

22. Alwazeer D. et al. Hydrogen incorporation into butter improves its
microbial and chemical stability, biogenic amine safety, quality
attributes, and shelf-life // LWT. - 2024.- Vol.206: 116550. DOI
10.1016/j.lwt.2024.116550.

23. Alwazeer D. et al. Recent knowledge on the applications of molecular
hydrogen in plant physiology, crop production, and food processing //
Frontiers in Food Science and Technology. - 2024. - Vol.4: 1501046.
DOI: 10.3389/frfst.2024.1501046.

\emph{{\bfseries Information about the authors}}

Melissova A. - PhD student, Shakarim University, Semey, Kazakhstan,
e-mail: aknurmelissova@gmail.com;

Daiyrbekova A. - Master student, Shakarim University, Semey, Kazakhstan,
e-mail: a.dayirbekova\_03@mail.ru:

Nurbekova M. - Master student, Shakarim University, Semey, Kazakhstan,
e-mail:
Zhanbekkyzy90@bk.ru;

Bolkenov B. - PhD, Shakarim University, Semey, Kazakhstan, e-mail:
b.bolkenov@shakarim.kz;

Bekbayev K. - Candidate of Technical Sciences, Associate Professor;
Shakarim University, Semey, Kazakhstan, e-mail: k\_bekbaev@mail.ru;

Okassov A.- PhD student, Shakarim University, Semey, Kazakhstan, e-mail:
oar20@mail.ru.

\emph{{\bfseries Сведения об авторах}}

Мелісова A.A.-докторант, Шәкәрім Университет, Семей, Казахстан, e-mail:
aknurmelissova@gmail.com;

Дайырбекова A.Д. -- магистрант, Шәкәрім Университет, Семей, Казахстан,
e-mail: a.dayirbekova\_03@mail.ru;

Нұрбекова М. Ж.- магистрант, Шәкәрім Университет, Семей, Казахстан,
e-mail: Zhanbekkyzy90@bk.ru;

Болкенов Б. Т. - PhD, Шәкәрім Университет, Семей, Казахстан, e-mail:
b.bolkenov@shakarim.kz;

Бекбаев К. С. - кандидат технических наук, ассоциированный профессор,
Шәкәрім Университет, Семей, Казахстан, e-mail: k\_bekbaev@mail.ru;

Окасов А.Р. - докторант, Шәкәрім Университет, Семей, Казахстан, e-mail:
oar20@mail.ru.\
