\id{IRSTI 65.59.03}{}

{\bfseries EXPERIMENTAL STUDIES OF THE GASTROINTESTINAL SYSTEM --}

{\bfseries QARTA OF THE HORSE}

{\bfseries \tsp{1}A.T. Kostanova}
{\bfseries \envelope 
, \tsp{1}Sh.B. Baytukenova}
{\bfseries ,
\tsp{2}S.B. Baytukenova}


\emph{\tsp{1}NCJSC «S.Seifullin Kazakh Agro Technical
Research University», Astana, Kazakhstan,}

\emph{\tsp{2}JSC « K.Kulazhanov Kazakh University of
Technology»,Astana, Kazakhstan}

\texorpdfstring{\envelope Corresponding
author:
\href{mailto:\%20anel_kostanova@mail.ru}{anel\_kostanova@mail.ru}
}{\envelope Corresponding author: anel\_kostanova@mail.ru }

The present study provides a comprehensive investigation into the
biochemical, nutritional, and physicochemical properties of qarta, the
terminal segment of the horse's gastrointestinal tract, traditionally
consumed in Kazakh cuisine. Despite its cultural importance and
increasing market interest, little is known about the compositional
profile of this intestinal material. Boiled qarta samples were analyzed
for mass fraction of moisture, fat, protein, ash, pH, water activity
(a\tsb{w}), and functional properties such as water-holding,
water-binding, and fat-hiding capacities. Furthermore, cholesterol
levels, free amino acid profiles, and fatty acid composition were
determined using standardized GOST protocols and modern analytical
equipment.

The results revealed that qarta is characterized by a low cholesterol
content (29.9±0.72 mg/100g), a high proportion of unsaturated fatty
acids (MUFA - 42.5\%, PUFA -- 17.7\%), and a substantial amount of free
and essential amino acids (total 41.3\%). Functional parameters such as
water and fat-binding capacities remained high after boiling, indicating
technological suitability for meat processing.

These findings highlight qarta as a valuable source of high-quality
protein and lipids, with potential applications in health-oriented and
functional food products. The study contributes new insights into the
valorization of underutilized meat by-products and supports the
development of novel formulations basedon equine tissues.

{\bfseries Key words:} horse meat, qarta, gastrointestinal system,
nutritional composition.

{\bfseries ЭКСПЕРИМЕНТАЛЬНОЕ ИССЛЕДОВАНИЕ ЖЕЛУДОЧНО-КИШЕЧНОГО ТРАКТА}

{\bfseries ЛОШАДИ ҚАРТА}

{\bfseries \tsp{1}А.Т. Костанова\envelope ,
\tsp{1}Ш.Б. Байтукенова, \tsp{2}С.Б.
Байтукенова}

\emph{\tsp{1}НАО «Казахский агротехнический
исследовательский университет им. С.Сейфуллина», Астана, Казахстан,}

\emph{\tsp{2}АО «Казахский университет технологий и бизнеса
им. К.Кулажанова», Астана, Казахстан,}

\href{mailto:e-mail:\%20anel_kostanova@mail.ru}{e-mail:
anel\_kostanova@mail.ru}

В настоящем исследовании дана всесторонняя характеристика химического
состава, пищевой и физико-химической ценности қарта -- терминального
отдела желудочно-кишечного тракта лошади, традиционно употребляемого в
казахской кухне. Несмотря на культурное значение и растущий интерес со
стороны рынка, состав этого субпродукта остается малоизученным. Образцы
қарта были проанализированы по показателям массовой доли влаги,
содержании жира, белка, золы, pH, водной активности
(a\tsb{w}), а также по функциональным свойствам -
влагоудерживающей, водосвязывающей и жироудерживающей способности. Кроме
того, были определены уровни холестерина, профиль свободных аминокислот
и состав жирных кислот с применением стандартов ГОСТ и современных
аналитических методов.

Результаты показали, что қарта характеризуется низким содержанием
холестерина (29.9±0.72 мг/100г), высоким содержанием ненасыщенных жирных
кислот (МНЖК - 42.5\%, ПНЖК - 17.7\%) и значительным уровнем свободных и
незаменимых аминокислот (всего 41.3\%). Функциональные свойства
сохранялись после термообработки, что указывает на технологическую
пригодность қарта для переработки в мясной промышленности.

Полученные данные позволяют рассматривать қарта как ценный источник
полноценного белка и липидов с высоким потенциалом для производства
деликатесных мясных продуктов питания.

{\bfseries Ключевые слова:} конина, қарта, желудочно-кишечный тракт,
пищевая ценность.

{\bfseries ЖЫЛҚЫНЫҢ АСҚАЗАН-ІШЕК ЖОЛЫ - ҚАРТАНЫ ЭКСПЕРИМЕНТТІК ЗЕРТТЕУ}

{\bfseries \tsp{1}А.Т. Костанова\envelope ,
\tsp{1}Ш.Б. Байтукенова, \tsp{2}С.Б.
Байтукенова}

\emph{\tsp{1}«С.Сейфуллин атындағы Қазақ агротехникалық
зерттеу университеті» КеАҚ, Астана, Қазақстан,}

\emph{\tsp{2}«Қ. Құлажанов атындағы Қазақ технология және
бизнес университеті» АҚ, Астана, Қазақстан,}

\href{mailto:e-mail:\%20anel_kostanova@mail.ru}{e-mail:
anel\_kostanova@mail.ru}

Бұл зерттеу жұмысы дәстүрлі қазақ деликатесі саналатаын қарта өнімінің -
жылқының тоқ ішек бөлігінің - химиялық, тағамдық және физика-химилық
қасиеттерін кешенді түрде сипаттайды. Мәдени маңыздылығына қарамастан
және нарықта қызығушылықтың артуына қарамастан, бұл өнім түрінің құрамы
жеткілікті дәрежеде зерттелмеген. Қарта үлгілері ылғалдың массалық
үлесі, май, ақуыз, күл, pH, су белсенділігі (a\tsb{w}),
сонымен қатар технологиялық қасиеттері - ылғал-май ұстағыштық пен су
байланыстыру қабілеті бойынша зерттелді. Сонымен қатар холестерин
мөлшері, бос аминқышқылдары мен май қышқылдарының құрамы анықталды (СТ
ҚР стандарттары және заманауи аналитикалық әдістерімен).

Зерттеу нәтижелері бойынша қарта төмен холестерин құрамымен (29.9±0.72
мг/100г), жоғары моно- және полиқанықпаған май қышқылдарымен (МНМҚ -
42.5\%, ПНМҚ - 17.7\%) және едәір мөлшердегі бос және алмастырылмайтын
аминқышқылдарымен (41.3\%) ерекшеленетіні анықталды. Термоөңдеуден кейін
де технологиялық қасиеттері жоғары деңгейде сақталған.

Осы нәтижелер қарта өнімін ақуыздар мен майлардың толыққанды көзі
ретінде және денсаулыққа пайдалы функционалды өнімдерге арналған шикізат
ретінде қарастыруға мүмкіндік береді.

{\bfseries Түйін сөздер:} жылқы еті, қарта, ішек-қарын жүйесі, тағамдық
құндылық.

{\bfseries Introduction.} Horses are unique hindgut fermenters, utilizing a
well-developed caecum and colon for microbial digestion of structural
carbohydrates and absorption of volatile fatty acids. Despite widespread
knowledge of their digestive anatomy, the biochemical composition of
equine intestinal tissues, particularly the final segment (rectum, known
as qarta in Kazakh culture), remains underexplored. While muscle,
adipose, and milk tissues of horses have been frequently characterized
for amino and fatty acid profiles {[}1{]}, little is known about the
free amino acids, lipid fractions, and physicochemical properties of
qarta.

In Kazakhstan and other Central Asian countries, growing attention has
been paid to the valorization of traditional meat by-products as sources
of high value nutrients and functional materials. The rational use of
edible offal contributes to sustainable meat processing and the creation
of functional semi-finished products rich in bioactive compounds.
Similar studies emphasize the importance of domestic research in
developing technologies for the efficient utilization of by-products
within national meat industries {[}2{]}.

Building upon this background of sustainable processing and traditional
product valorization, several domestic studies have focused on improving
the quality and sustainability of national meat products. Recent works
have demonstrated the development of new-generation Kazakh horse-meat
products enriched with natural antioxidants such as goji extract and
buckwheat flour, which enhanced oxidative stability and product quality
{[}3{]}.

In Central Asian cuisine, qarta has long been consumed for its unique
texture and rich nutritional content. Preliminary evidence suggests that
equine intestines may contain high levels of unsaturated fatty acids and
free amino acids beneficial for health and digestion (e.g., glutamic
acid, taurine, oleic and linoleic acids).

The chemical composition of horse gastrointestinal tissues - measured by
their content of water, fat, protein, minerals, carbohydrates, and other
substances - offers essential insight into the nutritional quality of
horse meat. These intestinal by-products are increasingly valued in the
market due to their distinctive levels of proteins, fats, and glycogen.
The high concentration of essential amino acids, balanced fatty acid
profiles, and the presence of both macro- and micronutrients make horse
meat and its gastrointestinal components especially suitable for
individuals with higher nutritional requirements.

Horse meat is particularly digestible due to its high protein, vitamin
B12, and iron content, coupled with a low fat (approximately 3\%) and
cholesterol level {[}4{]}. Meat quality is a comprehensive measure that
includes nutritional, sensory, hygienic, technological, and processing
characteristics {[}5{]}. Compared to ruminant or pig meat, horse meat
and its intestinal parts are richer in water, proteins, glycogen, iron,
and water-soluble vitamins, while containing lower levels of lipids and
fat-soluble vitamins, giving them distinct dietary advantages {[}6{]}.

Regional studies conducted in Kazakhstan confirm that horse meat and its
by-products possess a favorable amino acid balance, high digestibility,
and good safety indicators under local production conditions. These
findings reinforce the nutritional benefits of horse-derived products
for functional and health-oriented food applications {[}7{]}.

With muscle fibers comprising about 70\% of horse meat and minimal fat
tissue, the meat has favorable dietary properties. Fat content typically
ranges between 0.5\% and 3.0\%, with unsaturated fatty acids accounting
for a larger proportion (approximately 56-60\%) than saturated ones
(around 40\%) {[}8{]}. Key unsaturated fatty acids include linoleic,
linolenic, palmitic, and oleic acids.

Qarta, also known as digestive system in Kazakh culture, is a
traditional delicacy prepared from the terminal portion of the horse's
large intestine, including the rectum. It is distinct in structure,
consisting of a dense outer layer of connective tissue and a fat-rich
inner surface, which contributes to its unique texture and taste after
cooking. Traditionally, qarta undergoes thorough cleaning and multiple
soaking steps to remove odors. It is typically boiled in salted water,
often with spices like black pepper and bay leaf, although variations
include smoking or drying. Once prepared, it is sliced and served as an
appetizer or main dish. In Kazakh culture, qarta is highly valued as a
symbol of hospitality, often served to esteemed guests during festive
gatherings.

From a nutritional standpoint, qarta differs notably from lean horse
meat. While horse meat is recognized for its high protein, low fat, and
favorable unsaturated fatty acid profile, qarta contains significantly
higher fat levels on its inner surface, increasing its energy value and
influencing its sensory properties. Analytical data from experimental
studies that horse meat, including gastrointestinal components.

Research by Dufey and Segato et al. {[}9,10{]} found that as horses age,
fat content increases while water and protein levels decline, though
cholesterol levels remain stable throughout aging.

Fat accumulation also increases with age. The proportion of abdominal
fat rises from 9.4\% at 6 months to 14.2\% by 30 months. Similarly,
subcutaneous and visceral fat increase with age, whereas intramuscular
fat decreases-from 51.4\% at 6 months to 43.8\% at 30 months.

Genetics significantly influence both the efficiency of horse meat
production and the meat' s quality. Although no horse
breeds are raised exclusively for meat, heavy cold-blooded breeds are
well-suited for this purpose. Meat is also sourced from warm-blooded
horses that are no longer viable due to age or injury. Existing studies
confirm a strong relationship between genotype and both qualitative and
quantitative slaughter traits. For example, analyzed slaughter
performance across five French heavy horse breeds aged between 12 and 30
months, demonstrating these genetic correlations.

{\bfseries Materials and methods.}~In accordance with the set goal and
objectives the scheme of research organisation was developed.
Experimental work of the research was carried out in the laboratories of
the departments of «Technology of food and processing industries»,
accredited testing laboratory «Food safety», «Research laboratory for
quality assessment and safety of food products», «National Center for
Biotechnology» at the «Laboratory of genetics and biochemistry of
microorganisms». Theoretical and practical skills were learnt during an
internship abroad at the Swedish University of Agricultual Sciences,
Department of «Molecular Sciences», Uppsala, Sweden.

The study was conducted on gastrointestinal raw materials of equine
origin - specifically, the terminal segment of the large intestine
(rectum), traditionally referred to as Qarta in Kazakh culture. Samples
were obtained from clinically healthy horses aged 1 to 3 years
post-slaughter at a certified meat-processing facility.

Determination of the total composition of meat and meat products.
Determination of the chemical composition gives the opportunity to get
an idea of the quality of meat and meat products, their nutritional
value, depending on the quantitative ratio of moisture, protein fat and
mineral substances. The content of basic nutrients in meat products is
determined by their recipe and the nature of technological processing.

Determination of moisture content. GOST 9793-2016 {[}11{]}. Methods for
determining moisture. A sample of ground product weighing 3-5 g, taken
to an accuracy of 0.001, dried in a metal bouquet with a glass rod in a
desiccator at 150\tsp{o}C for 1 hour or in the apparatus SAL
at 150\tsp{о}C for 15 min.

Moisture content is calculated according to the formula:



where, x\tsb{1} - moisture content, \%;

m\tsb{1} - weight of the bunker with the sample before drying,
g;

m\tsb{2} - mass of the bucket with the sample after drying, g;

m - weight of the bucket, g.

Determination of fat content. GOST 23042-2015 {[}12{]}. Methods of fat
determination. The dried sample after determination of moisture is
quantitatively transferred into a bouquet and poured 10-15 ml of solvent
(petroleum or ethyl ether). Extraction of fat is carried out for 3-4 min
with 4-5 times repeated frequency. During the process the suspension is
periodically stirred with a glass rod and the solvent is poured each
time with the extracted fat. After the last pour the residual solvent is
evaporated in air. Buex with defatted sample is dried in a desiccator at
105\tsp{o}C for 10 min.

The fat content is determined according to the formula:



where, x\tsb{2} - fat content, \%;

m\tsb{1} - weight of bouquet with sample after drying before
degreasing, g;

m\tsb{2} - mass of the bunker with the sample after
degreasing, g;

m\tsb{0} - weight of the sample, g.

Determination of ash content. GOST 31727-2012 {[}13{]}. Method for
determining the mass fraction of total ash. After degreasing the
contained bunks are transferred into a pre-calcined and weighed
crucible. The residue of the suspension from the walls of the bouquet is
washed off with a small amount of solvent, which is then removed by
heating on a water bath until it disappears. In the crucible to the dry
defatted suspension add 1 ml of magnesium acetate.

The crucible with the suspension is charred on an electric cooker, then
placed for 30 min in a muffle furnace, inside which the temperature is
500-600\tsp{o}C.

In the same way 1 ml of magnesium acetate is mineralised.

Ash content is calculated according to the formula:



where, x\tsb{3} - ash content, \%;

m\tsb{1} - mass of ash, g;

m\tsb{2} - mass of magnesium oxide obtained after
mineralisation of magnesium acetate solution, g;

m\tsb{0} - mass of suspension, g.

Determination of protein content. GOST 25011-2017 {[}14{]}. Methods for
determining the mass fraction of protein. Protein content is determined
by calculation according to the formula:



where, x - protein content, \%;

x\tsb{1} - moisture content, \%;

x\tsb{2} - fat content, \%.

Ash content, petroleum or ethyl ether, magnesium acetate solution.

GOST 32886-2014 {[}15{]}. Determination of cholesterol content by gas
chromatographic method. This method is used to quantify the amount of
cholesterol present in various meat and poultry samples.~The standard
outlines the procedures for sample preparation, analysis, and
calculation of results, ensuring consistent and reliable measurements of
cholesterol levels.~

Determination of amino acids composition. GOST 34132-2017 {[}16{]}.
Chromatographic method for determining the amino acid composition of
meat proteins, including poultry meat, offal, meat and meat-containing
products, as well as products made from poultry meat.

Determination of fatty acid composition. GOST 34987-2023 {[}17{]}. This
standard applies to slaughter products (meat, offal) of all types of
slaughtered animals and poultry, as well as meat products (including
canned meat and meat--containing products), including poultry meat
(hereinafter referred to as meat products), and establishes the
following methods for determining fatty acid composition: - a method for
determining the mass fraction of fatty acids in the form of methyl
esters using gas chromatography with a flame ionization detector
(GC-PID) with a measurement range from 0.1\% to 100\% - a method for
determining the mass fraction of saturated fatty acids using
near-infrared spectroscopy (BIC spectroscopy) with a measurement range
from 0.4\% to 80.0 \%.

Determination of fat-holding capacity (FHC). GOST 23042-2015 {[}18{]}.
This standard applies to meat and meat products and establishes methods
for determining fat content and fat-holding capacity (FHC), as well as
moisture-holding capacity (MHC). The fat-holding capacity is determined
by thermal treatment of the sample followed by gravimetric analysis of
fat released. The moisture-holding capacity is calculated based on mass
loss after heating in a water bath at 70±2\tsp{o}C for 15
minutes and subsequent pressing and weighing.

GOST 25179-90 {[}19{]}. This standard describes a refractometric method
for determining the protein mass fraction, which. Is widely adapted in
research and analytical practice for meatraw material. In meat analysis,
this method is used to estimate the refractive index of aqueous meat
extracts, indirectly reflecting the protein content, and is applied to
assess moisture-retaining properties.

Determination of water-binding capacity (WBC). Method of Vartanyan
(Pressing Method) {[}20{]}. This mthod is used for determining the
water-binding capacity (WBC) of meat and is based on compressing a
ground meat sample between filter paper layers under a standardized
pressure (1 kg load for 10 minutes at room temperature). The moisture
area (exudate) formed is measured using planimetry.

\fig{p/image24}{}

%% \begin{longtable}[]{@{}
%%   >{\raggedright\arraybackslash}p{(\linewidth - 4\tabcolsep) * \real{0.4783}}
%%   >{\raggedright\arraybackslash}p{(\linewidth - 4\tabcolsep) * \real{0.2174}}
%%   >{\raggedright\arraybackslash}p{(\linewidth - 4\tabcolsep) * \real{0.3044}}@{}}
%% \toprule\noalign{}
%% \begin{minipage}[b]{\linewidth}\centering
%% {\bfseries Horse (1-3 years)}
%% \end{minipage} & \begin{minipage}[b]{\linewidth}\centering
%% {\bfseries Carcass of horse}
%% \end{minipage} & \begin{minipage}[b]{\linewidth}\centering
%% {\bfseries Digestive system of horse}
%% \end{minipage} \\
%% \midrule\noalign{}
%% \endhead
%% \bottomrule\noalign{}
%% \endlastfoot
%% \end{longtable}

{\bfseries Fig.1 - Anatomical diagram of the horse and its
gastrointestinal system}

Figure 1 provides a schematic overview of the horse's digestive anatomy.
The entire gastrointestinal tract is depicted, highlighting the large
intestine and its terminal section -- the rectum -- traditionally known
as qarta in Kazakh culture. The caecum, colon, and rectum are clearly
visualized, establishing anatomical context for the studied tissue.

A detailed figure 2 of the horse's large intestine, emphasizing the
colon's length, thickness, and internal structural features. The image
offers visual clarity for understanding where qarta originates and its
relation to other digestive organs.

\fig{p/image25}{}

{\bfseries Fig.2 - The anatomy of the gastrointestinal tract in horse}

Figure 3 shows the external and internal surface of qarta. The outer
layer appears dense and fibrous, composed largely of connective tissue,
while the inner layer displays visible fat deposits -- a key contributor
to qarta's flavor and texture upon cooking.

\fig{p/image26}{}\fig{p/image27}{}

\def\labelenumi{\alph{enumi})}

1. b)

{\bfseries Fig.3 - Intestinal raw materials of horse meat - qarta (a)
external; b) internal)}

\fig{p/image28}{}

{\bfseries Fig.4 - Cross-sectional view of Qarta, showing fat-muscle
structure}

The cross-section reveals the layered architecture of qarta: a fatty
inner surface juxtaposed with muscular and connective tissue structures.
This figure 4 reinforces qarta's technological characteristics relevant
for culinary and meat-processing applications.

{\bfseries Results and discussion.} Chemical composition of fresh
gastrointestinal tract from cattle, pig, and horse, as well as the
boiled products derived from these animals. Their analysis revealed that
the water activity (a\tsb{w}) values in boiled meats from
bovine, horse, and pig are consistent with expected standards for such
products. Notably, the pH levels were generally high, except in the case
of horse meat, which may indicate ongoing proteolytic activity driven by
endo- and exoenzymes produced by microorganisms involved in the curing
and ripening process (Table 1, 2).

Fresh qarta demonstrated high moisture (60.46\%) and fat (21.61\%)
content, with moderate protein levels (9.56\%). The pH value (6.12\%)
and high water acivity (0.94\%) suggest susceptibility to microbial
growth, requiring proper processing. Functional indicators (WHC, WBC,
FHC) were relatively high, indicating good technological potential.

{\bfseries Table 1 - pH,} {\bfseries a\tsb{w} content of fresh
gastrointestinal tract of horse after thawing}

%% \begin{longtable}[]{@{}
%%   >{\centering\arraybackslash}p{(\linewidth - 2\tabcolsep) * \real{0.5301}}
%%   >{\centering\arraybackslash}p{(\linewidth - 2\tabcolsep) * \real{0.4699}}@{}}
%% \toprule\noalign{}
%% \begin{minipage}[b]{\linewidth}\centering
%% {\bfseries Name of indicators, unit of measurement}
%% \end{minipage} & \begin{minipage}[b]{\linewidth}\centering
%% {\bfseries Actual results}
%% \end{minipage} \\
%% \midrule\noalign{}
%% \endhead
%% \bottomrule\noalign{}
%% \endlastfoot
%% pH under the hood, \% & 6.12±0.06 \\
%% a\tsb{w}, \% & 0.94±0.01 \\
%% Mass fraction of moisture, \% & 60.46±0,91 \\
%% Mass fraction of proteins, \% & 9.56±0.14 \\
%% Mass fraction of fats, \% & 21.61±0.32 \\
%% Water-holding capacity (WHC) , \% & 45.87±0.20 \\
%% Water-binding capacity (WBC) , \% & 24.96±0.96 \\
%% Fat-holding capacity (FHC) , \% & 44.91±0.14 \\
%% Ash, \% & 1.3±0.09 \\
%% \end{longtable}

Boiling led (Table 2) to a reduction in moisture (52.10\%) and fat
(15.84\%), while protein content increased significantly to 19.11\% due
to water loss. Functional parameters (WHC, WBC, FHC) slightly decreased
but remained within acceptable technological limits. Ash content
increased to 7.7\% reflecting mineral concentration after cooking.

{\bfseries Table 2 - pH, a\tsb{w} and approximate content in
final boiled gastrointestinal tract}

%% \begin{longtable}[]{@{}
%%   >{\centering\arraybackslash}p{(\linewidth - 2\tabcolsep) * \real{0.5301}}
%%   >{\centering\arraybackslash}p{(\linewidth - 2\tabcolsep) * \real{0.4699}}@{}}
%% \toprule\noalign{}
%% \begin{minipage}[b]{\linewidth}\centering
%% {\bfseries Name of indicators, unit of measurement}
%% \end{minipage} & \begin{minipage}[b]{\linewidth}\centering
%% {\bfseries Actual results}
%% \end{minipage} \\
%% \midrule\noalign{}
%% \endhead
%% \bottomrule\noalign{}
%% \endlastfoot
%% pH under the hood, \% & 5.81±0.15 \\
%% a\tsb{w}, \% & 0.90±0.02 \\
%% Mass fraction of moisture, \% & 52.10±0.93 \\
%% Mass fraction of proetins, \% & 19.11±0.28 \\
%% Mass fraction of fats, \% & 15.84±0.24 \\
%% Water-holding capacity (WHC) , \% & 38.41±0.57 \\
%% Water-binding capacity (WBC) , \% & 18.84±0.28 \\
%% Fat-holding capacity (FHC) , \% & 38.41±0.57 \\
%% Ash, \% & 7.7±0.22 \\
%% \end{longtable}

Among all the meats studied, boiled horse meat exhibited the lowest
cholesterol content, a finding of particular nutritional interest. Horse
meat stood out for its notably significant level of polyunsaturated
fatty acids. Meat products had a moderate level of these fatty acids,
whereas bovine products had the lowest (Table 3). Cholesterol level was
measured at 29.9mg/100g, which is notably low compared to traditional
red meats.This reinforces the dietary suitability of horse-derived
gastrointestinal products for individuals with cardiovascular concerns.

{\bfseries Table 3 - Levels of cholesterol (mg/100 g) in final boiled meat
products}

%% \begin{longtable}[]{@{}
%%   >{\centering\arraybackslash}p{(\linewidth - 4\tabcolsep) * \real{0.5301}}
%%   >{\centering\arraybackslash}p{(\linewidth - 4\tabcolsep) * \real{0.4691}}
%%   >{\centering\arraybackslash}p{(\linewidth - 4\tabcolsep) * \real{0.0007}}@{}}
%% \toprule\noalign{}
%% \begin{minipage}[b]{\linewidth}\centering
%% {\bfseries Name of indicators, unit of measurement}
%% \end{minipage} & \begin{minipage}[b]{\linewidth}\centering
%% {\bfseries Actual results}
%% \end{minipage} & \begin{minipage}[b]{\linewidth}\centering
%% \end{minipage} \\
%% \midrule\noalign{}
%% \endhead
%% \bottomrule\noalign{}
%% \endlastfoot
%% Cholesterol, \% &
%% \multicolumn{2}{>{\centering\arraybackslash}p{(\linewidth - 4\tabcolsep) * \real{0.4699} + 2\tabcolsep}@{}}{%
%% 29.9±0.72} \\
%% \end{longtable}

The fatty acid profile of boiled qarta showed dominance of
monounsaturated (MUFA -- 42.5\%) and polyunsaturated fatty acids (PUFA
-- 17.7\%), especially oleic and linoleic acids. Saturated fatty acids
(SFA) accounted for 39.84\%. This composition highlights qarta's
favorable lipid profile with potential health benefits (Table 4).

{\bfseries Table 4 - Content of fatty acids (\%) of gastrointestinal raw
material of final boiled qarta}

%% \begin{longtable}[]{@{}
%%   >{\raggedright\arraybackslash}p{(\linewidth - 2\tabcolsep) * \real{0.5305}}
%%   >{\raggedright\arraybackslash}p{(\linewidth - 2\tabcolsep) * \real{0.4695}}@{}}
%% \toprule\noalign{}
%% \begin{minipage}[b]{\linewidth}\centering
%% {\bfseries Name of indicators, unit of measurement}
%% \end{minipage} & \begin{minipage}[b]{\linewidth}\centering
%% {\bfseries Actual results}
%% \end{minipage} \\
%% \midrule\noalign{}
%% \endhead
%% \bottomrule\noalign{}
%% \endlastfoot
%% Saturated fatty acids (SFA), \% & \\
%% C10:0 capric fatty acid & 0.095±0.005 \\
%% C12:0 lauric fatty acid & 0.159±0.009 \\
%% C14:0 miristinic fatty acid & 3.265±0.163 \\
%% C15:0 pentadecanoic fatty acid & 0.492±0.025 \\
%% C16:0 palmitic fatty acid & 28.996±1.450 \\
%% C17:0 margaric fatty acid & 0.606±0.030 \\
%% C18:0 stearic fatty acid & 5.892±0.295 \\
%% C21:0 geneticosanic fatty acid & 0.170±0.009 \\
%% C22:0 behenoic fatty acid & 0.055±0.003 \\
%% C23:0 tricosanoic fatty acid & 0.113±0.006 \\
%% Monounsaturated fatty acid (MUFA), \% & \\
%% C14:1 (cis-9) myristoleic fatty acid & 0.229±0.011 \\
%% C15:1 (cis-10) pentadecenoic fatty acid & 0.046±0.002 \\
%% C16:1 (cis-9) palmitoleic fatty acid & 5.999±0.300 \\
%% C17:1 (cis-10) margarinoleic fatty acid & 0.632±0.032 \\
%% C18:1 (trans-9) oleinic fatty acid & 0.023±0.001 \\
%% C18:1 (cis-9) oleinic fatty acid & 35.025±1.751 \\
%% C20:1 (cis-11) eicosenoic fatty acid & 0.393±0.020 \\
%% C22:1 (cis-13) erucic fatty acid & 0.048±0.002 \\
%% C24:1 (cis-15) selacholic fatty acid & 0.059±0.003 \\
%% Polyunsaturated fatty acid (PUFA), \% & \\
%% C18:2n6c linoleic fatty acid & 14.420±0.721 \\
%% C18:3n6 Y- linoleic fatty acid & 3.012±0.151 \\
%% C18:3n3 linoleic fatty acid & 0.035±0.002 \\
%% C20:3n6c (cis-8,11,14) eicosatrienoic fatty acid & 0.235±0.012 \\
%% Total SFA & 39.843±1.995 \\
%% Total MUFA & 42.454±2.122 \\
%% Total PUFA & 17.702±0.886 \\
%% \end{longtable}

Boiled qarta contained a high total level of free amino acids (13~177.93
mg/100g), including essential amino acids (5~455.56mg/100g or 41.3\%).
Dominant amino acids included glutamic fatty acid, aspartic fatty acid
and leucine. This confirms qarta as a valuable source of high-quality,
bioavailable protein.

{\bfseries Table 5 - Free amino acids (mg/100g) in final boiled qarta}

%% \begin{longtable}[]{@{}
%%   >{\centering\arraybackslash}p{(\linewidth - 2\tabcolsep) * \real{0.5195}}
%%   >{\centering\arraybackslash}p{(\linewidth - 2\tabcolsep) * \real{0.4805}}@{}}
%% \toprule\noalign{}
%% \begin{minipage}[b]{\linewidth}\centering
%% {\bfseries Name of indicators}
%% \end{minipage} & \begin{minipage}[b]{\linewidth}\centering
%% {\bfseries Actual results}
%% \end{minipage} \\
%% \midrule\noalign{}
%% \endhead
%% \bottomrule\noalign{}
%% \endlastfoot
%% ASP & 1334.96±133.50 \\
%% THR & 645.45±64.55 \\
%% SER & 607.69±60.77 \\
%% GLU & 2056.64±205.66 \\
%% PRO & 647.45±24.55 \\
%% GLY & 602.09±60.21 \\
%% ALA & 722.37±72.24 \\
%% VAL & 487.06±48.71 \\
%% CYS & 210.48±21.05 \\
%% MET & 330.76±33.08 \\
%% ILE & 558.74±55.87 \\
%% LEU & 1044.75±104.48 \\
%% TYR & 480.41±48.04 \\
%% PHE & 599.30±59.93 \\
%% LYS & 1216.08±121.61 \\
%% HIS & 573.42±57.34 \\
%% ARG & 975.52±97.55 \\
%% TOTAL & 13~177.93 \\
%% Essential amino acids (\%) & 5~455.56 (41.3\%) \\
%% \end{longtable}

{\bfseries Conclusion.} The present study comprehensively characterizes the
biochemical, nutritional, and functional properties of qarta, as
traditional Kazakh meat delicacy derived from the terminal segment of
the horse's gastrointestinal tract. The analysis revealed that qarta is
a valuable source of high-quality protein (notably rich in essential
amine acids), unsaturated fatty acids (particularly oleic and linoleic
acids), and minerals. Despite its anatomical origin, qarta displayed
favorable technological characteristics, including strong water- and
fat-binding capacities, even after thermal processing.

Notably, qarta contains significantly lower cholesterol levels than
conventional red meats, while offering a superior lipid profile --
making it a promising ingredient for health-oriented and delicatessen
food products. The considerable content of free and essential amina
acids further enhances its biological value and digestibility,
supporting its potential role in dietary formulations for vulnerable or
nutritionally demanding populations.

These findings contribute novel scientific data to the underexplored
area of equine by-product utilization and offer a foundation for the
valorization of qarta in meat science, gastronomy, and food technology.
Given the increasing market demand for culturally rooted, protein-rich,
and nutritionally balanced foods, qarta holds considerable potential
both as a traditional product and as a functional ingredient in modern
meat processing and product development.

{\bfseries Литература}

1. Petrov K. A. Dudareva, L. V., Nokhsorov, V. V., Stoyanov, K. N., \&
Makhutova, O. N.~ Fatty acid content and composition of the Yakutian
horses and their main food source: living in extreme winter conditions
//Biomolecules.-2020.-Vol.10(2). - P.315.
~\href{https://doi.org/10.3390/biom10020315}{DOI 10.3390/biom10020315}.

2. Alibekov R. S. Alibekova Z. I., Bakhtybekova A. R., Taip F. S.,
Urazbayeva K. A., Kobzhasarova Z. I. Review of the slaughter wastes and
the meat by-products recycling opportunities // Frontiers in Sustainable
Food Systems. -2024.- Vol.8.- P.1-17.
\href{https://doi.org/10.3389/fsufs.2024.1410640}{DOI
10.3389/fsufs.2024.1410640}.

3. Uzakov Y. M., Kaldarbekova M. A., Kuznetsova O. N. Improved
technology for new-generation Kazakh national meat products // Foods and
Raw Materials.- 2020.-Vol.8(1)- P.76--83.
https://doi.org/10.21603/2308-4057-2020-1-76-83.

4. Dobranić V. et al. Chemical composition of horse meat //MESO: Prvi
hrvatski časopis o mesu. -- 2009. - Vol.11.(1)1. - P.62-38.

5. Martin-Rosset W., Boccard R., Jussiaux M., Robelin J., Trillaud-Geyl
C. (1980): Rendement et composition des carcasses du poulain de
boucherie // Bulletin Technique Centre de Recherches Zootechniques et
Veterinaires de Theix. -1980.- № 41. P.57-64. ISSN~0395-7519.

6. Martin-Rosset W. (2001)Horse meat production and characteristics.52
nd Annual Meeting EAAP, Budapest,- P.26-29.

7. Maikanov B. S., Ismagulova G. T., Auteleyeva L. T., Kemeshov Z. O.,
Zhanabayeva D. K. Assessment of quality and safety of meats from various
animal species in the Shuchinsk-Burabay resort zone,
Kazakhstan//Veterinary World.-2021- Vol.14(6).- P.1615-1621. DOI
10.14202/vetworld.2021.1615-1621.

8. Makray S. et al. Evaluation of dietary value of horse meat //Acta
agriculturae Slovenica. Supplement. - 1998. -- Vol.30. - С.209-212.
DOI 10.14720/aas-s.1998.30.19628

9. Dufey P. A. Proprietes sensorielles et physico-chimiques de la viande
de chevaux de differentes categories d' age //Revue
suisse d' agriculture.-1999.- Vol.31(3)- P.157-161.

10. Segato, S., Cozzi, G., \& Andrighetto, I. (1999) Effect of animal
morphotype, sex and age on quality of horse meat imported from Poland.
In: Proceedings of the ASPA XIII Congress (pp.674-676), Piacenza
(Italy), June 21st--24th.

11. ГОСТ 9793-2016. Мясо и мясные продукты. Метод определения массовой
доли влаги. - М.: Стандаринформ, 2016. - 6 с.

12. ГОСТ 23042-2015. Мясо и мясные продукты. Метод определения массовой
доли жира. - М.: Стандаринформ, 2015. - 10 с.

13. ГОСТ 31727-2012. Мясо и мясные продукты. Метод определения массовой
доли золы. - М.: Стандаринформ, 2012. - 8 с.

14. ГОСТ 25011-2017. Мясо и мясные продукты. Метод определения массовой
доли белка. - М.: Стандаринформ, 2017. - 9 с.

15. ГОСТ 32886-2014. Мясо и мясные продукты. Определение содержания
холестерина методом газовой хроматографии. - М.: Стандаринформ, 2014.

16. ГОСТ 34132-2017. Мясо и мясные продукты. Метод определения
аминокислотного состава методом ионообменной хроматографии. - М.:
Стандаринформ, 2017. - 18 с.

17. ГОСТ 34987-2023. Продукты убоя животных и птицы. Метод определения
жирнокислотного состава (ГХ--ПИД, NIRS). - М.: Стандаринформ, 2023. - 22
с.

18. ГОСТ 23042-2015. Мясо и мясные продукты. Метод определения массовой
доли жира / Межгосударственный совет по стандартизации, метрологии и
сертификации. - М.: Стандаринформ, 2016. - 13 с.

19. ГОСТ 25179-90. Молоко и молочные продукты. Рефрактометрический метод
определения белка / Госкомитет СССР по стандартам. - М.: Издательство
стандартов, 1991. - 7 с.

20. Антипова Л. В., Глотова И. А., Рогов И. А. Методы исследования мяса
и мясных продуктов. - М.: Колос, 2001.- 376 с. ISBN 5-10-003612-5

{\bfseries References}

1. Petrov K. A. Dudareva, L. V., Nokhsorov, V. V., Stoyanov, K. N., \&
Makhutova, O. N.~ Fatty acid content and composition of the Yakutian
horses and their main food source: living in extreme winter conditions
//Biomolecules.-2020.-Vol.10(2). - P.315.
~\href{https://doi.org/10.3390/biom10020315}{DOI 10.3390/biom10020315}.

2. Alibekov R. S. Alibekova Z. I., Bakhtybekova A. R., Taip F. S.,
Urazbayeva K. A., Kobzhasarova Z. I. Review of the slaughter wastes and
the meat by-products recycling opportunities // Frontiers in Sustainable
Food Systems. -2024.- Vol.8.- P.1-17.
\href{https://doi.org/10.3389/fsufs.2024.1410640}{DOI
10.3389/fsufs.2024.1410640}.

3. Uzakov Y. M., Kaldarbekova M. A., Kuznetsova O. N. Improved
technology for new-generation Kazakh national meat products // Foods and
Raw Materials.- 2020.-Vol.8(1)- P.76--83.
https://doi.org/10.21603/2308-4057-2020-1-76-83.

4. Dobranić V. et al. Chemical composition of horse meat //MESO: Prvi
hrvatski časopis o mesu. -- 2009. - Vol.11.(1)1. - P.62-38.

5. Martin-Rosset W., Boccard R., Jussiaux M., Robelin J., Trillaud-Geyl
C. (1980): Rendement et composition des carcasses du poulain de
boucherie // Bulletin Technique Centre de Recherches Zootechniques et
Veterinaires de Theix. -1980.- № 41. P.57-64. ISSN~0395-7519.

6. Martin-Rosset W. (2001)Horse meat production and characteristics.52
nd Annual Meeting EAAP, Budapest,- P.26-29.

7. Maikanov B. S., Ismagulova G. T., Auteleyeva L. T., Kemeshov Z. O.,
Zhanabayeva D. K. Assessment of quality and safety of meats from various
animal species in the Shuchinsk-Burabay resort zone,
Kazakhstan//Veterinary World.-2021- Vol.14(6).- P.1615-1621. DOI
10.14202/vetworld.2021.1615-1621.

8. Makray S. et al. Evaluation of dietary value of horse meat //Acta
agriculturae Slovenica. Supplement. - 1998. -- Vol.30. - С.209-212.
DOI 10.14720/aas-s.1998.30.19628

9. Dufey P. A. Proprietes sensorielles et physico-chimiques de la viande
de chevaux de differentes categories d' age //Revue
suisse d' agriculture.-1999.- Vol.31(3)- P.157-161.

10. Segato, S., Cozzi, G., \& Andrighetto, I. (1999) Effect of animal
morphotype, sex and age on quality of horse meat imported from Poland.
In: Proceedings of the ASPA XIII Congress (pp.674-676), Piacenza
(Italy), June 21\tsp{st}-24th.

11. GOST 9793-2016. Mjaso i mjasnye produkty. Metod opredelenija
massovoj doli vlagi. - M.: Standarinform, 2016. - 6 s. {[}in Russian{]}

12. GOST 23042-2015. Mjaso i mjasnye produkty. Metod opredelenija
massovoj doli zhira. - M.: Standarinform, 2015. - 10 s. {[}in Russian{]}

13. GOST 31727-2012. Mjaso i mjasnye produkty. Metod opredelenija
massovoj doli zoly. - M.: Standarinform, 2012. - 8 s. {[}in Russian{]}

14. GOST 25011-2017. Mjaso i mjasnye produkty. Metod opredelenija
massovoj doli belka. - M.: Standarinform, 2017. - 9 s. {[}in Russian{]}

15. GOST 32886-2014. Mjaso i mjasnye produkty. Opredelenie soderzhanija
holesterina metodom gazovoj hromatografii. - M.: Standarinform, 2014.
{[}in Russian{]}

16. GOST 34132-2017. Mjaso i mjasnye produkty. Metod opredelenija
aminokislotnogo sostava metodom ionoobmennoj hromatografii. - M.:
Standarinform, 2017. - 18 s. {[}in Russian{]}

17. GOST 34987-2023. Produkty uboja zhivotnyh i pticy. Metod
opredelenija zhirnokislotnogo sostava (GH--PID, NIRS). - M.:
Standarinform, 2023. - 22 s. {[}in Russian{]}

18. GOST 23042-2015. Mjaso i mjasnye produkty. Metod opredelenija
massovoj doli zhira / Mezhgosudarstvennyj sovet po standartizacii,
metrologii i sertifikacii. - M.: Standarinform, 2016. - 13 s. {[}in
Russian{]}

19. GOST 25179-90. Moloko i molochnye produkty. Refraktometricheskij
metod opredelenija belka / Goskomitet SSSR po standartam. - M.:
Izdatel' stvo standartov, 1991. - 7 s. {[}in Russian{]}

20. Antipova L. V., Glotova I. A., Rogov I. A. Metody issledovanija
mjasa i mjasnyh produktov. - M.: Kolos, 2001.- 376 s. ISBN
5-10-003612-5. {[}in Russian{]}

\texorpdfstring{\emph{{\bfseries Information about the
authors}}}{Information about the authors}

Kostanova A.T.- PhD student, NJSC «S.Seifullin Kazakh Agrotechnical
Research University», Astana, Kazakhstan, e-mail:
anel\_kostanova@mail.ru;

Baytukenova Sh. B. - candidate of technical sciences, associate
professor, NJSC «S.Seifullin Kazakh Agrotechnical Research University»,
Astana, Kazakhstan, e-mail: baytukenova75@mail.ru;

\texorpdfstring{Baytukenova S.B. - candidate of technical
sciences, associate professor, JSC «K.Kulazhanov Kazakh University of
Technology and Business», Astana, Kazakhstan, e-mail:
saule7272@mail.ru.}{Baytukenova S.B. - candidate of technical sciences, associate professor, JSC «K.Kulazhanov Kazakh University of Technology and Business», Astana, Kazakhstan, e-mail: saule7272@mail.ru.}

\emph{{\bfseries Сведения об авторах}}

Костанова А.Т. - докторант НАО «Казахский агротехнический
исследовательский университет им. С.Сейфуллина», Астана, Казахстан,
e-mail:
anel\_kostanova@mail.ru;

Байтукенова Ш.Б. - кандидат технических наук, и.о. ассоциированного
профессора НАО «Казахский агротехнический исследовательский университет
им. С.Сейфуллина», Астана, Казахстан, e-mail: baytukenova75@mail.ru;

Байтукенова С.Б. - кандидат технических наук, и.о. ассоциированного
профессора АО «Казахский университет технологии и бизнеса им. К.
Кулажанова», Астана, Казахстан, e-mail:
saule7272@mail.ru.\
