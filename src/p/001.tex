\id{ҒТАМР 65.63.33}{}

\begin{header}
\swa{}{АҚ ҚАЙЫҢ ҚАБЫҒЫНАН ТРИТЕРПЕНОИДТАРДЫ БӨЛУ ТЕХНОЛОГИЯСЫ}

\tsp{1}С.А. Карденов\envelope,
\tsp{2}С.Б. Байтукенова,
\tsp{1}Ш.Б. Байтукенова,
\tsp{1}Э.Ч. Базылханова,
\tsp{1}Ж.С. Ажгереева
\end{header}

\begin{affil}
\tsp{1}С.Сейфуллин атындағы Қазақ агротехникалық зерттеу университеті, Астана, Қазақстан,

\tsp{2}Қ.Құлажанов атындағы Қазақ технология және бизнес университеті, Астана, Қазақстан

\corrauthor{Корреспондент-автор:Askerbekovsk@mail.ru}
\end{affil}

Өсімдік тектес заттар тағамдық қоспалар мен медицина саласында
дәрі-дәрмектер ретінде кеңінен қолданылады. Ақ қайың қабығының химиялық
құраммы, тритерпеноидтарды бөліп алу әдістері және олардың биологиялық
белсенділігі бойынша зерттеулер жасалы. Қайың қабығынан
тритерпеноидтарды бөліп алудың тиімді әдісі ұсынылды. Тритерпеноидтарды
ультрадыбыстық белсендіру және экстракциялау басқа әдістермен
салыстырғанда, экологиялық таза және тиімді әдіс болып табылды.

Отандық шикізаттың орасан зор қоры, қайың қабығынан алынған биологиялық
белсенді заттардың зерттеу бойынша жүйелі жұмыстары аз болғандықтан
экстрактивті заттар кешенінің қасиеттерін, олардың физика-химиялық
сипаттамаларын зерттеуді жалғастыруға көп мүмкіндік береді.
Тритерпеноидтар ағзаға ір түрлі пайдалы қасиеттері бар биологиялық
белсенді органикалық қосылыстар. Тағамға биологиялық белсенді
тритерпеноидтарды қосу өнімнің сапасын жақсартады, өнімдердің
қышқылдылығын төмендетіп, олардың сақтау мерзімін ұзартады.

Қайың қабығынан алынған тритерпеноидтар сүтқышқылды өнімдердің сапасы
мен пайдалы қасиеттерін жақсарта алатын функционалды ингредиент ретінде
сүтқышқылды өнімдер өндірушілердің назарын аударуда. Алынған нәтижелерді
қортындылай келе, қайың қабығынан бөлінген тритерпеноиды бар экстрактіге
физика-химиялық талдау жасалды (сығындының балқу температурасы еру
температурасы сығынды құрамындағы дәрумендер таниндер құрғақ заттар
анықталды). Сығындының тазалығы құрамы ИҚ және УФ спектроскопиясы арқылы
анықталды. Сығындының микробиолгиялық талдау жасалып, оның
Staphylococcus aureus Bacillus subtilis Escherichia coli Pseudomonas
aeruginosa Candida albicans Lactobacillus plantarum микроорганизімдеріне
қарсы антимикробтық белсенділігі және олардың бактериялардың,
вирустардың және саңырауқұлақтардың өсуін жоятыны немесе тоқтататаны
анықталды.

{\bfseries Түйін сөздер:} тритерпеноид, бетулин, мацерация, сүтқышқылды
өнімдер, айран, қымыз.

\begin{header}
ТЕХНОЛОГИЯ ВЫДЕЛЕНИЯ ТРИТЕРПЕНОИДОВ ИЗ БЕЛОЙ БЕРЕЗЫ

\tsp{1}С.А. Карденов,
\tsp{2}С.Б. Байтукенова,
\tsp{1}Ш.Б. Байтукенова,
\tsp{1}Э.Ч. Базылханова,
\tsp{1}Ж.С. Ажгереева
\end{header}

\begin{affil}
\tsp{1}НАО «Казахский агротехнический исследовательский университет им. С.Сейфуллина» Астана, Казахстан,

\tsp{2}АО «Казахский университет технологии и бизнеса им. К. Кулажанова» Астана, Казахстан,

e-mail:Askerbekovsk@mail.ru
\end{affil}

Растительные вещества широко используются в качестве пищевых добавок и
лекарственных средств в медицине. Проведены исследования химического
состава коры белой березы, методов выделения тритерпеноидов и их
биологической активности. Предложен эффективный метод выделения
тритерпеноидов из коры березы. Установлено, что ультразвуковая активация
и экстракция тритерпеноидов является экологически чистым и эффективным
методом по сравнению с другими методами.

Огромные запасы отечественного сырья в сочетании с отсутствием системных
работ по изучению биологически активных веществ, извлекаемых из березы,
предоставляют широкие возможности для дальнейшего изучения свойств
комплекса экстрактивных веществ и их физико-химических характеристик.
Тритерпеноиды --- биологически активные органические соединения,
обладающие разнообразными полезными свойствами для организма. Добавление
в пищевые продукты биологически активных тритерпеноидов улучшает
качество продукции, снижает кислотность продуктов и увеличивает сроки их
хранения. Тритерпеноиды из березы привлекают внимание производителей
кисломолочных продуктов как функциональный ингредиент, способный
улучшить качество и полезные свойства кисломолочных продуктов.

Обобщая полученные результаты, был проведен физико-химический анализ
экстракта, содержащего тритерпеноиды, выделенные из коры березы
(определены температура плавления экстракта, температура растворения,
содержание витаминов, дубильных веществ и сухих веществ в экстракте).
Чистоту экстракта определяли методами ИК- и УФ-спектроскопии.
Микробиологический анализ экстракта показал, что он обладает
антимикробной активностью в отношении микроорганизмов Staphylococcus
aureus, Bacillus subtilis, Escherichia coli, Pseudomonas aeruginosa,
Candida albicans и Lactobacillus plantarum, а также убивает или
подавляет рост бактерий, вирусов и грибков.

{\bfseries Ключевые слова:} тритерпеноид, бетулин, мацерация, кисломолочные
продукты, кефир, кумыс.

\begin{header}
USE OF TRITERPENOIDS FROM WHITE BIRCH IN FERMENTED MILK PRODUCTS

\tsp{1}S.A. Kardenov,
\tsp{2}S.B. Baitukenova,
\tsp{1}Sh.B. Baitukenova,
\tsp{1}E.Ch. Bazylkhanova,
\tsp{1}Zh.S. Azhgereeva
\end{header}

\begin{affil}
\tsp{1}«Kazakh Agrotechnical Research University named after S. Seifullin», Astana, Kazakhstan,

\tsp{2}«Kazakh University of Technology and Business named after K. Kulazhanov», Astana Kazakhstan,

e-mail: Askerbekovsk@mail.ru
\end{affil}

Plant substances are widely used as food additives and medicines in
medicine. Studies have been conducted on the chemical composition of
white birch bark, methods for isolating triterpenoids and their
biological activity. An effective method for isolating triterpenoids
from birch bark is proposed. It is established that ultrasonic
activation and extraction of triterpenoids is an environmentally
friendly and effective method compared to other methods.

Huge reserves of domestic raw materials, combined with the lack of
systematic work on the study of biologically active substances extracted
from birch, provide ample opportunities for further study of the
properties of the complex of extractive substances and their
physicochemical characteristics. Triterpenoids are biologically active
organic compounds that have a variety of beneficial properties for the
body. Birch triterpenoids are attracting the attention of dairy
producers as a functional ingredient capable of improving the quality
and beneficial properties of fermented milk products.

Summarizing the obtained results, a physicochemical analysis of the
extract containing triterpenoids isolated from birch bark was carried
out (the melting point of the extract, the dissolution temperature, the
content of vitamins, tannins and dry substances in the extract were
determined). The purity of the extract was determined by IR and UV
spectroscopy. Microbiological analysis of the extract showed that it has
antimicrobial activity against the microorganisms Staphylococcus aureus,
Bacillus subtilis, Escherichia coli, Pseudomonas aeruginosa, Candida
albicans and Lactobacillus plantarum, and also kills or inhibits the
growth of bacteria, viruses and fungi.

{\bfseries Keywords:} Triterpenoid, betulin, maceration, fermented milk
products, kefir, kumiss.

\begin{multicols}{2}
{\bfseries Кіріспе.} Қазіргі таңда амдамдардың денсаулығын сақтау және
нығайту кез-келген өркениетті елдердің маңызды міндеті болып табылады.
Елімізде соңғы жылдарда тамақ өндірісі саласында халықты пайдалы
тағаммен қамтамсыз ету мақсатында функциональды ингредиенттермен
байытылған жаңа өнімдерді дайындау қарқынды дамуда.Өнімдердің тағамдық
құндылығы -- ағзаның қоректік заттармен қанағаттандырылуы олардың
энергетикалық құндылығымен, дәмдік қасиеттерімен және құрамымен
айқындалатын дәрежесі болып табылады.

Осыған байланысты соңғы жылдары өсімдік шикізатынан биологиялық белсенді
тағамды қоспаларды дайындау мен зерттеу жұмыстары ғалымдар арасында
үлкен қызығушылық тудыруда.Солардың бірі ақ қайың қабығының құрамында
кездесетін белсенді заттар.

Ғылыми зерттеу жұмысыың өзектілігі: Қайың қабығынан тритерпеноидтарды
бөлу технологиясын жетілдіру және оларды биологиялық белсенді
компоненттер мен функционалды ингредиенттер ретінде пайдалануды ұсыну.

Ғылыми зерттеу жұмысының жаңалығы: Заманауи технологияларды қолдана
отырып, ақ қайың қабығынан тритерпеноидтарды бөліп алудың оңтайлы
параметрлерін құрастыру.

Ғылыми зерттеу жұмысының міндеттері:

- заманауи технологияларды пайдалана отырып, қайың қабығынан
тритерпеноидтарды бөліп алу әдістерін жасау;

- тритерпеноидтарды сандық анықтау үшін заманауи аналитикалық әдістерді
(ЖҚХ, ИҚ спектрофотометр) қолдану арқылы ақ қайың қабығына фитохимиялық
зерттеу жүргізу;

- тритерпеноидтардың биологиялық белсенділігін зерттеу;

- тритерпеноидтардың антимикробтық белсенділігін зерттеу;

- тритерпеноидтардың антиоксидантық белсенділігін зерттеу;

- тритерпеноидтардың тағамның функционалдық қасиеттеріне әсерін зерттеу;

- қайыңнан тритерпеноидтарды бөліп алу мақсатында ультрадыбысты
активацияның оңтайлы параметрлерін әзірлеу {[}1{]}.

{\bfseries Материалдар мен әдістер.} Өсімдік тектес шикізаттардың құрамында
кездесетін антиоксиданттар майлы сүт өнімдерінің тотығу бұзылу
процестерін басуға қабілетті. Табиғи антиоксиданттар табиғи
метаболикалық агенттер ретінде дененің химиялық гомеостазын бұзбайды
және иммундық жүйенің реакциясын тудырмайды. Дегенмен кейбір табиғи
антиоксиданттарды қолдануда белгілі бір қиындықтар бар.

Олардың ерекшелігі -- сығындылар өсімдіктің өзіне тән дәмі мен иіс
қасиеттерін сақтайды, бұл дайын өнімнің органолептикалық көрсеткіштеріне
теріс әсер етуі мүмкін. Осыған орай бүгінгі таңда оңтайлы
антиоксиданттарды іздеу сүт өнеркәсібі үшін өзекті мәселе. Перспективалы
табиғи антиоксиданттардың бірі - бетулин. Химиялық жағынан бұл лупан
сериясының пентациклді тритерпенді спирті. Ол өнімдердің тотығуға
төзімділігін арттырады, бір мезгілде ең күшті табиғи консервант,
эмульгатор, антисептикалық және биостимулятор болып табылады {[}2{]}.

Негізгі зерттеу нысаны реттінде ақ қайың қабығы алынды. Зерттеуде
Қостанай өлкесінде өсетін Betula Pendula жаңа кесілген қайың қабығы
пайдаланылды. Қайың қабығының ылғалдылығы 3,05\%, Қайыңның қабығы
қабынуға, вирусқа, бактерияға қарсы және адам денсаулығына басқа да
пайдалы қасиеттері бар бетулин және оның туындылары сияқты
тритерпеноидтардың құнды көзі болып табылады.

{\bfseries Нәтижелер және талқылау.} Қайың әр түрлі салалар үшін бағалы
шикізат болып табылады және оның барлық бөлшкетері -- бұтақтары,
бүршіктері, жапырақтары, қайың қабығы, шырындары әсіресе сыртқы қабығы
кеңінен пайдаланылады {[}3{]}.

Эксперименттік зерттеулер бетулин сүт ортасының ашытуын өзгертпейтінін,
сүтте кездесетін аминқышқылдарымен әрекеттеспейтінін және жақсы
тұрақтандырғыш бола отырып, сүт пен сүтқышқылды өнімдердің
қауіпсіздігіне әсер ететін аминқышқылдарының тұрақтандырғыш қасиеттерін
күшейтетінін көрсетті. Соңғысы термиялық өңдеуге ұшыраған сүт пен
сүтқышқылды өнімдердің сақтау мерзімін ұзарту үшін де маңызды, өйткені
сүт өнімдеріндегі сүт пен сүт фракциясының бөлінуіне әкелетін
процестерге әсер ететін аминқышқылдарының тұрақтандырғыш қасиеттерінің
төмендеуі олардың сақтау мерзімін қысқартады. Сүт өңдеу зауыттарында
сүтті және сүт өнімдерін өңдеудің технологиялықпроцесінде бетулинді
қолдану гомогенизация процесінің ұзақтығын қысқартуға мүмкіндік береді,
осылайша нативті ксантиноксидазаның сүттің сулы фазасына түсу
ықтималдығын азайтады. Бетулин иіссіз, сондықтан сүттің органолептикалық
қасиеттерін өзгертпейді, оның өзіне тән иммуномодуляциялық, гастро және
гепатопротекторлық қасиеттеріне байланысты оның тағамдық құндылығын
арттырады. Қайың қабығы сығындысының сүт қышқылы микроорганизмдерінің
дамуына әсері зерттелді {[}4{]}.

Қайың ағаштарының кең таралғандығына байланысты бұл ағаштың қабығы
өнеркәсіптік масштабта тритерпеноидтарды алу үшін қолжетімді шикізат
болып табылады. Сонымен қатар, қайың қабығын пайдалану қоршаған ортаға
зиян тигізбейді, өйткені оны өңдеу улы химиялық заттарды қолдануды
қажететпейді. Көптеген зерттеулер қайың қабығынан алынған
тритерпеноидтардың тиімділігі мен қауіпсіздігін растайды, бұл оны
дәрілік заттар мен тағамдық қоспаларды өндіру үшін құнды шикізат етеді.

Зерттеуде келесі негізгі аналитикалық жабдықтар пайдаланылды:Тазартылған
және кептірілген қайын қабығын ұнтақтау мақсатында зертханалық ұсақтағыш
dr. Корнер, II модельі қолданылды.

Қайың қабығын активтендіру үшін Ningbo Lawson Smarttech компаниясы
шығарған маркасы DH-3000F-370.Ультрадыбыстыдиспергатор қолданылды.
Құрылғының корпусы тот баспайтын болаттан жасалған, бұл оның беріктігін
және коррозияға төзімділігін қамтамасыз етеді. Құрылғының ішкі бөлігі
ультрадыбыстық түрлендіргіш циркуляциялық сорғы мен өңдеу контейнерінен
тұрады. DH-3000F-370 ультрадыбыстық толқындар әсерінен пайда болатын
кавитация принципі бойынша жұмыс істейді {[}5{]}.

Қайың қабығын экстракциялап тритерпеноидтарды бөліп алу мақсатында
Сокслет аппараты қолданылды. Сокслет экстракторы үздіксіз еріткіш
экстракция процесін пайдалана отырып, қатты шикізаттан нақты
қосылыстарды бөліп алу үшін пайдаланылатын негізгі құрал болып табылады.
Экстрактор штативтен, конденсатордан және колба мен құмды моншадан
тұрады. Экстрагент іретінде келесі еріткіш спирттер пайдаланылды:
этанол, ацетон, хлораформ метанол. Қабықтан бөлінген экстрактіні тазалап
тек тритерпеноидтарды бөліп алу үшін бағанлы хроматография мен ЖҚХ әдісі
қолданылды. Тритерпеноидтарды жұқа қабат хроматографиясы (ЖҚХ/ТСХ)
арқылы сәйкестендіру үшін, PTSKH-AF-A алюминий субстратында Sorbfil
маркалы пластиналарды қолдандық. Элюция жүйесі іретінде гексан
қолданылды {[}6{]}.

Ультрадыбыстық белсендендіруді пайдаланып, эксперименттер жасау әдісі
оңтайландырылды. Ультрадыбыстық қуаты, өңдеу уақыты және экстракция
температурасының параметрлері өзгертілді. Тритерпеноидтерді алу
тиімділігі ең жоғары 150 Вт қуатта, 30 минут ультрадыбыстық өңдеу уақыты
және 50 \tsp{0}С температурада жетті. Осы жағдайда шығым
классикалық әдістермен салыстырғанда 35\% дейін артты.

Тиртерпеноидты сандық талдау үшін жоғары тиімділікті сұйықты
хроматография ултокүлгін детокторы қолданылды. Ультродыбыстық экстракция
мақсатты қосылыстардың жоғары концентрациясын қамтамасыз етті:
бетулинның мөлшері құрғақ шикізаттың 1 грамында 12,5 мг-ға жетті.

ИҚ-спектрофтометр сығындыларға тән тритерпеноидты функционалды топтардың
болуын нақтылады.

Оңтайлы ультрадыбыстық белсендіру параметрлерін пайдалану арқылы алынған
сығындылар маңызды биологиялық белсенділікті көрсетті. Олар
тритерпеноидтардыңжоғары концентрациясына жататын DPPH талдауларында
айқын антиоксиданттық әсер көрсетті.

Зерттеудің негізгі нысаны тритерпеноидтар зертханада алынғандықтан
қосылыстарды бөліп алу барысында қателіктер кетеді. Осы себепті
тритерпеноидтардың химиялық құрылымынның дәлдігін анықтау үшін WTW
компаниясышығарған, маркасы PhotoLab 7100 спектрометрі қолданылды.
Тритерпеноидты қаңқалардың бірнеше түрі ИҚ спетрометр арқылы анықталды.
Қайың қабығынан тритерпеноидтарды бөліп алудың технологиясы 1-суретке
сәйкес келеді.
\end{multicols}

\fig{p/image4}[1 - сурет. Қайынның қабығынан тритерпеноидтарды бөліп
алудың технологиялық сұлбасы]

Алынған сығындының тазалығын тәжірибенің дәлдігін тексеру және әр
тритерпеноидтың сығындыдағы мөлшерін анықтау үшін спектрофотометрия
әдісі қолданылды.

\fig{p/image5}[2 - сурет. Бетулин мен оның қосылыстарының спектрі]

\begin{multicols}{2}
Қайың сығындысының құрамында гидроксил және α, β қаныққан карбонилді
қосылыстағы топтардан тұратын тритерпоноид бар екендігі анықталды.
3200-3500 см (-1) аймағындағы кең жолақ гидросикл топтарындағы О-Н
баланыстарының созылу тербелістердің жұтылуына сәйкес келеді. Ол бетулин
молекуласында бірнеше гидропкил топтарының болатынын көрсетеді.
1680-1690 см (-1) аймағындағы жолақ α, β қанықпаған карбонилді
қосылыстағы С=О байланысының созылу тербелістерінің жұтылуына сәйкес
келеді. Ол дегеніміз қайың сығындысындағы бетулинді құрайтын α, β
қанықпаған кетондарға тән екені анықталды. Қайың сығындысының
құрамындағы функционалдық топтар 1-кестеде көрсетілген.
\end{multicols}

\tcap{1 - кесте. Спектр негізіндегі қайың сығындысының құрамындағы функционалды топтар}
\begin{longtblr}[
  label = none,
  entry = none,
]{
  width = \linewidth,
  colspec = {Q[15]Q[137]Q[298]Q[463]},
  cells = {c},
  cells = {font = \small},
  hlines,
  vlines,
}
№ & \textbf{Тритерпеноид атауы} & \textbf{Тритерпеноидтарға сәйкес шыңдар (пик) (см\^(-1))} & \textbf{Интерпертация}                                                            \\
1 & Бетулин                     & 17,6903                                                   & α, β қанқықпаған карбонилді қосылыстағы карбонил тобының созылу тербелісі.        \\
2 & Бетанол                     & 16,4584                                                   & α, β қанықпаған қосылыстардағы тербелісі С=C қас байланыстың созылу тербелістері. \\
3 & Лупеол                      & 14,8489                                                   & Метилен топтарының деформациялық тербелістері.                                    \\
4 & Олеанол қышқылы             & 13,6036                                                   & Метил топтарының деформация тербелісі.                                            \\
5 & Бетулин қышқылы             & 11,3072                                                   & Карбоксил топтарындағы С-О байланысының созылу тербелістері.                      
\end{longtblr}

\begin{multicols}{2}
Бетулинді басқа ұқсас қосылыстрдан ерекшелендіретін негізгі
артықшылықтары, қолжетімді шикізат базасы, шикізаттағы негізгі заттың
жоғары мөлшері, тритерпеноитарды бөліп алудың қарапайымдылығы. Бетулин
-- ол дәмсіз, иіссіз ұнтақ, түсі ақшыл ашық. Бетулиннің балқу
температурасы 240-260\tsp{0}С, молекуласының инертті
қасиеттері қасиеттерін өзгертпей ұзақ сақтау мерзімін қамтамасыз етеді,
улы емес, оттегі мен күн сәулесіне төзімді. Бетулин органикалық
ерткіштерде ериді, эмульциялаушы және құрылым түзуші қасиетке ие, май
және май эмульциясын түзелі. Мұндай технологиялық қасиеттер бетулинді
зерттеушілер үшін тиімді етеді, өйткені олардайын өнімнің дәміне әсер
етпейді және пробиотиктерге қарағанда өнімді термиялық өңдеуге мүмкіндік
береді. Сығындыны қолданудың қауіпсіздігі В.В. Закусов атындағы Ресей
медицина ғылымдары академиясының Фармакология ғылыми зерттеу
институтының дәрілік токсиология зертханасында, жаңа фармакологиялық
қойылатын талаптарға сәйкес дәлелденеді. Экстракт улы, мутагенді емес,
репродуктивті, уыттылық және аллергендік қасиеті жоқ. Осы аталған
зертханада қабынуғы қарсы қаиеттері мен ашық антиаллергиялық әсерлері
расталды. Халықаралық жіктеу бойынша ол аз уытты заттардың төртініші
класына жатады. Тағамдық қоспалар және тамақ өнеркәсібі үшін шикізат
ретінде бетулиннің микробиологиялық көрсеткіштеріне жүргізілен
зерттеулер бетулиннің СанЕмН 2.3.2.1078 талаптарына сәйкес келеді
{[}7{]}.

Азық-түлікке бетулин қоспасын енгізу өнімнің сапасын жақсартады
антиоксиданттық әсері бар өнімдерге енгізгенде қышқыл мен пероксид санын
азайтады, өнімнің сақтау мерзімін арттырады. Бетулиннің күнделікті
диеталық өнімдерге қосылуы функционалы пайдалы тағам өнімдерін, арнайы
мақсаттағы функционалды өнімдерді әзірлеуге мүмкіндік береді. Бетулин
қосылған функционалдық тағамдарды тұтыну кезінде адам ағзасына
емдік-профилактикалық әсер етеді. Толықтықтан, асқазан-ішек жолдарының
ауруларынан қорғайды, қан ментіндердегі холестеринді, қатерлі ісік және
басқа да көптеген аурулардың пайда болу қаупін төмендетеді {[}8{]}.

Ақ қайың (Betula pendula) қабығы -- тритерпеноидтарға, әсіресе бутулин
мен оның туындыларына бай табиғи шикізат көзі екені анықталған. Бұл
қосылыстар микробтарға қарсы, қабынуға қарсы және антиоксиданттық
қасиеттерімен ерекшеленеді. Ол дегеніміз, функционалды тағам өнімдерін
байытуда және табиғи қоспалар ретінде қолдануға мүмкіндік береді.
Ультрадыбыстық активацияны қолдана отырып, қайың қабығынан
тритерпеноидтарды тиімді бөліп алу үшін оңтайлы технологиялық
параметрлер анықталды.

Ультрадыбыстық экстракцияның артықшылықтары: Экологиялық қаупсіз --
органикалық еріткіштердің көлемі азаяды; Қысқа уақыт ішінде жоғары
шығымы -- жасуша қабырғаларын тиімді бұзады; Дәстүрлі әдістермен
салыстырғанда энергия шығыны төмен; Биологиялық белсенді заттар
ыдырамайды.
\end{multicols}

\tcap{2 - кесте. Оңталы параметрлерді таңдау}
\begin{longtblr}[
  label = none,
  entry = none,
]{
  width = \linewidth,
  colspec = {Q[10]Q[154]Q[231]Q[152]},
  cells = {c},
  cells = {font = \small},
  cell{2}{4} = {r=6}{},
  vlines,
  hline{1-2,8} = {-}{},
  hline{3-7} = {1-3}{},
}
№ & \textbf{Параметр}          & \textbf{Ұсынылатын диапазон}             & \textbf{Ескерту}                                                                               \\
1 & Ультрадыбыстық қуат        & 100-300 Вт                               & Параметрлер шикізаттың ұнтақталу дәрежесіне және тиртерпеноид түріне байланысты өзгеруі мүмкін \\
2 & Жұмыс жиілігі              & 10-40 кГц                                &                                                                                                \\
3 & Экстракция уақыты          & 20-40 минут                              &                                                                                                \\
4 & Еріткіш түрі               & Этанол (70\%) немесе эанол-су қоспасы    &                                                                                                \\
5 & Шикізат / еріткіш қатынасы & 1:10 – 1:20 (массалық қатынас)           &                                                                                                \\
6 & Температура                & 30-50 \textsuperscript{0}С (жоғары емес) &                                                                                                
\end{longtblr}

\begin{multicols}{2}
Ультрадыбыстық активация -- қайың қабығынан тритерпеноидтарды бөліп
алудың тиімді, жылдам және экологиялық таза әдісі. Технологиялық
параметрлерді оңтайландыру арқылы өнімнің шығымын арттырып, сапалы
биологиялық белсенді қоспалар алуға мүмкіндік береді. Авторлардың
зерттеулері бойынша ултардыбыспен өсімдіктен экстарция жасаудың
тиімділігі экдистероидтар мен флаваноидтардың шығымдарымен негізделген
{[}9{]}.

Физика-химиялық қасиеттеріне, сондай-ақ биологиялық фармаколиялық
белсенділігінің кең спектріне байланысты бетулин әр түрлі тағам
өнімдеріне қосуға болатын переспективалы табиғи биологиялық белсенді
қоспа болып табылады. Сол себепті де тритерпеноидтар суда нашар
еритіндігінен майлы сүт өнімдеріне тағамдық қоспа ретінде енгізген жөн.
Эксперименттік зерттеулер бетулин сүт ортасының ұюын өзгертпейтінін,
сүтте кездесетін амин қышқылдарымен әрекеттеспейтінін және жақсы
тұрақтандырғыш бола отырып сүт өнімдерінің қаупсіздігіне әсер ететін
аминқышқылдарының тұрақтандырығыш қасиеттерін күшейтетінін көрсетті.
Соңғысы термиялық өңдеуге ұшыраған сүт өнімдерінің сақтау мерзімін
ұзарту үшін де маңызды. Өйткені сүт өнімдеріндегі сүт пен сүт
фракциясының бөлінуіне әкелетін үдерістерге әсер ететін аминқышқыларының
тұрақтандырғыш қасиеттерінің төмендеуі олардың сақтау мерзімін
қысқартады. Сүт және сүтқышқылды өнімдерді өңдеу зауттарында сүт
өнімдерін өңдеудің технологиялық үдерістерінде бетулинді қолдану
гомогенизация үдерісінің ұзақтығын қысқартуға мүмкіндік береді {[}10{]}.

{\bfseries Қорытынды.} Экстракция процессі мен экстрактінің сапасына
тікелей әсер ететін барлық параметрлер: қайың қабығының ылғалдылығы мен
ұнтақталу дәрежесі, температура ультрадыбыстың жиілігі мен қуаты,
экстракция уақыты зерттеліп, қайың қабығынан тритерпеноидтарды бөліп
алудың оңтайлы параметрлері анықталды. Сығындының антооксиданттық
белсенділігіне талдау жүргізіліп, тритерпеноидтар тотығу стрессін және
жасуша зақымдануын тудыратын бос радикалдарды бейтараптандырады.
Жасушаларды ультракүлгін сәулеленуден, токсиндерден және басқа зиянды
факторлардан туындаған зақымданудан қорғап, иммундық жүйені күшейтетіні
анықталды.

Сығындыны сүт қышқылды өнімдерге қосудың тиімділігі қымыздың сақтау
мерзімін 15 күнге ал айраннынның сақтау мерзімін 7 күннен 10 күнге
ұзартатыны анықталды.
\end{multicols}

\begin{center}
{\bfseries Әдебиеттер}
\end{center}

\begin{refs}
1. Погребняк Л. В., Погребняк А. В. Перспективы использования внешней
коры деревьев и кустарников семейства берёзовые (Betulaceae) в качестве
источника биологически активных и вспомогательных веществ //Известия
Самарского научного центра Российской академии наук. -2015. - Т.17
(5-1). - С.174-178.

2. Юферова А. А., Сударева М. А., Дубняк Я. В. Применение природных
антиоксидантов в технологии молочных продуктов//Технологии пищевой и
перерабатывающей промышленности АПК-продукты здорового питания. - 2021.
- №.2. - С.98-107.

3. Косяков Д. С., Ульяновский Н. В., Фалев Д. И. Определение
тритерпеноидов коры березы методом жидкостной тандемной
хроматомасс-спектрометрии //Масс-спектрометрия. - 2013. - Т.10(4). - С.
237-242.

4. Белякова А. Ю., Погребняк А. В., Погребняк Л. В. Физико-химические и
биологические свойства компонентов внешней коры березы //Современные
проблемы науки и образования. -- 2015. - №.2(2). -С.492-492.

5. Зобкова З.С., Федотова О.Б., Фурсова Т.П., Зенина Д.В., Гаврелина
А.Д., Щелигинова Н.Р. Исследование антимикробных свойств
бетулиносодержащего экстракта в молочных продуктах //Молочная
промышленность. - 2017. - №.1. - С.50-52.

6. Тургенбаева Ш.Ш. Хатибаева А.Ш. Получение экстрактивных веществ
берёзы. // Universum: химия и биология электронный научный журнал.
-2020. № 8 (74).

7. Миназова Г. И. Тонкослойная хроматография в анализе природного сырья
//Башкирский химический журнал. - 2010.- Т.17(5). - С.105-107.

8. Юферова А. А., Сударева М. А., Дубняк Я. В. Применение природных
антиоксидантов в технологии молочных продуктов//Технологии пищевой и
перерабатывающей промышленности АПК--продукты здорового питания.- 2021.-
№.2.- С.98-107.

9. Зибарева Л. Н., Филоненко Е. С. Влияние ультразвукового воздействия
на экстракцию биологически активных соединений растений семейства
Caryophyllaceae //Химия растительного сырья. - 2018. - №.2. - С.
145-151.

10. Москалев Е. В., Поняев А. И. Здоровые продукты питания с применением
биологически активной добавки-бетулина // Балтийский морской форум:
материалы VII Междунар. Балт. мор. форума, 7-12 окт.2019 г.- 2019. -
Т.5.- С.73-77.
\end{refs}

\begin{center}
{\bfseries References}
\end{center}

\begin{refs}
1. Pogrebnjak L. V., Pogrebnjak A. V. Perspektivy ispol' zovanija
vneshnej kory derev' ev i kustarnikov semejstva berjozovye
(Betulaceae) v kachestve istochnika biologicheski aktivnyh i
vspomogatel' nyh veshhestv //Izvestija Samarskogo nauchnogo centra
Rossijskoj akademii nauk. -2015.  - T.17 (5-1). - S.174-178. {[}in
Russian{]}

2. Juferova A. A., Sudareva M. A., Dubnjak Ja. V. Primenenie prirodnyh
antioksidantov v tehnologii molochnyh produktov//Tehnologii pishhevoj i
pererabatyvajushhej promyshlennosti APK-produkty zdorovogo pitanija. -
2021. - №.2. - S.98-107. {[}in Russian{]}

3. Kosjakov D. S., Ul' janovskij N. V., Falev D. I.
Opredelenie triterpenoidov kory berezy metodom zhidkostnoj tandemnoj
hromatomass-spektrometrii //Mass-spektrometrija. - 2013. - T.10(4). -
S.237-242. {[}in Russian{]}

4. Beljakova A. Ju., Pogrebnjak A. V., Pogrebnjak L. V.
Fiziko-himicheskie i biologicheskie svojstva komponentov vneshnej kory
berezy //Sovremennye problemy nauki i obrazovanija. -- 2015. - №.2(2).
-S.492-492. {[}in Russian{]}

5. Zobkova Z.S., Fedotova O.B., Fursova T.P., Zenina D.V., Gavrelina
A.D., Shheliginova N.R. Issledovanie antimikrobnyh svojstv
betulinosoderzhashhego jekstrakta v molochnyh produktah //Molochnaja
promyshlennost'. - 2017. - №.1. - S.50-52. {[}in
Russian{]}

6. Turgenbaeva Sh.Sh. Hatibaeva A.Sh. Poluchenie jekstraktivnyh
veshhestv berjozy. // Universum: himija i biologija jelektronnyj
nauchnyj zhurnal. -2020. № 8 (74).

7. Minazova G. I. Tonkoslojnaja hromatografija v analize prirodnogo
syr' ja //Bashkirskij himicheskij zhurnal. - 2010.-
T.17(5). - S.105-107. {[}in Russian{]}

8. Juferova A. A., Sudareva M. A., Dubnjak Ja. V. Primenenie prirodnyh
antioksidantov v tehnologii molochnyh produktov//Tehnologii pishhevoj i
pererabatyvajushhej promyshlennosti APK--produkty zdorovogo pitanija.-
2021. - №.2.- S.98-107. {[}in Russian{]}

9. Zibareva L. N., Filonenko E. S. Vlijanie
ul' trazvukovogo vozdejstvija na jekstrakciju
biologicheski aktivnyh soedinenij rastenij semejstva Caryophyllaceae
//Himija rastitel' nogo syr' ja. - 2018. -
№.2. - S.145-151. {[}in Russian{]}

10. Moskalev E. V., Ponjaev A. I. Zdorovye produkty pitanija s
primeneniem biologicheski aktivnoj dobavki-betulina // Baltijskij
morskoj forum: materialy VII Mezhdunar. Balt. mor. foruma, 7-12 okt.
2019 g.- 2019. - T.5.- S.73-77. {[}in Russian{]}
\end{refs}

\begin{info}
\hspace{1em}\emph{{\bfseries Сведения об авторах}}

Карденов С.А. - т.ғ.к. қауым. профессор м.а., С.Сейфуллин атындағы Қазақ
агротехникалық зерттеу университеті, Астана, Қазақстан, e-mail:
Askerbekovsk@mail.ru;

Байтукенова С.Б. - т.ғ.к., қауым. профессор, Қ.Құлажанов атындағы
Қазақ технология және бизнес университеті, Астана, Казахстан, e-mail:
~saule7272 @mail.ru;

Байтукенова Ш.Б. - т.ғ.к., қауым. профессор, С.Сейфуллин атындағы
Қазақ агротехникалық зерттеу университеті, Астана, Қазақстан, e-mail:
baytukenova75@mail.ru;

Базылханова Э.Ч. - PhD доктор, қауым. профессор м.а., С.Сейфуллин
атындағы Қазақ агротехникалық зерттеу университеті, Астана, Қазақстан,
e-mail: 66bel@bk.ru;

Ажгереева Ж.С. - т.ғ.м., оқытушы С.Сейфуллин атындағы Қазақ
агротехникалық зерттеу университеті, Астана, Қазақстан, e-mail:
zhuldyz\_09.11@mail.ru.

\hspace{1em}\emph{{\bfseries Information about the authors}}

Kardenov S.A. - Candidate of Technical Sciences, Acting Associate
Professor, S.Seifullin Kazakh Agrotechnical Research University,
Astana, Kazakhstan, e-mail: Askerbekovsk@mail.ru;

Baitukenova S.B. - Candidate of Technical Sciences, Associate
Professor, K.Kulazhanov Kazakh University of Technology and Business,
Astana, Kazakhstan, e-mail: s.baitukenova@mail.ru;

Baitukenova Sh.B. - Candidate of Technical Sciences, Associate
Professor, S.Seifullin Kazakh Agrotechnical Research University,
Astana, Kazakhstan, e-mail: baytukenova75@mail.ru;

Bazylkhanova E.Ch. - PhD doctor, Acting Associate Professor,
S.Seifullin Kazakh Agrotechnical Research University, Astana,
Kazakhstan, e-mail: 66bel@bk.ru;

Azhgereeva Zh.S. - Master of Technical Sciences, lecturer, S.Seifullin
Kazakh Agrotechnical Research University, Astana, Kazakhstan, e-mail:
zhuldyz\_09.11@mail.ru.
\end{info}
