\id{МРНТИ 65.59.91}{}

{\bfseries РАЗРАБОТКА И ИССЛЕДОВАНИЕ ФУНКЦИОНАЛЬНО-ТЕХНОЛОГИЧЕСКИХ СВОЙСТВ
БЕЛКОВО-ЖИРОВОЙ ЭМУЛЬСИИ ИЗ КУРИНЫХ СУБПРОДУКТОВ}

{\bfseries \tsp{1}А.К.
Суйчинов}{\bfseries ,
\tsp{2}Э.К.
Окусханова}{\bfseries ,
\tsp{1}Г.А.
Капашева}{\bfseries \envelope 
,
\tsp{1}Ж.С.Есимбеков}

\emph{\tsp{1}Семейский филиал ТОО «Казахский
научно-исследовательский институт перерабатывающей и пищевой
промышленности», Семей, Казахстан,}

\emph{\tsp{2} НАО «Шәкәрім Университеті», Семей, Казахстан}

\corrauthor{Корреспондент-автор:gena.89.89@mail.ru}

В статье представлена разработка белково-жировой эмульсии (БЖЭ) на
основе субпродуктов птицы: куриной кожи, куриного жира и белкового
гидролизата. Целью исследования являлось создание стабильной
эмульсионной системы с оптимальными функционально-технологическими
свойствами для использования в составе мясных продуктов. Были изучены
четыре варианта рецептуры с разным содержанием куриного жира
(13,5-15,0\,\%), гидролизата (5,3- 6,8\,\%), при постоянном содержании
куриной кожи (36,4\,\%) и воды (43,3\,\%). Активность воды в образцах
БЖЭ составила от 0,9798 до 0,9968, что указывает на высокую влажность и
необходимость строгого соблюдения условий хранения. Значения pH
варьировались в пределах 4,82-4,99, соответствуя требованиям
микробиологической безопасности. Наиболее высокая влагоудерживающая
способность (78,74\,\%) зафиксирована у варианта 1, в то время как
жироудерживающая способность достигла максимума у варианта 4 (62,4\,\%).
Эмульгирующая способность увеличивалась от 46\,\% до 56\,\%, а
стабильность эмульсии колебалась от 10\,\% до 22\,\%. Органолептические
исследования показали, что все образцы имели однородную консистенцию и
характерный мясной аромат, при этом вариант 3 был признан наиболее
сбалансированным по внешнему виду, текстуре и запаху. Таким образом,
рецептура 3 обеспечивает оптимальное соотношение функциональных и
органолептических свойств. Разработанная технология БЖЭ позволяет
рационально использовать побочное сырье птицы, снижать себестоимость
продукции и расширять ассортимент мясных изделий за счёт внедрения
безотходных технологий.

{\bfseries Ключевые слова:} куриная шкурка, куриный жир, активность воды,
pH, белково-жировая эмульсия, гидролизат, функционально-технологические
свойства.

{\bfseries ТАУЫҚ СУБӨНІМІНЕН АЛЫНҒАН АҚУЫЗ-МАЙ ЭМУЛЬСИЯСЫНЫҢ
ФУНКЦИОНАЛДЫҚ-ТЕХНОЛОГИЯЛЫҚ ҚАСИЕТТЕРІН ӘЗІРЛЕУ ЖӘНЕ ЗЕРТТЕУ}

{\bfseries \tsp{1}А.К. Суйчинов, \tsp{2}Э.К.
Окусханова, \tsp{1}Г.А. Капашев\envelope ,
\tsp{1}Ж.С.Есимбеков}

\emph{\tsp{1}\href{https://rpf.kz/?lang=kk}{Қaзақ қайта
өңдеу және тағам өнеркәсіптері ғылыми зерттеу институты}}

\emph{(Семей филиалы), Семей, Қазақстан,}

\emph{\tsp{2}«Шәкәрiм Университетi» КЕҚ, Семей, Қазақстан,}

e-mail: gena.89.89@mail.ru

Мақалада құс субөнімдеріне негізделген ақуыз-май эмульсияның (АМЭ)
жасалуы ұсынылған. Құрамында тауық терісі, тауық майы және ақуыз
гидролизаты бар, бұл эмульсияны әзірлеудегі мақсат - ет өнімдерінде
қолдануға жарамды, тұрақты және оңтайлы функционалды-технологиялық
қасиеттері бар эмульсиялық жүйе жасау. Зерттеу барысында тауық майы
(13,5-15,0 \%) мен гидролизаттың (5,3-6,8 \%) мөлшері әртүрлі төрт
рецептура нұсқасы қарастырылды, ал тауық терісі (36,4 \%) мен су (43,3
\%) тұрақты компоненттер ретінде сақталды. АМЭ үлгілеріндегі су
активтілігі 0,9798-0,9968 аралығында болды, бұл олардың ылғалдылығының
жоғары екенін және сақтау шарттарын қатаң сақтаудың маңыздылығын
көрсетеді. pH мәндері 4,82-4,99 шегінде ауытқып, микробиологиялық
қауіпсіздік талаптарына сәйкес келді. Ең жоғары ылғал ұстау қабілеті
1-нұсқада (78,74 \%) байқалса, ең жоғары май ұстау қабілеті 4-нұсқада
(62,4 \%) тіркелді. Эмульгирлеу қабілеті 46 \%-дан 56 \%-ға дейін артты,
ал эмульсия тұрақтылығы 10 \%-дан 22 \%-ға дейін ауытқыды.
Органолептикалық зерттеулер барлық үлгілердің біртекті консистенциясы
мен тән ет иісіне ие екенін көрсетті, ал 3-нұсқа сыртқы түрі, құрылымы
және иісі жағынан ең теңгерімді деп танылды. Осылайша, 3-рецептура
функционалды және органолептикалық қасиеттердің оңтайлы арақатынасын
қамтамасыз етеді. Жасалған АМЭ технологиясы құс қалдықтарын тиімді
пайдаланып, өнімнің өзіндік құнын төмендетуге және ет өнімдерінің
ассортиментін кеңейтуге мүмкіндік береді.

{\bfseries Түйінді сөздер:} тауық терісі, тауық майы, су активтілігі, pH,
белокты-майлы эмульсия, гидролизат, функционалды-технологиялық
қасиеттер.

{\bfseries DEVELOPMENT AND RESEARCH OF FUNCTIONAL AND TECHNOLOGICAL
PROPERTIES OF PROTEIN-FAT EMULSION FROM CHICKEN OFFAL}

{\bfseries \tsp{1}A. K. Suichinov, \tsp{2}E.K.
Okuskhanova, \tsp{1}G. A. Kapasheva\envelope ,
\tsp{1}Zh.S.Yesimbekov}

\emph{\tsp{1}Kazakh Scientific Research Institute of
Processing and Food Industry}

\emph{(Semey branch), Semey, Kazakhstan,}

\emph{\tsp{2} NJC «Shakarim University», Semey Kazakhstan,}

e-mail: :
gena.89.89@mail.ru

This article presents the development of a protein-fat emulsion (PFE)
based on poultry by-products: chicken skin, chicken fat, and protein
hydrolysate. The aim of the study was to create a stable emulsion system
with optimal functional and technological properties for use in meat
products. Four formulation variants were studied with varying contents
of chicken fat (13.5-15.0\%) and hydrolysate (5.3-6.8\%), while
maintaining constant levels of chicken skin (36.4\%) and water (43.3\%).
The water activity in the PFE samples ranged from 0.9798 to 0.9968,
indicating high moisture content and the need for strict storage
conditions. The pH values varied between 4.82 and 4.99, meeting
microbiological safety requirements. The highest water-holding capacity
(78.74\%) was recorded in variant 1, while the fat-holding capacity
reached a maximum in variant 4 (62.4\%). The emulsifying capacity
increased from 46\% to 56\%, and the emulsion stability ranged from 10\%
to 22\%. Sensory evaluations showed that all samples had a uniform
consistency and characteristic meat aroma, with variant 3 being
recognized as the most balanced in terms of appearance, texture, and
aroma. Thus, formulation 3 provides an optimal balance of functional and
sensory properties. The developed PFE technology allows for the rational
use of poultry by-products, reducing product cost and expanding the
range of meat products through the implementation of zero-waste
technologies.

{\bfseries Keywords:} chicken skin, chicken fat, water activity, pH,
protein-fat emulsion, hydrolysate, functional and technological
properties.

{\bfseries Введение.} В современной пищевой промышленности одной из
приоритетных задач является повышение качества продукции при
одновременном снижении себестоимости. В этом направлении особое значение
приобретает эффективное использование побочного сырья, образующегося при
переработке мяса птицы, включая куриную кожу. Являясь дешёвым и богатым
белком сырьём, куриная кожа встречается в больших объёмах, однако
высокое содержание жировой и соединительной тканей ограничивает её
прямое применение в готовых мясных изделиях {[}1{]}. Для преодоления
этих ограничений предлагается использовать куриную кожу в составе
белково-жировых эмульсий. Такой подход позволяет вовлечь в переработку
низкосортное, но функционально ценное сырьё, экономить качественное
мясное сырьё и расширять ассортимент мясной продукции {[}2{]}.

Совершенствование технологий переработки мясного сырья, включая
субпродукты, представляет собой не только производственную, но и
социально значимую задачу. Эффективное использование имеющихся ресурсов
открывает возможности для создания нового поколения качественных и
доступных пищевых продуктов {[}3{]}. Использование куриной кожи при
создании белково-жировой эмульсии обеспечивает пищевую промышленность
недорогим белоксодержащим сырьём, позволяет высвободить часть
высококачественного мясного сырья, а также расширить ассортимент мясной
продукции. В связи с этим применение куриной кожи в производстве
белково-жировой эмульсии представляет собой актуальное направление
безотходной технологии {[}4{]}.

Куриная кожа считается одним из основных видов побочного сырья,
образующегося в процессе переработки мяса птицы. Её химический состав
характеризуется высоким содержанием жира - около 45 \% {[}5{]} и белка -
около 9 \% (во влажной массе), что делает её перспективным источником
для получения липидов, белков и их производных. Особенно ценна кожа с
области шеи и бёдер, так как она содержит 1-17 \% белков, 20-25 \% жира,
а также обогащена витаминами (A, B1, B2, B3, PP, C, E) и кальцием
{[}6,7{]}.

В производстве мясной продукции широко используются эмульсии на основе
белково-жировых и водно-жировых смесей, способствующие улучшению
текстурно-механических характеристик, функционально-технологических
показателей, а также повышению пищевой и биологической ценности изделий
{[}8{]}. Эмульсия представляет собой многокомпонентную дисперсную
систему, в которой дисперсной фазой является жир, а дисперсионной средой
--- вода {[}1{]}. Для достижения наилучших результатов при использовании
таких систем важно обеспечить совместимость ингредиентов за счёт
рационального подбора их функциональных свойств {[}9{]}.

Таким образом, куриная кожа, являясь побочным продуктом переработки мяса
птицы, представляет собой ценный источник белка, жира, витаминов и
минеральных веществ. Несмотря на морфологические особенности,
затрудняющие её прямое использование в мясных изделиях, создание
белково-жировой эмульсии на основе куриной кожи является эффективным
технологическим решением. Белково-жировая эмульсия способствует
стабилизации сырья, улучшению функционально-технологических свойств
мясных продуктов, а также снижению себестоимости производства за счёт
вовлечения низкокачественного сырья. Такой подход расширяет ассортимент
мясных изделий, обеспечивает рациональное использование ресурсов и
способствует внедрению безотходных технологий, что делает его актуальным
направлением в пищевой промышленности.

Целью данной работы является разработка белково-жировой эмульсии из
субпродуктов птицы и оценка её функционально-технологических свойств.

{\bfseries Материалы и методы.} Объектами исследования являлись
белково-жировые эмульсии из субпродуктов птицы, приготовленные с
использованием куриного жира, куриной кожи, белкового гидролизата и воды
по четырём рецептурным вариантам для каждого типа продукции.

В рамках разработки белково-жировой эмульсии на основе куриных
субпродуктов были сформированы четыре рецептурных варианта с
изменяющимся содержанием куриного жира и гидролизата. Постоянными
компонентами служили куриная кожа и вода, обеспечивающие стабильность
текстурных и эмульгирующих характеристик. Изменение концентраций
жирового компонента и белкового гидролизата позволило оценить их влияние
на функционально-технологические свойства готовой эмульсии. В таблице 1
представлены разные варианты рецептуры.

{\bfseries Таблица 1 -- Состав белково-жировой эмульсии}

%% \begin{longtable}[]{@{}
%%   >{\raggedright\arraybackslash}p{(\linewidth - 8\tabcolsep) * \real{0.3139}}
%%   >{\centering\arraybackslash}p{(\linewidth - 8\tabcolsep) * \real{0.1906}}
%%   >{\centering\arraybackslash}p{(\linewidth - 8\tabcolsep) * \real{0.1461}}
%%   >{\centering\arraybackslash}p{(\linewidth - 8\tabcolsep) * \real{0.1589}}
%%   >{\centering\arraybackslash}p{(\linewidth - 8\tabcolsep) * \real{0.1906}}@{}}
%% \toprule\noalign{}
%% \multirow{2}{=}{\begin{minipage}[b]{\linewidth}\raggedright
%% {\bfseries Компоненты}
%% \end{minipage}} &
%% \multicolumn{4}{>{\centering\arraybackslash}p{(\linewidth - 8\tabcolsep) * \real{0.6861} + 6\tabcolsep}@{}}{%
%% \begin{minipage}[b]{\linewidth}\centering
%% {\bfseries Вариант рецептуры, \%}
%% \end{minipage}} \\
%% & \begin{minipage}[b]{\linewidth}\centering
%% {\bfseries 1}
%% \end{minipage} & \begin{minipage}[b]{\linewidth}\centering
%% {\bfseries 2}
%% \end{minipage} & \begin{minipage}[b]{\linewidth}\centering
%% {\bfseries 3}
%% \end{minipage} & \begin{minipage}[b]{\linewidth}\centering
%% {\bfseries 4}
%% \end{minipage} \\
%% \midrule\noalign{}
%% \endhead
%% \bottomrule\noalign{}
%% \endlastfoot
%% Куриный жир & 13,5 & 14 & 14,5 & 15 \\
%% Куриная шкурка & 36,4 & 36,4 & 36,4 & 36,4 \\
%% Гидролизат & 6,8 & 6,3 & 5,8 & 5,3 \\
%% Вода & 43,3 & 43,3 & 43,3 & 43,3 \\
%% Итого & 100 & 100 & 100 & 100 \\
%% \end{longtable}

\emph{Методика приготовления белково-жировой эмульсии.}

\emph{Технологический процесс - Подготовка ингредиентов:}

В качестве основных компонентов использовались белковый гидролизат,
вода, куриная кожа и куриный жир. Все ингредиенты предварительно
очищались, а при необходимости - измельчались до требуемого размера.

\emph{Обработка куриной кожи:}

Куриная шкурка замачивалась в 3\%-ном растворе уксусной кислоты при
температуре 4-6 °C в течение 12-24 часов. Такая обработка способствовала
размягчению соединительных тканей и снижению микробной обсеменённости
сырья.

Белковый гидролизат из куриных ног готовили согласно патента № 9667 от
11.10.2024г. Гидролизат предварительно растворяли в воде при температуре
16-18 °C и перемешивали до получения однородной массы. Куриная кожа и
куриный жир измельчались на мясорубке до размера частиц 2-3 мм.
Гидролизат объединяли с измельчённой куриной шкуркой и жиром. Смесь
тщательно перемешивали в течение 1-2 минут до получения однородной
эмульсии. Полученную эмульсионную массу охлаждали до температуры 0-4 °C.
Время охлаждения составляло 1,5-2 часа для стабилизации текстуры и
повышения устойчивости эмульсии. Готовую эмульсию помещали в герметичные
контейнеры и хранили при температуре 0-4 °C не более 48 часов.

Контрольные испытания проводились через 2-4 часа после приготовления.
Оценивались физико-химические показатели, влаго- и жироудерживающая
способность, эмульгирующая способность, стабильность эмульсии, активная
кислотность (pH), вязкость, напряжение сдвига, а также органолептические
характеристики.

\emph{Определение активности воды}

Определение активности воды (aw) - с использованием прибора LabTouch-aw
(Италия) {[}10{]}.

\emph{Определение рН.} Активную кислотность среды (рН) определяли
потенциометрическим методом на приборе
рН-метр-Seven2Go\tsp{TM}, погружением электрода в раствор с
фиксацией значения рН на шкале прибора. Раствор (водную вытяжку)
готовили из измельченного продукта с водой (в соотношении 1:10). рН
измеряли после настаивания в течение 30 минут при температуре
20\tsp{0}С {[}11{]}.

\emph{Органолептическая оценка.} Органолептическую оценку проводили в
соответствии с ГОСТ 31470--2012 {[}12{]}.

\emph{Определение функционально-технологических свойств}

Функционально-технологические свойства белково-жировой эмульсии, такие
как влагоудерживающая способность (ВУС), жироудерживающая способность
(ЖУС), эмульгирующая способность (ЭС) и стабильность эмульсии (СЭ),
определялись по методике Тимошенко и соавторов {[}13{]}.

\emph{Статистический анализ}

Обработку результатов измерений осуществляли с помощью программы
Microsoft Excel 2016. Результаты анализов были статистически значимы при
p ≤ 0,05. Данные представлены как среднее значение ± стандартное
отклонение.

{\bfseries Обсуждение и результаты.} \emph{Исследование активности воды}

С целью оценки микробиологической устойчивости и стабильности
белково-жировых эмульсий, полученных по рецептуре, была проведена
регистрация показателей активности воды (aw) в четырёх вариантах
образцов. Результаты представлены в таблице 2.

{\bfseries Таблица 2 - Результаты измерения активности воды в БЖЭ}

%% \begin{longtable}[]{@{}
%%   >{\centering\arraybackslash}p{(\linewidth - 2\tabcolsep) * \real{0.2500}}
%%   >{\centering\arraybackslash}p{(\linewidth - 2\tabcolsep) * \real{0.7500}}@{}}
%% \toprule\noalign{}
%% \begin{minipage}[b]{\linewidth}\centering
%% {\bfseries № образца}
%% \end{minipage} & \begin{minipage}[b]{\linewidth}\centering
%% {\bfseries Активность воды}
%% \end{minipage} \\
%% \midrule\noalign{}
%% \endhead
%% \bottomrule\noalign{}
%% \endlastfoot
%% Вариант 1 & 0,9798 \\
%% Вариант 2 & 0,9912 \\
%% Вариант 3 & 0,9851 \\
%% Вариант 4 & 0,9968 \\
%% \end{longtable}

Анализ полученных данных показывает, что активность воды в образцах БЖЭ
варьируется в пределах от 0,9798 до 0,9968. Эти значения характерны для
эмульсионных продуктов с высоким содержанием влаги. Показатель aw выше
0,95 свидетельствует о потенциальной возможности роста микроорганизмов,
поэтому при производстве и хранении данных эмульсий необходимо строго
соблюдать температурный режим и сроки хранения.

\emph{Определение активной кислотности среды (pH)}

Для оценки кислотности белково-жировых эмульсий, приготовленных по
четырём различным вариантам рецептуры, была проведена серия измерений
значения рН при комнатной температуре. Результаты представлены в таблице
3. {\bfseries Таблица 3 - Определение концентрации водородных ионов в БЖЭ}

%% \begin{longtable}[]{@{}
%%   >{\centering\arraybackslash}p{(\linewidth - 8\tabcolsep) * \real{0.1737}}
%%   >{\centering\arraybackslash}p{(\linewidth - 8\tabcolsep) * \real{0.2070}}
%%   >{\centering\arraybackslash}p{(\linewidth - 8\tabcolsep) * \real{0.1746}}
%%   >{\centering\arraybackslash}p{(\linewidth - 8\tabcolsep) * \real{0.2232}}
%%   >{\centering\arraybackslash}p{(\linewidth - 8\tabcolsep) * \real{0.2215}}@{}}
%% \toprule\noalign{}
%% \begin{minipage}[b]{\linewidth}\centering
%% {\bfseries Показатель}
%% \end{minipage} & \begin{minipage}[b]{\linewidth}\centering
%% {\bfseries Вариант 1}
%% \end{minipage} & \begin{minipage}[b]{\linewidth}\centering
%% {\bfseries Вариант 2}
%% \end{minipage} & \begin{minipage}[b]{\linewidth}\centering
%% {\bfseries Вариант 3}
%% \end{minipage} & \begin{minipage}[b]{\linewidth}\centering
%% {\bfseries Вариант 4}
%% \end{minipage} \\
%% \midrule\noalign{}
%% \endhead
%% \bottomrule\noalign{}
%% \endlastfoot
%% рН & 4,82 & 4,99 & 4,85 & 4,82 \\
%% t, °C & 25 & 22,9 & 23 & 21,1 \\
%% \end{longtable}

По полученным данным значения рН всех образцов находились в диапазоне
4,82--4,99, что указывает на слабокислую среду в эмульсиях.
Незначительные колебания температуры не оказали существенного влияния на
итоговые показатели кислотности. Эти значения соответствуют допустимым
нормам для мясных эмульсионных систем и обеспечивают микробиологическую
стабильность продукта.

\emph{Исследование функционально-технологических свойств белково-жировых
эмульсий (БЖЭ).} В рамках исследования были проведены испытания по
оценке функционально-технологических характеристик разработанных
белково-жировых эмульсий (БЖЭ). Анализу подверглись следующие
показатели: влагоудерживающая способность (ВУС, \%), жироудерживающая
способность (ЖУС, \%), эмульгирующая способность (ЭС, \%) и стабильность
эмульсии (СЭ, \%). Результаты исследования функционально-технологических
свойств БЖЭ приведены в рисунке 1.

{\bfseries Рис.1 - Экспериментальные данные по различным вариантам БЖЭ}

Все данные представлены в виде сравнительной диаграммы для четырёх
экспериментальных вариантов.

Максимальное значение ВУС зафиксировано у варианта 1 (78,74\%), что
свидетельствует о высокой способности данной эмульсии связывать воду. По
мере перехода к варианту 4 этот показатель постепенно снижался до
73,6\%. Показатель ЖУС продемонстрировал рост от 41\% (вариант 1) до
62,4\% (вариант 4), что указывает на улучшенное удержание жира в
последних рецептурах. Эмульгирующая способность (ЭС) возрастала от
варианта 1 к вариантам 3 и 4, достигая наибольшего значения 56\%. Это
свидетельствует о более эффективной стабилизации жира в водной фазе при
соответствующей рецептуре. Стабильность эмульсии (СЭ) варьировалась от
10\% у варианта 4 до 22\% у варианта 3, что указывает на различную
степень насыщения эмульсий воздухом, влияющую на консистенцию и текстуру
продукта.

На основании полученных данных можно заключить, что варианты 2 и 3
обеспечивают оптимальный баланс между исследуемыми показателями. ни
характеризуются высокой влаго- и жироудерживающей способностью, а также
хорошей стабильностью и взбиваемостью. Эти рецептуры представляют
практический интерес для использования в производстве мясных продуктов.

\emph{Органолептическое исследование белково-жировой эмульсии}

В рамках оценки качества разработанных белково-жировых эмульсий (БЖЭ)
были проведены органолептические исследования по следующим показателям:
внешний вид, консистенция, запах и цвет. Результаты представлены в
таблице 4.

{\bfseries Таблица 4 - Органолептические показатели БЖЭ}

%% \begin{longtable}[]{@{}
%%   >{\raggedright\arraybackslash}p{(\linewidth - 8\tabcolsep) * \real{0.2158}}
%%   >{\raggedright\arraybackslash}p{(\linewidth - 8\tabcolsep) * \real{0.2030}}
%%   >{\raggedright\arraybackslash}p{(\linewidth - 8\tabcolsep) * \real{0.1814}}
%%   >{\raggedright\arraybackslash}p{(\linewidth - 8\tabcolsep) * \real{0.2357}}
%%   >{\raggedright\arraybackslash}p{(\linewidth - 8\tabcolsep) * \real{0.1640}}@{}}
%% \toprule\noalign{}
%% \multirow{2}{=}{\begin{minipage}[b]{\linewidth}\centering
%% {\bfseries Наименование показателя}
%% \end{minipage}} &
%% \multicolumn{4}{>{\centering\arraybackslash}p{(\linewidth - 8\tabcolsep) * \real{0.7842} + 6\tabcolsep}@{}}{%
%% \begin{minipage}[b]{\linewidth}\centering
%% {\bfseries Характеристика}
%% \end{minipage}} \\
%% & \begin{minipage}[b]{\linewidth}\centering
%% {\bfseries Вариант 1}
%% \end{minipage} & \begin{minipage}[b]{\linewidth}\centering
%% {\bfseries Вариант 2}
%% \end{minipage} & \begin{minipage}[b]{\linewidth}\centering
%% {\bfseries Вариант 3}
%% \end{minipage} & \begin{minipage}[b]{\linewidth}\centering
%% {\bfseries Вариант 4}
%% \end{minipage} \\
%% \midrule\noalign{}
%% \endhead
%% \bottomrule\noalign{}
%% \endlastfoot
%% Внешний вид & Однородная масса светло-кремового цвета & Однородная масса
%% с глянцем & Однородная масса светло-кремового цвета & Однородная масса
%% светло-кремового цвета \\
%% Консистенция & Мягкая, умеренно плотная & Более эластичная и плотная &
%% Ещё более густая и пластичная & Плотная, с характерной жирностью \\
%% Запах & Умеренный мясной запах & Сильнее выражен аромат куриного жира &
%% Сбалансированный мясной и жировой запах & Ярко выраженный куриный жирный
%% запах \\
%% Цвет & Светло-жёлтый с розоватым оттенком & Кремовый & Светло-бежевый &
%% Светло-бежевый \\
%% \end{longtable}

Внешний вид: во всех вариантах эмульсия имела однородную структуру,
вариант 2 выделялся наличием глянца, придающего массе привлекательный
внешний облик.

Консистенция: с увеличением доли жира наблюдалось повышение плотности и
пластичности эмульсии. Вариант 4 имел наиболее плотную текстуру с
выраженной жирностью, тогда как вариант 1 характеризовался как мягкий и
умеренно плотный.

Запах: у варианта 2 аромат куриного жира выражен сильнее всего. Вариант
3 отличался наиболее сбалансированным мясным и жировым ароматом. Вариант
4 имел ярко выраженный запах куриного жира, а у варианта 1 аромат был
умеренным.

Цвет: эмульсии имели светлые оттенки от светло-жёлтого с розоватым
подтоном (вариант 1) до светло-бежевого (варианты 3 и 4).

Таким образом, все образцы обладают приемлемыми органолептическими
свойствами, а вариант 3 признан наиболее сбалансированным по
совокупности признаков.

{\bfseries Обсуждение и результаты.} Результаты исследования показали, что
изменения в рецептурном составе БЖЭ оказывают существенное влияние на её
физико-химические и органолептические характеристики. Значения
активности воды (0,9798-0,9968) указывают на необходимость соблюдения
строгих условий хранения для предотвращения микробиологической порчи.
Измерения pH (4,82-4,99) свидетельствуют о слабокислой среде,
способствующей повышению микробиологической устойчивости эмульсий.

Функционально-технологические свойства изменялись в зависимости от
соотношения компонентов. С увеличением доли жира повышалась
жироудерживающая способность (от 41 до 62,4\%), тогда как
влагоудерживающая способность снижалась (от 78,74 до 73,6\%).
Эмульгирующая способность и стабильность варьировались в пределах
50-56\% и 10-22\% соответственно, демонстрируя улучшение при оптимальном
балансе белка и жира.

Органолептическая оценка показала, что вариант 3 обладает наилучшим
сочетанием запаха, консистенции и внешнего вида. Эти данные
подтверждают, что рецептуры 2 и 3 обладают наилучшими потребительскими и
технологическими характеристиками, обеспечивая стабильность,
функциональность и привлекательность продукта.

{\bfseries Выводы.} Разработка белково-жировой эмульсии на основе куриной
кожи, жира и гидролизата показала высокую эффективность использования
побочного сырья птицы. Проведённые исследования подтвердили, что такие
эмульсии обладают приемлемыми технологическими и потребительскими
характеристиками, а вариант 3 рецептур является оптимальным. Полученные
результаты могут быть использованы в производстве мясных изделий с целью
снижения себестоимости, расширения ассортимента и внедрения безотходных
технологий. Работа представляет практическую ценность для предприятий
мясоперерабатывающей отрасли.

\emph{{\bfseries Финансирование.}} \emph{Данные исследования проводились в
рамках проекта № BR24992938 - «Научное обоснование, разработка и
внедрение прогрессивных технологических процессов, методов и решений
комплексной переработки тушек птицы Министерством науки и высшего
образования Республики Казахстан на 2024- 2026 год}

{\bfseries Литература}

1. Дашиева Л. Б. Диссер.Разработка белково-жировой эмульсии для рубленых
полуфабрикатов из мяса птицы.Автореферат дис.. канд.техн.наук:
05.18.04. -Улан-Удэ, 2013.- 110 c.

2. Smolinska T., Wieslaw K., Kopec W. Effect of skin addition on the
technological properties of comminuted chicken meat emulsions//
International Journal of Food Science \& Technology. -2007. -Vol.23(5).
-P.441 - 446. DOI
\href{https://doi.org/10.1111/j.1365-2621.1988.tb00600.x}{10.1111/j.1365-2621.1988.tb00600.x}.

3. Суйчинов А.К., Окусханова Э.К., Капашева Г.А., Жүзжасарова Г.Е.,
Туменов С.Н. Физико-химические показатели мяса курицы и утки и их
субпродуктов// Вестник Университета Шакарима. Серия технические науки.
-2025. --T.1(17). -С.179-186.
\href{https://doi.org/10.53360/2788-7995-2025-1(17)-23}{DOI
10.53360/2788-7995-2025-1(17)-23}.

4. Мансветова Е. В. Разработка технологии вареных колбасных изделий с
использованием белково-жировых эмульсий на основе камедей. Автореферат
дис..канд.техн.наук: 05.18.04 --Москва. -2010. -157c.

5. Прянишников~В.В. Эмульсия из куриной шкурки в технологиях мясных
продуктов // Рациональное питание, пищевые добавки и биостимуляторы.
-2016. -№ 5. -C.27-29.

6. Farmani J., Roshani S., Hosseini Gabous H. Physicochemical properties
of chicken fat as affected by rendering condition // Advances in Food
Sciences. -2016. -Vol.38 (1). -P.35-43.

7. Jamshid F.,
\href{https://onlinelibrary.wiley.com/authored-by/Mohammadnezhad/Sedigheh}{Mohammadnezhad}
S. Rheological and functional characterization of gelatin and fat
extracted from chicken skin for application in food technology // Food
Science and Nutrition. -2025. -Vol.10 (6). -P.1908-1920. DOI
\href{https://doi.org/10.1002/fsn3.2807}{10.1002/fsn3.2807}.

8. Жаринов А.И., Юрков С.Г. Технико-технологические аспекты
приготовления мясных эмульсий // Мясная индустрия. -2006. -№1. -С.
31-35.

9. Окусханова Э.К., Асенова Б.К., Ребезов М.Б., Есимбеков Ж.С., Зинина
О.В. Разработка технологии и рецептуры мясорастительного паштета с
применением белкового обогатителя // Вестник Алматинского
технологического университета. -2017. -№ 1. -С.51-57.

10. ГОСТ ISO 21807-2015. Микробиология пищевой продукции и кормов.
Определение активности воды/М.: Стандартинформ. - 2016. -9 с.
\url{https://meganorm.ru/Data2/1/4293754/4293754690.pdf-} Дата
обращения: 21.05.2025

11. СТ РК ИСО 2917-2009. Мясо и мясные продукты. Определение рН.
Контрольный метод/ Астана: Госстандарт Республики Казахстан. -2010. -16
с.
http://standarts.nism.gov.kg/uploads/demo/pdf/555e6c26b5edc1e20ca0287aab664679.pdf.
Дата обращения 21.05.2025

12. ГОСТ 31470- 2012. Мясо птицы, субпродукты и полуфабрикаты из мяса
птицы Методы органолептических и физико-химических исследований/М.:
Стандартинформ. -2013. -4 с.
\url{https://internet-law.ru/gosts/gost/52629/.Дата} обращения:
21.05.2025

13. Тимошенко Н. В., Патиевой А. М., Патиевой С. В., Нестеренко А. А.,
Н.В.Кенийз. Методические указания к лабораторно-практической работе
«Идентификация качественного состава мясных изделий» //Краснодар.
КубГАУ. -2015 -32 с.

{\bfseries References}

1. Dashieva L. B. Disser.Razrabotka belkovo-zhirovoj
jemul' sii dlja rublenyh polufabrikatov iz mjasa
pticy.Avtoreferat dis.. kand.tehn.nauk: 05.18.04.-Ulan-Udje, 2013.- 110
c. {[}in Russian{]}

2. Smolinska T., Wieslaw K., Kopec W. Effect of skin addition on the
technological properties of comminuted chicken meat emulsions//
International Journal of Food Science \& Technology. -2007. -Vol.23(5).
-P.441 - 446. DOI
\href{https://doi.org/10.1111/j.1365-2621.1988.tb00600.x}{10.1111/j.1365-2621.1988.tb00600.x}.

3. Sujchinov A.K., Okushanova Je.K., Kapasheva G.A., Zhүzzhasarova G.E.,
Tumenov S.N. Fiziko-himicheskie pokazateli mjasa kuricy i utki i ih
subproduktov// Vestnik Universiteta Shakarima. Serija tehnicheskie
nauki. -2025. -T.1(17). -S.179-186. DOI
10.53360/2788-7995-2025-1(17)-23.{[}in Russian{]}

4. Mansvetova E. V. Razrabotka tehnologii varenyh kolbasnyh izdelij s
ispol' zovaniem belkovo-zhirovyh
jemul' sij na osnove kamedej. Avtoreferat
dis..kand.tehn.nauk: 05.18.04 --Moskva. -2010. -157c. {[}in Russian{]}

5. Prjanishnikov V.V. Jemul' sija iz kurinoj shkurki v
tehnologijah mjasnyh produktov // Racional' noe pitanie,
pishhevye dobavki i biostimuljatory. -2016. -№ 5. -C.27-29. {[}in
Russian{]}

6. Farmani J., Roshani S., Hosseini Gabous H. Physicochemical properties
of chicken fat as affected by rendering condition // Advances in Food
Sciences. -2016. -Vol.38 (1). -P.35-43.

7. Jamshid F.,
\href{https://onlinelibrary.wiley.com/authored-by/Mohammadnezhad/Sedigheh}{Mohammadnezhad}
S. Rheological and functional characterization of gelatin and fat
extracted from chicken skin for application in food technology // Food
Science and Nutrition. -2025. -Vol.10 (6). -P.1908-1920. DOI
\href{https://doi.org/10.1002/fsn3.2807}{10.1002/fsn3.2807}.

8. Zharinov A.I., Jurkov S.G. Tehniko-tehnologicheskie aspekty
prigotovlenija mjasnyh jemul' sij // Mjasnaja industrija.
-2006. -№1. -S.31-35. {[}in Russian{]}

9. Okushanova Je.K., Asenova B.K., Rebezov M.B., Esimbekov Zh.S., Zinina
O.V. Razrabotka tehnologii i receptury
mjasorastitel' nogo pashteta s primeneniem belkovogo
obogatitelja // Vestnik Almatinskogo tehnologicheskogo universiteta.
-2017. -№ 1. -S.51-57. {[}in Russian{]}

10. GOST ISO 21807-2015. Mikrobiologija pishhevoj produkcii i kormov.
Opredelenie aktivnosti vody/M.: Standartinform. - 2016. -9 s.
https://meganorm.ru/Data2/1/4293754/4293754690.pdf- Data obrashhenija:
21.05.2025. {[}in Russian{]}

11. ST RK ISO 2917-2009. Mjaso i mjasnye produkty. Opredelenie rN.
Kontrol' nyj metod/ Astana: Gosstandart Respubliki
Kazahstan. -2010. -16 s.
http://standarts.nism.gov.kg/uploads/demo/pdf/555e6c26b5edc1e20ca0287aab664679.pdf.
Data obrashhenija 21.05.2025. {[}in Russian{]}

12. GOST 31470- 2012. Mjaso pticy, subprodukty i polufabrikaty iz mjasa
pticy Metody organolepticheskih i fiziko-himicheskih issledovanij/M.:
Standartinform. -2013. -4 s.
https://internet-law.ru/gosts/gost/52629/.Data obrashhenija:
21.05.2025{[}in Russian{]}

13. Timoshenko N. V., Patievoj A. M., Patievoj S. V., Nesterenko A. A.,
N.V.Kenijz. Metodicheskie ukazanija k laboratorno-prakticheskoj rabote
«Identifikacija kachestvennogo sostava mjasnyh izdelij» //Krasnodar.
KubGAU. -2015 -32 s. {[}in Russian{]}

\emph{{\bfseries Сведения об авторах}}

Суйчинов А.К. - ассоц. профессор, PhD, Семейский филиал ТОО «Казахский
научно-исследовательский институт перерабатывающей и пищевой
промышленности», Семей Казахстан, e-mail:
asuychinov@gmail.com;

Окусханова Э.К. - ассоц. профессор, PhD, НАО «Шәкәрім Университет»,
Семей, Казахстан,

e-mail:
eokuskhanova@gmail.com;

Капашева Г.А. - магистр технических наук; Семейский филиал ТОО
«Казахский научно-исследовательский институт перерабатывающей и пищевой
промышленности», Семей, Казахстан, e-mail:
gena.89.89@mail.ru;

Есимбеков Ж.С. - ассоц. профессор, PhD, Семейский филиал ТОО «Казахский
научно-исследовательский институт перерабатывающей и пищевой
промышленности», Семей, Казахстан, e-mail:
ezhanibek@mail.ru.

\emph{{\bfseries Information about the authors}}

Suychinov A.K. - assoc. Professor, PhD; Kazakh Research Institute of
Processing and Food Industry (Semey Branch). Semey, Kazakhstan, e-mail:
asuychinov@gmail.com;

Okuskhanova E.K. - assoc. Professor, PhD, NJC «Shakarim University»,
Semey, Kazakhstan, e-mail:
eokuskhanova@gmail.com;

Kapasheva G.A. - Master of Technical Sciences; Kazakh Research Institute
of Processing and Food Industry (Semey Branch). Semey, Kazakhstan,
e-mail:
gena.89.89@mail.ru;

Yessimbekov Z.S. - assoc. Professor, PhD, Kazakh Research Institute of
Processing and Food Industry (Semey Branch). Semey, Kazakhstan, e-mail:
ezhanibek@mail.ru.\
