\id{МРНТИ 65.59.31}{}

\begin{header}
\swa{}{ӨНДІРІЛГЕН МАШ ЕЗБЕСІН КҮРКЕТАУЫҚ ЕТ ӨНІМДЕРІ ТЕХНОЛОГИЯСЫНДА ҚОЛДАНУ}

З.К. Конарбаева,
З.Т. Нурсеитова\envelope,
Г.Ж. Нурынбетова,
Б.Ж. Мулдабекова,
А.Қ. Әбді
\end{header}

\begin{affil}
М.Әуезов атындағы ОҚУ, Шымкент, Қазақстан

\corrauthor{Корреспондент-автор: nur.zeinep@mail.ru}
\end{affil}

Биологиялық белсенді құрам бөліктердің жетіспеуі және тапшылығы бар
теңестірілмеген тамақтану алиментарлық сипаттағы аурулардың пайда
болуына және ағзаның тозуына алып келеді. Теңестірілген тамақтануға
арналған жаңа тамақ өнімдерін жасаудың ілгерлемелі бағыттарының бірі ет
шикізатын ішінара өсімдік текті шикізаттарға алмастыра отырып,
қолданылатын қоректік заттардың базасын кеңейту. Бұл жұмыста отандық
«Жасыл дән» сұрыпына жататын маш дақылының (бұршақ дақылдылар ішіндегі
Винга тобы) езбесін күрке тауық етінен жасалған жартылай шикізатқа қосу
ұсынылған. Күрке тауық етінен жасалған езбеге машты қосу етті-өсімдікті
езбенің келесідей технологиялық қасиеттерін жақсартатыны байқалған:
маштың езбесі мөлшерінің артуымен рН мөлшері артады, ылғал
байланыстырғыштық қабілеті 60\%-дан 71\%-ға дейін артады, ылғал ұстап
тұрғыштық қабілеті 55\%-дан 61,2\%-ға дейін артады. Күрке тауық етінен
жасалған ет езбесіне машты қосу жартылай шикізаттар құрамындағы
ақуыздардың, майлардың, көмірсулардың мөлшерінің артуына алып келеді.
Күрке тауық етінен жасалған жартылай шикізаттың ең жақсы сапа
көрсеткіштеріне маш езбесін 15\% мөлшерінде қосу нәтижесінде қол
жеткізілді. Ет жартылай шикізаттарының рецептурасында машты өңдеу өнімін
қолдану құрамында теңестірілген құнды заттары бар тамақтану өнімдерінің
түрін арттырады, алынған ет езбесі жүйесінің технологиялық, құрылымдық
қасиеттерін жақсартады.

{\bfseries Түйін сөздер:} теңестірілген тамақтану, қоректік заттар, күрке
тауық еті, маш езбесі, технология, нутриенттік әлеует

\begin{header}
ПРИМЕНЕНИЕ ПРОРОЩЕННОЙ ПАСТЫ ИЗ МАША В ТЕХНОЛОГИИ ПРОДУКТОВ ИЗ МЯСА ИНДЕЙКИ

З.К. Конарбаева,
З.Т. Нурсеитова\envelope,
Г.Ж. Нурынбетова,
Б.Ж. Мулдабекова,
А.Қ. Әбді
\end{header}

\begin{affil}
Южно-Казахстанский~университет им.~М.Ауезова, Шымкент, Казахстан,

e-mail: nur.zeinep@mail.ru
\end{affil}

Несбалансированное питание при недостатке и дефиците биологически
активных компонентов может вызывать заболевания алиментарной природы и в
целом может привести к истощению организма. Перспективным направлением в
разработке новых пищевых продуктов для сбалансированного питания
является расширение базы используемых питательных веществ за счет
частичной замены мясного сырья на растительное. В данной работе был
предложен и изучен полуфабрикат из мяса индейки с добавлением пюре
кашица из зернобобовой культуры маш (вид рода Вигна семейства Бобовые)
сорта отечественной селекции- «Жасыл Дән». Было выявлено, что при
введении в фарш из индейки пюре из маша улучшаются технологические
характеристики мясо-растительного фарша: с увеличением концентрации пюре
из маша увеличивалось значение рН, влагосвязывающая способность возросла
с 60 \% до 71 \%, влагоудерживающая способность возросла с 55 \% до 61,2
\%. Добавление пюре из маша в состав фарша из индейки приводит к
увеличению содержания в полуфабрикатах белков, жиров, углеводов.
Наилучшее качество полуфабриката из мяса индейки выявлено в образце с
15\% добавкой пюре из маша. Использование пюре из маша при производстве
мясных полуфабрикатов расширяет ассортимент продуктов питания с
повышенной сбалансированностью питательных веществ, улучшает
структурные, технологические свойства полученных фаршевых систем.

{\bfseries Ключевые слова:} сбалансированное питание, питательные вещества,
мясо индейки, пюре из маша, технологический, нутриентный потенциал.

\begin{header}
APPLICATION OF SPROUTED MUNG BEAN PASTE IN THE TECHNOLOGY OF TURKEY MEAT PRODUCTS

Z.K. Konarbaeva,
Z.T. Nurseytova\envelope,
G.Zh. Nurynbetova,
B.Zh. Muldabekova,
A.K. Abdi
\end{header}

\begin{affil}
SKU named after M.Auezov, Shymkent, Kazakhstan,

e-mail: nur.zeinep@mail.ru
\end{affil}

An unbalanced diet with a lack and deficiency of biologically active
components can cause diseases of an alimentary nature and, in general,
can lead to exhaustion of the body. A promising direction in the
development of new food products for a balanced diet is to expand the
base of nutrients used by partially replacing meat with vegetable raw
materials. In this work, we proposed and studied a semi---finished
product of turkey meat with the addition of purеe from the leguminous
culture of mash (a species of the genus Vigna of the Legume family) of a
variety of domestic breeding - "Zhasyl Dan". It was found that when mash
purеe is introduced into turkey minced meat, the technological
characteristics of minced meat and vegetable meat improve: with an
increase in the concentration of mash purеe, the pH value increased, the
moisture binding ability increased from 60\% to 71\%, and the moisture
retention ability increased from 55\% to 61.2\%. The addition of mash
puree to turkey minced meat leads to an increase in the content of
proteins, fats, carbohydrates in semi-finished products. The best
quality of the turkey meat semi-finished product was revealed in a
sample with 15\% added puree from mash. The use of mash puree in the
production of semi-finished meat products expands the range of food
products with an increased balance of nutrients, improves the structural
and technological properties of the resulting stuffing systems.

{\bfseries Keywords:} balanced nutrition, nutrients, turkey meat, mung bean
purеe, technological, nutritional potential

\begin{multicols}{2}
{\bfseries Кіріспе.} Тамақ өнеркәсібіндегі дамытудың бағыттарының бірі жаңа
өнімдер жасау. Жаңа тамақтану өнімдерін жасау оның тағамдық және
биологиялық құндылығын, физико-химиялық және функциональдық қасиеттерін,
қауіпсіздігін жақсартуға және тамақ өнімдерінің қол жетімділігін
арттыруға бағытталған.

Еліміздегі тұрғындарды қауіпсіз азық-түлік өнімдерімен қамтамасыз етудің
ең маңызды мәселесі ол ақуызбен қамтамасыз ету. Ақуыздардың бірқатар
альтернативтік нұсқаларына қарамастан, жануартекті ақуыздар әлі күнге
дейін негізгі ақуыздың көзі болып саналады. Адамның тамақтануындағы
еттің маңызын асыра бағалау өте қиын. Ол тек ғана ақуыздар мен
калорияның көзі ғана емес, сонымен қатар, адам ағзасына қажетті
дәрумендердің, оңай сіңетін темір мен мырыштыңда көзі болып саналады.
Қазіргі уақытта адамның тамақтану рационында ең кең таралған жануартекті
ақуыздар олар құс етіне тән {[}1{]}.

Құс еттерінің ішінде әсіресе күркетауық етіне соңғы жылдары ерекше назар
аударылуда.

Күркетауық еті аминқышқылды құрамы теңестірілген ақуыздың көзі болып
саналады, соның нәтижесінде ол жақсы қорытылады және жақсы сіңеді.
Күркетауық етінің биологиялық толық құндылығына оның құрамында жоғары
мөлшерде кездесетін триптофан және оксипролин аминышқылдары кепіл бола
алады {[}2, 3{]}.

Еттің осы түріне деген еліміздегі сұраныстың күн санап артуына
байланысты Қазақстан Республикасының ауыл шаруашылық министрлігі
күркетауық етін шығаратын жеті кәсіпорынды салуды жоспарлаған.2020 жылы
жаңа құс фермасы ашылды: Алматы облысында, Атырау облысында, Жамбыл
облысында, Қостанай облысында, Маңғыстау облысында, Түркістан облысында
және Солтүстік Қазақстан облысында. Ашылған фермалардың барлығының
өнімділігі жылына 50 мың тоннаны құрайды. Түркістан облысының Бадау
ауылында «Ордабасы құс» ЖШС құс фермасы кешенінде жылына 4830 мың тонна
күркетауық өсіру мен оның етін өңдеу жүргізіледі.

Күркетауық етін өндіру саласы Қазақстанда да, әлемде де жақсы
перспективаға ие. Бұл салауатты тамақтанудың белең алуымен және
күркетауық етінен жасалған өнімдерге деген сұраныстың артуымен
байланысты {[}4{]}.

Соңғы жылдары, жануарлар мен өсімдіктекті шикізаттарды бірге қолдану
бойынша тенденцияның артқанын байқауға болады. Себебі, адамдар жануар
текті өнімдерінің қоршаған ортаға әсері туралы және өсімдік текті
өнімдерді көп мөлшерде пайдаланудың пайдасы туралы көбірек біле түсуде.
Бұршақ дақылдары мен дәнді дақылдарды пайдалана отырып жасалған құрама
ет өнімдері құрамында қаныққан майлар мен холестериннің мөлшері аз,
алмаспайтын амин қышқылдарының мөлшері жоғары, құрамы тағамдық талшықтар
мен фитохимиялық заттарға бай экономикалық тұрғыда тиімді өнім болып
саналады.

Маш немесе жасыл бұршақ негізінен Қазақстанда, Азия елдерінде және
әлемнің басқа елдерінде өсірілетін бұршақ дақылдас өсімдік. Бұл
құрамында оңай сіңетін ақуыздар ( 20-32\%), көмірсулар (53,3-67,1\%),
майлар (0,71-1,85\%), дәрумендер, минералды заттар мен жасунық мөлшері
жоғары болатын ылғал сүйгіш бұршақ дақылды өсімдік. Маштың құрамында
танин, фитин қышқылы, гемагглютин, полифенол және өте аз мөлшерде
трипсин ингибиторлары болатын антинутриенттер болады. Бұған қоса маштың
құрамында қабынуға қарсы, микробка қарсы сияқты фармацевтикалық маңызды
қасиеттерге ие алкалоидтер, флавоноидтер, сапониндер, фенолдар,
гликозидтер мен пептидты биоактивтер сияқты фитохимиялық заттар болады
{[}5{]}.

Машты зерттеу бойынша жүргізілген ғылыми-зерттеу жұмыстарын қорытындылай
келе, машты табиғи күйінде де және кез-келген өңделген күйде пайдалану
адам денсаулығы үшін маңызды рөл атқарады және функциональды бағыттағы
түрлі өнімдер жасауда қолданылуы мүмкін {[}6{]}.

Жұмыстың мақсаты машты өңдеу өнімін күрке тауық етінен жасалған ет
өінімі өндірісінде қолданудың технологиялық және нутриенттік потенциалын
зерттеу және негіздеу.

Зерттеу нысаны келесілер: 1 сұрыпты күрке тауық еті («Дәмді ет» тауарлық
маркасы арқылы тіркелген «Ордабасы құс» ЖШС өнімі), маш езбесін 10-нан
20-ға дейін қосу арқылы алынған ет езбесі және одан алынған жартылай
шикізат. Қазіргі таңда Қазақстан Республикасында маштың «Жасыл дән»
селекциясына жататын жалғыз сұрыпы қолданылады {[}7{]}.

{\bfseries Материалдар мен әдістер.} Зерттеу материалдары ретінде
күркетауықтың сан сүбе еті (МЕМСТ 31473-2012 бойынша) {[}8{]} мен
бауырын қосу арқылы өндірілген маш езбесімен (МЕМСТ 35050-2023 бойынша)
{[}9{]} байытылған ет өнімі қолданылды. Ет езбесінің ылғал
байланыстырғыштық қабілеті оның технологиялық дайындаудың түрлі
сатыларындағы қасиетін анықтайды және дайын ет өнімдерінің су ұстағыштық
қабілетіне, олардың сапасы мен шығымына әсер етеді. Оларды Грау-Гамм
әдісін қолдана отырып анықтайды. Әдісте етті фильтр қағазының үстіне
қойып ақырындап сығымдау барысында бөлінетін су дағының мөлшерімен
анықталады. Дақ мөлшері еттін су ұстағыштық қабілетіне байланысты
болады. Алдымен еттің ылғалдылығын анықтап алады. Содан соң барып ет
сынамасының 0,3 г диаметрі 15-20 мм болатын полиэтиленнен жасалған
пиалаға салады, содан соң барып оны фильтр қағазына салады бұл ретте
фарт полиэтилен пиаланың астында болуы тиіс. Сынама бетіне тура
астындағыдай пластинкамен жабады және оның үстіне салмағы 1 кг болатын
жүкті 10 мин қояды. Содан соң барып жүкті алады, фильтр қағазынан ет
езбесін алады және фильтр қағазындағы дақты қаламмен сызып оны олшеп
алады. Тәжірибелік тұрғыдан 1 см\tsp{2} аудан 804 мл суды
сіңірген деп қабылдау ұсынылған. Байланысқан су мөлшері келесі формула
арқылы анықталады:
\vspace{-0.5em}
\begin{equation}
В_{1} = [(А - 8,4 \times Б) \times 100]: M
\end{equation}
\vspace{-0.8em}
\begin{equation}
B_{2} = [(А - 8,4 \times Б) \times 100] : A
\end{equation}

мұндағы \(В_{1}\) -- байланысқан ылғалдың мөлшері, \% ет
салмағына қатысты;

\(В_{2}\) -- байланысқан ылғалдың мөлшері, \% жалпы ылғалға
байланысты;

\(А\) -- сынамадағы ылғал мөлшері, мг;

\(8,4\) -- 1 см ылғал дақтағы су мөлшері, мг;

\(Б\) -- ылғал дақтың ауданы, см\tsp{2};

\(М\) -- ет езбесі сынамасының салмағы, мг.

Ылғал ұстап тұрғыштық қабілеті өнімнің ылғалдылығынан байланысқан ылғал
мөлшерін алып тастау арқылы келесі формуламен анықталады:
\vspace{-0.3em}
\begin{equation}
ЫҰҚ = W_{\text{ылғ}} - ЫБҚ
\end{equation}

Жартылай шикізаттың органолептикалық сапасы (дәмі, иісі, консистенциясы,
сыртқы түрі) дегустациялау әдісімен бес балдық шкала бойынша МЕМСТ
9959-2015 {[}10{]} сай анықталды.

рН мөлшері станционарлы зертханалық рН метрде анықталды.

Майдың салмақтық үлесі МЕМСТ 23042-2015 сай анықталады. Ылғалдың
салмақтық үлесі анықтағаннан кейінгі алынған кептірілген сынаманы арнайы
бюкстерге салып, үстінен еріткіш ретінде 10-15 мл этил эфирін құяды.
Өнім құрамындағы майды экстрациялау әр бір 3-4 минут сайын 4-5 рет
қайталау арқылы жүргізілді. Еріткішті соңғы рет төгіп тастаған соң,
ауада буландырады. Майсыздандырылған сынамасы кептіргіш шкафта 10 минут
бойына 105\tsp{0}С температура кептіру арқылы анықтадық
{[}11{]}.

Ақуыздың мөлшерін МЕМСТ 25011-2017 сай Кельдаль әдісімен анықтадық. Әдіс
органикалық қосылыстарды концентрлі күкірт қышқылымен минерализациялап,
ары қарай азоттың мөлшерін аммиактың жалпы мөлшері бойынша анықтауға
негізділген {[}12{]}.

Көмірсулардың салмақтық үлесі есептеу арқылы анықталды. Зерттеудің
стандартты және арнайы әдістері қолданылды.

Зерттеу жұмыстары М.Әуезов атындағы ОҚУ «Тамақ инженериясы» кафедрасының
базасында жүргізілді.

Әр бір сапа көрсеткішті анықтау үш реттен жүргізіліп, оның арифметикалық
ортасы алынды.

Зертханалық жағдайда маш дақылын өндірдік. Ол үшін 500 г маш дәні өлшеп
алынды. Оны минеральды және өзгеде қоспаларынан тазалап, температурасы
25 \tsp{0}С сумен жудық. Содан соң барып оны жалпақ ыдысқа
салып, қараңғы жерге, бөлме температурасына 3 тәулікке қалдырып қойдық.
Әр тәулік сайын температрурасы 25 \tsp{0}С суды маштың
бетіне себелеп тұрдық. Маштың өскінділері 2-3 мм болған кезде алып,
ақырын суда мұхият бір жуып алдық. Өндірілген машты зертханалық блэндрге
салып, біркелкі масса пайда болғанша майдалап алдық.

Ет езбесін келесідей сатылардан тұрады: шикізаттарды сұрыптау және
дайындау, бланширлеу, майдалау, құрамына өнділірген маш езбесін 10-нан
20-ға дейін қосу арқылы және өзгеде рецептуралық құрам бөліктермен
араластыру, ыдыстарға құю, стерилдеу және суыту. алынған ет езбесі.

{\bfseries Нәтижелер мен талқылаулар.} Күркетауық еті - салауатты өмір
салтын ұстанғысы келетіндер үшін ең жақсы таңдау. Тамақтану саласындағы
әлемдік сарапшылар күрке тауық етін пайдаланудың артқышлығын дәлелдеген.
Күркетауық етінің орташа порциясы дәрумендерге деген қажіттіліктің 60\%
қамтамасыз етеді. Күркетауық етінің құрдамында К, Е, Д, РР, В тобының
дәрумендері, тіршілікке маңызды минералдар - калий, кальций, магний,
натрий, фосфор, мырыш, йод, күкірт, селен, марганец және т.б. жоғары
мөлшерде бар. Күркетауық еті мен басқа құс еттерінің химиялық құрамы мен
тағамдық құндылығын салыстырмалы талдауы 1-кестеде көрсетілген {[}13{]}.

1-кестедегі мәліметтерді талдай отырып, күркетауық етінің құрамындағы
ақуыз, натрий, фосфор басқаларға қарағанда көп екендігін және майдық
үлесі аз, сол арқылы басқа құс еттерімен «бәсекеге қабілетті»
көрсеткіштерге ие деп айта аламыз.

Құрамында ақуыз мөлшері жоғары жаңа дақылды іздеу және оның химиялық
құрамын жақсарту мақсатында машты зерттедік. Өңдеудің саналуан түрлері
маштың химиялық құрамына және оның функциональды қасиетінің өзгеруіне
алып келуі мүмкін.

Маштың химиялық құрамы басқа бұршақ дақылдарына ұқсас, бірақ өзіндік
ерекшеліктері бар. Маш дәнінде ақуыз мөлшері 24-28\% шамасында болып, ол
ноқат (20-22\%) пен бұршақтан (20-25\%) жоғары. Көмірсулар үлесі машта
орташа (55-60\%), бұл көрсеткіш ноқат пен фасольмен шамалас. Май мөлшері
машта аз (1-2\%), ол соямен салыстырғанда әлдеқайда төмен (18--20\%).
Маш минералды заттарға (темір, магний, калий) және дәрумендерге бай, ас
қорытуға жеңіл әрі аллергенділігі төмен {[}14{]}.
\end{multicols}

\tcap{1-кесте. Құс етінің химиялық құрамы мен тағамдық құндылығы}
\begin{longtblr}[
  label = none,
  entry = none,
]{
  cells = {c},
  cells = {font = \small},
  cell{1}{2} = {c=2}{},
  cell{1}{4} = {c=2}{},
  cell{1}{6} = {c=2}{},
  cell{1}{8} = {c=2}{},
  cell{1}{10} = {c=2}{},
  cell{7}{1} = {c=11}{},
  cell{14}{1} = {c=11}{},
  hlines,
  vlines,
}
\textbf{Құс түрлері} & \textbf{Балапандар} &      & \textbf{Қаздар} &      & \textbf{Күрке -тауық} &      & \textbf{Тауықтар} &      & \textbf{Үйректер} &      \\
Санат                & 1-ші                & 2-ші & 1-ші            & 2-ші & 1-ші                  & 2-ші & 1-ші              & 2-ші & 1-ші              & 2-ші \\
Су, мл               & 63,8                & 67,7 & 45              & 54,4 & 57,3                  & 64,5 & 61,9              & 68,1 & 5,6               & 56,7 \\
Ақуыздар, г          & 18,7                & 19,7 & 15,2            & 17   & 19,5                  & 21,6 & 18,2              & 21,2 & 15,8              & 17,2 \\
Майлар, г            & 16,1                & 11,2 & 39              & 27,7 & 22                    & 12   & 18,4              & 8,2  & 38                & 24,2 \\
Зола, г              & 0,9                 & 0,9  & 0,8             & 0,9  & 0,9                   & 1,1  & 0,8               & 0,9  & 0,6               & 0,9  \\
Минералды заттар, мг &                     &      &                 &      &                       &      &                   &      &                   &      \\
Na                   & 70                  & 88   & 91              & 99   & 90                    & 100  & 70                & 79   & 58                & 90   \\
K                    & 236                 & 242  & 240             & 274  & 210                   & 257  & 194               & 240  & 156               & 160  \\
Ca                   & 14                  & 12   & 12              & 14   & 12                    & 18   & 16                & 18   & 10                & 12   \\
Mg                   & 19                  & 22   & 30              & 34   & 19                    & 25   & 18                & 21   & 15                & 13   \\
P                    & 160                 & 175  & 165             & 179  & 200                   & 227  & 165               & 190  & 136               & 156  \\
Fe                   & 1,3                 & 1,7  & 2,4             & 2,4  & 1,4                   & 1,8  & 1,6               & 1,5  & 1,9               & 1,9  \\
Дәрумендер, мг       &                     &      &                 &      &                       &      &                   &      &                   &      \\
А                    & 0,04                & 0,03 & 0,02            & 0,02 & 0,01                  & 0,01 & 0,07              & 0,07 & 0,05              & 0,05 \\
В1                   & 0,09                & 0,11 & 0,08            & 0,09 & 0,05                  & 0,07 & 0,07              & 0,07 & 0,12              & 0,18 \\
В2                   & 0,15                & 0,16 & 0,23            & 0,26 & 0,22                  & 0,19 & 0,15              & 0,14 & 0,17              & 0,19 \\
РР                   & 6,1                 & 6,4  & 5,2             & 5,6  & 7,8                   & 8    & 7,7               & 7,8  & 5,8               & 6    
\end{longtblr}

\begin{multicols}{2}
Маш дақылы адамдардың физиологиялық ерекшелігіне байланысты аллергия
тудыруы мүмкін. Алайда, машттың аллергиялық қасиетінен құтылуға болады.
Ол үшін оны өндіру керек {[}15, 16{]}.

Соңғы жылдары өндірілген маш өнімін пайдалану тенденциясы салауатты
тамақтану принциптерін ұстанған адамдар арасында кеңінен таралуда.

Өндірілген машты зертханалық майдалағышқа салып, біркелкі масса пайда
болғанша майдалап алдық. Ары қарай оны күркетауық еті негізіндегі езбе
өніміні өндірісінде қолдану үшін пайдаландық.

Бірақ оған дейін, маш пен өндірілген маштың химиялық құрамы мен
минералды құрамы салыстырмалы түрде анықталды (2-кесте).

2-кестеден көріп тұрғанымыздай, өндірілген маш құрамында өндірілмеген
машпен салыстырғанда ақуыздың және күлдің мөлшері жоғары болды.
Зерттеулер көрсеткендей маш дақылын өндіру барысында оның ылғалдығы
артады, ферменттік жүйенің белсендірілуі нәтижесінде көмірсулар гидролиз
жүреді (көмірсулар мөлшері 3,5 тен 1,3 ке дейін азаяды). Бұл
функциональды бағыттағы өнім алуда өндірілген машты қолданудың
мақсаттылығын айқындайды.

Ет езбесінің тәжірибелік сынамаларында күркетауық саны сүбесінің жалпы
рецептуралық мөлшерінен 5, 10 және 15\% өндірілген маш езбесі қосылады.

Өнімге сұранысты қалыптастыру барысында шешуші рөлді оны
органолептикалық сапа көрсеткіштері алады, ал оның химиялық құрамы мен
тағамдық құндылығы көбінесе екінші кезекте болады.

Сондықтанда, өндірілген маш езбесінің күркетауық паштетінің
органолептикалық сапа көрсеткіштеріне әсері анықталды, анықтау нәтижесі
3-кестеде берілген.
\end{multicols}

\tcap{2-кесте. Маш және өндірілген маштың химиялық құрамын салыстырмалы түрде анықтау нәтижесі}
\begin{longtblr}[
  label = none,
  entry = none,
]{
  cells = {c},
  cells = {font = \small},
  cell{1}{1} = {r=2}{},
  cell{1}{2} = {c=2}{},
  vlines,
  hline{1,3-8} = {-}{},
  hline{2} = {2-3}{},
}
\textbf{Көрсеткіш атауы} & \textbf{Маштағы салмақтық үлесі, \%} &                         \\
                         & \textbf{Өндірілмеген маш}            & \textbf{Өндірілген маш} \\
Ылғалдың салмақтық үлесі & 12,0                                 & 57,9                    \\
Күлділік                 & 2,02                                 & 5,04                    \\
Ақуыздар                 & 20,6                                 & 28,9                    \\
Майлар                   & 2,5                                  & 0,7                     \\
Көмірсулар               & 3,5                                  & 1,3                     
\end{longtblr}

\tcap{3-кесте. Ет өнімінің сынамаларының органолептикалық сапа көрсеткіштерін анықтау нәтижесі}
\begin{longtblr}[
  label = none,
  entry = none,
]{
  width = \linewidth,
  colspec = {Q[91]Q[219]Q[213]Q[227]Q[204]},
  cells = {font = \small},
  row{1} = {c},
  cell{2}{1} = {c},
  cell{3}{1} = {c},
  cell{4}{1} = {c},
  cell{5}{1} = {c},
  hlines,
  vlines,
}
\textbf{Көрсет\-кіш атауы} & \textbf{Бақылау сынамасы}                                                   & \textbf{10\% маш езбесі қосылған сынама}                            & \textbf{15\% маш езбесі қосылған сынама}                                 & \textbf{20 \% маш езбесі қосылған сынама}                           \\
Консист\-енциясы           & Нәзік, жағылғыш, барлық көлемі бойынша біркелкі                             & Нәзік, жағылғыш, барлық көлемі бойынша біркелкі                     & Нәзік, жағылғыш, барлық көлемі бойынша біркелкі                          & Тығыздау, барлық көлемі бойынша біркелкі емес                       \\
Иісі                     & Паштетке тән хош иісті, бөгде иіссіз                                        & Аздап ғана өсімдікті-жаңғақты иіс сезіледі                          & Өсімдікті-жаңғақты иіс жақсы сезіледі                                    & Өсімдікті-жаңғақты иіс анық сезіледі                                \\
Дәмі                     & Нәзік етті, тұздылау, өнімге тән                                            & Аздап машқа тән дәм сезіледі, толық емес                            & Ет пен өсімдік дәмі теңескен, толық                                      & Өсімдік дәмі қанық сезіледі                                         \\
Сыртқы көрінісі          & Майда дисперсті, барлық көлемі бойынша бірдей, күкетауық етіне тән түске ие & Майда дисперсті, барлық көлемі бойынша бірдей, аздап жасыл реңге ие & Майда дисперсті, барлық көлемі бойынша бірдей, жасыл-ашық қоңыр реңге ие & Майда дисперсті, барлық көлемі бойынша бірдей, қоңыр-жасыл реңге ие 
\end{longtblr}

\begin{multicols}{2}
Ет өнімі құрамында өндірілген маш езбесінің мөлшерінің артуымен дайын
өнімнің түсі ашық жасылдан қоңыр жасыл түске дейін өзгереді. Бұл маш
құрамында өнімге түс беретін хлорофилл мен антиоксиданттардың болуынан.
Құрамына 10 және 15\% мөлшерінде маш езбесі қосылған тәжірибелік
сынамалардың консистенцясы бақылау сынамасымен бірдей нәзік, жағылғыш,
барлық көлемі бойынша біркелкі болады. Ал ет езбесінің консистенциясы
маш езбесінің мөлшері 20\%-ға дейін артуымен тығыздау бола бастайды.
Сонымен қатар, ет езбесі құрамындағы күркетауық бауырын күркетауық
жүрегімен алмастырудан пашттеттің консистенциясының аздап тығыздалуна
алып келеді.

Ет езбесі сынамаларының құрамында өндірілген маш езбесінің мөлшерінің
артуымен өнімнің дәмі мен иісінде өсімдікке тән дәм мен иіс күшейе
түседі, 10\% мөлшерінде маш езбесі қосылған тәжірибелік сынамада аздап
ғана өсімдікті-жаңғақты иіс сезілсе, 15\% мөлшерінде маш езбесі қосылған
тәжірибелік сынамада ет пен өсімдік дәмі теңескен, толық болады. Ал 20\%
мөлшерінде маш езбесі қосылған тәжірибелік сынамада өндірілген маштағы
ферментативтік үрдістердің белсенуінен өсімдік дәмі қанық сезіледі.

Ары қарай алынған жартылай шикізаттардың физико-химиялық және
технологиялық қасиеттері анықталды.

Құрамына 10\%, 15\% және 20\% қосылған етті-өсімдікті жартылай
шикізаттардың дайындалған сынамалары функционалдық және технологиялық
қасиеттері бойынша бағалану үшін рН мөлшері бойынша, ылғал ұстап
тұрғыштық қабілеті, ылғал байланыстырғыштық қабілеті бойынша бағаланды.
Ары қарай сынамалар құрамындағы негізгі құрам бөліктердің мөлшері
анықталды, бұл ретте ақуызды құрам бөліктердің мөлшерінің артуына баса
назар аударылды (1-сурет және 4-кесте).
\end{multicols}

\tcap{4-кесте. Етті-өсімдікті жартылай шикізаттың негізгі технологиялық көрсеткіштері}
\begin{longtblr}[
  label = none,
  entry = none,
]{
  width = \linewidth,
  colspec = {Q[173]Q[162]Q[200]Q[200]Q[200]},
  cells = {c},
  cells = {font = \small},
  cell{1}{1} = {r=2}{},
  cell{1}{2} = {r=2}{},
  cell{1}{3} = {c=3}{0.6\linewidth},
  vlines,
  hline{1,3-6} = {-}{},
  hline{2} = {3-5}{},
}
\textbf{Көрсеткіштер атауы}      & \textbf{Бақылау сынамасының мәні} & \textbf{Өндірілген маш езбесін әр түрлі мөлшерде қосылған етті-өсімдікті жартылай шикізаттың тәжірибелік сынамалары} &               &               \\
                                 &                                   & \textbf{10\%}                                                                                                        & \textbf{15\%} & \textbf{20\%} \\
рН                               & 6,84±0,01                         & 6,88±0,01                                                                                                            & 6,91±0,01     & 6,94±0,01     \\
Ылғал ұстап тұрғыштық қабілеті   & 55±0,2\%                          & 60,4±0,2\%                                                                                                           & 60,9±0,1\%    & 61,2±0,1\%    \\
Ылғал байланыстырғыштық қабілеті & 60 ±0,2\%                         & 62 ±0,2\%                                                                                                            & 70±0,2\%      & 71±0,2\%      
\end{longtblr}

{\bfseries 1-сурет. Етті-өсімдікті жартылай шикізаттың негізгі
технологиялық көрсеткіштері}

4 кестедегі және 1-суреттегі мәліметтерге сай күрке тауық етінен
жасалған ет езбесіне маш езбесі қосу етті-өсімдікті езбенің
технологиялық сипаттамаларын жақсартатынын көрсетті: маш ұнының мөлшері
артуымен рН мөлшері артады, ылғал байланыстарғыштық қабілеті 60\%-дан
71\%-ға дейін артады, ылғал ұстап тұрғыштық қабілеті 55\%-дан 61,2\%-ға
дейін артады. Бұл өзгерістер өсімдік қоспасында крахмалдың, жасунықтың
және ақуыздың көп мөлшерде болуымен түсіндіріледі, олар өнімді
технологиялық өңдеу барысында ет езбесі жүйесінде бос ылғалды ұстап
тұруға мүмкіндік береді.

Күрке тауық ет езбесі құрамына маш езбесін қосу жартылай шикізаттар
құрамындағы ақуыздардың, майлардың, көмірсулардың және тағамдық
талшықтардың мөлшерінің артуына алып келеді (5-кесте).

Ет өнімінің тәжірибелік сынамасының құрамына өндірілген маш езбесі
күркетауық саны сүбесінің жалпы мөлшерінің 15\% шегере отырып қосты және
ет өнімінің дәстүрлі рецептурасындағы бауыр күркетауық жүрегімен
алмастырылды.

Ет өнімінің бақылау сынамасы дәстүрлі рецептура негізінде күркетауық еті
мен бауырынан жасалынды.

\tcap{5-кесте. Ет өнімінің сынамаларының химиялық құрамын салыстырмалы түрде анықтау нәтижелері}
\begin{longtblr}[
  label = none,
  entry = none,
]{
  cells = {c},
  cells = {font = \small},
  hlines,
  vlines,
}
\textbf{Көрсеткіштердің атауы}          & \textbf{Бақылау үлгісі} & \textbf{Тәжірибелік үлгі} \\
Ылғалдың салмақтық үлесі, \%/100 г      & 65.2 ± 0.15             & 70.0 ± 0.20               \\
Ақуыздың салмақтық үлесі, \%/100 г      & 15.6 ± 0.10             & 25.8 ± 0.25               \\
Майдың салмақтық үлесі, \%/100 г        & 4.5 ± 0.05              & 4.6 ± 0.05                \\
Көмірсулардың салмақтық үлесі, \%/100 г & 7.1 ± 0.08              & 9.4 ± 0.10                
\end{longtblr}

\begin{multicols}{2}
5-кестеден көрініп тұрғандай, ет өнімінің тәжірибелік сынамаларының
құрамындағы ақуыздардың, күлдің, көмірсулардың салмақтық үлесінің
бақылау сынамасымен салыстырғанда айтарлықтай үлкен болған. Ет өнімінің
тәжірибелік сынамаларындағы ылғалдың салмақтық үлесі өндірілген маштың
ылғал байланыстырғыш қабілетінің жоғары болуынан бақылау сынамасымен
салыстырғанда жоғары болған. Ақуыздың салматық үлесі бақылау сынамасымен
салыстырғанда 10\% артқан. Бұл күркетауық жүрегі мен өндірілген маштың
ақуызды құрамының бай болуымен тікелей байланысты.

{\bfseries Қорытынды.} Күрке тауық етінен жасалған жартылай шикізаттың ең
жақсы сапа көрсеткішіне 15\% маш езбесі қосылған сынама ие болды. Ет
жартылай шикізатының құрамына 15\% мөлшерінде маш езбесін қосу бақылау
сынамасымен (маш езбесі қосылмаған жартылай шикізаты) салыстырғанда
ылғалдың мөлшерін 4,8\%-ға, ақуыздардың мөлшерін 10,2\%-ға мүмкіндік
береді. Май мен көмірсулардың мөлшеріде маш езбесі қосылған сынамаларда
(сәйкесінше 4,6\% және 9,4\%) жоғары болды.

Ет жартылай шикізаттары өндірісінде маш езбесін қолдану құрамында
теңестірілген қоректік заттарының мөлшері жоғары тамақтану өнімінің
ассортиментік түрін арттырады, алынған ет езбесі жүйесінің құрылымдық,
технологиялық қасиеттерін жақсартады.
\end{multicols}

\begin{center}
{\bfseries Әдебиттер}
\end{center}

\begin{refs}
1. Окусханова Э.К. Разработка технологии мясного паштета с применением
акустических методов обработки мясного и вторичного сырья:
диссертационная работа д-ра философии (PhD). Семей. -2018. - 153 с.

2. Ахмедова Т.П. Использование пищевых волокон для обогащения пищевых
продуктов // Орел: Изд-во ОрелГИЭТ. - 2012. - С.18--22.

3. Зачесова И.А., Страхова С.А., Кузина А.А. Разработка рецептуры
паштета из креветок с использованием пшеничной // Вестн. КрасГАУ. -2019.
-№ 2. -С.139--142.

4. Kozhabayeva S.А., Sartanova N.T. POULTRY SUBCOMPLEX OF KAZAKHSTAN:
PRODUCTION OF TURKEY MEAT// Problem of AgriMarket. -2021. -№3. -P.
100-107. \href{https://doi.org/10.46666/2021-3.2708-9991.11}{DOI
10.46666/2021-3.2708-9991.11}.

5. Sá A.G.A., Moreno Y.M.F., Carciofi B.A.M. Food processing for the
improvement of plant proteins digestibility//Critical Reviews in Food
Science and Nutrition~. -2020. --Vol.60 (20). --P.3367-- 3386.
\href{https://doi.org/10.1080/10408398.2019.1688249}{DOI
10.1080/10408398.2019.1688249}.

6. Chandel K.P.S, Lester R.N., Starling R.J. The wild ancestors of urid
and mung beans (Vigna mungo (L.) Hepper and V. radiata (L.) Wilczek)//
Botanical Journal of the Linnean Society. -2008. --Vol.89 (1). --P.85
- 96. DOI
\href{http://dx.doi.org/10.1111/j.1095-8339.1984.tb01002.x}{10.1111/j.1095-8339.1984.tb01002.x}.

7. Айтбаев Т.Е. Маш --- перспективная культура для Казахстана /
Agro-mart. URL:
\url{https://agro-mart.kz/mash-perspektivnaya-kultura-dlya-kazahstana}.
--Қаралған күні: 21.08.2025.

8. ГОСТ 31473-2012 Мясо индеек (тушки и их части). Общие технические
условия. -2013. -12c.

9. ГОСТ 35050-2023 Маш. Технические условия (с Поправкой).-2024. -14с.

10. ГОСТ 9959-2015 Мясо и мясные продукты. Общие условия проведения
органолептической оценкию -2016. -28с.

11. ГОСТ 23042-86 Мясо и мясные продукты. Методы определения жира.
-2010. -6с.

12. ГОСТ 25011-2017. Мясо и мясные продукты. Методы определения белка.
-2018. -25с.

13. Дубровская В.И., Гоноцкий В.А. Продукты из мяса индейки // Птица и
птицепродукты. - 2013. - №3. - С.30--32.

14. Kim JW., Kim H.S. Extraction and characterization of mung bean
proteins using different alkaline solutions// Food Sci Biotechnol.
-2024. -Vol.33(13). -P.3047-3056. DOI 10.1007/s10068-024-01624-x.

15. Эргашев А.Ш., Додаев К.О., Кобилова Г.И., Максумова Д.К.
Использование муки из проросших зёрен маш в производстве соус-паст //
Universum: технические науки. -2022. -№6(99). -С.34-37. DOI
10.32743/UniTech.2022.99.6.13893.

16. Мирходжаева Д.Д., Джахангирова Г.З. Анализ качества и биологическая
ценность машевой муки как потенциального сырья для хлебопекарного производства //
Universum: технические науки. -2020. -№ 8(77). -С.29-30.
\end{refs}

\begin{center}
{\bfseries References}
\end{center}

\begin{refs}
1. Okushanova Je.K. Razrabotka tehnologii mjasnogo pashteta s
primeneniem akusticheskih metodov obrabotki mjasnogo i vtorichnogo
syr' ja: dissertacionnaja rabota d-ra filosofii (PhD).
Semej. -2018. - 153 s.

2. Ahmedova T.P. Ispol' zovanie pishhevyh volokon dlja
obogashhenija pishhevyh produktov // Orel: Izd-vo OrelGIJeT. - 2012. -
S.18--22.

3. Zachesova I.A., Strahova S.A., Kuzina A.A. Razrabotka receptury
pashteta iz krevetok s ispol' zovaniem pshenichnoj //
Vestn. KrasGAU. -2019. -№ 2. -S.139--142.

4. Kozhabayeva S.А., Sartanova N.T. POULTRY SUBCOMPLEX OF KAZAKHSTAN:
PRODUCTION OF TURKEY MEAT// Problem of AgriMarket. -2021. -№3. -P.
100-107. \href{https://doi.org/10.46666/2021-3.2708-9991.11}{DOI
10.46666/2021-3.2708-9991.11}.

5. Sá A.G.A., Moreno Y.M.F., Carciofi B.A.M. Food processing for the
improvement of plant proteins digestibility//Critical Reviews in Food
Science and Nutrition~. -2020. --Vol.60 (20). --P.3367-- 3386.
\href{https://doi.org/10.1080/10408398.2019.1688249}{DOI
10.1080/10408398.2019.1688249}.

6. Chandel K.P.S, Lester R.N., Starling R.J. The wild ancestors of urid
and mung beans (Vigna mungo (L.) Hepper and V. radiata (L.) Wilczek)//
Botanical Journal of the Linnean Society. -2008. --Vol.89 (1). --P.85
- 96. DOI
\href{http://dx.doi.org/10.1111/j.1095-8339.1984.tb01002.x}{10.1111/j.1095-8339.1984.tb01002.x}.

7. Ajtbaev T.E. Mash --- perspektivnaja kul' tura dlja
Kazahstana / Agro-mart. URL:
https://agro-mart.kz/mash-perspektivnaya-kultura-dlya-kazahstana. --Date
of access: 21.08.2025.

8. GOST 31473-2012 Mjaso indeek (tushki i ih chasti). Obshhie
tehnicheskie uslovija. -2013. -12s.

9. GOST 35050-2023 Mash. Tehnicheskie uslovija (s Popravkoj).-2024.
-14s.

10. GOST 9959-2015 Mjaso i mjasnye produkty. Obshhie uslovija
provedenija organolepticheskoj ocenkiju -2016. -28s.

11. GOST 23042-86 Mjaso i mjasnye produkty. Metody opredelenija zhira.
-2010. -6s.

12. GOST 25011-2017. Mjaso i mjasnye produkty. Metody opredelenija
belka. -2018. -25s.

13. Dubrovskaja V.I., Gonockij V.A. Produkty iz mjasa indejki // Ptica i
pticeprodukty. - 2013. - №3. - S.30--32.

14. Kim JW., Kim H.S. Extraction and characterization of mung bean
proteins using different alkaline solutions// Food Sci Biotechnol.
-2024. -Vol.33(13). -P.3047-3056. DOI 10.1007/s10068-024-01624-x.

15. Jergashev A.Sh., Dodaev K.O., Kobilova G.I., Maksumova D.K.
Ispol' zovanie muki iz prorosshih zjoren mash v
proizvodstve sous-past // Universum: tehnicheskie nauki. -2022. -№6(99).
-S.34-37. DOI 10.32743/UniTech.2022.99.6.13893.

16. Mirhodzhaeva D.D., Dzhahangirova G.Z. Analiz kachestva i
biologicheskaja cennost'{} mashevoj muki kak potencial' nogo syr' ja
dlja hlebopekarnogo proizvodstva // Universum: tehnicheskie nauki.
-2020. -№ 8(77). -S.29-30.
\end{refs}

\begin{info}
\hspace{1em}\emph{{\bfseries Авторлар туралы мәліметтер}}

Конарбаева З.К. - PhD, М.Әуезов атындағы оңтүстік Қазақстан
университеті, Шымкент, Қазақстан, e-mail:
z.konarbayeva@auezov.edu.kz;

Нурсеитова З.Т. - т.ғ.к., доцент, М.Әуезов атындағы оңтүстік Қазақстан
университеті, Шымкент, Қазақстан, e-mail:
nur.zeinep@mail.ru;

Нурынбетова Г.Ж. - магистр, оқытушы. М.Әуезов атындағы оңтүстік
Қазақстан университеті, Шымкент, Қазақстан, e-mail:
gulnur\_ailan@mail.ru;

Мулдабекова Б.Ж.- т.ғ.к., профессор, алматы Технологиялық университеті,
Алматы, Қазақстан, e-mail:
bayan\_1004@mail.ru;

Әбді А.Қ. -- магистрант, М.Әуезов атындағы оңтүстік Қазақстан
университеті, Шымкент, Қазақстан, e-mail:
erkewa2001@icloud.com.

\hspace{1em}\emph{{\bfseries Information about the authors}}

Konarbayeva Z.K. - PhD, M.Auezov South Kazakhstan University, Shymkent,
Kazakhstan, e-mail:
z.konarbayeva@auezov.edu.kz;

Nurseitova Z.T. -- Candidate of Technical Sciences, Associate Professor,
M.Auezov South Kazakhstan University, Shymkent, Kazakhstan, e-mail:
nur.zeinep@mail.ru;

Nurynbetova G.Zh. - Master, Lecturer. M.Auezov South Kazakhstan
University, Shymkent, Kazakhstan, e-mail:
gulnur\_ailan@mail.ru;

Muldabekova B.Zh. - Candidate of Technical Sciences, Professor, Almaty
Technological University, Almaty, Kazakhstan e-mail:
bayan\_1004@mail.ru;

Abdi A.K. -- Master' s student, M. Auezov South
Kazakhstan University, Shymkent, Kazakhstan, e-mail:
erkewa2001@icloud.com.
\end{info}
