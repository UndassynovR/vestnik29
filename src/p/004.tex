\id{МРНТИ 65.59.03}{}

\begin{header}
\swa{}{ИССЛЕДОВАНИЕ МИКРОБИОЛОГИЧЕСКИХ И ФИЗИКО-ХИМИЧЕСКИХ ПОКАЗАТЕЛЕЙ ПОЛУКОПЧЁНОЙ КОЛБАСЫ ИЗ РАСТИТЕЛЬНОГО ЗАМЕНИТЕЛЯ МЯСА С ДОБАВЛЕНИЕМ ПЛАЗМЫ КРОВИ}

Э.Т. Кансейтова\envelope,
Ж.С. Желеуова,
А.Ж. Шиналиева,
М.Д. Абишев,
И.Р. Садырбаева
\end{header}

\begin{affil}
Южно-Казахстанский университет им. М. Ауэзова, Шымкент, Казахстан

\corrauthor{Корреспондент-автор: kanseitova@bk.ru}
\end{affil}

В данной работе представлена разработка технологии
полукопчёной колбасы функционального назначения с использованием
растительного сырья (соевый изолят и нут) и добавлением плазмы крови
убойных животных в качестве компонента животного происхождения.
Проведены исследования влияния различных концентраций плазмы крови (1\%,
2\%, 3\%) на пищевую ценность, органолептические, микробиологические и
физико-химические показатели колбасных изделий. Установлено, что
добавление 2\% плазмы крови приводит к умеренному увеличению
водосвязывающей способности (ВСС) и водоудерживающей способности (ВУС),
а также положительно влияет на органолептические показатели образца.
Такое улучшение функционально-технологических свойств происходит за счёт
усиления гелеобразования и повышения сочности продукта. Однако при
увеличении доли плазмы до 3 \% наблюдается чрезмерное уплотнение
текстуры, что негативно отражается на потребительских свойствах
продукта.

Установлено, что добавление 2\% плазмы крови обеспечивает оптимальное
соотношение белков, углеводов, влаги и жира, улучшает влагоудерживающую
способность и сохраняет стабильную консистенцию продукта. При этом
продукт соответствует санитарно-гигиеническим требованиям и сохраняет
высокие органолептические характеристики в течение срока хранения.
Результаты подтверждают технологическую и биологическую целесообразность
использования плазмы крови в составе растительных колбасных изделий для
повышения их функциональной и пищевой ценности.

{\bfseries Ключевые слова:} плазма крови, соя, нут, полукопченая колбаса,
заменитель мяса, растительное сырье.

\begin{header}
ӨСІМДІК ТЕКТІ ЕТ АЛМАСТЫРҒЫШЫ ҚОСЫЛҒАН ҚАН ПЛАЗМАСЫМЕН ЖАРТЫЛАЙ ЫСТАЛҒАН ШҰЖЫҚТЫҢ МИКРОБИОЛОГИЯЛЫҚ ЖӘНЕ ФИЗИКА-ХИМИЯЛЫҚ КӨРСЕТКІШТЕРІН ЗЕРТТЕУ

Э.Т. Кансейтова\envelope,
Ж.С. Желеуова,
А.Ж. Шиналиева,
М.Д. Абишев,
И.Р. Садырбаева
\end{header}

\begin{affil}
М. Әуезов атындағы Оңтүстік Қазақстан университеті, Шымкент, Қазақстан,

e-mail: kanseitova@bk.ru
\end{affil}

Бұл жұмыста жануар тектес компонент ретінде сойыс малдарының қанының
плазмасын қосып, өсімдік текті шикізат (соя изолятты және ноқат)
негізінде функционалды мақсаттағы жартылай ысталған шұжық технологиясы
әзірленді. Әртүрлі концентрациядағы қан плазмасының (1\%, 2\%, 3\%)
шұжық өнімдерінің тағамдық құндылығына, органолептикалық,
микробиологиялық және физ\-ика-химиялық көрсеткіштеріне әсері зерттелді.

Зерттеу нәтижелері бойынша, 2\% қан плазмасын енгізу су байланыстыру
(СБҚ) және су ұстау қабілетін (СҰҚ) орташа деңгейде арттырып,
органолептикалық қасиеттерін жақсартатыны анықталды.
Функционалды-технологиялық көрсеткіштердің мұндай артуы өнімнің гель
түзу қабілетінің күшеюімен және шырындылығының жоғарлауымен байланысты.
Алайда плазма үлесі 3\%-ға дейін арттырылғанда, текстураның шамадан тыс
тығыздануы байқалып, бұл өнімнің тұтынушылық қасиеттеріне кері әсер
етті.

Осылайша, 2\% плазма қосу ақуыздар, көмірсулар, ылғал және майдың
оңтайлы арақатынасын қамтамасыз етеді, су ұстау қабілетін арттырады және
өнімнің тұрақты консистенциясын сақтайды. Сонымен қатар, шұжық
санитарлық-гигиеналық талаптарға сәйкес келеді және сақтау мерзімі бойы
жоғары органолептикалық сипаттамаларын сақтайды. Алынған нәтижелер қан
плазмасын өсімдік негізіндегі шұжық өнімдерінің құрамында қолданудың
технологиялық және биологиялық тұрғыдан мақсатқа сай екенін дәлелдейді.

{\bfseries Түйін сөздер:} қан плазмасы, соя, ноқат, жартылай ысталған
шұжық, ет алмастырғыш, өсімдік тектi шикізат.

\begin{header}
STUDY OF MICROBIOLOGICAL AND PHYSICO-CHEMICAL PARAMETERS OF SEMI-SMOKED SAUSAGE MADE FROM A PLANT-BASED MEAT SUBSTITUTE WITH THE ADDITION OF BLOOD PLASMA

E.T. Kanseitova\envelope,
Zh.S. Zheleuova,
A.Zh. Shinalieva,
M.D. Abishev,
I.R. Sadyrbayeva
\end{header}

\begin{affil}
M. Auezov South Kazakhstan University, Shymkent, Kazakhstan,

e-mail: kanseitova@bk.ru
\end{affil}

This article presents the development of a technology for semi-smoked
sausages of functional purpose, based on plant raw materials (soy
isolate and chickpea) with the addition of animal-derived plasma. The
effect of different concentrations of blood plasma (1\%, 2\%, 3\%) on
the nutritional value, organoleptic, microbiological, and
physicochemical parameters of sausage products was investigated.

It was found that the addition of 2\% blood plasma moderately increased
water-binding capacity (WBC) and water-holding capacity (WHC), while
also improving the organoleptic characteristics of the samples. This
improvement in functional-technological properties is associated with
enhanced gel formation and increased juiciness of the product. However,
when the plasma content was increased to 3\%, excessive texture
densification was observed, negatively affecting the consumer qualities
of the sausage.

Thus, the addition of 2\% plasma ensures an optimal balance of proteins,
carbohydrates, moisture, and fat, improves water-holding capacity, and
maintains a stable product consistency. At the same time, the sausage
meets sanitary and hygienic requirements and retains high organoleptic
characteristics throughout its shelf life. The obtained results confirm
the technological and biological feasibility of using blood plasma in
plant-based sausage products to enhance their functional and nutritional
value.

{\bfseries Keywords:} blood plasma, soy, chickpea, semi-smoked sausage,
meat substitute, plant-based raw mate\-rials.

\begin{multicols}{2}
{\bfseries Введение.} Современные потребности потребителей и переход на
растительные альтернативы замены мяса неуклонно растет в последние
несколько лет, как в дальнем, так и в ближнем зарубежье. Растительную
альтернативу мясных продуктов производят так, чтобы сымитировать вкус,
внешний вид, текстуру, пищевую ценность и белковую основу.

Обеспечение качества и безопаснос

ти пищевых продуктов остается одной из приоритетных задач пищевой
промышленности и санитарного надзора, новых разработанных изделий. В
этой связи исследование микробиологических и физико-химических
показателей полукопченой колбасы из растительного заменителя мяса с
добавлением плазмы сельскохозяйственных животных представляет собой
важное направление в области контроля качества и безопасности пищевых
продуктов, способствующее повышению эффективности технологических
процессов и защите здоровья населения {[}1{]}.

Одним из важнейших показателей, определяющих органолептическую оценку и
экономическую эффективность производства, является влагоудерживающая
(ВУС) и влагосвязывающая (ВСС) способность фарша. Не способность
удерживать влагу при внешнем воздействии и степень фиксации влаги в
структуре белков на молекулярном уровне, может привести к потере массы
при термообработке, ухудшению структуры, сочности и внешнего вида.

Таким образом, результаты исследования микробиологических и
физико-химических показателей, влагоудерживающей и влагосвязывающей
способности полукопчёной колбасы из растительного заменителя мяса с
добавлением плазмы, влияющих на данные характеристики, имеют значение
при совершенствовании технологии, повышении качества и
конкурентоспособности.

В колбасном производстве используют широкий ассортимент крови убойных
животных и ее компонентов, которая является ценным источником животного
белка и других необходимых человеку компонентов - жиров, углеводов,
ферментов, витаминов и минеральных веществ. Кровь животных содержит
60-70\% плазмы и 30-40\% взвешенных эритроцитов по весу {[}2{]}.

По литературным данным плазма -- жидкая часть, не содержащая клеток
крови {[}3{]}, содержит 91\% воды, 7\% белков, 1\% минеральных веществ,
в ней присутствуют все белки крови, за исключением гемоглобина {[}4{]}.
При этом в ней содержится более 100 различных белков, из которых
наиболее представительными являются сывороточные белки альбумин,
γ-глобулины, α-глобулины, β-глобулины и фиброген {[}5{]}, которые
используются из-за различных функциональных свойств {[}6{]}.

Результаты исследований {[}7, 8{]} показывают, что использование белков
плазмы крови можно рассматривать как связующее вещество для производства
различных пищевых продуктов.

{\bfseries Материалы и методы.} Объектами исследования служили опытные
образцы полукопчёной колбасы из растительного заменителя мяса с
добавлением плазмы сельскохозяйственных животных, основным сырьем
которых являлись соя, нут и кровь убойных животных.

Полукопчёная колбаса из растительного заменителя мяса с добавлением
плазмы была приготовлена в соответствии с существующими технологическими
инструкциями, технологический процесс осуществлялся с соблюдением
санитарных правил для предприятий мясной промышленности при первичной
обработке растительного сырья (соя, нут) в соотношении 70/30 {[}9,
10{]}. Приготовление фарша на растительной основе проводили в мешалке с
внесением плазмы (1\%, 2\%, 3\%), добавлением поваренной соли,
растительного масла (подсолнечное и кокосовое) и пряностей (перец
душистый и кориандр молотый), перемешивая в течение 8-10 мин. Затем
наполняли оболочку, обвязывая батоны, после усадки проводили
термообработку, сушку, обжарку, варку, охлаждение при температуре
20\tsp{о}С и ниже, копчение проводили при температуре
43±7\tsp{о}С, 12-24 ч, далее проводили сушку и отправляли на
хранение.

При разделении фракции цельной крови использовали метод
центрифугирования (Stegler СМ-300-06, 4000 об/мин, 15 мин) с добавлением
антикоагулянта в соотношении 1:10 (4\% раствор цитрата натрия и лимонной
кислоты).

Влагосвязывающую и влагоудерживающую способность образцов определяли в
испытательной лаборатории АО «Алматинский технологический университет»,
методом прессования и Вартаняна соответственно.

Микробиологическую оценку опытного образца проводили в испытательной
лаборатории ТОО «Нутритест» по общепринятым методикам определения БГКП
(ГОСТ 31747-2012), сульфитредуцирующих клостридий (ГОСТ 29185-2014)
31746-2012), бактерий вида Staphylococcus aureus (ГОСТ 31746-2012),
токсичных элементов (свинец, кадмий (ГОСТ30178-96), мышьяка (ГОСТ
31628-2012), ртуть (ГОСТ 26927-86)), пестицидов, ДДТ и его метаболитов
(СТ РК 2011-2010), микотоксинов (ГОСТ 30711-2001).

Определение физико-химических показателей - массовой доли белка (ГОСТ
26889-86), влаги (ГОСТ 9793-2016), жира (ГОСТ 23042-2015) и углеводов
(И.М. Скурихин, вып.1, 1978) - опытных образцов проводили по
общепринятым методикам.

{\bfseries Обсуждение и результаты.} Разработана технология получения
полукопченой колбасы функционального назначения из продуктов животного
(плазмы - крови сельскохозяйственных животных) и растительного
происхождения (соевый изолят и нут).

В ходе исследования были определены органолептические характеристики
полукопченой колбасы из растительного сырья с добавлением плазмы крови
убойных животных. Установлено, что добавление плазмы крови в указанных
дозировках не оказывало существенного влияния на такие показатели, как
внешний вид, вкус и цвет продукта. Эти параметры во всех опытных
образцах соответствовали установленным требованиям для данного вида
колбасных изделий.

Исследованы пищевая безопасность, микробиологические, органолептические
и физико-химические показатели полученных функциональных полукопченых
колбас из растительного заменителя мяса.

Исследования пищевой ценности проводили для оценки влияния добавления
плазмы крови в различных концентрациях (1\%, 2\%, 3\%), таблица 1.
\end{multicols}

\tcap{Таблица 1 - Пищевая ценность полукопченой колбасы из растительного заменителя мяса}
\begin{longtblr}[
  label = none,
  entry = none,
]{
  width = \linewidth,
  colspec = {Q[188]Q[177]Q[223]Q[223]Q[223]},
  cells = {c},
  cells = {font = \small},
  hlines,
  vlines,
}
\textbf{Наиме\-нование показателей} & \textbf{Полукопченая колбаса из растительного заменителя мяса (контроль)} & {\textbf{Полукопченая колбаса из растительного заменителя мяса с добавлением 1\% плазмы крови}\\\textbf{(1 образец)}} & {\textbf{Полукопченая колбаса из растительного заменителя мяса с добавлением 2\% плазмы крови}\\\textbf{(2 образец)}} & {\textbf{Полукопченая колбаса из растительного заменителя мяса с добавлением 3\% плазмы крови}\\\textbf{(3 образец)}} \\
Массовая доля влаги, \%           & 49,95±2,5                                                                 & 47,28±2,36                                                                                                            & 47,27±2,36                                                                                                            & 47,3±2,37                                                                                                             \\
Массовая доля жира, \%            & 3,34±0,2                                                                  & 3,39±0,2                                                                                                              & 4,49±0,67                                                                                                             & 5,8±0,35                                                                                                              \\
Массовая доля углеводов, \%       & 19,81±1,0                                                                 & 22,69±1,13                                                                                                            & 22,56±1,13                                                                                                            & 20,07±1,0                                                                                                             \\
Массовая доля белка, \%           & 16,90±1,0                                                                 & 24,34±1,46                                                                                                            & 24,49±1,47                                                                                                            & 24,54±1,47                                                                                                            \\
Энерге\-тическая ценность, ккал     & 183,99                                                                    & 214,94                                                                                                                & 215,51                                                                                                                & 228,08                                                                                                                
\end{longtblr}

\begin{multicols}{2}
В таблице 1 представлены данные о влиянии добавления плазмы крови в
различных концентрациях (1\%, 2\%, 3\%) на показатели пищевой ценности
полукопченой колбасы, изготовленной из растительного заменителя мяса. В
качестве контрольного образца выступает продукт без добавления плазмы.

С увеличением доли плазмы крови наблюдается тенденция к снижению
массовой доли влаги по сравнению с контрольным образцом от 49,95\% в
контроле до 47,3\% при всех уровнях добавления, что свидетельствует о
более плотной структуре продукта за счет увеличения содержания белков и
жиров.

Массовая доля жира повышается по мере увеличения дозировки плазмы от
3,34\% в контрольной группе до 5,8\% при добавлении 3\% плазмы, что
может быть связано с содержанием липидов в плазме крови.

Содержание углеводов, увеличивается при добавлении 1\% и 2\% плазмы,
однако при добавлении 3\% снижается, приближаясь к контрольному
значению.

Наиболее значительные изменения наблюдаются в содержании белка с 16,9\%
в контрольной пробе до 24,54 при добавлении плазмы крови, что
свидетельствует об эффективности использования плазмы как белкового
обогатителя.

Соответственно энергетическая ценность колбасы возрастает с 183,99 ккал
до 228,08 ккал при максимальной концентрации плазмы (3\%).

Таким образом, добавление плазмы крови способствует повышению пищевой и
энергетической ценности растительной полукопченой колбасы за счет
увеличения содержания белка и жира. Добавление плазмы крови привело к
росту калорийности продукта, это объясняется увеличением содержания
белков в зависимости от концентрации плазмы крови.

Функционально-технологические характеристики опытных образцов описаны в
таблице 2.
\end{multicols}

\tcap{Таблица 2 - Функционально-технологические характеристики}
\begin{longtblr}[
  label = none,
  entry = none,
]{
  width = \linewidth,
  colspec = {Q[104]Q[279]Q[279]Q[279]},
  cells = {c},
  cells = {font = \small},
  hlines,
  vlines,
}
\textbf{Наимено\-вание показателей} & {\textbf{Полукопченая колбаса из растительного заменителя мяса с добавлением 1\% плазмы крови}\\\textbf{(1 образец)}} & {\textbf{Полукопченая колбаса из растительного заменителя мяса с добавлением 2\% плазмы крови}\\\textbf{(2 образец)}} & {\textbf{Полукопченая колбаса из растительного заменителя мяса с добавлением 3\% плазмы крови}\\\textbf{(3 образец)}} \\
ВСС, \%                           & 44,89±0,59                                                                                                            & 49,38±0,73                                                                                                            & 54,78±0,73                                                                                                            \\
ВУС, \%                           & 42,47±0,55                                                                                                            & 47,89±0,61                                                                                                            & 48,92±0,51                                                                                                            
\end{longtblr}

\begin{multicols}{2}
По данным таблицы 2, при увеличении содержания плазмы от 1 до 2\%
наблюдается значительное повышение как ВСС (на 4,49\%), так и ВУС (на
5,42\%). Максимальные значения ВСС (54,78\%) и ВУС (48,92\%)
зафиксированы при добавлении 3\% плазмы.
\end{multicols}

{\bfseries Рис.1 - Диаграмма функционально-технологических характеристик образцов}

\begin{multicols}{2}
По данным таблицы 2 и рисунка 1 наблюдается умеренное увеличение ВСС и
ВУС, но наблюдается изменение органолептических показателей.
Установлено, что добавление 2\% плазмы крови приводит к умеренному
увеличению водосвязывающей способности (ВСС) и водоудерживающей
способности (ВУС), а также положительно влияет на органолептические
показатели образца. Такое улучшение функционально технологических
свойств происходит за счет усиления гелеобразования и повышения сочности
продукта. Однако при увеличении доли плазмы до 3 \% наблюдается
чрезмерное уплотнение текстуры, что негативно отражается на
потребительских свойствах продукта.

Однако влияние добавки плазмы крови прослеживалось в изменении
структурно-механических свойств, в частности консистенции ухудшения
текстуры продукта, проявляющееся в виде рыхлой консистенции, что может
свидетельствовать о нарушении оптимального соотношения компонентов и
ухудшении структурной целостности колбасы.
\end{multicols}

\tcap{Таблица 3 - Результаты микробиологического исследования и пищевой безопасности}
\begin{longtblr}[
  label = none,
  entry = none,
]{
  row{1} = {c},
  cell{2}{2} = {c},
  cell{3}{2} = {c},
  cell{4}{2} = {c},
  cell{5}{2} = {c},
  cell{6}{2} = {c},
  cell{7}{2} = {c},
  cell{8}{2} = {c},
  cell{9}{2} = {c},
  cell{10}{2} = {c},
  cell{11}{2} = {c},
  cell{12}{2} = {c},
  cells = {font = \small},
  hlines,
  vlines,
}
\textbf{Исследуемый показатель}                                            & \textbf{Результаты исследования} \\
Сульфитредуцирующие клостридии, в 1г                                       & не обнаружено в 1 г продукта     \\
БГКП                                                                       & не обнаружено в 1 г продукта     \\
бактерии вида Staphylococcus aureus                                        & не обнаружено в 1 г продукта     \\
{Количество мезофильных аэробных и \\факультативно-анаэробных микроорганизмов} & 0,82х10\textsuperscript{3 }КОЕ/г \\
Свинец                                                                     & не обнаружено                    \\
Кадмий                                                                     & не обнаружено                    \\
Мышьяк                                                                     & не обнаружено                    \\
Ртуть                                                                      & не обнаружено                    \\
Афлатоксин В1                                                              & не обнаружено                    \\
ГХЦГ (α, β, γ-изомеры)                                                     & не обнаружено                    \\
ДДТ и его метаболиты                                                       & не обнаружено                    
\end{longtblr}

\begin{multicols}{2}
БГКП (ГОСТ 31747-2012), (ГОСТ 29185-2014) 31746-2012), бактерии вида
Staphylococcus aureus (ГОСТ 31746-2012), токсичные элементы
(ГОСТ30178-96), (ГОСТ 31628-2012), (ГОСТ 26927-86)), пестициды, ДДТ и
его метаболиты (СТ РК 2011-2010).

По результатам микробиологической оценки в опытном образце с добавлением
2\% плазмы присутствие БГКП, сульфитредуцирующих клостридии, бактерии
вида Staphylococcus aureus, токсичных элементов (свинец, кадмий, мышьяк,
ртуть), пестицидов (ГХЦГ (α, β, γ-изомеры), ДДТ и его метаболиты),
микотоксинов (афлатоксин В\tsb{1}) не было выявлено.
Количество мезофильных аэробных и факультативно-анаэробных
микроорганизмов 0,82х10\tsp{3} КОЕ/г, что значительно меньше
нормативного показателя

Из выше полученных данных наиболее сбалансированным и технологически
целесообразным является добавление двух процентов плазмы крови, при
котором достигается оптимальное соотношение органолептических и
функционально-технологических характеристик полукопченой колбасы на
растительной основе.

Добавление плазмы крови в рецептуру полукопченой колбасы значительно
улучшает пищевую и биологическую ценность продукта. Все образцы
демонстрируют улучшенные показатели пищевой ценности и могут
рассматриваться как альтернатива традиционным мясным ингредиентам с
целью повышения белка.

Добавление плазмы крови в количестве 1, 2, 3 \% в рецептуру полукопчёной
колбасы на растительной основе способствует повышению пищевой и
биологической ценности продукта, в частности, увеличению
массовой доли белка (до 24,5 \%) и энергетической ценности (до 228,08
ккал).

Установлено, что наиболее сбалансированным вариантом с точки зрения
функционально-технологических, органолептических и микробиологических
показателей является образец с добавлением 2 \% плазмы крови.
Данный образец демонстрирует влагоудерживающую способность (47,89 \%),
стабильную консистенцию, отсутствие отклонений по микробиологическим и
санитарно-гигиеническим нормам.

Органолептические исследования показали, что внешний вид, вкус и цвет
продукта не изменяются при введении плазмы крови, а консистенция
ухудшается только при максимальной дозировке (3 \%).

Установлено, что добавление 2\% плазмы крови приводит к умеренному
увеличению водосвязывающей способности (ВСС) и водоудерживающей
способности (ВУС), а также положительно влияет на органолептические
показатели образца. Такое улучшение функционально-технологических
свойств становится возможным за счет усиления гелеобразования и
повышения сочности продукта. Однако при увеличении доли плазмы до 3 \%
наблюдается чрезмерное уплотнение текстуры, что негативно отражается на
потребительских свойствах продукта.

{\bfseries Выводы}. Результаты, полученные в ходе проведенных исследований,
позволяют сделать следующие выводы: установлена возможность и
целесообразность использования плазмы крови при производстве
полукопчёной колбасы из растительного заменителя мяса.

Проведённые микробиологические испытания образца, где подтверждена
безопасность образцов с добавлением 2\% плазмы крови, свидетельствуют о
том, что продукт пригоден для употребления и соответствует нормативным
требованиям.

Таким образом, использование 2\% плазмы крови в производстве
полукопчёной колбасы из растительного заменителя мяса является
оптимальным решением, позволяющим повысить белковую ценность
продукта, улучшить его технологические свойства и сохранить
безопасность и стабильность в течение срока хранения.

\emph{{\bfseries Финансирование.} Данные исследования проводились в рамках
проекта АР19679729 «Разработка технологии полукопченой колбасы из
растительного заменителя мяса функционального назначения» и
финансируется Комитетом науки Министерства науки и высшего образования
Республики Казахстан.}
\end{multicols}

\begin{center}
{\bfseries Литература}
\end{center}

\begin{refs}
1. Ахметова~В.Ш., Машанова~Н.С. Функционалдык тамактануга арналган
к¥рама ет 0німінін технологиясы// Вестник Университета Шакарима.- 2020.
-№ 4(92). - С.108-111.

2. Ofori J.A., Yun-Hwa Hsieh P. Issues related to the use of blood in
food and animal feed //Critical reviews in food science and
nutrition.-2014.- Vol.54(5).- P.687-697. DOI 10.1080/10408398.2011.605229.

3. Zaitsev S. Dynamic surface tension measurements as general approach
to the analysis of animal blood plasma and serum //~Advvances in colloid
interface science.- 2016.-Vol.235.-P.201-213. DOI
10.1016/j.cis.2016.06.007.

4. Del Hoyo P., Moure F., Rendueles M., Díaz M. Demineralization of
animal blood plasma by ion exchange and ultrafiltration//Meat
science.-2007.-Vol.76(3).-P.402-410. DOI 10.1016/j.meatsci.2006.06.014.

5. Tarté R. Meat Protein Ingredients // Woodhead Publishing Limited.-
2011 - Chapter:~4. - P.56-91. DOI 10.1533/9780857093639.56.

6. Clara S.F. Bah, Alan Carne, Michelle A. McConnell, Sonya Mros, Alaa
El-Din A. Bekhit, Production of bioactive peptide hydrolysates from
deer, sheep, pig and cattle red blood cell fractions using plant and
fungal protease preparations//Food Chemistry.-2016.-Vol.202.-P.458-466.
DOI 10.1016/j.foodchem.2016.02.020.

7. Chiroque R. G. S., Santiago H. P. C., Espinoza L. E., Quispe L. A.
M., Tirado L. R. P., Villegas L. M. N., Villegas M. A. N., A Review of
Slaughterhouse Blood and its Compounds, Processing and Application in
the Formulation of Novel Non-Meat Products // Current Research Nutrition
and Food Science. -2023.-Vol.11(2).- P.534-548. DOI
\href{http://dx.doi.org/10.12944/CRNFSJ.11.2.06}{10.12944/CRNFSJ.11.2.06}

8. Suter D.A., Sustek E., Dill C.W., Marshall W.H., Carpenter Z.L. A
method for measurement of the effect of blood protein concentrates on
the binding forces in cooked ground beef patties // Journal of Food
Science. -1976.-Vol.41 (6).-P.1428-1432. DOI
10.1111/j.1365-2621.1976.tb01188.x

9. Желеуова Ж.С., Кансейтова Э.Т., Тасполтаева А.Р., Балабекова А.С.,
Қоштаева Г.Е. Өсімдік негізіндегі ет алмастырғыштан жасалған жартылай
ысталған шұжықтың аминқышқылдары мен май қышқылдарының құрамын зерттеу//
Механика және технологиялар.-2024.- №3(85). -
Б.115-127\emph{.~}\href{https://doi.org/10.55956/ZZIV9315}{DOI
10.55956/ZZIV9315}

10. Желеуова~Ж.С., Шингисов~А.У., Тасполтаева~А.Р., Кансейтова~Э.Т.,
Бердембетова~А.Т. Исследование органолептических показателей и
минерального состава полукопченой колбасы из растительного
сырья.~Вестник Университета Шакарима.-2024.- № 3(15).-
C.118-127.~\href{https://doi.org/10.53360/2788-7995-2024-3(15)-17}{DOI
10.53360/2788-7995-2024-3(15)-17}.
\end{refs}

\begin{center}
{\bfseries References}
\end{center}

\begin{refs}
1. Ahmetova V.Sh., Mashanova N.S. Funkcionaldyk tamaktanuga arnalgan
kуrama et оnіmіnіn tehnologijasy// Vestnik Universiteta Shakarima.-
2020. -№ 4(92). - S.108-111.{[}in Kazakh{]}

2. Ofori J.A., Yun-Hwa Hsieh P. Issues related to the use of blood in
food and animal feed //Critical reviews in food science and
nutrition.-2014.- Vol.54(5).- P.687-697. DOI 10.1080/10408398.2011.605229.

3. Zaitsev S. Dynamic surface tension measurements as general approach
to the analysis of animal blood plasma and serum //~Advvances in colloid
interface science.- 2016.-Vol.235.-P.201-213. DOI
10.1016/j.cis.2016.06.007.

4. Del Hoyo P., Moure F., Rendueles M., Díaz M. Demineralization of
animal blood plasma by ion exchange and ultrafiltration//Meat
science.-2007.-Vol.76(3).-P.402-410. DOI 10.1016/j.meatsci.2006.06.014.

5. Tarté R. Meat Protein Ingredients // Woodhead Publishing Limited.-
2011 - Chapter:~4. - P.56-91. DOI 10.1533/9780857093639.56.

6. Clara S.F. Bah, Alan Carne, Michelle A. McConnell, Sonya Mros, Alaa
El-Din A. Bekhit, Production of bioactive peptide hydrolysates from
deer, sheep, pig and cattle red blood cell fractions using plant and
fungal protease preparations//Food Chemistry.-2016.-Vol.202.-P.458-466.
DOI 10.1016/j.foodchem.2016.02.020.

7. Chiroque R. G. S., Santiago H. P. C., Espinoza L. E., Quispe L. A.
M., Tirado L. R. P., Villegas L. M. N., Villegas M. A. N., A Review of
Slaughterhouse Blood and its Compounds, Processing and Application in
the Formulation of Novel Non-Meat Products // Current Research Nutrition
and Food Science. -2023.-Vol.11(2).- P.534-548. DOI
\href{http://dx.doi.org/10.12944/CRNFSJ.11.2.06}{10.12944/CRNFSJ.11.2.06}

8. Suter D.A., Sustek E., Dill C.W., Marshall W.H., Carpenter Z.L. A
method for measurement of the effect of blood protein concentrates on
the binding forces in cooked ground beef patties // Journal of Food
Science. -1976.-Vol.41 (6).-P.1428-1432. DOI
10.1111/j.1365-2621.1976.tb01188.x

9. Zheleuova Zh.S., Kansejtova Je.T., Taspoltaeva A.R., Balabekova A.S.,
Қoshtaeva G.E. Өsіmdіk negіzіndegі et almastyrғyshtan zhasalғan
zhartylaj ystalғan shұzhyқtyң aminқyshқyldary men maj қyshқyldarynyң
құramyn zertteu// Mehanika zhәne tehnologijalar.- 2024.- №3(85). -
B.115-127. DOI 10.55956/ZZIV9315. {[}in Kazakh{]}

10. Zheleuova Zh.S., Shingisov A.U., Taspoltaeva A.R., Kansejtova Je.T.,
Berdembetova A.T. Issledovanie organolepticheskih pokazatelej i
mineral' nogo sostava polukopchenoj kolbasy iz
rastitel' nogo syr' ja. Vestnik
Universiteta Shakarima.-2024.- № 3(15).- C.118-127. DOI
10.53360/2788-7995-2024-3(15)-17. {[}in Russian{]}
\end{refs}

\begin{info}
\hspace{1em}\emph{{\bfseries Сведения об авторах}}

Кансейтова Э.Т.- кандидат сельскохозяйственных наук, доцент,
Южно-Казахстанский университет им. М. Ауэзова, Шымкент, Казахстан,
e-mail: kanseitova@bk.ru;

Желеуова Ж.С.-PhD, доцент, Южно-Казахстанский университет им. М.
Ауэзова, Шымкент, Казахстан, e-mail: zhozi\_tima@mail.ru; ORCID:

Шиналиева А.Ж. - докторант, Южно-Казахстанский университет им.М.
Ауэзова, Шымкент, Казахстан, e-mail: ainur\_09\_09@mail.ru; .

Абишев М.Д. - кандидат технических наук, доцент, Южно-Казахстанский
университет им. М. Ауэзова, Шымкент, Казахстан, e-mail:
marat.abishev.63@ mail.ru;

Садырбаева И.Р.- старший преподаватель, Южно-Казахстанский университет
им. М. Ауэзова, Шымкент, Казахстан, e-mail:
indirasadyrbaeva76@mail.ru.

\hspace{1em}\emph{{\bfseries Information about authors}}

Kanseitova E.T. - Candidate of Agricultural Sciences, Associate
Professor, M. Auezov South Kazakhstan University, Shymkent,
Kazakhstan, e-mail: kanseitova@bk.ru;

Zheleuova Zh. S. -- PhD, Associate Professor, M. Auezov South
Kazakhstan University, Shymkent, Kazakhstan, e-mail:
zhozi\_tima@mail.ru;

Shinaliyeva A. Zh. - Doctoral Student, M. Auezov South Kazakhstan
University, Shymkent, Kazakhstan, e-mail: ainur\_09\_09@mail.ru;

Abishev M. J/ - Candidate of Technical Sciences, Associate Professor,
M.  Auezov South Kazakhstan University, Shymkent, Kazakhstan, e-mail:
marat.abishev.63@ mail.ru;

Sadyrbaeva I. R. - Senior Lecturer, M. Auezov South Kazakhstan
University, Shymkent, Kazakhstan, e-mail: indirasadyrbaeva76@mail.ru.
\end{info}
