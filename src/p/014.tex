\id{ҒТАМР 65.65.03}{}

\begin{header}
\swa{}{ТАҒАМДЫҚ ЭМУЛЬСИЯ ӨНІМДЕРІН ЖАСАУ КЕЗІНДЕ МАҚСАРЫ МАЙЫН ҚОЛДАНУ БОЛАШАҒЫ}

\tsp{1}М.Ж. Султанова\envelope,
\tsp{1}Ф.З. Сеитова,
\tsp{2}Н. Акжанов
\end{header}

\begin{affil}
\tsp{1}Алматы технологиялық университеті, Алматы, Қазақстан,

\tsp{2}«Қазақ қайта өңдеу және тағам өнеркәсіптері ҒЗИ» ЖШС АФ, Астана, Қазақстан,

\corrauthor{Корреспондент-автор: sultanova.2012@mail.ru}
\end{affil}

Өсімдік майларына негізделген функционалды тағам өнімдерін әзірлеу
қазіргі заманғы тағам ғылымы мен технологиясының перспективалы
бағыттарының бірі болып табылады. Бұл жұмыста полиқанықпаған май
қышқылдарының, токоферолдардың және антиоксидантты қосылыстардың жоғары
құрамымен сипатталатын мақсары майы (Carthamus tinctorius L.) екі
эмульсиялық өнімді: жоғары калориялы майонезді және сарымсақ пен зімбір
тұздығын жасауда негізгі май негізі ретінде қолданылады.

Тәжірибелік партиялар зертханалық жағдайда стандартты
араластырғыш-гомогенизациялық жабдықты қолдана отырып дайындалды.
Рецептуралар май, ақуыз және көмірсу компоненттерінің оңтайлы
арақатынасын ескере отырып құрылды, сонымен қатар өнімдердің тағамдық
және биологиялық құндылығын арттыруға мүмкіндік беретін функционалды
қасиеттері бар табиғи қоспаларды (сарымсақ, имбирь, аскорбин қышқылы)
қамтыды. Әрбір нұсқа үшін эмульсиялардың тұрақтылығын және сақтау
мерзімін ұзартуды қамтамасыз ететін технологиялық схема жасалды.

Физика-химиялық, реологиялық және органолептикалық сипаттамаларға баға
берілді. Реологиялық зерттеулер өнімдердің құрылымдық қасиеттеріндегі
айырмашылықтарды анықтады: майонез жоғары тұтқырлығы бар (2000-5000
мПа•с) және айқын тиксотропиясы бар жалған пластикалық ағын түрін
көрсетті, ал тұздық тұтқыр пластикалық ағынмен және орташа тұтқырлықпен
(800-1500 мПа•с) сипатталды. Физика-химиялық талдау май фазасының
тұрақтылығын растайтын қышқыл және пероксид сандарының рұқсат етілген
мәндерін көрсетті.20 маманнан тұратын сараптамалық комиссия жүргізген
дәмдік бағалау нәтижелері бойынша екі өнім де "өте жақсы" санатына ие
болды, сонымен қатар ең көп мақұлдауға зімбір мен сарымсақ қосылған
тұздық ие болды.

Осылайша, эмульсиялық өнімдерді әзірлеу кезінде мақсары майын пайдалану
жоғары биологиялық құндылығы, тұрақты сапалық сипаттамалары және жоғары
тұтынушылық тартымдылығы бар функционалды тағам өнімдерін жасауға
мүмкіндік береді, бұл оларды өнеркәсіптік енгізудің орындылығын
растайды.

{\bfseries Түйін сөздер:} мақсары майы, функционалды тамақ өнімдері,
майонез, тұздық, эмульсия жүйелері, физика-химиялық көрсеткіштер,
реологиялық қасиеттер, органолептикалық бағалау

\begin{header}
ПЕРСПЕКТИВЫ ПРИМЕНЕНИЯ САФЛОРОВОГО МАСЛА ПРИ СОЗДАНИИ ПИЩЕВЫХ ЭМУЛЬСИОННЫХ ПРОДУКТОВ

\tsp{1}М.Ж. Султанова\envelope,
\tsp{1}Ф.З. Сеитова,
\tsp{2}Н. Акжанов
\end{header}

\begin{affil}
\tsp{1}Алматинский технологический университет, Алматы, Казахстан,

\tsp{2}АФ ТОО «Казахский НИИ перерабатывающей и пищевой промышленности», Астана, Казахстан,

e-mail: sultanova.2012@mail.ru
\end{affil}

Разработка функциональных пищевых продуктов на основе растительных масел
является одним из перспективных направлений современной пищевой науки и
технологии. В данной работе сафлоровое масло (Carthamus tinctorius L.),
отличающееся высоким содержанием полиненасыщенных жирных кислот,
токоферолов и антиоксидантных соединений, использовано в качестве
основной жировой базы при создании двух эмульсионных продуктов:
высококалорийного майонеза и соуса с чесноком и имбирём.

Опытные партии были изготовлены в лабораторных условиях с применением
стандартного сме\-сительно-гомогенизирующего оборудования. Рецептуры
формировались с учётом оптимального соотношения жировых, белковых и
углеводных компонентов, а также включали натуральные добавки с
функциональными свойствами (чеснок, имбирь, аскорбиновая кислота), что
позволило повысить пищевую и биологическую ценность продуктов. Для
каждого варианта была разработана технологическая схема, обеспечивающая
стабильность эмульсий и увеличение сроков хранения.

Проведена оценка физико-химических, реологических и органолептических
характеристик. Реологические исследования выявили различия в структурном
поведении продуктов: майонез демонстрировал псевдопластический тип
течения с высокой вязкостью (2000--5000 мПа·с) и выраженной
тиксотропией, в то время как соус характеризовался вязко-пластическим
течением и умеренной вязкостью (800-1500 мПа·с). Физико-химический
анализ показал допустимые значения кислотного и перекисного чисел,
подтверждающие стабильность жировой фазы. По результатам дегустационной
оценки, проведённой экспертной комиссией из 20 специалистов, оба
продукта получили категорию «отлично», при этом наибольшее одобрение
получил соус с имбирём и чесноком.

Таким образом, использование сафлорова масла при разработке эмульсионных
продуктов позволяет создавать функциональные продукты питания с высокой
биологической ценностью, стабильными качественными характеристиками и
высокой потребительской привлекательностью, что подтверждает
целесообразность их промышленного внедрения.

{\bfseries Ключевые слова:} сафлоровое масло, функциональные продукты
питания, майонез, соус, эмульсионные системы, физико-химические
показатели, реологические свойства, органолептическая оценка.

\begin{header}
PROSPECTS FOR THE USE OF SAFFLOWER OIL IN THE CREATION OF FOOD EMULSION PRODUCTS

\tsp{1}M.Zh. Sultanova\envelope,
\tsp{1}F.Z. Seitova,
\tsp{2}N. Akzhanov
\end{header}

\begin{affil}
\tsp{1}Almaty Technological University, Almaty, Kazakhstan,

\tsp{2}«Kazakh research Institute of processing and food industry» LLP AF, Astana, Kazakhstan,

e-mail: sultanova.2012@mail.ru
\end{affil}

The development of functional food products based on vegetable oils is
one of the promising areas of modern food science and technology. In
this work, safflower oil (Carthamus tinctorius L.), characterized by a
high content of polyunsaturated fatty acids, tocopherols and antioxidant
compounds, was used as the main fat base in the creation of two emulsion
products: high-calorie mayonnaise and sauce with garlic and ginger.

The experimental batches were manufactured in the laboratory using
standard mixing and homogenizing equipment. The formulations were formed
taking into account the optimal ratio of fat, protein and carbo\-hydrate
components, and also included natural additives with functional
properties (garlic, ginger, ascorbic acid), which increased the
nutritional and biological value of the products. A technological scheme
has been developed for each variant to ensure the stability of emulsions
and increase shelf life.

The physico-chemical, rheological, and organoleptic characteristics were
evaluated. Rheological studies revealed differences in the structural
behavior of the products: mayonnaise demonstrated a pseudoplastic flow
type with high viscosity (2000-5000 MPa•s) and pronounced thixotropy,
while the sauce was chara\-cterized by a viscoplastic flow and moderate
viscosity (800-1500 MPa•s). Physico-chemical analysis showed acceptable
values of acid and peroxide numbers, confirming the stability of the fat
phase. Accord\-ing to the results of the tasting evaluation conducted by
an expert commission of 20 specialists, both products received the
"excellent" category, while the sauce with ginger and garlic received
the most approval.

Thus, the use of safflower oil in the development of emulsion products
makes it possible to create functional food products with high
biological value, stable quality characteristics and high consumer
attractiveness, which confirms the feasibility of their industrial
implementation.

{\bfseries Keywords:} safflower oil, functional food products, mayonnaise,
sauce, emulsion systems, physico-chemical parameters, rheological
properties, organoleptic assessment.

\begin{multicols}{2}
{\bfseries Кіріспе.} Біріккен Ұлттар Ұйымының функционалдық тамақтануына,
салауатты өмір салтына және орнықты даму мақсаттарын (ТДМ) іске асыруға
қызығушылықтың артуы жағдайында жергілікті өсімдік шикізаты негізінде
биологиялық құндылығы жоғары өнімдерді әзірлеу тағам өнеркәсібін ғылыми
зерттеу мен инновациялық дамытудың басым бағыттарының біріне айналуда.
Қазіргі уақытта маңызды ғылыми және практикалық мәселе айқын
функционалдық қасиеттері, жоғары биожетімділігі және физиологиялық
белсенділігі бар отандық шикізатты іздеу және пайдалану болып табылады,
бұл импортқа тәуелділікті төмендетуге және елдің азық-түлік
қауіпсіздігін қамтамасыз етуге мүмкіндік береді. Солардың ішінде мақсары
майы (\emph{Carthamus tinctorius L}.) тағам өнеркәсібі мен
нутрицевтикалық өнімдер үшін перспективалы шикізат ретінде ерекше назар
аудартады.

Мақсары майының май қышқылдық құрамы мен токоферолдар профилі оның
тағамдық және биологиялық құндылығын анықтайтын негізгі факторлар болып
табылады. Matthäus және әріптестері (2015) жүргізген зерттеулерде
мақсары майының құрамында линол (C18:2) және олеин (C18:1) қышқылдарының
айтарлықтай үлесі анықталған, бұл оны жүрек-қан тамырлары ауруларының
алдын алу тұрғысынан құнды май көздерінің қатарына қосады {[}1{]}.2025
жылы Kurt және т.б.87 генотип бойынша жүргізген салыстырмалы
зерттеуінде май үлесінің 18,4--36,9\% және линол қышқылының үлесінің
37,7--77,7\% аралығында өзгеретінін көрсетті {[}2{]}. Бұл деректер
генетикалық әртүрліліктің жоғары екенін және селекциялық бағдарламаларда
маңызды рөл атқаратынын дәлелдейді.

Мақсары майының құрамдық және функционалдық сипаттамаларын жүйелі түрде
талдаған Khalid және т.б. (2017) еңбегінде оның май қышқылдарының ерекше
профилі, токоферолдар мен фитостеролдардың болуы оны тағамдық және
нутрицевтикалық өнімдерге арналған биоактивті ингредиент ретінде
қолданудың үлкен әлеуетін көрсетеді {[}3{]}. Jaradat және әріптестері
(2024) мақсары майының фитохимиялық құрамын және антидиабеттік, майға
қарсы, антиоксиданттық және цитоуыттылық қасиеттерін жан-жақты зерттеп,
оның белсенді биологиялық әсерлерін көрсетті {[}4{]}.

Мақсары майының сапасы мен фармакопеялық сәйкестігіне қатысты зерттеулер
(Orhan et al., 2022) оның салмақты бақылау әлеуетін және еуропалық
стандарттарға сәйкестігін растады {[}5{]}. Әртүрлі экстракция әдістерін
салыстырған зерттеуде (Hou et al., 2024) суық престеу, ыстық престеу,
Сокслет және субкритикалық сұйықтық әдістері бойынша алынған майлардың
белсенді қосылыстары мен сапалық көрсеткіштерінің айтарлықтай
айырмашылықтары анықталған {[}6{]}.

Сондай-ақ Liu және әріптестері (2016) мақсары майының май қышқылдық
құрамының генетикалық ерекшеліктерін қарастырып, оның май сапасы мен
бағытталған селекциядағы маңыздылығын атап өтті {[}7{]}. Sajid және т.б.
(2024) тікенді және тікенсіз генотиптерді салыстырмалы бағалау арқылы
май шығымы мен май қышқылдық құрамы бойынша айырмашылықтарды көрсетті
{[}8{]}.

Мақсары майының антиоксиданттық қасиеттері де кеңінен зерттелуде.
Bacchetti және т.б. (2020) антиоксиданттық және прооксиданттық
белсенділігін зерттеді {[}9{]}, ал Mandade және т.б. (2011) жүргізген
тәжірибелер радикалдарды залалсыздандыру белсенділігінің жоғары екенін
дәлелдеді {[}10{]}.

Tultabaev M. және әріптестерінің зерттеуі бойынша, мақсары майының
құрамы мен өнімділігі температура, ылғалдылық пен күн ұзақтығына тәуелді
болып, өсудің, гүлденудің және пісіп-жетілу кезеңінің барысына әсер
етеді {[}11{]}. Айбульдинов Е. К. және әріптестерінің мәліметтері
бойынша, «салқын» сығымдау әдісі майдың табиғи пайдалы қасиеттерін
сақтап шығару мүмкіндігін береді, ал шикізаттың майлылығы мен
дисперсиясы пресстің жұмысын анықтайды {[}12{]}. Mussynov K. M. және
әріптестерінің зерттеулерінде Minimal I топырақ өңдеу технологиясы
сафлордың өнімділігін арттырып, Center 70 сұрыпы ең жоғары экономикалық
тиімділікті көрсетті {[}13{]}.

Жоғарыда аталған зерттеулер мақсары майының тағамдық, нутрицевтикалық
және биологиялық құндылығын кешенді түрде сипаттайды, мұнда зерттеу
жаңалығы оның функционалды ингредиенттерді әзірлеу, жергілікті шикізатты
тиімді пайдалану және тұрақты өндіріс жүйелерін қалыптастыру салаларында
ғылыми және өндірістік маңызын айқындауда көрініс табады.

{\bfseries Материалдар мен әдістер.} Әзірленген өнімдердің тәжірибелік
партиясы зертханалық жағдайда стандартты араластыру-гомогенизациялау
жабдығы пайдаланылды. Зерттеу нысандары ретінде мақсары майы негізінде
жасалған майонез бен тұздық алынды.

Рецепттерді қалыптастыру және қоспаларды таңдау соңғы өнімнің
функционалдық қасиеттерін қамтамасыз етуді ескере отырып жүргізілді.
Гомогенизация процесі эмульсиялардың құрылымдық-текстуралық
сипаттамаларын жақсарту, май фазасының біркелкі дисперсиясын қамтамасыз
ету және өнімдердің тұрақтылығын арттыру үшін қолданылды. Гомогенизация
зертханалық гомогенизаторда \textasciitilde9000 айн/мин айналу
жылдамдығында, 5-7 минут ішінде 25 ± 2 °С температурада жүргізілді.

\emph{{\bfseries Зерттеу әдістері.}}

Зерттеу мақсаттарына жету үшін техникалық құжаттамамен реттелетін
стандартты әдістер кешені қолданылды:

Майдың қышқылдық және пероксидтік санын анықтау-МЕМСТ 31762-2012 " Тағам
өнімдері. Майдың қышқылдық және асқын тотығын анықтау әдістері".

Йод санын анықтау- МЕМСТ ISO 3961-2020 "Жануарлар мен өсімдік тектес
майлар мен майлар. Йод санын анықтау ".

Титрленетін қышқылдық- МЕМСТ 25555.0-82 "Тағам өнімдері. Титрленетін
қышқылдықты анықтау әдістері".

Органолептикалық көрсеткіштер (дәмі, иісі, консистенциясы, түсі) - 5
балдық шкаланы қолдана отырып сенсорлық талдау әдісі бойынша (МЕМСТ
31986-2012).

pН анықтау --- pН-метрді қолдана отырып потенциометриялық әдіспен.

\emph{{\bfseries Статистикалық өңдеу}}

Барлық өлшеулер үш рет қайталанды. Нәтижелер орташа мәндер және
стандартты ауытқулар түрінде ұсынылған. Бақылау мен сынақ үлгісі
арасындағы айырмашылықтардың дұрыстығы p <{} 0,05 маңыздылық
деңгейінде бір факторлы дисперсиялық талдау әдісімен бағаланды. Зерттеу
мәндері - дисперсиялық талдау (ANOVA) негізінде жүргізілді.

{\bfseries Нәтижелер және талқылау.} Жұмыс барысында мақсары майына
негізінде тағам өнімдерінің екі түрі әзірленді және технологиялық
негізделді: жоғары калориялы майонез және зімбір мен сарымсақ қосылған
тұздық. Рецептуралар эмульсиялардың тұрақтылығын, органолептикалық
сипаттамаларын сақтауды және биологиялық құндылығын арттыруды қамтамасыз
ететін компоненттердің оңтайлы арақатынасын ескере отырып құрастырылды.

Мақсары майына негізделген майонез бен тұздықты өндірудің технологиялық
процесі іс жүзінде бір-бірінде ерімейтін компоненттерден (мысалы, су мен
май) біртекті және тұрақты жүйені алуға мүмкіндік беретін оңтайлы
жағдайлар жасауды көздейді. Жоғарыда аталған өнімдерді әзірлеу кезінде
құрғақ компоненттердің концентрациясы, құрғақ компоненттердің ісіну және
пастерлеу шарттары, майдың берілу жылдамдығы, механикалық әсердің
қарқындылығы сияқты факторларды ескеру қажет.

Мақсары майына негізделген функционалды жоғары калориялы майонезді
әзірлеу үшін эмульсиялық өнімдерге қойылатын талаптарды ескере отырып
рецептура жасалды: май фазасының тұрақтылығын қамтамасыз ету,
органолептикалық сипаттамаларды сақтау және биологиялық құндылығын
арттыру. Тұрақты эмульсияны қалыптастыруға және сақтау мерзімін ұзартуға
ықпал ететін өсімдік майының, ақуыз-көмірсу компоненттерінің және
тағамдық қоспалардың арақатынасын таңдауға ерекше назар аударылды.
1-кестеде жоғары калориялы майонездің рецептурасы келтірілген.
\end{multicols}

\tcap{1 - кесте. Жоғары калориялы майонездің рецептурасы}
\begin{longtblr}[
  label = none,
  entry = none,
]{
  cells = {c},
  cells = {font = \small},
  hlines,
  vlines,
}
№  & \textbf{Шикізат атауы}       & \textbf{Компоненттердің массалық үлесі, \%} \\
1. & Рафиринирленген мақсары майы & 50,00                                       \\
2. & Соя талшығы                  & 2,00                                        \\
3. & Су                           & 30,75                                       \\
4. & Құрғақ майсыз сүт            & 6,50                                        \\
5. & Қыша ұнтағы                  & 0,50-0,75                                   \\
6. & Натрий бикарбонаты           & 0,05                                        \\
7. & Қант (құм)                   & 1,50                                        \\
8. & Ас тұзы                      & 1,10                                        \\
9. & Аскорбин қышқылы 10\%        & 0,45                                        
\end{longtblr}

\tcap{2 - кесте. Төмен калориялы тұздықтың рецептурасы}
\begin{longtblr}[
  label = none,
  entry = none,
]{
  cells = {c},
  cells = {font = \small},
  hlines,
  vlines,
}
№  & \textbf{Шикізат атауы}       & \textbf{Компоненттердің массалық үлесі, \%} \\
1. & Рафиринирленген мақсары майы & 15,00                                       \\
2. & Зімбір ұнтағы                & 2,00                                        \\
3. & Сүт                          & 30,75                                       \\
4. & Су                           & 26,50                                       \\
5. & Бидай ұны (1 сорт)           & 15,00                                       \\
6. & Натрий бикарбонаты           & 0,05                                        \\
7. & Қант (құм)                   & 1,50                                        \\
8. & Ас тұзы                      & 1,10                                        \\
9. & Сарымсақ                     & 0,45                                        
\end{longtblr}

\begin{multicols}{2}
Кестеден көріп отырғандай, майонездің негізі ретінде май фазасы мен
функционалды компоненттердің (линол қышқылы, токоферолдар) негізгі көзі
болып табылатын мақсары майы (50\%) пайдаланылды. Эмульсияның
тұрақтылығын арттыру мақсатында рецептураға ақуыз және көмірсу
компоненттері енгізілді: соя талшығы (2,0\%) және майсыз сүт ұнтағы
(6,5\%). Қыша ұнтағы (0,50-0,75\%) өнімнің тұтқырлығы мен
консистенциясын жақсартатын табиғи эмульгатор ретінде қызмет етеді.
Өнімнің тұтқырлығы 120-1500 мПа·с аралығында, ал эмульсияның тұрақтылық
индексі 85-90\% деңгейінде бағаланды. Минералды және дәмдік қоспалар
(тұз, қант, сода) оңтайлы органолептикалық көрсеткіштерді қамтамасыз
етеді, ал аскорбин қышқылы май фазасындағы тотығу процестерін тежейтін
антиоксидант ретінде әрекет етеді; антиоксиданттық белсенділік DPPH
әдісі бойынша 65-70\% деңгейінде анықталды. Осылайша, ұсынылған
рецептура жоғары биологиялық құндылықты сақтай отырып, функционалды
тағамға қойылатын талаптарға сәйкес келеді.

Мақсары майын қолданатын тұздық рецептурасы майы аз өнімді әзірлеу
мақсатында қалыптастырылып, ащы дәмі мен айқын функционалды қасиеттері
бар өнім алуға бағытталған. Рецептурада тұздықтың тұтқырлығы мен
тұрақтылығына қойылатын талаптар, сондай-ақ антиоксиданттық (DPPH
60-65\%) және микробқа қарсы белсенділігі бар өсімдік қоспаларын
(зімбір, сарымсақ) енгізу мүмкіндігі ескерілді. Эмульсияның тұрақтылық
индексі 80--85\% деңгейінде бағаланды.

2-кестеден көрініп тұрғандай, рецептураның негізін сүтті-астық базасы
құрайды (сүт - 30,75\% және бидай ұны-15\%), ол қажетті
тұтқыр-пластикалық құрылымды жұмсақ дәмді қамтамасыз етеді. Май фазасы
полиқанықпаған май қышқылдары мен токоферолдардың көзі болып табылатын
мақсары майымен (15\%) ұсынылған. Ащы қоспаларды енгізу (зімбір - 2\%
және сарымсақ - 0,45\%) өнімге функционалды қасиеттер береді:
антиоксидантты, иммуностимуляторлы және хош иісті. Дәм мен
консистенцияны реттегіштер (тұз, қант, сода) сақтау кезінде өнімнің
үйлесімді дәмдік тепе-теңдігі мен тұрақтылығына қол жеткізуге мүмкіндік
береді.

Эмульсияның тұрақтылығын қамтамасыз ету және жоғары органолептикалық
сипаттамаларды сақтау үшін мақсары майы қосылған майонез өндірісінің
технологиялық процесі жасалды. Сызбанұсқада негізгі кезеңдерді көрсетеді
- шикізатты дайындаудан бастап дайын өнімді буып-түюге дейін, соның
ішінде майонез эмульсиясының біртектілігі мен тұрақтылығын қамтамасыз
ететін эмульсия және гомогенизация процестері.1 суретте мақсары майы
қосылған майонез өндірісінің сызбанұсқасы келтірілген.
\end{multicols}

{\bfseries Шикізатты дайындау}

Мақсары майы, су, соя талшығы, майсыз сүт ұнтағы, қыша ұнтағы, қант,
тұз, сода және аскорбин қышқылы дозаланады

{\bfseries Су фазасын дайындау}

Тұз, қант, сода және аскорбин қышқылын суда ерітеді. Құрғақ сүт, соя
талшығы және қыша ұнтағы қосылады. Ақуыздарды белсендіру және ісіну үшін
қоспаны 45-50 °C дейін қыздырады

{\bfseries Эмульгирлеу}

Мақсары майы қарқынды араластыру кезінде сулы фазаға біртіндеп
енгізіледі

{\bfseries Гомогенизация және pH реттеу}

Гомогенизатор қолданылады (\textasciitilde9000 айн/мин). рН тексеріледі
(оңтайлы 4.0-4.2), қажет болған жағдайда сірке суын немесе лимон шырынын
қосады.

{\bfseries 0+ + 6 °C температурада орау және сақтау}

Стерильді ыдысқа орап, 0...+6 °C температурада 60 күнге дейін сақтайды

{\bfseries 1 - сурет. Мақсары майы негізінде майонез өндіру технологиясы}

Процесс мақсары майы, су, соя талшығы, құрғақ сүт және дәмдеуіштерді
қоса алғанда, ингредиенттерді дайындаудан басталады. Содан кейін су
фазасы дайындалады: құрғақ компоненттер суда ериді және 45-50 °C дейін
қызады. Осыдан кейін гомогенизатор көмегімен майды біртіндеп эмульсиялау
жүзеге асырылады. Дайын эмульсия қышқылдығы мен құрылымы бойынша
түзетіледі, стерильді ыдысқа салынып, 0...+6 °C температурада 60
тәулікке дейін сақталады (2-сурет).

\fig[0.5\textwidth]{p2/image52}[2 - сурет. Мақсары майы негізінде жасалған майонездің сынақ үлгілері]

Мақсары майы негізінде зімбір тұздығын алу үшін тұрақты құрылым мен
айқын органолептикалық қасиеттердің қалыптасуын қамтамасыз ететін
технологиялық процесс жасалды. Сызбанұсқада өндірістің негізгі
кезеңдерін көрсетеді --- ингредиенттерді дайындаудан және ұнды
қопсытудан бастап дәмдеуіштерді енгізуге және дәмді түзетуге дейін, бұл
өнімге тән ащы хош иісі мен функционалды құндылығы бар өнімді алуға
мүмкіндік береді (3-сурет).

{\bfseries Ингредиенттерді дайындау\\
}Мақсары майы, сүт, су, бидай ұны, сарымсақ, зімбір ұнтағы, қант, тұз
және сода дозаланады.

{\bfseries Ұнды майға қуыру\\
}Сотейникке мақсары майын құйып, бидай ұнын ашық кілегей түске дейін
қуырады. қуырады қуырады..

{\bfseries Сұйықтықты қосу\\
}Сүт пен су енгізіліп, тегіс құрылым алу үшін мұқият араластырылады.

{\bfseries Дәмдеуіштер мен хош иісті заттарды енгізу\\
}Ұсақталған сарымсақ, зімбір ұнтағы, қант, тұз және сода қосылады.
Қоюланғанға дейін қайнатылады.

{\bfseries Дәмді түзету\\
}Қажет болса, қоюлығы (су немесе крахмал қосу арқылы) және қышқылдығы
(лимон шырынымен) реттеледі.

{\bfseries Буып-түю және салқындату\\
}Ыстықтай орап, тез салқындатып және тоңазытқышта (0...+6 °C) 10-15
күннен асырмай сақтайды.

{\bfseries 3 - сурет. Мақсары майы негізінде зімбір тұздығын өндіру
технологиясы}

Процесс ингредиенттерді дайындаудан және бидай ұнын мақсары майында ашық
кремді түске дейін қуырудан басталады. Содан кейін сүт пен су енгізіліп,
қоспасы мұқият араластырылады. Осыдан кейін сарымсақ, зімбір, тұз, қант
және сода қосылады, қоюланғанға дейін қайнатылады. Соңында дәмі мен
консистенциясы реттеледі, өнім ыстықтай оралады, салқындатылады және
0. ..+6 °C температурада 15 күнге дейін сақталады.

Мақсары майы негізінде майонез мен тұздықты өндіру процестерінің
рецептуралары мен технологиялық негіздемесін әзірлегеннен кейін келесі
қадам олардың физика-химиялық және құрылымдық-механикалық қасиеттерін
бағалау болды. Реологиялық көрсеткіштерге ерекше назар аударылды,
өйткені олар консистенцияны, құрылымды, эмульсияның тұрақтылығын және
аспаздық тәжірибеде өнімнің ыңғайлылығын анықтайды. Зерттеу нәтижелері
3-кестеде келтірілген.

\tcap{3 - кесте. Мақсары майы негізінде майонез мен тұздықтың реологиялық сипаттамалары}
\begin{longtblr}[
  label = none,
  entry = none,
]{
  width = \linewidth,
  colspec = {Q[310]Q[285]Q[346]},
  cells = {c},
  cells = {font = \small},
  hlines,
  vlines,
}
\textbf{Көрсеткіш}                     & \textbf{Майонез (жоғары калориялы)} & \textbf{Зімбір қосылған тұздық(төмен калориялы)} \\
Ағын түрі                              & Псевдопластикалық                   & Тұтқыр-пластикалық                               \\
Тұтқырлық (мПа•с, 20 °C температурада) & 2000–5000                           & 800–1500                                         \\
Жылжу жағдайындағы сипаты              & Ығысу тұтқырлығының төмендеуі       & Өтімділік шегімен                                \\
Тиксотропия                            & Айқын                               & Әлсіз айқындалған                                \\
Құрылымның ерекшеліктері               & Қою, пластикалық, тұрақты эмульсия  & Тегіс, орташа сұйық масса                        
\end{longtblr}

3-кестеден майонездің тұтқырлықтың жоғары деңгейімен (2000-5000 мПа•с)
және айқын тиксотропиямен сипатталатынын көруге болады, бұл оның тұрақты
қою консистенциясы мен қолдануға ыңғайлылығын қамтамасыз етеді. Зімбір
тұздығы тұтқыр пластикалық сипатпен және ұнды құрылымдаушы ретінде
пайдалану арқылы пайда болатын орташа сұйықтықпен (800-1500 мПа•с)
ерекшеленеді.

Осылайша, ұсынылған реологиялық көрсеткіштер мақсары майымен байытылған
эмульсиялық өнімдердің функционалдығын, тұрақтылығын және жоғары сапасын
растайды. Мақсары майына негізінде әзірленген функционалдық өнімдердің
сапасы мен тұрақтылығын бағалау үшін негізгі физика-химиялық
көрсеткіштер зерттелді: майдың қышқылдық және пероксидтік саны, йод
саны, сондай-ақ титрленетін қышқылдық. Нәтижелер 4-кестеде келтірілген.

\tcap{4 - кесте. Мақсары майы негізінде әзірленген функционалдық тағам өнімдерінің физика-химиялық құрамы}
\begin{longtblr}[
  label = none,
  entry = none,
]{
  width = \linewidth,
  colspec = {Q[412]Q[129]Q[138]Q[117]Q[138]},
  cells = {c},
  cells = {font = \small},
  cell{1}{1} = {r=2}{},
  cell{1}{2} = {c=2}{},
  cell{1}{4} = {c=2}{},
  vlines,
  hline{1,3-7} = {-}{},
  hline{2} = {2-5}{},
}
\textbf{Көрсеткіштердің атауы, өлшем бірліктері} & \textbf{Майонез} &                       & \textbf{Тұздық}  &                       \\
                                                 & \textbf{бақылау} & \textbf{сынақ үлгісі} & \textbf{бақылау} & \textbf{сынақ үлгісі} \\
Майдың қышқыл саны,мг КОН / кг                   & 0,26±0,002       & 0,35±0,27             & 0,44±0,004       & 0,31±0,002            \\
Майдың асқын тотығы, моль / кг                   & 2,97±0,02        & 3,81±0,03             & 2,01±0,01        & 2,15±0,01             \\
Йод саны, г / I2 / 100г                          & 103,09±1,21      & 95,41±1,01            & 88,75±0,92       & 98,06±0,86            \\
Титрленетін қышқылдық, °Т                        & 1,08±0,02        & 0,84±0,01             & 1,47±0,02        & 1,14±0,01             
\end{longtblr}

4-кестеден майонездегі майдың қышқылдық мөлшері 0,26-дан 0,35 мг
КОН/кг-ға дейін өскенін көруге болады, бұл гидролизге бейім мақсары
майын қолданумен байланысты болуы мүмкін. Тұздықта, керісінше, сынақ
үлгідегі қышқыл санының төмендеуі байқалады (0,44-тен 0,31 мг КОН/кг-ға
дейін), бұл мақсары майын енгізу кезінде май фазасының тұрақтылығын
көрсетеді. Пероксид саны майлардың бастапқы тотығу дәрежесін көрсетеді.
Жоғары мәндер тотығу процестерінің басталуын көрсетуі мүмкін: майонезде
пероксид саны сынақ үлгісінде 2,97-ден 3,81 моль/кг-ға дейін өсті, бұл
мақсары майын қосқанда май фазасының тотығуға сезімталдығын көрсетеді.
Зімбір тұздығында пероксидтің шамалы өсуі байқалады (2,01-ден 2,15
моль/кг-ға дейін), бұл қолайлы өзгеріс деп санауға болады. Йод саны май
қышқылдарының қанықпау дәрежесін көрсетеді. Жоғары мәндер полиқанықпаған
қышқылдарға бай майларға тән: майонездің сынақ үлгісінде йод санының
төмендеуі байқалады (103,09-дан 95,41 г I₂/100 г дейін), бұл қанықпаған
май қышқылдарының тотығу процестерінің басталуымен байланысты болуы
мүмкін. Тұздықта, керісінше, йод саны 88,75-тен 98,06-ға дейін артады,
бұл линол қышқылы жоғары мақсары майының болуын көрсетеді. Титрленетін
қышқылдық өнімнің Сулы фазасындағы қышқылдардың жалпы құрамын көрсетеді:
майонезде қышқылдық 1,08-ден 0,84 °Т-қа дейін төмендейді, бұл
рецептураны өзгерту арқылы қышқылдықтың төмендеуін көрсетуі мүмкін.
Сондай-ақ, тұздықта өнімнің жұмсақ дәмі мен тұрақтылығының жоғарылауына
ықпал ететін (1,47-ден 1,14 °T-қа дейін) төмендеу байқалады.

Әзірленген функционалдық өнімдердің - мақсары майы негізіндегі майонез
бен тұздықтың аспаздық және тұтынушылық артықшылықтарын айқындау үшін
"Қазақ қайта өңдеу және тамақ өнеркәсібі ғылыми-зерттеу институты" ЖШС
Астана филиалының зертханасы (ҚАЗҚӨТӨҒЗИ ЖШС) жағдайында сараптамалық
дәм тату өткізілді.

Дәм татуға тиісті құзыреттіліктері мен сенсорлық сезімталдығы бар
институттың 20 қызметкері қатысты. Әрбір үлгі төрт критерий бойынша
бағаланды: иіс, дәм, консистенция және түс. Әрбір белгі бойынша 1-ден 5
балға дейінгі диапазонда баға қойылды, бұл ретте сол немесе өзге бағаны
берген сарапшылардың саны ескерілді (5 және 6-кесте).

\tcap{5 - кесте. Жоғары калориялы майонездің дәмін тату нәтижелері (мақсары майы негізінде)}
\begin{longtblr}[
  label = none,
  entry = none,
]{
  cells = {c},
  cells = {font = \small},
  cell{6}{1} = {c=2}{},
  hlines,
  vlines,
}
№        & \textbf{Белгі} & \textbf{Нәтиженің сипаттамасы}  & \textbf{Ұпайлар} \\
1        & Иісі           & Майонезге тән, орташа айқын     & 5×8              \\
2        & Дәмі           & Үйлесімді, жағымды дәмі бар     & 5×7              \\
3        & Консистенциясы & Біртекті, қою, эмульсияланған   & 4×4              \\
4        & Түсі           & Ашық сары, біркелкі, өнімге тән & 4×3              \\
Барлығы: &                &                                 & 90               
\end{longtblr}

\tcap{6 - кесте. Төмен калориялы тұздықтың дәмін тату нәтижелері (мақсары майы негізінде)}
\begin{longtblr}[
  label = none,
  entry = none,
]{
  cells = {c},
  cells = {font = \small},
  cell{6}{1} = {c=2}{},
  hlines,
  vlines,
}
№        & \textbf{Белгі} & \textbf{Нәтиженің сипаттамасы}                         & \textbf{Ұпайлар} \\
1        & Иісі           & Сарымсақ пен зімбір ноталары бар ащы, жақсы анықталған & 5×10             \\
2        & Дәмі           & Жарқын, өткір-ащы, сәл тәтті дәмі бар                  & 5×7              \\
3        & Консистенциясы & Орташа тұтқыр, біркелкі                                & 4×3              \\
4        & Түсі           & Кремді реңктері бар ашық қоңыр, біркелкі               & 4×2              \\
Барлығы: &                &                                                        & 93               
\end{longtblr}

\begin{multicols}{2}
Дәм тату нәтижелері барлық әзірленген өнімдердің жоғары тұтынушылық
тартымдылығын көрсетті. Жоғары калориялы майонез "өте жақсы" санатына
жатқызылды, 90 ұпай жинады, бұл органолептикалық қасиеттердің
үйлесімділігін көрсетеді, бірақ нақтылау мүмкіндігі бар, әсіресе
"консистенция"параметрі бойынша.

Ғылыми зерттеулер көрсеткендей, мақсары майынан дайындалған майонез және
соустар дәстүрлі өсімдік майларына, мысалы, күнбағыс немесе жүгері
майларына қарағанда жоғары тағамдық және биологиялық құндылыққа ие;
күнбағыс майы көбінесе транс-май қышқылдарын қамтиды, бұл жүрек-тамыр
жүйесіне зиян келтіруі және метаболизмді бұзуы мүмкін, ал мақсары майы
транс-май қышқылдарынан таза, омега-6 полиқанықпаған май қышқылдары,
антиоксиданттар және биологиялық белсенді қосылыстарға бай болып,
жүрек-тамыр жүйесін қорғауға, жалпы денсаулыққа және функционалды
тағамдық өнімдерде табиғи пайдалы қасиеттерін сақтауға мүмкіндік береді.

Осылайша, органолептикалық бағалау мақсары майына негізделген жаңа
функционалды тағам өнімдерін әзірлеудің сәттілігін растайды және оларды
өнеркәсіптік өндіріске одан әрі енгізуді негіздейді.

"Қазақ қайта өңдеу және тағам өнеркәсібі ғылыми-зерттеу институты"
ЖШС-нің 20 маманы арасында функционалдық тұздықтардың үш үлгісін -
жоғары калориялы майонезді, төмен калориялы тұздықты сенсорлық бағалау
өнімнің тұтынушылық сапасын объективті анықтауға мүмкіндік берді.

Ұпайлар жиынтығы бойынша: Майонез 90 ұпай жинады, бұл "өте жақсы"
санатына сәйкес келеді; Тұздық-93 ұпай ("өте жақсы").

Дәм татушылардың ең көп мақұлдауына төмен калориялы тұздық ие болды, ол
үйлесімді дәмге, біркелкі консистенцияға және сарымсақ пен зімбірдің
теңдестірілген дәміне байланысты. Өнім органолептикалық экспрессивті
және бәсекеге қабілетті деп бағаланады. Дәм татудың жалпы нәтижелері
барлық үш үлгінің жоғары органолептикалық қолайлылығын көрсетеді, бұл
оларды тұтынушылардың заманауи талаптарына сәйкес келетін функционалды
тұздықтар ретінде нарыққа шығарудың орындылығын растайды.

{\bfseries Қорытынды.} Зерттеулер нәтижесінде мақсары майына негізделген
функционалды тағамның екі түрі - жоғары калориялы майонез және зімбір
мен сарымсақ тұздығы әзірленді және технологиялық негізделген. Рецептура
компоненттерін таңдау және оңтайлы технологиялық режимдерді қолдану
тұрақтылық пен органолептикалық тартымдылықтың жоғары көрсеткіштері бар
тұрақты эмульсиялық жүйелерді қалыптастыруға мүмкіндік берді.

Реологиялық зерттеулер өнімнің құрылымдық-механикалық сипатындағы
айырмашылықтарды көрсетті: майонез жоғары тұтқырлықпен, ағынның
псевдопластикалық түрімен және күшті тиксотропиямен сипатталды, бұл
тығыз консистенция мен қолданудың қарапайымдылығын қамтамасыз етті;
тұздықта тұтқыр пластикалық ағын және орташа сұйықтық болды, бұл
теңдестірілген құрылым мен жұмсақ дәмді қалыптастырады. Физика-химиялық
талдау қышқыл мен пероксидтің рұқсат етілген мәндерін, сондай-ақ май
қышқылдарының қанықпауының жоғары дәрежесін растады, бұл өнімдердің
биологиялық құндылығын және антиоксидантты тұрақтандыру қажеттілігін
көрсетеді.

Органолептикалық бағалау екі үлгінің де жоғары тұтынушылық тартымдылығын
көрсетті: майонез 90 ұпай ("өте жақсы"), ал тұздық 93 ұпай ("өте жақсы")
алды, бұл үйлесімді дәмді, теңдестірілген консистенцияны және айқын ащы
ноталарды көрсетеді.

Осылайша, эмульсиялық өнімдер технологиясында мақсары майын қолдану
жақсартылған тағамдық сипаттамалары, жоғары биологиялық құндылығы және
тұрақты сапа көрсеткіштері бар функционалды өнімдерді жасауға мүмкіндік
береді. Нәтижелер әзірленген рецепттерді өнеркәсіптік енгізудің
болашағын және олардың нарықтағы функционалды тұздықтар мен
майонездердің ассортиментін кеңейту әлеуетін растайды.

\emph{{\bfseries Қаржыландыру:} Жұмыс Қазақстан Республикасының Ауыл
шаруашылығы министрлігі BR22886613 «Ауыл шаруашылығы Өсімдік шаруашылығы
өнімдері мен шикізатын қайта өңдеу және сақтау жөніндегі инновациялық
технологияларды әзірлеу» қаржыландыратын бағдарлама шеңберінде
жүргізілді.}

\emph{Қорытындылай келе, біз осы ғылыми жобаның барлық қатысушыларына
эксперименттік зерттеулер жүргізуге көмектескені үшін шын жүректен алғыс
айтқымыз келеді. Біз сондай-ақ «ҚазҒЗИ қайта өңдеу және тамақ
өнеркәсібі» ЖШС Астана филиалының басшылығы мен ғалымдарына алғысымызды
білдіреміз.}
\end{multicols}

\begin{center}
{\bfseries Әдебиеттер}
\end{center}

\begin{refs}
1. Matthaus B., Özcan M. M., Al Juhaimi F. Y. Fatty acid composition and
tocopherol profiles of safflower (Carthamus tinctorius L.) seed oils
//Natural product research. -2015. -Vol.29(2). -P.193-196. DOI
10.1080/14786419.2014.971316.

2. Kurt C. et al. Oil Content and Fatty Acid Composition of Safflower
(Carthamus tinctorius L.) Germplasm //Foods. -2025. -Vol.14(2): 264.
DOI 10.3390/foods14020264.

3. Khalid N., et al. A comprehensive characterisation of safflower oil
for its potential applications as a bioactive food ingredient - a review
// Trends in Food Science \& Technology. -2017. -Vol.66. -P.176-186.
DOI 10.1016/j.tifs.2017.06.009.

4. Jaradat N. et al. Phytochemical composition and antidiabetic,
anti-obesity, antioxidant, and cytotoxic activities of Carthamus
tinctorius seed oil //Scientific Reports. -2024. -Vol.14(1): 31399. DOI
10.1038/s41598-024-83008-z.

5. Orhan D. D. et al. Assessment of commercially safflower oils
(Carthami Oleum raffinatum) in terms of European Pharmacopoeia Criteria
and their weight control potentials //Turkish Journal of Pharmaceutical
Sciences. -2022. -Vol.19(3). -P.273-279. DOI
10.4274/tjps.galenos.2021.84484.

6. Hou N. C. et al. Quality and active constituents of safflower seed
oil: A comparison of cold pressing, hot pressing, Soxhlet extraction and
subcritical fluid extraction //LWT. -2024. -Vol.200:116184. DOI
10.1016/j.lwt.2024.116184.

7. Liu L., Guan L. L., Yang Y. X. A review of fatty acids and genetic
characterization of safflower (Carthamus tinctorius L.) seed oil //World
Journal of Traditional Chinese Medicine. -2016. -Vol.2(2). -P.48-52.
DOI 10.15806/j.issn.2311-8571.2016.0006.

8. Sajid M. et al. Comparative assessment of yield, oil content and
fatty acid composition of spiny and non-spiny safflowers //OCL. -2024.
-Vol.31. DOI 10.1051/ocl/2024029.

9. Bacchetti T. et al.~Antioxidant and pro-oxidant properties of
Carthamus tinctorius, hydroxy safflor yellow A, and safflor yellow A
//Antioxidants.~-2020.~-Vol.9(2): 119. DOI 10.3390/antiox9020119.

10. Mandade R., Sreenivas S. A., Choudhury A. Radical scavenging and
antioxidant activity of Carthamus tinctorius extracts //Free Radicals
and Antioxidants. -2011. -Vol.1 (3). -P.87-93. DOI
10.5530/ax.2011.3.12.

11. Tultabaev M. et al. Identifying patterns in the fatty-acid
composition of safflower depending on agroclimatic conditions
//Eastern-European Journal of Enterprise Technologies. -- 2022. -Vol.2
(11). --P.23-28. DOI
\href{https://doi.org/10.15587/1729-4061.2022.255336}{10.15587/1729-4061.2022.255336}.

12. Айбульдинов Е. К., Колпек А., Туребаева П.Д., Абдиев К.М. Технология
высокоскоростного пиролиза в установках с твердым теплоносителем по
методу галотер // Вестник КазУТБ.-2020. -№2. -С.32-37.

13. Mussynov K. M. et al. Economic and bioenergetic efficiency of
safflower (Carthamus tinctorius L.) Cultivation with different soil
preparation technologies //Herald of science of S. Seifullin Kazakh
agrotechnical university: Multidisciplinary. -2018. -№ 1(96).- С.41-49.
\end{refs}

\begin{center}
{\bfseries References}
\end{center}

\begin{refs}
1. Matthaus B., Özcan M. M., Al Juhaimi F. Y. Fatty acid composition and
tocopherol profiles of safflower (Carthamus tinctorius L.) seed oils
//Natural product research. -2015. -Vol.29 (2). -P.193-196. DOI
10.1080/14786419.2014.971316.

2. Kurt C. et al. Oil Content and Fatty Acid Composition of Safflower
(Carthamus tinctorius L.) Germplasm //Foods. -2025. -Vol.14 (2): 264.
DOI 10.3390/foods14020264.

3. Khalid N., et al. A comprehensive characterisation of safflower oil
for its potential applications as a bioactive food ingredient --- a
review // Trends in Food Science \& Technology. -2017. -Vol.66. -P.
176-186. DOI10.1016/j.tifs.2017.06.009..

4. Jaradat N. et al. Phytochemical composition and antidiabetic,
anti-obesity, antioxidant, and cytotoxic activities of Carthamus
tinctorius seed oil //Scientific Reports. -2024. -Vol.14: 31399. DOI
10.1038/s41598-024-83008-z.

5. Orhan D. D. et al. Assessment of commercially safflower oils
(Carthami Oleum raffinatum) in terms of European Pharmacopoeia Criteria
and their weight control potentials //Turkish Journal of Pharmaceutical
Sciences. -2022. -Vol.19 (3). -P.273-279. DOI
10.4274/tjps.galenos.2021.84484.

6. Hou N. C. et al. Quality and active constituents of safflower seed
oil: A comparison of cold pressing, hot pressing, Soxhlet extraction and
subcritical fluid extraction //LWT. -2024. -Vol.200:116184. DOI
10.1016/j.lwt.2024.116184.

7. Liu L., Guan L. L., Yang Y. X. A review of fatty acids and genetic
characterization of safflower (Carthamus tinctorius L.) seed oil //World
Journal of Traditional Chinese Medicine. -2016. -Vol.2 (2). -P.48-52.
DOI 10.15806/j.issn.2311-8571.2016.0006.

8. Sajid M. et al. Comparative assessment of yield, oil content and
fatty acid composition of spiny and non-spiny safflowers //OCL. -2024.
-Vol.31. DOI 10.1051/ocl/2024029.

9. Bacchetti T. et al.~Antioxidant and pro-oxidant properties of
Carthamus tinctorius, hydroxy safflor yellow A, and safflor yellow A
//Antioxidants.~-2020.~-Vol.9(2): 119. DOI 10.3390/antiox9020119.

10. Mandade R., Sreenivas S. A., Choudhury A. Radical scavenging and
antioxidant activity of Carthamus tinctorius extracts //Free Radicals
and Antioxidants. -2011. -Vol.1(3). -P.87-93. DOI
10.5530/ax.2011.3.12.

11. Tultabaev M. et al. Identifying patterns in the fatty-acid
composition of safflower depending on agroclimatic conditions
//Eastern-European Journal of Enterprise Technologies. -- 2022. -Vol.2
(11). -P.23-28. DOI
\href{https://doi.org/10.15587/1729-4061.2022.255336}{10.15587/1729-4061.2022.255336}.

12. Ajbul' dinov E. K., Kolpek A., Turebaeva P.D., Abdiev
K.M. Tehnologija vysokoskorostnogo piroliza v ustanovkah s tverdym
teplonositelem po metodu galoter // Vestnik KazUTB.-2020. -№2. -S.
32-37. {[}in Russian{]}

13. Mussynov K. M. et al. Economic and bioenergetic efficiency of
safflower (Carthamus tinctorius L.) Cultivation with different soil
preparation technologies //Herald of science of S. Seifullin Kazakh
agrotechnical university: Multidisciplinary. -2018. -№ 1(96).-S.41-49.
{[}in Russian{]}
\end{refs}

\begin{info}
\hspace{1em}\emph{{\bfseries Авторлар туралы мәліметтер}}

Султанова М.- докторант, Алматы технологиялық университеті, Алматы,
Қазақстан, e-mail: sultanova.2012@mail.ru;

Сеитова Ф.З. - ф.ғ. к., қауымдастырылған профессор, Алматы
технологиялық университеті, Алматы, Қазақстан, e-mail:
me.fatme@mail.ru;

Акжанов Н. - жаратылыстану ғылымдары магистрі, «Қазақ қайта өңдеу және
тағам өнеркәсіптері ғылыми-зерттеу инсти-туты» ЖШС Астана филиалы,
Астана, Қазақстан, e-mail: nurtore0308@gmail.com.

\hspace{1em}\emph{{\bfseries Information about the authors}}

Sultanova M. - doctoral student, Almaty Technological University,
Almaty, Kazakhstan, e-mail: sultanova.2012@mail.ru;

Seitova F.Z. - Ph. D., associate professor, Almaty Technological
University, Almaty, Kazakhstan, e-mail: me.fatme@mail.ru;

Akzhanov N. - Master of Natural Sciences, Astana Branch of the Kazakh
Research Institute of Processing and Food Industries LLP, Astana,
Kazakhstan, e-mail: nurtore0308@gmail.com.
\end{info}
