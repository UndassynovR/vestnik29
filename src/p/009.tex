\id{ҒТАМР 65.65.03}{}

\begin{header}
\swa{}{ӨСІМДІК ШИКІЗАТЫ НЕГІЗІНДЕ ЭМУЛЬСИЯЛЫҚ ӨНІМДЕРДІ АЛУ ТЕХНОЛОГИЯСЫН ӘЗІРЛЕУ}

\tsp{1}М.Е. Смагулова,
\tsp{2}Ш.Ж. Жасқайрат\envelope,
\tsp{2}Н.С. Машанова,
\tsp{2}М.Е. Бекболатова
\end{header}

\begin{affil}
\tsp{1}К.Кулажанов атындағы Қазақ технология және бизнес университеті, Астана, Қазақстан,

\tsp{2}С.Сейфуллин атындағы Қазақ агротехникалық зерттеу университеті, Астана, Қазақстан

\corrauthor{Корреспондент-автор: shynarai\_92@mail.ru}
\end{affil}

Бұл мақалада өсімдік шикізаты негізінде эмульсиялық өнімдерді алу
технологиясын әзірлеуде эмульсиялық өнімнің рецептурасы,
органолептикалық және физико-химиялық көрсеткіштері, сонымен қатар
өнімнің тағамдық және энергетикалық құндылығы май қышқылдық құрамының
талдау нәтижелері көрсетілген. Талдау нәтижелері бойынша эмульсиялық
өнімнің органолептикалық және физика-химиялық көрсеткіштері дәстүрлі
тұздықпен салыстыра отырып анықталды, яғни сыртқы түрі қаймақ тәрізді,
консистенциясы біртекті, иіс пен дәм бойынша нәзік дәм, ащы емес,
қышқылдау, ашық сары түске ие. Эмульсиялық өнімдердің физика-химиялық
көрсеткіштері: ылғалдылығы - 51,3\%, pH қышқылдығы - 3,6, эмульсия
тұрақтылығы - 100 \%. Эмульсиялық өнімнің тағамдық құндылығы: май -
45\%, ақуыз - 3\%, көмірсу - 4\%, энергетикалық құндылығы -- 432 ккал.
Эмульсиялық өнімнің май қышқылдық құрамы: линолен кышқылы -- 9,8; линол
қышқылы -- 27,2 көрсетті, яғни ω-3:ω-6 полиқанықпаған май қышқылдарының
қатынасы 1:4-ке тең.

{\bfseries Түйін сөздер}: майонез, физико-химиялық көрсеткіштер,
органолептикалық көрсеткіштер, энергетикалық құндылық, тағамдық
құндылық, май қышқылдық құрам

\begin{header}
ИССЛЕДОВАНИЕ ПОДСОЛНЕЧНОГО И ЛЬНЯНОГО МАСЕЛ ПРИ ПОЛУЧЕНИИ ЭМУЛЬСИОННЫХ ПРОДУКТОВ

\tsp{1}М.Е. Смагулова,
\tsp{2}Ш.Ж. Жасқайрат\envelope,
\tsp{2}Н.С. Машанова,
\tsp{2}М.Е. Бекболатова
\end{header}

\begin{affil}
\tsp{1}Казахский университет технологии и бизнеса им. К.Кулажанова, Астана, Казахстан,

\tsp{2}Казахский агротехнический исследовательский университет им. С.Сейфуллина, Астана, Казахстан,

e-mail: shynarai\_92@mail.ru
\end{affil}

В данной статье представлены рецептура, органолептические и
физико-химические показатели эмульсионного продукта при разработке
технологии получения эмульсионных продуктов на основе растительного
сырья, а также результаты анализа жирнокислотного состава пищевой и
энергетической ценности продукта. По результатам анализа были определены
органолептические и физико-химические показатели эмульсионного продукта
по сравнению с традиционным соусом, т. е. по внешнему виду
кремообразный, по консистенции однородный, по запаху и вкусу тонкий
вкус, не горький, не кисловатый, имеет светло-желтый цвет.
Физико-химические показатели эмульсионных продуктов: влажность-51,3\%,
кислотность pH - 3,6, стабильность эмульсии - 100\%. Пищевая ценность
эмульсионного продукта: жиры-45\%, белки - 3\%, Углеводы - 4\%,
Энергетическая ценность - 432 ккал. Жирнокислотный состав эмульсионного
продукта:линоленовая кислота - 9,8; линолевая кислота - 27,2, т. е.
соотношение ω-3:ω-6 полиненасыщенных жирных кислот равно 1: 4.

{\bfseries Ключевые слова}: майонез, физико-химические показатели,
органолептические показатели, энергетическая ценность, пищевая ценность,
жирнокислотный состав

\begin{header}
INVESTIGATION OF SUNFLOWER AND LINSEED OILS IN THE PRODUCTION OF EMULSION PRODUCTS

\tsp{1}M.E. Smagulova,
\tsp{2}Sh.Zh. Zhaskairat\envelope,
\tsp{2}N.S. Mashanova,
\tsp{2}M.E. Bekbolatova
\end{header}

\begin{affil}
\tsp{1}K. Kulazhanov\tsp{1}Kazakh University of Technology and Business, Astana, Kazakhstan,

\tsp{2}Saken Seifullin Kazakh Agrotechnical Research University Astana, Kazakhstan,

e-mail: shynarai\_92@mail.ru
\end{affil}

This article presents the formulation, organoleptic and physicochemical
indicators of the emulsion product in the development of technology for
obtaining emulsion products based on plant raw materials, as well as the
results of the analysis of the fatty acid content of the food and energy
value of the product. According to the results of the analysis, the
organoleptic and physico-chemical indicators of the emulsion product
were determined in comparison with traditional sauce, that is, the
appearance is creamy, the consistency is homogeneous, the taste is
delicate in smell and taste, not bitter, souring, has a light yellow
color. Physico-chemical indicators of emulsion products: humidity -
51.3\%, PH acidity - 3.6, emulsion stability - 100 \%. Nutritional value
of the emulsion product: fat - 45\%, protein - 3\%, carbohydrate - 4\%,
energy value - 432 kcal. Fatty acid composition of the emulsion product:
linolenic acid -- 9.8; linoleic acid -- 27.2, i.e. the ratio of ω-3:ω-6
polyunsaturated fatty acids is 1:4.

{\bfseries Keywords:} mayonnaise, physico-chemical indicators, organoleptic
indicators, energy value, nutritional value, fatty acid content

\begin{multicols}{2}
{\bfseries Кіріспе.} Эмульсиялық май өнімдерін пайдалану - бұл халықтың
тамақтану құрылымын оңтайландыру және диетаны қажетті майлармен толтыру
тәсілі. Өсімдік майлары ағзаны адамның физиологиялық белсенді май
қышқылдарымен қамтамасыз етеді, қандағы холестеринді төмендетеді және
атеросклероздың алдын алуға көмектеседі. Дегенмен, дәстүрлі түрде
еліміздің халқы күнбағыс майын тамақ үшін пайдаланады, ал өндірушілер
майонез өнімдерін дайындау үшін де тек күнбағыс майын пайдаланады. Майды
тұтыну май қышқылдары мен биологиялық белсенді заттардың құрамын
ескерусіз жүреді. Өсімдік майларының биологиялық құндылығы -
фосфолипидтер, стеролдар, майда еритін дәрумендер, эфир майларына
байланысты {[}1{]}.

Эмульсиялық тамақ өнімдерінің тұтынушылық қасиеттерін қалыптастыру
кезінде жоғары беріктігі бар шоғырланған және жоғары концентрацияланған
эмульсияларды алу міндеті қойылады. Егер тек май мен су фазалары
гомогенизацияланса, жүйе әртүрлі механизмдер, соның ішінде тамшылардың
флоккуляциясы, коалесценция, Оствальдтың жетілуі және гравитациялық
бөліну арқылы тез бұзылады. Сондықтан тамақ эмульсияларының
формулаларында дисперсті жүйенің фазаларының бірінде еритін
эмульгаторларды қолдану қажет {[}2{]}.

Майонез құрамының көп компонентті болуын және эмульгаторлық компонент
ретінде шикізаттың түрлі ассортиментін пайдалану мүмкіндігін ескерсек,
өңдірілетін өнім ассортиментін кеңейту перспективалары өте әсерлі болып
табылады. Азық-түлік технологиясын дамытудағы негізгі үрдіс
физиологиялық функционалды тамақтануға арналған өнімдерді өндіру болып
табылады {[}3{]}.

Майонез, осы тұрғыдан алғанда, перспективалы тағам болып табылады. Бұл
күрделі, көп компонентті жүйе, сондықтан рецептке дайын өнімнің
биологиялық құндылығын арттыратын қоспаларды енгізу мүмкіндігі бар.
Мұндай қоспалар ретінде ақуыздар, майлар, көмірсулар, дәрумендер,
минералдар бар табиғи кешендерді енгізу өте тиімді. Майонездің
компоненттік құрамы оларды функционалдық ингредиенттердің барлық
түрлерімен байытуға мүмкіндік береді: тағамдық талшықтармен,
витаминдермен, минералды заттармен, омега - 3 май қышқылдарымен, майда
еритін антиоксиданттармен және т. б. {[}4{]}.

Майонез және соған ұқсас тұздықтар құрамында полиқанықпаған май
қышқылдары және түрлі дәмдік қоспалары жоғары сұйық өсімдік майларына
негізделген көп компонентті өнімдер болып табылады. Майонез өнімдерінде
ақуыз, май ингредиенттерінің, судың болуы биохимиялық және
автокаталитикалық тотығу реакцияларының белсендірілуіне әкелетін
микробиологиялық, гидролитикалық, тотығу процестерінің қатар жүруіне
мүмкіндік береді. Бұл тамақ өнімдерінің органолептикалық қасиеттерінің
нашарлауына әкеліп соқтырады, сонымен қатар оның тағамдық құндылығын,
соның ішінде биологиялық құндылығын төмендетеді, бұл өмірлік маңызды май
қышқылдарының тотығуымен, сондай-ақ каротиноидтардың, токоферолдардың
және басқа биологиялық белсенді заттардың жойылуымен байланысты {[}5{]}.

{\bfseries Материалдар мен әдістер.} Зерттеу нысаны ретінде С.Сейфуллин
атындағы Қазақ агротехникалық университетінің «Тамақ және қайта өңдеу
өндірісінің технологиясы» кафедрасындағы «Майлы дақылдарды қайта өңдеуге
арналған тәжірибелік-өндірістік цехында» өндірілген зығыр және күнбағыс
майлары алынды. Жұмыс барысында қолданылатын барлық компоненттер
қолданыстағы нормативтік-техникалық құжаттаманың талаптарына сәйкеc.

\emph{Эмульсиялық өнімді зерттеу әдістері.} Өнімнің эмульсиялық
тұрақтылығын анықтау ГОСТ 31762-2012. Эмульсияның тұрақтылығының
көрсеткіші -- күшті механикалық және жылулық әсерлердің нәтижесінде
майонезден бөлінетін майдың мөлшері. Майонез эмульсиясы жоғары
температурада (45°С-тан жоғары) тұрақты емес және қыздырылған кезде
құрамға кіретін басқа өнімдердің тамшы тәрізді ұсақ, анық көрінетін
қоспалары бар таза өсімдік майына оңай ыдырайды.

Ылғалдың массалық үлесі ГОСТ 31762-2012 сәйкес анықталды. Бұл әдіс өнім
сапасы жағынан күдіктер туғызғанда және ылғалдың массалық үлесі
1,0\%-дан 95,0\%-ға дейінгі аралықта болғанда қолданылады.

Өнімнің pH -- көрсеткішін анықтау. Потенциометриялық әдіс талданатын
үлгіге батырылған екі электродтың (өлшеу және эталондық электрод)
арасындағы потенциалдар айырмасын өлшеуге негізделген.

Өнімнің энергетикалық құндылығы макронутриенттердің мөлшерін
энергетикалық құндылық факторларымен (кДж) қайта есептеу арқылы
анықталды: ақуыздар мен көмірсулар үшін - 4, майлар үшін - 9.

Дайын өнімнің органолептикалық көрсеткіштерін бағалау ГОСТ 28283-2015
сәйкес дегустация әдісімен өткізілді. Келесі көрсеткіштер бақыланады:
иісі, дәмі, консистенциясы, сыртқы түрі және түсі, олар сандық балл
арқылы өрнектелді. Бағалау 5 балдық шкала бойынша жүргізілді, содан
кейін профилограммалар салынды.

Өнімнің май қышқылды құрамын анықтау -- капиллярлық газ хроматографиясы
әдісі арқылы орындалды.

Сығынды үлгілерін зерттеу әдістері. Primula veris (L.) сығындыларында
жеке фенолдық қосылыстар өнімділігі жоғары сұйық хроматография (HPLC)
әдісімен анықталды.

Сығындының құрамындағы суда еритін дәрумендерді анықтау үшін жоғары
тиімді сұйық хроматография (HPLC) әдісі ГОСТ 34151-2017 сәйкес
қолданылды. С дәруменінің құрамы L(+)-аскорбин және L(+)-дегидроаскорбин
қышқылдарының қосындысы түрінде анықталады.

{\bfseries Нәтижелер мен талқылау.} Primula veris сығындысын дайындау және
оның эмульсиялық өнімге әсері. Primula veris L. кептірілген жапырақтары
IKA type A 11 негізгі аналитикалық диірменінде ұсақталды , содан кейін
100 г ұсақталған материал өлшеніп, экстракциялық колбаларға салынып, 500
мл еріткішпен араластырылды.1 а, ә, б, в - суретте Primula veris
сығындысын дайындалуы көрсетілген.

Термостат функциясы бар Polsonic Sonic 22 ультрадыбыстық ваннасына 70 °C
температурада 60 минутқа орналастырылды. Экстракция үшін 1:3 қатынасында
дистилденген су және 70\% спирт бар этанол қолданылды. Ультрадыбыспен
экстракция аяқталып, кейін алынған сығынды сүзгі қағазы арқылы сүзілді.
Содан кейін RV 10 digital V айналмалы буландырғышта сүзінді 30 минут
ішінде айдалып, тұнған сұйықтық, яғни алынған дайын қою сығынды бөлек
таза пробиркаларға құйылды. Primula veris сығындысын дайындауда
Tarapatskyy M т.б. авторлар терең зерттеулер жүргізген {[}6,7{]}.

P. Veris-те анықталған фенолдық қосылыстар мөлшері 1-кестеде
көрсетілген.
\end{multicols}

\begin{figs}[1 - сурет. Primula veris сығындысын дайындау]
  \fig[0.24\textwidth][4cm]{p/image29}[а]
  \fig[0.24\textwidth][4cm]{p/image30}[ә]
  \fig[0.24\textwidth][4cm]{p/image31}[б]
  \fig[0.24\textwidth][4cm]{p/image32}[в]
\end{figs}

\tcap{1-кесте. - P. Veris-те анықталған фенолдық қосылыстар (мг/100 г)}
\begin{longtblr}[
  label = none,
  entry = none,
]{
  cells = {c},
  cells = {font = \small},
  hlines,
  vlines,
}
\textbf{Атаулары}     & \textbf{Мөлшері} \\
Катехин               & 312,11           \\
Ориентин              & 204,23           \\
Рутозид               & 630,83           \\
Изорамнетин-рутинозид & 740,24           \\
Изорамнетин-гликозид  & 448,45           \\
Астрагалин            & 185,07           \\
Хлорген қышқылы       & 72,84            
\end{longtblr}

\begin{multicols}{2}
1-кестеде көрсетілгендей P. Veris-те анықталған фенолдық қосылыстардың
ішінде ең көп изорамнетин-рутинозид - 740,24 және рутозид - 630,83
флавоноидтары байқалды.5-ші суретте Primula veris сығындысындағы
фенолды қосылыстардың хроматограммасы келтірілген.
\end{multicols}

\fig[0.5\textwidth]{p/image33}[2-сурет. Primula veris сығындысындағы фенолды қосылыстардың хроматограммасы]

\begin{multicols}{2}
Иванова Д.Ф. ғылыми-тәжірибелік жұмысында Primula veris сығындысының
құрамындағы фенолды қосылыстар мен аскорбин қышқылын анықтаған {[}8{]}.
Зерттеу нәтижесінде сығынды құрамында алты флавоноидты қосылыстардың
болуын растайды, атап айтқанда: ориентин (лютеолин-8-C-глюкозид),
рутозид (кверцетин 3-О-рутинозид), изорамнетин-3-о-рутинозид,
изорамнетин-3-о-глюкозид, астрагалин (кемпферол-3-О-глюкозид) және (+)-
катехин. Сонымен қатар басқа ғалымдардың еңбектерінен тиамин (B1),
рибофлавин (B2), ниацин (B3), пантотен қышқылы (B5), пиридоксин (B6),
фолий қышқылы (B9), цианокобаламин (B12) және аскорбин қышқылы (C)
сияқты басқа да витаминдерді тиімді анықтауға болатыны көрсетілген
{[}9,10{]}.

P. Veris-те анықталған витаминдердің мөлшері 2-кестеде көрсетілген.
\end{multicols}

\tcap{2 - кесте. P. Veris-те анықталған суда еритін витаминдер (мг/100 г) {[}8{]}.}
\begin{longtblr}[
  label = none,
  entry = none,
]{
  cells = {c},
  cells = {font = \small},
  hlines,
  vlines,
}
Атаулары                & Мөлшері \\
Аскорбин қышқылы        & 3,01    \\
Фолий қышқылы           & 4,47    \\
Пиридоксина гидрохлорид & 5,29    \\
Тиамина гидрохлорид     & 8,74
\end{longtblr}


3-ші суретте Primula veris сығындысындағы суда еритін витаминдердің
хроматограммасы келтірілген {[}8{]}.

\fig[0.5\textwidth]{p/image34}[3 - сурет. Primula veris сығындысындағы суда еритін витаминдердің хроматограммасы]

\begin{multicols}{2}
Рецепт құрамына кіретін шикізаттың оңтайлы қатынасын және эмульсиялық
өнім дайындаудың технологиялық параметрлерін таңдау үшін күнбағыс майы
мен зығыр майының қасиеттері мен эмульсиялық өнімнің сапасына әсерін
зерттеуге бағытталған жұмыстарға талдау жасалды.

Осы мақсатта «Салатный» майонезді соусының негізгі шикізаттарына зығыр
майын 25\% мөлшерде ауыстыра отырып, сынама майонезді соус дайындау
жүзеге асырылды.3-кестеде эмульсиялық өнімнің рецетурасы көрсетілген.
\end{multicols}

\tcap{3 - кесте. Эмульсиялық өнімдердің рецептуралары}
\begin{longtblr}[
  label = none,
  entry = none,
]{
  cells = {c},
  cells = {font = \small},
  hlines,
  vlines,
}
\textbf{Шикізат атауы} & \textbf{Бақылау үлгісі} & \textbf{Май қоспасынан жасалған өнім} \\
Күнбағыс майы, л       & 450                     & 337,5                                 \\
Зығыр майы, л          & -                       & 112,5                                 \\
Қант, кг               & 20                      & 15                                    \\
Ас тұзы, кг            & 20                      & 13                                    \\
Лимон қышқылы, кг      & -                       & 6,5                                   \\
Сірке қышқылы, л       & 8                       & -                                     \\
Жұмыртқа ұнтағы, кг    & 50                      & 48                                    \\
Қыша ұнтағы, кг        & 10                      & 7,5                                   \\
Крахмал, кг            & 22                      & -                                     \\
Су, л                  & 420                     & 445                                   \\
Экстракт, кг           & -                       & 15                                    \\
Жалпы:                 & 1000                    & 1000                                  
\end{longtblr}

\begin{multicols}{2}
Зертханалық жағдайда тәжірибе үлгісі мен бақылау үлгілері алынып,
талданды. Өнімнің органолептикалық және физика-химиялық көрсеткіштері
сыртқы түрі, консистенция, иіс пен дәм бойынша дәстүрлі тұздықпен
салыстыра анықталды.4-кестеде өнімнің органолептикалық көрсеткіштері
келтірілген.
\end{multicols}

\tcap{4 - кесте. Эмульсиялық өнімдердің органолептикалық көрсеткіштері}
\begin{longtblr}[
  label = none,
  entry = none,
]{
  width = \linewidth,
  colspec = {Q[200]Q[252]Q[257]Q[98]Q[98]},
  cells = {c},
  cells = {font = \small},
  cell{1}{1} = {r=2}{},
  cell{1}{2} = {r=2}{},
  cell{1}{3} = {r=2}{},
  cell{1}{4} = {c=2}{0.175\linewidth},
  vlines,
  hline{1,3-6} = {-}{},
  hline{2} = {4-5}{},
}
\textbf{Көрсеткіштер}       & \textbf{Тәжірибе үлгісі}          & \textbf{Бақылау үлгісі}             & \textbf{Орташа балл} &                  \\
                            &                                   &                                     & \textbf{Тәжірибе}    & \textbf{Бақылау} \\
Сыртқы түрі, консистенциясы & Біртекті қаймақ тәрізді, біртекті & Біртекті қаймақ тәрізді, біртекті   & 4,9                  & 4                \\
Дәмі мен иісі               & Нәзік дәм, ащы емес, қышқылдау    & Жағымды, өткірлеу, қышаның дәмі бар & 4,98                 & 3,9              \\
Түсі                        & Ашық сары                         & Ақ                                  & 5                    & 4,5              
\end{longtblr}

\begin{multicols}{2}
Дайын өнімнің органолептикалық көрсеткіштері С. Сейфуллин атындағы Қазақ
агротехникалық университетінің Тамақ және қайта өңдеу өндірісінің
технологиясы кафедрасының меңгерушісіне, оқытушыларына дегустация
жүргізіп, тәжірибе үлгісін оңтайлы деп анықталды. Кестелерде көрсетілген
деректер өнім үлгілерінің барлық мәндер үшін ГОСТ Р 53590-2009 және ГОСТ
31761-2012 сәйкестігін көрсетеді.5-ші кестеде өнімнің физика-химиялық
көрсеткіштері келтірілген.
\end{multicols}

\tcap{5 - кесте. Эмульсиялық өнімдердің физика-химиялық көрсеткіштері}
\begin{longtblr}[
  label = none,
  entry = none,
]{
  cells = {c},
  cells = {font = \small},
  hlines,
  vlines,
}
Көрсеткіштер атауы       & Тәжірибе үлгісі & Бақылау үлгісі \\
Ылғалдылық, \%           & 51,3            & 50,5           \\
pH                       & 3,6             & 3,9            \\
Эмульсия тұрақтылығы, \% & 100,0           & 98,0           
\end{longtblr}

\begin{multicols}{2}
Әзірленген өнімнің тағамдық құндылығы маңызды май қышқылдары мен
фосфолипидтердің, органикалық қышқылдар, таниндер, сапониндер, С және Е
дәрумендері болуына негізделген.

Эмульсиялық өнімнің ақуыз, көмірсу және майдың массалық үлестері бойынша
өнімнің энергетикалық құндылығы өздеріне тиеселі коэффиценттерге
көбейтіле отырып есептелді және 6 - кестеге салынды.
\end{multicols}

\tcap{6 - кесте. Эмульсиялық өнімнің тағамдық және энергетикалық құндылығы}
\begin{longtblr}[
  label = none,
  entry = none,
]{
  width = \linewidth,
  colspec = {Q[198]Q[69]Q[90]Q[113]Q[227]Q[231]},
  cells = {c},
  cell{1}{1} = {r=2}{},
  cell{1}{2} = {c=3}{0.272\linewidth},
  cell{1}{5} = {c=2}{0.458\linewidth},
  vlines,
  hline{1,3-4} = {-}{},
  hline{2} = {2-6}{},
}
Өнім           & Массалық үлесі, \% &       &         & Энергетикалық құндылық (100 г.) &      \\
               & май                & ақуыз & көмірсу & Ккал                            & кДж  \\
Зерттеу үлгісі & 45,0               & 3,0   & 4,0     & 432                             & 1808 
\end{longtblr}

\begin{multicols}{2}
Әзірленген өнімнің тағамдық құндылығы Е витаминінің көзі болып
табылатын, омега қышқылдары бойынша теңдестірілген майлардың қоспасы
болып табылатын май негізін рецептураға қосумен және денсаулықты сақтау
үшін қажетті биологиялық белсенді заттардың қосылуымен байланысты.7 -
кестеде эмульсиялық өнімдердің май қышқылдық құрамы көрсетілген.
\end{multicols}

\tcap{7 - кесте. Эмульсиялық өнімдердің май қышқылдық құрамы}
\begin{longtblr}[
  label = none,
  entry = none,
]{
  width = \linewidth,
  colspec = {Q[352]Q[154]Q[210]Q[217]},
  cells = {c},
  cell{1}{1} = {r=2}{},
  cell{1}{2} = {r=2}{},
  cell{1}{3} = {c=2}{0.427\linewidth},
  vlines,
  hline{1,3-14} = {-}{},
  hline{2} = {3-4}{},
}
Май қышқылының атауы & Белгіленуі & Нәтижесі       &                 \\
                     &            & Бақылау үлгісі & Тәжірибе үлгісі \\
Миристин             & С14:0      & 0,03           & 0,02            \\
Пальмитин            & С16:0      & 3,30           & 2,40            \\
Пальмитолеин         & С16:1      & 0,06           & 0,07            \\
Стеарин              & С18:0      & 2,93           & 2,05            \\
Олеин                & С18:1      & 12,37          & 10,24           \\
Линол                & С18:2      & 35,10          & 27,20           \\
Линолен              & С18:3      & 0,05           & 9,80            \\
Арахин               & С20:0      & 0,25           & 0,11            \\
Гондоин              & С20:1      & 0,07           & 0,06            \\
Беген                & С22:0      & 0,25           & 0,32            \\
Лигноцерин           & С24:0      & 0,04           & 0,05            
\end{longtblr}

\begin{multicols}{2}
Кестедегі зерттеу нәтижесі бойынша зерттеу үлгісінде линолен кышқылы --
9,8; линол қышқылы -- 27,2 көрсетті, яғни ω-3:ω-6 полиқанықпаған май
қышқылдарының қатынасы 1:4-ке тең, бұл дұрыс тамақтану үшін ұсынылған
нормаға сәйкес келуін дәлелдейді. Ал, бақылау үлгісінде линолен қышқылы
-0,05; линол қышқылы - 35,10; сәйкесінше бұл нәтиже нормадан ауытқып
кеткенін байқадық.

{\bfseries Қорытынды.} Жаңа эмульсиялық өнімнің шикізаты ретінде ω-3:ω-6
полиқанықпаған май қышқылдарының адам ағзасына пайдалы қатынасы ескеріле
отырып, қолданылатын күнбағыс майы мен зығыр майының оңтайлы қатынасы
75\%:25\% екені анықталды. Эмульсиялық өнімдерді өндіруде өсімдік тектес
компоненттерді қолданудың орындылығы негізделді, еңгізілген Primula
veris сығындысы майонез тұздығымен үйлесетіндігі, оның органолептикалық
және реологиялық көрсеткіштерін жақсартатаны анықталды. Зерттеу
нәтижелері кестелерде келтірілген, сонымен қатар алынған эмульсиялық
өнім «Эмульсия тәрізді майлы тағамдық өнім» ҚР пайдалы модель патентімен
расталды {[}11{]}.
\end{multicols}

\begin{center}
{\bfseries Әдебиеттер}
\end{center}

\begin{refs}
1. Гаврилова Д. В. Разработка и товароведная оценка майонеза и
майонезного соуса для здорового питания с пектином //дис.. канд. тех.
наук: 05.18.15 - Москва, 2014. - 147 c.

2. Феофилактова О. В. Научное и практическое обоснование технологии
фортификации биоактивными комплексами эмульсионных пищевых продуктов
//дис. докт.тех.наук: 4.3.3 -- Екатеринбург, 2024. - 310 c.

3. Журавко Е.В., Грузинов Е.В. Майонез «Диабетический» с экстрактом
стевии // Масложировая промышленность. -№ 2.-2004. -С.41-42.

4. D Guzey and D. J. McClements. Formation, stability and properties of
multilayer emulsions for application in the food industry // Advances in
Colloid and Interface Science 128. -2006.- Vol.128-130. -Р.227-248. DOI
10.1016/j.cis.2006.11.021.

5. Жакова К. И., Бабодей В.Н., Пчельникова А.В. Окислительные процессы в
жировых эмульсионных продуктах прямого типа. Анализ качественных
показателей сырья, используемого при производстве эмульсионных продуктов
прямого типа //Пищевая промышленность: наука и технологии.-2025.-
Т.17(4). - С.35-43.

6. Tarapatskyy M., Gumienna A., Sowa P., Kapusta I., Puchalski C.
Bioactive Phenolic Compounds from Primula veris L.// Influence of the
Extraction Conditions and Purification. Molecules. - 2021. - 26(4). -
997. DOI 10.3390/molecules26040997.

7. Tatiana B. Schreiner, Madalena M. Dias, Maria Filomena Barreiro and
Simão P. Pinho. Saponins as Natural Emulsifiers for Nanoemulsions //
Journal of Agricultural and Food Chemistry. -2022. -Vol.70(22).
-P.6573-6590. DOI 10.1021/acs.jafc.1c07893.

8. Иванова Д. Ф.. Фитохимическое изучение, разработка и стандартизация
лекарственных средств на основе первоцвета весеннего (primula veris l.)
// дис... кан. тех. Наук: 14.04.02-Уфа, 2017. -194 c.

9. Sori C.A., Fayissa G.R. Comparative investigation of the level of
vitamin C in wild edible plants consumed at North Shoa Zone, Oromia
region, Ethiopia//Discov Food.-2025.-Vol.5(300).- P.1-13/
\href{https://doi.org/10.1007/s44187-025-00614-0}{DOI
10.1007/s44187-025-00614-0}

10. Meos A., Saage P., Arak E. Content of ascorbic acid in common
cowslip (Primula veris L.) compared to common foodplants and orange
juices. Acta Biologica Cracoviensia Series Botanica, -2017. -Vol.1(59).
-P.113--120. DOI 10.1515/abcsb-2016-0020.

11. Эмульсия тәрізді майлы тағамдық өнім. Пайдалы модельге патент №7144,
27.05.2022
\end{refs}

\begin{center}
{\bfseries References}
\end{center}

\begin{refs}
1. Gavrilova D. V. Razrabotka i tovarovednaja ocenka majoneza i
majoneznogo sousa dlja zdorovogo pitanija s pektinom //dis.. kand. teh.
nauk: 05.18.15 - Moskva, 2014. - 147 c. {[}in Russian{]}

2. Feofilaktova O. V. Nauchnoe i prakticheskoe obosnovanie tehnologii
fortifikacii bioaktivnymi kompleksami jemul' sionnyh
pishhevyh produktov //dis. dokt.teh.nauk: 4.3.3 -- Ekaterinburg, 2024. -
310 c. {[}in Russian{]}

3. Zhuravko E.V., Gruzinov E.V. Majonez «Diabeticheskij» s jekstraktom
stevii // Maslozhirovaja promyshlennost'. -№ 2.-2004. -S.
41-42. {[}in Russian{]}

4. D Guzey and D. J. McClements. Formation, stability and properties of
multilayer emulsions for application in the food industry // Advances in
Colloid and Interface Science 128. -2006.- Vol.128-130. -Р.227-248. DOI
10.1016/j.cis.2006.11.021.

5. Zhakova K. I., Babodej V.N., Pchel' nikova A.V.
Okislitel' nye processy v zhirovyh
jemul' sionnyh produktah prjamogo tipa. Analiz
kachestvennyh pokazatelej syr' ja,
ispol' zuemogo pri proizvodstve
jemul' sionnyh produktov prjamogo tipa //Pishhevaja
promyshlennost': nauka i tehnologii.-2025.- T.17(4). - S.
35-43. {[}in Russian{]}

6. Tarapatskyy M., Gumienna A., Sowa P., Kapusta I., Puchalski C.
Bioactive Phenolic Compounds from Primula veris L.// Influence of the
Extraction Conditions and Purification. Molecules. - 2021. - 26(4). -
997. DOI 10.3390/molecules26040997.

7. Tatiana B. Schreiner, Madalena M. Dias, Maria Filomena Barreiro and
Simão P. Pinho. Saponins as Natural Emulsifiers for Nanoemulsions //
Journal of Agricultural and Food Chemistry. -2022. -Vol.70(22).
-P.6573-6590. DOI 10.1021/acs.jafc.1c07893.

8. Ivanova D. F.. Fitohimicheskoe izuchenie, razrabotka i
standartizacija lekarstvennyh sredstv na osnove pervocveta vesennego
(primula veris l.) // dis... kan. teh. Nauk: 14.04.02-Ufa, 2017. -194 c.
{[}in Russian{]}

9. Sori C.A., Fayissa G.R. Comparative investigation of the level of
vitamin C in wild edible plants consumed at North Shoa Zone, Oromia
region, Ethiopia//Discov Food.-2025.-Vol.5(300).- P.1-13/
\href{https://doi.org/10.1007/s44187-025-00614-0}{DOI
10.1007/s44187-025-00614-0}

10. Meos A., Saage P., Arak E. Content of ascorbic acid in common
cowslip (Primula veris L.) compared to common foodplants and orange
juices. Acta Biologica Cracoviensia Series Botanica, -2017. -Vol.1(59).
-P.113--120. DOI 10.1515/abcsb-2016-0020.

11. Эмульсия тәрізді майлы тағамдық өнім. Пайдалы модельге патент №7144,
27.05.2022. 
\end{refs}

\begin{info}
\hspace{1em}\emph{{\bfseries Авторлар туралы мәліметтер}}

Смагулова М.Е. -- химия ғылымдарының кандидаты, жетекші ғылыми
қызметкер, К.Кулажанов атындағы Қазақ технология және бизнес
университеті, Астана, Қазақстан, қаласы, e-mail:
mirgul.smagulova@bk.ru;

Жасқайрат Ш.Ж.- т.ғ.м., С.Сейфуллин атындағы Қазақ агротехникалық
зерттеу университетінің 3-курс докторанты, Астана Қазақстан, e-mail:
shynarai\_92@mail.ru;

Машанова Н.С. - техника ғылымдарының докторы, аға оқытушы, С.Сейфуллин
атындағы Қазақ агротехникалық зерттеу университеті, Астана, Қазақстан,
e-mail: nurmashanova@gmail.com;

Бекболатов М.Е. - т.ғ.м., С.Сейфуллин атындағы Қазақ агротехникалық
зерттеу университетінің 3-курс докторанты, Астана, Қазақстан, e-mail:
mariambekbolatova93@gmail.com.

\hspace{1em}\emph{{\bfseries Information about the authors}}

Smagulova M. E.- candidate of chemical sciences, Leading Researcher,
K.  Kulazhanov Kazakh University of Technology and Business , Astana,
Kazakhstan, e-mail: mirgul.smagulova@bk.ru;

Zhaskairat Sh. Zh. - мaster of technical sciences, Saken Seifullin
Kazakh Agrotechnical Research University, 3rd-year PhD student,
Astana, Kazakhstan, e-mail: shynarai\_92@mail.ru;

Mashanova N.S. - doctor of technical sciences, senior lecturer, Saken
Seifullin Kazakh Agrotechnical Research University, Astana, Kazakhstan
, e-mail: nurmashanova@gmail.com;

Bekbolatova M.E. - master of technical sciences, Saken Seifullin
Kazakh Agrotechnical Research University, 3rd-year PhD student,
Astana, Kazakhstan , e-mail: mariambekbolatova93@gmail.com.
\end{info}
