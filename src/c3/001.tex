\id{МРНТИ 61.45.36}{};
27.35.30

{\bfseries МАТЕМАТИЧЕСКАЯ МОДЕЛЬ ЭКСТРАКЦИИ НА ОСНОВЕ КИНЕТИКИ
ЭКСТРАКЦИОННОГО ПРОЦЕССА}

{\bfseries \tsp{1,2}К.М.
Сулейменов}\fig{c3/image1}{}{\bfseries ,
\tsp{3}Г.Қ.
Мамытбекова}\fig{c3/image1}{}{\bfseries ,
\tsp{3}Ж.
Нұрымов}\fig{c3/image1}{}{\bfseries ,
\tsp{3-4} К.Г.
Сатенов}\fig{c3/image1}{},

{\bfseries \tsp{3,4,5}Е.М.Сүлеймен}\fig{c3/image1}{}\envelope 

\emph{\tsp{1}Евразийский национальный университет им. Л.Н.
Гумилева, Астана, Казахстан,}

\emph{\tsp{2}Общественный фонд «TEMS-Technology, Education
and Mathematical Science», Астана, Казахстан,}

\emph{\tsp{3}Казахский университет технологии и бизнеса
им.К. Кулажанова, Астана, Казахстан,}

\emph{\tsp{4}ТОО КМГ Инжиниринг, Aстана, Казахстан,}

\emph{\tsp{5}НАО Кокшетауский университет им. Ш. Уалиханова,
Кокшетау, Казахстан}

\envelope Корреспондент-автор:
syerlan75@yandex.kz

В данной работе рассматривается математическая модель экстракционного
процесса, основанная на кинетике массообмена между растительным сырьем и
экстрагентом. Основное внимание уделено моделированию изменения
концентраций сухих растворимых веществ в экстрагенте и в исходном
растительном материале при экстракции. Разработана система
дифференциальных уравнений, описывающая динамику экстрагирования,
которая была решена аналитическими методами. В качестве примера проведен
расчет для растения \emph{Achillea asiatica} Serg., и получены
выражения, описывающие изменение концентраций во времени. Математическая
модель описывает процесс экстракции, применена для обработки
лабораторных данных и может быть использована для прогнозирования и
оптимизации технологических параметров. В работе использованы данные по
экстракции растений рода \emph{Achillea} и \emph{Polygonum} с
использованием смеси этанол-хлороформ в качестве экстрагента. Полученные
результаты способствуют улучшению понимания массообменных процессов и
могут быть применимы в фармацевтической и пищевой промышленности.

{\bfseries Ключевые слова:} экстракция, кинетика, массообмен,
математическое моделирование, Achillea, растительное сырье, экстрагент.

{\bfseries ЭКСТРАКЦИЯЛЫҚ ПРОЦЕСТІҢ КИНЕТИКАСЫНА НЕГІЗГЕН ЭКСТРАКЦИЯНЫҢ
МАТЕМАТИКАЛЫҚ МОДЕЛІ}

{\bfseries \tsp{1,2}К.М. Сулейменов, \tsp{3}Г.Қ.
Мамытбекова , \tsp{3}Ж. Нұрымов, \tsp{3-4}К.Г.
Сатенов, \tsp{3,4,5}Е.М.Сүлеймен\envelope }

\emph{\tsp{1}Л.Н. Гумилев атындағы Еуразия ұлттық
университеті Л.Н. Гумилев, Астана, Қазақстан,}

\emph{\tsp{2}«TEMS-Технология, білім және математика ғылымы»
қоғамдық қоры, Астана, Қазақстан,}

\emph{\tsp{3}Қ. Құлажанов атындағы Қазақ технология және
бизнес университетінің Астана, Қазақстан,}

\emph{\tsp{4} «KMG Engineering» ЖШС, Астана, Қазақстан,}

\emph{\tsp{5}Ш. Уалиханов атындағы Көкшетау университеті,
Көкшетау, Қазақстан,}

\emph{e-mail:syerlan75@yandex.kz}

Бұл мақалада өсімдік шикізаты мен экстрагент арасындағы массаалмасу
кинетикасына негізделген экстракциялық процестің математикалық моделі
қарастырылады. Зерттеуде экстрагент пен өсімдік құрамындағы құрғақ
еритін заттардың концентрациясының уақыт бойынша өзгеруін сипаттайтын
дифференциалдық теңдеулер жүйесі ұсынылды және аналитикалық әдістермен
шешілді. \emph{Achillea asiatica} Serg. шөбі мысалында модельдің
тиімділігі көрсетілді. Математикалық модель экстракциялық процесті
сипаттайды, лабораториялық деректерді өңдеуге пайдаланылды және болжау
мен технологиялық параметрлерді тиімділеуге қолданылуы мүмкін. Зерттеу
барысында \emph{Achillea} және \emph{Polygonum} туысына жататын
өсімдіктердің этанол-хлороформ қоспасымен экстракциясы жүргізілді.
Ұсынылған модель экстракция технологиясын оңтайландыруға және
фармацевтикалық, тағам өнеркәсібінде қолдануға мүмкіндік береді.

{\bfseries Түйін сөздер:} экстракция, кинетика, массаалмасу, математикалық
модельдеу, Achillea, өсімдік шикізаты, экстрагент.

{\bfseries MATHEMATICAL MODEL OF EXTRACTION BASED ON THE KINETICS OF THE
EXTRACTION PROCESS}

{\bfseries \tsp{1,2}K.M. Suleimenov, \tsp{3}G.K.
Mamytbekova, \tsp{3}Zh. Nurymov, \tsp{3-4}K.G.
Satenov,\tsp{3,4,5}Ye.M. Suleimen\envelope }

\emph{\tsp{1}L.N. Gumilyov Eurasian National University,
Astana, Kazakhstan,}

\emph{\tsp{2}TEMS-Technology, «TEMS-Technology, Education
and Mathematical Science», Education and Public Foundation, Astana,
Kazakhstan,}

\emph{\tsp{3}K. Kulajanov Kazakh University of Technology
and Business, Astana, Kazakhstan,}

\emph{\tsp{4} KMG Engineering LLP, Astana, Kazakhstan,}

\emph{\tsp{5} Sh. Ualikhanov Kokshetau University,
Kokshetau, Kazakhstan,}

\emph{e-mail:syerlan75@yandex.kz}

This paper presents a mathematical model of the extraction process based
on the kinetics of mass transfer between plant material and the
extracting solvent. The study focuses on modeling the changes in
concentrations of soluble solids in the solvent and the raw plant matrix
over time. A system of differential equations was developed to describe
the extraction dynamics and solved analytically. As an example, the
extraction kinetics of Achillea asiatica Serg. was modeled, and
analytical expressions were derived to describe concentration changes.
The mathematical model describes the extraction process, has been
applied to the processing of laboratory data, and can be used for
predicting and optimizing technological parameters. The extraction of
several species from the Achillea and Polygonum genera was performed
using an ethanol-chloroform solvent system. The results enhance the
understanding of mass transfer mechanisms and can be applied to optimize
extraction parameters in the pharmaceutical and food industries.

{\bfseries Keywords}: extraction, kinetics, mass transfer, mathematical
modeling, Achillea, plant material, solvent.

{\bfseries Введение.} Экстракция биологически активных веществ (БАВ) из
лекарственных растений, а также из сухих растений является ключевым
процессом в фармацевтической, косметической и пищевой промышленности
{[}1{]}. Растения семейства \emph{Achillea} (тысячелистник),
\emph{Polygonum} (горец) и \emph{Hedysarum} (копеечник) обладают
уникальным химическим составом, включая флавоноиды, фенольные
соединения, полисахариды, сапонины и эфирные масла {[}2-4{]}. Эти
соединения находят широкое применение в производстве препаратов с
антиоксидантными, противовоспалительными и антимикробными свойствами
{[}5{]}.

Целью исследований работы является разработтка математической модели
процесса экстракции с использованием смеси этанол-хлороформ в качестве
экстрагента.

Основной задачей экстракции является извлечение целевых соединений с
высокой эффективностью при минимальных затратах и без разрушения
структуры биоактивных веществ. Для достижения этой цели используются
различные растворители, температурные режимы и технологии, такие как
сверхкритическая экстракция, ультразвуковая обработка, микроволновая
экстракция и традиционная экстракция жидкость-жидкость {[}6-7{]}.
Приведем некоторые исследования, посвященные процессу экстракции.

В работе {[}8{]} изучается процесс экстракции из пористых частиц
масличного материала, поры которого частично заполнены маслом, может
быть представлен как диффузионный процесс в пористом теле, при этом
рассматривается случай, когда макропоры частицы первоначально заполнены
чистым растворителем, а микропоры - маслом.

При этом рассматриваются массовые балансы макропоры и микроаоры, которые
описываются следующими дифференциальными уравнениями (обозначения см. в
работе):

\[\frac{\partial}{\partial z}\left\lbrack - D_{a}\frac{\partial C_{a}}{\partial z} \right\rbrack + \frac{\partial C_{a}}{\partial t} + nP_{a}\frac{A_{i}}{A_{a}}\left. \  - D_{i}\frac{\partial C_{i}}{\partial y} \right|_{y = 0} = 0,\]

\[\frac{\partial}{\partial y}\left\lbrack - D_{i}\frac{\partial C_{i}}{\partial y} \right\rbrack + \frac{\partial C_{i}}{\partial t} + \frac{P_{i}}{A_{i}}\frac{\partial q}{\partial t} = 0\]

В работе {[}9{]} изучается ферментная экстракция (EAE) биоактивных
соединений из псевдоплодов дикой розы (Rosa canina L.), также известной
как шиповник, с использованием коммерческого целлюлолитического
ферментного препарата \({Cellic}^{®}\), CTec3 HS. Изучалось влияние
времени экстракции, соотношения твердого и жидкого вещества и загрузки
фермента на общее содержание фенолов (TPC) и общее содержание
флавоноидов (TFC). Было использовано несколько математических моделей
для описания процесса экстракции биологически активных соединений из
растительных материалов, в том числе законы первого порядка, второго
порядка, Пелега и мощности. Экстракция из твердой фазы в жидкую может
рассматриваться как противоположную операцию адсорбции (массоперенос
растворителя между твердой фазой и растворителем). Уравнение, обычно
применяемое для адсорбции, может быть также адаптировано для процесса
экстракции, где кинетика часто описывается реакцией первого порядка.
Кинетическая модель первого порядка может быть записана в виде уравнения
(обозначения см. в работе):

\[\frac{dC_{t}}{dt} = k\left( C_{s} - C_{t} \right)\]

Разными методами проводится процесс экстракции биологически активных
веществ (БАВ) из растительного сырья. Многое зависит от района
произрастания, условий сушки и хранения растительного сырья, а также
дальнейшего метода переработки. Кроме того, качественная переработка
зависит от следующих факторов - температура процесса, вида экстрагента и
объема, химического состава сырья, от этих факторов зависит выбор
технологических параметров процесса {[}8-10{]}.

{\bfseries Материалы и методы.} \emph{Математическая модель экстракционного
процесса.} Данная работа посвящена изучению математической модели
экстракции одного вещества, поэтому будем изучать кинетику
экстракционного вещества, которая определяется системой дифференциальных
уравнений.

\(\left\{ \begin{array}{r}
\frac{dC_{x}}{dt} = - \alpha\left( C_{x} - C_{f} \right), \\
\frac{dC_{f}}{dt} = \alpha\left( C_{x} - C_{f} \right) - \gamma C_{f}
\end{array} \right.\ \) (1)

Здесь \(C_{x}(0)\) - заданная начальная масса сухих растворимых веществ
в сырье; \(C_{f}\) - масса сухих растворимых веществ в экстрагенте,
причем \(C_{f}(0) = 0\); \(\alpha\) -- скорость массообмена, т.е.

\[\alpha = \frac{1}{g}\frac{dg}{dt},\]

где \(g\) - полная масса экстрагента (растворителя).

В работе {[}11{]} для процесса экстрагирования использовалась
соответствующая математическая модель (1). Отличием применения
математической модели в данной работе является сам процесс
экстрагирования, а именно, в работе {[}11{]} процесс экстрагирования
осуществляется посредством сжиженного диоксида углерода, в данной работе
используются с смеси этанол-хлороформ в качестве экстрагента.

Целью изучения математической модели (1) является определение
функционального изменения масс при массопереносе из сухого растительного
сырья в растворимый вид экстракта.

Интегрируя или же решая численно систему (1), можем получить прямую
задачу массообмена для указанных условий, т.е. при определенном значений
\(\alpha\) определить \(C_{f}(t),\ C_{x}(t)\), а также т обратную задачу
определения параметра по результатам проведенного эксперимента. Будем
считать, что скорость массообмена, а также параметр \(\gamma\)
известными, т.е. константами, так как при выборе определенного растения
и известной температуры, они определяются отдельно.

Систему (1) также можно решить и методом исключения, для этого первое
уравнение продифференцируем по переменной \(t\), тогда

\[\frac{d^{2}C_{x}}{dt^{2}} = - \alpha\frac{dC_{x}}{dt} + \alpha\frac{dC_{f}}{dt} = - \alpha\left\lbrack - \alpha\left( C_{x} - C_{f} \right) \right\rbrack + \alpha\left\lbrack \alpha\left( C_{x} - C_{f} \right) - \gamma C_{f} \right\rbrack =\]

\[= \alpha^{2}C_{x} - \alpha^{2}C_{x} + \alpha^{2}C_{x} - \alpha^{2}C_{x} - \alpha\gamma C_{f} = {2\alpha}^{2}C_{x} + \left( {2\alpha}^{2} - \alpha\gamma \right)C_{f}\]

Тогда, получим равносильную систему для (1)

\(\left\{ \begin{array}{r}
\frac{dC_{x}}{dt} = - \alpha\left( C_{x} - C_{f} \right), \\
\frac{d^{2}C_{x}}{dt^{2}} = {2\alpha}^{2}C_{x} + \alpha(2\alpha - \gamma)C_{f}
\end{array} \right.\ \) (2)

Теперь, первое уравнение умножим на \((2\alpha - \gamma)\) и сложим со
вторым уравнением, тогда

\(\frac{d^{2}C_{x}}{dt^{2}} + (2\alpha - \gamma)\frac{dC_{x}}{dt} - \alpha\gamma C_{x} = 0\)
(3)

Так как \(\alpha,\gamma = const\), то уравнение (3) решается методом
характеристического уравнения

\[k^{2} + (2\alpha - \gamma)k - \alpha\gamma = 0,\]

отсюда

\[\left\{ \begin{array}{r}
k_{1} = \frac{- (2\alpha - \gamma) - \sqrt{4\alpha^{2} + \gamma^{2}}}{2}, \\
k_{2} = \frac{- (2\alpha - \gamma) + \sqrt{4\alpha^{2} + \gamma^{2}}}{2},
\end{array} \right.\ \]

следовательно,

\(C_{x} = N_{1}e^{k_{1}t} + N_{2}e^{k_{2}t},\) (4)

где \(N_{1},N_{2}\) - произвольные константы, которые определяются
начальными и конечными значениями. Аналогично определяется и \(C_{f}\).

При \(t = 0\) очевидно, что \(C_{x} = 0\), так что
\(N_{1} + N_{2} = 0,\ \ \ а\ при\ t = 3,\ \)концентрации для

каждого растения различны, поэтому рассматриваем отдельно для каждого
растения.

{\bfseries Обсуждение и результаты.} Сначала, приведем экспериментальные
данные.

{\bfseries Таблица 1- Лабораторное получение экстрактов из растительного
сырья}

%% \begin{longtable}[]{@{}
%%   >{\raggedright\arraybackslash}p{(\linewidth - 8\tabcolsep) * \real{0.1300}}
%%   >{\centering\arraybackslash}p{(\linewidth - 8\tabcolsep) * \real{0.1305}}
%%   >{\centering\arraybackslash}p{(\linewidth - 8\tabcolsep) * \real{0.2465}}
%%   >{\centering\arraybackslash}p{(\linewidth - 8\tabcolsep) * \real{0.2175}}
%%   >{\raggedright\arraybackslash}p{(\linewidth - 8\tabcolsep) * \real{0.2755}}@{}}
%% \toprule\noalign{}
%% \begin{minipage}[b]{\linewidth}\raggedright
%% {\bfseries Название травы}
%% \end{minipage} & \begin{minipage}[b]{\linewidth}\centering
%% {\bfseries Масса сырья, г}
%% \end{minipage} & \begin{minipage}[b]{\linewidth}\centering
%% {\bfseries Масса неочищенного экстракта, г}
%% \end{minipage} & \begin{minipage}[b]{\linewidth}\centering
%% {\bfseries Масса очищенного экстракта, г}
%% \end{minipage} & \begin{minipage}[b]{\linewidth}\centering
%% {\bfseries Растворители}
%% \end{minipage} \\
%% \midrule\noalign{}
%% \endhead
%% \bottomrule\noalign{}
%% \endlastfoot
%% \emph{Achillea asiatica} Serg. & 500 & 60 & 38 & Этанол: хлороформ =
%% 1:1 \\
%% \emph{Achillea nobilis} L. & 2000 & 130 & 44 & Этанол: хлороформ =
%% 1:1 \\
%% \emph{Achillea salicifolia} Besser & 500 & 57 & & Этанол: хлороформ =
%% 1:1 \\
%% \emph{Polygonum aviculare} L. & 2000 & 273 & & Этанол: хлороформ =
%% 1:1 \\
%% \emph{Polygonum acerosum} Ledeb. ex Meisn. & 1500 & 109 & & Этанол:
%% хлороформ = 1:1 \\
%% \end{longtable}

Например, рассмотрим решение уравнения для травы \emph{Achillea
asiatica} Serg.

Тогда при \(t = 3\), \(C_{x} = 60\) для неочищенного экстракта, отсюда
получим систему уравнений

\(\left\{ \begin{array}{r}
N_{1} + N_{2} = 0, \\
N_{1}e^{k_{1}t} + N_{2}e^{k_{2}t} = 60
\end{array} \right.\ \) (5)

Сначала определим коэффициенты \(k_{1},\ k_{2}\). Так как \(\gamma\) -
интенсивность потерь вещества из второго резервуара, например, из-за
распада, метаболизма или выведения, а такой потери нет, то
\(\gamma = 0\).

Теперь, определим коэффициент \(\alpha\). Для этого запишем численно
производную \(\frac{dg}{dt}\) в виде конечной разности в промежутке
\(\lbrack 0,\ 3\rbrack\) в виде

\[\frac{dg}{dt} \approx \frac{g_{3} - g_{0}}{3 - 0} = \frac{0,44 - 0,5}{3} = - 0,2\]

Тогда, при \(g = 500\), получим \(\alpha = - \frac{0,2}{0,5} = - 0,4.\)
Отсюда

\[k_{1} = \frac{0,8 - 0,8}{2} = 0,\ k_{2} = \frac{0,8 + 0,8}{2} = 0,8\]

Отсюда уравнение (4) примет вид

\[C_{x} = N_{1} + N_{2}e^{k_{2}t} = N_{1} + N_{2}e^{0,8t}\]

Таким образом, в этом случае, система (5) преобразуется

\(\left\{ \begin{array}{r}
N_{1} + N_{2} = 0, \\
N_{1} + N_{2}e^{0,8 \bullet 3} = 60
\end{array} \right.\ \) (6)

Отсюда, вычитая из второго равенства первое, получим

\[N_{2}\left( e^{0,8 \bullet 3} - 1 \right) = 60,\ N_{2} = \frac{60}{e^{0,8 \bullet 3} - 1} = \frac{60}{10} = 6\]

Тогда \(N_{1} = - 6\), следовательно

\[C_{x} = 6\left( e^{0,8t} - 1 \right)\]

Таким образом, процесс экстрагирования для травы Achillea asiatica Serg
происходит в виде

\(C_{x} = 6\left( e^{0,8t} - 1 \right)\) (7)

Теперь, определим изменения \(C_{f}\) -- концентрация сухих растворимых
веществ в экстрагенте. Данное изменение определим из уравнения

\(\frac{dC_{f}}{dt} = \alpha\left( C_{x} - C_{f} \right) \Longleftrightarrow \frac{dC_{f}}{dt} = 0,8\left( C_{f} - 6\left( e^{0,8t} - 1 \right) \right),\)
(8)

которое решим методом вариации постоянной. Для этого решим однородное
уравнение

\[\frac{dC_{f}}{dt} = 0,8C_{f} \Longleftrightarrow \frac{C_{f}'}{C_{f}} = 0,8dt\  \Longleftrightarrow \ C_{f} = C{(t)e}^{0,8t}\]

Тогда, подставляя в уравнение (8), будем иметь

\[C'{\ (t)e}^{0,8t} + 0,8C{(t)e}^{0,8t} = 0,8C{(t)e}^{0,8t} - 4,8\left( e^{0,8t} - 1 \right)\  \Longleftrightarrow\]

\[\Longleftrightarrow \ C'{\ (t)e}^{0,8t} = - 4,8\left( e^{0,8t} - 1 \right)\  \Longleftrightarrow \ C'(t) = - 4,8\frac{e^{0,8t} - 1}{e^{0,8t}}\  \Longleftrightarrow\]

\[\Longleftrightarrow C(t) = - \left( 6e^{- 0,8t} - 4,8t \right)\]

Отсюда

\[C_{f} = - \left( 6e^{- 0,8t} - 4,8t \right)e^{0,8t} = - 4,8te^{0,8t} + 6\]

Окончательно

\[\left\{ \begin{array}{r}
C_{x} = 6\left( e^{0,8t} - 1 \right), \\
C_{f} = - 4,8te^{0,8t} + 6
\end{array} \right.\ \]

{\bfseries Выводы.} В процессе проведения экстрагирования масса сухих
растворимых веществ в сырье изменяется по закону

\[C_{x} = 6\left( e^{0,8t} - 1 \right),\]

а масса сухих растворимых веществ в экстрагенте происходит по закону

\[C_{f} = - 4,8te^{0,8t} + 6\]

Из полученных результатов видно, что масса сухих растворимых веществ в
сырье увеличивается и максимальное значение зависит от конечного
значения времени, в то же время масса сухих растворимых веществ в
экстрагенте уменьшается и минимальное значение также зависит от
конечного значения времени. При этом масса сухих растворимых веществ в
экстрагенте может и не быть равной нулю, что будет означать
существование остаточного значения массы.

Отметим, что математическое моделирование позволяет расширить масштабы
процессов экстракции путем определения соответствующих коэффициентов,
зависящих от экспериментальных данных.

Аналогично и для других растений.

\emph{{\bfseries Финансирловпние:}} Данное исследование финансировалось
Комитетом науки Министерства науки и высшего образования Республики
Казахстан (грант BR 24992761)».

При написании статьи при переводе на иностранные языки и проверки
орфографии, стилистики и пунктуации использовались данные ИИ --
ChatGPT-4o {[}12{]}.

{\bfseries Литература}

1. Handa S.S. et al. Extraction Technologies for Medicinal and Aromatic
Plants // Trieste: International Centre for Science and High Technology.
- 2008. - 263 p.

2. Dias M.I., Barros L., Dueñas M., Pereira E., Carvalho A.M., Alves
R.C., Oliveira M.B.P.P., Santos-Buelga C., Ferreira I.C.F.R. Chemical
composition of wild and commercial Achillea millefolium L. and
bioactivity of the methanolic extract, infusion and decoction // Food
Chemistry. - 2013. - Vol.141 (4). - P.4152- 4160.
DOI~\href{https://doi.org/10.1016/j.foodchem.2013.07.018}{10.1016/j.foodchem.2013.07.018}.

3. Idoudi S., Tourrette A., Bouajila J., Romdhane M., Elfalleh W. The
genus \emph{Polygonum}: An updated comprehensive review of its
ethnomedicinal, phytochemical, pharmacological activities, toxicology,
and phytopharmaceutical formulation // Heliyon. - 2024. - Vol.10, No.
8. - e28947.

DOI~\href{https://doi.org/10.1016/j.heliyon.2024.e28947}{10.1016/j.heliyon.2024.e28947}.

4. Dong Y., Tang D., Zhang, N., Li, Y., Zhang, C., Li, L., Li, M.
Phytochemicals and biological studies of plants in genus Hedysarum //
Chemistry Central Journal. - 2013. - Vol.7(124). - P.~124. DOI
\href{https://doi.org/10.1186/1752-153X-7-124}{10.1186/1752-153X-7-124}.

5. Parham S., Kharazi A.Z., Bakhsheshi-Rad H.R., Nur H., Ismail A.F.,
Sharif S., Krishna~S.R., Berto F. Antioxidant, Antimicrobial and
Antiviral Properties of Herbal Materials // Antioxidants (Basel). -
2020. - Vol.9(12). - P.1309. DOI
\href{https://doi.org/10.3390/antiox9121309}{10.3390/antiox9121309}

6. Wu W.-L., Tan Z.-Q., Wu G.-J., Yuan L., Zhu W.-L., Bao Y.-L., Pan
L.-Y., Yang Y.-J., Zheng J.-X. Deacidification of crude low-calorie
cocoa butter with liquid--liquid extraction and strong-base anion
exchange resin // Separation and Purification Technology. - 2013. - Vol.
102. - P.~163--172. DOI
\href{https://doi.org/10.1016/j.seppur.2012.10.014}{10.1016/j.seppur.2012.10.014}.

7. Букин А.А., Беляев П.С., Однолько В.Г., Ткач Л.И., Щербаков С.А.
Математическая модель массопереноса при многоступенчатой экстракции из
растительного сырья сжиженным диоксидом углерода // Известия вузов.
Пищевая технология. - 2011. - № 2--3.- С.69--70.

8. Василенко В.В., Кошевой Е.П., Косачев В.С. Математическая модель
массопереноса при экстракции в бидисперсном адсорбирующем поровом объеме
частиц масличного материала // Известия вузов. Пищевая технология. -
2007. - № 2. - С.47- 66.

9. Lemoni, Z.; Kalantzi, S.; Lymperopoulou, T.; Tzani, A.; Stavropoulos,
G.; Detsi, A.; Mamma, D.Kinetic Modeling and Biological Activities of
Rosa canina L.Pseudo-Fruit Extracts Obtained via

Enzyme-Assisted Extraction// Antioxidants. -2025. - Vol.14(5):558.
\href{https://doi.org/10.3390/antiox14050558}{DOI
10.3390/antiox14050558}.

10. Reverchon E., De Marco I. Supercritical fluid extraction and
fractionation of natural matter // The Journal of Supercritical Fluids.
- 2006. - Vol.38(2).- P.146-166.
\href{https://doi.org/10.1016/j.supflu.2006.03.020}{DOI~10.1016/j.supflu.2006.03.020}.

11. Ramaswamy H.S., Marcotte M. Food Processing: Principles and
Applications/ Boca Raton: CRC Press. -2006. - 482 p. ISBN 9781587160080,
9780429204791.12. OpenAI. ChatGPT (GPT-4o) {[}Large language model{]}.
\url{https://chat.openai.com}. --Дата обращения: 08.06.2025.

{\bfseries References}

1. Handa S.S. et al. Extraction Technologies for Medicinal and Aromatic
Plants // Trieste: International Centre for Science and High Technology.
- 2008. - 263 p.

2. Dias M.I., Barros L., Dueñas M., Pereira E., Carvalho A.M., Alves
R.C., Oliveira M.B.P.P., Santos-Buelga C., Ferreira I.C.F.R. Chemical
composition of wild and commercial Achillea millefolium L. and
bioactivity of the methanolic extract, infusion and decoction // Food
Chemistry. - 2013. - Vol.141 (4). - P.4152- 4160.
DOI~\href{https://doi.org/10.1016/j.foodchem.2013.07.018}{10.1016/j.foodchem.2013.07.018}.

3. Idoudi S., Tourrette A., Bouajila J., Romdhane M., Elfalleh W. The
genus \emph{Polygonum}: An updated comprehensive review of its
ethnomedicinal, phytochemical, pharmacological activities, toxicology,
and phytopharmaceutical formulation // Heliyon. - 2024. - Vol.10, No.
8. - e28947.
DOI~\href{https://doi.org/10.1016/j.heliyon.2024.e28947}{10.1016/j.heliyon.2024.e28947}.

4. Dong Y., Tang D., Zhang, N., Li, Y., Zhang, C., Li, L., Li, M.
Phytochemicals and biological studies of plants in genus
\emph{Hedysarum} // Chemistry Central Journal. - 2013. - Vol.7(124). -
P.~124. DOI
\href{https://doi.org/10.1186/1752-153X-7-124}{10.1186/1752-153X-7-124}.

5. Parham S., Kharazi A.Z., Bakhsheshi-Rad H.R., Nur H., Ismail A.F.,
Sharif S., Krishna~S.R., Berto F. Antioxidant, Antimicrobial and
Antiviral Properties of Herbal Materials // Antioxidants (Basel). -
2020. - Vol.9(12). - P.1309. DOI
\href{https://doi.org/10.3390/antiox9121309}{10.3390/antiox9121309}.

6. Wu W.-L., Tan Z.-Q., Wu G.-J., Yuan L., Zhu W.-L., Bao Y.-L., Pan
L.-Y., Yang Y.-J., Zheng J.-X. Deacidification of crude low-calorie
cocoa butter with liquid--liquid extraction and strong-base anion
exchange resin // Separation and Purification Technology. - 2013. - Vol.
102. - P.~163--172. DOI
\href{https://doi.org/10.1016/j.seppur.2012.10.014}{10.1016/j.seppur.2012.10.014}.

7. Bukin A.A., Beljaev P.S., Odnol' ko V.G., Tkach L.I.,
Shherbakov S.A. Matematicheskaja model'{} massoperenosa
pri mnogostupenchatoj jekstrakcii iz rastitel' nogo
syr' ja szhizhennym dioksidom ugleroda // Izvestija
vuzov. Pishhevaja tehnologija. - 2011. - № 2-3: 6970.{[}in Russian{]}

8. Vasilenko V.V., Koshevoj E.P., Kosachev V.S. Matematicheskaja
model'{} massoperenosa pri jekstrakcii v bidispersnom
adsorbirujushhem porovom ob\#eme chastic maslichnogo materiala //
Izvestija vuzov. Pishhevaja tehnologija. - 2007. - № 2. - S.47-66.
{[}in Russian{]}

9. Lemoni, Z.; Kalantzi, S.; Lymperopoulou, T.; Tzani, A.; Stavropoulos,
G.; Detsi, A.; Mamma, D.Kinetic Modeling and Biological Activities of
Rosa canina L.Pseudo-Fruit Extracts Obtained via

Enzyme-Assisted Extraction// Antioxidants. -2025. --Vol.14(5):558.
\href{https://doi.org/10.3390/antiox14050558}{DOI
10.3390/antiox14050558}.

10. Reverchon E., De Marco I. Supercritical fluid extraction and
fractionation of natural matter // The Journal of Supercritical Fluids.
- 2006. - Vol.38(2).- P.146-166.
\href{https://doi.org/10.1016/j.supflu.2006.03.020}{DOI~10.1016/j.supflu.2006.03.020}.

11. Ramaswamy H.S., Marcotte M. Food Processing: Principles and
Applications/ Boca Raton: CRC Press. -2006. -- 482 p. ISBN:
9781587160080, 9780429204791.

12. OpenAI. ChatGPT (GPT-4o) {[}Large language model{]}.
\url{https://chat.openai.com}. --Date of access: 08.06.2025.

\emph{{\bfseries Cведения об авторах}}

Сулейменов К.М. - кандидат физико-математических наук, PhD, Евразийский
национальный университет им. Л.Н. Гумилева, Астана, Казахстан;
Общественный фонд «TEMS-Technology, Education and Mathematical Science»,
Астана, Казахстан, e-mail:
kenessarymath@gmail.com;

Мамытбекова Г.K. - магистр ветеринарных наук, Казахский университет
технологии и бизнеса им. К. Кулажанова, Астана, Казахстан, e-mail:
gulnur4284@mail.ru;

Нұрымов Ж.Ж.- магистр химических наук, Казахский университет технологии
и бизнеса им. К. Кулажанова, Астана, Казахстан, e-mail:
njd-jainar@mail.ru;

Сатенов К. Г. - кандидат химических наук, Казахский университет
технологии и бизнеса им. К. Кулажанова, ТОО «КМГ Инжиниринг», Астана,
Казахстан, e-mail:
K.Satenov@kmge.kz;

Сүлеймен Е.М. - кандидат химических наук, PhD, Казахский университет
технологии и бизнеса им. К. Кулажанова, Астана, Казахстан, ТОО «КМГ
Инжиниринг», Астана, Казахстан, Кокшетауский университет им. Ш.
Уалиханова, Кокшетау, Казахстан, e-mail:
Syerlan75@yandex.kz.

\emph{{\bfseries Information about the authors}}

Suleimenov K.M.- Candidate of Physics and Mathematics Sciences,PhD, L.N.
Gumilyov Eurasian National University, Astana, Kazakhstan, Public
Foundation "TEMS-Technology, Education and Mathematical Science",
Astana, Kazakhstan, e-mail:
kenessarymath@gmail.com;

Mamytbekova G.K. - Master of Veterinary Sciences, Kazakh
University of Technology and Business named after K. Kulajanov, Astana,
Kazakhstan, e-mail:
gulnur4284@mail.ru;

Nurymov Zh. Zh. - Master of Chemical Sciences, Kazakh University of
Technology and Business named after K. Kulajanov, Astana, Kazakhstan,
e-mail:
njd-jainar@mail.ru;

Satenov K.G. - candidate of chemical sciences, K. Kulazhanov Kazakh
University of Technology and Business, KMG Engineering'' LLP, Astana,
Kazakhstan, e-mail:
K.Satenov@kmge.kz;

Suleimen Ye.M. - candidate of Chemical Sciences, PhD, K. Kulazhanov
Kazakh University of Technology and Business, Astana, Kazakhstan, KMG
Engineering'' LLP, Astana, Kazakhstan, Sh. Ualikhanov Kokshetau
University, Kazakhstan, e-mail:
Syerlan75@yandex.kz.\