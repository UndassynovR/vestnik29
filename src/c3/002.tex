\id{IRSTI 31.15.15}{}

{\bfseries MOLECULAR-LEVEL ASSESSMENT OF THE CORROSION INHIBITION
PROPERTIES OF ONO-TYPE SCHIFF BASES USING DFT CALCULATIONS AND MOLECULAR
DYNAMICS SIMULATIONS}

{\bfseries R.Vardanyan}\fig{c3/image1}{}{\bfseries ,
E.Kozhevnikova}\fig{c3/image1}{}{\bfseries ,
N.Akatyev}\fig{c3/image1}{}\envelope 

\emph{M.Utemisov West Kazakhstan University, Uralsk, Kazakhstan}

\envelope Corresponding author:
nikolay.akatyev@wku.edu.kz

This study presents a molecular-level evaluation of two ONO-type Schiff
bases using density functional theory (DFT) and molecular dynamics (MD)
simulations to elucidate their anticorrosion performance. The Schiff
bases, synthesized from \emph{o}-aminophenol with
\emph{o}-hydroxyacetophenone (SB1) and
3,5-di-\emph{tert}-butylsalicylaldehyde (SB2), were investigated in
terms of their electronic structure, reactivity descriptors, and
adsorption behavior on the Fe(110) surface. DFT calculations at the
B3LYP/6-311+G(d,p) level in an aqueous phase revealed key differences in
molecular properties such as frontier orbital energies, dipole moment,
energy gap, and charge distribution. Mulliken population analysis and
Fukui functions identified the primary reactive centers, emphasizing the
role of heteroatoms in surface adsorption interactions. Global
reactivity descriptors were also evaluated to compare the
inhibitors'{} electron-donating and accepting
capabilities. Complementary MD simulation provided insight into the
effect of the geometry of molecules and substituents on the adsorption
process. SB1 showed a more favorable space orientation and stronger
surface interaction, while steric hindrance from bulky \emph{tert}-butyl
groups in SB2 limited its adsorption efficacy. This integrated
computational approach offers a detailed understanding of
structure--activity relationships in corrosion inhibitors and supports
the rational design of next-generation materials for metal protection.

{\bfseries Keywords:} density functional theory (DFT), quantum chemical
calculations, molecular dynamics (MD) simulation, organic corrosion
inhibitors, Schiff bases, quantum chemical descriptors, adsorption.

{\bfseries ТФТ-ЕСЕПТЕУЛЕРІ МЕН МОЛЕКУЛАЛЫҚ ДИНАМИКАЛЫҚ ҮЛГІЛЕУ АРҚЫЛЫ
ONO-ТИПТІ ШИФФ НЕГІЗДЕРІНІҢ КОРРОЗИЯҒА ҚАРСЫ ҚАСИЕТТЕРІН МОЛЕКУЛАЛЫҚ
ДЕҢГЕЙДЕ БАҒАЛАУ}

{\bfseries Р. Варданян, Е. Кожевникова, Н. Акатьев\envelope }

\emph{М.Өтемісов атындағы Батыс Қазақстан университеті, Орал,
Қазақстан,}

e-mail:
nikolay.akatyev@wku.edu.kz

Осы зерттеуде коррозияға қарсы қасиеттерін анықтау мақсатында ONO-тиіпті
екі Шифф негізіне тығыздық функционалы теориясы (ТФТ) және молекулалық
динамика (МД) әдістері қолданылып, молекулалық деңгейде бағалау
жүргізілді. \emph{О}-аминфенол мен \emph{о}-гидроксиацетофеноннан (SB1),
сондай-ақ 3,5-ди-\emph{трет}-бутилсалицил альдегидтен (SB2) синтезделген
Шифф негіздерінің электрондық құрылымы, реактивтілік сипаттамалары және
Fe(110) бетіне адсорбциялану тәртібі зерттелді. B3LYP/6-311+G(d,p)
деңгейінде су фазасында жүргізілген DFT есептеулері молекулалық
қасиеттерде - шекаралық орбитальдар энергиясы, дипольдік момент,
энергетикалық аралық және зарядтардың таралуы секілді - негізгі
айырмашылықтарды көрсетті. Малликен бойынша зарядтарды тарату және Фукуи
функцияларын талдау реакцияға бейім алғашқы орталықтарды анықтап, беткі
адсорбциялық әрекеттесулердегі гетероатомдардың рөлін айқындады.
Ингибиторлардың электрон беру және қабылдау қабілеттерін салыстыру үшін
жаһандық реактивтілік дескрипторлары да бағаланды. Қосымша MD модельдеуі
молекулалық геометрия мен орынбасар топтардың адсорбция процесіне әсерін
нақтылауға мүмкіндік берді. SB1 молекуласы метал бетіне қолайлырақ
орналасып, күштірек әрекеттесу көрсетсе, SB2 құрамындағы көлемді
трет-бутил топтары стереоқиындық тудырып, оның адсорбциялық тиімділігін
шектеді. Мұндай кешенді есептік әдіс коррозия ингибиторларының құрылымы
мен белсенділігі арасындағы өзара байланысты тереңірек түсінуге жол
ашады және металдарды қорғауға арналған жаңа буындағы материалдарды
ұтымды жобалауға ықпал етеді.

{\bfseries Түйін сөздер:} тығыздық функционалы теориясы (ТФТ),
кванттық-химиялық есептеулер, молекулалық-динамикалық (МД) модельдеу,
органикалық коррозия ингибиторлары, Шифф негіздері, кванттық-химиялық
дескрипторлар, адсорбция.

{\bfseries ОЦЕНКА АНТИКОРРОЗИОННЫХ СВОЙСТВ ОСНОВАНИЙ ШИФФА ONO-ТИПА НА
МОЛЕКУЛЯРНОМ УРОВНЕ С ИСПОЛЬЗОВАНИЕМ DFT-РАСЧЕТОВ И
МОЛЕКУЛЯРНО-ДИНАМИЧЕСКОГО МОДЕЛИРОВАНИЯ}

{\bfseries Р. Варданян, Е. Кожевникова, Н. Акатьев\envelope }

\emph{Западно-Казахстанский университет им. М.Утемисова, Уральск,
Казахстан,}

e-mail:
nikolay.akatyev@wku.edu.kz

В настоящем исследовании на молекулярном уровне проведена оценка
антикоррозионных свойств двух оснований Шиффа ONO-типа с использованием
методов теории функционала плотности (ТФП) и молекулярной динамики (МД).
Основания Шиффа, синтезированные из \emph{о}-аминофенола,
\emph{о}-гидроксиацетофенона (SB1) и
3,5-ди-\emph{трет}-бутилсалицилового альдегида (SB2), были исследованы с
точки зрения их электронной структуры, характеристик реакционной
способности и поведения при адсорбции на поверхности Fe(110). Расчеты
методом DFT на уровне B3LYP/6-311+G(d,p) в водной фазе выявили ключевые
различия в молекулярных свойствах, таких как энергии граничных
орбиталей, дипольный момент, энергетическая щель и распределение заряда.
Анализ распределения зарядов по Малликену и функций Фукуи выявили
первичные реакционные центры, подчеркнув роль гетероатомов в
поверхностных адсорбционных взаимодействиях. Для сравнения способности
ингибиторов отдавать и принимать электроны также были оценены глобальные
дескрипторы реакционной способности. Дополнительное МД-моделирование
позволило получить представление о влиянии геометрии молекул и
заместителей на процесс адсорбции. SB1 показал более благоприятную
пространственную ориентацию и более сильное поверхностное
взаимодействие, в то время как стерические препятствия из-за объёмных
\emph{трет}-бутильных групп в SB2 ограничивали эффективность его
адсорбции. Такой комплексный вычислительный подход обеспечивает
детальное понимание взаимосвязи структуры и активности ингибиторов
коррозии и способствует рациональному проектированию материалов нового
поколения для защиты металлов.

{\bfseries Ключевые слова:} теория функционала плотности (ТФП),
квантово-химические расчеты, молекулярно-динамическое (МД)
моделирование, органические ингибиторы коррозии, основания Шиффа,
квантово-химические дескрипторы, адсорбция.

{\bfseries Introduction.} Corrosion is a pervasive issue that leads to
significant material degradation and economic losses across various
industries. Organic corrosion inhibitors have emerged as an effective
and versatile solution to mitigate corrosion, particularly in aggressive
environments such as acidic media. These inhibitors function by
adsorbing onto the metal surface, forming a protective barrier that
reduces the interaction between the metal and the corrosive environment.
Organic inhibitors, often containing heteroatoms like nitrogen, oxygen,
and sulfur, as well as π-electron systems, exhibit strong adsorption due
to their ability to donate and accept electrons {[}1{]}. Their
efficiency, environmental compatibility, and tunable molecular
structures make them a focus of extensive research for sustainable
corrosion protection. These inhibitors are widely used in acidic
environments for applications such as descaling, acid pickling, and
oil-well protection, demonstrating high inhibition efficiency {[}2{]}.

Schiff bases offer several advantages over other organic corrosion
inhibitors. Schiff bases contain imine (−C=N−) groups and heteroatoms
(nitrogen, oxygen), which provide multiple active sites for adsorption
onto metal surfaces, enhancing their inhibition efficiency {[}3{]}.
Their molecular structure allows strong interactions with metal
surfaces, forming stable protective films that effectively block
corrosion processes {[}4{]}. In addition, Schiff bases can be easily
synthesized and modified to optimize their properties for specific
environments, making them adaptable to various industrial
applications{[}5{]}.

The mechanisms of organic inhibitors, including adsorption and electron
transfer, are well-studied using computational and experimental
approaches {[}6{]}. Quantum chemical calculations and molecular dynamics
simulations, play a crucial role in the research and development of
modern corrosion inhibitors. Quantum chemical methods, such as Density
Functional Theory (DFT), are used to calculate molecular reactivity
descriptors (e.g., HOMO-LUMO energies, dipole moments, Fukui functions)
to predict the adsorption behavior and inhibition efficiency of
inhibitors on metal surfaces {[}7{]}. MD simulations provide insights
into the dynamic interactions between inhibitor molecules and metal
surfaces, including adsorption configurations, binding energies, and
surface coverage. This helps evaluate the stability and effectiveness of
inhibitors under realistic conditions {[}8{]}. These methods enable the
design and optimization of inhibitors by identifying key structural
features. These methods also help to explain the experimental results.

In this work, the inhibitory properties of two ONO-type Schiff bases
(Figure 1) synthesized by condensation of \emph{o}-aminophenol and
\emph{o}-hydroxyacetophenone (SB1), as well as \emph{o}-aminophenol and
3,5-di-\emph{tert}-butylsalicylaldehyde (SB2), were investigated using
DFT calculations and MD simulation.



{\bfseries Fig.1 - Chemical structure with atomic enumeration of
investigated ONO-type Schiff bases}

{\bfseries Materials and methods.} All quantum chemical calculations were
performed on a desktop PC with Windows 11, a 12th generation Intel(R)
Core (TM) i7-12700H 2.30 GHz, 16GB RAM) with GAMESS software {[}9{]}.
Initially, full geometry optimization was achieved using the Molecular
Mechanics (MM+) force field. The results from MM+ were further selected
as input and re-optimized using semi-empirical PM3 to obtain the
equilibrium geometry. Final geometry optimization and quantum chemical
parameters were obtained by density functional theory (DFT)
calculations. To calculate optimized geometrical structure, atomic
charges, and energies, the Lee-Yang-Parr Becke's three-parameter hybrid
functional method (B3LYP) with the split-valence double-zeta polarized
basis set 6-311+G(d,p) was used. All DFT calculations were done in the
aqueous phase because it is well-known that the electrochemical
corrosion always appears in an aqueous medium. The solvent
(H\tsb{2}O) was incorporated via the Conductor-like Screening
Model (COSMO) {[}10{]}.

The following quantum chemical indices, describing global reactivity
were considered: the energy of the highest occupied molecular orbital
(E\emph{\tsb{HOMO}}), the energy of the lowest unoccupied
molecular orbital (E\emph{\tsb{LUMO}}), energy gap
(ΔE\emph{\tsb{gap}}), the ionization potential (IP), the
electron affinity (EA), electronegativity (χ), global hardness (η),
softness (σ), electrophilicity (ω) and nucleophilicity (ε) indexes,
back-donation energy (E\emph{\tsb{b-d}}), total negative
charge (TNC, TNC/\emph{n}, where \emph{n} - total number of atoms in
molecule), and fraction of transferred electrons (ΔN) were considered in
accordance with the following equations {[}11{]}:

ΔE\emph{\tsb{gap}} = E\emph{\tsb{LUMO}} -
E\emph{\tsb{HOMO}} (1)

IP = - E\emph{\tsb{HOMO}} (2)

EA = - E\emph{\tsb{LUMO}} (3)

(4)

(5)

(6)

(7)

(8)

(9)

(10)

where χ\emph{\tsb{Fe}} and χ\emph{\tsb{inh}} - the
absolute electronegativity of the iron atom and the inhibitor molecule,
respectively; η\emph{\tsb{Fe}} and η\emph{\tsb{inh}}
the absolute hardness of the iron atom and the inhibitor molecule,
respectively. The electronegativity of iron is 7 eV, and the total
hardness is 0 eV.

According to the Koopman's theorem, the HOMO energy is related to IP
while the LUMO energy is related to EA. Electronegativity (χ) and global
hardness (η) were evaluated based on the finite difference approximation
as linear combinations of the calculated IP and EA. The global softness
(σ) is the reciprocal of the global hardness. The global
electrophilicity index (ω) describes the stabilization of a molecule
after the acquisition of an additional number of electrons. The
nucleophilicity index (ε) is the reciprocal of the global
electrophilicity index {[}12{]}.

\emph{Molecular dynamics simulation.} The intensity of the interactions
between the examined inhibitor and metal surface was determined using
molecular dynamics (MD) simulations. MD simulation is a modern
computational technique crucial for theoretical studies of the
interaction between inhibitors and metal surfaces.

The MD simulations were carried out by using the GROMACS 2020.4 software
{[}13{]}. Adsorption energy was calculated by Equation (11):

E\emph{\tsb{adsorption}} = E\emph{\tsb{total}} -
(E\emph{\tsb{Fe(110)}} + E\emph{\tsb{inhibitor}})
(11)

where E\emph{\tsb{adsorption}} is the adsorption energy (or
interaction energy) corresponds to the amount of energy released or
absorbed as 1 mole of adsorbate molecules is adsorbed on the adsorbent,
E\emph{\tsb{total}} is the total energy of the whole system
(surface Fe(110) + inhibitor molecule), E\emph{\tsb{Fe(110)}}
is the energy of the metal and E\emph{\tsb{inhibitor}} is the
energy of the inhibitor molecule. The binding energy is regarded as the
reciprocal of the adsorption energy of the inhibitor molecule as follows
(12):

E\emph{\tsb{binding}} = -E\emph{\tsb{adsorption}}
(12)

where E\emph{\tsb{binding}} is the binding energy.

The Fe (110) metal structure was chosen due to its lowest surface
energy, thermodynamic stability and the largest area of the Fe crystal
{[}14{]}.

{\bfseries Results and discussion.} The inhibitory properties of Schiff
bases are investigated using quantum chemical methods, focusing on the
electronic structure and reactivity descriptors. Density Functional
Theory (DFT) calculations provided insights into molecular orbitals,
charge distribution, and adsorption behavior on metal surfaces to
evaluate the mechanisms underlying their corrosion inhibition
performance.

\emph{Frontier molecular orbitals (FMO) energies and molecular
electrostatic potential (MEP).} The electronic properties of Schiff
bases, including their reactivity and adsorption behavior on metal
surfaces, can be effectively analyzed using Frontier Molecular Orbitals
(FMOs) and Molecular Electrostatic Potential (MEP). Optimized geometry,
HOMO, LUMO energies and density distribution, energy gap
(ΔE\emph{\tsb{gap}}), and MEP for SB1 and SB2 molecules
calculated using DFT/B3LYP/6-311+G(d,p) method are presented in Figure
2. The positively charged lobe is indicated by the blue color, and the
negatively charged lobe is indicated by the red color.

\fig{c3/image10}{}

{\bfseries Fig.2 - HOMO, LUMO energies and density distribution, energy gap
(ΔE\emph{\tsb{gap}}), and MEP for SB1 (A) and SB2 (B)
mole­cules calculated using DFT/B3LYP/6-311+G(d,p) method}

As evident from the optimized structures, both Schiff bases belong to
the C\tsb{1} point group and exhibit non-planar geometries.
This conformational behavior can be attributed to the free rotation
around C*C σ-bonds and the steric repulsion between similarly charged
functional groups, such as -OH.

The MEP maps reveal electron-rich regions (depicted in red) and
electron-deficient regions (in blue). Both SB1 and SB2 exhibit areas of
negative electrostatic potential localized around oxygen and nitrogen
atoms, suggesting these sites as potential adsorption centers on the
metal surface. Notably, SB1 displays more intense and spatially confined
electron-rich zones, indicating stronger and more specific interaction
sites for bonding with metal. The high electron density around the donor
atoms (O and N) enhances the likelihood of chemisorption, which is one
of the key factors for the mechanism in the inhibition of metal
corrosion.

In the SB1 molecule, the HOMO is primarily localized on the phenolic
ring and the azomethine (-CH=N) group, particularly around the oxygen
and nitrogen atoms. This indicates that these regions are electron-rich
and capable of donating electrons to the vacant \emph{d}-orbitals of
metal atoms during adsorption. The LUMO is also distributed over the
azomethine group and partially over the adjacent aromatic ring,
suggesting that these sites are favorable for accepting back-donated
electrons from the metal surface, thereby enhancing adsorption
stability. In contrast, in the SB2 molecule, the HOMO is delocalized
over the aromatic ring but exhibits lower electron density on the
heteroatoms compared to SB1, implying a reduced ability to donate
electrons to the metal. The LUMO is primarily located on the azomethine
moiety and parts of the aromatic system. Its delocalized nature suggests
a reduced capacity for back-donation, potentially lowering the
efficiency of adsorption. Therefore, the co-localization of the HOMO and
LUMO near the donor atoms (N and O) in SB1 indicates a strong potential
for bidirectional interactions with the metal surface, involving both
electron donation and back-donation. In contrast, the FMOs in SB2 are
more dispersed and less concentrated near key donor sites, suggesting
weaker and less specific interactions with the metal surface.

The HOMO energy reflects a molecule's ability to donate electrons to the
metal surface. SB1 exhibits a higher HOMO energy compared to SB2,
indicating a stronger electron-donating capability, which promotes
adsorption onto the metal surface and facilitates the formation of a
protective film. The LUMO energy is associated with the molecule's
capacity to accept electrons. SB2 possesses a lower LUMO energy,
suggesting a greater ability to accept electrons, which may enhance
back-donation interactions from the metal. Furthermore, SB1 displays a
smaller energy gap (ΔE\emph{\tsb{gap}}) than SB2, indicating
higher chemical reactivity and a stronger potential for interaction with
the metal surface.

The obtained results indicate that SB1 is more chemically reactive and
has stronger inhibition potential than SB2, owing to its higher HOMO
energy (enhanced electron-donating ability), lower
ΔE\emph{\tsb{gap}} (greater chemical reactivity), and more
favorable MEP distribution (stronger adsorption affinity toward the
metal surface).

\emph{Mulliken atomic charge distribution (MACD) and Fukui functions
analysis.} Mulliken charge distribution analysis provides essential
information on the electronic structure of corrosion inhibitor molecules
by estimating the partial atomic charges {[}15{]}. Atoms with higher
negative charges, typically heteroatoms such as nitrogen, oxygen, or
sulfur, are identified as potential adsorption centers capable of
donating electrons to the vacant orbitals of metal atoms. Additionally,
the charge distribution influences total molecular polarity and
orientation during adsorption on metal surfaces, and consequently is a
valuable tool for understanding and predicting the corrosion inhibition
performance of organic molecules.

In turn, Fukui functions are mathematical descriptors derived from
Density Functional Theory (DFT) that quantify the reactivity of specific
atomic sites in a molecule {[}16{]}. They indicate how the electron
density of a molecule changes upon gaining or losing electrons, making
them useful for identifying the most reactive sites in a molecule
potentially suitable for nucleophilic, electrophilic or radical attack.
Therefore, Fukui functions are integral to understanding the electronic
properties and reactivity of corrosion inhibitors at the molecular
level.

The condensed Fukui functions are calculated using the finite-difference
approximation as follows:


(For nucleophilic attack)


(For electrophilic attack)


(For radical attack)

The results of the calculation of Mulliken atomic charges and the values
of the Fukui functions for ONO-type Schiff bases are presented in Table
1. For simplicity, the charges and Fukui functions of the hydrogen (H)
atoms were removed.

{\bfseries Table 1 - Mulliken atomic charges and Fukui functions calculated
using DFT/B3LYP/6-311+G(d,p) for ONO-type Schiff bases}

%% \begin{longtable}[]{@{}
%%   >{\centering\arraybackslash}p{(\linewidth - 14\tabcolsep) * \real{0.1033}}
%%   >{\centering\arraybackslash}p{(\linewidth - 14\tabcolsep) * \real{0.1493}}
%%   >{\centering\arraybackslash}p{(\linewidth - 14\tabcolsep) * \real{0.1493}}
%%   >{\centering\arraybackslash}p{(\linewidth - 14\tabcolsep) * \real{0.1493}}
%%   >{\centering\arraybackslash}p{(\linewidth - 14\tabcolsep) * \real{0.1493}}
%%   >{\centering\arraybackslash}p{(\linewidth - 14\tabcolsep) * \real{0.1493}}
%%   >{\centering\arraybackslash}p{(\linewidth - 14\tabcolsep) * \real{0.1493}}
%%   >{\centering\arraybackslash}p{(\linewidth - 14\tabcolsep) * \real{0.0011}}@{}}
%% \toprule\noalign{}
%% \begin{minipage}[b]{\linewidth}\centering
%% Atoms*
%% \end{minipage} & \begin{minipage}[b]{\linewidth}\centering
%% P\emph{\tsb{k}}(N)
%% \end{minipage} & \begin{minipage}[b]{\linewidth}\centering
%% P\emph{\tsb{k}}(N+1)
%% \end{minipage} & \begin{minipage}[b]{\linewidth}\centering
%% P\emph{\tsb{k}}(N-1)
%% \end{minipage} & \begin{minipage}[b]{\linewidth}\centering
%% 
%% \end{minipage} & \begin{minipage}[b]{\linewidth}\centering
%% 
%% \end{minipage} & \begin{minipage}[b]{\linewidth}\centering
%% 
%% \end{minipage} & \begin{minipage}[b]{\linewidth}\centering
%% \end{minipage} \\
%% \midrule\noalign{}
%% \endhead
%% \bottomrule\noalign{}
%% \endlastfoot
%% \multicolumn{8}{@{}>{\centering\arraybackslash}p{(\linewidth - 14\tabcolsep) * \real{1.0000} + 14\tabcolsep}@{}}{%
%% {\bfseries SB1}} \\
%% C(1) & 0.2454 & 0.3213 & -0.3244 & 0.0759 & -0.0790 & -0.0015 & \\
%% C(2) & 0.0902 & -0.1359 & -0.1046 & -0.0457 & 0.0144 & -0.0156 & \\
%% C(3) & -0.0669 & -0.0746 & 0.0116 & -0.0077 & -0.0785 & -0.0431 & \\
%% C(4) & -0.1032 & -0.0974 & -0.0842 & 0.0058 & -0.0190 & -0.0066 & \\
%% C(5) & -0.0688 & -0.1289 & -0.0572 & -0.0601 & -0.0116 & -0.0358 & \\
%% C(6) & -0.1900 & -0.0480 & -0.2005 & {\bfseries 0.1420} & 0.0105 & 0.0762
%% & \\
%% C(7) & 0.2863 & 0.1802 & 0.3241 & -0.1061 & -0.0378 & -0.0719 & \\
%% N(8) & -0.5385 & -0.5096 & -0.5697 & 0.0289 & {\bfseries 0.0312} & 0.0300
%% & \\
%% C(9) & 0.1197 & 0.1951 & 0.2121 & 0.0754 & -0.0924 & -0.0085 & \\
%% C(10) & -0.0682 & -0.1382 & -0.0699 & -0.0700 & 0.0017 & -0.0341 & \\
%% C(11) & -0.0977 & -0.0827 & -0.0602 & 0.0150 & -0.0375 & -0.0112 & \\
%% C(12) & -0.0844 & -0.1067 & -0.0689 & -0.0223 & -0.0155 & -0.0189 & \\
%% C(13) & -0.1040 & -0.1654 & -0.0334 & -0.0614 & -0.0706 & -0.0660 & \\
%% C(14) & 0.1784 & 0.2870 & 0.1939 & 0.1086 & -0.0155 & {\bfseries 0.0465}
%% & \\
%% O(15) & -0.3654 & -0.4101 & -0.3417 & -0.0447 & -0.0237 & -0.0342 & \\
%% O(16) & -0.3734 & -0.4759 & -0.2864 & -0.1025 & -0.0870 & -0.0947 & \\
%% C(17) & -0.3276 & -0.2665 & -0.3067 & 0.0611 & -0.0209 & 0.0201 & \\
%% \multicolumn{8}{@{}>{\centering\arraybackslash}p{(\linewidth - 14\tabcolsep) * \real{1.0000} + 14\tabcolsep}@{}}{%
%% {\bfseries SB2}} \\
%% C(1) & 0.1309 & 0.3687 & 0.2464 & {\bfseries 0.2378} & -0.1155 &
%% {\bfseries 0.1766} & \\
%% C(2) & 0.0683 & -0.0389 & -0.0519 & -0.1072 & 0.1202 & -0.1137 & \\
%% C(3) & 0.0701 & 0.0521 & -0.0446 & -0.0180 & 0.1147 & -0.0663 & \\
%% C(4) & 0.0437 & -0.0314 & -0.0373 & -0.0751 & 0.0810 & -0.0780 & \\
%% C(5) & -0.0425 & 0.0249 & -0.0694 & 0.0674 & 0.0269 & 0.0202 & \\
%% C(6) & -0.0881 & -0.1802 & -0.1207 & -0.0921 & 0.0326 & -0.0623 & \\
%% C(7) & 0.1598 & 0.2850 & 0.1061 & 0.1252 & 0.0537 & 0.0357 & \\
%% N(8) & -0.3201 & -0.5249 & -0.5653 & -0.2048 & {\bfseries 0.2462} & -0.2255
%% & \\
%% C(9) & 0.0426 & 0.1361 & 0.1816 & 0.0935 & -0.1390 & 0.1162 & \\
%% C(10) & -0.0075 & -0.0350 & -0.1255 & -0.0275 & 0.1180 & -0.0727 & \\
%% C(11) & -0.1335 & -0.1027 & -0.0999 & 0.0308 & -0.0336 & 0.0322 & \\
%% C(12) & -0.0566 & -0.0446 & -0.1326 & 0.0120 & 0.0760 & -0.0320 & \\
%% C(13) & -0.1342 & -0.0829 & -0.1334 & 0.0513 & -0.0008 & 0.0260 & \\
%% C(14) & 0.2204 & 0.2120 & 0.1496 & -0.0084 & 0.0708 & -0.0396 & \\
%% O(15) & -0.3689 & -0.3665 & -0.3835 & 0.0024 & 0.0146 & -0.0061 & \\
%% O(16) & -0.4093 & -0.2961 & -0.4622 & 0.1132 & 0.0529 & 0.0301 & \\
%% C(18) & -0.2399 & -0.2536 & -0.1980 & -0.0137 & -0.0419 & 0.0141 & \\
%% C(19) & -0.2875 & -0.2919 & -0.2502 & -0.0044 & -0.0373 & 0.0164 & \\
%% \end{longtable}

* - \emph{atomic enumeration} \emph{is in accordance with Fig.1.}

The Mulliken atomic charge distribution analysis shows that regions of
high electron density are located on the heteroatoms (nitrogen and
oxygen) of both molecules. Therefore, these atoms serve as potential
adsorption centres and contribute to the formation of a protective film
on the metal surface. In both molecules, the nitrogen atom of the imine
group (N8) has the highest electron density and is identified as the
primary electrophilic centre, as confirmed by the corresponding Fukui
function values.

Furthermore, the presence of \emph{tert}-butyl substituents increases
the susceptibility of nitrogen atoms to electrophilic attack, which is
attributed to the electron donor properties of \emph{tert}-butyl groups,
increasing the total electron density in the molecule. This effect also
makes the C(1) atom a potential center of radical attack.

Carbon atoms C(6) in SB1 and C(1) in SB2 were identified as potential
centers for nucleophilic attack in both molecules. The shift in the
nucleophilic reaction center is likewise attributed to the presence of
\emph{tert}-butyl groups. Acting as first-kind orientants, they increase
the electron density in the aromatic system, particularly at the
\emph{ortho}- and \emph{para}-positions. In the presence of a hydroxyl
group, this results in a significant enhancement of electron density at
the C(1) atom.

The combination of high electron density and favorable Fukui function
values at key heteroatoms strongly supports the effective adsorption of
the investigated inhibitor molecules onto the steel surface. This charge
distribution further underpins the ability of the Schiff bases to
interact with the steel surface through both electron donation and
acceptance mechanisms, thereby enhancing their corrosion inhibition
efficiency.

\emph{Molecular reactivity descriptors.} Molecular reactivity
descriptors are quantitative parameters that characterize the electronic
structure and chemical reactivity of molecules. In the study of
corrosion inhibitors, these descriptors are used to predict and
understand the interaction between inhibitor molecules and metal
surfaces. These descriptors are essential for designing and optimizing
corrosion inhibitor structures for high efficiency and stability.
Quantum chemical descriptors calculated for the investigated Schiff
bases molecules in aqueous phase are listed in Table 2.

{\bfseries Table 2 - Molecular reactivity descriptors calculated for
ONO-type Schiff bases calculated using DFT/B3LYP/6-311+G(d,p) in aqueous
phase}

%% \begin{longtable}[]{@{}
%%   >{\centering\arraybackslash}p{(\linewidth - 4\tabcolsep) * \real{0.6119}}
%%   >{\centering\arraybackslash}p{(\linewidth - 4\tabcolsep) * \real{0.1847}}
%%   >{\centering\arraybackslash}p{(\linewidth - 4\tabcolsep) * \real{0.2034}}@{}}
%% \toprule\noalign{}
%% \begin{minipage}[b]{\linewidth}\centering
%% {\bfseries Descriptor}
%% \end{minipage} & \begin{minipage}[b]{\linewidth}\centering
%% {\bfseries SB1}
%% \end{minipage} & \begin{minipage}[b]{\linewidth}\centering
%% {\bfseries SB2}
%% \end{minipage} \\
%% \midrule\noalign{}
%% \endhead
%% \bottomrule\noalign{}
%% \endlastfoot
%% Total energy (E\emph{\tsb{tot}}), a.u. & -746.25 & -1021.14 \\
%% Dipole moment (μ), D & 2.8142 & 1.7517 \\
%% Ionization energy (IP), eV & 5.6900 & 9.4940 \\
%% Electron affinity (EA), eV & 1.6480 & 4.2780 \\
%% Electronegativity (\emph{χ}), eV & 3.6690 & 6.8860 \\
%% Chemical hardness (\emph{η}), eV & 2.0210 & 2.6080 \\
%% Chemical softness (σ), eV & 0.4948 & 0.3834 \\
%% Electrophilicity index (ω), eV & 6.7308 & 23.7085 \\
%% Nucleophilicity index (\emph{ε}), eV & 0.1486 & 0.0422 \\
%% Back-donation energy (E\emph{\tsb{b-d}}, eV) & -0.5053 &
%% -0.6520 \\
%% Fraction of transferred electrons (Δ\emph{N}) & -0.8241 & -0.0219 \\
%% TNC & -2.4783 & -3.5265 \\
%% TNC/\emph{n} & -0.0826 & -0.0691 \\
%% Inhibition efficiency (in 0.5 mol·dm\tsp{-3}
%% H\tsb{2}SO\tsb{4}) {[}17{]}, \% & 86,55 & 56,05 \\
%% \end{longtable}

The calculated total energy values (E\emph{\tsb{tot}})
indicate greater stability of SB2 molecule. The dipole moment (μ)
reflects the polarity of the inhibitor molecule and provides insights
into its adsorption behavior. A higher dipole moment generally indicates
stronger molecular polarity, which can enhance the interaction between
the inhibitor and the metal surface through electrostatic forces.
Molecules with significant dipole moments tend to learn more effectively
on the metal surface, promoting the formation of a compact and stable
protective film. The calculation results indicate that SB1 has a higher
dipole moment and, accordingly, has more intense inhibitory properties,
which is consistent with experimental data.

Ionization energy (IE) and electron affinity (EA) are key descriptors of
the electronic properties of corrosion inhibitors. Ionization energy
reflects the ability of a molecule to donate electrons; a lower IE
indicates that the molecule can more easily donate electrons to the
vacant \emph{d}-orbitals of metal atoms, enhancing adsorption. Electron
affinity measures the tendency of a molecule to accept electrons.
Therefore, a higher EA suggests a stronger ability to accept electrons
from the metal surface. Together, these parameters describe the
molecule's ability to engage in donor - acceptor interactions with the
metal, stabilizing the adsorbed layer and improving corrosion inhibition
efficiency. The calculated values of the ionization energy indicate that
SB1 is a more effective inhibitor, which is also consistent with the
experimental results. The values of electron affinity indicate that the
most effective inhibitor should be SB2. Most likely, this result is due
to the presence of \emph{tert}-butyl groups, which make charges on
potential electrophilic centers more positive, thereby increasing
acceptor properties of the whole molecule.

Electronegativity (\emph{χ}) also plays a critical role in describing
the properties of corrosion inhibitors. Molecules with high
electronegativity tend to attract electrons, enabling them to form
stable interactions with the partially positive metal atoms. A higher
electronegativity value for SB2 molecule implies it' s
more effective interaction with the metal surface and therefore has
higher inhibitory properties. However, the lower inhibition efficiency
is likely because the large \emph{tert}-butyl groups create significant
sterical hindrance.

Chemical hardness (\emph{η}) and softness (σ) are also important
descriptors for understanding the reactivity of corrosion inhibitors. In
corrosion inhibition, soft molecules are often more effective because
their higher polarizability and greater reactivity enhance their ability
to adsorb onto the metal, form stable complexes, and protect the
surface. Thus, chemical softness is typically associated with higher
inhibition efficiency. A higher chemical softness value and a lower
chemical hardness value for SB1 clearly indicate its more effective
inhibitory properties.

Electrophilicity (ω) and nucleophilicity (\emph{ε}) indexes provide
insight into the chemical reactivity of corrosion inhibitors. High
nucleophilicity and low electrophilicity show enhanced adsorption of
inhibitor molecules on metal surfaces, and therefore they are critical
for predicting and describing the efficiency of corrosion inhibitors. As
can be seen, the values of both parameters also confirm the high
effectiveness of SB1 as a corrosion inhibitor.

Back-donation energy (E\emph{\tsb{b-d}}) describes the ability
of inhibitor molecules to accept electron density from the metal surface
into their LUMO. A negative back-donation energy indicates that, in
addition to donating electrons to the metal, the inhibitor can also
accept electrons from the metal' s surface. This two-way
electron transfer strengthens the adsorption interaction, leading to the
formation of a more stable protective film. The results of the
calculations indicate that both molecules, especially SB2, tend to
accept electrons from the metal. However, the sterical hindrance caused
by tert-butyl groups does not allow the orbitals to be at the distance
necessary for back-donation and thereby reduces its inhibitory
effectiveness.

The fraction of transferred electrons (ΔN) quantifies the extent of
electron transfer from a corrosion inhibitor molecule to the metal
surface. Generally, a higher ΔN value corresponds to a greater ability
of the molecule to stabilize on the metal surface through
donor--acceptor interactions, enhancing its corrosion inhibition
efficiency. A negative ΔN suggests that both inhibitor molecules can
accept electrons from the metal, which leads to the reduction of the
adsorption intensity. A more negative ΔN value calculated for SB1
clearly indicates its lower propensity to accept electrons from the
metal surface and, therefore, its higher inhibition efficiency.

The total negative charge (TNC) represents the sum of all negative
atomic charges within a molecule, providing an overall measure of its
electron-donating ability. A more negative TNC value indicates a greater
capacity for electron donation to the metal surface, promoting stronger
sorption and enhancing corrosion protection. The ratio TNC/\emph{n},
where \emph{n} is the number of atoms in the molecule, normalizes the
total negative charge relative to molecular size, allowing for more
accurate comparisons between inhibitors of different molecular weights.
Higher TNC/\emph{n} values suggest a more efficient distribution of
electron density, which correlates with improved adsorption strength and
inhibition efficiency. The values of these parameters for the studied
molecules indicate a more efficient distribution of electron density in
molecule SB1 and, consequently, its higher inhibition efficiency.

Thus, most of the calculated parameters clearly confirm the high
inhibition efficacy of SB1 compared to SB2. The insignificant
differences between the calculation results and the experimental data
are explained by the peculiarity of the structure of SB2 molecule, in
particular, the presence of two bulky \emph{tert}-butyl groups and their
steric effects.

\emph{Molecular dynamic simulation.} Molecular dynamics (MD) simulation
is used to study the dynamic behavior of molecules and their
interactions. MD simulations provide insights into inhibitor molecules
interaction with metal surfaces, including adsorption configurations,
binding energies, and the formation of protective films. MD simulations
help quantify the effectiveness of inhibitors by analyzing parameters
like surface coverage, interaction energy, and molecular orientation. By
simulating different molecular structures, MD makes it possible to
identify characteristics that increase adsorption and corrosion
resistance and thus help design more effective inhibitors. This approach
significantly complements experimental results and quantum chemical
methods, providing a dynamic perspective on inhibitor performance.

The top and side views of the optimal configurations of both studied
Schiff bases on the Fe(110) surface are shown in Figure 3.

\fig{c3/image17}{}

{\bfseries Fig.3 - Side and top views of the most appropriate configu­ration
for adsorption of SB1 (A) and SB2 (B) mole­cules on the Fe (1}{\bfseries 10)
surface obtained by MD simulations}

As can be seen, the SB1 molecule lying flat on the Fe(110) surface. The
corresponding values of the adsorption energy indicate a more effective
interaction between the SB1 molecules and the Fe(110) surface. The
presence of two bulky \emph{tert}-butyl substituents in SB2 structure
prevents effective adsorption of the inhibitor due to sterical
hindrance. As a result, the inhibitor molecule is positioned at an
angle, which also prevents effective interaction with the metal surface
of all potential sorption centres, leading to the decrease of inhibition
efficiency.

\emph{Proposed adsorption mechanism.} Based on the results obtained, a
mechanism of sorption of the studied ONO-type Schiff bases on the
Fe(110) surface was proposed and shown in Figure 4.



{\bfseries Fig.4 - Proposed adsorption mechanism for SB1 and SB2 on the Fe
(110) surface in accordance}

{\bfseries with calculated descriptors}

As can be seen, the SB1 molecule has a flatter orientation relative to
the metal surface than the SB2. This is due to the presence of bulky
\emph{tert}-butyl substituents, which create steric hindrance to more
effective adsorption. More positive changes in heteroatoms in SB2
molecule also reduce the adsorption efficiency due to weaker
interactions with the metal surface.

SB1 exhibits higher inhibition efficiency (86.55\%) due to its favorable
electronic properties, including a higher dipole moment (2.8142 D),
lower ionization energy (5.6900 eV), and greater chemical softness
(0.4948 eV). These parameters enhance its ability to donate electrons to
the metal surface, forming stable donor-acceptor interactions.
Additionally, SB1' s flatter orientation relative to the
metal surface facilitates effective adsorption through its heteroatoms,
as supported by its higher total negative charge (TNC = -2.4783) and
fraction of transferred electrons (ΔN=−0.8241). In contrast, SB2 shows
lower inhibition efficiency (56.05\%) due to steric hindrance caused by
bulky \emph{tert}-butyl groups, which limit its adsorption capacity.
Despite its higher electronegativity (6.8860 eV) and electron affinity
(4.2780 eV), these properties result in weaker donor-acceptor
interactions and reduced adsorption intensity. The steric effects also
prevent optimal alignment of adsorption centers, further decreasing its
effectiveness. The back-donation of electrons from the \emph{d}-orbitals
of the metal enhances the interaction of SB1 with the Fe(110) surface.
Despite the fact that the LUMO energy values for SB2 indicate its better
ability to accept electrons from the metal surface, more efficient
back-donation is observed for SB1 due to the localization of LUMO on
both aromatic systems and the absence of steric hindrance.

{\bfseries Conclusion.} This study employed density functional theory (DFT)
and molecular dynamics (MD) simulations to evaluate the corrosion
inhibition performance of two ONO-type Schiff bases. The quantum
chemical calculations revealed that SB1 possesses more favorable
electronic properties for corrosion inhibition, including a higher HOMO
energy, lower energy gap (ΔE\emph{\tsb{gap}}), greater dipole
moment, and enhanced chemical softness, all of which contribute to its
excellent electron-donating ability and reactivity. The molecular
electrostatic potential (MEP) maps and frontier molecular orbital (FMO)
distributions indicated that SB1 provides more effective adsorption
centers localized on heteroatoms, promoting stronger interactions with
the metal surface. Mulliken charge analysis and Fukui function
evaluations further supported the predominance of nitrogen and oxygen
atoms as active sites for adsorption, particularly in SB1. The
calculated global reactivity descriptors consistently pointed to
SB1' s higher inhibitory efficiency. Moreover, MD
simulations confirmed that SB1 adopts a favorable flat orientation on
the Fe(110) surface, maximizing its surface coverage and interaction
energy, while steric hindrance from bulky \emph{tert}-butyl substituents
in SB2 decreases its adsorption efficiency. In summary, SB1 demonstrated
excellent corrosion inhibition performance compared to SB2, both
theoretically and experimentally. These findings highlight the value of
integrated approaches for the rational design of effective corrosion
inhibitors.

{\bfseries References}

1. Al-Amiery A.A., Isahak W.N.R.W., Al-Azzawi W.K. Corrosion Inhibitors:
Natural and Synthetic Organic Inhibitors//Lubricants. - 2023. - Vol.
11(4):174. \href{https://doi.org/10.3390/lubricants11040174}{DOI
10.3390/lubricants11040174}.

2. Yang H.-M. Role of Organic and Eco-Friendly Inhibitors on the
Corrosion Mitigation of Steel in Acidic Environments-A State-of-Art
Review//Molecules. - 2021. - Vol.26(11): 3473.
\href{https://doi.org/10.3390/molecules26113473}{DOI
10.3390/molecules26113473}.

3. Lordjames A. et al. Schiff Bases as Effective and Sustainable
Corrosion Inhibitors // Saudi Jourbal of Engineering and Technology.
-2025. -Vol.10 (04). -P.127-136.
\href{https://doi.org/10.36348/sjet.2025.v10i04.002}{DOI
10.36348/sjet.2025.v10i04.002}.

4. Govindasamy R. et al. Improved corrosion inhibition by heterocyclic
compounds on mild steel in acid medium//Corrosion Reviews. -2022. -Vol.
40 (2). -P.137-148.
\href{https://doi.org/10.1515/corrrev-2021-0045}{DOI
10.1515/corrrev-2021-0045}.

5. Senapati R. et al. Amino acid Schiff bases as efficient corrosion
inhibitor for mild steel in aqueous H2SO4: a comparative
study//\href{https://www.researchgate.net/journal/BCREC-Engineering-Science-TransactionBEST-2582-9068?_tp=eyJjb250ZXh0Ijp7ImZpcnN0UGFnZSI6InB1YmxpY2F0aW9uIiwicGFnZSI6InB1YmxpY2F0aW9uIn19}{BCREC
Engineering \& Science Transaction(BEST)}. -2023. - Vol.4 (01). --
P.1-10.DOI
\href{https://doi.org/10.61429/BEST2023E03}{10.61429/BEST2023E03}.

6. Shaker L.M. et al. Understanding the mechanism of organic corrosion
inhibitors through density functional theory//Koroze a ochrana
materiálu. -2024. -Vol.68 (1). -P.9--21.
\href{https://doi.org/10.2478/kom-2024-0002}{DOI 10.2478/kom-2024-0002}.

7. Mamand D.M., Qadr H.M. Corrosion inhibition efficiency and quantum
chemical studies of some organic compounds: theoretical
evaluation//Corrosion Reviews. - 2023. - Vol.41 (4). - P.427-441.
\href{https://doi.org/10.1515/corrrev-2022-0085}{DOI
10.1515/corrrev-2022-0085}.

8. Bhaskara S. et al\emph{.} Evaluation of Corrosion Inhibition
Efficiency of Aluminum Alloy 2024 by Diaminostilbene and Azobenzene
Schiff Bases in 1 M Hydrochloric Acid // International Journal of
Corrosion//\href{https://onlinelibrary.wiley.com/journal/7572}{International
Journal of Corrosion}. -2021. -Vol.2021(1). -P.1-20.
\href{https://doi.org/10.1155/2021/5869915}{DOI 10.1155/2021/5869915}.

9. Schmidt M.W. et al. General atomic and molecular electronic structure
system//\href{https://onlinelibrary.wiley.com/journal/1096987x}{Journal
of Computational Chemistry}. -1993. -Vol.14(11). -P.1347-1363.
\href{https://doi.org/10.1002/jcc.540141112}{DOI 10.1002/jcc.540141112}.

10. Klamt A. The COSMO and COSMO-RS solvation models//WIREs
Computational Molecular Science. - 2018. - Vol.8 (1): e1338.
\href{https://doi.org/10.1002/wcms.1338}{DOI 10.1002/wcms.1338}.

11. Akatyev N. et al\emph{.} A new potential of sodium
anthraquinone-2-sulfonate as a corrosion inhibitor for carbon steel in
0.5 M H2SO4 // Journal of Electrochemical Science and Engineering. -
2025. - Vol.15 (2):2512. \href{https://doi.org/10.5599/jese.2512}{DOI
10.5599/jese.2512}.

12. Akatyev N. Semi-empirical investigation of zinc(ii) salicylate.
comparison with x-ray structure//Vestnik KazUTB. -2024. -Vol.3(24). -P.
236-249. \href{https://doi.org/10.58805/kazutb.v.3.24-514}{DOI
10.58805/kazutb.v.3.24-514}.

13. Abraham M.J. et al\emph{.} GROMACS: High performance molecular
simulations through multi-level parallelism from laptops to
supercomputers // SoftwareX. -2015. -Vol.1-2. - P.19--25.
\href{https://doi.org/1016/j.softx.2015.06.001}{DOI
1016/j.softx.2015.06.001}.

14. Mazlan N. et al\emph{.} Density functional theory and molecular
dynamics simulation studies of bio-based fatty hydrazide-corrosion
inhibitors on Fe (1 1 0) in acidic media // Journal of Molecular
Liquids. - 2022. - Vol.347: 118321.
\href{https://doi.org/10.1016/j.molliq.2021.118321}{DOI
10.1016/j.molliq.2021.118321}.

15. Kotupalli M.R. \emph{et al.} Corrosion Inhibition of Mild Steel by
\emph{Plumeria rubra} Flower Extract: Electrochemical and Computational
Study//ChemistrySelect. -2024. -Vol.9 (46): e202403617.
\href{https://doi.org/10.1002/slct.202403617}{DOI
10.1002/slct.202403617}.

16. Ogunyemi B.T. Quantum Chemical and Reactivity-Based Evaluation of
Aspartic Acid and its Oligomers as Eco-Friendly Corrosion Inhibitors for
Iron Surfaces//IJRASET. International Journal for Research in Applied
Science and Engineering Technology (IJRASET). -2025. -Vol.13 (3). -P.
3585--3593. \href{https://doi.org/10.22214/ijraset.2025.68891}{DOI
10.22214/ijraset.2025.68891}.

17. Asanova D.Zh., Serikkalieva A.A., Akat' ev N.V.
Issledovanie ingibirujushhih svojstv osnovanij Shiffa ONO-tipa //
Dostizhenija molodyh uchenyh: himicheskie nauki: Sbornik tezisov X
Vserossijskoj molodezhnoj konferencii. Ufimskij universitet. -2025. -С.
46-47. ISBN 978-5-7477-6091-2. {[}in Russian{]}

\emph{{\bfseries Information about the authors}}

Akatyev N.V. - Candidate of Chemical Sciences, senior lecturer, M.
Utemisov West Kazakhstan University, Uralsk, Kazakhstan, e-mail:
nikolay.akatyev@wku.edu.kz;

Vardanyan R.D. - student, M. Utemisov West Kazakhstan University,
Uralsk, Kazakhstan, e-mail:
darupandandtary@gmail.com;

Kozhevnikova E.A. - student, M. Utemisov West Kazakhstan University,
Uralsk, Kazakhstan, e-mail:
katerinakozevnikova853@gmail.com.

\emph{{\bfseries Сведения об авторах}}

Акатьев Н.В. - кандидат химических наук, старший преподаватель,
Западно-Казахстанский университет им. М.Утемисова, Уральск, Казахстан,
e-mail:
nikolay.akatyev@wku.edu.kz;

Варданян Р. Д.-- студент, Западно-Казахстанский университет им.
М.Утемисова, Уральск, Казахстан, e-mail:
darupandandtary@gmail.com;

Кожевникова Е.А. -- студент, Западно-Казахстанский университет им.
М.Утемисова, Уральск, Казахстан, e-mail:
katerinakozevnikova853@gmail.com.\