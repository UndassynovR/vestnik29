\id{ҒТАМР 52.13.17}{}

{\bfseries МИНЕРАЛДЫҚ ШИКІЗАТТЫҢ ЭКОНОМИКАЛЫҚ СИПАТТАМАЛАРЫН ЕСКЕРЕ ОТЫРЫП,
ТАУ-КЕН ЖҰМЫСТАРЫНЫҢ НЕГІЗГІ КЕЗЕҢДЕРІНІҢ СТРАТЕГИЯСЫН ТАҢДАУ}

{\bfseries А.У.Кожантов}{\bfseries \envelope ,
Е.Х.Абен}{\bfseries ,
С.С.Мырзахметов}{\bfseries ,Д.К.
Ахметканов}

\emph{Сәтбаев университеті, Алматы, Казахстан}

\corrauthor{Корреспондент-автор:a.kozhantov@satbayev.university}

Мақалада негізгі және қосалқы (ілеспе) пайдалы компоненттерді өндіру
кезінде алынған барлық түпкілікті өнімдердің құнын және осы өнімдерді
өндіруге жұмсалатын шығындарды ескере отырып, қорлардың жай-күйін
бағалау ұсынылған, ілеспе пайдалы қазбаларды пайдалануға тартудың
экономикалық тиімділігі - калькуляциялық блоктың 1 тоннасын өндіруге
және қайта өңдеуге жұмсалатын шығындарға есептелген, қарастырылып
отырған пайдалы қазбаны қайта өңдеу кезінде өндірілген түпкілікті өнім
құнының қатынасына тең жағдай коэффициенті. Мысал ретiнде Жайрем
полиметалл және марганец кен орнының базасында келтiрiлген және күрделi
кен орындарында тау-кен жұмыстарын дамытуды, өндiрудiң толықтығын және
көп компоненттi кендердiң қажеттi сапасын қамтамасыз ететiн әзiрленген
әдiстеменi сынау нәтижелерi алынды. Кешендi кен орындарын игеру кезiнде
жер қойнауын пайдалану талаптарын толық қанағаттандыратын тау-кен
жұмыстарын дамыту бағытын негiзделген таңдау тиiмдiлiкке неғұрлым елеулi
әсер ететiнi анықталды. Көп компонентті кендердің әртүрлі типтері мен
сорттарын іріктеу және игеру кезінде күрделі кен орындарын таңдау және
игеру тәртібі үшін ашудың ең төменгі коэффициентінің критерийін
пайдалануға жол берілмейді.

Кен орындарын кешенді игерудің экономикалық тиімділігін бағалау
әдістемесі жай-күйінің коэффициенті (К\tsb{с}) негізінде
әзірленді. Осы коэффициент барлық түпкілікті өнім құнының (негізгі және
ілеспе компоненттерден) 1 тонна кенді өндіруге және қайта өңдеуге
арналған шығындарға қатынасы ретінде есептеледі.

Күрделі көп компонентті кен орындарын игерудің толықтығы мен
рентабельділігін арттыру үшін әдістемені енгізу.

Оңайлатылған өлшемдерге негізделген қолданыстағы тәсілдер (мысалы,
ашудың ең төменгі коэффициенті) барлық бағалы компоненттерді оңтайлы
алуды қамтамасыз етпейді және экономикалық әлеуетті жоғалтуға әкеледі.

Барлық пайдалы қазбаларды экономикалық бағалау негізінде тау-кен
жұмыстарын дамыту стратегиясын оңтайландыру.

Тау-кен жұмыстарының бағытын негiздi таңдау жекелеген техникалық
параметрлердi барынша азайтуға қарағанда жалпы тиiмдiлiкке көбiрек әсер
ететiнi анықталды. Бұл жер қойнауын ұтымды пайдалану қағидаттарының
сақталуын қамтамасыз етеді. Жайрем полиметалл және марганец кен орнында
қорларды алу толықтығы, көп компонентті кендердің қажетті сапасы, ілеспе
пайдалы қазбаларды пайдалануға тартудың экономикалық орындылығы.

{\bfseries Түйін сөздер:} карьер бөлімі, кондиционды коэффициенті, карьер,
ашық игеру, пайдалы қазбалар, қайта өңдеу.

{\bfseries ВЫБОР ОСНОВНЫХ ЭТАПОВ ГОРНЫХ РАБОТ С УЧЕТОМ ЭКОНОМИЧЕСКИХ
ХАРАКТЕРИСТИК МИНЕРАЛЬНОГО СЫРЬЯ}

{\bfseries А.У. Кожантов\envelope , Е.Х.Абен, С.С.Мырзахметов,
Д.К. Ахметканов}

\emph{Satbayev University, Алматы, Казахстан,}

е-mail:
a.kozhantov@satbayev.university

Экономическая эффективность вовлечения в эксплуатацию попутных полезных
ископаемых, по которым в статье представлена оценка состояния запасов с
учетом стоимости всех конечных продуктов, полученных при производстве
основных и попутных (сопутствующих) полезных компонентов, и затрат на
производство этих продуктов - коэффициент состояния, равный отношению
стоимости конечной продукции, добытой при переработке рассматриваемого
полезного ископаемого, рассчитанный на затраты на добычу и переработку 1
тонны калькуляционного блока. В качестве примера получены результаты
испытаний разработанной методики, приведенной на базе Жайремского
полиметаллического и марганцевого месторождения и обеспечивающей
развитие горных работ а сложных месторождениях, полноту добычи и
необходимое качество многокомпонентных руд. Установлено, что более
значительный эффект на эффективность оказывает обоснованный выбор
направления развития горных работ, полностью удовлетворяющий требованиям
недропользования при освоении комплексных месторождений. При отборе и
разработке различных типов и сортов многокомпонентных руд не допускается
использование критерия минимального коэффициента вскрытия для порядка
выбора и разработки сложных месторождений.

Методика оценки экономической эффективности комплексной разработки
месторождений разработана на основе коэффициента состояния (Кс). Данный
коэффициент рассчитывается как отношение стоимости всей конечной
продукции (от основных и сопутствующих компонентов) к затратам на добычу
и переработку 1 тонны руды.

Внедрение методики для повышения полноты и рентабельности разработки
сложных многокомпонентных месторождений.

Существующие подходы, основанные на упрощенных критериях (например,
минимальный коэффициент раскрываемости), не обеспечивают оптимальное
получение всех ценных компонентов и приводят к потере экономического
потенциала.

Оптимизация стратегии развития горных работ на основе экономической
оценки всех полезных ископаемых.

Установлено, что обоснованный выбор направления горных работ оказывает
большее влияние на общую эффективность, чем минимизация отдельных
технических параметров. Это обеспечит соблюдение принципов рационального
использования недр. Полнота извлечения запасов на Жайремском
полиметаллическом и марганцевом месторождении, необходимое качество
многокомпонентных руд, экономическая целесообразность вовлечения в
эксплуатацию попутных полезных ископаемых.

{\bfseries Ключевые слова:} разрез карьера, коэффициента норма качества и,
борта карьера, открытая разработка, полезные ископаемые, переработка.

{\bfseries SELECTION OF THE MAIN STAGES OF MINING OPERATIONS TAKING INTO
ACCOUNT THE ECONOMIC CHARACTERISTICS OF MINERAL RAW MATERIALS}

{\bfseries A. Kozhantov\envelope , E. Aben, S. Myrzakhmetov, D.
Akhmetkanov}

\emph{Satbayev University, Almaty, Kazakhstan,}

е-mail:
a.kozhantov@satbayev.university

Economic efficiency of utilization of associated minerals, for which the
article presents an estimate of the state of reserves, taking into
account the cost of all final products obtained during the production of
basic and associated (by-product) useful components, and production
costs of these products - a condition coefficient equal to the ratio of
the value of the final product extracted during processing of the
mineral under consideration, calculated for the cost of extraction and
processing of 1 ton of the calculation unit. As an example, the results
of testing the developed methodology presented on the basis of the
Zhayremskoye deposit and ensuring the development of mining operations
in complex deposits, the completeness of production and the required
quality of multicomponent ores were obtained. It has been established
that a more significant effect on efficiency is exerted by a reasonable
choice of the direction of development of mining operations, which fully
meets the requirements of subsoil use in the development of complex
fields. When selecting and developing various types and grades of
multicomponent ores, it is not allowed to use the criterion of the
minimum opening factor for the procedure for selecting and developing
complex deposits.

The methodology for assessing the economic efficiency of integrated
field development was developed on the basis of the condition
coefficient (Kc). This ratio is calculated as the ratio of the cost of
all final products (from the main and related components) to the cost of
mining and processing 1 ton of ore.

Implementation of a methodology to improve the completeness and
profitability of development of complex multi-component fields.

Existing approaches based on simplified criteria (for example, a minimum
rate of disclosure) do not provide optimal receipt of all valuable
components and lead to a loss of economic potential.

Optimization of mining development strategy based on economic evaluation
of all minerals.

It has been established that a reasonable choice of mining direction has
a greater impact on the overall efficiency than minimizing individual
technical parameters. This will ensure compliance with the principles of
rational use of subsoil. Completeness of reserves recovery at
Zhayremskoye deposit, required quality of multicomponent ores, economic
feasibility of associated mineral resources utilization.

{\bfseries Keywords:} quarry section, coefficient of air conditioning,
sides of the quarry, open-pit mining, minerals, processing.

{\bfseries Кіріспе}. Зерттеудің әдістемелік негізі кешенді кен орындарын
игерудің экономикалық тиімділігін бағалауға арналған коэффициенті
аталатын интегралдық көрсеткішке негізделген. Бұл әдістеме тек
математикалық-экономикалық тәсілді ұсынумен шектелмей, оның іс жүзіндегі
қолдану алгоритмін, бастапқы деректердің көздерін және оңтайландыру
процедураларын егжей-тегжейлі сипаттайды. Пайдалы компоненттердің
экономикалық шекті шығындарды қамтамасыз етеді. Негізгі және ілеспе
(сопутствующих) компоненттерді, сондай-ақ оларды алуға жұмсалатын
шығындарды және пайдалы қазбаларды пайдалану тиімділігін қоса алғанда,
барлық түпкілікті өнімнің құнына негізделген қорлар сапасының нормаларын
экономикалық бағалау сапа нормасының коэффициентін қолдана отырып жүзеге
асырылуы мүмкін. Осы коэффициент белгілі бір блок шегінде өндірілетін
пайдалы қазбаның 1 тоннасынан алынатын дайын өнім құнының өндіру және
қайта өңдеу шығындарына қатынасын білдіреді. Есептеу формуласы
компоненттердің мазмұны, сапа коэффициенттері және алу коэффициенттері,
сондай-ақ тасымалдау және қайта өңдеу шығындары сияқты көптеген
параметрлерді қамтиды, соның арқасында компаниялар әзірлеуді жалғастыру,
өндірістік процестерді оңтайландыру және инфрақұрылымға инвестициялар
салу қажет {[}1, 2{]}:

,
(1)

мұнда α\tsb{1i} - қорлардағы і-компоненттің үлесі (пайызбен);

k\tsb{k} - пайдалы қазбаларды өндіру кезінде олардың сапасының
өзгеруін көрсететін коэффициент (үлеспен);

\fig{g/image16}{}\tsb{оi},
\fig{g/image17}{}\emph{\tsb{i}}
- қайта байыту және металлургиялық қайта өңдеу кезінде пайдалы
компоненттерді алу коэффициенттері (үлестерде);

с\tsb{мi} - байыту өнімін (концентратты) металлургиялық
кәсіпорынға дейін тасымалдаудың жиынтық құны және бір тонна үшін
теңгемен көрсетілген соңғы өнімнің (металдың) бір тоннасының өзіндік
құнындағы металлургиялық өңдеу құны;

Ц\tsb{мi}- алынатын түпкілікті өнім (і-түрі) құнының тоннасына
теңгемен көрсетілген шамасы.

c\tsb{д} - байыту фабрикасына дейін өндіруге, тасымалдауға
және бір тонна өндірілген пайдалы қазбаны байытуға жұмсалатын жалпы
шығындар.

,

немесе

,
(2)

мұнда с\tsb{об} - алынған минералдық шикізаттың бір бірлігінің
сипаттамасын жақсартуға күтілетін шығындар.

Кенді белгілі бір блоктан (немесе 1 тоннасынан) алынатын барлық
түпкілікті өнімнің құндылығының осы блокты өндіруге және қайта өңдеуге
кеткен жалпы шығындарға қатынасы ретінде анықталады. Бұл көрсеткіш
негізгі және ілеспе пайдалы компоненттерді кешенді есепке алуға
мүмкіндік береді.

Нәтижесінде, жалпы жағдайда сапа нормасының коэффициентін:

(3)

мұнда \emph{α\tsb{1i}} - қорлардағы і-компоненттің құрамы,\%;

\emph{k\tsb{к}} - өндіру кезінде пайдалы қазба сапасының
өзгеру коэффициенті, өлшемсіз көрсеткіш;

\fig{g/image16}{}\tsb{об\emph{i}},
\fig{g/image17}{}\tsb{м\emph{i}}
- байыту және металлургиялық қайта өңдеу кезінде i-компонентті алу
коэффициенттері тиісінше мөлшерсіз, бірлік үлесі;

\emph{с\tsb{м i}}- концентратты металлургиялық кәсіпорынға
дейін тасымалдауға және оны қайта өңдеуге жұмсалатын шығындар, бір тонна
өнімнің өзіндік құнында (теңге/тонна), тг/т;

\emph{с\tsb{д}} - өндіруге, байыту фабрикасына дейін
тасымалдауға және байытуға жұмсалған жалпы шығындар, өндірілген ресурс
құнының үлесінде;

\emph{Ц\tsb{мi} -} \emph{i}-компоненттің түпкілікті өнімінің
бағасы, теңге/тонна.

Тау-кен байыту комбинатының қызмет ету мерзімін ұзарту үшін оның қуатын
арттырусыз қорларды қайта өңдеудің орындылығын бағалау мына формула
бойынша жүргізіледі {[}2, 3{]}:

(4)

Теңдеуге тиісті мәндерді қою арқылы мыналарды аламыз:



яғни, барлық қорлар жиынтығында кондициялық емес болып шығады. Алайда
жиынтық қорларды құрайтын блоктарда пайдалы компоненттердің құрамы
әртүрлі және оларды жеке бағалау орынды. Мұны әрбір блок немесе
шекаралық жағдайлар үшін жеке сапа нормасының коэффициентін есептей
отырып жасауға болады.

(5)

Мысалға, мыс компоненті үшін:





Тау-кен байыту кәсіпорнының қуатын кеңейтпей, оның қызмет ету мерзімін
ұзарту мақсатында осы қорларды қайта өңдеуге тартуды бағалауды {[}3,
4{]}:

(6)





(7)

,
(8)

мұндағы α\emph{\tsb{i}} - қорлардағы i-компоненттің құрамы,
\%;

k\tsb{к} - өндіру кезінде пайдалы қазба сапасының өзгеру
коэффициенті, бірлік үлесі;

\fig{g/image16}{}\tsb{об\emph{i}},
\fig{g/image17}{}\tsb{м\emph{i}}
- байыту және металлургиялық қайта бөлу кезінде пайдалы і-компонентті
алу коэффициенті, бірлік үлесі;

-
байыту өнімін (концентратты) металлургия зауытына дейін тасымалдауға
және түпкілікті өнім (металл) тоннасының өзіндік құнындағы
металлургиялық қайта бөлуге арналған шығындар сомасы, dollar/т;

С\tsb{D} - байыту фабрикасына дейін өндіру, тасымалдау және
өндірілген пайдалы қазбаның бірлігін байыту шығындарының сомасы, dollar
/т;

\emph{-}
түпкілікті \emph{i}-өнімнің алынатын құндылығы, dollar /т;

\emph{C\tsb{об}} - байыту өнімін (концентратты) металлургия
зауытына дейін тасымалдауға және түпкілікті өнім (металл) тоннасының
өзіндік құнындағы металлургиялық бөлуге жұмсалған шығындар сомасы,
dollar /т;

\emph{C\tsb{Д}} - өндірілген пайдалы қазбаның бірлігін
байытуға жұмсалатын алдағы шығындар, dollar/т;

Ц\tsb{i} - 1 т өндірілетін пайдалы қазбадан негізгі және і-ші
ілеспе пайдалы компонентті алу кезінде алынатын түпкілікті өнімнің
алынатын құндылығы;

V\tsb{i} - ашудың жиынтық көлемі;

P\tsb{i} - игеру басталғаннан і жылға дейін өндірілген пайдалы
қазбаның жиынтық көлемі;

с\tsb{i.пр} - баланстан тыс қорларды алудың алдағы
шығындарының өтімділігі;

\emph{C} - тау-кен массасының кейбір көлемін әзірлеу;

Р\tsb{s} - пайдалы қазба қорлары, s-сорты (типі);

с\emph{\tsb{i}} - аршу жұмыстарына жұмсалған шығындарды
өтеусіз өндіруге жұмсалған шығындар сомасы, dollar /т;

с\tsb{в} - аршу жыныстарының бірлігін жоюға арналған шығындар,
dollar /м\tsp{3};

V\emph{\tsb{B}} - аршу жұмыстарының көлемі,
м\tsp{3};

Д - оны сатудан күтуге болатын жиынтық кірістер (пайдалар), онда олардың
С/D қатынасы салынған шығындарды пайдалану дәрежесін сипаттайды;

\emph{k -} кеннің түрлері мен сорттарының саны;

Z\tsb{pj}- жұмыс аймағының j-жағдайы үшін р-көкжиектегі шығын
коэффициенті;

Z\tsb{ji}- i-ші қима үшін р-көкжиек түбінің j-ші контурында да
қайталанады;

\emph{k\tsb{ВС.кос} -} ашу коэффициентінің рұқсат етілген
мәні;

\fig{g/image33}{}-
жиынтық құндылығы, оны өндіруге жол берілетін тұйықтау шығындары ретінде
қабылданатын 1 т кешенді кендерден алынған дайын өнім түріндегі барлық
компоненттердің {[}4, 5{]}.

{\bfseries Материалдар мен әдістер.} Зерттеу пайдалы қазбалар қорларын
игерудің экономикалық тиімділігін бағалауға негізделеді. Құрал ретінде
көрсеткішке негізделген әдіс қолданылады - сапа нормасы және (Кк)
{[}1{]}. Бұл коэффициент зерттелетін блокта өндірілген бір тонна кеннен
алынған түпкілікті өнім құнының оны өндіру мен қайта өңдеудің жиынтық
шығындарына арақатынасы ретінде айқындалады. Мұндай тәсіл қайта өңдеу
кезінде негізгі ресурстарды да, ілеспе пайдалы компоненттерді де
пайдалану тиімділігін жан-жақты талдауға мүмкіндік береді.



(9)

Минералдық ресурстарды кешенді есепке алу кезінде графоаналитикалық
әдісті пайдалана отырып тау-кен жұмыстарын жүргізудің неғұрлым ұтымды
бағытын айқындау кен орнында анықталған пайдалы қазбалардың немесе кен
қорларының әрбір түрі үшін де сапа нормасының коэффициентін есептеуге
негізделеді. Тау-кен жұмыстарын дамыту бағытын оңтайландыру
графоаналитикалық әдіс келесі ретпен карьерді тереңдетудің әрбір ықтимал
бағыты үшін (1-суреттегі a, b, c, ... ж сызықтары) әрбір кен түрі
бойынша өндіруге жарамды кен көлемі анықталды. Алынған көлемдерге сәйкес
кен түрінің коэффициентіне көбейтіліп, тең кондициялы көлемдері көрінді.
Бұл әртүрлі компоненттерді бір экономикалық өлшемге келтіруге мүмкіндік
береді. Карьердің көлденең қимасындағы пайдалы қазбалардың көлемін
талдағаннан кейін тереңдетудің түрлі бағыттары кезінде және оларды
тиісті коэффициенттерге көбейту сапа нормасы және тау-кен жұмыстарын
ұйымдастырудың барынша тиімді нұсқасы анықталады, ол ресурстарды барынша
тиімді пайдалануға және өндіру тиімділігін арттыруға мүмкіндік береді.
Соңғы қадам ретінде әрбір тереңдету бағыты үшін аршу жұмыстарының көлемі
(V) мен шығындары (C) бағаланды. Ең төменгі көбейтіндісін ұсынатын бағыт
тау-кен жұмыстарын дамытудың оңтайлы бағыты ретінде таңдалды.

А.И. Арсентьев әзірлеген әдістен айырмашылығы критерийі бойынша
минералдық ресурстарды кешенді пайдалану кезінде графаналитикалық
әдіспен тау-кен жұмыстарын дамытудың оңтайлы бағытын айқындаудың мәні
мынада. Кен орнындағы пайдалы қазбалардың әрбiр түрi, кен түрлерi үшiн
немесе кондициялар жобасының нұсқалары бойынша шектелген кен қорлары
үшiн де сапа нормасының коэффициентiн анықтайды. Карьердің көлденең
қимасында тереңдетудің барлық ықтимал бағыттары бойынша карьердің жұмыс
аймағы шаблонының көмегімен пайдалы қазбалардың көлемі әрбір түрі
бойынша жеке өлшенеді. Өлшенген көлемдер пайдалы қазбаның осы түрі үшін
де сапа нормасын тиісті коэффициентке көбейтеді және содан кейін
карьердің жұмыс аймағының әрбір жағдайы үшін қосылады.

Алынған деректер бойынша үдемелі мәндердің көмекші кумулятивтік кестесі
Р k
= ,
\emph{Н} карьерін тереңдету кезеңдеріне байланысты құрылады.

,
(10)

Оқшаулағыштарды сапа нормасы мен оське тең құру үшін ординатты тең
сегменттерге бөледі және Р1, Р2,..., Pn көлденең сызықтарын жүргізеді.
Олардың белгілі бір сапа нормасына сәйкес келетін пайдалы қазбалар
көлемінің тәуелділігін көрсететін кестелермен қиылысу нүктелері өндіру
тереңдігін білдіреді. Бұл нүктелер көлденең қимаға көшіріледі (1-суретті
қараңыз), тиісті сапа нормасына сәйкес келетін пайдалы қазбалар
көлемдерінің бірдей мәндерінің белгісі бойынша қосылады және
тереңдетудің барлық ықтимал бағыттары үшін алынатын пайдалы қазбалардың
тең көлемін көрсететін оқшаулау сызықтарын алады {[}4- 6{]}.

Бұдан әрі біздің зерттеулеріміз көмір қасиеттерінің вариациялары мен
электромагниттік өрістің сипаттамаларын ескере отырып, көмір қабаттарын
қыздыру процестерін неғұрлым егжей-тегжейлі талдау және жетілдіру
мақсатында тереңдетілуі мүмкін.

Қосымша, бұл зерттеу төбе және табан жыныстарындағы жылу ысыраптарын
ескере отырып, көмір массасын қыздыру процесінің тиімділігін арттыруға,
сондай-ақ жиілік, қуат және сәулелендіргіштің нысаны сияқты
электромагниттік өріс параметрлерін оңтайландыруға бағытталуы мүмкін.
Осы мақсаттарға қол жеткізу үшін температуралық режимді неғұрлым дәл
болжау және реттеу үшін сандық әдістерді пайдалана отырып, көп
параметрлік талдауды ескеретін неғұрлым күрделі математикалық модельді
әзірлеу қажет. Зерттеудің қисынды аяқталуы есептік деректерді
эксперименттік тексеруді және көмір массасын қыздырудың оңтайлы
шарттарын анықтауды қамтитын зертханалық және далалық эксперименттер
болады. Заманауи анықтамалық шешім қабылдау (Surpac Systems, Micromine,
Datamine) k -коэффициенті әртүрлі даму сценарийлерін салыстыру үшін
сандық негіз ретінде қызмет етеді. Техникалық шешімдерді таңдау тікелей
экономикалық нәтижелерге байланысты. Кен орнын жоғары дәлдікпен
экономикалық бағалауға мүмкіндік береді.

Карьердің көлденең қимасы (1-суретті қараңыз) аршу жұмыстарының өзіндік
құны бойынша айтарлықтай ерекшеленетін тау-кен және басқа да аймақтарды
бөле отырып, аршу жыныстарын тасымалдау қашықтығына және игеру
тереңдігінің деңгейіне байланысты аймақтарға бөлінеді. Әрбір бөлінген
аймақ үшін аршу жыныстарын өндіруге арналған шығындар есептеледі.

Дәстүрлі тәсілдермен (мысалы, тек ашу коэффициентін минимизациялау)
салыстырмалы талдау жүргізілді. Ашу коэффициентінің төмендігіне ғана
негізделген тәсіл (1a-сурет) жоғары құнды, бірақ алу қиын орналасқан кен
блоктарын игеруден бас тартуға әкеп соғуы мүмкін, бұл өз кезегінде
экономикалық жоғалуына себеп болады.

Карьерді тереңдетудің барлық ықтимал бағыттары желілерінің пайдалы
қазбалардың тең кондициялы көлемдерінің оқшаулағыштарымен қиылысында
карьердің жұмыс аймағының шаблонының көмегімен аршу жыныстарының көлемін
өлшейді және оларды жоюға жұмсалатын тиісті шығындарға көбейтеді.

Карьерді тереңдетудің әрбір кезеңі үшін
c\tsb{B}V\tsb{B}. ең төменгі мәндеріне сүйене
отырып, оның оңтайлы бағыты айқындалады.

Егер аршу жұмыстарының шығындарындағы айырмашылық аз болса, оларды
елемеуге болады және ұтымды бағыт аршу жұмыстарының ең аз көлемі бойынша
таңдалады.

1-суретте карьердің көлденең қимасы берілген, ол аршу жыныстарын
тасымалдау қашықтығы мен өндіру тереңдігіне қарай аймаққа бөлінеді (аршу
жыныстарын жою жөніндегі жұмыстарға айтарлықтай әртүрлі шығыстармен
сипатталатын таулы және басқа аймақтарға). Бөлінген аймақтардың
әрқайсысы үшін аршу жыныстарын алуға арналған шығындар есептеледі. Аршу
жыныстарының көлемін анықтау және оларды жоюға жұмсалатын жиынтық
шығындарды есептеу үшін карьерді тереңдетудің барлық ықтимал
бағыттарының сызықтарын пайдалы қазбалардың тең кондициялы көлемдерінің
оқшаулағыштарымен қиып өту жүргізіледі. Ол үшін карьердің жұмыс
аймағының трафареті пайдаланылады.

\emph{а}

\emph{б}

\emph{в}

\emph{г}

\emph{д}

\emph{е}

\emph{ж}

\emph{φ}

\emph{№1}

\emph{{\bfseries в\tsb{1}}}

\emph{{\bfseries г\tsb{0}}}

\emph{№2}

\emph{№3}

\emph{г\tsb{2}}

\emph{г\tsb{3}}

\emph{г\tsb{4}}

\emph{з}

\emph{1}

\emph{2}

\emph{3}

\emph{4}

\emph{5}

\emph{6}

\emph{7}

\emph{8}

P\tsb{0}

P\tsb{1}

P\tsb{2}

Р\tsb{3}

P\tsb{4}

\emph{б}

\emph{б}

\emph{в}

\emph{г}

\emph{д}

\emph{е}

\emph{ж}

\emph{φ}

\emph{№1}

\emph{г\tsb{0}}

\emph{№2}

\emph{№3}

\emph{ж\tsb{1}}

\emph{е\tsb{3}}

\emph{е\tsb{4}}

\emph{φ}

\emph{з}

\emph{1}

\emph{2}

\emph{3}

\emph{4}

\emph{5}

\emph{6}

\emph{7}

\emph{8}

\emph{\tsb{0}}

P\tsb{1}k\tsb{кд}

\emph{ж\tsb{2}}

P\tsb{2}k\tsb{кд}

P\tsb{3}k\tsb{кд}

P\tsb{4}k\tsb{кд}

\fig{g/image37}{}

\emph{{\bfseries ж\tsb{1}}}

\emph{{\bfseries ж\tsb{2}}}

{\bfseries 1-сурет. Үлгілік карьердің көлденең қимасы}

\emph{а, б, в, г, д, е, ж, з - тереңдетудің ықтимал бағыттарының
сызықтары; φ - карьердің жұмыс борты еңісінің бұрышы;
Р\tsb{0}, Р\tsb{1} ,Р\tsb{2}
,Р\tsb{3} ,Р\tsb{4} - пайдалы қазба көлемдерінің
оқшаулау желісі;
г\tsb{0}в\tsb{1}г\tsb{2}г\tsb{3}г\tsb{4}
- тереңдету-1 бағытындағы сызық; Р\tsb{1}k\tsb{кд},
Р\tsb{2}k\tsb{кд}
Р\tsb{3}k\tsb{кд},Р\tsb{4}k\tsb{кд}
- пайдалы қазбалардың тең кондициялы көлемдерінің оқшаулағыштары;
ж\tsb{1}ж\tsb{2}е\tsb{3}е\tsb{4}
- 2-тереңдету бағытының сызығы}

Кешенді кен орнын игеру шарттары үшін екі қағидат бойынша тереңдету
бағыты айқындалды: 1а-суретте - оның сапасын (Р\tsb{1} . . . ,
Р\tsb{4}) және пайдаланыла бастағаннан бастап ашудың ең
төменгі орташа коэффициентін
(г\tsb{0}в\tsb{1}г\tsb{2}г\tsb{3}г\tsb{4}
- сызығы) ескермегенде пайдалы қазбалар көлемдерінің оқшаулау сызықтары
негізінде - 1б-суретте - пайдалы қазбалардың тең кондициялы көлемдерінің
оқшаулау сызықтарына негізделген ұсынылған әдіс бойынша
(Р\tsb{1}k\tsb{кд},\ldots.Р\tsb{4}k\tsb{кд})
және алу үшін \emph{c\tsb{B}V\tsb{B}} ең аз мәндері,

Көрсетілген әдістер бойынша анықталған тереңдету бағыттарының сызықтарын
салыстыру қорлар сапасының нормасы бағытты таңдауға елеулі әсер ететінін
көрсетеді. Бірінші жағдайда тереңдету сызығы кен денесі бойынша өтеді.
1-бағыт үшін V=f(P) функциясының графигі осы параметр кезінде аз көлемді
куәландыра отырып, 2-бағыт үшін төмен орналасқан, Алайда, c=(Ц)
функциясының графигінде 1-тереңдету бағыты үшін функциясының жағдайы
2-бағыт үшін жоғары, бұл осы бағыттағы үлкен шығындарды көрсетеді.

Тереңдету бағыттарын айқындау нәтижелерін бағалау үшін екі бағыт үшін V
= f (Р) және c = f (Ц) (2-сурет) функцияларының графиктері құрылған.

Бұл 1-шi тереңдетудi бағыттау кезiнде өнiмге бiрдей шығындарға қол
жеткiзу үшiн 2-шi тереңдетудi бағыттауға қарағанда уақыт бойынша ерте де
үлкен салымдарды жүзеге асыру қажеттiгiн куәландырады.

\emph{V\tsb{1}=f(P)}

\emph{V\tsb{2}=f(P)}

\emph{С\tsb{2}=f(Ц)}

\emph{С\tsb{1}=f(Ц)}

V - аршу жұмыстарының көлемі;

С - пайдалы қазбаны өндіруге және байытуға арналған шығындар сомасы;

Р - пайдалы қазбаны жылдық өндіру;

Ц - өндірілген пайдалы қазбадан алынатын түпкілікті өнімнің құндылығы.

2-сурет. Уақытқа немесе басқа да нәтижелерге байланысты нәтижелдің
жинақталған шамасын көрсететін модельдеу графигі

Екінші суретте уақытқа немесе басқа параметрлерге байланысты біршама
нәрсенің (бұл жағдайда - өндірілген пайдалы қазбаның) жинақталған
(жиынтық) шамасын көрсететін: V-аршу жұмыстарының көлемі; С-пайдалы
қазбаны өндіруге және байытуға арналған шығындар сомасы; Р-пайдалы
қазбаны жылдық өндіру; Ц-өндірілген пайдалы қазбадан алынатын түпкілікті
өнімнің құндылығы. График, мысалы, тереңдетудің 1 және 2 бағытындағы
қазбалардың өткен метрі - бұл пайдалы қазбаларға қол жеткізу үшін
тау-кен қазбаларын қазып игеру өнімділігін дамыту.

{\bfseries Нәтижелер және талқылау.} Аршу коэффициентін игеру басталғаннан
бастап орташа ең төменгі критерийді пайдаланған кезде тереңдету
бағытының таңдалған сызығы пайдалы қазбалар қорларының ең жоғары
көлеміне бейім екені белгілі, көбінесе жұмыс борттары шекті жағдайларға
жеткен нүктелерде, яғни аршу жұмыстарының қарқынды түрде игеру керек
{[}5, 6{]}.

Есептеулер мынаны көрсетті: тек негізгі компонент бойынша есептегенде,
кейбір шекаралық блоктар k <1.0 көрсеткішіне ие болды
(экономикалық тиімсіз). Алайда, ілеспе компоненттерді есепке алу кезінде
осы блоктардың k \textgreater1.2-ге дейін өскені байқалды, бұл оларды
кешенді игерудің экономикалық орындылығын растады. Осылайша, минералдық
ресурстарды кешенді пайдалану кезінде тау-кен жұмыстарын дамытудың
оңтайлы бағытын айқындау тәсілдерінің бірі пайдалы қазбалардың тең
кондициялы көлемдерінің құруға негізделген графоаналитикалық әдіс болып
табылады. Графоаналитикалық талдау арқылы анықталған тау-кен жұмыстарын
дамытудың оңтайлы бағыты жай техникалық параметрлерді минимизациялау
негізінде таңдалған бағытпен салыстырғанда, жоба бойынша таза
келтірілген құнды шамамен 15\%-ға арттыруға мүмкіндік беретіні
көрсетілді.

Бұл ретте жұмыс аймағын басқару тәсілдері академик К.Н. Трубецкий
жүйелендірген және 1-кестеде көрсетілген жалпы заңдылықтарға сәйкес
келеді. Негізгі реттеуші фактор ретінде карьерлердің даму деңгейін және
олардың кеңістікте орналасуын басқару болып табылады. Жоғарыда
көрсетілгендей, бұл фактор кемердің алға жылжу жылдамдығымен және
тау-кен жұмыстарының тереңдеу қарқынымен айқындалады. Ұсынылған нақты
алгоритмдермен, дереккөздермен және есептеу процедураларымен қамтамасыз
етілген, бұл оны қайталауға және басқа да кешенді кен орындарына
қолдануға мүмкіндік береді. Көрсетілген коэффициент (k) және оны
пайдаланатын графоаналитикалық әдіс көп компонентті кен орындарын игеру
стратегиясын негіздеу үшін тиімді және іс жүзінде қолданылатын құрал
ретінде дәлелденді.

{\bfseries 1-кесте. Өндіру режимдерін оңтайландыру және тұрақты өнімділікті
қамтамасыз ету кезінде терең карьерлердегі жұмыс саласын бақылау
тәсілдері}

%% \begin{longtable}[]{@{}
%%   >{\centering\arraybackslash}p{(\linewidth - 4\tabcolsep) * \real{0.2862}}
%%   >{\centering\arraybackslash}p{(\linewidth - 4\tabcolsep) * \real{0.4000}}
%%   >{\centering\arraybackslash}p{(\linewidth - 4\tabcolsep) * \real{0.3138}}@{}}
%% \toprule\noalign{}
%% \begin{minipage}[b]{\linewidth}\centering
%% {\bfseries Тау-кен жұмыстарының режимін реттеу әдістері}
%% \end{minipage} & \begin{minipage}[b]{\linewidth}\centering
%% {\bfseries Жұмыс кеңістігін қалыптастыруды басқару тәсілдері}
%% \end{minipage} & \begin{minipage}[b]{\linewidth}\centering
%% {\bfseries Тау-кен геометриялық есептеулер кезінде борттың орнын ауыстыру
%% тәртібі}
%% \end{minipage} \\
%% \midrule\noalign{}
%% \endhead
%% \bottomrule\noalign{}
%% \endlastfoot
%% 1 & 2 & 3 \\
%% \multirow{2}{=}{\centering\arraybackslash а) тау-кен жұмыстарын
%% төмендету бағытының өзгеруі} & 1. Дайындық қазбасын салу орнын және
%% жұмыс алаңының параметрлерін таңдау &
%% \multirow{2}{=}{\centering\arraybackslash Тұрақты жылдамдықпен барлық
%% бағыттар бойынша үздіксіз орын ауыстыру} \\
%% & 2. Кен денесіне қатысты фронт жағдайын және жұмыс бағытын таңдау \\
%% \multirow{4}{=}{\centering\arraybackslash в) кен орнын игерудің жалпы
%% тәртібін өзгерту} & 1. Жұмыс аймағын тереңдету &
%% \multirow{3}{=}{\centering\arraybackslash 1. Борт учаскесінде тереңдігі
%% бойынша және жоспарда тұрақты жылдамдықпен үздіксіз орын ауыстыру} \\
%% & 2. Жұмыстардың алаңдық дамуы \\
%% & 3. Кезеңдік қазып-игеру \\
%% & 4. Оқшауланған карьер-учаскелермен өңдеу & 2. Борттың немесе оның
%% учаскесінің кезең-кезеңмен жылжуы \\
%% \multirow{3}{=}{\centering\arraybackslash с) Кен массасын алу орнының
%% карьер ауданы бойынша өзгеруі} & 1. Жоспарда кемер учаскелерін өңдеу &
%% \multirow{3}{=}{\centering\arraybackslash 1. Тереңдігі бойынша және
%% жоспарда ауыспалы жылдамдықпен борт учаскесін жылжыту
%% 
%% 2. Борт учаскесін кезең-кезеңмен ауыстыру} \\
%% & 2. Борттар учаскесін биіктігі бойынша өңдеу \\
%% & 3. Борттар мен кемерлер учаскелерін консервациялау \\
%% \multirow{3}{=}{\centering\arraybackslash d) өңделетін борттардың
%% конструкциясы мен еңісінің өзгеруі} & 1. Жұмыс алаңдарының жобалық енін
%% өңдеу & 1. Шоғырлану аймақтарын өңдеу \\
%% & 2. Ауыспалы ені бар алаңдармен өңдеу & 2. Шоғырлану аймақтарын
%% өңдеу \\
%% & 3. Тік бортпен өңдеу & 3. Тік бортты жылжыту \\
%% \end{longtable}

Тау-кен техникалық сипаттамаларға негізделген және таңдалған әдіспен
айқындалатын берілген даму векторы кезінде мүмкін болатын карьердің
әлеуетті өнімділігі ең жоғары экономикалық қайтарымға қол жеткізуге
мүмкіндік береді, тау-кен техникалық факторлар мен сипатталған әдісті
қолдана отырып, тау-кен жұмыстарын дамытудың айқындалатын бағыты
негізінде есептелген карьердің қол жеткізілетін өнімділігі ең жоғары
экономикалық тиімділікті қамтамасыз етеді. Графоаналитикалық талдау
коэффициентінің максималды мәнін сақтай отырып, қазып алу ісін
жүргізудің оңтайлы бағытын анықтауға мүмкіндік берді, бұл ресурстарды
жалпы қазып алу коэффициентін арттыруға әкелді {[}5, 6{]}.

{\bfseries Қорытынды.} Тереңдетудің оңтайлы бағытын анықтаудың ұсынылған
графоаналитикалық әдісін қолдану саласын атап өткен жөн. Темір жол
көлігін пайдаланған жағдайда көлік схемасын ескере отырып, карьерді
дамыту стратегиясын қалыптастыруға қойылатын талаптар ескерілмейді.
Циклдық-ағынды технологияны қолдану кезінде нұсқалық есептеулер жүргізу
талап етіледі. Дегенмен, бұл талдау процесінде анықталған қолданыстағы
критерилиалды тәуелділіктер мен негізгі заңдылықтарды жоймайды {[}5, 6,
7{]}.

Жоғары көрсеткіштер мен тиімділікке қол жеткізуге әсер ететін түйінді
фактор көп компонентті кен орындарын пайдалану кезінде ресурстарды
ұтымды пайдалану қағидаттарына толық сәйкес келетін тау-кен
процестерінің даму бағытын дұрыс айқындау болып табылады.

Көп компонентті кендердің әртүрлі типтері мен сорттарын селективті
өндіру және қайта өңдеу жағдайларында кешенді кен орындарын игеру
кезінде ашу коэффициентінің ең төменгі критерийі тау-кен жұмыстарын
дамытудың ұтымды тәртібін таңдау үшін пайдаланылмайды, өйткені ол
пайдалы қазбаларды өндіруге және қайта өңдеуге жұмсалатын шығындардағы
айырмашылықтарды және тауарлық өнімнің бірнеше түрін алуды ескермейді
бірдей емес құндылық.

Минералдық ресурстарды кешенді игеру кезінде карьердегі тау-кен
жұмыстарын дамытудың оңтайлы бағытын айқындау әдістерінің бірі - бұл
бірдей сапа нормасымен пайдалы қазбаларды өндірудің тең көлемін
көрсететін оқшаулағыштарды құруға негізделген графоаналитикалық тәсіл.
Дәстүрлі тәсілдерге қарағанда кешенді кен орындарын игеру шарттарында
тиімдірек екендігі салыстырмалы талдау арқылы көрсетілді.

Көрсетілген суреттерде тау-кен жұмыстары басталғаннан бері тау-кен
қазбаларын төсеудің екі әртүрлі учаскесі (бағыты) үшін жинақталған
өндірілген пайдалы қазбалардың жиынтық көлемін көрсетеді. Бұл кесте
олардың тиімділігін, игеру жылдамдығын және таңдалған уақыт аралығындағы
жалпы өндіруге жалпы үлесін көрнекі бағалауға мүмкіндік береді. Басқа
кен орындарына бейімдеу үшін нарықтық бағалар мен өңдеу
технологияларының өзгеруіне сезімталдықты талдау жүргізу ұсынылады {[}8
- 11{]}.

{\bfseries Әдебиеттер}

1. Шитарев В.Г., Салманов О.Н. Параметры карьеров при комплексном
использовании недр. // Недра. -1990. - 112 с. ISBN: 5-247-00778-6.

2. Кожантов А.У. Открытая разработка комплексных месторождений при
взаимосвязанных процессов горного производства (при добыче комплексных
полезных ископаемых из недр земли). // Монография. - Алматы: НАО КазНИТУ
им. К.И. Сатпаева. - 2025. - 134 с. ISBN 978-601-323-571-4.

3. Гурьевский Б.А., Мустафина А.М., Долгов Ю.Ф., Евсин В.Г., Каратаев
М.М. Скоростное строительство и освоение глубоких карьеров. // Алма-Ата.
- 1977. - 255с.

4. Кенжебаев А. Комплексное обеспечение рационального недропользования
при открытой разработке крутопадающих залежей. Дис\ldots{}
докт.техн.наук: 05.15.03.- // Алматы, 1997. - 381с.

5. Трубецкой К.Н. и др. Справочник - Открытые горные работы / М. Недра.
- 1994. -590c. ISВN 5 -900697-01-0

6. Дауренбекова А. Н., Кожантов А.У. Основы горной технологии. //Учебное
пособие. Алматы: КазНИТУ имени К.И. Сатпаева. -2017.--174с. ISBN
978-601-323-073-03

7. Дауренбекова А. Н., Кожантов А.У. Тау-кен технологиясының негіздері:
Оқу құралы. //Алматы: Қ.И.Сәтбаев атындағы ҚазҰТЗУ. - 2017. - 161 б.
ISBN 978-601-323-056-6

8. Liu J. B., Dai H. Y., Wang H., Shynar A. and Madimarova G.
Three-Dimensional Geological Modeling of Mining Subsidence Prediction
Based on the Blocks // Advanced Materials Research. - 2014.- Vol.905.-
P.697-701.
\href{https://doi.org/10.4028/www.scientific.net/AMR.905.697}{DOI
10.4028/www.scientific.net/AMR.905.697}.

9. Lozynskyi V, Falshtynskyi V, Kozhantov A, Kieush L and Saik P.
Increasing the underground coal gasification efficiency using
preliminary electromagnetic coal mass heating // IOP Conf. Ser. Earth
Environ. Sci. - 2024.- Vol.1348(1): 012045.012045
\href{https://doi.org/10.1088/1755-1315/1348/1/012045}{DOI
10.1088/1755-1315/1348/1/012045}.

10. Sayk P., Lozinsky V., Anisimov O., Akimov O, Kozhantov A., Mamaikin
O. Managing the process of underground coal gasification // Naukovyi
Visnyk Natsionalnoho Hirnychoho Universytetu - 2023. - № 6. - P.25-30.
\href{https://doi.org/10.33271/nvngu/2023-6/025}{DOI
10.33271/nvngu/2023-6/025}

11. Rakishev B.R., Auezova A.M., Kuttybaev A.E., Kozhantov A.U.
Specifications of the rock massifs by the block sizes// Naukovyi Visnyk
Natsionalnoho Hirnychoho Universytetu. - 2014. -№ 6. - P.22-27.

{\bfseries References}

1. Shitarev V.G., Salmanov O.N. Parametry kar' erov pri
kompleksnom ispol' zovanii nedr. // Nedra. -1990. - 112
s. ISBN: 5-247-00778-6. {[}in Russian{]}

2. Kozhantov A.U. Otkrytaja razrabotka kompleksnyh mestorozhdenij pri
vzaimosvjazannyh processov gornogo proizvodstva (pri dobyche kompleksnyh
poleznyh iskopaemyh iz nedr zemli). // Monografija. - Almaty: NAO
KazNITU im. K.I. Satpaeva. - 2025. - 134 s. ISBN 978-601-323-571-4.
{[}in Russian{]}

3. Gur' evskij B.A., Mustafina A.M., Dolgov Ju.F., Evsin
V.G., Karataev M.M. Skorostnoe stroitel' stvo i osvoenie
glubokih kar' erov. // Alma-Ata. - 1977. -- 255 s. {[}in
Russian{]}

4. Kenzhebaev A. Kompleksnoe obespechenie racional' nogo
nedropol' zovanija pri otkrytoj razrabotke
krutopadajushhih zalezhej. Dis\ldots{} dokt.tehn.nauk: 05.15.03.- //
Almaty, 1997. - 381s. {[}in Russian{]}

5. Trubeckoj K.N. i dr. Spravochnik - Otkrytye gornye raboty / M. Nedra.
- 1994. -590c. ISVN 5 -900697-01-0/ {[}in Russian{]}

6. Daurenbekova A. N., Kozhantov A.U. Osnovy gornoj tehnologii.
//Uchebnoe posobie. Almaty: KazNITU imeni K.I. Satpaeva. -2017.--174s.
ISBN 978-601-323-073-03/ {[}in Russian{]}

7. Daurenbekova A. N., Kozhantov A.U. Tau-ken tehnologijasynyң
negіzderі: Oқu құraly. //Almaty: Қ.I.Sәtbaev atyndaғy ҚazҰTZU. - 2017. -
161 b. ISBN 978-601-323-056-6. {[}in Kazakh{]}

8. Liu J. B., Dai H. Y., Wang H., Shynar A. and Madimarova G.
Three-Dimensional Geological Modeling of Mining Subsidence Prediction
Based on the Blocks // Advanced Materials Research. - 2014.- Vol.905.-
P.697-701.
\href{https://doi.org/10.4028/www.scientific.net/AMR.905.697}{DOI
10.4028/www.scientific.net/AMR.905.697}.

9. Lozynskyi V, Falshtynskyi V, Kozhantov A, Kieush L and Saik P.
Increasing the underground coal gasification efficiency using
preliminary electromagnetic coal mass heating // IOP Conf. Ser. Earth
Environ. Sci. - 2024.- Vol.1348(1): 012045.012045
\href{https://doi.org/10.1088/1755-1315/1348/1/012045}{DOI
10.1088/1755-1315/1348/1/012045}.

10. Sayk P., Lozinsky V., Anisimov O., Akimov O, Kozhantov A., Mamaikin
O. Managing the process of underground coal gasification // Naukovyi
Visnyk Natsionalnoho Hirnychoho Universytetu - 2023. - № 6. - P.25-30.
\href{https://doi.org/10.33271/nvngu/2023-6/025}{DOI
10.33271/nvngu/2023-6/025}

11. Rakishev B.R., Auezova A.M., Kuttybaev A.E., Kozhantov A.U.
Specifications of the rock massifs by the block sizes// Naukovyi Visnyk
Natsionalnoho Hirnychoho Universytetu. - 2014. -№ 6. - P.22-27.

\emph{{\bfseries Автор туралы мәліметтер}}

Кожантов А.У. - т.ғ.к., Сәтбаев университеті, Алматы, Қазақстан, e-mail:
a.kozhantov@satbayev.university;

Абен Е.Х. - т.ғ.к., қауымдастырылған профессор, Сәтбаев университеті,
Алматы, Қазақстан, e-mail:
y.aben@satbayev.university;

Мырзахметов С.С. - т.ғ.к., қауымдастырылған профессор, Сәтбаев
университеті, Алматы, Қазақстан, e-mail:
s.myrzakhmetov@satbayev.university;

Ахметканов Д.К. -- т.ғ.к., қауымдастырылған профессор, Сәтбаев
университеті, Алматы, Қазақстан Республикасы, e-mail:
d.akhmetkanov@satbayev.university.

\emph{{\bfseries Information about the authors}}

Kozhantov A.- Candidate of Technical Sciences, Satbayev University,
Almaty, Kazakhstan, e-mail: a.kozhantov@satbayev.university;

Aben Y. - Candidate of Technical Sciences, Associate Professor of
Satbayev University, Almaty, Kazakhstan, e-mail:
y.aben@satbayev.university;

Myrzakhmetov S. - Candidate of Technical Sciences, Associate Professor
of Satbayev University, Almaty, Kazakhstan, e-mail:
s.myrzakhmetov@satbayev.university;

Akhmetkanov D. - Candidate of Technical Sciences, Associate Professor of
Satbayev University, Almaty, Kazakhstan, e-mail:
d.akhmetkanov@satbayev.university.
