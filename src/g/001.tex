\id{IRSTI 52.13.07}{}

{\bfseries THERMAL AND CRYOGENIC FRACTURE INTERACTIONS IN GRANITE: FRACTURE
TOUGHNESS TESTS FOR GEOTHERMAL DEVELOPMENT}

{\bfseries \tsp{1}D.T.Begaliyev}{\bfseries \envelope ,
\tsp{1}N.M.Zhumagaliyev}{\bfseries ,
\tsp{1}A.S.Kupeshova}{\bfseries ,
\tsp{2}A.T.Zholbasarova}{\bfseries \envelope ,
\tsp{1}A.B.Kydrashov}{\bfseries \envelope ,
\tsp{1}B.Zh.Burkhanov}{\bfseries ,
\tsp{3}A.R
Abdiev}

\emph{\tsp{1}NJSC ``West Kazakhstan Agrarian and Technical
University named after Zhangir khan'',Uralsk, Kazakhstan,}

\emph{\tsp{2}NJSC ``Caspian State University of Technologies
and Engineering named after S.Yessenov'', Aktau, Kazakhstan,}

\emph{\tsp{3}Kyrgyz Mining and Metallurgical Institute named
after academician U. Asanaliev, Bishkek, Kyrgyzstan}

\corrauthor{Corresponding-authors:dastan.begaliyev@wkau.kz,}
a.kydrashov@mail.ru

The use of liquid nitrogen (LN₂) as a freezing agent has been shown to
help enhance the permeability of the rock, making it easier for fluids
to pass through, which improves how much water or steam can flow from
the reservoir. This is especially useful in Hot Dry Rock (HDR) systems.
This study looks at how exposing granite to low temperatures affects its
strength after it has been cooled down after being heated. Fracture
Toughness Tests were done on granite samples after they were first baked
at different temperatures from 100\,°C to 400\,°C before being quickly
cooled with LN₂ to copy what happens when something gets suddenly cooled
quickly.

The tested granite samples were taken from a place that is around 50 km
away from Astana, in the Akmola region of Kazakhstan. Experimental
measurements helped find important mechanical values like the maximum
load the sample could handle, how much stress builds up at the tip of
the crack, and how much energy it takes to make the sample break. The
results show that using liquid nitrogen makes it much easier for cracks
to start and spread in the sample, and doing LN₂ treatment at higher
pre-treatment temperatures makes it happen even faster. A consistent
decrease in both the stress intensity factor and fracture energy as the
temperature went up could be seen. These findings show that using LN₂ to
freeze and crack the rocks can be a useful way to help boost the flow of
geothermal energy from certain types of hard rocks.

{\bfseries Keywords}: geothermal energy, Kazakhstan, cryogenic fracturing,
liquid nitrogen, fracture toughness test, intensity factor, fracture
energy, granite, HDR( hot dry rocks).

{\bfseries ГРАНИТТЕГІ ТЕРМИЯЛЫҚ ЖӘНЕ КРИОГЕНДІК СЫНЫҚТАРДЫҢ ӨЗАРА
ӘРЕКЕТТЕСУІ: ГЕОТЕРМИЯЛЫҚ ДАМУҒА АРНАЛҒАН СЫНЫҚТАРДЫҢ}

{\bfseries БЕРІКТІГІН СЫНАУ}

{\bfseries \tsp{1}Д.Т. Бегалиев\envelope ,
\tsp{1}Н.М. Жумагалиева, \tsp{1}А.С. Купешова,
\tsp{2}А.Т. Жолбасарова, \tsp{1}А.Б.
Кыдрашов\envelope ,}

{\bfseries Б.Ж. \tsp{1}Бурханов, \tsp{3}А.Р.
Абдиев}

\emph{\tsp{1}«Жәңгір хан атындағы Батыс Қазақстан
аграрлық-техникалық университеті» КеАҚ ,Орал, Қазақстан,}

\emph{\tsp{2}«Ш. Есенов атындағы Каспий технологиялар және
инжиниринг университеті» КеАҚ, Ақтау, Қазақстан,}

\emph{\tsp{3}Академик У. Асаналиев атындағы Қырғыз тау-кен
металлургиялық институты, Бішкек, Қырғызстан,}

e-mail:dastan.begaliyev@wkau.kz,
a.kydrashov@mail.ru

Сұйық азотты (LN₂) мұздату агенті ретінде пайдалану геотермалды
потенциалы бар тау жыныстарының өткізгіштігін арттыруға көмектесетіні
дәлелденді. Бұл сұйықтықтардың өтуін жеңілдетеді және сол арқылы жер
асты резервуарларынан су немесе бу шығу мүмкіндігін жақсартады. Ондай
тау жыныстары әдетте ыстық құрғақ күйде болады (hot dry rock, HDR). Бұл
зерттеу гранитті қыздырып, оны сұйық азотпен тез арада салқындатудан
кейін оның беріктігіне қалай әсер ететінін қарастырады. Бұл жүргізілген
сынақтар 100 °C-тан 400 °C-қа дейінгі әртүрлі температурада қыздырылған
граниттерді қамтыды.

Сыналған гранит үлгілері Астанадан шамамен 50 км қашықтықта,
Қазақстанның Ақмола облысында орналасқан жерден алынды. Эксперименттік
өлшеулер гранит тастарының көтере алатын максималды жүктеуін,
жарықшақтың ұшында қанша қысым жиналатыны және үлгіні бұзу үшін қанша
энергия қажет болатыны сияқты маңызды механикалық мәндерді табуға
көмектесті. Нәтижелер сұйық азотты пайдалану гранит тастарындағы
жарықтардың пайда болуын және таралуын айтарлықтай жеңілдететінін, ал
өңдеуге дейінгі жоғарырақ температурада қыздырылуы және сұйық азотқа
батырылуы оны одан да жылдамырақ ететінін көрсетеді. Температура
көтерілген сайын қысым қарқындылығы факторының да, сыну энергиясының да
тұрақты төмендеуін байқауға болады. Бұл нәтижелер құрғақ ыстық тау
жыныстарын мұздату және жару үшін LN₂ пайдалану қатты жыныстардың
белгілі бір түрлерінен геотермалдық энергия ағынын арттырудың пайдалы
әдісі болуы мүмкін екенін көрсетеді.

{\bfseries Түйін сөздер}: геотермалды энергия, Қазақстан, криогендік жару,
сұйық азот, жарықшаққа төзімділік сынағы, қарқындылық коэффициенті,
жарықшақ энергиясы, гранит, HDR (құрғақ ыстық жыныстар).

{\bfseries ВЗАИМОДЕЙСТВИЕ ТЕРМИЧЕСКОГО И КРИОГЕННОГО РАЗРУШЕНИЯ ГРАНИТА:
ИСПЫТАНИЯ НА ТРЕЩИНОСТОЙКОСТЬ ПРИ РАЗРАБОТКЕ ГЕОТЕРМАЛЬНЫХ
МЕСТОРОЖДЕНИЙ}

{\bfseries \tsp{1}Д.Т. Бегалиев\envelope ,
\tsp{1}Н.М. Жумагалиева, \tsp{1}А.С. Купешова,
\tsp{2}А.Т. Жолбасарова,}

{\bfseries \tsp{1}А.Б.
Кыдрашов}{\bfseries \envelope ,
\tsp{1}Б.Ж. Бурханов, \tsp{3}А.Р. Абдиев}

\emph{\tsp{1}НАО «Западно-Казахстанский аграрно-технический
университет им. Жангир хана», Уральск, Казахстан,}

\emph{\tsp{2}НАО «Каспийский университет технологии и
инжиниринга им. Ш.Есенова»,Актау, Казахстан,}

\emph{\tsp{3}Кыргызский горно-металлургический институт им.
академика У.Асаналиева ,Бишкек, Кыргызстан,}

e-mail:dastan.begaliyev@wkau.kz,
a.kydrashov@mail.ru

Было доказано, что использование жидкого азота (LN₂) в качестве
замораживающего агента помогает повысить проницаемость породы, облегчая
прохождение жидкостей, что позволяет увеличить объем воды или пара,
которые могут вытекать из резервуара. Это особенно полезно для горячих
сухих горных пород (hot dry rock, HDR). В этом исследовании
рассматривается влияние воздействия низких температур на прочность
гранита после его охлаждения после нагрева. Испытания на
трещиностойкость были проведены на образцах гранита после того, как они
были сначала нагреты при различных температурах от 100 °C до 400 °C, а
затем быстро охлаждены с помощью LN₂, чтобы воспроизвести термошок,
когда что-то внезапно быстро охлаждается.

Протестированные образцы гранита были взяты из места, расположенного
примерно в 50 км от Астаны, в Акмолинской области Казахстана.
Экспериментальные измерения помогли определить важные механические
параметры, такие как максимальная нагрузка, которую может выдержать
образец, величина напряжения, возникающего на вершине трещины, и
количество энергии, затрачиваемой на разрушение образца. Результаты
показывают, что использование жидкого азота значительно облегчает
образование и распространение трещин в образце, а обработка LN₂ при
более высоких температурах предварительной обработки позволяет сделать
это еще быстрее. С повышением температуры наблюдалось последовательное
снижение коэффициента интенсивности напряжений и энергии разрушения. Эти
результаты показывают, что использование LN₂ для замораживания и
растрескивания горных пород может быть полезным способом увеличения
потока геотермальной энергии из определенных типов твердых пород.

{\bfseries Ключевые слова}: геотермальная энергия, Казахстан, криогенный
разрыв, жидкий азот, испытание на устойчивость к разрушению, коэффициент
интенсивности напряжений, энергия разрушения, гранит, HDR (горячие сухие
породы).

{\bfseries Introduction.} With the world looking to reduce emissions and
use environmentally friendly energy, geothermal power has become vital
for making this happen. Interest has grown in Hot Dry Rock (HDR)
formations due to their ability to provide temperatures that range from
150 °C to over 500°C at depths of around 5000-6000 meters {[}1-4{]}.
Innovative though they are, HDR systems do not work as well for business
purposes due to their naturally low ability to transfer heat. As a
result, it is important to apply new stimulation methods to unleash all
the advantages of these reservoirs {[}5-9{]}.

One popular trend now uses liquid nitrogen (LN₂) in what is known as
cryogenic fracturing. LN₂-based stimulation uses the low temperature and
pressure conditions at -196\,°C to quickly contract high-temperature
rock. The heat shock leads to the development of many small fractures
that aid the movement of fluids within the rock {[}10-14{]}. Unlike
traditional hydraulic fracturing which can lead to chemical pollution
and wastes a lot of water, LN₂ stimulation is both cleaner and more
sustainable. The brittleness of granite at low temperatures allows it to
form many fractures, contributing to the overall development of complex
systems {[}15-20{]}.

This research focuses on checking how granite from the Akmola region in
Kazakhstan responds when it is subjected to cryogenic fracturing. For
our experiment, granite samples were first heated to different high
temperatures and rapidly cooled with liquid nitrogen. Then, the fracture
toughness was tested to measure the thermal damage done. Cracking
behavior was investigated by using fracture energy and mode I stress
intensity factor at different thermal conditions. This study aims to
find better, more environmentally friendly ways to stimulate wells by
examining the effectiveness of using liquid nitrogen on HDR geothermal
reservoirs.

{\bfseries Materials and methods.} Granite samples were collected from
rocks that stick out of the ground not far from Vyacheslavka in Akmola,
Kazakhstan. For the fracture toughness tests, the samples were made into
a semicircular shape, which can be seen in Figure 1. Prior to testing,
all the specimens were dried out in an oven for about 24 hours at 50
degrees to get rid of any moisture on their surface. They were then put
in a furnace and heated for two hours at 100, 200, and 400 degrees
Celsius. After heating, the samples were lowered straight into a liquid
nitrogen bath for an hour to simulate exposure to cryogenic temperatures
(such as those used in space). To make sure there was no more moisture
left, the samples were dried again at 50 ⁰C for an hour. The fracture
toughness tests were then carried out with the GCTS UCT-1000 testing
machine, which you can see on the right in Figure 1.

\fig{g/image2}{}

{\bfseries Fig.1 - Granite specimen dimensions and setup}

{\bfseries Results and discussion.} The graph in Figure 2 shows the results
of fracture toughness tests done on granite after subjecting them to
different hot and cold temperatures. The sample only treated at 50\,°C
acted as the reference, and the other samples were heated to 100\,°C,
200\,°C, and 400\,°C before being frozen in liquid nitrogen for one
hour. Each test was duplicated twice to improve the reliability of the
data and cut down on measurement mistakes. The specimens without any
treatment reached their highest peak load at 50\,°C, while peak load in
the others became lower with both thermal and cryogenic treatment. With
temperatures going up during preheating, more and larger microcracking
was found. Elevated temperatures caused more damage to the material,
which LN₂ only made worse, resulting in a drop in its ability to hold
weight.

\fig{g/image3}{}

{\bfseries Fig.2 - Load vs crack mouth opening curves from fracture
toughness tests}

The same fracture toughness test samples showed the load-displacement
curves that can be seen in Figure 3. The untreated samples at 50 ⁰C had
the biggest strengths and could handle more pressure in the experiments,
since their graphs showed steeper curves. In contrast, when samples were
put through a mix of heat and freezing, they were less strong when
stretched and moved more before breaking, especially in the samples that
were heated up to 400°C and then cooled down with liquid nitrogen. This
behavior shows that over time, the material gets weaker because of
things like expansion when it gets hot, the development of tiny cracks,
and shrinkage when it gets cold. As these effects add up, the stone
slowly gets softer and less sturdy, which then results in it being less
able to hold up to tension.

\fig{g/image4}{}

{\bfseries Fig.3 - Load vs axial displacement curves after fracture
toughness tests}

The results of Figure 4 show the stress intensity factor results from
fracture toughness tests, with each pre-treatment compared to the
reference group preheated at 50\,°C. When the pre-heating temperature is
raised, the material strength is seen to decrease, indicated by a drop
in peak stress intensity factors. In both experiments, the sharpest drop
in the value of stress intensity factor was found under the 400\,°C +
LN₂ condition, with a value of 0.91 MPa·m¹ᐟ². Cooling the granite
rapidly using LN₂ after heating it to a high temperature turned out to
be the best type of experiment for inflicting internal damage on the
sample. Matching results between both experiments and the trend seen
support the accuracy of the findings.

\fig{g/image5}{}

{\bfseries Fig.4 - Mode I stress intensity factor vs crack mouth opening
curves after fracture toughness tests}

Figure 5 highlights the difference in mechanical behavior of granite
when subjected to thermal and cryogenic loading, by using fracture
energy as a measure. Fracture energy shows the total effort needed to
break apart a sample of rock and is a sign of toughness. The best
fracture-resistance was seen in the granite samples preheated only to
50\,°C, as they were still largely intact and without damage.

Fracture energy was reduced by an average of 0.906 J when the granite
was heated at 100\,°C and then quickly cooled with LN₂ in the first two
experimental trials. It seems that when rocks go through a brief heating
period and then a sudden, extreme cooling, it causes internal fractures
that lower the strength of the rock. If thermal pre-treatment is made
more severe, the material becomes more vulnerable to cracking and
breaking. Compressive tests for samples treated at 200\,°C and then
exposed to LN₂ resulted in a fracture energy of 0.939 J, showing that
the material had grown more fragile.

Among all cases, the ones heated at 400°C and cooled with LN₂ showed the
lowest fracture energy of 0.831 J. The significant change from the
reference condition means that major internal damage and a loss of
bonding between materials occurred as a result of the thermal and
cryogenic stress during the test. All in all, the results clearly show
that higher preheat temperatures before LN₂ treatment result in
considerably softer granite, proving the method's success in improving
fracturing of rocks for geothermal use.

\fig{g/image6}{}

{\bfseries Fig.5 - Fracture energy vs process after fracture toughness
tests}

Table 1 displays the fracture toughness results of granite specimens put
through different thermal shock conditions. Without any treatment, the
K\tsb{I} factor was measured at 1.71 MPa·m¹ᐟ². Still, raising
the temperature of the specimens and then suddenly lowering it with
liquid nitrogen caused an obvious drop in their fracture toughness.
There was a drop of 17.8\% in the K\tsb{I} value in both
trials when the temperature was set to 100\,°C and LN\tsb{2}
was added. Significant decreases in content occurred at higher degrees
of preheating. The first and second trials both saw decreases of 31.4\%
and 28.4\% after using 200\,°C + LN₂. Using 400\,°C + LN₂ caused the
most noticeable declines, with a change of 45.8\% and 47.4\% in trials 1
and 2, respectively. We clearly see a certain pattern in the results. A
higher temperature during the preheating stage tends to bring down the
fracture toughness of granite before it is treated with LN₂. The
research confirms that the level of damage to rocks after LN₂ exposure
increases along with the duration of preheating, showing a proper
relationship between the damage caused by long-term exposure and
high-temperature conditions.

{\bfseries Table 1 - Effect of increase in temperature along with
LN\tsb{2} to the peak values of mode I intensity factor
values}

%% \begin{longtable}[]{@{}
%%   >{\centering\arraybackslash}p{(\linewidth - 10\tabcolsep) * \real{0.0980}}
%%   >{\centering\arraybackslash}p{(\linewidth - 10\tabcolsep) * \real{0.1065}}
%%   >{\centering\arraybackslash}p{(\linewidth - 10\tabcolsep) * \real{0.1584}}
%%   >{\centering\arraybackslash}p{(\linewidth - 10\tabcolsep) * \real{0.1808}}
%%   >{\centering\arraybackslash}p{(\linewidth - 10\tabcolsep) * \real{0.2230}}
%%   >{\centering\arraybackslash}p{(\linewidth - 10\tabcolsep) * \real{0.2332}}@{}}
%% \toprule\noalign{}
%% \begin{minipage}[b]{\linewidth}\centering
%% {\bfseries Specimen}
%% \end{minipage} & \begin{minipage}[b]{\linewidth}\centering
%% {\bfseries Treatment}
%% \end{minipage} & \begin{minipage}[b]{\linewidth}\centering
%% {\bfseries Peak value of K\tsb{I}}
%% \end{minipage} & \begin{minipage}[b]{\linewidth}\centering
%% {\bfseries Stepwise Difference}
%% \end{minipage} & \begin{minipage}[b]{\linewidth}\centering
%% {\bfseries Difference from Baseline}
%% \end{minipage} & \begin{minipage}[b]{\linewidth}\centering
%% {\bfseries Average value of K\tsb{I} peaks}
%% \end{minipage} \\
%% \midrule\noalign{}
%% \endhead
%% \bottomrule\noalign{}
%% \endlastfoot
%% ~ & ~ & {\bfseries (MPa*m\tsp{1/2})} & {\bfseries (\%)} &
%% {\bfseries (\%)} & {\bfseries (MPa*m\tsp{1/2})} \\
%% 1 & \multirow{2}{=}{\centering\arraybackslash No} & 1.72 &
%% \multirow{2}{=}{\centering\arraybackslash -} &
%% \multirow{2}{=}{\centering\arraybackslash -} &
%% \multirow{2}{=}{\centering\arraybackslash 1.71} \\
%% 2 & & 1.69 \\
%% 1 & \multirow{2}{=}{\centering\arraybackslash 100+LN\tsb{2}} &
%% 1.41 & -17.8 & -17.8 &
%% \multirow{2}{=}{\centering\arraybackslash 1.40} \\
%% 2 & & 1.39 & -17.7 & -17.7 \\
%% 1 & \multirow{2}{=}{\centering\arraybackslash 200+LN\tsb{2}} &
%% 1.18 & -16.5 & -31.4 &
%% \multirow{2}{=}{\centering\arraybackslash 1.19} \\
%% 2 & & 1.21 & -13.1 & -28.4 \\
%% 1 & \multirow{2}{=}{\centering\arraybackslash 400+LN\tsb{2}} &
%% 0.93 & -21.0 & -45.8 &
%% \multirow{2}{=}{\centering\arraybackslash 0.91} \\
%% 2 & & 0.89 & -26.5 & -47.4 \\
%% \end{longtable}

{\bfseries Conclusion.} Akmola granite samples were studied through an
experimental process using liquid nitrogen (LN₂) temperatures to see how
it affected preconditioned granite. The evaluation involved carrying out
fracture toughness tests, checking the load-displacement curve, and
getting readings of the fracture energy and stress intensity factors.
The key findings from the study are as follows:

- Preheating over 400\,°C with LN\tsb{2} caused a continuous
reduction in load-bearing, and the group with the highest preheated
temperature showed the least ability to support a high load.

- When materials were pre-treated at higher temperatures along with
liquid nitrogen, the amount of fracture energy they absorbed reduced.
Combining heat and cryogenic cracking greatly reduced the amount of
effort required to overcome the resistance of the granite.

- Calculations of K\tsb{I}, a standard measure for toughness,
revealed that LN₂ makes cracks grow more effectively at higher
preheating temperatures.

- All of the samples proved to fail in the same qualitative way during
treatment under LN₂, still variations were noticed in the fracture
energy and toughness at different temperature levels.

The results taken together show that exposing granite to heat and
flash-freezing it significantly reduces its strength, making it a useful
strategy for helping break up rocks in geothermal energy production.

\emph{{\bfseries Funding}. This research was financially supported by
Nazarbayev University (Funder Project Reference:021220CRP2022).}

\emph{I would like to express my sincere gratitude to Dr. Sotirios
Longinos for his valuable guidance, support, and insightful suggestions
throughout the course of this research}.

{\bfseries References}

1. McDaniel B.W., Grundmann S.R., Kendrick W.D., Wilson D.R., Jordan
S.W. Field applications of cryogenic nitrogen as a hydraulic fracturing
fluid // SPE annual technical conference and exhibition. Society of
Petroleum Engineers. - 1997. DOI
\href{https://doi.org/10.2118/38623-MS}{10.2118/38623-MS}.

2. Anderson R.L., Ratcliffe I., Greenwell H.C., Williams P.A., Cliffe S.
and Coveney P.V. Clay swelling-a challenge in the oilfield //
Earth-Science Reviews. - 2010. - Vol.98(3). - P.201-216. DOI
10.1016/j.earscirev.2009.11.003.

3. Longinos S.N. and Hazlett R. Cryogenic fracturing efficacy in granite
rocks: Fracture toughness and brazilian test differences after elevated
temperatures and liquid nitrogen exposure //~Geoenergy Science and
Engineering.- 2025.- Vol.244. -P.213436. DOI
10.1016/j.geoen.2024.213436.

4. Dai F. and Xia K.W., 2013. Laboratory measurements of the rate
dependence of the fracture toughness anisotropy of Barre granite
//~International Journal of Rock Mechanics and Mining Sciences.- 2013. -
Vol.~60. - P.57-65. DOI 10.1016/j.ijrmms.2012.12.035.

5. Longinos S.N., Skrzypacz P. and Hazlett R. Cryofracturing
effectiveness using liquid nitrogen on granite: Fracture toughness and
Brazilian tests following freezing-thawing cycle// Results in
Engineering.- 2024. - Vol.21 (2): 101895. DOI
\href{http://dx.doi.org/10.1016/j.rineng.2024.101895}{10.1016/j.rineng.2024.101895}.

6. Zhuang D., Yin T., Li Q., Wu Y., Chen Y. and Yang Z.,. Fractal
fracture toughness measurements of heat-treated granite using hydraulic
fracturing under different injection flow rates // Theoretical and
Applied Fracture Mechanics.- 2022.-Vol.119:103340. DOI
\href{https://doi.org/10.1016/j.tafmec.2022.103340}{10.1016/j.tafmec.2022.103340}.

7. Longinos S., Tuleugaliyev M., Begaliyev D. and Hazlett R., 2023.
Effects of High Temperature and Liquid Nitrogen Cooling: A Case Study of
granite rocks from Kazakhstan //~CEST-18th International Conference on
Environmental Science and Technology. -- 2023. DOI
\href{http://dx.doi.org/10.30955/gnc2023.00161}{10.30955/gnc2023.00161}.

8. Yang R., Hong C., Liu W., Wu X., Wang T., Huang Z. "Non-contaminating
cryogenic fluid access to high-temperature resources: Liquid nitrogen
fracturing in a lab-scale Enhanced Geothermal System // Renewable
Energy. -2021. - Vol.165. - P.125-138. DOI
10.1016/j.renene.2020.11.006.

9. Ishida T., Aoyagi K., Niwa T., Chen Y., Murata S., Chen Q., Nakayama
Y. Acoustic emission monitoring of hydraulic fracturing laboratory
experiment with supercritical and liquid CO\tsb{2 //}
Geophysical Research Letters. - 2012. - Vol.39(16): 16309. DOI
\href{https://doi.org/10.1029/2012GL052788}{10.1029/2012GL052788}.

10. Gallup D.L. Production engineering in geothermal technology: a
review // Geothermics. - 2009. - Vol.38(3). - P.326--334.
\href{https://doi.org/10.1016/j.geothermics.2009.03.001}{DOI
10.1016/j.geothermics.2009.03.001}.

11. Begaliyev D.~Laboratory investigation of cryofracturing potential to
stimulate geothermal reservoirs in Kazakhstan~/ Doctoral dissertation,
Nazarbayev University. - 2024.

12. Zhang D. and Tingyun Y. Environmental impacts of hydraulic
fracturing in shale gas development in the United States // Petroleum
Exploration and Development. - 2015. - Vol.42(6). - P.876-883. DOI
\href{https://doi.org/10.1016/S1876-3804(15)30085-9}{10.1016/S1876-3804(15)30085-9}.

13. Yin T., Bai L., Li X., Li X. and Zhang, S. Effect of thermal
treatment on the mode I fracture toughness of granite under dynamic and
static coupling load // Engineering Fracture Mechanics. -- 2018. -Vol.
199. - P.143-158. DOI
\href{https://doi.org/}{10.1016/j.engfracmech.2018.05.035}.

14. Cai C., Huang Z., Li G., Gao F., Wei J. and Li R. Feasibility of
reservoir fracturing stimulation with liquid nitrogen jet // Journal of
Petroleum Science and Engineering. - 1016. -Vol.144. - P.59-65. DOI
10.1016/j.petrol.2016.02.033.

15. Longinos S.N., Begaliyev D. and Hazlett R. Experimental and
simulation study of cryogenic stimulation of granites from Akmola region
in Kazakhstan // Geomechanics for Energy and the Environment. - 2025. -
Vol.~41. - P.100635. DOI
\href{https://doi.org/10.1016/j.gete.2024.100635}{10.1016/j.gete.2024.100635}.

16. Zhenlong G., Sun Q., Yang T., Luo T., Hailiang J., and Yang D.
Effect of high temperature on mode-I fracture toughness of granite
subjected to liquid nitrogen cooling //Engineering Fracture Mechanics. -
2021. - Vol.252:107834. DOI
\href{https://doi.org/10.1016/j.engfracmech.2021.107834}{10.1016/j.engfracmech.2021.107834}.

17. Breede K., Dzebisashvili K., Liu X., Falcone G. A systematic review
of enhanced (or engineered) geothermal systems: past, present and
future//Geotherm Energy.- 2013.-Vol.11-4) - P.2-27.DOI
10.1186/2195-9706-1-4.

18. Ge Z., Sun Q., Xue L. and Yang T. The influence of microwave
treatment on the mode I fracture toughness of granite//~Engineering
Fracture Mechanics.-2021.-Vol.249:107768. DOI
\href{https://doi.org/10.1016/j.engfracmech.2021.107768}{10.1016/j.engfracmech.2021.107768}.

19. Longinos S.N. Pore structure analysis by mercury intrusion and
nitrogen adsorption after LN2 treatment: An experimental study for
granite rocks in `O'field, Kazakhstan //~Geoenergy Science and
Engineering. -- 2025. -- Vo.252. - P.213953. DOI
\href{https://doi.org/10.1016/j.geoen.2025.213953}{10.1016/j.geoen.2025.213953}.

20. Zhang S. et al. Numerical and experimental analysis of hot dry rock
fracturing stimulation with high-pressure abrasive liquid nitrogen jet
// Journal of Petroleum Science and Engineering. - 2018. - Vol.163. -
P.156-165. DOI
\href{https://doi.org/10.1016/j.petrol.2017.12.068}{10.1016/j.petrol.2017.12.068}.

\emph{{\bfseries Information~about~the~authors}}

\emph{{\bfseries \hfill\break
}}Begaliyev D.T.- NJSC ``West Kazakhstan Agrarian and Technical
University named after Zhangir khan'', master of technical sciences,
Uralsk, Kazakhstan,
dastan.begaliyev@wkau.kz;

Zhumagaliyeva N.M.- NJSC ``West Kazakhstan Agrarian and Technical
University named after Zhangir khan'', master of technical sciences,
Uralsk, Kazakhstan,
zhumagaliyeva.nurzhamal@inbox.ru;

Kupeshova A.S.- NJSC ``West Kazakhstan Agrarian and Technical University
named after Zhangir khan'', senior lecturer, Uralsk, Kazakhstan,
kupeshova.altynay@mail.ru;

Zholbasarova A.T.- NJSC ``Caspian State University of Technologies and
Engineering named after S.Yessenov'', candidate of technical sciences,
associate professor, Aktau, Kazakhstan,
аkshyryn.zholbassarova@yu.edu.kz;

Kydrashov A.B.- NJSC ``West Kazakhstan Agrarian and Technical University
named after Zhangir khan'', PhD doctor, associate professor, Uralsk,
Kazakhstan,
a.kydrashov@mail.ru;

Burkhanov B.Zh.- NJSC ``West Kazakhstan Agrarian and Technical
University named after Zhangir khan'', candidate of technical sciences,
associate professor, Uralsk, Kazakhstan,
aruka73@mail.ru;

Abdiev A.R.- Kyrgyz Mining and Metallurgical Institute named after
academician U.Asanalieva, doctor of technical sciences, professor,
Bishkek, Kyrgyzstan,
arstanbek.abdiev@kstu.kg.\
