\id{IRSTI 86.40}{}

{\bfseries SAFETY ENGINEERING FOR WIND TURBINE HEIGHT OPERATIONS: A
CRITICAL REVIEW OF ISO/OSHA/EN STANDARDS}

{\bfseries \tsp{1}A.M. Koshayeva}
{\bfseries \envelope ,
A.B.Bekmagambetov}
{\bfseries ,
\tsp{1}A.M.Rakhmetova}
{\bfseries ,
\tsp{1}E.A.Kulmagambetova}
{\bfseries ,
\tsp{2}G.K.Daumova}


\emph{\tsp{1}Republican Research Institute for Occupational
Safety, Astana, Kazakhstan,}

\emph{\tsp{2}Kazakhstan Regional Branch of the Republican
Research Institute for Occupational Safety, Ust-Kamenogorsk, Kazakhstan}

\corrauthor{Corresponding author:koshayeva12@gmail.com}

The article examines the main standards that apply to high-altitude work
in the wind energy industry. Such documents are necessary to regulate
and ensure safety at all stages of the life cycle of "green" energy
production, with special attention to be paid to the most hazardous
areas of activity, including high-altitude work. Based on this, a
critical analysis of the requirements and standards in the selected
countries is carried out, the standards used and the percentage of
uncountable cases in the process of high-altitude work at stations are
compared. At the same time, each country adheres to the requirements of
uniform international standards, but, if necessary, adopts its own
requirements, specifically for certain types of stations, types of work,
etc. In conclusion, a number of problematic aspects are put forward, the
solution of which will contribute to increasing the level of safety at
wind stations.

{\bfseries Keywords:} wind energy, green energy, wind power plants,
high-altitude work, standards, safety, labor protection, injuries.

{\bfseries ЖЕЛ ТУРБИНАЛАРЫНДАҒЫ ТЕХНИКАЛЫҚ ЖҰМЫСТАРДЫҢ ҚАУІПСІЗДІГІ:
ISO/OSHA/EN СТАНДАРТТАРЫН ТАЛДАУ}

{\bfseries \tsp{1}А.М.Кошаева\envelope ,
\tsp{1} Ә.Б.Бекмағамбетов, \tsp{1}А.М.Рахметова,
\tsp{1} Э.А.Құлмағамбетова, \tsp{2}Г.К.Даумова}

\emph{\tsp{1}Еңбекті қорғау жөніндегі Республикалық
ғылыми-зерттеу институты, Астана, Қазақстан,}

\emph{\tsp{2}Еңбекті қорғау жөніндегі Республикалық
ғылыми-зерттеу институтының Шығыс-Қазақстан өңірлік бөлімшесі, Өскемен,
Қазақстан,}

e-mail:
koshayeva12@gmail.com

Мақалада жел энергетикасы саласындағы биіктік жұмыстарына қолданылатын
негізгі стандарттар қаралады. Мұндай құжаттар «жасыл» энергия өндірудің
өмірлік циклінің барлық кезеңдерінде қауіпсіздікті реттеу және
қамтамасыз ету үшін қажет, бұл ретте қызметтің неғұрлым қауіпті
түрлеріне, оның ішінде биіктік жұмыстарына ерекше назар аударылады. Осы
негізде таңдалған елдерде талаптар мен стандарттарға сыни талдау
жүргізіледі, пайдаланылатын стандарттар салыстырылады және
станциялардағы биіктік жұмыстары процесінде есептеуге жатпайтын
жағдайлар пайызы салыстырылады. Бұл ретте әрбір ел бірыңғай халықаралық
стандарттардың талаптарын ұстанады, бірақ қажет болған кезде арнайы
станциялардың белгілі бір типтері, жұмыс түрлері және т.б. үшін өзінің
жеке талаптарын қабылдайды. Қорытындысында жел станцияларындағы
қауіпсіздік деңгейін арттыруға ықпал ететін бірқатар проблемалық
аспектілер алға тартылады.

{\bfseries Түйін сөздер:} жел энергиясы, жасыл энергия, жел электр
станциялары, биіктік жұмыстары, стандарттар, қауіпсіздік, еңбекті
қорғау, жарақаттар.

{\bfseries БЕЗОПАСНОСТЬ ТЕХНИЧЕСКИХ РАБОТ НА ВЕТРОВЫХ ТУРБИНАХ: КРИТИЧЕСКИЙ
АНАЛИЗ СТАНДАРТОВ ISO/OSHA/EN}

{\bfseries \tsp{1}А.М.Кошаева\envelope ,
\tsp{1}А.Б.Бекмагамбетов, \tsp{1}А.М.Рахметова,
\tsp{1}Э.А.Кулмагамбетова, \tsp{2}Г.К.Даумова}

\emph{\tsp{1}Республиканский научно-исследовательский
институт по охране труда, Астана, Казахстан,}

\emph{\tsp{2}Восточно-Казахстанское региональное отделение
Республиканского научно-исследовательского института по охране труда,
Усть-Каменогорск, Казахстан,}

e-mail:
koshayeva12@gmail.com

В статье рассматриваются основные стандарты, применимые к высотным
работам в ветроэнергетической отрасли. Такие документы необходимы для
регулирования и обеспечения безопасности на всех этапах жизненного цикла
производства «зеленой» энергии, при этом особое внимание уделяется
наиболее опасным видам деятельности, в том числе высотным работам. На
этом основании проводится критический анализ требований и стандартов в
выбранных странах, сравниваются используемые стандарты и процент не
поддающихся подсчету случаев в процессе высотных работ на станциях. При
этом каждая страна придерживается требований единых международных
стандартов, но при необходимости принимает свои собственные требования,
специально для определенных типов станций, видов работ и т. д. В
заключение выдвигается ряд проблемных аспектов, решение которых будет
способствовать повышению уровня безопасности на ветровых станциях.

{\bfseries Ключевые слова:} ветровая энергия, зеленая энергия, ветровые
электростанции, высотные работы, стандарты, безопасность, охрана труда,
травмы.

{\bfseries Introduction.} The popularity of using wind turbines is growing
every year, this is due to the fact that wind energy has become one of
the fundamental alternatives in energy production. Wind energy does not
harm the environment, its production does not emit greenhouse gases and
complies with the principles of the "green" economy, which all
developing countries of the world strive to comply with. Thus, the
government of many countries is currently moving away from the use of
coal, oil, nuclear fuel and switching to the use of new, clean types of
energy {[}1{]}. In this regard, there is a significant increase in the
popularity of wind energy. Every year, the number of wind stations
around the globe increases, their combined growth rate per year is about
11.6\%.

At the same time, this growth is due to a number of factors, mainly the
availability of financial and human resources that are used in the
design and operation of installations, the achievements of scientific
and technological progress, legislative support, as well as public
assistance and assistance of numerous companies. However, with the
development of a new industry, more precise standardization is required,
as well as the development of new standards and requirements for both
equipment and workers themselves. In the event of inaccuracies in these
documents, violations or non-compliance with standards, some problems
may arise.

With the increase in the number of wind power plants, the number of
workers on them also increases, that is, the primary task should be
labor protection and their health {[}2{]}. The duties and functions
performed by workers at wind energy facilities are usually unique, this
is due to the fact that in other industries there is no installation,
assembly and maintenance of wind turbines {[}3{]}. The working
environment is also unique, which provokes possible difficulties in the
field of labor protection. Innovative work processes may contain new
hazards that employees and administrative personnel must provide for and
enshrine in the rules of conduct, as well as in the relevant documents
(standards).

Wind energy is a new, modern direction in all countries, new workers may
not fully understand what dangers await them when performing their
duties. In addition, the technology in this area is developing quite
rapidly, which can lead to a shortage of qualified personnel, and
inexperienced workers create a great danger for themselves if they have
not been trained and therefore can expose themselves to threats and
violate safety installations.

Despite the fact that wind energy is considered quite useful for the
environment, its production itself is not always useful and safe for
workers. Often, at wind farms, employees can encounter dangers that
ultimately lead to negative or fatal consequences. In this regard, the
study and critical analysis of safety standards in this area is very
relevant, since it is necessary to point out existing problem areas and
maintain a high level of labor protection. Workplaces in this area must
meet all safety requirements and not have a negative impact on the
health of workers.

Daily, in order to reduce the possibility of negative and dangerous
phenomena on personnel working at wind farms, their maintenance and
inspection are carried out. It is organized under the safest conditions,
since almost all wind turbines in developed countries have a short
service life, and their manufacturers claim that the percentage of
breakdowns is minimal, however, information about possible problems is
presented poorly and incompletely, it is mainly supplemented during
operation. Also, many workers point out that the danger in this area is
at the same level as in any other industry. For example, falls from a
height can occur both when working in the mining industry and when
working on a wind turbine.

However, it is necessary to accept and take into account other specific
conditions that may arise, for example, weather conditions or being in
hard-to-reach areas. The combination of such conditions and the lack of
experience and necessary qualifications of workers in the field of wind
energy leads to problems with their health and safety in general. Over
the past decade, according to experts, accidents at wind turbines have
become more frequent, and their number is directly dependent on the
number of new turbines, that is, the more turbines are built, the more
accidents there are.

SWIF analyzes the number and basic information about accidents and
incidents at various production facilities, including wind farms. Thus,
since 1970, the number of accidents has reached 2,500 units, with most
of them occurring in the last 3-5 years. On average, there are about 200
accidents per year that lead to negative consequences, and safety
requirements are violated. However, not all accidents are fatal; for the
same period since 1970, only 213 fatal cases were recorded. They mainly
occurred during the construction or maintenance of wind turbines and
turbines. Other cases occurred with workers in the transport sector,
they were not directly related to the production of wind energy {[}4{]}.

Also, since 2012, the company began recording data on the impact of wind
turbines on human health, data on noise, vibration, etc. were analyzed.
It is worth noting that not only workers suffer from such impacts, but
also residents of nearby settlements. In recent years, 36 requests have
been submitted and processed, in which people indicated that they have a
negative impact from wind turbines. It is worth noting that every year
the number of requests only increases, since in 2012 only 6 people
sought help, and in 2023 already 12. Many scientists believe that these
figures differ from the real ones, since wind turbines can have a more
detrimental effect on both workers and residents of nearby settlements.

Based on this, a more detailed study of labor protection in the field of
wind energy and an analysis of standards and requirements in various
countries is needed.

{\bfseries Materials and мethods.} In order to analyze the level of
injuries and safety of high-altitude work at wind power plants, it is
necessary to conduct a study of the standards that are used for labor
protection in this area, consider possible risks, the number of injury
cases, and formulate problematic aspects and directions that can be
proposed to minimize possible risks and threats. As part of the
scientific work, a critical analysis of the main safety standards of
various countries was carried out, statistics of victims of accidents
related to labor activities, including fatalities at wind power plants,
were considered.

The purpose of this scientific article is to identify the existing
methods of ensuring personnel safety at wind power plants in various
countries, as well as to study possible problematic aspects {[}5{]}.

At the same time, special methods of collecting information were used to
achieve the goal: statistical analysis, with its help the number of
injuries and other safety violations in the field of wind energy was
considered; topographic method, with its help the location of wind power
plants in a particular country was considered; monographic method, with
its help the analysis of literature on the research topic is carried
out, as well as the content of standards that are applied in this area,
but in different countries.

{\bfseries Results and discussions.} Currently, there are two main aspects
that can cause problems in the field of labor protection and safety at
wind farms. The first aspect is related to the fact that international
regulations have not been developed properly, there are problems in
standardization. The second aspect is related to the inexperience of
workers and their low level of training and qualifications.
Let' s consider these aspects in more detail.

So, to begin with, it is worth paying attention to the training of
workers in this area. As was said earlier, the wind energy industry
began to actively develop relatively recently, in this regard, there is
still no base of workers who would have significant experience and
skills in the industry in question. Despite the fact that the number of
jobs and graduates with similar training areas is growing every year,
there is still a shortage of experts. The rapid development of the
industry does not correspond to the speed of personnel training, many
inexperienced workers can participate in the process of casing
installations, turbines, which ultimately leads to safety violations and
injuries {[}6{]}. The main reason that can lead to safety violations and
injuries to workers with little experience is the lack of a single
standard that would regulate the practical skills of workers in the wind
energy sector. Currently, there are many different standards that define
the rules of conduct for workers, but there is no common one that would
regulate safety rules in the wind energy sector. Small companies do not
have the necessary funds to train workers and familiarize them with
various standards when working on land or at sea, in this regard, they
have a low level of training and qualifications, which entails possible
violations. Larger companies that have funds for training and
familiarizing workers with all standards can often neglect this, since
time is needed for training, and money can be directed to a more
attractive direction.

However, all personnel who are at the wind farm are required to know the
minimum safety requirements, which can be useful in various areas of
work. Let' s consider the safety requirements in more
detail.

The first group of requirements relates to activities at height. In this
case, the employee must know the safety precautions when working with
turbines, take into account all the risks and dangers that he may
encounter. He must also know the rules for handling personal protective
equipment and the rules of conduct in an emergency. The employee must
have rescue and evacuation devices. The second group of requirements
includes the rules of conduct in the water, the employee must know how
to move from a ship to a pier or to a wind turbine, how the global
marine communication system is used, what personal protective equipment
exists and how they are used in the water. The third group includes the
rules for manual handling of goods, it includes knowledge of the methods
of safe transportation of goods and special equipment for work,
identifying aspects that can negatively affect health and cause injury,
ways to prevent traumatic events and solve problems in the field of wind
energy associated with improper movement of cargo and equipment {[}7{]}.
The fourth is the standard rules for first aid, this includes knowledge
and skills in providing assistance in case of injuries and illnesses,
rules of conduct in emergency situations, methods of providing first aid
to victims. The final group is behavior in case of fire hazard, workers
must know the rules for preventing fires, be able to identify sources of
ignition and quickly respond to a possible fire, assess the consequences
of a fire. In general, it is worth noting that all companies operating
in this industry strive to provide their employees with all the
necessary knowledge, skills and abilities, with which they can not only
perform assigned tasks, but also ensure a high level of safety. For
this, special courses, training and retraining programs, training
programs can be created, but at the same time, they are not always
sufficient in full. According to scientists, in order to better improve
the level of knowledge of personnel in the field of safety, it is
necessary to create special courses on the basis of research institutes
that would contain programs aimed at training workers in the wind energy
sector. Within the framework of such programs, courses on the operation
of turbines, their maintenance and safety at stations can be organized.

With the development of the wind energy sector, more and more experts in
this field appear, with their help it is possible to reduce the shortage
of unskilled personnel, transfer useful experience and information.
Also, it is worth considering the creation of a special unified standard
that will help in training new specialists. In addition, another
solution for attracting highly qualified personnel may be their
transition from other related industries. Such specialists will also be
in demand in wind energy, since they will have experience and all the
necessary knowledge. An example can be workers from the oil or gas
sector, since their working conditions are similar to the area in
question. All the presented conditions will have a positive impact on
the wind energy industry in one way or another {[}8,9{]}.

Next, it is worth moving on to considering specific requirements and
standards that can be used in the field of labor protection in the
framework of wind energy in different countries.

To begin with, it is worth paying attention and considering what
standards in the field of safety of high-altitude work at a wind power
plant are adopted in Kazakhstan. As in other developed countries, the
government of Kazakhstan is aimed at developing a "green" economy, in
this regard, a decision has also been made to develop the wind energy
sector. Thus, with the help of investments of about 100 billion dollars,
the development of a new type of energy in the country began in 2014. In
order for the new industry to bring only a positive effect within the
framework of the development program of the Republic of Kazakhstan and
the joint project of the Ministry of Energy and Mineral Resources of the
Republic of Kazakhstan, special sites for the construction of wind power
systems (TPS) with a good wind climate, located in various regions of
Kazakhstan were selected (Figure 1).

\fig{g/image12}{}

{\bfseries Fig.1 - Location of wind turbines in the Republic of
Kazakhstan}

Also, as part of the study, energy scientists developed a special atlas,
which indicates areas with the strongest wind flows. An approximate
assessment of the wind energy resources of Kazakhstan is included in the
atlas of the main winds, it indicates that in "50,000 sq. at a height of
80 meters on an area of more than km, the
average annual wind speed is more than 7 m / s. Thus, based on the
presented data, investment projects were prepared, which were then
implemented and thereby new wind stations were built. So, over time,
their total capacity reached \$ 3 billion per year. kWh, these stations
can generate about 1000 MW of electricity. "Based on the indicators of
the program, which is aimed at the development of this area, it follows
that by 2030 the capacity of these installations should reach about \$ 5
billion per year. kWh. Noticeable prospects are due to the fact that the
Republic of Kazakhstan is rich in winds, it is necessary to use them in
the right direction and produce "clean" energy.

According to some data, the theoretical wind generator of Kazakhstan
amounted to about 1820 billion dollars kWh.hour per year. "Considering
the power density of the wind power plant at the level of 10 MW / km2
and the presence of significant free spaces, it is possible to predict
the possibility of installing a wind power plant capacity of several
thousand MW in Kazakhstan. The average annual wind speed in the
Dzungarian Gate area at a height of 50 meters is 9.7 m / s, and the wind
flow density is about 1050 W / m2. This allows you to produce 4400 kW
per year. Here, a special place is occupied by the electric power
purpose. The availability of free space is about 1 billion dollars
allows you to install a wind power plant capacity of several MW with an
annual output of kWh electricity is required. Currently, a pilot thermal
power plant with a capacity of 5 MW is being built in this area. If the
experience is successful, the operational capacity of the thermal power
plant can be increased to 50 MW."

As in other developed countries, wind energy does not cease to lose its
momentum of development and in Kazakhstan, following foreign experience,
more and more construction of new wind installations is planned, the
first stations were launched in the capital of the Republic of
Kazakhstan, then, having shown high results, construction continued in
other areas, so stations were opened in Mangistau and Akmola regions
{[}10{]}. Thus, currently in the Republic of Kazakhstan there are about
83 stations, the total capacity of which reaches 1410 MW.

In order to organize the work of these stations in the country, one of
the leading and successful companies currently operates, VISTA
INTERNATIONAL LLP. In 2014, it implemented the first project in the
field of wind energy, so with its help, the Kordai wind power plant was
put into operation. Today, this station is considered advanced, it is
equipped with the most modern and innovative equipment, it also has the
support of various ministries and departments, including the Ministry of
Industry and New Technologies of the Republic of Kazakhstan (MIINT). To
build the station, the company' s employees considered
numerous foreign experiences, to work out issues related to safety,
equipment, technology, they adopted the experience of countries such as
China, Holland and Germany. Studying the experience in foreign
countries, calculating all the costs and logistics, a wind power plant
was built in Kazakhstan. At the same time, it had a positive effect not
only on the environment, but also provided an opportunity for the
employment of a large number of citizens. Thus, based on this
experience, other wind power plants were put into operation in the
country. At the same time, the main aspect for their successful work
remains occupational health and safety of high-altitude work.
High-altitude work at wind farms is usually associated with the
installation, maintenance and repair of equipment located at a
considerable height, which requires special attention to safety issues.
The main risks include falls from height, which can occur due to
insufficient insurance, improper use of personal protective equipment or
unexpected exposure to external factors such as strong wind or rain. For
example, strong gusts of wind can lead to loss of balance and falls,
which makes it necessary to take wind speed into account when planning
work at height. It is also important to consider that low temperatures
can cause frostbite and reduce the performance of employees, which
requires them to be especially careful and use protective clothing. In
addition, precipitation such as rain or snow can worsen the grip of
shoes on the surface, increasing the likelihood of slips and falls
{[}11{]}. Workers must be trained to recognize adverse weather
conditions and take the necessary precautions, such as the use of
special equipment and personal protective equipment. It is also worth
noting that climate change as a result of global warming may increase
the frequency of extreme weather events, which requires a revision of
safety standards and adaptation of working practices at height.

Taking climate factors into account in the process of risk assessment
and planning of work at wind farms is becoming essential to ensure the
safety and health of workers, as well as to effectively perform tasks in
a changing environment. An important aspect is the need for regular
monitoring of meteorological conditions. The use of modern technologies,
such as weather stations and weather forecasting systems, allows for
early assessment of risks and the adoption of measures to minimize them.
For example, if strong winds are expected, work at height may be
suspended until conditions improve.

Practical risk analysis and assessment in the context of high-altitude
work at wind farms in Kazakhstan require an integrated approach that
takes into account both the specific conditions of the region and the
general trends of the transition to a green economy. An important aspect
is the identification of risks associated with working at height,
including both physical and organizational risks. In order to assess how
safety is organized at wind farms in the Republic of Kazakhstan, it is
necessary to first analyze the total number of workers at all facilities
(Figure 2).

{\bfseries Fig.2 - Total number of personnel and those engaged in
high-altitude work in the Republic of Kazakhstan, people}

Thus, it can be noted that every year there is a steady increase in the
number of personnel, while about 65\% of the total number is engaged in
high-altitude work, this is the personnel who are directly involved in
servicing turbines, their repair and daily inspection. In order to
ensure the safety of their activities in the field of labor protection,
certain standards in the field of wind energy have been developed.
Further, it is worth analyzing in more detail the content of the
presented safety standards and identifying how rationally they are
applied in practice. The first standard is the IEC 61400-2:2013 Standard
for low-power wind turbines. Often, when characterizing wind turbines,
many imagine huge turbines that are located at the height of a
twenty-story building, however, over time, slightly different
installations have also begun to gain popularity, they are slightly
smaller in size, mainly such installations are located near residential
buildings, while their power is also lower than that of large
installations. Such devices are covered by the text of the IEC
61400-2:2013 standard "Wind turbines - Part 2: Low-power wind turbines"
{[}12{]}.

This standard is necessary to regulate the safety rules for handling
such small installations. It specifies the requirements for the quality
of materials, recommendations for the efficiency of their use,
installation, maintenance, and operation rules. In addition, the
standard contains safety requirements for the subsystems of these
installations, including a description of how the protection mechanism
is arranged, how the supporting structures, foundation, connections, and
much more should be installed.

Next comes the ASTM D5741 standard, it is necessary to correctly select
the surface and location of the wind turbine. In order for the
installation to produce maximum power, it must be directed to the place
where the strongest air flow is located. With the help of this standard,
it is possible to estimate the wind at the surface and at a certain
height. In addition, this standard defines the conditions under which
the operation of turbines is possible, and which weather conditions are
considered unsuitable for its maintenance. Also, this document can be
applied in other areas of industry and in agriculture. To meet all the
requirements, in accordance with the standard, the company must purchase
special devices for changing. wind speed, temperature, etc., their
presence guarantees the reliability of the result and the correct
implementation of the theoretical aspects specified in this act.

Other standards, which are developed on the basis of consensus and can
be applied to other areas, also make an important contribution to the
development of wind energy. One of these is ISO 29404:2015 and ISO
29400:2015, they are necessary for the correct operation of offshore
wind turbines. Such installations are considered the most complex, they
are located in the open sea or ocean, and they also assume the highest
efficiency {[}13{]}.

Currently, in the Republic of Kazakhstan, it is planned to install
similar wind turbines in the Caspian Sea, the frontal plans should
assume that work in this area should be carried out only by experts in
the field of wind energy, who are familiar with all the safety rules not
only on the shore, but also in the water. In order to increase the level
of safety in this case, knowledge of the ISO 29404:2015 Standard "Ships
and marine technology - Offshore wind energy - Data flow within the
supply chain" is required. This standard defines the content and format
of messages that allow monitoring the physical movement of wind
turbines, as well as their components, both during the construction
phase and at the start of operation. These requirements and messages can
be useful not only at the beginning of work on a wind farm, but also at
the time of its breakdowns and maintenance. The ISO 29404:2015 standard
can be used by all companies and their employees when performing
activities with wind turbines, turbines and other components. However,
the standard does not apply to substations. A slightly different ISO
29400:2015 Standard "Ships and marine technology - Offshore wind energy
- Port and marine operations" is also necessary to ensure safety during
work with offshore wind turbines. It defines the procedure for offshore
operations, provides recommendations for the design of various
installations associated with the deployment or repair of offshore
helicopters. In particular, it contains requirements for the design and
use of equipment, how systems and equipment are analyzed, methods and
procedures for the safe execution of work on offshore wind turbines.

The next Standard is IEC 60076-16 Ed.1.0 b:2011 for power transformers
for wind turbines. Transformers are the main and indispensable component
of each wind farm, they are necessary so that the generated energy
obtained by wind can be converted into alternating current, which is
then used in the central network. Such devices are covered by the IEC
60076-16 Ed.1.0 b:2011 "Power transformers - Part 16: Transformers for
wind turbines", despite the fact that basically all safety requirements
concern high-altitude work performed with turbines, it is worth
remembering that improper operation and maintenance of transformers can
also lead to negative consequences for both workers and the population
of the country {[}10{]}.

The final standard that is used in this area is Standard AS 4959-2010
"Measurement, prediction and evaluation of noise from wind turbines", it
also does not define safety requirements for high-altitude work, but is
no less important, since it defines noise and vibration requirements,
their excess can also have a negative impact on workers and residents of
the country.

Thus, it can be said that in the Republic of Kazakhstan a large number
of standards are used that are aimed at improving safety when working
with wind turbines at height. However, despite this, injuries and other
accidents occur. According to the National Statistics Bureau of the
Republic of Kazakhstan and information from the Ministry of Labor and
Social Protection of the Population of the Republic of Kazakhstan, over
the past 10 years, about 18 thousand people have been injured in the
industrial sector while performing assigned tasks. Of these, 2,356
people were injured from falling from a height, of which about 96 people
were engaged in the maintenance and repair of wind turbines. A
significant number of victims is due to the fact that many workers do
not have sufficient knowledge and experience when working at wind farms
and high-altitude work, do not comply with safety requirements and are
not familiar with the applicable standards.

Let us consider the dynamics of injuries and deaths during high-altitude
work at wind farms in the Republic of Kazakhstan (Figure 3).

{\bfseries Fig.3 - Number of victims, including fatalities, during
high-altitude work at wind farms in the Republic of Kazakhstan, people}

That is, it can be said that about 0.4\% of wind farm workers are
injured and neglect safety requirements when performing high-altitude
work. The most common cases of injury or death are falls from height,
and falls from different heights and to different depths (in the case of
wind farms located at sea) can cause harm to a person. In addition,
injuries are also possible when falling on a flat surface of the same
level. Falls account for more than 50\% of all accidents during
high-altitude work at wind farms. Similarly, falls from height and to
depth can lead to fatal consequences.

Another group of factors that leads to injuries during high-altitude
work is associated with damage to mechanisms; injuries occur when
working with faulty turbine blades. Also, this group can include cuts
and abrasions that occur when in contact with faulty tools, burns from
contact with harmful substances. In addition, other types of injuries
include those associated with improper actions or failure to use
personal protective equipment (PPE) {[}14{]}. For example, in practice
in the Republic of Kazakhstan there were cases of injury to a worker due
to wear of a safety rope and slings.

In this case, it is worth saying that workers at wind farms in the
Republic of Kazakhstan are aware that work is not provided for during
bad weather, in this regard, injuries associated with wind and
precipitation are quite rare. That is, measures in this area are being
implemented in full and even those employees who have been hired quite
recently are familiar with the standards in this area in detail.

The second country chosen for the analysis of safety standards was the
United States. Thus, OSHA standards are currently being implemented in
the country, they are rules that all organizations and employees in them
must follow in order to protect themselves from emergency situations.
They are based not only on theoretical aspects, they are based on
practice and can be supplemented and improved. OSHA standards are
divided into several groups: general industry, construction, marine
transport, agriculture. They are mandatory for implementation throughout
the country, but each state can offer its own requirements for safety
and labor protection at wind farms, in particular.

It is worth saying that, unlike Kazakhstan, wind energy is much better
developed in the USA, there are a large number of stations that are
located both at sea and on land. Also, in the USA, in general, the
"green" economy is better developed, the use of new types of energy for
electricity production in the country began in the 1970s, when the
government decided to combat the energy crisis with new technologies.
Already in 1974, a special program was created that involved research in
this area, and in 1980, the world' s first wind farm was
launched {[}15{]}.

Today, in the USA there are about 9 of the largest wind farms in terms
of capacity, they are located mainly in the states of California and
Texas, but there are also a large number of small stations that are
located in places with appropriate climatic conditions. Thus, the map of
the location of wind turbines in the USA looks like this (Figure 4).

\fig{g/image13}{}

{\bfseries Fig.4 - Location of wind farms in the USA}

Based on the number of wind farms, it is worth noting that they employ
significantly more workers than in the Republic of Kazakhstan and other
European countries. Thus, according to statistics, in 2024, more than
131 thousand people worked in this sector, which is 11 thousand more
than the previous year. That is, we can say that this industry in the
United States is attractive for employment, while out of the total
number, about 65-70\% of workers are engaged in high-altitude work
(Figure 5).

{\bfseries Fig.5 - Total number of personnel and those engaged in
high-altitude work in the USA, people}

The increase in the number of workers in this sector is due to the fact
that new wind farms have been launched and are planned to be launched in
the country between 2022 and 2026. The continued growth in the number of
stations entails the need for new labor. It is worth noting that labor
protection in the United States is carried out quite well, employers
adhere to the requirements of the OSHA standard and communicate them to
their employees, who undergo annual recertification and pass exams on
knowledge of safety standards.

Thus, according to OSHA standards, the following safety requirements
apply to high-altitude work at wind farms in the United States:

1. "Maintain vertical and horizontal lift range limits. Before using the
lift, operators must read and understand the
manufacturer' s instructions.

2. Control over power lines. Lift operators must be at least 10 meters
away from the widest power lines.

3. Use of fall protection equipment. These include seat belts,
shock-absorbing slings, and platform guards.

4. Stabilization of the lift. This is important, especially when working
in windy conditions. At wind speeds over 64.4 kilometers per hour (40
miles per hour) or 48.3 kilometers per hour (30 miles per hour), if the
work is related to the processing of materials, the employer is obliged
to take measures to protect employees from the dangerous effects of the
wind." In addition, the GWO (Global Wind Organisation) standard is also
used, it includes requirements related to working at height, actions in
case of fire hazard, first aid, survival at sea, and so on. The main
requirement is that when working at high-rise facilities, workers must
be equipped with special protective equipment, a fencing system, a
network protection system must be provided, workers must be equipped
with personal protective equipment. When performing repair work or
servicing turbines at wind farms, workers must generally be guided by
and adhere to OSHA standards, according to which special guards must be
provided when working at height, and in their absence, each worker must
have personal protective equipment. With the help of handrails or
netting, as well as personal protective equipment, it is possible to
avoid falls or other injuries {[}9{]}.

Also, workers often get injured not only when directly working with
turbines, but when climbing to them on fixed ladders. "When climbing a
fixed ladder (not 20 feet long) on these towers,
a ladder equipped with a cage or well shall have a landing platform
every 30 feet; a ladder not so equipped shall have a landing platform
every 20 feet. Ladder safety devices may be used on wind towers longer
than 20 feet in place of cage protection. In these cases, a landing
platform is not required." These requirements are also set out in the
OSHA standard.

However, despite the fact that there are safety requirements in the
United States, similar to other countries, there are often unfavorable
incidents, including injuries and fatalities. Thus, over the past 2024,
there were 240 accidents and incidents in the United States that
resulted in injuries to wind farm workers. Of these, 32 cases were
associated with strong winds, 9 with lightning strikes, another 4 with
tornadoes, cold snaps, storms and cyclones. Also, 35 cases were caused
by safety violations, 9 cases with problems in electrical voltage. At
the same time, the number of fatalities was 5 people (Figure 6). In this
case, the percentage of injuries from the total number of workers at
high-rise wind farms is about 0.3\%.

{\bfseries Fig.6 - Number of injured, including fatalities, during
high-altitude work at wind farms in the USA, people}

Based on the increase in injuries and accidents, wind energy companies
are deciding to conduct additional training and instruction for
employees. Since it is the training of new employees and monitoring
their activities that will help to avoid falls, injuries and other
injuries in the future. Also, it is important to monitor weather
conditions: in 2024, a tornado destroyed about 10 turbines in the United
States; in 2022, lightning struck several stations, which led to a fire
and fatalities. Thus, we note that it is necessary to pay attention to
safety in the field of labor protection in the event of various natural
disasters, the evacuation of employees during a tornado or storm in
particular {[}16{]}.

Also, unlike the first example, where it was said that basically all
workers do not have the proper qualifications and are quite young, the
United States has an aging workforce. Many professions in the wind
energy industry require not only physical strength, but also knowledge,
including work at height, which includes turbine repair and other
maintenance of wind turbines. This activity often has a negative impact
on the health of the staff, and then on their productivity.

Thus, elderly and mature workers are no longer able to perform the
assigned tasks, but at the same time, they have extensive experience,
which increases their expertise in all matters. In this regard, to
reduce the likelihood of accidents with elderly workers, it is necessary
to develop a schedule so that they can rest and not receive additional
stress. Also, it is necessary to give recommendations on changing
professions, moving to administrative staff or to the company office.

The third example that will be considered in this work is the European
experience. It is worth saying that the countries that prevail in the
production of "clean" energy in Europe are Germany, Great Britain and
Spain (Figure 7).

\fig{g/image14}{}

{\bfseries Fig.7 - Wind farm locations in Europe}

To begin with, let' s look at the safety standards that
are used in all European countries when performing high-altitude work at
wind farms {[}17{]}.

The first standard is the European standard EN 50308:2004 "Wind
turbines. Protection measures. Requirements for design, operation and
maintenance", it is considered the most complete, since it contains all
the necessary labor protection conditions in the field of wind energy,
and also considers all aspects from the moment of transportation of
turbines to the place of their operation. Let' s look at
the main aspects of this standard:

1. The design of wind farms should be carried out in such a way as to
reduce all possible risks to the health of workers and citizens.

2. The edges of turbines that are publicly accessible should not have
sharp corners, uneven surfaces and should not cause injury to workers.

3. Operational control of the turbine should be carried out without
climbing to it.

4. Control panels should be located in a place that prevents accidental
actions or unintentional errors of personnel.

5. During repair work or maintenance of turbines, blocking and forced
shutdown must be provided.

6. All safety measures must be observed to avoid accumulation of
hazardous toxic substances, flammable or explosive gases in any areas of
the wind turbine {[}8{]}.

It can be said that this standard, unlike those discussed earlier,
contains the most complete and detailed list of safety precautions;
familiarization with this standard can reduce the level of injuries and
other accidents at work. However, its requirements do not apply to
installations that were put into operation before 2004.

Also, when working at wind farms in Europe, quality standards ISO 9001,
ISO 14001 and OHSAS 1800 are actively used. These standards include
general requirements for labor protection, quality management,
management, however, they do not directly relate to wind energy, but
manufacturers strive to comply with them. Specifically, the IEC 61400
series applies to wind turbines; this standard covers requirements for
turbine design, reliability, service life, and installation. It contains
requirements for all aspects of the design, construction, and operation
of an offshore wind farm. This standard is necessary so that the
turbines perform their functions as reliably as possible during
operation. In addition, the National Institute for Occupational Safety
and Health (NIOSH) operates in Europe, which was organized back in 2009,
and courses and seminars are currently held on its basis, which are also
useful for health protection. An example is the seminar "Making Green
Workplaces Safe" within the framework of the "Prevention through Design"
initiative. At such meetings, specific problems in the field of labor
protection are discussed and new directions for minimizing risks in
production are proposed. Thus, one of the new issues is the need for
uniform standard codes for the entire life cycle of wind turbines
{[}18{]}.

In addition to this organization, the European Wind Energy Association
(EWEA) operates in Europe. Its mandate includes training new employees,
familiarizing them with safety rules in the industry and in specific
areas, which helps reduce risks and injuries. The organization publishes
all safety requirements that are relevant for managers and employees,
with special attention paid to first aid procedures. EWEA documents
contain requirements that relate to safety both on water bodies and on
land. EWEA proposals and organization documents can be considered
international and used for labor protection in other countries. Many
organizations that install and operate wind turbines are also interested
in developing and using safety standards. An example is RenewableUK,
which has created its own labor protection forum, which contains various
training courses, a description of currently available standards and
requirements. In addition, the G9 Wind Energy Health and Safety
Association (G9) has been created in Europe. Its main goal is safety and
labor protection in the wind energy industry. To achieve the goal, the
association' s management allocates resources for
conducting courses and training specialists, thereby reducing the risk
of injuries and other unpleasant situations. This association, in
addition to considering safety in the context of wind energy, also pays
attention to offshore operations that are associated with wind turbines
at sea or in the ocean and lifting operations that are associated with
construction.

It is worth noting that despite a fairly wide list of requirements and
standards in Europe, there is a shortage of articles on the safety of
wind turbines in the marine space, and there are also no requirements
for checking the durability of installations. For example, at present, a
standard has not been developed that would apply to medium-sized
turbines; standard requirements are used for them, similarly, there are
no requirements for small wind turbines. Despite the fact that each
company in this area is able to develop its own safety requirements,
they need to be unified within the framework of national or
international standards. Since with their help, safe work is possible.

As an example of a European country, it is worth considering the
experience of Great Britain and identifying the number of injuries and
other accidents when working at height. Thus, based on SafetyOn data,
the following information was obtained on the total number of incidents
at wind farms (Figure 8) {[}19{]}.

{\bfseries Fig.8 - Number of casualties, including fatalities, during
high-altitude work at wind farms in the UK, people.}

At the same time, according to statistics, the nature of accidents is
not specified, whether they occurred while working at height during
maintenance of wind turbines or not. However, the data show that work at
stations has a high risk to the health and safety of personnel. Workers
must understand that their health protection is exclusively in their
hands, ignoring safety requirements cannot lead to positive
consequences.

It is worth saying that the most common phenomenon that leads to
injuries and other accidents is a turbine blade breakage. Fire comes in
second place, on average, about 50 fires occur annually at various
stations. At the same time, turbine operators do not conduct a
sufficiently accurate inspection using mechanical means, which
ultimately aggravates the situation. In this case, the number of victims
is about 0.4\% of the total number of workers.

Thus, in order to systematize the information presented in the
conclusion, it is worthwhile to make a comparative analysis of the
standards considered in different countries and draw conclusions
regarding their effectiveness in application, comparing them with the
general level of injuries and other accidents during high-altitude work
at wind power stations (Table 1) {[}20{]}.

{\bfseries Table 1 - Comparative analysis}

%% \begin{longtable}[]{@{}
%%   >{\raggedright\arraybackslash}p{(\linewidth - 8\tabcolsep) * \real{0.1685}}
%%   >{\raggedright\arraybackslash}p{(\linewidth - 8\tabcolsep) * \real{0.1950}}
%%   >{\raggedright\arraybackslash}p{(\linewidth - 8\tabcolsep) * \real{0.2492}}
%%   >{\raggedright\arraybackslash}p{(\linewidth - 8\tabcolsep) * \real{0.1484}}
%%   >{\raggedright\arraybackslash}p{(\linewidth - 8\tabcolsep) * \real{0.2389}}@{}}
%% \toprule\noalign{}
%% \begin{minipage}[b]{\linewidth}\centering
%% {\bfseries Country}
%% \end{minipage} & \begin{minipage}[b]{\linewidth}\centering
%% {\bfseries Standards}
%% \end{minipage} & \begin{minipage}[b]{\linewidth}\centering
%% {\bfseries Problem areas}
%% \end{minipage} & \begin{minipage}[b]{\linewidth}\centering
%% {\bfseries Number of injuries and accidents}
%% \end{minipage} & \begin{minipage}[b]{\linewidth}\centering
%% {\bfseries Conclusion on the effectiveness of the standard}
%% \end{minipage} \\
%% \midrule\noalign{}
%% \endhead
%% \bottomrule\noalign{}
%% \endlastfoot
%% Republic of Kazakhstan & IEC 61400-2:2013
%% 
%% IEC 60076-16
%% 
%% ASTM D5741
%% 
%% ISO 29404:2015
%% 
%% ISO 29400:2015
%% 
%% AS 4959-2010 & 1.Lack of requirements for different types of wind
%% turbines.
%% 
%% 2. No mandatory safety training for young professionals.
%% 
%% 3. Fragmentation of standards and lack of clear coordination between them
%% create difficulties in their application. & 0.4\% of all those engaged
%% in high-altitude work & The most common cause of injury or death is a
%% fall from a height.
%% 
%% The second group of factors - injuries occur when working with faulty
%% turbine blades (cuts and abrasions).
%% 
%% Other types of injuries include those associated with improper actions
%% or failure to use personal protective equipment (PPE). For example, in
%% practice in the Republic of Kazakhstan there were cases of workers being
%% injured due to wear of a safety rope and slings. \\
%% United States of America & OSHA Standard
%% 
%% GWO (Global Wind Organization) Standard & 1.Safety precautions for
%% emergencies (storm, tsunami) are not fully developed.
%% 
%% 2. Low level of awareness of risks among older workers. & 0.3\% of all
%% those engaged in high-altitude work & Injuries are associated with
%% strong winds, lightning strikes, tornadoes, cold snaps, storms and
%% cyclones. Also, most cases of injuries are caused by safety violations,
%% problems with electrical voltage. \\
%% Europe (Great Britain) & European standard EN 50308:2004
%% 
%% ISO 9001
%% 
%% ISO 14001
%% 
%% OHSAS 1800 & 1.Lack of unified requirements for fire hazard.
%% 
%% 2. There is a shortage of articles devoted to the safety of wind turbines
%% in the marine space.
%% 
%% 3. There are no requirements for checking the durability of
%% installations. & 0.4\% of all those engaged in high-altitude work & The
%% most common occurrence that leads to injuries and other accidents is
%% turbine blade failure, followed by fire. \\
%% \end{longtable}

That is, it can be said that the standards governing the performance of
high-altitude work require constant updating and adaptation to modern
conditions and technologies in all countries. In particular, the study
highlights the current problems associated with the insufficient
effectiveness of existing standards, which can lead to increased risks
for workers at height {[}21{]}.

{\bfseries Conclusions.} Thus, based on the analysis of the number of
injuries and accidents at heights within wind stations, we can identify
some trends and main problem areas that are inherent in all the
countries considered. The main problem is that technical progress in the
wind energy industry is moving somewhat faster than the development of
new safety standards, and therefore even experienced workers can be
injured by improper operation, repair or maintenance of the turbine. It
can also be argued that the hazards found in wind power plants are not
too different from those that exist in other industries today (for
example, falls from height, manual handling), however, it is necessary
to take into account all the unique conditions that are present in the
wind energy industry, and therefore the use of previously adopted safety
standards and requirements for other industries is not always
appropriate and effective.

In this regard, in order to reduce injuries and accidents in the work of
wind turbines in the Republic of Kazakhstan, it is recommended to adopt
the experience of standardizing the rules of both the United States and
European countries.

It is necessary to create a specific standard that will specify all the
requirements for labor protection, as well as standards for each group
of wind turbines. For European countries, it is also recommended to
practice actions in case of fires, since some principles of fire safety
equipment of oil platforms and other energy facilities can also be
applied to turbines. For the US, it is worth paying more attention to
the occurrence of natural disasters and practicing storm safety rules
with employees. For all countries, it is preferable to study and
implement safety training for h igh-altitude work.

The basis can be the courses of the World Wind Organization (GWO). These
standards include modules on working at height, fire safety, first aid,
and other topics.

\emph{{\bfseries Funding.} The article presents the results of scientific
research obtained during the implementation of a scientific and
technical programme within the framework of programme-targeted financing
of the work of the Republican Research Institute for Occupational Safety
and Health of the Ministry of Labour and Social Protection of the
Population of the Republic of Kazakhstan (IRN BR22182667) on the topic
"Working conditions and occupational risks: classification, categories
and groupings in the context of the transition to the ``Green
Economy''".}

{\bfseries Литература}

1. Бюро национальной статистики Агентства по стратегическому планированию
и реформам республики Казахстан // Power technology.
\href{URL:\%20https://stat.gov.kz/ru/industries/social-statistics/stat-medicine/spreadsheets/}{URL:
https://stat.gov.kz/ru/industries/social-statistics/stat-medicine/spreadsheets/}.-Дата
обращения: 21.06.2025.

2. Husseini T. Golden hour: the paramedics saving lives on offshore
windfarms // Power technology. URL:
\url{https://www.power-technology.com/features/golden-hour-paramedics-saving-lives-offshore-windfarms/}.
-Дата обращения: 21.06.2025.

3. Гордиенко Д., Мордвинова А., Шебеко Ю., Лагози А., Некрасов В.
Пожарная безопасность морских стационарных платформ для добычи нефти и
газа на континентальном шельфе // Инженерная защита
научно-практический журнал. URL:
\url{http://territoryengineering.ru/bez-rubriki/pozharnaya-bezopasnost-morskih-statsionarnyh-platform-dlya-dobychi-nefti-i-gaza-na-kontinentalnom-shelfe/}.
-Дата обращения 21.06.2025.

4. Лебедева Е.О., Матузова С.Ю. Особенности обеспечения индивидуальной
безопасности при доставке персонала на нефтегазодобывающие платформы
арктического шельфа // Международный научно-исследовательский журнал.
-2014. - №1(20).- С.74-76.

5. Wind energy' s frequently asked questions (FAQ). URL:
https://www.ewea.org/wind-energy-basics/faq/ -Дата обращения:
21.06.2025.6. Wu X., Hu Y., Li Y., Yang J., et al. Foundations of offshore wind
turbines: A review // Renewable and Sustainable Energy Reviews. -2019.
-Vol.104. -P.379--393. DOI 10.1016/j.rser.2019.01.012.

7. BARD Offshore Wind Farm. -2023. URL: https://www.4coffshore.
com/windfarms/bard-offshore-1-germany-de23. -Дата обращения:
21.06.2025.8. What if one of the key solutions to fighting climate change was in our
ocean? /GWEC. -2021 URL: https://gwec.net/offshore-wind/. -Дата
обращения: 21.06.2025.

9. Global Offshore Wind Technical Potential/World Bank Group. -2023 URL:
\url{https://datacatalog.worldbank.org/search/dataset/0037787}. -Дата
обращения: 21.06.2025.

10. International Renewable Energy Agency, International Standardisation
in the Field of renewable Energy. -2013. URL:
https://www.irena.org/-/media/Files/IRENA/Agency/Publication/2013/International\_Standardisation\_-in\_the\_Field\_of\_Renewable\_Energy.pdf.
- Дата обращения: 21.06.2025.

11. International Qualification \& Training Center. // GWO BST
Occupational Safety Training. URL: https://iqtc-riga.com/. - Дата
обращения: 21.06.2025.

12. European Technology \& Innovation Platform on Wind Energy (ETIP Wind).
Strategic research and innovation agenda 2018 // European Technology
\& Innovation Platform on Wind Energy. -2018. URL:
https://etipwind.eu/library/reports/.- Дата обращения: 21.06.2025.

13. Ярош В. А., Ефанов А. В., Привалов Е. Е., Ястребов С. С. Электрические
станции и подстанции. Часть II : лабораторный практикум /Директ-Медиа.
- 2019. - 91 с. ISBN 978-5-4475-5320-3.

14. Гурин С.И. Пропаганда охраны труда // Руководитель автономного
учреждения. -2011. -№ 4.

15. Energy Institute. -2023. URL:
\url{https://www.energyinst.org/technical/}
publications/topics/offshore-safety/g9-2013-incident-data-report. -
Дата обращения: 21.06.2025.

16. Simone Peter. Women in wind energy // Women of Wind Energy Deutschland
e.V. URL: http://www.womenofwindenergy.de. - Дата обращения:
21.06.2025.17. WindEurope. Health \& safety in the wind industry. URL:
http://www.ewea.org/policy-issues/health-and-safety/gwo-standards/. -
Дата обращения: 21.06.2025.

18. Moore J., Bullard N. BNEF Executive Factbook. Power, transport,
buildings and industry, commodities, food and agriculture, capital
//BloombergNEF. -2021. -100 p.

19. RenewableUK. Health and safety on offshore windfarm // RenewableUK.
URL: http://www.renewableuk.com/en. - Дата обращения: 21.06.2025.

20. Offshore Wind Technical Potential. Analysis and Maps. URL:
https://esmap.org/esmap\_offshorewind\_tech-potential\_analysis\_maps.
- Дата обращения: 21.06.2025.

21. Castro-Santos L., Diaz-Casas V. Floating offshore wind farms /Springer
InternationalPublishing. -2016. - 193 p. DOI
10.1007/978-3-319-27972-5.

{\bfseries References}

1. Bjuro nacional' noj statistiki Agentstva po
strategicheskomu planirovaniju i reformam respubliki Kazahstan // Power
technology. URL:
https://stat.gov.kz/ru/industries/social-statistics/stat-medicine/spreadsheets/.-Data
obrashhenija: 21.06.2025.{[}in Russian{]}

2. Husseini T. Golden hour: the paramedics saving lives on offshore
windfarms // Power technology. URL:
\url{https://www.power-technology.com/features/golden-hour-paramedics-saving-lives-offshore-windfarms/.-}
Date of access:21.06.2025.

3. Gordienko D., Mordvinova A., Shebeko Ju., Lagozi A., Nekrasov V.
Pozharnaja bezopasnost'{} morskih stacionarnyh platform
dlja dobychi nefti i gaza na kontinental' nom
shel' fe // Inzhenernaja zashhita nauchno-prakticheskij
zhurnal. URL:
http://territoryengineering.ru/bez-rubriki/pozharnaya-bezopasnost-morskih-statsionarnyh-platform-dlya-dobychi-nefti-i-gaza-na-kontinentalnom-shelfe/.
-Data obrashhenija 21.06.2025. {[}in Russian{]}

4. Lebedeva E.O., Matuzova S.Ju. Osobennosti obespechenija
individual' noj bezopasnosti pri dostavke personala na
neftegazodobyvajushhie platformy arkticheskogo shel' fa
// Mezhdunarodnyj nauchno-issledovatel' skij zhurnal.
-2014. - №1(20).- S.74-76. {[}in Russian{]}

5. Wind energy' s frequently asked questions (FAQ). URL:
https://www.ewea.org/wind-energy-basics/faq/- Date of access:
21.06.2025.6.Wu X., Hu Y., Li Y., Yang J., et al. Foundations of offshore wind
turbines: A review // Renewable and Sustainable Energy Reviews. -2019.
-Vol.104. -P.379--393. DOI 10.1016/j.rser.2019.01.012.

7. BARD Offshore Wind Farm. -2023. URL: https://www.4coffshore.
com/windfarms/bard-offshore-1-germany-de23. - Date of access:
21.06.2025.8. What if one of the key solutions to fighting climate change was in
our ocean? /GWEC. -2021 URL: https://gwec.net/offshore-wind/. - Date of
access: 21.06.2025.

9. Global Offshore Wind Technical Potential/World Bank Group. -2023 URL:
\url{https://datacatalog.worldbank.org/search/dataset/0037787}. -Дата
обращения: 21.06.2025.

10. International Renewable Energy Agency, International Standardisation
in the Field of renewable Energy. -2013. URL:
https://www.irena.org/-/media/Files/IRENA/Agency/Publication/2013/International\_Standardisation\_-in\_the\_Field\_of\_Renewable\_Energy.pdf.
- Date of access: 21.06.2025.

11. International Qualification \& Training Center. // GWO BST
Occupational Safety Training. URL: https://iqtc-riga.com/. - Date of
access: 21.06.2025.

12. European Technology \& Innovation Platform on Wind Energy (ETIP
Wind). Strategic research and innovation agenda 2018 // European
Technology \& Innovation Platform on Wind Energy. -2018. URL:
https://etipwind.eu/library/reports/.- Date of access: 21.06.2025.

13. Jarosh V. A., Efanov A. V., Privalov E. E., Jastrebov S. S.
Jelektricheskie stancii i podstancii. Chast'{} II :
laboratornyj praktikum /Direkt-Media. - 2019. - 91 s. ISBN
978-5-4475-5320-3. {[}in Russian{]}

14. Gurin S.I. Propaganda ohrany truda // Rukovoditel'{}
avtonomnogo uchrezhdenija. -2011. -№ 4. {[}in Russian{]}

15. Energy Institute. -2023. URL:
\url{https://www.energyinst.org/technical/}
publications/topics/offshore-safety/g9-2013-incident-data-report. - Date
of access: 21.06.2025.

16. Simone Peter. Women in wind energy // Women of Wind Energy
Deutschland e.V. URL: http://www.womenofwindenergy.de. - Date of access:
21.06.2025.17.WindEurope. Health \& safety in the wind industry. URL:
http://www.ewea.org/policy-issues/health-and-safety/gwo-standards/. -
Date of access: 21.06.2025.

18. Moore J., Bullard N. BNEF Executive Factbook. Power, transport,
buildings and industry, commodities, food and agriculture, capital
//BloombergNEF. -2021. -100 p.

19. RenewableUK. Health and safety on offshore windfarm // RenewableUK.
URL: http://www.renewableuk.com/en. - Date of access: 21.06.2025.

20. Offshore Wind Technical Potential. Analysis and Maps. URL:
https://esmap.org/esmap\_offshorewind\_tech-potential\_analysis\_maps.-
Date of access: 21.06.2025.

21. Castro-Santos L., Diaz-Casas V. Floating offshore wind farms
/Springer InternationalPublishing. -2016. - 193 p. DOI
10.1007/978-3-319-27972-5.

\emph{{\bfseries Information about the authors}}

Koshayeva A.M. -- Master of Technical Sciences, the RRIOSH of the
Ministry of Health of the Republic of Kazakhstan, Astana, Kazakhstan,
e-mail:
koshayeva12@gmail.com;

Bekmagambetov A.B. - Higher law degree, candidate of legal sciences,
associate professor, the RRIOSH of the Ministry of Health of the
Republic of Kazakhstan, Astana, Kazakhstan, e-mail:
Adilet1979@mail.ru;

Rakhmetova A.M. - Candidate of Medical Sciences, associate professor the
RRIOSH of the Ministry of Health of the Republic of Kazakhstan, Astana,
Kazakhstan, e-mail:
ra\_anar@mail.ru;

Kulmagambetova E. A. - Candidate of Chemical Sciences, the RRIOSH of the
Ministry of Health of the Republic of Kazakhstan, Astana, Kazakhstan,
e-mail:
elya\_kulmagambet@mail.ru;

Daumova G. K.- Candidate of Technical Sciences, associate professor, the
RRIOSH of the Ministry of Health of the Republic of Kazakhstan,
Ust-Kamenogorsk, Kazakhstan, e-mail:
gulzhan.daumova@mail.ru.

\emph{{\bfseries Сведение об авторах}}

Кошаева А.М. - Магистр технических наук, РГП на ПХВ «Республиканский
научно-исследовательский институт по охране труда Министерства труда и
социальной защиты населения Республики Казахстан», Астана, Казахстан,
e-mail: \href{mailto:Koshayeva12@gmail.com}{koshayeva12@gmail.com};

Бекмағамбетов А.Б. - кандидат юридических наук, ассоциированный
профессор, РГП на ПХВ «Республиканский научно-исследовательский институт
по охране труда Министерства труда и социальной защиты населения
Республики Казахстан», Астана, Казахстан, e-mail:
Adilet1979@mail.ru;

Рахметова А.М. - кандидат медицинских наук, доцент, РГП на ПХВ
«Республиканский научно-исследовательский институт по охране труда
Министерства труда и социальной защиты населения Республики Казахстан»,
Астана, Казахстан, e-mail:
ra\_anar@mail.ru;

Кульмагамбетова Э.А. - кандидат химических наук, РГП на ПХВ
«Республиканский научно-исследовательский институт по охране труда
Министерства труда и социальной защиты населения Республики Казахстан»,
Астана, Казахстан, e-mail:
elya\_kulmagambet@mail.ru;

Даумова Г.К. - кандидат технических наук, ассоцированный профессор, РГП
на ПХВ «Республиканский научно-исследовательский институт по охране
труда Министерства труда и социальной защиты населения Республики
Казахстан», Усть - Каменогорск, Казахстан, e-mail:
gulzhan.daumova@mail.ru.\
