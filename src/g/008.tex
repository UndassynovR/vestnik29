\id{МРНТИ 87.53.13}{}

\begin{header}
\swa{}{ВЛИЯНИЕ ВНЕДРЕНИЯ ИЗМЕЛЬЧИТЕЛЕЙ ПИЩЕВЫХ ОТХОДОВ НА ЭКОСИСТЕМУ КАЗАХСТАНА}

\tsp{1}Д. С.Хайриев,
\tsp{2}А.К. Мукатай,
\tsp{3,4}Ж.Т. Даулетжанова\envelope
\end{header}

\begin{affil}
\tsp{1}ТОО «RedSeven» Астана, Казахстан,

\tsp{2}ТОО «Офис Коммерциализации» Астана, Казахстан,

\tsp{3}ТОО «Институт химии угля и технологии», Астана, Казахстан,

\tsp{4}Казахский университет технологии и бизнеса им. К. Кулажанова, Астана, Казахстан

\corrauthor{Корреспондент-автора: kaliyeva\_zhanna@mail.ru}
\end{affil}

В условиях роста доли органической фракции в составе ТКО и ограниченных
возможностей полигонного захоронения актуализируются технологии,
позволяющие перераспределить потоки пищевых отходов из твердых
коммунальных отходов в систему водоотведения. В статье рассматривается
технология измельчителей пищевых отходов (ИПО) как элемент
интегрированного управления органическими отходами и городскими сточными
водами. Цель работы - обобщить международный опыт применения ИПО и
оценить потенциальные экологические эффекты и инфраструктурные
ограничения при внедрении технологии в Казахстане. Методологическая
основа включает анализ научных публикаций и отчетов международных
организаций, сопоставление сценариев обращения с органикой
(полигон/раздельный сбор/компостирование/анаэробное сбраживание/передача
через ИПО) и качественную оценку влияния на парниковые выбросы, нагрузку
на систему очистки сточных вод и ресурсный потенциал (биогаз, извлечение
питательных веществ).

{\bfseries Ключевые слова:} пищевые отходы, измельчители пищевых отходов,
твердые коммунальные отходы, канализационные сточные воды, очистные
сооружения, анаэробное сбраживание, биогаз, экологическая оценка
жизненного цикла.

\begin{header}
ТАМАҚ ҚАЛДЫҚТАРЫН ҰСАҚТАҒЫШТАРДЫ ЕНГІЗУДІҢ ҚАЗАҚСТАН ЭКОЖҮЙЕСІНЕ ӘСЕРІ

\tsp{1}Д.С. Хайриев,
\tsp{2}А.К. Мукатай,
\tsp{3,4}Ж.Т. Даулетжанова\envelope
\end{header}

\begin{affil}
\tsp{1}«RedSeven» ЖШС, Астана, Қазақстан,

\tsp{2}«Офис Коммерциализации» ЖШС, Астана, Қазақстан,

\tsp{3}«Көмір химиясы және технологиясы институты» ЖШС, Астана, Қазақстан,

\tsp{4}Қ. Құлажанов атындағы Қазақ технология және бизнес университеті, Астана, Қазақстан,

e-mail: kaliyeva\_zhanna@mail.ru
\end{affil}

Қатты тұрмыстық қалдықтардың құрамындағы органикалық үлестің өсуі және
полигондық көму мүмкіндіктерінің шектелуі жағдайында органикалық
қалдықтарды басқарудың орнықты технологиялары маңызды бола түсуде. Бұл
мақалада тамақ қалдықтарын ұсақтағыштар (ИПО) технологиясы органикалық
қалдықтар ағынын қатты қалдықтар жүйесінен кәріз жүйесіне ауыстыратын
интеграцияланған шешім ретінде қарастырылады. Зерттеудің мақсаты - ИПО
қолданудың халықаралық тәжірибесін жинақтап, технологияны Қазақстанда
енгізу кезіндегі ықтимал экологиялық тиімділіктер мен инфрақұрылымдық
шектеулерді бағалау. Әдістемелік негіз ретінде ғылыми жарияланымдар мен
халықаралық ұйымдардың есептеріне шолу жасалды, органиканы өңдеудің
әртүрлі сценарийлері (полигон, бөлек жинау, компосттау, анаэробты ашыту,
ИПО арқылы тасымалдау) салыстырылды және парниктік газдар
шығарындыларына, кәріз желілеріне/тазарту құрылыстарына түсетін
жүктемеге, сондай-ақ ресурстық әлеуетке (биогаз, қоректік элементтерді
қалпына келтіру) ықпалдары сапалық тұрғыда талданды. Нәтижелер ИПО
енгізу тұрмыстық қалдықтардағы тағам фракциясын азайтып, полигондардағы
метан түзілуін төмендетуге мүмкіндік беретінін көрсетеді. Сонымен бірге
ағынды сулардағы органикалық жүктеме (БПК/ХПК, майлар) артып, тазарту
құрылыстарының жеткіліксіз қуаты жағдайында технологиялық тәуекелдер
туындауы ықтимал. Қорытындыда ИПО технологиясын кезең-кезеңімен,
пилоттық аумақтардан бастап енгізу, кәріз желілері мен тазарту
құрылыстарының өткізу қабілетін алдын ала бағалау, сондай-ақ анаэробты
ашыту жобаларымен үйлестіру және нормативтік базаны жаңарту қажеттігі
негізделеді.

{\bfseries Түйін сөздер}: тамақ қалдықтары, тамақ қалдықтарын ұсақтағыш,
қатты тұрмыстық қалдықтар, кәріз ағынды сулары, тазарту құрылыстары,
анаэробты ашыту, биогаз, өмірлік циклді бағалау.

\begin{header}
IMPACT OF INTRODUCING FOOD WASTE DISPOSERS ON KAZAKHSTAN'S ECOSYSTEM

\tsp{1}D.S. Khairiev,
\tsp{2}A.K. Mukatai,
\tsp{3,4}Zh.T. Dauletzhanova\envelope
\end{header}

\begin{affil}
\tsp{1}LLP ``RedSeven'', Astana, Kazakhstan,

\tsp{2}LLP ``Commercialization Office'', Astana, Kazakhstan,

\tsp{3}LLP ``Institute of Coal Chemistry and Technology'', Astana, Kazakhstan,

\tsp{4}K. Kulazhanov Kazakh University of Technology and Business, Astana, Kazakhstan,

e-mail: kaliyeva\_zhanna@mail.ru
\end{affil}

The growing share of organics in municipal solid waste (MSW) and the
limited capacity of landfilling make source-level solutions for
biodegradable waste increasingly relevant. This paper reviews food waste
disposers (FWDs) as an element of integrated management of household
organics and municipal waste\-water. The aim is to synthesize
international evidence on environmental benefits and infrastructure
const\-raints and to outline the key considerations for potential
implementation in Kazakhstan. The methodology combines a literature
review of peer-reviewed studies and institutional reports with
scenario-based compa\-rison of common organic waste pathways (landfilling,
separate collection, composting, anaerobic diges\-tion, and diversion via
FWDs into sewer systems). The analysis focuses on expected changes in
greenhouse gas emissions, the organic load to sewer networks and
wastewater treatment plants (BOD/COD and fats), and resource recovery
opportunities such as biogas production and nutrient recycling. The
reviewed evidence suggests that diverting food waste away from landfills
can reduce methane formation and lower the MSW organic fraction, while
co-treatment with sewage sludge may improve energy recovery where
anaerobic digestion is available. At the same time, widespread FWD
adoption can increase the organic and hydraulic loads on sewerage
infrastructure and treatment processes, which may pose operational risks
in systems with limited capacity. The paper concludes that Kazakhstan
should consider phased, pilot-based deployment coupled with capacity
assessments of sewer networks and wastewater treatment plants, alignment
with anaerobic digestion projects, and updates to technical and
regulatory requirements.

{\bfseries Keywords:} food waste; food waste disposers; municipal solid
waste; municipal wastewater; wastewater treatment plants; anaerobic
digestion; biogas; life cycle assessment.

\begin{multicols}{2}
{\bfseries Введение.} В условиях нарастающих экологических проблем,
связанных с переработкой и утилизацией органических отходов, а также
ограниченными возможностями полигонного захоронения, особую актуальность
приобретают технологии, обеспечивающие устойчивое обращение с
биологически разлагаемыми компонентами ТКО. Одним из таких решений
является использование измельчителей пищевых отходов (ИПО), способных
сократить нагрузку на систему сбора и транспортировки отходов, а также
уменьшить долю органики, направляемой на полигоны. Кроме того, внедрение
ИПО может повысить эффективность очистных сооружений при условии
модернизации инфраструктуры, а также снизить углеродный след и объёмы
метановых выбросов.

В работах посвящённых Казахстану, подчёркивается системный характер
отходообразования и слабая эффективность традиционного полигонного
сценария. Так, У.Б. Аскарова связывает рост негативного воздействия ТБО
с недостаточной развитостью инфраструктуры переработки и утилизации, что
усиливает экологические риски и требует модернизации подходов к
обращению с отходами {[}1{]}. В близком ключе Г.А. Джалилова описывает
«антропогенную эпоху ТКО» на примере Казахстана, акцентируя
накопительный эффект коммунальных отходов и необходимость перехода от
захоронения к более управляемым технологиям переработки {[}2{]}. Эти
работы задают контекст: пищевые отходы следует рассматривать не как
``фоновую'' фракцию ТКО, а как приоритетный объект управления из-за
высокой биодеградируемости и влияния на эмиссии при захоронении. Переход
к более эффективным моделям обращения с пищевыми отходами требует
количественной базы. В этом отношении полезны данные о морфологическом
составе ТКО и доле пищевой фракции в городских системах, включая Алматы,
представленные в источнике {[}3{]}. Несмотря на давность публикации,
подобные табличные сведения важны для обоснования актуальности темы, а
также для постановки задач исследования (оценка масштаба органической
фракции, оценка потенциального эффекта перераспределения потоков
органики). В связке с обзорными работами по Казахстану {[}1,2{]}
источник {[}3,4{]} позволяет аргументировать, что именно
органическая/пищевая фракция формирует значимую долю массы ТКО и
способна определять экологические последствия выбранной схемы обращения.
Заметный пласт литературы посвящён экономике и политике управления
пищевыми отходами - прежде всего в России, как более ``регуляторно
оформленном'' примере в регионе. А.П. Епишов систематизирует подходы
государственной политики в сфере переработки и утилизации пищевых
отходов, показывая, что институциональные механизмы (стимулы,
ответственность участников цепочки, инфраструктурные решения) определяют
практические сценарии и темпы внедрения технологий {[}5{]}. Л.К.
Субракова рассматривает экономику обращения с пищевыми отходами,
подчёркивая роль издержек логистики, сортировки/раздельного сбора и
переработки, а также вопрос распределения затрат между участниками
системы {[}6{]}. Эти источники важны для вашей темы тем, что внедрение
измельчителей пищевых отходов (ИПО) -- это не только технологический
выбор домохозяйства, но и элемент управленческой модели, влияющий на
финансовые параметры муниципальной системы: смещение органики из потока
ТКО в поток сточных вод меняет структуру затрат и ``ответственность''
между операторами ТКО и водоканалом. Для Казахстана экономическая рамка
раскрыта в работах, оценивающих политику сокращения ТБО и отношение
населения к мерам управления отходами. Так, С.С. Калиева и соавторы
применяют метод условной оценки (CV) для экономической оценки политики
по сокращению ТБО в РК, что позволяет обосновывать управленческие
решения через готовность общества поддерживать изменения и через оценку
эффектов политики {[}7{]}. В практическом поле близкую тематику
развивает Т.В. Петросянц, предлагая эколого-экономическую оценку и
варианты утилизации пищевых отходов в РК {[}8{]}. В совокупности {[}7{]}
и {[}4{]} задают важный для вашей статьи вывод: выбор технологии
(включая ИПО) должен обсуждаться не только через экологические
аргументы, но и через применимость в конкретных институциональных
условиях Казахстана (готовность платить, приоритеты модернизации
инфраструктуры, управляемость реализации).

Отдельное направление исследований касается переработки органической
фракции, включая пищевые отходы, через анаэробное сбраживание и
получение биогаза. Для обоснования технологической логики (и для
разделов ``Материалы и методы'' / ``Результаты и обсуждение'') ключевые
источники - публикации А.Я. Ванюшиной и Д.А. Даниловича. В части 1 они
позиционируют анаэробное сбраживание как базовую (ключевую) технологию
обработки осадков городских сточных вод и обсуждают её роль в
современных системах водоотведения {[}9{]}. В части 2 развивают тему,
описывая технологические аспекты и значимость процесса для устойчивости
очистных сооружений (энергоэффективность, управление осадком, интеграция
в общую схему) {[}10{]}. Для статьи это принципиально: внедрение ИПО
фактически увеличивает долю легкоразлагаемой органики, поступающей в
систему водоотведения, а значит, повышает актуальность технологий
обработки осадков (включая метантенки) и переводит вопрос ИПО из
``санитарно-бытового удобства'' в плоскость оптимизации работы городских
очистных сооружений. На уровне экспериментальной/прикладной проработки
переработки пищевых отходов внутри Казахстана важной выглядит работа
С.Б. Жапаровой и соавторов о термофильном сбраживании бытовых пищевых
отходов {[}11{]}. Термофильный режим в целом рассматривается как
потенциально более интенсивный по скорости процессов и по санитарной
безопасности продукта, но требующий более строгого управления
параметрами и энергоснабжения. Привлечение {[}11{]} позволяет вашей
статье опираться не только на концептуальные схемы, но и на региональную
научную базу, показывающую реализуемость биогазовых решений для пищевых
отходов. Это особенно полезно при обсуждении сценариев, где ИПО
рассматриваются как инструмент ``подведения'' органики к последующей
анаэробной переработке (на очистных сооружениях или на
специализированных объектах), при условии технической готовности
инфраструктуры.

На рисунке 1 представлен сравнительный анализ выбросов CO₂-эквивалента
при различных способах утилизации пищевых отходов.

\fig[0.3\textwidth]{g3/image33}[Рис.1 - Сравнительный углеродный след при утилизации пищевых отходов: свалка, компостирование, ИПО + сбраживание]

В более прикладной плоскости исследование проведем сравнение капитальных
и операционных затрат муниципалитетов до и после установки ИПО. Анализ
сводит экономическую эффективность к времени окупаемости, составляющему
в среднем от 2,5 до 4 лет, при условии получения побочных продуктов в
виде биогаза.
\end{multicols}

\tcap{Таблица 1 - Сравнительный анализ методов утилизации пищевых отходов}
\begin{longtblr}[
  label = none,
  entry = none,
]{
  width = \linewidth,
  colspec = {Q[10]Q[354]Q[140]Q[212]Q[202]},
  cells = {c},
  cells = {font = \small},
  hlines,
  vlines,
}
  & \textbf{Показатель}                  & \textbf{Свалка} & \textbf{Компостирование} & \textbf{ИПО + сбраживание} \\
1 & Средняя стоимость утилизации (USD/т) & 40–80           & 30–60                    & 20–50                      \\
2 & Среднее время разложения             & Месяцы–годы     & 2–6 месяцев              & 10–20 дней                 \\
3 & Потенциал энергопроизводства         & Нет             & Ограниченный (тепло)     & Высокий (биогаз)           \\
4 & Углеродный след (CO₂-экв.)           & 100\%           & 62\%                     & 55\%                       
\end{longtblr}

\begin{multicols}{2}
На рисунке 2 показано различие в выбросах парниковых газов в зависимости от
метода переработки отходов рассматривается роль ИПО в рамках «зелёной
сертификации» жилищных комплексов. Авторы подчёркивают, что наличие ИПО
в проекте способствует увеличению «зеленого рейтинга» и может
использоваться как критерий устойчивого домостроения.

Таким образом, анализ источников {[}1-11{]} показывает, что современная
дискуссия об обращении с пищевыми отходами в регионе строится вокруг
трёх взаимосвязанных блоков: (i) экологическая необходимость сокращения
полигонного захоронения органики и управления ТКО как антропогенным
фактором {[}1,2{]}; (ii) экономико-институциональная реализуемость и
принятие населением, определяющие выбор технологий и механизмов
внедрения {[}5-8{]}; (iii) технологическая база переработки органики
(особенно анаэробное сбраживание) как ключевой элемент устойчивой
системы, связанной как с сектором отходов, так и с сектором
водоотведения {[}10-11{]}. При этом источники указывают и на
исследовательский разрыв, который ваша статья может закрыть:
недостаточно проработан именно ``стык'' систем ТКО и водоотведения в
условиях Казахстана --- от обоснования доли пищевой фракции и мотивации
выбора технологии {[}3,4{]} до оценки готовности городской
инфраструктуры и институциональных механизмов согласования интересов
операторов ТКО и водоканала {[}5-7{]}. Это делает актуальным последующее
исследование, в котором внедрение ИПО оценивается не изолированно, а как
часть комплексной модели управления органическими отходами с учётом
технических и экономических ограничений конкретных городов Казахстана.

\fig[0.36\textwidth]{g3/image34}[Рис.2 - Сравнение выбросов парниковых газов при различных методах утилизации пищевых отходов]

{\bfseries Материалы и методы.} Использование измельчителей пищевых отходов
(ИПО) в условиях урбанизированной среды Казахстана представляет собой
как перспективную экологическую инициативу, так и вызов для устаревшей
инфраструктуры. Их внедрение оказывает разностороннее влияние на
коммунальные и экологические системы, включая твёрдые бытовые отходы
(ТБО), канализационные сети, очистные сооружения и климатические
показатели.

С одной стороны, ИПО позволяют перерабатывать органическую фракцию ТБО
на месте, что снижает общий объём вывозимых отходов на 20-25\% и
способствует уменьшению загрязнения на полигонах. Учитывая, что в
структуре ТБО до 60\% приходится на органику, установка ИПО в 50\%
домохозяйств Казахстана способна перераспределить до 1,5 млн тонн
органических отходов в канализационную систему. Это приводит к
значительному снижению массы гниющих отходов, уменьшению выбросов
метана, сокращению популяции грызунов и насекомых и снижению расходов на
вывоз мусора, особенно в малых городах.

С другой стороны, перераспределение органики в канализационные стоки
повышает нагрузку на систему водоотведения. Согласно расчётам,
биохимическая потребность в кислороде (BOD5) в сточных водах может
возрасти на 20--40\%, что требует дополнительной мощности аэротенков,
повышения аэрационной производительности и реконструкции иловых
площадок. На фоне того, что в более чем 20 городах Казахстана степень
износа очистных сооружений превышает 70-90\%, риск перегрузки
инфраструктуры возрастает. В Таразе, где отсутствуют КОС, сточные воды
сбрасываются через временные отстойники, что усугубляет угрозу
вторичного загрязнения окружающей среды.

Проведённый анализ показал, что только 36\% сточных вод в Казахстане
очищаются безопасно, в соответствии с национальными или местными
стандартами. Для сравнения, в странах с высоким уровнем экологической
эффективности, таких как Сингапур, Нидерланды или Южная Корея, этот
показатель превышает 99\%. Таким образом, при внедрении ИПО в Казахстане
необходима синхронизация технологических и нормативных мер с
национальной программой «Зелёный Казахстан».

Положительные климатические эффекты внедрения ИПО включают снижение
выбросов CO₂ и CH₄ с полигонов, потенциал повторного использования
органики для производства биогаза, а также переработку иловых осадков в
компост или удобрения. В крупных городах возможно применение технологий
анаэробного сбраживания для энергетической утилизации измельчённых
отходов. Это позволит интегрировать ИПО в круговую экономику и систему
устойчивого водоснабжения.

Тем не менее, внедрение ИПО требует комплексного подхода. Необходимо:
модернизировать существующие КОС и проектировать новые с учётом будущих
органических нагрузок; обновить СНиП и СанПиН с включением норм
эксплуатации ИПО; предусмотреть стандарты технической эксплуатации,
запрещающие сброс волокнистых и жиросодержащих отходов; обучить
операторов КОС новым требованиям; проводить просветительскую работу
среди населения о правилах эксплуатации ИПО.

По данным моделирования, установка ИПО в половине городских домохозяйств
позволит: снизить общий объём ТБО до 4-4,25 млн тонн в год (против
текущих 5,5 млн тонн); сократить частоту вывоза мусора на 30-50\%;
переработать до 1 млн тонн органических отходов на КОС с возможностью
получения биоудобрений или метана; сократить выбросы парниковых газов с
полигонов на 20-30\%.

Однако потенциальные риски включают засоры в устаревших трубах,
биообрастание и заиливание канализации, увеличение энергозатрат на
очистку сточных вод, а также вероятность эвтрофикации водоёмов при
сбросе недоочищенных стоков.

Таким образом, внедрение ИПО при текущем состоянии очистных сооружений
требует поэтапного, научно обоснованного внедрения с обязательным
технико-экономическим обоснованием. При выполнении необходимых
модернизационных мер использование ИПО может стать ключевым элементом
повышения экологической эффективности городов Казахстана и интеграции в
международные климатические инициативы. В Казахстане проведено множество
исследований, посвящённых анализу и очистке канализационных (сточных)
вод. Ниже представлена таблица с ключевыми публикациями, организациями и
направлениями исследований.

В Республике Казахстан в последние годы наблюдается рост научного
интереса к оценке состояния сточных вод, их очистке и воздействию на
окружающую среду. Исследования охватывают как лабораторные анализы
состава канализационных стоков, так и оценку состояния инфраструктуры,
разработку инновационных методов очистки и определение
токсикологического профиля иловых осадков.

Наиболее активно мониторинг качества сточных вод осуществляется на
крупных объектах, таких как ГКП "Тоспа Су" (г. Алматы), где
функционирует современная лаборатория, оснащённая химическим,
бактериологическим, гидробиологическим и физико-химическим отделениями.
Регулярный контроль осуществляется по таким показателям, как
температура, pH, содержание нефтепродуктов, тяжёлых металлов, взвешенных
веществ и микробиологическая нагрузка.

Проблемы с очисткой промышленных сточных вод остаются актуальными. По
данным ЮКГУ им. М. О. Ауэзова, многие предприятия не справляются с
технологическими нормативами по очистке, что связано с изношенным
оборудованием, отсутствием современных технологий, а также с нехваткой
квалифицированных специалистов. В качестве перспективных решений
рассматриваются внедрение мембранных биореакторов, фотокаталитической
очистки и реагентной коагуляции.

Анализ, опубликованный на платформе CyberLeninka и в других
академических источниках, свидетельствует о том, что в некоторых городах
степень износа канализационных очистных сооружений достигает 90\%.
Например, в Кокшетау, Сарань, Риддере и Аркалыке изношенность КОС
превышает допустимые эксплуатационные пределы, что ограничивает
возможность приема дополнительных органических нагрузок от ИПО.

Применение инновационных решений в Казахстане представлено, в частности,
технологией нитри-денитрификации на основе природного цеолита
Чанканайского месторождения, которая успешно используется в ряде
объектов Астаны и улучшает процессы биологической очистки. Также
проводятся токсикологические исследования иловых осадков с целью оценки
класса опасности, в том числе методом определения LD50 и биотестирования
на модельных организмах.

Таким образом, существующий научный и практический задел показывает, что
Казахстан обладает достаточным потенциалом для модернизации очистных
сооружений с ориентацией на приём сточных вод, насыщенных органикой,
включая переработанные ИПО отходы. Однако реализовать этот потенциал
возможно лишь при синхронизации технологических, нормативных и
образовательных изменений на всех уровнях.
\end{multicols}

\tcap{Таблица 2 - Основные направления исследований и мониторинг канализационных масс в Казахстане}
\begin{longtblr}[
  label = none,
  entry = none,
]{
  width = \linewidth,
  colspec = {Q[10]Q[140]Q[298]Q[460]},
  row{1} = {c},
  cell{2}{1} = {c},
  cell{2}{2} = {c},
  cell{3}{1} = {c},
  cell{3}{2} = {c},
  cell{4}{1} = {c},
  cell{4}{2} = {c},
  cell{5}{1} = {c},
  cell{5}{2} = {c},
  cell{6}{1} = {c},
  cell{6}{2} = {c},
  cells = {font = \small},
  hlines,
  vlines,
}
№  & \textbf{Направление исследования}               & \textbf{Описание}                                                                                               & \textbf{Примеры и результаты}                                                                                                                                                   \\
1. & Лабораторный анализ сточных вод                 & Изучение физико-хими- ческих и биологическихпараметров сточных вод для оценки их состава и степени загрязнения. & В лаборатории ДГКП "Тоспа Су" проводится анализ сточных вод на всех стадиях очистки, включая химической, бактерио-логический и физико-химичес-кий и физико-химический контроль. \\
2. & Оценка состояния очистных сооружений            & Анализ технического сос-тояния и эффективностиработы канализационных очистных сооружений.                       & В Казахстане очистные сооружения установлены в 62 из 89 городов, однако в некоторых городах износ сооружений превышает 90\%.                                                    \\
3. & Разработка и внедрение новых технологий очистки & Исследование и примене –ние современных методови материалов для повышения эффективнос-ти очистки сточных вод.   & В Астане применена технология нитри-денитрификации с использованием цеолита Чанканайского месторождения, что улучшило процессы биологической очистки.                           \\
4. & Токсикологиче\-ские исследования иловых осадков   & Оценка токсичности иловых осадков для определения их класса опасности и способов утилизации.                    & Проведены эксперименты по определению LD50 и установлению степени токсич-ности иловых отходов на теплокровных животных.                                                         \\
5. & Анализ промышленных сточных вод                 & Изучение состава и методов очистки сточных вод, образующихся в результате промышлен-ной деятельности.           & Рассмотрены методы очистки промышленных сточных вод, включая предварительную, биохимическую и доочистку, а также проблемы утилизации отходов очистки.                           
\end{longtblr}

\begin{multicols}{2}
Экологические и санитарные эффекты внедрения ИПО

Внедрение измельчителей пищевых отходов в масштабах городских
агломераций Казахстана имеет значительные экологические, санитарные и
климатические последствия. Одним из ключевых преимуществ технологии
является снижение доли органических компонентов в структуре твёрдых
бытовых отходов. По результатам моделирования, в случае установки ИПО в
50\% домохозяйств объём органической фракции ТБО может быть уменьшен на
30-60\% в зависимости от потребительских привычек населения и структуры
питания.

Это приводит к следующим положительным последствиям: сокращается масса
отходов, направляемых на полигоны, уменьшается частота вывоза мусора,
снижается уровень выбросов метана, образующегося при анаэробном
разложении органики, а также сокращается популяция грызунов и насекомых
в зоне хранения ТБО. Учитывая, что в Казахстане до 90\% отходов
по-прежнему захораниваются, интеграция ИПО представляет собой важный
элемент стратегии декарбонизации и устойчивого обращения с отходами.

Однако перераспределение органики из ТБО в сточные воды требует
готовности канализационной и очистной инфраструктуры. Основными рисками
являются увеличение BOD и COD показателей на 20-40\%, рост иловой
нагрузки, потенциальные засоры и биообрастание в трубопроводах при
недостаточной пропускной способности, особенно в системах с
гофрированными трубами или в населённых пунктах с септиками и выгребными
ямами.

Для минимизации рисков и обеспечения адаптивности инфраструктуры в таких
городах, как Астана, Алматы и Шымкент, возможно проведение следующих
мероприятий: увеличение мощности аэротенков, внедрение систем
анаэробного сбраживания органики с получением биогаза, обновление СНиП и
СанПиН с учётом новых органических нагрузок, а также внедрение
фильтрации тонкой очистки перед выпуском в водоёмы.

Экологический и климатический эффект заключается также в снижении
объёмов сжигаемой органики, уменьшении выбросов CO₂ и NOₓ, а также
возможности повторного использования сточных вод для орошения или
технических нужд при условии их качественной очистки. ИПО могут стать
элементом городской системы замкнутого цикла, где органика поступает в
биогазовые реакторы, а затем возвращается в виде энергии и удобрений.

Необходимые меры по успешному внедрению включают: проведение пилотных
региональных проектов, формирование нормативной базы, экономическое
стимулирование (в том числе субсидии на покупку ИПО), просветительскую
работу среди населения и обучение персонала очистных станций. Эти шаги
должны быть включены в реализацию стратегии «Зелёный Казахстан».

Таким образом, при условии технической готовности и согласованной
политики на государственном и муниципальном уровнях, массовое внедрение
ИПО способно существенно улучшить экологическую ситуацию в стране,
повысить эффективность обращения с органикой и сформировать элементы
циркулярной экономики в жилищно-коммунальном хозяйстве.

{\bfseries Результаты и обсуждение.} Моделирование снижения нагрузки на
полигоны и выбросов парниковых газов. Анализ структуры твёрдых бытовых
отходов (ТБО) в Казахстане показывает, что ежегодно формируется около
5--6 млн тонн ТБО, из которых от 40\% до 60\% составляют органические
пищевые отходы. Внедрение измельчителей пищевых отходов (ИПО) в 50\%
домохозяйств позволяет перераспределить существенную долю органики в
сточные воды, что приводит к ряду социально-экологических эффектов.

Помимо этого, сокращение объёма органики на полигонах позволяет
уменьшить выбросы метана (CH₄) на 20-30\%, что позитивно влияет на
климатическую политику страны и способствует выполнению обязательств по
Парижскому соглашению. Снижение затрат на транспортировку отходов может
достигать 20-50\%, особенно в малых и средних городах, что делает
технологию ИПО экономически привлекательной для муниципалитетов.

С другой стороны, перераспределение органики из ТБО в канализационные
стоки увеличивает нагрузку на очистные сооружения. При 100\% установке
ИПО суточный показатель BOD₅ (биохимическая потребность в кислороде)
может увеличиться на 20-40\%, а общее количество перерабатываемой
органики на КОС - на 1,2-1,5 млн тонн. Это приводит к необходимости
увеличения аэротрофных мощностей, пересмотра параметров иловых площадок
и внедрения энергоэффективных технологий очистки, таких как мембранные
реакторы и анаэробное сбраживание с получением биогаза.
\end{multicols}

\tcap{Таблица 3 - Потенциал сокращения объёма ТБО при массовом внедрении ИПО}
\begin{longtblr}[
  label = none,
  entry = none,
]{
  width = \linewidth,
  colspec = {Q[10]Q[272]Q[187]Q[315]Q[100]},
  cells = {c},
  cells = {font = \small},
  cell{5}{1} = {c=5}{},
  hlines,
  vlines,
}
№                                                    & \textbf{Показатель}       & \textbf{До внедрения ИПО} & \textbf{После внедрения ИПО (50\% домов)} & \textbf{Изменение} \\
1                                                    & Органические отходы в ТБО & 2,5-3млн тонн/год         & \textasciitilde{}1,25–1,5 млн тонн/год    & -50\%              \\
2                                                    & Общий объём ТБО           & 5,5 млн тонн              & 4-4,25 млн тонн                           & -20-25\%           \\
3                                                    & Частота вывоза мусора     & Высокая (1-3 р/нед)       & Снижена на 30–50\%                        & -                  \\
Примечание: если ИПО установлены в 50\% домохозяйств &                           &                           &                                           &                    
\end{longtblr}

\begin{multicols}{2}
Экологический эффект может быть дополнительно усилен за счёт интеграции
ИПО в биометанизационные комплексы. Биогаз, получаемый из переработанной
органики, может быть использован для генерации электроэнергии, отопления
или как транспортное топливо, тем самым снижая углеродный след и
обеспечивая устойчивое использование ресурсов.

В целом, расчёты показывают, что при внедрении ИПО в 50\% домохозяйств:
\end{multicols}

\tcap{Таблица 4 - Результаты эффективности при 50\% охвата применения ИПО}
\begin{longtblr}[
  label = none,
  entry = none,
]{
  cells = {c},
  cells = {font = \small},
  hlines,
  vlines,
}
№ & \textbf{Эффект}                                       & \textbf{Оценка}                          \\
1 & Снижение объёма свалок                                & -1,5 млн тонн/год (в среднем)            \\
2 & Снижение метановых выбросов                           & -20-30\% от текущих выбросов с полигонов \\
3 & Снижение затрат на транспортировку отходов            & -20-50\% (особенно в малых городах)      \\
4 & {Потенциал переработки на КОС в органические \\удобрения} & до 1 млн тонн/год                        
\end{longtblr}

\begin{multicols}{2}
Тем не менее, существующие инфраструктурные ограничения требуют учёта
ряда факторов: высокий уровень износа КОС (до 90\% в ряде городов),
отсутствие централизованной канализации в более чем 30\% населённых
пунктов, правовой вакуум в части применения ИПО в нормативных документах
(СНиП, СанПиН).

Таким образом, полноценное внедрение ИПО и реализация связанных
экологических выгод возможны только при параллельной модернизации
водоотводных и очистных систем, актуализации нормативной базы и запуске
региональных программ по устойчивому обращению с отходами. При
выполнении этих условий, Казахстан способен достигнуть значительного
прогресса в области декарбонизации, ресурсосбережения и устойчивого
развития коммунальной инфраструктуры.

Сточные воды - это воды, загрязненные отходами разного происхождения. В
большинстве случаев это бытовые и производственные отходы.

«В рейтинге IMD используется показатель «Безопасно очищенные сточные
воды», который рассчитывается как процент населения страны,
подключенного к очистным сооружениям. При расчете данного показателя
используется база данных ЦУР. Индикатор направлен на отслеживание
процентной доли сточных вод из различных источников (домохозяйства,
сфера услуг, промышленность и сельское хозяйство), которые очищаются в
соответствии с национальными или местными стандартами. В рейтинге IMD
2023 Казахстан по данному показателю занял 56 место со значением 35,7\%
(при этом стоит учитывать, что использовались данные за 2020 год).
Лидирующие позиции заняли следующие страны: Сингапур (100\%), Нидерланды
(99,8\%), Катар (99,5\%), Южная Корея (99,5\%) и Германия (99,3\%).
Замыкают рейтинг такие страны, как Таиланд (24,4\%), Колумбия (21,3 \%)
и Монголия (10,4\%)», -- сообщили авторы исследования. «При этом
наибольший процент очистки сточных вод имеют такие страны, как Кувейт,
Сингапур, Андорра, Гибралтар и Катар (все страны имеют стопроцентную
очистку). Многие развивающиеся страны обрабатывают менее 10\% (Малави,
Сенегал, Мали). Казахстан проводит безопасную очистку 36\% сточных вод».

В ходе проведённого научного исследования обоснована высокая
экологическая и инфраструктурная значимость применения измельчителей
пищевых отходов (ИПО) как эффективного инструмента управления
органической фракцией твёрдых коммунальных отходов (ТКО) в условиях
урбанизированной среды. Анализ современных технических решений,
зарубежного и казахстанского опыта, а также результаты моделирования
показали, что внедрение ИПО позволяет существенно сократить объёмы
органики, поступающей на полигоны, снизить выбросы парниковых газов,
уменьшить расходы на транспортировку и повысить санитарное состояние
городской среды.

Мета-анализ продемонстрировал, что успешное внедрение ИПО требует
многоуровневой поддержки - от нормативно-правового регулирования и
модернизации инфраструктуры до формирования потребительской культуры и
мотивации. Анализ зарубежных практик (США, Германия, Япония, Швеция)
показал, что ключевым фактором является интеграция ИПО в комплексную
систему обращения с отходами, где технологии утилизации и очистки
сочетаются с просветительской деятельностью и государственной политикой
по поддержке устойчивого развития.

Моделирование показало, что при оснащении измельчителями до 50\%
домохозяйств Казахстана возможно перерабатывать до 1,5 млн тонн
органических отходов ежегодно. Это позволит снизить общий объём ТКО на
20-25\%, одновременно сократив нагрузку на полигоны и канализационные
очистные сооружения. Таким образом, технология ИПО способна сыграть роль
катализатора экологической модернизации коммунальной отрасли страны.
Вместе с тем, исследование выявило ряд ограничений: изношенность сетей
водоотведения, дефицит производственных мощностей КОС, низкий уровень
нормативной готовности, а также ограниченное знание потребителей о
правилах эксплуатации ИПО. Эти факторы могут замедлить масштабирование
технологии и вызвать риски вторичного загрязнения или засоров в системах
водоотведения.

{\bfseries Выводы.} На основании проведённого анализа установлено, что
органическая фракция ТКО составляет до 50-60\% от их общего объема, а
переработка этих отходов на месте образования посредством ИПО позволяет
значительно сократить массу и объём твёрдых отходов, направляемых на
полигоны, тем самым снижая потребность в их транспортировке и
утилизации.

Для реализации проекта необходимо:

- разработать и утвердить нормативно-технические регламенты, санитарные
и строительные требования к установке и использованию ИПО в жилом и
коммерческом фонде;

- инициировать интеграцию ИПО в национальные и региональные программы по
обращению с отходами, устойчивому городскому развитию и декарбонизации
экономики;

- создать механизмы государственной поддержки - субсидии, налоговые
льготы, кредитные линии и участие в программах «зелёного
финансирования»;

Комплексная реализация вышеперечисленных мер позволит не только повысить
эффективность системы обращения с органическими отходами, но и
обеспечить экологическую и санитарную безопасность, снизить
антропогенную нагрузку на окружающую среду и создать устойчивую модель
городской среды. Внедрение ИПО при соблюдении технических, нормативных и
организационных условий способно стать неотъемлемым элементом стратегии
перехода Республики Казахстан к принципам циркулярной экономики,
устойчивого потребления и ресурсосбережения, а также повысить качество
жизни населения в долгосрочной перспективе.

\emph{{\bfseries Примечание}: В процессе подготовки статьи использовались
инструменты искусственного интеллекта, в том числе языковая модель
GPT-4, предоставляемая OpenAI Модель использовалась исключительно для
помощи в редактуре текста и структурировании научного изложения.}
\end{multicols}

\begin{center}
{\bfseries Литература}
\end{center}

\begin{refs}
1. Аскарова У.Б., Мустафаева Р.М. Проблемы утилизации твердых бытовых
отходов и их влияние на экологию в Казахстане // Актуальные проблемы
гуманитарных и естественных наук\emph{.-} 2014.- № 8(2).- С.12-14.

2. Джалилова Г.А. Антропогенная эпоха твердых коммунальных отходов//
Известия Санкт-Петербургского государственного технологического
института (технического университета). - 2013. -№ 19 (45). - С.
93--97\emph{.}

3. Байдалинова А.С., Байгиреева Ж.З., Ниязбекова Ш.У. Экономические
аспекты управления пищевыми отходами в Казахстане: проблемы, тенденции и
перспективы//Л.Н. Гумилев атындағы Еуразия Ұлттық университетінің
Хабаршысы.- 2025.-№ 2.-С.143-162. DOI
\href{https://doi.org/10.32523/2789-4320-2025-2-143-162}{10.32523/2789-4320-2025-2-143-162}.

4. Козлов Г.В., Ивахнюк Г.К. Морфологический состав твердых коммунальных
отходов по регионам мира в xx и начале XXI века (обзор)// Известия
Санкт-Петербургского государственного технологического института
(технического университета), Серия Экология и системы жизнеобеспечения.
-2014.- № 24 (50).- С.58-66.

5. Епишов А.П. Государственная политика в сфере переработки и утилизации
пищевых отходов в России // Вестник Российского экономического
университета им. Г.В. Плеханова\emph{.-} 2022.- № 6.- С.17-23. DOI
10.21686/2413-2829-2022-6-17-23.

6. Субракова Л.К. Экономика обращения с пищевыми отходами в России //
Вестник Воронежского государственного университета. Серия: Экономика и
управление\emph{.-} 2021.- № 1.- -С.37- 48. DOI
10.17308/econ.2021.1/3322.

7. Калиева С.С. и др. Экономическая оценка политики по сокращению ТБО в
Республике Казахстан с использованием метода условной оценки (CV) //
Вестник университета «Туран»\emph{.} 2024. - № 3. - С.23-34. DOI
10.46914/1562-2959-2024-1-3-23-34.

8. Петросянц Т.В. Эколого-экономическая оценка и утилизация пищевых
отходов в РК // Вестник науки.- 2023.-№ 12(69).-С.1050-1054.

9. Ванюшина А.Я., Данилович Д.А. Анаэробное сбраживание - ключевая
технология обработки осадков городских сточных вод (часть 1) //
Водоснабжение и санитарная техника\emph{.} 2013.- № 10\emph{.} С.58-65.

10. Данилович Д.А., Ванюшина А.Я. Анаэробное сбраживание - ключевая
технология обработки осадков городских сточных вод (часть 2) //
Водоснабжение и санитарная техника.- 2013. -№ 11. С.50-57.

11. Жапарова С.Б., Баязитова З.Е., Курманбаева А.С. и др. Термофильное
сбраживание бытовых пищевых отходов // Вестник Карагандинского
университета\emph{.} -2022.- №107(3).- С.56-66. DOI
10.31489/2022BMG3/56-66.
\end{refs}

\begin{center}
{\bfseries References}
\end{center}

\begin{refs}
1. Askarova U.B., Mustafaeva R.M. Problemy utilizacii tverdyh bytovyh
othodov i ih vlijanie na jekologiju v Kazahstane //
Aktual' nye problemy gumanitarnyh i estestvennyh nauk.-
2014. - № 8(2).- S.12-14. {[}in Russian{]}

2. Dzhalilova G.A. Antropogennaja jepoha tverdyh
kommunal' nyh othodov// Izvestija Sankt-Peterburgskogo
gosudarstvennogo tehnologicheskogo instituta (tehnicheskogo
universiteta). - 2013. -№ 19 (45). - S.93--97. {[}in Russian{]}

3. Bajdalinova A.S., Bajgireeva Zh.Z., Nijazbekova Sh.U. Jekonomicheskie
aspekty upravlenija pishhevymi othodami v Kazahstane: problemy,
tendencii i perspektivy//L.N. Gumilev atyndaғy Eurazija Ұlttyқ
universitetіnің Habarshysy.- 2025.-№ 2.-S.143-162.
DOI 10.32523/2789-4320-2025-2-143-162. {[}in Russian{]}

4. Kozlov G.V., Ivahnjuk G.K. Morfologicheskij sostav tverdyh
kommunal' nyh othodov po regionam mira v xx i nachale XXI
veka (obzor)// Izvestija Sankt-Peterburgskogo gosudarstvennogo
tehnologicheskogo instituta (tehnicheskogo universiteta), Serija
Jekologija i sistemy zhizneobespechenija. -2014.- № 24 (50).- S.58-66.
{[}in Russian{]}

5. Epishov A.P. Gosudarstvennaja politika v sfere pererabotki i
utilizacii pishhevyh othodov v Rossii // Vestnik Rossijskogo
jekonomicheskogo universiteta im. G.V. Plehanova.- 2022.- № 6.- S.
17-23. DOI 10.21686/2413-2829-2022-6-17-23. {[}in Russian{]}

6. Subrakova L.K. Jekonomika obrashhenija s pishhevymi othodami v Rossii
// Vestnik Voronezhskogo gosudarstvennogo universiteta. Serija:
Jekonomika i upravlenie.- 2021.- № 1.- -S.37- 48. DOI
10.17308/econ.2021.1/3322. {[}in Russian{]}

7. Kalieva S.S. i dr. Jekonomicheskaja ocenka politiki po sokrashheniju
TBO v Respublike Kazahstan s ispol' zovaniem metoda
uslovnoj ocenki (CV) // Vestnik universiteta «Turan».2024. - № 3. - S.
23-34. DOI 10.46914/1562-2959-2024-1-3-23-34. {[}in Russian{]}

8. Petrosjanc T.V. Jekologo-jekonomicheskaja ocenka i utilizacija
pishhevyh othodov v RK // Vestnik nauki.- 2023.-№ 12(69).-S.1050-1054.
{[}in Russian{]}

9. Vanjushina A.Ja., Danilovich D.A. Anajerobnoe sbrazhivanie -
kljuchevaja tehnologija obrabotki osadkov gorodskih stochnyh vod
(chast'{} 1) // Vodosnabzhenie i sanitarnaja tehnika.
2013. - № 10. S.58-65. {[}in Russian{]}

10. Danilovich D.A., Vanjushina A.Ja. Anajerobnoe sbrazhivanie -
kljuchevaja tehnologija obrabotki osadkov gorodskih stochnyh vod
(chast'{} 2) // Vodosnabzhenie i sanitarnaja tehnika.-
2013. -№ 11. S.50-57. {[}in Russian{]}

11. Zhaparova S.B., Bajazitova Z.E., Kurmanbaeva A.S. i dr.
Termofil' noe sbrazhivanie bytovyh pishhevyh othodov //
Vestnik Karagandinskogo universiteta. -2022.- №107(3).- S.56-66. DOI
10.31489/2022BMG3/56-66. {[}in Russian{]}
\end{refs}

\begin{info}
\hspace{1em}\emph{{\bfseries Сведения об авторах}}

Хайриев Д. С. - директор ТОО «RedSeven», Астана, Казахстан, e-mail:
d.hairiev@gmail.com;

Мукатай А. К. - директор ТОО «Офис Коммерциализации», Астана,
Казахстан, e-mail: ablay.m.k@gmail.com;

Даулетжанова Ж. Т. - доктор PhD, доцент Казахского университета
технологии и бизнеса им. К. Кулажанова, Астана, ведущий научный
сотрудник ТОО «Институт Химии угля и технологии», Астана, Казакстан
e-mail: kaliyeva\_zhanna@mail.ru.

\hspace{1em}\emph{{\bfseries Information about the authors}}

Khayriev D. S. - director of «RedSeven» LLP, Astana, Kazakhstan,
e-mail: d.hairiev@gmail.com;

Mukatai Ablaikhan Karimgazyuly -- Director of «Commercialization
Office» LLP, Astana, Kazakhstan, e-mail: ablay.m.k@gmail.com;

Dauletzhanova Zhanna Taumuratovna - PhD Doctor, Technology, Kazakh
University of Technology and Business named after K. Kulazhanov,
Leading Researcher of LLP "Institute of Coal Chemistry and
Technology", Astana, Kazakhstan, e-mail: kaliyeva\_zhanna@mail.ru.
\end{info}
