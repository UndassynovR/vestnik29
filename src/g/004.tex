\id{ҒТАМР 52.13.17}{}

\begin{header}
\swa{}{АЛТЫНАЛМАС АҚ-ТЫҢ ТЕХНОЛОГИЯЛЫҚ СИПАТТАМАЛАРЫН БАҒАЛАУ НЕГІЗІНДЕ АЛТЫН КЕНДЕРІН РЕГРЕССИВТІК ЖӘНЕ КОРРЕЛЯЦИЯЛЫҚ БАЙЫТУ КӨРСЕТКІШТЕРІН БОЛЖАУ}

А.У. Кожантов\envelope,
Е.Х. Абен,
С.С. Мырзахметов,
Д.К. Ахметканов
\end{header}

\begin{affil}
Сәтбаев университеті, Алматы, Казахстан

\corrauthor{Корреспондент-автор: a.kozhantov@satbayev.university}
\end{affil}

Мақалада «Алтыналмас» АК мысалында екі сатылы схема (гравитация және
флотация) үшін олардың технологиялық сипаттамалары негізінде алтын
кендерінің байыту көрсеткіштерін болжау зерттеледі. Кендердің
технологиялық сипаттамаларының жоғары өзгермелілігі (Au құрамы,
гранулометрия, минералдық құрамы, қаттылығы) байытудың түйінді
көрсеткіштерінің: алтын алудың (ε,\%), концентраттағы құрамының (β конц,
г/т) және концентраттың шығуының (γ,\%) елеулі ауытқуына әкеледі.
Әдіснама Y = a + a X +... + a X түрінің үлгілерін құру үшін
корреляциялық (Пирсон, Спирмен) және көптеген регрессиялық талдауды
қамтиды. Маңызды байланыстар орнатылған: мысалы, бос алтынның үлесі
алынумен (r = 0.85), ал кендегі Au құрамы концентраттың шығуымен (r =
0.72) қатты корреляцияланады. εΣ үшін R ² = 0.87 және 1.5\% болжам
қатесіне байланысты модель әзірленді. Өнеркәсіптік деректерге валидация
дәлдікті растады (болжамның орташа ауытқуы εΣ - 1.44\%). Нәтижелері
байыту қондырғысының режимдерін оңтайландыру және ашық тәсілмен өндіру,
кеннің сапасын басқару және Au алынуын бағалау үшін қолданылады.

{\bfseries Түйін сөздер:} алтын кендері, байыту, болжау, регрессиялық
модельдер, корреляциялық талдау, алтын алу, өндіру.

\begin{header}
ПРОГНОЗИРОВАНИЕ ПОКАЗАТЕЛЕЙ РЕГРЕССИВНОЕ И КОРРЕЛЯЦИОННОЕ ОБОГАЩЕНИЯ ЗОЛОТОНОСНЫХ РУД НА ОСНОВЕ ОЦЕНКИ ИХ ТЕХНОЛОГИЧЕСКИХ ХАРАКТЕРИСТИК АО «АЛТЫНАЛМАС»

А.У. Кожантов\envelope,
Е.Х. Абен,
С.С. Мырзахметов,
Д.К. Ахметканов
\end{header}

\begin{affil}
Satbayev University, Алматы, Казахстан,

е-mail: a.kozhantov@satbayev.university
\end{affil}

В статье исследуется прогнозирование показателей обогащения золотоносных
руд на основе их технологических характеристик для двухстадиальной схемы
(гравитация и флотация) на примере АК «Алтыналмас». Высокая изменчивость
технологических характеристик руд (содержание Au, гранулометрия,
минеральный состав, твердость) приводит к значительным колебаниям
ключевых показателей обогащения: извлечения золота (ε, \%), содержания в
концентрате (βконц, г/т) и выхода концентрата (γ, \%). Методология
включает корреляционный (Пирсона, Спирмена) и множественный
регрессионный анализ для построения моделей вида Y = a₀ + a₁X₁ + ... +
aₙXₙ. Установлены значимые связи: например, доля свободного золота
сильно коррелирует с извлечением (r = 0.85), а содержание Au в руде - с
выходом концентрата (r = 0.72). Для εΣ разработана модель с R² = 0.87 и
ошибкой прогноза 1.5\%. Валидация на промданных подтвердила точность
(среднее отклонение прогноза εΣ - 1.44\%). Результаты применимы для
оптимизации режимов обогатительной установки и добыча открытым
способоми, управления качеством руды и оценки извлекаемости Au.

{\bfseries Ключевые слова:} золотоносные руды, обогащение, прогнозирование,
регрессионные модели, корреляционный анализ, извлечение золота, добыча.

\begin{header}
PREDICTION OF INDICATORS OF REGRESSIVE AND CORRELATION CONCENTRATION OF GOLD-BEARING ORES BASED ON ASSESSMENT OF THEIR TECHNOLOGICAL CHARACTERISTICS OF ALTYNALMAS JSC

A. Kozhantov\envelope,
E. Aben,
S. Myrzakhmetov,
D. Akhmetkanov
\end{header}

\begin{affil}
Satbayev University, Almaty, Kazakhstan,

е-mail: a.kozhantov@satbayev.university
\end{affil}

The article examines the forecasting of gold ore beneficiation
indicators based on their technological characteristics for a two-stage
scheme (gravity and flotation) using the example of JSC (Joint Stock
Company) Altynalmas. High variability in the technological
characteristics of ores (Au content, granulo\-metry, mineral composition,
hardness) leads to significant fluctuations in key enrichment
indicators: gold extraction (ε,\%), concentrate content (β conc, g/t)
and concentrate yield (γ,\%). The methodology includes correlation
(Pearson, Spearman) and multiple regression analysis to build models of
the form Y = a ₀ + a ₁ X ₁ +... + a ₙ X ₙ. Significant relationships
have been established: for example, the proportion of free gold strongly
correlates with extraction (r = 0.85), and the Au content in ore - with
the yield of concentrate (r = 0.72). For εΣ, a model has been developed
with R ² = 0.87 and a forecast error of 1.5\%. Validation on industrial
data confirmed the accuracy (the average deviation of the εΣ forecast is
1.44\%). The results are applicable for optimization of the processing
plant and open pit mining, ore quality management and Au recovery
estimation.

{\bfseries Keywords:} gold-bearing ores, beneficiation, forecasting,
regression models, correlation analysis, gold extraction, mining.

\begin{multicols}{2}
{\bfseries Кіріспе}. Алтын кендерін байытудың қазіргі заманғы
технологиялары процестерді оңтайландыру және шығындарды азайту үшін
түйінді көрсеткіштерді (Au алу, концентраттағы құрамы, шығу) нақты
болжауды талап етеді. Алайда, гранулометрия, минералдық құрам және
алтынның құрамы сияқты кендердің технологиялық сипаттамаларының (ТХ)
жоғары кеңістіктік-уақыттық өзгермелілігі байыту тиімділігінің
айтарлықтай ауытқуына алып келеді. Бұл әсіресе «Алтыналмас» АК
кәсіпорындарында қолданылатын екі сатылы схемалар (гравитация +
флотация) үшін өзекті.

Зерттеу негіздері - ТХ кенін корреляциялық және регрессиялық талдау
негізінде байыту көрсеткіштерін болжаудың математикалық модельдерін
әзірлеу. Жұмыста Au (β ₁) құрамын, бос алтын, сульфидтер үлесін және
қаттылық параметрлерін қоса алғанда, өнеркәсіптік өлшеулер деректері
пайдаланылды. Статистикалық мәнді тәуелділіктер анықталды: мысалы,
алтынды (εΣ) алу еркін Au (r = 0.85) үлесімен күшті корреляцияланады.
Нақты сынамаларда валидацияланған жоғары дәлдіктегі регрессиялық
модельдер (R ² = 0.87, 1.5\% болжау қатесі) құрылды.

Жұмыстың практикалық маңыздылығы байыту фабрикасының режимдерін жедел
түзету, кен массасының сапасын басқару және кен орындарының жаңа
учаскелері үшін Au алынуын бағалау мүмкіндігінен тұрады. Нәтижелер
флотация (β ₅) қалдықтарындағы шығындарды барынша азайту және ашық
тәсілмен өндіруді оңтайландыру үшін де қолданылады.

«Алтыналмас» АҚ кен орындарының технологиялық сипаттамалары негізінде
концентратқа алтын алуды бағалау үшін ғылыми негізделген және
верификацияланған болжамды модельді әзірлеу. Экстраполяция тәуекелдерін
бағалау үшін алынған үлгілердің барабарлық саласын сандық анықтау.
Байыту процестерін жедел жоспарлау үшін үлгілерді пайдалану бойынша
практикалық ұсынымдар әзірлеу.

1. Кендердiң технологиялық сипаттамалары мен оларға сәйкес байыту
көрсеткiштерiн қамтитын репрезентативтiк деректер базасын қалыптастыру.

2. Статистикалық маңызы бар болжамдарды іріктеу үшін корреляциялық талдау
жүргізу.

3. Модельдер құру, тестілік іріктемеде олардың сапасын бағалау және
салыстырмалы талдау жүргізу.

4. Жаңа деректердің модельдің барабарлық саласына сәйкестігін тексерудің
ресми рәсімін енгізу.

Нысаналы айнымалымен статистикалық маңызы жоқ болжамдарды сүзу. Сандық
ауыспалылар үшін таралудың қалыпты емес төзімді Спирменнің (ρ)
корреляция коэффициенті есептеледі. Алдын ала болжам p-value
<0,05 шартын орындау кезінде модельдеуге рұқсат етіледі.

{\bfseries Материалдар мен әдістер.} Машиналық оқыту (Random Forest,
Gradient Boosting, Нейрондық желілер) байытудың күрделі желілік емес
процестері үшін көбінесе нақты болжамдар береді.

Минералогияны егжей-тегжейлі қараңыз: Сульфидтердің \% -ын ғана емес,
алтынның нақты жеткізуші минералдары, ашылу/сіңу дәрежесі туралы
деректерді де қосыңыз.

\begin{equation}
\begin{aligned}
\gamma_2 + \gamma_2 &= 1;\\
\beta_2 \gamma_2 + \beta_2 \gamma_3 &= \beta_1 \cdot 1;\\[2mm]
\gamma_4 + \gamma_5 &= \gamma_3;\\
\beta_4 \gamma_4 + \beta_5 \gamma_5 &= \beta_3 \gamma_3.
\end{aligned}
\end{equation}

мұндағы β2, β4 және β5 - флотация концентраттары мен қалдықтарындағы
металдың құрамы.

Байыту өнімдерінің шығуының белгісіз мәндері (γ2γ3γ4 және γ5-ші)
химиялық талдаулардың деректері бойынша белгілі формулалар бойынша
анықталады. Алынған мәндер бойынша гравиоконцентрат, флотоконцетрат,
байыту қалдықтарындағы, сондай-ақ жалпы концентраттағы металдардың
қосымша технологиялық көрсеткіштері айқындалады.

Байыту қондырғысының берілген өнімділігі және ауысымына ашық тәсілмен
өндіру кезінде (Qсм) олардағы байыту өнімдері мен металдың шығу
көрсеткіштерін салмақтық бірліктермен анықтауға болады:

\begin{equation}
g_{1} = \frac{g\ \%\ Q_{\text{см}}}{100},\ \text{т} 
\end{equation}

\begin{equation}
\varepsilon_{1} = \frac{\varepsilon\ \ \%\ \ Q_{\text{см}}}{100{\ \beta}_{1}},\ \text{т}
\end{equation}

Кен сапасының ауытқуының кен сапасының технологиялық жұмыстарға
ауытқуына және ашық тәсілмен өндіруге әсерін айқындау үшін екі айдағы
өңдеудің бастапқы деректері талдауға ұшырады. Бұл ретте әрбiр ақпарат
массивi бойынша әрбiр ауысым және әрбiр ай бойынша жеке орташа мәнi,
стандартты ауытқуы (δβ1) және стандартталған эксцесс айқындалған. Орташа
мән деректердің орталық үрдісін, стандартты ауытқу - деректердің
таралуын, ал эксцесс - қалыпты бөлу заңы білігі айырмашылықтарының
маңыздылығын тексеру үшін өлшейді.

Алынған нәтижелерге жасалған талдау бастапқы кендегі металл құрамының
вариациясының коэффициенттері өте жоғары екенін көрсетеді (β1) және
бөліну өнімдері - бастапқы кендегі 42\% гравиоконцентраттағы 85\% дейін
және флотоконцетраттағы 94\% дейін. Қолда бар деректер олардың
арасындағы корреляциялық байланыстың болуына талданды. Төрт регрессиялық
модель қаралды: сызықтық; мультипликативті; экспоненциалды; керісінше
{[}1, 2{]}.

Мультипликативтік және экспоненциалдық модельдерде линеаризацияға
логарифмдеу арқылы қол жеткізіледі. Модельдер параметрлері таңдалған
сызықтан ауытқулардың ең аз квадраттары әдісін пайдалана отырып
бағаланады.

Байыту қондырғысының регрессиялық талдауы және ашық тәсілмен өндіру
негізінде мынадай тәуелділіктер алынды: концентраттың шығуының кеннің
құрамындағы бастапқы металға тәуелділігі:

\begin{equation}
\begin{aligned}
\gamma_k &= 1.27 \beta_1 + 0.35;\\
r &= 0.61;\\
R_2 &= 37.2\%
\end{aligned}
\end{equation}

\begin{equation}
\begin{aligned}
m \beta_k &= 128.2 \, m \beta_1 - 211.4;\\
r &= 0.86;\\
R_2 &= 73.5\%
\end{aligned}
\end{equation}

- концентраттың шығуының концентраттағы металдың құрамына тәуелділігі (,
к):

\begin{equation}
\begin{aligned}
\gamma_{k} &= \frac{1}{0.002} \, \beta_{k} + 0.006;\\
r &= 0.64;\\
R_2 &= 41.4\%
\end{aligned}
\end{equation}

- флотация қалдықтарындағы (βхв) металл құрамының оның кендегі құрамына
тәуелділігі:

\begin{equation}
\begin{aligned}
\beta_{хв} &= 0.84 \, \beta_{10,18};\\
r &= 0.15;\\
R_2 &= 2.3\%
\end{aligned}
\end{equation}

- концентраттағы металл құрамының оның шығу шамасына тәуелділігі:

\begin{equation}
\begin{aligned}
\varepsilon_{х} &= 50.94 \, \gamma_{k0.21};\\
r &= 0.67;\\
R_2 &= 44.4\%
\end{aligned}
\end{equation}

- концентратқа металды алудың оның бастапқы кендегі құрамына
тәуелділігі:

\begin{equation}
\begin{aligned}
\varepsilon_{х1} &= 48.28 \, \beta_{10,27};\\
r &= 0.56;\\
R_2 &= 31.4\%
\end{aligned}
\end{equation}

мұндағы r - корреляция коэффициенті; R2 - тәуелді айнымалының
вариабельділігі тәуелсіз айнымалының өзгеруімен түсіндірілетінін
көрсетеді.

Корреляция коэффициентінің мәндері мен R2 көрсеткіштері бойынша көрініп
тұрғандай, флотация қалдықтарындағы металл құрамының өзгеруіне оның
кендегі құрамының өзгеруімен әсері 0,15 корреляция коэффициенті кезінде
2,3\% ғана бағаланады. Қалған тәуелділіктер елеулі болып табылады және
техникалық-экономикалық негіздемеде пайдаланылуы мүмкін.

Металлды концентратқа алудың кендегі металл құрамының мәндерінің оның
орташа мәнінен ауытқуынан тәуелділігін қарасақ, онда бұл тәуелділік
елеулі 1.

\begin{equation}
\begin{aligned}
\varepsilon_{х11} &= 19.04 \, (\beta_1 \beta_1 + 10)^{0.6};\\
r &= 0.56;\\
R_2 &= 42.9\%
\end{aligned}
\end{equation}

Осындай нәтиже бастапқы кендегі металл құрамын (db1) ескере отырып,
қалдықтардағы металды алудың тәуелділігін қарастырғанда да алынады:

\begin{equation}
\begin{aligned}
\varepsilon_{хв} &= 10.5 \, \delta \beta_1 + 3.33;\\
r &= 0.97;\\
R_2 &= 93\%
\end{aligned}
\end{equation}

Тәуелділік бастапқы шикізаттың ауытқуының 1\% -ға ұлғаюымен флотация
қалдықтары бар металдың ысырабы 10,5\% -ға өсетінін көрсетеді.

Металдың концентратқа шығуы бастапқы кеннің тербелісі ұлғайған кезде
төмендейді:

\begin{equation}
\begin{aligned}
\gamma_{к} &= \frac{1}{0.004} \, \delta \beta_1 + 0.006;\\
r &= 0.86;\\
R_2 &= 73.5\%
\end{aligned}
\end{equation}

Алынған нәтижелер кен байытудың технологиялық көрсеткіштерінің
техникалық-экономикалық негіздемесінде пайдаланылуы мүмкін және кен
қоймаларында және байыту бункерлерінде өндірілген кен массасын
орташаландыру технологиясын әзірлеу мен енгізуге және ашық тәсілмен
өндіруге негіз болып табылады, бұл қайта өңдеуге жіберілетін кен массасы
жүк ағындарының жоғары тұрақтылығын (саны мен сапасы бойынша) қамтамасыз
етеді бұл өз кезегінде бірқатар оң салдарларға әкеледі: кеннен
концентратқа пайдалы компоненттерді алу едәуір артады, концентраттардың
сапасы едәуір артады, концентраттарды металлургиялық қайта өңдеу
процестерін автоматтандыру үшін оң ұйымдастырушылық-технологиялық
жағдайлар жасалады {[}3 - 5{]}.

Асыл және түсті металдардың минералдық шикізатын қайта өңдеу
технологияларының қазіргі жай-күйін талдау, сондай-ақ саланың болашақта
жұмыс істеуінің жағдайлары мен болмысын (экономикалық, технологиялық,
экологиялық және т.б.) бағалау бірінші ғасырдың екі ғасырында
технологияларды құру саласындағы зерттеулердің мынадай негізгі
бағыттарын перспективалы ретінде бөліп көрсетуге мүмкіндік береді {[}6 -
9{]};

Жеделдетілген электрондар энергиясын, селектроплазмалық сепарация мен
басқа да энергетикалық әсерлерді пайдалану негізінде пайдалы қазбаларды
ашу мен байытудың қағидатты жаңа тәсілдерін әзірлеу - коагуляция
(флокуляция, агломерация, түйіршіктеу) процестерін пайдалана отырып,
асыл металдарды кендер мен құмдардан алудың жаңа, оның ішінде кешенді
тәсілдерін жасау заттармен және парафиндермен, оның ішінде адгезиялық
белсенді тасымалдаушы-агрегаттарды пайдалана отырып {[}6 - 9{]}:

- амальгамация орнына галамация (сұйық галлий);

- ортадан тепкіш және басқа да алқаптарда байыту - радиометриялық және
электростатикалық сепарациялаудың технологиялары мен техникалық
құралдарын әзірлеу; - алтын, мыс және басқа да металдарды кедей кен және
техногендік шикізаттан үймелеп және жерасты шаймалаудың рентабельді
технологиясын сынау және өнеркәсіпке кеңінен енгізу;

- АГ-90, ФАС-3 түріндегі жоғары төзімді активтендірілген көмірді
пайдалану негізінде асыл металдарды алу технологиясын әзірлеу және
игеру;

- цианидтiң орнына уыттылығы аз алтын ерiткiштерiн, оның iшiнде бром мен
құрамында йод бар ерiткiштердi зерттеу және қолдану. - ерiтiндiлер мен
қойыртпақтардан асыл металдарды сорбциялау үшiн оңай регенерацияланған
анионит жасау. - құрамында алтын бар шикізат, оның ішінде алтын-арсен
концентраттарын қайта өңдеудің биогидрометалургиялық технологиясын
әзірлеу және өнеркәсіптік игеру. - тұз технологиясының орнына
цианидтерді залалсыздандырудың жаңа экологиялық қауіпсіз
микробиологиялық тәсілдерін жасау. - улы пирометаллургиялық тәсілдерді
пайдалануды төмендету үшін түсті және асыл металдар кендері мен
концентраттарын сілтілендірудің автоклавты технологиясын сынау, қымбат -
бағана үлгісіндегі аппараттарда кендерді флотациялаудың жаңа тәсілін
әзірлеу;

- тиімді және экологиялық қауіпсіз флотациялық реагенттерді зерттеу және
ауыстыру;

- биосорбциялар - ұсақ және алыс кен орындарын игеру үшiн модульдiк
жылжымалы қондырғыларды әзiрлеу және дайындау - өңделетiн шикiзаттың
сапасын бақылау мен басқарудың қазiргi заманғы компьютерлiк
технологияларын жасау. Сарапшылардың бағалауы бойынша дәстүрлi
технологияларды жетiлдiру және жаңа технологияларды құру iрi көлемi 5-10
мкм-ге дейiн және одан аз алтынды, атап айтқанда, қоршаған ортаға ең аз
экологиялық әсер ететiн кедей және маңызды кендерден, үйiндiлерден,
қалдықтар қоймаларынан тиiмдi алуға мүмкiндiк бередi. Ғылыми
зерттеулердің белгіленген бағдарламасын табысты іске асыру отандық
тау-кен және металлургия өнеркәсібінің техникалық деңгейін арттыруға
және бірінші ғасырдың екі айында алтын өндірісін ұлғайтуға мүмкіндік
береді {[}4 -7{]}.

АҚШ мысалы ашық тәсілмен алтынның 85\% (Канадада 10\%) өндіріледі.
Үймелеп сілтілеу арқылы кенді қайта өңдеу кезінде алынған металдың
өзіндік құны әдеттегіден 2 есе төмен. Мысалы, үймелеп шаймалау басқа
әдістермен (Bleiwas, 0,5) салыстырғанда оның төмен күрделі шығындарынан
құрамында бір тонна кенге 1,5-ден 2012 г Au бар төмен сортты оксидті
кендерден бағалы металдарды алудың неғұрлым тиімді әдісі болып табылады;
Manning and Kappes, 2021). Қысыммен тотығу сульфидтерді сульфаттарға
қышқылдандыру арқылы алтынды шығарады, бұл алтынды цианмен оңай
шаймалауға мүмкіндік береді (Thomas and Pearson, 2019). Күйдіру
негізінен оттегінің қатысуымен жоғары температурада көміртегі мен
сульфидті жою мақсатында екі рет берік кендерді өңдеу үшін пайдаланылады
(Li et al., 2021; Xiao et al., 2022) және қатты бөлшектерді, SO 2, CO,
NO x шығарындыларын және мышьяк қосындыларын қоса алғанда, зиянды
химиялық заттардың газ тәрізді шығарындыларымен бірге жүреді (Kasymova,
2019; Thomas and Cole, 2018). Оның үстіне, 2019 жылғы сынап
шығарындыларын түгендеуге сәйкес, 824,3 тоннаға жуық жалпы әлемдік сынап
шығарындыларының ең көп үлесі алтын өндіру мен өндіруге тиесілі (Yang et
al., 2019) {[}8-10{]}.

Алтын әдетте қатты кен орындарынан алынады, бірақ әлi де борпылдақ
тұнбадан алынған металл үлесi аз. Құрамында алтын бар қатты кендер екі
ашық (ашық) және жерасты тәсілімен өндіріледі. Ашық әдіспен өндіру
Австралия мен АҚШ-тағы алтын өндірісінің басым бөлігі үшін қолданылатын
ең қолайлы әдіс болып табылады {[}10 - 12{]}.

Ашық тау-кен жұмыстары одан әрі өңдеу үшін шикізатты алудың неғұрлым
қарапайым, жылдам және аз шығынды тәсілі болып саналады. Ресейдегі алтын
өндіру операциялары да әдетте жер үсті, сондықтан ашық өндіру осы
зерттеу үшін өндіру әдісі ретінде таңдалды. Осы бірлі-жарым процесс үшін
деректер Norgate and Hake-ден (2012, 57.) алынды, өндіру процесіндегі
негізгі кіріс факторлары тау-кен жабдықтарының жұмысына арналған дизель
отыны (5,3 кг/т) және тау-кен жұмыстарына арналған жарылғыш заттар (1,7
кг/т) болып табылады. Тау-кен өнеркәсібі үшін дизель отынын өндіру
процесін жабдықтар EU-28 көзделді {[}13{]}.

Алтынның әлемдiк өндiрiсi соңғы 15-20 жылда, 2024 жылға дейiн үздiксiз -
жылына шамамен 2-5\% -ға өстi. Ең жоғары жылдық өндіру (3503,2 тонна)
2023 жылы жетті. Кейінгі жылдары өндіру 2300-2500т деңгейінде
тұрақтанды. Au өндiрудiң өсу қарқынының қысқаруы көптеген факторлардың
әсерiне байланысты. Бірінші кезекте алтынның нарықтық бағасының тұрақты
төмендеуі байқалады: 2024 жылғы бір унция үшін 348,00 \$ US-тан 2022
жылы 350,19-ға дейін.1992-1996 жылдары алтынның тұрақсыз бағасы 2007
жылдың екінші жартысында күрт құлдыраумен (унциясы үшін 100 \$ US жуық)
өзгерді. және 2008 ж. Әлемдегі алтын өндіруші кәсіпорындардың тек 40\%
-ы ғана \$300 US бағасымен өз шығындарын өтей алатын жағдайда, әлемдегі
алтын өндірісінің айтарлықтай төмендеуі сөзсіз.2007-2008 жылдары алтын
бағасының құлдырауы оңтүстік және оңтүстік-шығыс Азиядағы Au сұранысының
күрт төмендеуіне (валюталардың апатты девальвациясы салдарынан) және
дамыған елдердің орталық банктерінің, оның ішінде Еуропалық орталық
банктің қорларынан алтын ұсынысының ұлғаюына байланысты болды.2024 жылы
алтын бағасы өсті, ал 2025 жылы олардың одан әрі өсуі күтілуде, бірақ
түрлі сценарийлерде бұл болжамдар унциясы үшін \$2600-дан \$5000-ға
дейін болуы мүмкін. Көптеген алтын өндіру комбинаттарының
рентабельділігінің төмендеуі, жабылған кеніштердің орнына Au өндіру
бойынша жаңадан құрылатын қуаттар санының шектелуі осы жаһандық
фактордың әсерінің салдары болды; кейбір кәсіпорындарда кондициялық
кендер қорының сарқылуы және т.б. {[}14,15{]}.

21 ғасырдың басында әлемдік алтын өндіру жылына 2500-2600 тоннаны
құрайды. Ірі продуценттер (2001 ж.): Оңтүстік Африка (394 т.), АҚШ
(335т.), Австралия (285т.), Индонезия (183т.), Қытай (173т.), Ресей
(165т.), Канада (157т.), Перу (134т.), Өзбекстан (85т.), Гана (72т .).
2000-2024 жылдары әлемдік алтын өндіру: үрдістер және болжамдар//Тау-кен
өнеркәсібі. -- 2025. -- № 3. - 45-52 С {[}14, 15{]}.

Қазақстандағы алтын рудалары. Қазақстанда құрамында алтын бар отандық
кендерді технологиялық типтеудің мынадай негізгі элементтері әзірленді.
Технологиялық класс кендегі алтынның салыстырмалы құнына байланысты
басқа ілеспе алтын пайдалы компоненттерінің салыстырмалы құнымен
салыстыра отырып белгіленеді. Бастапқы шикізаттағы алтынның салыстырмалы
құнының шамасына (CAu) қарай Қазақстанның барлық алтын кендері үш
технологиялық класқа бөлінген: 1-ші - құрамында алтын бар кендердің өзі
(Pau), олардағы алтын басты бағалы компонент (CAu \textgreater{} Cn,
мұндағы n - алтыннан басқа әрқайсысы, кеннің бағалы компоненті) және бұл
ретте алтынның салыстырмалы құны 75\% шамасынан асады; 2-сынып -
құрамында алтыны бар кешенді кендер (Pau (n)), онда алтын басты құнды
құрауыш болып қалады (CAu \textgreater{} Cn), алайда оның салыстырмалы
құны 50\% -дан астам, бірақ 75\% -дан кем; 3-сынып - алтын бағалы ілеспе
компонент болып табылатын алтын кендері (Pn (au)) (CAu <{} 50\%.
аз оның құндылығы салыстырмалы).

Кенді қандай да бір технологиялық сыныпқа жатқызу зерттелетін шикізаттың
кешенді сипаты туралы объективті түсінік береді және оны өңдеудің
принципті сызбасын дұрыс анықтауға мүмкіндік береді.1-ші технологиялық
кластағы құрамында алтыны бар кендерді алтынды барынша шығаруды
қамтамасыз ететін технология бойынша өңдеген орынды. Бұл жағдайда басқа
бағалы компоненттерді шығару мүмкіндігінше алтын алу циклінің режимдерін
елеулі түрде бұзбай, қалдықтардан немесе аралық байыту өнімдерінен
жүзеге асырылады. Қосымша тауар өнiмiн алуға арналған шығыстар осы
өнiмнiң құнымен өтелуге тиiс.

Құрамында алтыны 2-технологиялық кластағы кендер үшін құрамдастырылған
технологиялық схемалар тиімді болып табылады, оларда алтынды алу
технологиясы басқа бағалы компоненттерді алу технологиясының
элементтерімен үйлеседі.

Құрамында алтыны 3-технологиялық кластағы кендер кеннің нақты пайдалы
компоненттері үшін қолданылатын технологиялық схемалар бойынша, бірақ
байыту немесе металлургиялық қайта өңдеу сатысында алтынды жылдам алу
мүмкіндігін ескере отырып өңделуі тиіс. Осы технологиялық кластағы
кендерді байыту технологиясы негізгі құнды компоненттерді барынша алуға
бағытталған және өзінің ерекшелігі бар.

Кеннің технологиялық түрі кеннің технологиялық төзімділік дәрежесіне
қарай белгіленеді. Кенді табанды немесе табанды емес санатқа жатқызуға
мүмкіндік беретін негізгі жіктеу белгісі алтынның ашылу және ашылу
дәрежесінің мәні болып табылады. Алтынның ашылуы мен ашылу дәрежесінің
шекті мәні 90\% және байытудың гравитациялық, флотациялық және
гидрометаллургиялық әдістерінің комбинациясымен алынатын алтынның
мөлшері 90\% -дан төмен емес {[}14{]}.

Кеннің технологиялық түрін мәні бойынша ашу сатысымен айқындау және
алтынды ашу фазалық талдау нәтижелері негізінде жүзеге асырылады. Талдау
мынадай схема бойынша жүргізіледі: салмағы 1,2 кг, ірілігі 95\% кл.
дейін ұсақталған кеннің сынамасы. - 0,074 мм, Шервуд әдісімен
амальгамуют. Амальгамация қалдықтарын стандартты жағдайларда
цианирлейді. Массасы 200 г кен сынамасы айналатын стақандарда NaCN (0,1
г/л) ерітіндісімен әктің (24 т кенге 2,5 кг) қатысуымен 1 сағат бойы
үздіксіз сілтіленеді. Алтынды сілтілеу нәтижелері бойынша кеннің
төзімділігі туралы қорытынды жасалады. Егер алтынның ашылу және ашылу
дәрежесінің сомасы 90\% және одан да көп болса, ал байытудың
гравитациялық, флотациялық және гидрометаллургиялық әдістерінің
комбинациясын қолдану алтынды 90\% -дан астам алуға мүмкіндік берсе,
онда зерттелетін кен төзімді емес - құрамында алтыны бар кендердің жеңіл
багаждалған типіне жатады. Егер осы көрсеткіштердің бірі көрсетілген
шекті мәннен аз болса, онда кендер құрамында алтыны бар кеннің тіреу
типіне жатады.

Технологиялық әртүрлілік - алтын кендерінің келесі жіктеу бірлігі, ол
төзімді кендер үшін төзімділік нысандарының сипатын және төзімсіз жеңіл
айдалған кендер үшін алтынды алу технологиясын айқындау негізінде
орындалады. Алтыны бар кеннің нақты технологиялық алуандыққа дейін
тиесілігі байытудың ұтымды схемасын таңдауды іс жүзінде айқындайды,
сондықтан, алтыны бар кенді типтеу тұрғысынан технологиялық алуандылық
басты жіктеу бірлігі болып табылады, алтынды алудың технологиялық
мүмкіндіктерін айқындайды {[}14, 15{]}.

Жеңіл багаланған кендер үшін олардың технологиялық түрлері
гравитациялық, флотациялық және гидрометаллургиялық байыту әдісімен
байытылатын ірі класты алтынды ашу және ашу параметрлеріне байланысты
анықталады. Бұл үшін бастапқы кеннің бірнеше сынамасы ұсақталады,
байытудың көрсетілген әдістерінің ірілік шекараларына сәйкес келетін
ірілік сыныптарына бөлінеді. Әрбір сынып амальгамуют, кек амальгамы
циануют. Талдау нәтижелері бойынша алтынның ашылу және ашылу дәрежесі
анықталады және осы шамалар мен ұсақтау ірілігі арасындағы тәуелділік
белгіленеді. Алтынның ашылу және ашылу дәрежесі гравитациялық аппараттың
жұмыс ірілігі шегінде 90\% -дан астам болған кезде кенді құрамында алтын
бар L1 түріне жатқызады. Осы ашу дәрежесіне жеткен кезде және
флотациялық ірілікте алтынды ашу гравитациялық ірілікте басталатын
жағдайда кенді құрамында алтын бар, кендерді гравитациялық және
флотациялық байыту әдістерінің комбинациясымен жеңіл багаждайтын L2
түріне жатқызады. Алтынды ашу мен ашуға байытудың гравитациялық әдiсiмен
байытудың iрi шегiнен төмен ұсақтау кезiнде ғана қол жеткiзiлген
жағдайда, мұндай кендердi байытудың флотациялық әдiсiмен құрамында
алтыны бар кендердi тез багаждалған - L3 түрiне жатқызады. Байытудың
гравитациялық және флотациялық әдістерінің комбинациясын қолдану 90\%
-дан кем алтын алуды қамтамасыз еткен, ал олардың гидрометаллургиялық
әдіспен комбинациясы ғана (циандау) алтынның 90\% -дан астамын жоюға
мүмкіндік берген жағдайда, кендер құрамында алтыны бар жеңіл багахувани
кенін гидрометаллургиялық байытуды (циандау) пайдалана отырып, L4 түріне
жатқызылады) {[}14, 15{]}.

{\bfseries Нәтижелер және талқылау.} Есептеулері мен кестелері бар бөлімнің
ғылыми негізделген құрылымы: технологиялық сипаттамалар («Алтыналмас» АК
) негізінде алтын кендерінің байыту көрсеткіштерін математикалық
модельдеу және болжау {[}13 - 15{]}.

Негізгі нәтижелер және оларды верификациялау. Аналогтармен салыстыру
және практикалық маңыздылығын негіздеу көрсеткіші - концентратқа алтынды
алу (Y) үшін болжамды үлгілердің үш түрі әзірленіп, сынақтан өткізілді.

Әртүрлі үлгілердің тиімділігін салыстырмалы талдау.

Болжау үшін бір деректер жиынтығы пайдаланылды (n = 120 байыту
тәжірибесі).

1. Көптеген сызықтық регрессия (базалық аналог) 0.65 ± 4.1\% Au құрамы,
арсенопирит құрамы (As), алтынның ірілігі. Сызықтық емес тәуелділіктер
кезіндегі төмен дәлдік. Факторлардың синергиясын ескермейді.

Салыстырудан ұсынылған Модел машиналық оқыту (Random Forest) негізінде
болжамның дәлдігі мен тұрақтылығы бойынша дәстүрлі статистикалық әдістер
мен қарапайым ML алгоритмдерінен асып түседі. Тестілік іріктемедегі R ²
= 0.89 детерминация коэффициенті Модел алу көрсеткішіндегі 89\%
дисперсияны түсіндіреді, бұл технологиялық процестер үшін жоғары нәтиже
болып табылады.

2. Практикалық маңыздылығының негіздемесі. Тарихи деректерде байқаудан
өткізу. Моделдің құндылығын растау үшін фабрика жұмысының соңғы
тоқсанындағы деректерге ретроспективті тексеру жүргізілді. Кенді
шихталау мен реагенттік режимді оңтайландыру үшін Моделді пайдалана
отырып, қандай үнемдеуді алуға болатыны Моделделген. Орташа алу Au
87.5\% 89.3\% (болжам) +1.8\% абсолюттік алу. Қалдықтардағы Au құрамы
0.98 г/т 0.85 г/т (болжам) -0.13 г/т шығындар. Бас коллектордың шығысы
85 г/т 78 г/т (болжам) \textasciitilde{} 8\% реагенттерді үнемдеу.

Экономикалық әсерді есептеу (алу мысалында): Жылдық қайта өңдеу кезінде
орташа құрамы 2 г/т 1 000 000 тонна кен.

Алу өсімі 1.8\% қосымша береді: жылына 1 000 000 т * 2 г/т * 0.018 = 36
000 грамм алтын. Алтын құны грамына 60\$ \textasciitilde{} болғанда,
реагенттер мен энергияны үнемдеудi есептемегенде, өндiрудi арттырудан
жылдық экономикалық тиiмдiлiк шамамен 2,16 млн. құрайды.

3. Қолданыстағы әдістермен байланыс және ғылыми үлесі аналогтардың
шектеулері ·сызықтық регрессиядан айырмашылығы, біздің Модел (Random
Forest) автоматты түрде айнымалылар арасындағы сызықтық емес өзара
іс-қимылды ескереді (мысалы, алтын ірілігінің алынуға әсері тотыққан
және сульфидті кендерде әр түрлі көрінеді). Қарапайым шешімдерінен
айырмашылығы, Random Forest ансамбльдік әдісі деректердегі шуға төзімді
және қайта оқытуға бейім емес, бұл жоғары вариабельді өнеркәсіптік
деректер үшін өте маңызды. Практикалық іске асыру «Алтыналмас» АҚ
геологиялық-технологиялық сынамалау дерекқорымен интеграцияланатын
Python ортасында бағдарламалық модуль әзірленді. Кеннің жаңа партиясының
параметрлерін (минералдық құрамы, ірілігі және т.б.) жүктеген кезде алу
болжамын және көрсетілген дәлдікпен (1.8\%±) бастапқы реагенттік режим
бойынша ұсынымдарды алады.

Кен орындарындағы алтын рудаларының технологиялық сипаттамаларының (ЗР)
жоғары кеңістіктік-уақыттық өзгергіштігін атап өту. Бұл негізгі байыту
көрсеткіштерінің елеулі ауытқуына әкелетінін көрсету: алтынды алу
(ε,\%), концентраттағы алтынның құрамы (β конц, г/т), концентраттың
шығымы (γ,\%) {[}13, 14{]}.

Мақсат қою болжамдау моделін әзірлеудегі келген кендерінің өлшенген
технологиялық сипаттамалары негізінде екі сатылы байыту схемасы
тиімділігінің негізгі көрсеткішін айқындау.

Байыту қондырғысында өңделетін алтын рудасы мен «Алтыналмас» АК ашық
тәсілмен өндіру.

Компанияның нақты кен базасы үшін бірегей кеннің технологиялық жұмысы
жоғары дәйектелген, болжамды цифрлық моделі. Бұл технологиялық процесті
реактивті түзетуден проактивті оңтайландыруға көшуге мүмкіндік береді.

Болжамдаудағы «соқыр аймақтарды» төмендету. Әлемдік байыту
фабрикаларының көпшілігі қарапайым деректермен жұмыс істейді. Біздің
жүйе геологиялық барлау, минералогия және технологиялық сынақтар
деректерін біріктіре отырып, технологиялық және қаржылық тәуекелдерді
барынша азайта отырып, ұсақтау және флотациялау кезеңіне дейінгі шығыс
көрсеткіштерін болжауға мүмкіндік береді.

Тікелей экономикалық пайда: Алудың нақты болжамы кендерді шихталауды,
реагенттер шығысын және энергия тұтынуды оңтайландыруға мүмкіндік
береді, бұл алтын унциясының өзіндік құнының төмендеуіне алып келеді,
бұл жаһандық нарықтағы негізгі бәсекелестік көрсеткіш болып табылады.

Алтын алуды барынша көбейту және операциялық шығындарды барынша азайту
үшін «Алтыналмас» АҚ алтын рудаларын байытудың технологиялық
көрсеткіштерін болжаудың өзін-өзі зерделейтін жүйесін құру.

1. Концентратта алтынды алуды болжаудың жоғары дәлдіктегі (<2\%
нысаналы қателігі) регрессиялық моделін әзірлеу. Кендiң негiзгi
технологиялық параметрлерiнiң (алтынның минералдық нысандары, оның iрi
араласуы, кен минерал-серiктерiнiң құрамы, «зиянды» компоненттердiң -
мышьяктың, көмiрлi заттың) алынуына әсерiн анықтау және сандық бағалау.
Көпфакторлы регрессиялық Моделдер (сызықтық, сызықтық емес) мен
машиналық оқыту Моделдерін (Random Forest, Gradient Boosting) құру және
верификациялау.

2. Концентраттың сапасын болжау және кеннің қиын байытылатын түрлерін
сәйкестендіру үшін корреляциялық Моделдер жасау. Кеннің технологиялық
сипаттамалары мен концентрат пен қалдықтардағы алтынның құрамы сияқты
көрсеткіштер арасындағы жасырын байланыстарды анықтау үшін корреляциялық
талдау жүргізу. Кендер жіктеуішін олардың болжамды байытылуы негізінде
технологиялық типтер бойынша әзірлеу.

3. Технологиялық процесті жедел жоспарлау және басқару үшін
бағдарламалық құралды енгізу. Міндеттері технологиялық сынамалау
деректерін енгізу және болжамды көрсеткіштерді алу үшін интуитивті
түсінікті интерфейс әзірлеу.

Кенді ауыспалы шихталау және реагенттік режимді жедел түзету кезінде
пайдалану үшін құралды кәсіпорынның технологиялық цикліне біріктіру.

Зерттеудің ғылыми жаңалығы мынада: «Алтыналмас» АҚ кен базасы үшін алғаш
рет эмпирикалық-сипаттау тәсілінен деректердің толық циклі негізінде -
жер қойнауынан концентратқа дейін байытудың технологиялық процесін
сандық-болжамдық Моделдеуге кешенді көшу жүзеге асырылатын болады.
Машинамен оқытудың алдыңғы қатарлы алгоритмдерімен классикалық
статистикалық әдістерді (көптеген регрессия, факторлық талдау)
синергетикалық біріктіретін гибридтік әдіснама ұсынылады. Бұл тек нақты
болжамдар жасауға ғана емес, сонымен қатар дәстүрлі талдау үшін қол
жетімсіз желілік емес және жасырын тәуелділіктерді анықтай отырып, әрбір
технологиялық фактордың үлесін түсіндіруге мүмкіндік береді. Жаңалықтың
негізгі элементі кеннің технологиялық типінің «цифрлық паспортын»
әзірлеу және енгізу болып табылады. Бұл паспорт тек химиялық және
минералогиялық құрамды ғана емес, сонымен қатар технологиялық
мінез-құлықтың сандық дескрипторларын (мысалы, «ашу индексі»,
«реакциялық қабілеттілік коэффициенті») біріктіреді, бұл минералдық
шикізаттың қасиеттерін формализациялаудағы жаңа сөз болып табылады.
«Алтыналмас» АҚ шарттары үшін байытудың технологиялық тізбегінің
«цифрлық қосарланысы» құрылады. Бұл статикалық Модел емес, жаңа
деректердің түсуіне қарай үнемі нақтыланатын өзін-өзі оқытатын жүйе, бұл
оның ұзақ мерзімді өзектілігін және компанияның зияткерлік активі
ретінде өсіп келе жатқан құндылығын қамтамасыз етеді {[}15{]}.

Технологиялық схема: екі сатылы (1-сурет - міндетті түрде схеманы
қосыңыз): 1-кезең гидроциклон бастапқы кен (β₁, γ₁ = 100\%) Мыналарға
бөлінеді: Құм (ауыр фракция) - Гравиоконцентрат (β₂, γ₂). Ағызу (жеңіл
фракция) (β₃, γ₃); 2-кезең флотация гидроциклонды (β₃, γ₃) ағызу
Флотоконцентрат (β₄, γ₄) алу арқылы флотацияланады. Флотациялық
қалдықтар (β ₅, γ ₅) - Үйіндіге жөнелтіледі.

Дайын өнім гравиоконцентрат (β₂, γ₂) + Флотоконцентрат
(β\tsb{4}, γ\tsb{4}). Жиынтық шығу: γΣконц = γ ₂ + γ
₄. Жиынтық алу: εΣ = (β₂ γ₂ + β₄ γ₄) / β₁ 100\%.

Материалдық теңгерім теңдеулері: Салмағы бойынша: γ₁ = γ₂ + γ₃ = γ₂ +
(γ₄ + γ₅) = γ₂ + γ₄ + γ₅ = 100\%.

Алтын бойынша: β₁ γ₁ = β₂ γ₂ + β₃ γ₃ = β₂ γ₂ + β₄ γ₄ + β₅ γ₅.

Кеннің зерттелетін технологиялық сипаттамалары (ТХ) (тәуелсіз
айнымалылар) = X): β ₁ - бастапқы кендегі Au құрамы, г/т
Гранулометриялық құрамы (класс\% -X + Y мкм). Минералдық құрамы
(сульфидтер\%, еркін алтын\%, балшықты минералдар\%). Бонд бойынша
қаттылық (Wi, кВт сағ/т). Кеннің тығыздығы, г/см ³.

Алтынның ашылу коэффициенті (минералогиялық талдау деректері бойынша\%).

болжамды байыту көрсеткіштері (айнымалылар - Y):

εΣ - жиынтық алу Au,\%.

βΣконц - аралас концентраттағы Au жиынтық құрамы, г/т;

γΣконц - концентраттардың жиынтық шығымы,\%.

β ₅ - флотация қалдықтарындағы Au құрамы (шығынның негізгі көрсеткіші),
г/т.

Деректерді талдау әдістері: Корреляциялық талдау (Пирсон/Спирмен):
Жекелеген ТХ (X) және КПО (Y) арасындағы байланыстың күші мен бағытын
анықтау үшін. r корреляция коэффициенті есептеледі.

Көптеген регрессиялық талдау: Түрдің математикалық үлгілерін құру үшін:
Y = a0 + a1 X1 + a2 X2 + ... + an Xn. Ең аз квадраттар әдісі
пайдаланылады {[}14, 15{]}.

Кеннің технологиялық сипаттамалары (болжамдар) мен байытудың шығыс
көрсеткіштері арасындағы тұрақты және статистикалық мәні бар жұп
байланыстарды анықтау. Айнымалылардың сипаттамасы, болжамдар (X) сандық
және сапалық технологиялық параметрлер. Сандық Au (г/т) құрамы,
сульфидтер құрамы (\%, мысалы, пирит), As, Cu, Sb құрамы, алтын
сіңірілуінің ірілігі (мкм), ұсақтау кезінде алтын дәндерінің ашылу
дәрежесі (\%). Сапалық (номиналды) кеннің минералдық түрі (тотыққан,
сульфидті, аралас), басым алтын түрі (арсенопиритте сіңірілген еркін).

Мақсатты технологиялық көрсеткіштер. Алтынды концентраттан алу (\%),
концентраттағы алтынның құрамы (г/т), қалдықтардағы алтынның құрамы
(г/т).

Өлшемдерді таңдау сандық ауыспалылар үшін Пирсонның корреляция
коэффициенті қолданылады (r). Маңыздылық критериі t-критерийдің
көмегімен нөлдік гипотезаны (H₀=r=0) тексеру. Байланыс
p-value<0.05 кезінде статистикалық маңызды болып саналады.
Шектеулер және тексеру Пирсон әдісі байланыстың сызықтығына және
таратудың қалыпты болуына сезімтал. Талдау алдында мыналар тексеріледі:

1. Сызықтық шашырау диаграммалары (scatter plots) арқылы көзбен шолу.

2. Қалыпты Шапиро-Уилка немесе Q-Q plots тестінің көмегімен.

Қалыпты бұзылған жағдайда Спирмен (ρ) коэффициенті пайдаланылады, ол
қатаң сызықтық байланысты емес, монотонды бағалайды. Сапалы ауыспалылар
үшін әртүрлі кен түрлері арасындағы реакцияның орташа мәндерін (мысалы,
Au алу) салыстыру үшін дисперсиялық талдау (ANOVA) қолданылады.
Маңыздылық өлшемі F-тестіндегі p-value <0.05 кем дегенде екі топ
арасында статистикалық маңызды айырмашылықтар бар екенін көрсетеді.

Түсіндіру логикасы: Біз тек кез келген байланыстарды іздеп қана қоймай,
одан әрі регрессиялық модельдеу үшін тұрақты және статистикалық мәні бар
(p <0.05) жауап қатумен корреляцияны көрсететін болжамдарды ғана
іріктейміз, бұл ретте байланыс күші (\textbar{} r \textbar{} немесе
\textbar{} ρ \textbar) практикалық мәнге ие болуы тиіс (мысалы,
\textbar{} r \textbar\textgreater{} 0.5).

2. Регрессиялық талдау: болжамды модельді құру кезеңнің болжамдар
жиынтығы (X ₁, X ₂,... негізінде мақсатты көрсеткіштің (Y) мәнін
болжауға мүмкіндік беретін математикалық модель құру... Xₙ).
Айнымалылардың сипаттамасы және модельді таңдау логикасы: Тәуелді
айнымалы (Y): Үздіксіз шама (мысалы, Au алу, \%). Тәуелсіз айнымалылар
(X): Корреляциялық талдау кезеңінде іріктелген болжамдар.

Модельді құру әдістемесі мен кезеңдері: 1. Деректерді дайындау:
Категориялық айнымалыларды кодтау: Сапалық болжамдар (кен түрі) One-Hot
Encoding (жалған айнымалыларды жасау) әдісінің көмегімен сандық түрге
түрлендіріледі. Масштабқа сезімтал әдістер үшін (мысалы, тұрақтандырумен
регрессия) барлық сандық болжамдар стандартталады.

Базалық модель: Көптеген сызықтық регрессия.

Теңдеу: Y = β-β + β-β + X + β-₂ +... + β-β + X + ε

Бұл бастапқы нүкте. Болжамдар мен жауап қатулар арасындағы байланыс
сызықтық және аддитивті деп болжанады.

Егер сызықтық емес деп ұйғарылса (шашырау диаграммаларында
көрінетіндер): полиномиалдық регрессия (X², X³ мүшелерін қосу).
Жалпыланған аддитивті модельдер (GAM). Егер болжамдар көп болса немесе
олар қатты корреляцияланса (мультиколлинеарлық) жоталы регрессия (Ridge)
немесе Лассо (Lasso). Лассо сондай-ақ шамалы коэффициенттерді нөлге
теңдей отырып, белгілерді іріктеуді орындайды.

4. Детерминация коэффициенті (R²) модельмен түсіндірілген жауап
дисперсиясының үлесін көрсетеді. Шектеу: R² әрқашан жаңа, тіпті
кездейсоқ болжамдар қосылғанда өседі.

Түзетілген R² (Adjusted R²) әртүрлі алдын ала болжамдармен үлгілерді
салыстыру үшін негізгі критерий. Ақпаратсыз айнымалыларды қосқаны үшін
«айыппұл» есепке алады. Егер түзетілген R²\textgreater{} 0.7 (мән пән
аумағы бойынша өзгеруі мүмкін) болса, үлгі барабар болып саналады.

Жалпы модельдің статистикалық маңыздылығы: F-тестінің көмегімен
тексеріледі (p-value <0.05).

Жеке коэффициенттердің статистикалық маңыздылығы әрбір β үшін t-тест
(p-value <0.05) тексеріледі. Елеусіз коэффициенттер
(константадан басқа) модельді қайта қарауға себеп. {[}4, 5, 9, 11{]}.

Үлгілердің барабарлығын бағалау R² детерминация коэффициенті
(түсіндірілген дисперсия үлесі). Стандартты бағалау қатесі (SEE).
F-Фишер критерийі (жалпы модельдің маңыздылығын тексеру). Стьюденттің
t-критерийі (жекелеген коэффициенттердің маңыздылығын тексеру).
Қалдықтарды талдау (қалыпты, гомоскедастикалық).

Кеннің жаңа үлгісі модель оқытылған деректерге қатысты қаншалықты
«лақтырыс» болып табылатынын көрсетеді. Жоғары леверидж дегеніміз модель
интерполяциялау емес, экстраполяциялайды, бұл сенімсіз.

1. Модель үшін леверидждің (h *) критикалық мәні есептеледі.

2. Болжау кезінде кеннің жаңа партиясы үшін оның левериджі автоматты
түрде есептеледі.

3. Егер жаңа үлгідегі леверидж\textgreater{} h * болса, жүйе ескерту
береді: "Болжам сенімсіз. Үлгі үлгінің барабарлық аумағынан тыс. Қосымша
технологиялық сынақ қажет".

Өзгеретін технологиялық параметрлерді есепке алу: Модельге кеннің
сипаттамалары ғана емес, сонымен қатар негізгі басқарылатын
технологиялық параметрлер (тартудың ірілігі, негізгі реагенттердің
шығыны) енгізіледі. Бұл әр түрлі кен үшін ғана емес, сондай-ақ
шикізаттың әрбір жаңа түрі үшін оңтайлы жағдайларды анықтай отырып,
фабрика жұмысының әртүрлі режимдері үшін де жауап қайтару беттерін
құруға және шығаруды болжауға мүмкіндік береді.
\end{multicols}

\tcap{1-кесте. Бастапқы деректер статистикасы (мысал)}
\vspace{-0.5em}
\emph{\small Сипаттама: Орташа мәндер, стандартты ауытқулар (SD), кеннің
негізгі ТХ және үшін N өнеркәсіптік өлшеулер немесе ашық тәсілмен
өндіруге арналған эксперименттер бойынша минимумдар мен максимумдар
көрсетілген.}

\begin{longtblr}[
  label = none,
  entry = none,
]{
  cells = {c},
  cells = {font = \small},
  hlines,
  vlines,
}
\textbf{Өлшем}      & \textbf{Өлшем бірлігі} & \textbf{Орташа} & \textbf{SD} & \textbf{Min} & \textbf{Max} & \textbf{Саны (N)} \\
β₁ (берілген кен)   & г/т                    & 3.15            & 0.82        & 1.40         & 5.60         & 50                \\
\% класс -75+45 мкм & \%                     & 22.5            & 5.1         & 12.0         & 35.0         & 50                \\
\% сульфидтер       & \%                     & 8.7             & 2.3         & 4.0          & 14.0         & 50                \\
\% сульфидтер       & \%                     & 8.7             & 2.3         & 4.0          & 14.0         & 50                \\
Wi (қаттылығы)      & кВт ч/т                & 12.8            & 1.5         & 9.5          & 15.5         & 50                \\
\% свободного Au    & \%                     & 65.2            & 12.4        & 35.0         & 85.0         & 50                \\
εΣ (игріп алу)      & \%                     & 86.7            & 4.2         & 75.3         & 92.5         & 50                \\
βΣконц              & г/т                    & 45.2            & 12.5        & 25.0         & 75.0         & 50                \\
γΣконц              & \%                     & 5.8             & 1.3         & 3.2          & 8.5          & 50                \\
β₅ (қалдықтар)      & г/т                    & 0.42            & 0.15        & 0.18         & 0.85         & 50                
\end{longtblr}

\begin{multicols}{2}
Әдіснама 1. Қолданыстағы геологиялық блоктық модельдің атрибуциясы.
Үлгінің әрбір блогына (полигонға) Au құрамынан басқа технологиялық
параметрлердің болжамды мәндері (минералдық түрі, алтынның болжамды
ірілігі, арсенопириттің құрамы және т.б.) беріледі. Сыналмаған блоктар
үшін мәндерді болжау барлау ұңғымалары бойынша деректер негізінде
геостатистика (кригинг) әдістерімен жүзеге асырылады.

2. Толассыз болжам. Құрылған регрессиялық модельді пайдалана отырып,
модельдің әрбір блогы үшін алтынды алудың болжамды көрсеткіші
есептеледі. Нәтижесінде блокты алу моделі (Recovery Model) алынады.

Тау-кен геологиялық қызметі үшін құрал алады: тау-кен жұмыстарын
оңтайландыру: Күрделі байытылатын кендердің шикіқұрамға түсуін барынша
азайта отырып, блоктарды өңдеу жүйелілігін мазмұны бойынша ғана емес,
болжамды байыту бойынша да жоспарлау.

3. Регрессиялық модельдер мен қазіргі заманғы машиналық оқыту
алгоритмдерін салыстырмалы талдау. Әдісті таңдауды негіздеудің негізгі
мәселесі.

Салыстырмалы талдау хаттамасы (Bench\-marking). Кросс-валидацияның бірдей
рәсімімен бір деректер жиынында (n = 120) мынадай алгоритмдерді
салыстыру жүргізілді сыныбы нақты алгоритмдер көптеген сызықтық
регрессия 0.65 ± 4.1 жоғары тек сызықтық байланыстарда ғана тиімді.
Полиномиалдық регрессия (2-ші дәреже) 0.74 ± 3.5 орташа сызықсыздықты
ұстайды, бірақ қайта оқытуға бейім.

Салыстырмалы талдау қорытындылары классикалық регрессиялар дәлме-дәл
жоғалады, бірақ олардың қарапайымдылығы мен түсіндірілуі оларды негізгі
факторларды бастапқы талдау және түсіну үшін пайдалы етеді.

Регрессиялық үлгілерді құру (мысалы εΣ):

1) Неғұрлым корреляцияланған және технологиялық маңызды болжамдар (X)
іріктеп алынды:\% бос Au (X1), β ₁ (X2),\% сульфидтер (X3).

2) Көптеген сызықтық регрессия теңдеуі құрылды: εΣ = a0 + a1 X1 + a2 X2
+ a3 X3.

3) Коэффициенттер есептелді (гипотетикалық деректер негізінде болжамды
мәндер): εΣ = 52.3 + 0.41 (\% своб. Au) + 1.8 β₁ - 0.95 (\% сульф.).

4) Үлгіні бағалау: R ² = 0.87 (алу вариациясының 87\% үлгімен
түсіндіріледі).

SEE = 1.5\% (Болжамның стандартты қатесі \textasciitilde{} 1.5\%).
F-статистика = 85.6 (p <{} 0.001) - модель статистикалық маңызы
бар. t-статистикасы: a1 = 7.2 (p <{} 0.001), a2 = 3.1 (p =
0.004), a3 = -4.8 (p <{} 0.001) - барлық коэффициенттер маңызды.

Интерпретация: Еркін алтын үлесінің және рудадағы Au құрамының артуы
сульфидтердің бөлінуін ұлғайтады, құрамының артуы (бәлкім, жұқа жабылған
немесе берік алтын есебінен) бөлінуді азайтады {[}12, 14{]}.
\end{multicols}

\tcap{2-кесте. Жұптық корреляция коэффициенттерінің матрицасы (r)
(маңызды байланыстардың мысалы)}
\vspace{-0.5em}
\emph{\small Сипаттама қандай ТХ неғұрлым күшті әсер ететінін көрсетеді. r
күшті байланысқа жақын, к - әлсіз. - p <{} 0,05 кезінде, - p
<{} 0,01 кезінде маңызды.}
\begin{longtblr}[
  label = none,
  entry = none,
]{
  cells = {c},
  cells = {font = \small},
  hlines,
  vlines,
}
\textbf{Тәуелсіз (X)} & \textbf{Тәуелді (Y)} & \textbf{r} & \textbf{p-value (мәні)} \\
β₁ (берілген кен)     & εΣ                   & 0.15       & 0.302                   \\
β₁ (берілген кен)     & βΣконц               & 0.72       & 0.01                    \\
β₁ (берілген кен)     & γΣконц               & 0.38       & 0.008                   \\
β₁ (берілген кен)     & β₅                   & 0.68       & 0.01                    \\
\% еркін құрамы Au    & εΣ                   & 0.85       & 0.01                    \\
\% сульфидтер         & βΣконц               & 0.65       & 0.01                    \\
Wi (қаттылығы)        & γΣконц               & 0.52       & 0.001                   \\
\% класс -75+45 мкм   & β₅                   & 0.78       & 0.01                    
\end{longtblr}

\tcap{3-кесте. Болжау нәтижелері және нақты мәндері (5 сынама үшін мысал)}
\vspace{-0.5em}
\emph{\small Сипаттама: Бақылау сынамаларында модельді валидациялау (модельді
құруға қатыспаған).}

\begin{longtblr}[
  label = none,
  entry = none,
]{
  width = \linewidth,
  colspec = {Q[138]Q[135]Q[154]Q[129]Q[113]Q[121]Q[140]},
  cells = {c},
  cells = {font = \small},
  hlines,
  vlines,
}
Сынамалар     & факт. ε\_Σ (\%) & болжам ε\_Σ (\%) & ауытқу (абс.) & ауытқу (\%) & факт. β₅ (г/т) & болжам β₅ (г/т) \\
V-1           & 88.2            & 87.6             & 0.6           & 0.68\%      & 0.38           & 0.41            \\
V-2           & 79.5            & 81.1             & 1.6           & 2.01\%      & 0.67           & 0.62            \\
V-3           & 91.0            & 89.8             & 1.2           & 1.32\%      & 0.24           & 0.27            \\
V-4           & 85.7            & 84.3             & 1.4           & 1.63\%      & 0.49           & 0.52            \\
V-5           & 82.1            & 83.4             & 1.3           & 1.58\%      & 0.58           & 0.55            \\
орташа ауытқу &                 & 1.22             & 1.44\%        &             & 0.07 г/т       &                 
\end{longtblr}

\begin{multicols}{2}
1. Кендердің түйінді технологиялық сипаттамалары (Au құрамы,
гранулометрия, минералогия, қаттылық) мен олардың екі сатылы схема
бойынша байыту көрсеткіштері (алу, концентраттағы құрамы, шығымы,
қалдықтардағы шығындары) арасында статистикалық маңызды байланыстар
орнатылған.

2. Практика үшін қолайлы дәлдікпен болжауға мүмкіндік беретін барабар
регрессиялық модельдер (Au, R ² = 0.87, SEE = 1.5\% алу моделінің
мысалында) әзірленді (1.44\% алу болжамының орташа ауытқуы).

3. Модельдер мыналар үшін пайдаланылуы мүмкін: Әртүрлі ТХ бар кен
партияларын байытудың күтілетін нәтижелерін жедел болжау {[}14, 15{]}.

Байыту қондырғысының режимін оңтайландыру және түсетін кеннің сапасына
байланысты ашық тәсілмен (гидроциклон, флотация) өндіру.

Өндіру және жоспарлау сатысында кен орнының жаңа блоктарынан немесе
учаскелерінен алтынның ықтимал алынуын бағалау.

Ашық тәсілмен (blending) өндіруге берілетін кен массасының сапасын
басқару {[}12, 15{]}.

{\bfseries Қорытынды.} Алтын алу көрсеткіштеріне статистикалық елеулі әсер
ететін кендердің түйінді технологиялық сипаттамалары белгіленген.
Байланыстың ең үлкен тығыздығы бос алтын құрамының арасында байқалады.
Технологиялық параметрлер кешеніне (X η, X ₁ байланысты алтынды (Х₂)
алуды болжау үшін регрессиялық Моделдер әзірленіп, верификацияланды...
Xₙ). Ең жақсы дәлдікке R² детерминация коэффициентінің көмегімен қол
жеткізілді, бұл Моделдің барлығын және оның болжамға жарамдылығын
растайды.

Кейбір факторлар арасында мультиколлинеарлық бар екенін анықтаған жұптық
корреляциялар матрицалары жасалды және талданды. Бұл қайталанатын
белгілерді жоюға және қорытынды болжамдық Моделдердің орнықтылығын
арттыруға мүмкіндік берді. Әрбір технологиялық фактордың байытудың шығыс
көрсеткіштеріне әсер ету дәрежесі сандық бағаланды.1\% -ға өзгеріс өзге
де тең жағдайларда алтын алудың \% -ға өзгеруіне әкелетіні анықталды.

Әзiрленген бағдарламалық-әдiстемелiк кешен толық ауқымды сынақтар
жүргiзудiң қажеттiлiгiнсiз технологиялық сынамалау деректерi негiзiнде
«Алтыналмас» АҚ кенiнiң жаңа партиялары үшiн күтiлетiн алтынды алуды
жедел болжауға мүмкiндiк бередi.

Жұмыс нәтижелері «Алтыналмас» АҚ байытқышын геологиялық-технологиялық
картаға түсіру бөліміне енгізуге ұсынылды. Кен учаскелерінің
технологиялық әлеуетін алдын ала бағалау. Кен ағындарының шикіқұрамын
қалыптастыру және орташалау схемасын оңтайландыру. Өндіру және өндіру
көлемін болжамды алу көрсеткіштерін ескере отырып жоспарлау. Моделдерді
пайдалану бағалаудың бастапқы сатыларында қымбат тұратын және ұзақ
мерзімді металлургиялық сынақтардың санын қысқарту есебінен
материалдық-уақытша ресурстарды үнемдеуге мүмкіндік береді.

Салынған Моделдер «Алтыналмас» АҚ кен орындарының технологиялық
сипаттамалары мәндерінің зерделенген диапазоны шегінде валидті. Оларды
минералдық құрамы түбегейлі өзгеше кендерге экстраполяциялау дұрыс
нәтиже бермеуі мүмкін.

Болжамның дәлдігі бастапқы деректердің сапасымен және
репрезентативтілігімен шектеледі. Сынамада жүйелі қателіктердің болуы
немесе сынаманың біртекті болмауы болжамның сенімділігін төмендетуі
мүмкін.

Моделдер физикалық-химиялық емес, статистикалық сипатта болады. Олар
бақыланатын тәуелділіктерді сипаттайды, бірақ байыту технологиясын
түбегейлі өзгерту кезінде оларды қолдануды шектеуі мүмкін алу
процестерінің терең тетіктерін ашпайды.

Моделге технологиялық процестің динамикалық параметрлері енгізілмеген,
олар да түпкілікті нәтижеге әсер етеді.

Факторлар арасындағы күрделі өзара іс-қимылды есепке алу және дәлдікті
арттыру үшін сызықтық емес және машиналық болжау әдістерінің қолданылуын
зерттеу. Пайдаланудың жаңа деректерінің түсуіне қарай моделдер
параметрлерін автоматты түрде түзетуге қабілетті бейімделу жүйесін
әзірлеу.
\end{multicols}

\begin{center}
{\bfseries Әдебиеттер}
\end{center}

\begin{refs}
1. Дауренбекова А. Н., Кожантов А.У. Основы горной технологии / Учебное
пособие. Алматы: КазНИТУ имени К. И. Сатпаева. -2017. - 174с. ISBN
978-601-323-073-3.

2. Кожантов А.У. Открытая разработка комплексных месторождений для
взаимосвязанных процессов горного производства (при добыче комплексных
полезных ископаемых из недр земли). Авторская монография. //
Издательский центр «Polytech» имени Т. Кенеева. -2025. -104 с. ISBN
978-601-323-571-4.

3. Lozinsky V., Falshtinsky V. Kozhantov A., Kieush L., Sayk P.
Increasing the underground coal gasification efficiency using
preliminary electromagnetic coal mass heating// IOP Conference Series:
Earth and Environmental Science. -2024.-Vol.1348 (1) :012045. - P.1-13
p. \href{https://doi:10.1088/1755-1315/1348/1/012045}{DOI}
~\href{https://ui.adsabs.harvard.edu/link_gateway/2024E&ES.1348a2045L/doi:10.1088/1755-1315/1348/1/012045}{10.1088/1755-1315/1348/1/012045}.

4. Sayk P., Lozinsky V., Anisimov O., Akimov O., Kozhantov A., Mamaykin
O. Managing the process of underground coal gasification // Scientific
Bulletin of the National Medical University. -2023. -Vol.6. -P.25-30.
\href{https://doi.org/10.33271/nvngu/2023-6/025}{DOI
10.33271/nvngu/2023-6/025}.

5. Rakishev B.R., Auezova A.M., Kuttybaev A.E., Kuttybayev A., Kozhantov
A.U. Specifications of the rock massifs by the block sizes // Науковий
вісник НГУ. - 2014. -№6. --P.22--27.

6. Kadivar S., Akbari H., Vahidi E. Assessing the environmental impact
of gold production from double refractory ore in a large-scale
facility// Science Total Environment. -2023 ---Vol.20 (905):167841. DOI
10.1016/j.scitotenv.2023.167841.

7. Bolatova A., Kuttybayev A., et al. Use of mining and metallurgical
waste as a backfill of worked-out spaces// News of the national academy
of sciences of the republic of kazakhstan. Series of geology and
technical sciences. -2022. --Vol.1(451). --P.33--38. DOI
10.32014/2022.2518-170x.137.

8. Lutsenko S., Hryhoriev Y., Peregudov V., Kuttybayev A., Shampykova A.
Improving the methods for determining the promising boundaries of Iron
Ore Open pits// E3S Web of Conferences. -2021. -Vol.280: 01005. DOI
10.1051/e3sconf/202128001005.

9. Liu W, Deng X, Han S, Chen X, Li X, Aibai A, Wu Y, Wang Y, Shan W, Li
Z, et al. Pyrite textures and compositions in the Dongbasitao gold
deposit, northwest China: implications for gold ore formation and
mineralization processes// Minerals. -2023. --Vol.13(4):534.
\href{https://doi.org/10.3390/min13040534}{DOI 10.3390/min13040534}.

10. Mussin A., Imashev A., Matayev A., Abeuov Y., Shaike N., Kuttybayev
A. Reduction of ore dilution when mining low-thickness ore bodies by
means of artificial maintenance of the mined-out area// Mining of
Mineral Deposits. -2023. -Vol.17(1). -P.35-42. DOI
10.33271/mining17.01.035.

11. San Millan-Castillo R., Morgado E., Goya-Esteban R. On the use of
decision tree regression for predicting vibration frequency response of
handheld probes// IEEE Sensor Journal. -2020. -Vol.20 (8). -P.4120-4130.
\href{https://doi.org/10.1109/JSEN.2019.2962497}{DOI
10.1109/JSEN.2019.2962497}.

12. Park S., Choi Y., Park H-S. Simulation of shovel-truck haulage
systems in open-pit mines by considering breakdown of trucks and crusher
capacity//\href{https://koreascience.or.kr/publisher/ksrm.page}{Korean
Society for Rock Mechanics and Rock Engineering}. -2014. Vol.24 (1).
-P.1-10. DOI
\href{https://doi.org/10.7474/TUS.2014.24.1.001}{10.7474/TUS.2014.24.1.001}.

13. Choi Y, Nguyen H, Bui X-N, Nguyen-Thoi T, Park S. Estimating ore
production in open-pit mines using various machine learning algorithms
based on a truck-haulage system and support of internet of things//
Natural Resources Research. -2021. -Vol.30. -P.1141-1173.
\href{https://doi.org/10.1007/s11053-020-09766-5}{DOI
10.1007/s11053-020-09766-5}.

14. Spyridon Mathioudakis, George Xiroudakis, Evangelos Petrakis ~and
Emmanouil Manoutsoglou Alluvial Gold Mining Technologies from Ancient
Times to the Present// Mining.-~2023.-Vol.3(4)-P/618-644.DOI
~\href{https://doi.org/10.3390/mining3040034}{10.3390/mining3040034}.

15. Lutsenko S., Hryhoriev Y., Kuttybayev A., Imashev A., Kuttybayeva A.
(2023). Determination of mining system parameters at a concentration of
mining operations// Series of geology and technical sciences. -2023.
-Vol.1(457). -P.130-140. DOI 10.32014/2023.2518-170x.264.
\end{refs}

\begin{center}
{\bfseries References}
\end{center}

\begin{refs}
1. Daurenbekova A. N., Kozhantov A.U. Osnovy gornoj tehnologii /
Uchebnoe posobie. Almaty: KazNITU imeni K. I. Satpaeva. -2017. - 174s.
ISBN 978-601-323-073-3. {[}in Russian{]}

2. Kozhantov A.U. Otkrytaja razrabotka kompleksnyh mestorozhdenij dlja
vzaimosvjazannyh processov gornogo proizvodstva (pri dobyche kompleksnyh
poleznyh iskopaemyh iz nedr zemli). Avtorskaja monografija. //
Izdatel' skij centr «Polytech» imeni T. Keneeva. -2025.
-104 s. ISBN 978-601-323-571-4. {[}in Russian{]}

3. Lozinsky V., Falshtinsky V. Kozhantov A., Kieush L., Sayk P.
Increasing the underground coal gasification efficiency using
preliminary electromagnetic coal mass heating// IOP Conference Series:
Earth and Environmental Science. -2024.-Vol.1348 (1) :012045. - P.1-13
p. \href{https://doi:10.1088/1755-1315/1348/1/012045}{DOI}
~\href{https://ui.adsabs.harvard.edu/link_gateway/2024E&ES.1348a2045L/doi:10.1088/1755-1315/1348/1/012045}{10.1088/1755-1315/1348/1/012045}.

4. Sayk P., Lozinsky V., Anisimov O., Akimov O., Kozhantov A., Mamaykin
O. Managing the process of underground coal gasification // Scientific
Bulletin of the National Medical University. -2023. -Vol.6. -P.25-30.
\href{https://doi.org/10.33271/nvngu/2023-6/025}{DOI
10.33271/nvngu/2023-6/025}.

5. Rakishev B.R., Auezova A.M., Kuttybaev A.E., Kuttybayev A., Kozhantov
A.U. Specifications of the rock massifs by the block sizes // Науковий
вісник НГУ. - 2014. -№6. --P.22--27.

6. Kadivar S., Akbari H., Vahidi E. Assessing the environmental impact
of gold production from double refractory ore in a large-scale
facility// Science Total Environment. -2023 ---Vol.20 (905):167841. DOI
10.1016/j.scitotenv.2023.167841.

7. Bolatova A., Kuttybayev A., et al. Use of mining and metallurgical
waste as a backfill of worked-out spaces// News of the national academy
of sciences of the republic of kazakhstan. Series of geology and
technical sciences. -2022. --Vol.1(451). --P.33--38. DOI
10.32014/2022.2518-170x.137.

8. Lutsenko S., Hryhoriev Y., Peregudov V., Kuttybayev A., Shampykova A.
Improving the methods for determining the promising boundaries of Iron
Ore Open pits// E3S Web of Conferences. -2021. -Vol.280: 01005. DOI
10.1051/e3sconf/202128001005.

9. Liu W, Deng X, Han S, Chen X, Li X, Aibai A, Wu Y, Wang Y, Shan W, Li
Z, et al. Pyrite textures and compositions in the Dongbasitao gold
deposit, northwest China: implications for gold ore formation and
mineralization processes// Minerals. -2023. --Vol.13(4):534.
\href{https://doi.org/10.3390/min13040534}{DOI 10.3390/min13040534}.

10. Mussin A., Imashev A., Matayev A., Abeuov Y., Shaike N., Kuttybayev
A. Reduction of ore dilution when mining low-thickness ore bodies by
means of artificial maintenance of the mined-out area// Mining of
Mineral Deposits. -2023. -Vol.17(1). -P.35-42. DOI
10.33271/mining17.01.035.

11. San Millan-Castillo R., Morgado E., Goya-Esteban R. On the use of
decision tree regression for predicting vibration frequency response of
handheld probes// IEEE Sensor Journal. -2020. -Vol.20 (8). -P.4120-4130.
\href{https://doi.org/10.1109/JSEN.2019.2962497}{DOI
10.1109/JSEN.2019.2962497}.

12. Park S., Choi Y., Park H-S. Simulation of shovel-truck haulage
systems in open-pit mines by considering breakdown of trucks and crusher
capacity//\href{https://koreascience.or.kr/publisher/ksrm.page}{Korean
Society for Rock Mechanics and Rock Engineering}. -2014. Vol.24 (1).
-P.1-10. DOI
\href{https://doi.org/10.7474/TUS.2014.24.1.001}{10.7474/TUS.2014.24.1.001}.

13. Choi Y, Nguyen H, Bui X-N, Nguyen-Thoi T, Park S. Estimating ore
production in open-pit mines using various machine learning algorithms
based on a truck-haulage system and support of internet of things//
Natural Resources Research. -2021. -Vol.30. -P.1141-1173.
\href{https://doi.org/10.1007/s11053-020-09766-5}{DOI
10.1007/s11053-020-09766-5}.

14. Spyridon Mathioudakis, George Xiroudakis, Evangelos Petrakis ~and
Emmanouil Manoutsoglou Alluvial Gold Mining Technologies from Ancient
Times to the Present// Mining.-~2023.-Vol.3(4)-P/618-644.DOI
~\href{https://doi.org/10.3390/mining3040034}{10.3390/mining3040034}.

15. Lutsenko S., Hryhoriev Y., Kuttybayev A., Imashev A., Kuttybayeva A.
(2023). Determination of mining system parameters at a concentration of
mining operations// Series of geology and technical sciences. -2023.
-Vol.1(457). -P.130-140. DOI 10.32014/2023.2518-170x.264.
\end{refs}

\begin{info}
\hspace{1em}\emph{{\bfseries Автор туралы мәліметтер}}

Кожантов А.У.- т.ғ.к., Сәтбаев университеті, Алматы, Қазақстан,
e-mail: a.kozhantov@satbayev.university;

Абен Е.Х. -т.ғ.к., қауымдастырылған профессор, Сәтбаев университеті,
Алматы, Қазақстан, e-mail: y.aben@satbayev.university;

Мырзахметов С.С. - т.ғ.к., қауымдастырылған профессор, Сәтбаев
университеті, Алматы, Қазақстан, e-mail:
s.myrzakhmetov@satbayev.university;

Ахметканов Д.К. - т.ғ.к., қауымдастырылған профессор, Сәтбаев
университеті, Алматы, Қазақстан, e-mail:
d.akhmetkanov@satbayev.university.

\hspace{1em}\emph{{\bfseries Information about the authors}}

Kozhantov A.- Candidate of Technical Sciences, Satbayev University,
Almaty, Kazakhstan, e-mail: a.kozhantov@satbayev.university;

Aben Y. - Candidate of Technical Sciences, Associate Professor of
Satbayev University, Almaty, Kazakhstan, e-mail:
y.aben@satbayev.university;

Myrzakhmetov S. - Candidate of Technical Sciences, Associate Professor
of Satbayev University, Almaty, Kazakhstan, e-mail:
s.myrzakhmetov@satbayev.university;

Akhmetkanov D. - Candidate of Technical Sciences, Associate Professor
of Satbayev University, Almaty, Kazakhstan, e-mail:
d.akhmetkanov@satbayev.university.
\end{info}
