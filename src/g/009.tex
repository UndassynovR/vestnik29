\id{МРНТИ 52.01.76}{}

{\bfseries БЕЛАРУСЬ РЕСПУБЛИКАСЫНДАҒЫ ЗИЯНДЫ ЖӘНЕ ҚАУІПТІ ЕҢБЕК ЖАҒДАЙЛАРЫ}

{\bfseries Л.М.
Актаева}\fig{g3/image35}{}{\bfseries ,
Ә.Б.
Бекмағамбетов}{\bfseries ,
А.М.
Рахметова}{\bfseries ,
Э.А.
Құлмағамбетова}\envelope ,

{\bfseries А.О.
Абжаппарова}{\bfseries ,
\tsp{1}Н.Б.
Әбдрахманова}{\bfseries ,
Н.Т.
Сағындықова}

\emph{Қазақстан Республикасы Еңбек және халықты әлеуметтік қорғау
министрлігінің}

\emph{Еңбекті қорғау жөніндегі республикалық ғылыми-зерттеу институты
ШЖҚ РМК, Астана, Қазақстан}

\envelope Корреспондент - автор:
elya\_kulmagambet@mail.ru

Денсаулыққа қауіпті жағдайларда жұмыс істейтін қызметкерлер үшін
әлеуметтік кепілдіктер тақырыбы Беларусь Республикасында өзекті болып
табылады және егжей-тегжейлі қарауды қажет етеді. Мұндай жағдайларда
жұмыс істеу денсаулыққа ауыр зардаптарға әкелуі мүмкін, бұл мұндай
жұмысшылар үшін қосымша қорғауды қамтамасыз етуді қажет етеді.

Бұл жұмыста зиянды және қауіпті еңбек жағдайларында жұмыс істейтін
адамдар үшін Беларусь заңнамасында қарастырылған әлеуметтік
кепілдіктердің негізгі түрлері қарастырылады.

Өтемақы, қысқартылған жұмыс уақыты, медициналық тексерулер мен еңбек
жағдайлары, сондай-ақ кәсіптік аурулар немесе жарақаттар туындаған
жағдайда құқықтық қорғау және төлемдер мәселелеріне ерекше назар
аударылады.

Зерттеу барысында қолданыстағы әлеуметтік қорғау жүйесінің күшті жақтары
да, қызметкерлердің қауіпсіздігі мен денсаулығын тиімдірек қамтамасыз
ету үшін оны жетілдірудің мүмкін жолдары да анықталды.

{\bfseries Түйін сөздер:} әлеуметтік қорғау, еңбек жағдайы, кепілдіктер,
зиянды және қауіпті еңбек жағдайлары, өтемақылар

{\bfseries ВРЕДНЫЕ И ОПАСНЫЕ УСЛОВИЯ ТРУДА В РЕСПУБЛИКЕ БЕЛАРУСЬ}

{\bfseries Л.М.Актаева, А.Б. Бекмагамбетов, А.М. Рахметова}, {\bfseries Э.А.
Кульмагамбетова\envelope ,}

{\bfseries А.О. Абжаппарова, Н.Б. Абдрахманова, Н.Т. Сагиндикова}

\emph{РГП на ПХВ Республиканский научно-исследовательский институт по
охране труда}

\emph{Министерства труда и социальной защиты населения Республики
Казахстан, Астана, Казахстан,}

e-mail:elya\_kulmagambet@mail.ru

Тема социальных гарантий для работников, трудящихся в условиях, опасных
для здоровья, является актуальной в Республике Беларусь и требует
детального рассмотрения. Работа в таких условиях может иметь серьезные
последствия для здоровья, что делает необходимым обеспечение
дополнительной защиты для таких работников.

В данной работе рассматриваются основные виды социальных гарантий,
предусмотренные белорусским законодательством для людей, трудящихся во
вредных и опасных условиях труда.

Особое внимание уделяется вопросам компенсаций, сокращенного рабочего
времени, медицинских обследований и условий труда, а также правовой
защите и выплатам в случае возникновения профессиональных заболеваний
или травм.

В ходе исследования выявлены как сильные стороны существующей системы
социальной защиты, так и возможные пути её совершенствования для более
эффективного обеспечения безопасности и здоровья работников.

{\bfseries Ключевые слова}: социальная защита, условия труда, гарантии,
вредные и опасные условия труда, компенсации.

{\bfseries HARMFUL AND DANGEROUS WORKING CONDITIONS IN THE REPUBLIC OF
BELARUS}

{\bfseries L.M. Aktayeva, A.B. Bekmagambetov, A.M. Rakhmetova},
{\bfseries E.A. Kulmagambetova\envelope ,}

{\bfseries A.O. Abzhapparova, N.B. Abdrakhmanova, N.T. Sagindykova}

\emph{\tsp{1}RSE at the National Research Institute for
Occupational Safety of the Ministry of Labor and Social Protection of
the Population of the Republic of Kazakhstan, Astana, Kazakhstan,}

e-mail:elya\_kulmagambet@mail.ru

The topic of social guarantees for workers working in conditions
dangerous to their health is relevant in the Republic of Belarus and
requires detailed consideration. Working in such conditions can have
serious health consequences, which makes it necessary to provide
additional protection for such workers.

This paper examines the main types of social guarantees provided by the
Belarusian legislation for people working in harmful and dangerous
working conditions.

Special attention is paid to compensation, shorter working hours,
medical examinations and working conditions, as well as legal protection
and payments in case of occupational diseases or injuries.

The study identified both the strengths of the existing social
protection system and possible ways to improve it to better ensure the
safety and health of employees.

{\bfseries Keywords}: social protection, working conditions, guarantees,
harmful and dangerous working conditions, compensation

{\bfseries Кіріспе.} Қазіргі өндірісте салауатты және қауіпсіз еңбек
жағдайларын толық қамтамасыз ету көбінесе қол жетпейтін идеал болып
табылады. Сондықтан қауіпсіз технологияларды енгізу мүмкін емес немесе
экономикалық тұрғыдан қолайсыз жерлерде өндірістік әлеуметтік-техникалық
жүйелердегі адам факторын есепке алу зиянды және қауіпті жағдайларда
жұмыс істегені үшін өтемақы ұсыну жолымен жүреді.

Біздің елімізде зиянды және (немесе) қауіпті еңбек жағдайларында жұмыс
істегені үшін өтемақы жүйесі жұмыс күшінің ықтимал және нақты тозу
қаупін өтеу қажет деген ережеге сүйене отырып құрылды. Өндірістік
жағдайлар жұмыскердің денсаулығына кері әсер ететіні белгілі жай.

Сондықтан, бұл жағдайда жұмыс күшінің нақты тозуы қалыпты еңбек
жағдайымен, жұмыс қабілеттілігі мен денсаулығының жоғалуымен
салыстырғанда ертерек белең алады деп саналады. Негізінде бұл жүйеге
КСРО-да қалыптасқан қағидалар тірек болған.

\emph{{\bfseries Зиянды және (немесе) қауіпті еңбек жағдайларында жұмыс
істегені үшін берілетін өтемақының жалпы сипаттамасы.}} ТМД елдерінде ол
бұрынғы күйінде қызмет етеді немесе оның функциялары ұжымдық-шарттық
реттеу тетігі арқылы жүзеге асырылуда. Зиянды өтеу мәселелерін шешуде
Беларусь Республикасы Конституциясының 46, 47, 61-баптары {[}1{]} және
Беларусь Республикасы Еңбек кодексінің нормалары {[}2{]} құқықтық негіз
болып табылады. Еңбек кодексінің 11-бабының 7-тармағына сәйкес қызметкер
әлеуметтік сақтандыруға, зейнетақымен қамсыздандыруға және кәсіптік
ауруы, еңбек жарақаты, мүгедектігі жағдайында айтулы кепілдіктерге
құқылы.226-баптың 13-тармағында жұмыскердің өмірі мен денсаулығына
келтірілген зиянды өтеу міндеті белгіленген, ал 224-бапқа сәйкес жұмыс
беруші жұмыскердің өндірістегі жазатайым оқиғалардан және кәсіптік
аурулардан сақтандырылуы міндеттелген {[}3{]}.

{\bfseries Материалдар мен әдістер.} \emph{Қысқартылған жұмыс күні және
қосымша демалыс}. Беларусь Республикасының Еңбек кодексінің 113
бабына сәйкес еңбек жағдайлары зиянды және (немесе) қауіпті жұмыстарда
жұмыс істейтін қызметкерлер үшін жұмыс уақытының қысқартылған ұзақтығы
-- аптасына 36 сағаттан аспайтын болып белгіленеді. Қысқартылған жұмыс
күні уақытпен қорғау функциясын орындайды - зиянды факторлардың әсер ету
уақытын азайту, жұмысшыларды олардың әрекет ету аймағынан шамалы уақытқа
алшақтату арқылы жұмысшыларға зиянды еңбек жағдайларының әсер етуін
төмендетеді

Нәтижесінде жұмыс күшінің артық шығыны азаяды, оның қалпына келтірілуі
жеңілдейді.

Қосымша демалыс уақытпен оңалту функциясын орындайды - берілген демалыс
кезеңінде (негізгі демалыспен бірге) қызметкер еңбек міндеттерін орындау
мүмкіндіктерін қалпына келтіруі керек.

Қазіргі уақытта өтемақылардың осы түрлерін қолдану шарттарын реформалау
бойынша, олар бүгінгі күннің нақты талаптарына жауап беруі, ұйымдардағы
еңбек жағдайларының жай-күйімен тығыз байланысты болуы, жұмыс беруші мен
қызметкерлерді еңбек жағдайларын жақсартуға ынталандыру тетігінің
элементіне айналуы үшін қарқынды жұмыс жүргізілуде. Бұл норма Беларусь
Республикасының Еңбек кодексіне енгізілген өзгерістерге сәйкес
қолданысқа енгізіледі {[}4{]}.

Көрсетілген өтемақылар жұмыскерлерге еңбек жағдайлары бойынша жұмыс
орындарын аттестаттаумен міндетті байланыста берілетін болады.

{\bfseries \emph{Ерекше еңбек жағдайларына байланысты жұмысы үшін жасына
байланысты зейнетақы}.} Жалпы, зейнетақының бұл түріне жұмсалған
шығындар жалпы зейнетақы жүйесі қаражатының 4,5 - 5\% немесе еңбек
зейнетақыларына жұмсалған шығыстардың 5,5 - 5\% құрады. Мерзімінен бұрын
зейнетақыға арналған шығыстар еңбек зейнетақыларын (жалпы зейнетақы
жүйесінің қаражаты) төлеуге арналған халықты әлеуметтік қорғау қорының
қаражатынан қаржыландырылады.

Ерекше (зиянды және қауіпті) еңбек жағдайларына байланысты жасына
байланысты мерзімінен бұрын зейнетақымен қамсыздандыруды ұсынудың
құқықтық базасы «Зейнетақымен қамсыздандыру туралы» Беларусь Республика
Заңының 12-16-бабы болып табылады {[}5{]}. Мұндай зейнетақы еңбек
жағдайлары бойынша жұмыс орындарын аттестаттау негізінде тағайындалады
және меншік нысанына қарамастан барлық ұйымдар мен басқа шаруашылық
жүргізуші субъектілер үшін бірыңғай процесс болып табылады. Қазіргі
уақытта ерекше еңбек жағдайларында жұмыс істегені үшін жасына байланысты
зейнетақы алуға құқық беретін өндірістердің, жұмыстардың, кәсіптердің,
лауазымдар мен көрсеткіштердің жаңа тізімдері қолданысқа енгізілді
{[}6{]}.

\emph{{\bfseries Жоғары мөлшерде еңбекақы төлеу.}} Зиянды және (немесе)
қауіпті еңбек жағдайлары бар жұмыстарда істейтін жұмыскер жоғары
мөлшерде еңбекақы төлемін алуға құқылы.

Біріншіден, Беларусь Республикасы қызметкерлерінің Бірыңғай тарифтік
сеткасын қолдану тәртібі туралы нұсқаулыққа сәйкес {[}7{]}, жұмыстың
технологиялық түрлері, экономикалық қызмет түрлері және де жұмысы зиянды
және (немесе) қауіпті жағдайларда атқарылады деп ұйғарылатын салалар
бойынша орындалатын жұмыстардың технологиялары мен күрделілігіне,
өндіріс түрі мен салалық тиістілігіне қарай жұмысшылардың тарифтік
ставкаларын (жалақыларын) арттыру коэффициенттері белгіленеді. Бұл норма
жанама түрде аумақтарды залалсыздандыру кезінде, химиялық өндірістерде,
суды тазарту және хлорлау кезінде, мұнай өнімдерін, резеңке және
пластмасса бұйымдарын, шыны талшықты, тау-кен өнеркәсібінде және т. б.
өндіруде, сондай-ақ жекелеген жұмыстарды жүргізу кезінде, мысалы,
дефектоскопиямен, қол еңбегімен және т. б. байланысты зиянды өндірістік
факторлардың әсеріне байланысты жұмыстарға қатысты.

Екіншіден, жұмыс орындарын аттестаттау нәтижелері бойынша еңбек
жағдайларына олардың зияндылық дәрежесіне қарай тарифтік ставкалар мен
лауазымдық төлемақыларға қосымша ақылар белгіленеді.

Зиянды және ауыр еңбек жағдайларында жұмыс істегені үшін ең аз
кепілдендірілген қосымша ақының мөлшерін айқындау кезінде БР Үкімет
кеңесі белгілейтін бірінші разрядты тарифтік ставка қолданылады. Жалға
алушының ең төменгі кепілдендірілген төлемдерден төмен қосымша төлем
мөлшерін белгілеуге құқығы жоқ. Жалға алушы бірінші разрядты тарифтік
мөлшерлемені БР мин мөлшерінен асатын мөлшерде қолданған жағдайда,
қосымша төлем сомаларын есептеу жалға алушы қолданатын бірінші разрядты
тарифтік мөлшерлемеден өнімнің (жұмыстардың, көрсетілетін қызметтердің)
өзіндік құнына толық көлемде төленген қосымша төлем сомаларын жатқыза
отырып жүргізіледі.

\emph{{\bfseries Емдік-профилактикалық тамақтану.}} Кейбір өндірістерде
қолайсыз өндірістік факторлардың адам ағзасына әсерін әрдайым техникалық
шаралар, жеке және ұжымдық қорғаныс құралдары, еңбек пен демалысты
ұйымдастыру арқылы жол бермеу мүмкін болмайды. Мұндай жағдайларда
олардың әсері емдік-профилактикалық тамақ, сүт немесе оған теңестірілген
тамақ өнімдерін беру жолдарымен бәсеңдетіледі немесе жойылады.

Сүт беру мәселелері зиянды заттармен жұмыс істеу кезінде жұмысшыларды
сүтпен немесе оған теңестірілген азық-түлікпен тегін қамтамасыз ету
ережелерімен {[}8{]} және профилактикалық мақсатта жұмыс істеу кезінде
сүт немесе оған теңестірілген тамақ өнімдерін пайдалану көрсетілген
зиянды заттардың тізімімен реттеледі {[}9{]}.

Жұмыскерлерді сүтпен тегін қамтамасыз ету мәселелерін шешкен кезде
зиянды заттармен жұмыс істеуге қызметкердің олармен байланысын
анықтайтын кез-келген кәсіби қызмет, оның ішінде оларды өндіру, қолдану,
сақтау, тасымалдау, пайдалану, жөндеу және тазарту, технологиялық
жабдықтар, ыдыстар, қондырғы, қорғаныс құралдары және басқа да
өндірістік процестер, сондай-ақ I және II класты жұмыстарға сәйкес
келетін жұмыс орнындағы белсенділігі бар ашық радионуклидті көздермен
жұмыс істеу кіретінін есте ұстаған жөн.

{\bfseries Нәтижелер және талқылау\emph{. }}Қазіргі кезеңдегі
проблемалар. Соңғы жылдары өтемақыны пайдаланатын жұмысшылар
саны көптеген позициялар бойынша тұрақты өсу тенденциясына ие, бұл
Беларусь Республикасының өнеркәсібінде жиі кездеседі.

Қазіргі уақытта ең көп қолданылатын өтемақылардың бірі ретінде
қарастыруға болатын зиянды және ауыр еңбек жағдайларында жұмыс істегені
үшін қосымша ақы алатын жұмысшылар саны артып келеді. Өтемақының осы
түрін пайдаланатын жұмыскерлердің үлес салмағы 2011 жылдан бастап 1,5
есе өсті.

Беларустағы қазіргі жағдай қолданыстағы өтемақы тетіктерін сақтау еңбек
жағдайларын жақсартуға елеулі кедергілердің бірі болып табылады деп
есептеуге мүмкіндік береді. Қолданыстағы экономикалық механизм
өндірістік орта факторларының жақсаруын және осы негізде кез-келген
өтемақыны пайдаланатын жұмысшылар санының, сондай-ақ байланысты
шығындардың азаюын ынталандырмайды. Сондықтан өтемақы жүйесі
кәсіпорындарды аса күйзелтпейді, олардың жұмыстарының экономикалық
көрсеткіштеріне әсері де шамалы. Жұмысшылар да, жұмыс берушілер де
жеңілдіктердің жоғалуына әкелетін еңбек жағдайларын жақсартқысы келмейді
- жұмысшылар негізгі жалақыға қандай да бір қосымша алуға қанағаттанады,
ал жұмыс берушілер оларды кадрларды тарту және бекіту мақсатында жиі
пайдаланады.

Өтемақы (компенсация) жүйесінің елеулі кемшілігі - жұмысшыларды
өндірістік орта факторларының қолайсыз әсерінен әлеуметтік қорғау
мәселесін тұтастай шешпестен, бұл еңбек жағдайларын жақсарту жөніндегі
қызметті ынталандырмайтындығы. Бұл қолайсыз еңбек жағдайлары жұмыс істеп
тұрған және жаңадан іске қосылған кәсіпорындарда айтарлықтай дәрежеде
сақталатынын түсіндіреді.

Бұл жүйені шамадан тыс орталықтандыру жауапкершілікті азайтады және
жұмыс берушінің дербестігін шектейді. Өтемақы берудің құқықтық негізі
болып табылатын кәсіптердің, лауазымдардың, жұмыстардың, өндірістердің
тізімдері мен тізімдері белгілі бір жұмыс орнындағы еңбек жағдайларын
бағалаудың объективтілігімен әлсіз байланысты, сондықтан өтемақылар,
әдетте, еңбек жағдайындағы өзгерістерді ескермей, ресми негізде
беріледі.

Еуроодақтық бірде-бір елде жарты ғасырға жуық уақыттан бері мұндай жүйе
жоқ. Жаһанданудың ғаламдық процестері олардың әлеуметтік жүйелерінің
құқықтық және әкімшілік құрылымдарын технологиялар мен экономиканың
дамуымен сәйкестендіруге мәжбүрлейді. Қазір денсаулық қайтарымсыз
инвестиция ретінде қарастырылмайды -- еңбек қауіпсіздігі мен еңбекті
қорғауға салынған инвестициялар адами капиталдың өсуіне апарады және
қазіргі мен болашақ ұрпақтың өмірлік әлеуметтік табысына айналады.
Еңбекті қорғауға және оның қауіпсіздігін арттыруға барынша көп
инвестиция салатын елдер ең жоғары еңбек өнімділігіне және бәсекеге
қабілетті экономикаға ие екендігі анықталды.

Ғаламдық ауқымдағы бәсекелестіктің өсуі жұмыс берушілердің еңбекті
қорғау шараларының құнын, нақты өтемақыларды қосқанда, барынша азайтуға
ұмтылуына мәжбүр жағдайға алып келеді. Бұл объективтi түрде iс-шараларды
мақсатты әзiрлеу қағидаларын белгiлеудi, жарақаттану мен
сырқаттанушылықтың алдын алу жөнiндегi шығыстарды әрбiр жеке субъектiге
қатысты бақылау қажеттiлiгiн көздейдi.

Альтернативтік нұсқа ретiнде өндірістік қауіп-қатерге қарсы денсаулық
сақтау үшін алдын алу шараларын қабылдау керісінше әдептiлік болып
есептеледi. Ал оған жұмыскерлердің зиянды және қауіпті жағдайларда жұмыс
істеуден бас тартуға құқылылығы туралы шыншыл ақпарат жатады, оларға
қатысты санкцияларға тыйым салынады. Бұл норма ХЕҰ-ның «Еңбек
қауіпсіздігі және еңбекті қорғау және еңбек ортасы туралы» № 155
конвенциясында бекітілген және тиісінше біздің заңнамамызда қолданылады
{[}10{]}.

Жұмыс берушiнiң қызметi жұмыскерлердiң денсаулығы мен еңбек қабiлетiн
сақтауды көздейді және сақтандыру шығындары мен жұмыскерлерге өтемақыны
төмендету жолымен өндiрiстiң тиiмдiлiгiн арттыруға бағытталған. Бұл
ұйғарымдар үнемді көзқарас тұрғысынан шаралар таңдауға әсер етеді және
де жеке жұмысшы үшін оңтайлы шараларды бүкіл кәсіпорын үшін тиімді
шаралармен қалай теңестіруге болатыны жайлы көбінесе дилемма туындатады.
Жұмыскерлердi қорғаныш құралдарымен, емдеу және денсаулықты нығайту
жөнiндегi қызметтермен қамтамасыз ету сақтандыру төлемдерiне бөлінетін
қаражатты үнемдеу жолымен шығындарды азайтуға мүмкiндік береді. Оңалту
шаралары мен сақтандыру төлемдерiнiң арасындағы тәуелді байланыс жұмыс
берушi мен жұмыскерлердiң өкiлдi органдары арасындағы тарифтiк шарттарда
белгiленуi мүмкiн. Жұмыс орны бойынша денсаулық сақтау бағдарламаларын
жасау арқылы жұмыс берушi зиянды өндiрiстiк факторлардың әсерiн өтеуге
қатысуы мүмкiн. Оларды жоспарлау және iске асыру ұстанымдары мемлекеттiк
деңгейде әзiрленген. Iске асырылған бағдарламалардың көпшiлiгiн бағалау
олардың тиiмдiлiгi жоғары екендiгiн көрсетедi.

Батыс елдерінде тамақтану бағдарламалары біздің елімізде қабылданған
емдік-профилактикалық тамақтанудың аналогы болып табылады, олар көбінесе
жалпы денсаулық сақтау бағдарламаларының бір бөлігі болады. Тамақтану
мәселелеріне байланысты ақпараттық материалдар парақшалар, жазбаша
хаттар, плакаттар, бюллетеньдер және электрондық пошта хабарламалары
түрінде жұмыскерлерге де, олардың отбасы мүшелеріне де таратылады.
Отбасы мүшелерінің талғамдарын қалыптастыратын отбасы әйелдері
болғандықтан, олар арнайы сабақтар мен семинарларға шақырылады. Осыған
орай, тамақтану орындарында диетологтың ұсынымдарында көрсетілген тиісті
тамақтардың жұмыскерлер үшін қол жетімді болуы маңызды. Көптеген жұмыс
берушілер жұмыста тамақтану пункттерін ішінара немесе толық
субсидиялайды.

Тәуекелді (қатерді) регламенттеудің тағы бір бағыты қатерлі ісікке
шалдығуға соқтыратын қауіпті салаларды көрсету болып табылады.
Болжамдарға сай, алдағы онжылдықта көптеген индустриялық елдерде қатерлі
ісік аурулары өлім-жітімнің негізгі себебіне айналуы мүмкін. Жұмыс
берушілермен канцерогендік заттарға шалдыққан қызметкерлерге толық
медициналық тексеру жүргізіледі, олардың жұмысқа қабілетсіздік туындауы
мен өлім себептеріне талдау жасалады. Зиянды факторлармен күресу жұмыс
орнында әлеуетті канцерогендердің әсерін болдырмауды немесе
мүмкіндігінше олардың әсерін азайтуды, сондай-ақ канцерогенді
материалдарды нақты таңбалауды және осындай материалдарды пайдалану,
сақтау және жою ережелеріне жұмысшыларды ұдайы оқытуды қамтиды.

Дамыған елдердегі қабылданған сақтандыру жүйелерінде өндірісте жарақат
алған немесе кәсіптік ауруға шалдыққан жұмыскерлерге медициналық көмек
көрсетіледі немесе ақшалай өтемақы төленеді. Алайда, әртүрлі
мемлекеттерде қабылданған жүйелер арасында олардың бағыныштылығына,
қолданылу аясына, сақтандыру жәрдемақыларын төлеу тәртібіне, жазатайым
оқиғалардың алдын алуда мудделілік дәрежесіне және техникалық қамтамасыз
етуге қатысты айырмашылықтар орын алады. АҚШ әр штатының тәуелсіз
сақтандыру жүйесі бар, және онда жеке сақтандыру компаниялары рөлдерінің
маңызы зор. Францияда, керісінше, сақтандыру жүйесі толығымен мемлекетке
бағынады және еңбекті қорғау мәселелерімен айналысатын әкімшілік
органдар жүйесінің бөлігі болып табылады.

Германияда зиянды және (немесе) қауіпті еңбек жағдайларында еңбектену
нәтижесінде денсаулыққа зиян келтіргені үшін кепілдіктер белгілеу
сақтандыру институты аясында жүзеге асырылады. Өндірістегі жарақаттану
мен кәсіптік аурушаңдыққа қатысты мәселелерді реттеу мен шешуді
сақтандыру компаниялары кәсіптік одақтармен тығыз байланыста бола тұра,
өздерінің мойындарына алады. Жазатайым оқиғалардан сақтандыру жүйесі --
бұл заңмен белгіленген бес әлеуметтік сақтандыру саласының бірі. Бұл өз
ережелері мен нормаларын қабылдауға, алдын-алу шараларын нақтылауға және
қалыптастыруға құқығы бар жеткілікті тәуелсіз жүйе.

1970 жылы қабылданған АҚШ-тың өндірістік қауіпсіздік және денсаулық
актісі (Тhe US Occupational Safety and Health Act) жұмыс берушілер мен
жұмысшылардың қауіпсіз және салауатты еңбек жағдайларына қол жеткізу
ісінде дербес, бірақ өзара байланысты жауапкершіліктері мен құқықтарының
болуын қарастырады. Соңғы онжылдықта АҚШ-та денсаулыққа қауіп дәрежесін
бағалау (HRA) танымалдылыққа ие болды. Бұл жұмыскерлердің денсаулықтарын
сақтаудың маңыздылығын түсінуіне және басшылықтың осы мақсатқа жетудегі
іс-әрекеттердің қажеттілігін түсінуіне ықпалын тигізеді. Онымен «өмір
сүру нормасы» деп аталатын халықтың белгілі санаты үшін бағдар ретінде
оңтайлы өмір сүру ұзақтығы анықталады. Мұнда нақты адамның
ерекшеліктерін ескеретін жеке ұстаным қолданылады. Мәселен, қалыпты қан
қысымы және қандағы холестерин деңгейі төмен, жақсы генетикалық тарихы
бар, физикалық жаттығулармен айналысатын және жұмыс орнында қауіпсіздік
талаптарын орындайтын адам, белгілі бір жұмыс орнының қауіптілігіне
қарамастан, тәуекел дәрежесінің жақсы бағасын ала алады. HRA жүргізу
жауапкершілігі әдетте еңбекті қорғау бойынша жауапты басшыға жүктеледі.
Қазіргі уақытта бұл мәселелермен жұмысшылар мен жұмыс берушілердің
мүдделерін білдіретін қоғамдық ұйымдар немесе екі жақты еңбекті қорғау
комитеттері көбірек айналысады. Осыған байланысты біз сонымен қатар
еңбекті қорғау комитеттері («Еңбекті қорғау туралы» Беларусь
Республикасы Заңының жобасына сәйкес ұсынылатын) институтына үлкен үміт
артамыз.

Еңбекті қорғау жөніндегі іс-шараларды өткізу үшін соттар жұмыскердің
пайдасына өтемақы тағайындаған жұмыскерлердің еңбекке қабілеттілігінен
айырылу жағдайлары сөзсіз дәлел болып табылады. Егер жұмыскердің
мерзімінен бұрын зейнетке шығуын оның еңбекке қабілеттілігін қолайсыз
жағдайлардағы жұмысқа байланысты жоғалту салдары екеніне сот жұмыс
берушіні жауапты деп тапқан болса да. Мұндай дәлел жұмыскерлерге
келтірілген зиянның салдарынан ақшалай өтемақы төленген жағдайларға да
қатысты. Егер жарақат пен сырқаттанудың жұмысқа қатыстылығын дәлелдеуге
болатын болса, онда жұмыс берушілер заң бойынша медициналық көмек пен
табыстың азаюына жауапты. Бұл көмек жүйесі ауыр жарақаттар мен ауруларға
шалдыққан жұмысшыларға қатысты оңды, себебі бұл жерде еңбек
жағдайларының рөлі айдан анық. Сондықтан да жұмыскер, сот органдарының
көмегінсіз, өз орнына тезірек орала алуы үшін жұмыс беруші уақтылы және
тиімді емдеуге мүдделі.

Айта кететін жай, бұрынғы КСРО мен Шығыс Еуропа елдері Еуропалық Одаққа
кірген кезде, зиянды және (немесе) қауіпті жағдайларда жұмыс істегені
үшін өтемақы беруге қатысты құқықтық нормалар жойылады, өйткені олар
қолданыстағы еңбекті қорғау идеологиясына қайшы келеді. Бұл, әдетте,
кәсіподақтар тарапынан қандай да бір ауқымды акциялар мен шерулердің
болуына апармайды. Ал бізде болса, сол бұрынғыдай зиянды жағдайда жұмыс
істейтін адамның қосымша өтемақы алмауы мүмкін емес деп саналады. Біраз
уақыт бұрын Ресей Федерациясының Вологда қаласындағы бірқатар
кәсіпорындарда еуропалық еңбекті қорғау жүйелерін енгізу бойынша
эксперимент жүргізілген болатын {[}11{]}. Онда жұмыс жағдайлары
түбегейлі жақсарды, бірақ жұмысшылардың өтемақылары бәрібір сақталған.

\emph{{\bfseries Болашақ перспективалар.}} Қазіргі уақытта Беларусь
Республикасында жазатайым оқиғалардан және кәсіптік аурулардан
сақтандыру жүйесі аясында қолданылатын қағидаттарға зиянды және (немесе)
қауіпті жағдайларда еңбектенгені үшін өтемақы жүйелеріне бағдарланған
бағытқа байланысты еңбек саласындағы саралаудың негізгі критерийі
кәсіптік тәуекелдер болып есептеледі.

Жұмыскердің осы жағдайда болуы оған нақты өндірістік жағдайларға тән
кәсіби тәуекелдердің тұрақты әсер ету мүмкіндігін анықтайды. Сондықтан
белгілі бір өндірістермен және технологиялық процестермен тығыз
байланыста өтемақы жүйесін құру шынайы болып көрінеді.

Бұл жағдайда қаралып отырған өндірістік қызмет Беларусь Республикасы
Еңбек және әлеуметтік қорғау министрлігінің 2006 жылғы 29 наурыздағы №
38 {[}11{]} қаулысымен (бұдан әрі -- Тізбе) бекітілген «зиянды және
(немесе) қауіпті жағдайлары бар жұмыс түрлері» тізбесімен
регламенттелген жұмыстармен сәйкестендірілуі тиісті. Осы негіздемелік
құжатқа толығырақ тоқталу керек.

Елде қолданылатын салаларды, экономикалық қызмет түрлерін, өндірістер
мен жұмыс түрлерін жіктеу қағидаттары «Халық шаруашылығы салалары»
жалпыодақтық жіктеуішінде (ХШСЖК) және болашақта ХШСЖК ауыстыруға тиіс
«Экономикалық қызмет түрлері» (ЭҚЖК) Беларусь Республикасының
жалпымемлекеттік жіктеуішінде белгіленген. Бұл жіктеуіштер Беларусь
Республикасының техникалық-экономикалық және әлеуметтік ақпаратты жіктеу
мен код беру бірыңғай жүйесінің (ТЭӘА БККЖ) құрамдас бөлігі болып
табылады. ЭҚЖК әзірлеу Еуропалық одақтың экономикалық қызмет түрлерінің
жіктеуішіне (ЕОҚК) негізделген.

Қазіргі уақытта ел экономикасының қолданыстағы статистикасы ХШСЖЖ
негізделеді, өндірістегі жазатайым оқиғалардан және кәсіптік аурулардан
әлеуметтік сақтандыру тарифтері де ХШСЖЖ бөлінісінде белгіленеді;
Жұмыстар мен жұмысшы кәсіптерінің бірыңғай тарифтік-біліктілік
анықтамалығы~ (БТБА) мен Басшылар, мамандар және өзге де қызметшілер
лауазымдарының біліктілік~ анықтамалығының (БА) көптеген бөлігі де ХШСЖЖ
ескере отырып қалыптастырылған.

Барлық белгілі жіктеуіштер, әдетте, қарапайым жүйеден күрделі жүйеге
дейінгі мақсаттар мен нәтижелердің иерархиялық құрылымымен құрылады және
сәйкесінше объектілердің белгілі бір жиынтығының ағаш діңгегі тәрізді
құрылымы болып табылады. Бұл құрылымның жоғарғы жағында біртұтас жіктеу
- тамыр түйіні (Тізімдегі түбір түйіні, мысалы, «02 Металл өңдеу
өндірісі») болады. Ол осы классификацияның барлық объектілеріне қатысты.
Түбірден төмен орналасқан түйіндер жіктелетін объектілердің жалпы
жиынтығының ішкі жиындарына жататын барынша нақты жіктеулер болып
табылады.

Әзірленген Тізімде әлемдік тәжірибеде қабылданған жіктеудің иерархиялық
әдісі және код берудің сериялық-реттік әдісі қолданылады. Жіктеу
тереңдігі 3 деңгейлі болады. Толық код ұзындығы 7 сандық ондықты
қамтиды. Жіктеу деңгейіне байланысты жіктеу объектілерін код беру екі,
төрт және жеті белгілермен жүзеге асырылады:

1-ші деңгей (бөлім) - кәсіптердің (мамандықтардың) біліктілік
сипаттамаларын жалпылаудың салалық деңгейі -- БТБА және БА
шығарылымдарының атаулары негізінде тұжырымдалған айдарлар болып
табылады, белгілеу екі-цифрлық топтан тұрады (мысалы, «02 Металл өңдеу
өндірісі»);

2-ші деңгей (топ) - кәсіптердің (мамандықтардың) біліктілік
сипаттамаларын жалпылаудың өндірістік деңгейі - БТБА және БА
шығарылымдары бөлімдерінің атаулары, топтың белгіленуі бөлімнің
белгіленуінен және топ нөмірінен тұрады, яғни алдыңғы деңгейге қатысты
тағы екі-цифрлық топпен толықтырылады (мысалы, «0201 Құю (металл)
жұмыстары»);

3-ші деңгей (кіші топ) - зиянды және қауіпті еңбек жағдайлары бар жұмыс
түрі, кіші топтың белгіленуі алдыңғы деңгейдің белгіленуінен және реті
бойынша берілген нөмірді білдіретін үш-цифрлық топтан тұрады (мысалы,
«0201001 Автоклавтарға қызмет көрсету»).

Кез-келген жіктеуіштің, классификацияның негізгі қағидасы - бір сыныпқа
жатқызылған бірліктер әр түрлі топтар мен сыныптарға жатқызылған
бірліктерге қарағанда бір-бірінен аз ерекшеленеді. Бұл жағдайда жиынтық
бірлігі болып зиянды және (немесе) қауіпті еңбек жағдайлары бар жұмыс
түрі табылады. Бірліктерді сыныпқа біріктіру жіктеудің негізі ретінде
қызмет ететін белгілі бір жіктеу белгісі бойынша жүзеге асырылады.
Тізімде жіктеудің негізгі тұғырнамасы болып еңбек жағдайлары
белгілерінің салалық және өндірістік ортақтығы табылады. Зиянды және
қауіпті еңбек жағдайлары бар жұмыс түрлерін бөлу БТБА және БА тиісті
шығарылымдарына топтастырылған кәсіптердің біліктілік сипаттамаларын
талдау негізінде жүргізілді. Осылайша, Беларусь Республикасы
экономикасының барлық салаларының зиянды және (немесе) қауіпті
жағдайлары бар өндірістердің барлық түрлері мен жұмыс түрлерін толық
қамтуға мүмкіндік туды. Бұл - сапалық қасиет, сондықтан жіктеудің негізі
бола алады.

Біздің экономикамызды жалпы әлемдік экономикалық жүйеге интеграциялау
қажеттілігі еңбекті қорғау саласындағы қызметті ретке келтіру жөніндегі
бірыңғай қағидаттарды ұстану қажеттігін талап етеді. Оның градиенті
(белгілі шаманың ең жоғары өсім бағыты) өндірістегі жазатайым
оқиғалардан және кәсіптік аурулардан әлеуметтік сақтандыру жүйесі
аясында жұмыс беруші мен жұмыскердің еңбек жағдайларын жақсартуға
қызығушылығын арттыру мен зиянды және қауіпті жағдайларда жұмыс істегені
үшін өтемақы жүйесінен болашақта мүлдем бас тартуға мүмкіндік беретін
жалпы тетіктерді біркелкі пайдалануға бағытталуы тиіс. Экономикалық
дамыған елдерде қабылданған нормаларға жақындау кезең-кезеңімен
жүргізілуі мүмкін. Өтемақы жүйесін реформалаудың бірінші кезеңі жоғарыда
аталған Тізбенің қолданысқа енгізілуімен байланысты өткізілді.

Беларусь Республикасы Еңбек кодексінің 225-бабында көзделген еңбек
жағдайлары бойынша өтемақы алуға құқығы бар жұмыскерлердің кәсіптері мен
санаттарының тізбелері өтемақылардың барлық түрлері үшін әзірленбеген,
ал қолданыстағылары (№1, № 2 Тізімдерді қоспағанда) бір-бірімен
байланысты емес және өзгеріске ұшыраған нормативтік құқықтық құжаттар
мен өндіріс жағдайларына сәйкестендірілмеген. Сондықтан, осы Тізімді
енгізе отырып, кепілдіктер мен өтемақылардың бүкіл жүйесін ретке келтіру
мүмкіндігі туындайды, бұл оны зиянды және (немесе) қауіпті жағдайларда
жұмыс үшін өтемақының барлық түрлері белгіленуі ықтимал өндірістерді,
жұмыс түрлерін жіктеуге арналған жүйе құраушы құжат ретінде қарастыруға
мүмкіндік береді. Зиянды және (немесе) қауіпті еңбек жағдайлары бар
салалардағы, өндірістердегі, жұмыс түрлеріндегі жүргізілетін жүйелеу
жұмыстары осындай жағдайларда жұмыс үшін кепілдіктер мен өтемақылар
беруде бірегей ұстанымды қамтамасыз етуге мүмкіндік береді.

\emph{Екінші кезеңге деген} қажеттілік өтемақылардың (компенсациялар)
барлық жиынтығын ретке келтіруге ұмтылумен негізделеді. Қазіргі уақытта
берілетін өтемақылар жұмыскермен осыған дейін оның қай түрінің
алылғанына қарамастан берілуі мүмкін, бұл олардың орнын
өтеу(компенсациялық) функциясын айшықтаудын орнына бұлыңғыр етеді.
Сонымен қатар, барлық өтемақылар бір тізбекке қосылуы керек, мұнда
компенсациялардың кейбіреулерін қолдану фактісі бұрын басқа қандай
өтемақылар берілгеніне байланысты болады. Ол үшін өтемақының әр түрі
бойынша нормативтік шектеулер мен критерийлер әзірленіп, енгізілуі
керек. Егер өндірістік ортаның параметрлері белгіленген критерийлерден
асып кетсе, туындайтын жаңа тәуекелдерді өтеу негізгі өтемақыларды
толықтыруға арналған кейбір өтемақылар түріндегі бонустарды қолдану
арқылы жүзеге асырылады. Бұл кезеңде кейбір өтемақыларды өзгелеріне,
соның ішінде ақшалай баламаға ауыстыру тетіктері ұсынылуы керек.

Еңбек жағдайлары бойынша жұмыс орындарын аттестаттау мұндағы
өтемақылардың өздерін беру құқығын негіздеудің және оларды кешендеу
көлемін белгілеудің бірден бір тетігі болып табылады. Дегенмен,
қолданыстағы тәжірибе көрсеткендей, жұмыс орындарын аттестаттауды
осылайша ұғыну оның негізгі \emph{- еңбек жағдайларын жақсарту
мақсатында} \emph{өндірісті талдау} мән-мағынасын кемітеді. Оны
өтемақылардың барлық түрлеріне қатысты қарау олардың шама-шарықсыз өсіп
келе жатқан берілу көлемін шектеу үшін уақытша шара ретінде ғана қызмет
ете алады.

Сондықтан, соңғы, \emph{үшінші кезеңде} өндірістік тәуекелдерді алдын
ала (превентивтік) өтеуден зиянды және (немесе) қауіпті еңбек
жағдайларында жұмыс істейтін жұмыскерлердің әлеуметтік сақтандыру
қағидаттарына сәйкес денсаулықтың нақты зақымдануын атаулы өтеуге көшу
қажет. Еңбек зияндылығы мен қауіптілік дәрежесі сақтандыру тарифтерінің
мөлшерін атаулы белгілеу кезінде барабар көрініс табуы тиіс.

{\bfseries Қорытынды.} Жоғарыда айтылғандарды қорытындылай келе, өтемақы
берудің заманауи ұстанымы зиянды және (немесе) қауіпті еңбек
жағдайларының әсерін өтейтін барлық механизмдердің әрекеттесуін
біріктіру және ортақ жүйеге байланыстыру болып табылатынын атап өткен
жөн. Сондықтан Беларусь Республикасындағы мемлекеттік кепілдіктер мен
өтемақылар жұмысшыларға зиянды және (немесе) қауіпті жағдайларда
еңбектенгені үшін өтемақы беру профилактикалық сипатта емес, өз
мақсатына өндірістік жарақаттану мен кәсіптік сырқаттанушылықты алдын
ала болжайтын, анықтайтын, ескертетін сипатта болуы тиіс.

Мақалада «Қазақстан Республикасы Еңбек және халықты әлеуметтік қорғау
министрлігінің Еңбекті қорғау жөніндегі республикалық ғылыми - зерттеу
институты» ШЖҚ РМК-ның ғылыми - зерттеу жұмыстарын бағдарламалық -
нысаналы қаржыландыру шеңберінде «Қазіргі контексте зиянды еңбек
жағдайларында жұмыспен қамтылған адамдарға қатысты әлеуметтік
кепілдіктердің мемлекеттік тетігін трансформациялау» (ЖТН BR 22182673)
тақырып бойынша ғылыми-техникалық бағдарламаны іске асыру барысында
алынған ғылыми зерттеулердің нәтижелері ұсынылған.

{\bfseries Әдебиеттер}

1. Конституция Республики Беларусь 1994 года Беларусь (с изменениями и
дополнениями, принятыми на республиканских референдумах 24 ноября 1996
г., 17 октября 2004 г. и 27 февраля 2022 г. - 2022.

2. Трудовой кодекс Республики Беларусь: принят Палатой представителей 26
июля 1999 г., № 296 - 3 (с изменениями и дополнениями по состоянию на
01.01.2025 г.): одобр. Советом Республики 30 июня 1999 г. -1999.

3. Указ Президента Республики Беларусь «О страховой деятельности» от 25
августа 2006 г. № 530 // Национальный реестр правовых актов Республики
Беларусь (НРПА РБ). -2006. -№ 143,1/7866.

4. Закон Республики Беларусь «О внесении изменений и дополнений в
Трудовой кодекс Республики Беларусь» от 20 июля 2007 г. №272-З //
Звезда. - 2007.

5. Закон Республики Беларусь «О пенсионном обеспечении» от 17 апреля
1992 г. № 1596 -- XII // Ведомости Верховного Совета Республики
Беларусь. -1992. -№ 17. -С.275.

6. Список производств, работ, профессий, должностей и показателей на
подземных работах, на работах с особо тяжелыми условиями труда,
занятость которых дает право на пенсию по возрасту и за работу с особыми
условиями труда (Список № 1), Список производств, работ, профессий,
должностей и показателей на работах с вредными и тяжелыми условиями
труда, занятость которых дает право на пенсию по возрасту и за работу с
особыми условиями труда (Список № 2): утверждены Советом Министров
Республики Беларусь от 25 мая 2005 г. № 536 // НРПА РБ. -2005. - № 87,
5/ 16012.

7. Инструкция о порядке применения Единой тарифной сетки работников
Республики Беларусь: утверждена постановлением Министерства труда и
социальной защиты Республики Беларусь от 20 сентября 2002 г. № 123
(действует в ред. постановления Министерства труда и социальной защиты
Республики Беларусь от 22 декабря 2006 г. № 162 // НРПА РБ. -2007. - №
43, 8/15743.

8. Постановление Советом Министров Республики Беларусь «О бесплатном
обеспечении работников молоком или равноценными пищевыми продуктами при
работе с вредными веществами» от 27 февраля 2002 г. № 260 // НРПА РБ.
-2002. -№ 29, 5/10048.

9. Перечень вредных веществ, при работе с которыми в профилактических
целях показано употребление молока или равноценных пищевых продуктов:
утвержден постановлением Министерства труда и социальной защиты
Республики Беларусь и Министерства здравоохранения Республики Беларусь
от 19 марта 2002 г. № 34/12 // НРПА РБ. -2002. -№43, 8/7942.

10. Закон Республики Беларусь «О ратификации Конвенции 155 «О
безопасности и гигиене труда и производственной среде» Международной
организации труда» от 5 мая 1999 г. // НРПА РБ. -1999. -36, 2/28.

11. Волошина Т.Н., Грицкевич И.И., Кляуззе В.П. Предоставление доплат за
работу во вредных и опасных условиях. / Минск: НИИ труда. -2006. ISBN
985-6793-06-8.

{\bfseries References}



1. Konstitucija Respubliki Belarus'{} 1994 goda
Belarus'{} (s izmenenijami i dopolnenijami, prinjatymi na
respublikanskih referendumah 24 nojabrja 1996 g., 17 oktjabrja 2004 g. i
27 fevralja 2022 g.. - 2022. {[}in Russian{]}.

2. Trudovoj kodeks Respubliki Belarus': prinjat Palatoj
predstavitelej 26 ijulja 1999 g., №296 -- 3 (s izmenenijami i
dopolnenijami po sostojaniju na 01.01.2025 g.): odobr. Sovetom
Respubliki 30 ijunja 1999 g. -1999. {[}in Russian{]}.

3. Ukaz Prezidenta Respubliki Belarus'{} «O strahovoj
dejatel' nosti» ot 25 avgusta 2006 g. № 530 //
Nacional' nyj reestr pravovyh aktov Respubliki
Belarus'{} (NRPA RB). -2006. -№ 143,1/7866. {[}in
Russian{]}.

4. Zakon Respubliki Belarus'{} «O vnesenii izmenenij i
dopolnenij v Trudovoj kodeks Respubliki Belarus'» ot 20
ijulja 2007 g. №272-Z // Zvezda. - 2007. {[}in Russian{]}.

5. Zakon Respubliki Belarus'{} «O pensionnom
obespechenii» ot 17 aprelja 1992 g. № 1596 -- XII // Vedomosti
Verhovnogo Soveta Respubliki Belarus'. - 1992. - № 17. --
S.275. {[}in Russian{]}.

6. Spisok proizvodstv, rabot, professij, dolzhnostej i pokazatelej na
podzemnyh rabotah, na rabotah s osobo tjazhelymi uslovijami truda,
zanjatost'{} kotoryh daet pravo na pensiju po vozrastu i
za rabotu s osobymi uslovijami truda (Spisok № 1), Spisok proizvodstv,
rabot, professij, dolzhnostej i pokazatelej na rabotah s vrednymi i
tjazhelymi uslovijami truda, zanjatost'{} kotoryh daet
pravo na pensiju po vozrastu i za rabotu s osobymi uslovijami truda
(Spisok № 2): utverzhdeny Sovetom Ministrov Respubliki
Belarus'{} ot 25 maja 2005 g. № 536 // NRPA RB. -- 2005.
- № 87, 5/ 16012. {[}in Russian{]}.

7. Instrukcija o porjadke primenenija Edinoj tarifnoj setki rabotnikov
Respubliki Belarus': utverzhdena postanovleniem
Ministerstva truda i social' noj zashhity Respubliki
Belarus'{} ot 20 sentjabrja 2002 g. № 123 (dejstvuet v
red. postanovlenija Ministerstva truda i social' noj
zashhity Respubliki Belarus'{} ot 22 dekabrja 2006 g. №
162 // NRPA RB. -- 2007. - № 43, 8/15743. {[}in Russian{]}.

8. Postanovlenie Sovetom Ministrov Respubliki Belarus'{}
«O besplatnom obespechenii rabotnikov molokom ili ravnocennymi
pishhevymi produktami pri rabote s vrednymi veshhestvami» ot 27 fevralja
2002 g. № 260 // NRPA RB. -- 2002. - № 29, 5/10048. {[}in Russian{]}.

9. Perechen'{} vrednyh veshhestv, pri rabote s kotorymi v
profilakticheskih celjah pokazano upotreblenie moloka ili ravnocennyh
pishhevyh produktov: utverzhden postanovleniem Ministerstva truda i
social' noj zashhity Respubliki Belarus'{}
i Ministerstva zdravoohranenija Respubliki Belarus'{} ot
19 marta 2002 g. № 34/12 // NRPA RB. - 2002. - №43, 8/7942. {[}in
Russian{]}.

10. Zakon Respubliki Belarus'{} «O ratifikacii Konvencii
155 «O bezopasnosti i gigiene truda i proizvodstvennoj srede»
Mezhdunarodnoj organizacii truda» ot 5 maja 1999 g. // NRPA RB. - 1999.
- 36, 2/28.{[} in Russian{]}.

11. Voloshina T.N., Grickevich I.I., Kljauzze V.P. Predostavlenie doplat
za rabotu vo vrednyh i opasnyh uslovijah. / Minsk: NII truda. - 2006.
ISBN 985-6793-06-8. {[}in Russian{]}.

\emph{{\bfseries Авторлар туралы мәліметтер}}

Актаева Л.М. - медицина ғылымдарының докторы, Қазақстан
Республикасы Еңбек және халықты әлеуметтік қорғау министрлігінің Еңбекті
қорғау жөніндегі республикалық ғылыми-зерттеу институты» ШЖҚ РМК (ҚР
ЕХӘҚМ РҒЗИ), Астана, Қазақстан, e-mail: rniiot@ rniiot.kz;

Бекмағамбетов Ә.Б. - заң ғылымдарының кандидаты, қауымдастырылған
профессор, Қазақстан Республикасы Еңбек және халықты әлеуметтік қорғау
министрлігінің Еңбекті қорғау жөніндегі республикалық ғылыми-зерттеу
институты» ШЖҚ РМК (ҚР ЕХӘҚМ РҒЗИ), Астана, Қазақстан, e-mail:
Adilet1979@mail.ru;

Рахметова А.М. - медицина ғылымдарының кандидаты, доцент, биомониторинг
және еңбек гигиенасы бөлімінің басшысы, «Қазақстан Республикасы Еңбек
және халықты әлеуметтік қорғау министрлігінің Еңбекті қорғау жөніндегі
республикалық ғылыми-зерттеу институты» ШЖҚ РМК, Астана, Қазақстан,
e-mail: ra\_anar@mail.ru;

Құлмағамбетова Э.А. - химия ғылымдарының кандидаты, «Қазақстан
Республикасы Еңбек және халықты әлеуметтік қорғау министрлігінің Еңбекті
қорғау жөніндегі республикалық ғылыми-зерттеу институты» ШЖҚ РМК,
Астана, Қазақстан, e-mail:
elya\_kulmagambet@mail.ru;

Абжаппарова А.О. - экономика ғылымдарының магистрі, «Қазақстан
Республикасы Еңбек және халықты әлеуметтік қорғау министрлігінің Еңбекті
қорғау жөніндегі республикалық ғылыми-зерттеу институты» ШЖҚ РМК,
Астана, Қазақстан, e-mail:
ms.aybope@mail.ru;

Әбдрахманова Н.Б. - тіршілік қауіпсіздігі және қоршаған ортаны қорғау
ғылымдарының магистрі, «Қазақстан Республикасы Еңбек және халықты
әлеуметтік қорғау министрлігінің Еңбекті қорғау жөніндегі республикалық
ғылыми-зерттеу институты» ШЖҚ РМК, Астана, Қазақстан, e-mail:
nazgul122@mail.ru;

Сагиндикова Н.Т. - техника және технология ғылымдарының магистрі,
«Қазақстан Республикасы Еңбек және халықты әлеуметтік қорғау
министрлігінің Еңбекті қорғау жөніндегі республикалық ғылыми-зерттеу
институты» ШЖҚ РМК, Астана, Қазақстан, e-mail:
nursag79@mail.ru.

\emph{{\bfseries Information about the authors}}

Aktayeva L.M. - Doctor of Medical Sciences, The Republican
Scientific Research Institute of Occupational Safety and Healt, RNIIOT
of the Ministry of Health of the Republic of Kazakhstan Astana,
Kazakhstan, e-mail: rniiot@ rniiot.kz;

Bekmagambetov A.B. - higher law degree, candidate of legal sciences,
Associate Professor, RRIOSH of the Ministry of Health of the Republic of
Kazakhstan, Astana, Kazakhstan, e-mail:
Adilet1979@mail.ru;

Rakhmetova A.M. - Candidate of Medical Sciences, Associate Professor,
RRIOSH of the Ministry of Health of the Republic of Kazakhstan, Astana,
Kazakhstan, e-mail:
ra\_anar@mail.ru;

Kulmagambetova E. A. -Ph.D., RRIOSH of the Ministry of Health of the
Republic of Kazakhstan, Astana, Kazakhstan, e-mail:
elya\_kulmagambet@mail.ru;

Abzhapparova A. - m.sc. in economics, RRIOSH of the Ministry of Health
of the Republic of Kazakhstan, Astana, Kazakhstan, e-mail:
ms.aybope@mail.ru;

Abdrakhmanova N.B.- Master' s Degree in Life Safety and
Environmental Protection, the Research Institute of the Ministry of
Health of the Republic of Kazakhstan, Astana, Kazakhstan, e-mail:
nazgul122@mail.ru;

Sagindikova N. T. - Master of Engineering and Technology, the Research
Institute of the Ministry of Health of the Republic of Kazakhstan,
Astana, Kazakhstan, e-mail:
nursag79@mail.ru.\
