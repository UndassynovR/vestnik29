\id{ҒТАМР 52.47.19}{}

\begin{header}
\swa{}{ШТАНГАЛЫ СОРАПТАРМЕН ЖАБДЫҚТАЛҒАН ҰҢҒЫМАЛАРДЫҢ СОҢҒЫ ИГЕРУ САТЫСЫНДАҒЫ ЖҰМЫСЫН ДИАГНОСТИКАЛАУ ЖӘНЕ ТАЛДАУ ӘДІСТЕМЕСІ}

\tsp{1}М.Ж. Досжанов,
\tsp{2}Ю.А. Нифонтов,
\tsp{3}Ғ.Ж. Тасболат\envelope,
\tsp{1}Е.Т. Қодар,
\tsp{4}Н.С.Сүлейменов,
\tsp{1}Л.~Бисенбайқызы,
\tsp{1}А.М. Төлегенова
\end{header}

\begin{affil}
\tsp{1}Қызылорда «Болашақ» университеті, Қызылорда, Қазақстан,

\tsp{2}Санкт-Петербург мемлекеттік теңіз техникалық университеті, Санкт-Петербург, Ресей,

\tsp{3}Қ. Құлажанов атындағы Қазақ технология және бизнес университеті, Астана, Қазақстан,

\tsp{4}Қорқыт Ата атындағы Қызылорда университеті, Қызылорда, Қазақстан

\corrauthor{Корреспондент-автор: galymzhan\_zh@mail.ru}
\end{affil}

Ұңғымалық штангалы сорғы қондырғылары (ШСҚ) аз дебитті және жоғары сулы
қабаттарды игеру кезінде кеңінен қолданылатын, мұнай ұңғымаларының
сенімді және үнемді пайдалану жабдығы болып табылады. Штангалы сорғылар
көмегімен мұнай өндіру -- олардың қарапайымдылығы, тиімділігі және
сенімділігімен түсіндірілетін жасанды көтерудің ең кең таралған
әдістерінің бірі. Алайда ШСҚ пайдалану кезінде күрделі факторлар жиі
туындайды: сорғының қабылдауында жоғары газ құрамы, айдалатын
сұйықтықтағы механикалық қоспалардың жоғары мөлшері, сорғы-компрессорлік
құбырларда (СКҚ) парафиндер мен асфальт-шайыр заттарының шөгуі.

Бұл мақалада кен орындарын игерудің соңғы кезеңінде ШСҚ жұмысын кешенді
диагностикалау әдіснамасы ұсынылған. Сулылығы жоғары шартты ұңғымаларға
тән ақауларды анықтау үшін динамограммалар мен эхограммаларды талдауға
ерекше назар аударылған. Қондырғылардың жұмысын оңтайландыру бойынша
тәжірибелік ұсыныстар әзірленген, оларға газ бөлгіштерді қолдану және
қабаттың түпкі аймағына гидроимпульстік әсер ету кіреді. Қазақстан кен
орындарының жағдайына ұсынылған шешімдердің тиімділігін растайтын
өндірістік зерттеулердің нәтижелері келтірілген.

Әдіснама қабатқа импульстік әсер етудің толық циклын бақылауға және
мұнай өндіруді арттыру технологияларының тиімділігін талдауға мүмкіндік
береді. Зерттеу нәтижелері кен орындарын игерудің соңғы кезеңінде ШСҚ
пайдалану сенімділігін арттыруға және жабдықтың жөндеу аралық
мерзімдерін ұзарту бойынша тәжірибелік маңызға ие.

{\bfseries Түйін сөздер:}~штангалы сорғы қондырғылары, диагностика,
динамограмма, эхограмма, жоғары суланған ұңғымалар, газ құрамы, игерудің
соңғы кезеңі.

\begin{header}
МЕТОДИКА ДИАГНОСТИКИ И АНАЛИЗА РАБОТЫ СКВАЖИН, ОБОРУДОВАННЫХ ШТАНГОВЫМИ НАСОСАМИ, НА ПОЗДНЕЙ СТАДИИ РАЗРАБОТКИ

\tsp{1}М.Ж. Досжанов,
\tsp{2}Ю.А. Нифонтов,
\tsp{3}Г.Ж. Тасболат\envelope,
\tsp{1}Е.Т. Кодар,
\tsp{4}Н.С. Сулейменов,
\tsp{1}Л.~Бисенбайкызы,
\tsp{1}А.М. Толегенова
\end{header}

\begin{affil}
\tsp{1}Кызылординский университет «Болашак», Кызылорда, Казахстан,

\tsp{2}Санкт-Петербургский государственный морской технический университет, Санкт-Петербург, Россия,

\tsp{3}Казахский университет технологии и бизнеса имени К. Кулажанова, Астана, Казахстан,

\tsp{4}Кызылординский университет им. Коркыт Ата, Кызылорда, Казахстан,

е-mail: galymzhan\_zh@mail.ru
\end{affil}

Скважинные штанговые насосные установки (ШСНУ) являются надежным и
экономичным эксплуатационным оборудованием нефтяных скважин, широко
применяемым при разработке малодебитных и высокообводненных пластов.
Добыча нефти при помощи штанговых насосов -- один из самых
распространенных способов искусственного подъема, что объясняется их
простотой, эффективностью и надежностью. Однако при эксплуатации ШСНУ
часто возникают осложняющие факторы: высокое газосодержание на приеме
насоса, повышенное содержание механических примесей в откачиваемой
жидкости, отложение парафинов и асфальтосмолистых веществ в
насосно-компрессорных трубах (НКТ).

В данной статье предложена комплексная методика диагностики работы ШСНУ
на поздней стадии разработки месторождений. Особое внимание уделено
анализу динамограмм и эхограмм для идентификации специфических
неисправностей, характерных для условий высокообводненных скважин.
Разработаны практические рекомендации по оптимизации работы установок,
включая применение газосепараторов и гидроимпульсного воздействия на
призабойную зону пласта. Представлены результаты промысловых
исследований, подтверждающие эффективность предложенных решений для
условий казахстанских месторождений.

Методика позволяет проводить мониторинг полного цикла импульсных
воздействий на пласт и анализировать эффективность применяемых
технологий увеличения нефтеотдачи. Результаты исследований имеют
практическую значимость для повышения эксплуатационной надежности ШСНУ и
увеличения межремонтных периодов работы оборудования на завершающей
стадии разработки месторождений.

{\bfseries Ключевые слова:}~штанговые насосные установки, диагностика,
динамограмма, эхограмма, высокообводненные скважины, газосодержание,
поздняя стадия разработки.

\begin{header}
METODOLOGY FOR DIAGNOSING AND ANALYZING THE PERFORMANCE OF WELLS EGUIPPED WITH ROD PUMPS AT THE LATE STRAGE OF DEVELOPMENT

\tsp{1}M.Zh. Doszhanov,
\tsp{2}I.A. Nifontov,
\tsp{3}G.Zh. Tasbolat\envelope,
\tsp{1}E.T. Kodar,
\tsp{4}N.S. Suleimenov,
\tsp{1}L.~Bisenbaikyzy,
\tsp{1}A.M. Tolegenova
\end{header}

\begin{affil}
\tsp{1}Kyzylorda Bolashak University, Kyzylorda, Kazakhstan,

\tsp{2}Saint-Petersburg state marine technical University, Saint-Petersburg, Russia,

\tsp{3}K. Kulazhanov Kazakh University of Technology and Business, Astana, Kazakhstan,

\tsp{4}Korkyt Ata Kyzylorda University, Kyzylorda, Kazakhstan,

е-mail: galymzhan\_zh@mail.ru
\end{affil}

Sucker rod pump installations (SRPIs) are reliable and cost-effective
production equipment for oil wells, widely used in the development of
low-rate and highly water-cut reservoirs. Oil production using sucker
rod pumps is one of the most common artificial lift methods, which is
explained by their simplicity, efficiency, and reliability. However,
during the operation of SRPIs, complicating factors often arise: high
gas content at the pump intake, increased content of mechanical
impurities in the produced fluid, deposition of paraffins and
asphaltene-resinous substances in the tubing strings.

This article proposes a comprehensive methodology for diagnosing the
operation of SRPIs at the late stage of field development. Special
attention is paid to the analysis of dynamometer cards and echograms to
identify specific malfunctions characteristic of highly water-cut wells.
Practical recommendations have been developed for optimizing the
operation of installations, including the use of gas separators and
hydraulic pulse impact on the near-wellbore zone of the reservoir. The
results of field studies confirming the effectiveness of the proposed
solutions for the conditions of Kazakhstani fields are presented.

The methodology allows monitoring the full cycle of impulse impacts on
the reservoir and analyzing the effectiveness of the applied
technologies for enhanced oil recovery. The research results are of
practical importance for improving the operational reliability of SRPIs
and increasing the inter-repair periods of equipment operation at the
final stage of field development.

{\bfseries Keywords:} sucker rod pump installations, diagnostics,
dynamometer card, echogram, highly water-cut wells, gas content, late
stage of development.

\begin{multicols}{2}
{\bfseries Кіріспе.} Қазіргі уақытта Қазақстандағы ірі мұнай кен
орындарының едәуір бөлігі игерудің соңғы сатысында орналасқан, бұл кезең
ұңғымалар өнімі құрамындағы судың жоғары мөлшерімен сипатталады. Ұңғыма
өнімінің сулануы штанг колоннасында коррозиялық үдерістердің дамуына
ықпал етеді, ал бұл процестер штангалы ұңғымалық сорап қондырғыларын
пайдалану барысында туындайтын динамикалық ауыспалы жүктемелермен
бірігіп, металл құрылымына деструктивті әсерін күшейтеді. ШСҚ қолдану
шарттары мен оларды пайдалану тәжірибесі көрсеткендей, металл
штангаларда шаршау үдерістерінің (динамикалық жүктемелер мен коррозияның
бірлескен әсері) дамуына түрлі факторлар ықпал етеді, нәтижесінде
тереңдік сорап жабдықтарының бұзылуы мен істен шығуына алып келеді.
Мұндай факторларға ұңғыма өнімінің сулануы, ілеспе судың минералдану
деңгейі, штанг колоннасының материалы, ұңғыма оқпанының еңіс бұрышы
(үйкеліс пен коррозияның жиынтық әсері), сулануға бейімділік (ұңғыма
үзіліспен жұмыс істеген жағдайда, сораптағы ағып кетулер немесе төгу
клапанының болу), штангалардың жаңалығы және сорап жетегінің жұмыс
режимі жатады {[}1,2{]}.

Бұл зерттеу мұнай кен орындарын игерудің соңғы сатысында штангалы сорғы
қондырғыларының (ШСҚ) жұмысын кешенді диагностикалау әдістемесін
әзірлеуге арналған. Тақырыптың өзектілігі соңғы кезеңдегі игеруде жиі
кездесетін жоғары сулану, газ факторының өсуі және қабат энергиясының
төмендеуі сияқты қиындықтармен байланысты. Бұл факторлар ШСҚ жұмысына
кері әсер етіп, өнімділікті төмендетеді және жабдықтың тозуын тездетеді.

Зерттеудің мақсаты - соңғы кезеңдегі ШСҚ жұмысын диагностикалауға және
оңтайландыруға бағытталған кешенді әдістеме әзірлеу.

Зерттеу міндеттері: соңғы кезеңдегі негізгі техникалық қиындықтарды
талдау (сулану, газ факторы, тозу), динамограмма мен эхограмма
деректерін жинақтау және жүйелеу, газ бөлгіштерді қолдану тиімділігін
бағалау, гидроимпульстік әсердің суланған қабаттарға әсерін зерттеу,
алынған нәтижелер негізінде практикалық ұсыныстар әзірлеу.

Зерттеудің ғылыми жаңалығы динамограмма мен эхограммаларды біріктіретін
интеграцияланған диагностикалық тәсілдің әзірленуінде жатыр.

Қазақстан үшін маңыздылығы - елімізде ірі мұнай-газ кен орындарының бірі
Оңтүстік Торғай ойпаты ағымдағы кезеңде игерудің соңғы сатысында тұр,
бұл жетілдірілген диагностика әдістерін қолдануды талап етеді.
\end{multicols}

\fig[0.85\textwidth]{g2/image53}[1-сурет. Оңтүстік Торғай ойпаты кен орнындағы ұңғымалық сорғы жүйесі істен шығуының салыстырмалы көрсеткіші]

\begin{multicols}{2}
Зерттеу нәтижелері ШСҚ жұмысын үздіксіз бақылау жүйесін құруға және
жабдық ақауларын алдын алуға мүмкіндк береді. Бұл әсіресе қазіргі
экономикалық жағдайда операторларға айтарлықтай үнемдеуге мүмкіндік
береді.

Соңғы кезеңдегі игеруде ШСҚ диагностикасының тиімді әдістемесін әзірлеу
мұнай өндіру тиімділігін арттырудың маңызды факторы болып табылады.
Сондай-ақ жөндеу аралық кезеңін ұзарту және жабдықтың істен шығуын
азайту өндірістік тұрғыдан ғана емес, ғылыми тұрғыдан да маңызды
міндетке айналады. Жоғарыда аталған кен орнын зерттеу негізінде бірнеше
ұңғымалардағы ШСҚ және ЭОСҚ (электр ортадан тепкіш сорғы қондырғы)
сорғылар жүйесінің істен шығу себептеріне салыстырмалы талдау жүргізілді
(1-сурет).

Талдау жүйесі жабдық сенімділігін арттыруға негізделген диагностика
әдістерін жетілдіруге және ұңғымаларды игерудің соңғы кезеңінде тиімді
басқару шешімдерін ұсынуға мүмкіндік береді.

Мұнай мен газ өндіруді оңтайландыру мақсатында ұңғымалардың жұмысын
бақылау және талдау үшін динамограммалар қолданылады. Олар ақауларды
уақытылы анықтап, жою бойынша белгіленген іс-шараларды дер кезінде
жүзеге асыруға мүмкіндік береді. Мұндай ақауларға жеткілікті көңіл
бөлінбеген жағдайда, олар айтарлықтай экономикалық шығындарға алып келуі
мүмкін {[}3,4{]}.

ШСҚ сорғылары келесі жағдайларда жарамсыз деп танылады:

- плунжер (бөлінбейтін сораптар үшін) ұзындығы кемінде 1200 мм болатын
сорап-

компрессорлық құбырлармен (СКҚ) қосылған цилиндрге өтпеген жағдайда;

- плунжердің төлқұжатта көрсетілген нөмері мен өлшемі нақты мәндерге
сәйкес

келмеген жағдайда; егер нөмері сәйкес сәйкес келмесе, бірақ өлшемі
сәйкес келсе, пайдалану төлқұжатына нақты деректер енгізіледі;

- хром жабынының тұтастығы бұзылған жағдайда (қабыршақтану, сызаттар,
жарықтар

және т.б.);

- сорап құрамында кем дегенде бір пайдаланылған бөлшек анықталған
жағдайда;

- хромдау процесінен кейін цилиндр мен плунжер беттерінде дөрекі өңдеу
іздері

байқалған жағдайда.

Ұңғымаларда жүргізілетін жұмыстардың сипаты көбінесе олардың пайдалану
тәсіліне, ұңғымаға түсірілген жер асты жабдығының түріне, сондай-ақ жер
асты және жер үсті жабдықтарының жұмыс істеу жағдайына байланысты.
Ұңғымаларды пайдалану тәсілдері мен қолданылатын жабдықтар туралы
білімсіз маман тек жұмысты орындай алмай қалады, сонымен қатар мұндай
жұмыстар апатқа немесе жазатайым оқиғаға алып келуі мүмкін.

Аталған факторлар жер қойнауынан мұнайды неғұрлым толық өндіруді
қамтамасыз етуге бағытталған кен орындарын игеру кезінде туындайтын ең
өзекті ғылыми-техникалық мәселелердің бірі болып табылады. Бұл мәселені
шешуді күрделендіретін жағдай -- мұнай кен орындарының басым
көпшілігінің игеру жұмыстарының кеш сатысында өтуі. Мұндай жағдайда
ауқымды материалдық-техникалық және еңбек ресурстарын тарту қажеттілігі
туындайды. Қазіргі уақытта көптеген кен орындарында мұнай өндіруді
тұрақтандыру міндеті қойылып отыр. Өнімді қабаттар, негізінен, жоғары
аймақтық және қабаттық біртексіздігімен, бастапқы мұнаймен қанығуы мен
өткізгіштігінің төмендігімен, сондай-ақ кең көлемді су-мұнай
аймақтарының болуымен сипатталады (2-сурет) {[}5,6{]}.

Бұл зерттеу жұмысының әдіснамасы жүйелік талдау қағидаттарына
негізделген. Зерттеу барысында ұңғыма жұмысының гидродинамикалық
процестері кешенді түрде қарастырылып, динамограммалар мен эхограммалар
талданып, олардың негізінде сорап қондырғысының тиімділігін арттыруға
арналған инженерлік ұсыныстар әзірленді.
\end{multicols}

\fig{g2/image54}[2-сурет. Ұңғымалардың өнімділігі және сұйықтықты көтеру биіктігі бойнша шекаралары]

\begin{multicols}{2}
{\bfseries Әдеби шолу.} Ұңғымалық штангалы сорап қондырғылары қарапайым
құрылымымен, сенімділігі және қызмет көрсету тиімділігімен ерекшеленіп,
мұнай өндірудің соңғы кезеңдерінде кеңінен қолданылады.3-суретте арқан
штангаларымен жабдықталған ұңғымалық сорап қондырғысының типтік схемасы
көрсетілген. Қондырғы негізінен жер бетіндегі жетек, штангалар тізбегі
және плунжерлі сораптан тұрады. Жетек қозғалысы штангалар арқылы ұңғыма
түбіне беріледі де, плунжердің ілгерілемелі қозғалысын қамтамасыз етеді.
Бұл жүйе мұнай қабатынан сұйықты жер бетіне шығару процесін жүзеге
асырады.

Қазіргі таңда кен орындарын игеру объектілері бойынша мұнай қайтарымын
арттырудың үшінші реттік әдістерін қолдануға қатысты ауқымды зерттеулер
жинақталған. Алайда мұнай алуды арттыруға бағытталған бірқатар
технологияларды қолдану жиі тиімсіз болып жатады, бұл көбінесе олардың
нақты геологиялық-өндірістік жағдайларға жеткілікті бейімделмеуіне
байланысты. Аталған әдістерді ұтымды пайдалану қажеттілігінің
туаындауына байланысты, мұнай қорларын геологиялық-технологиялық
тұрғыдан сәйкестендіру, құрылымдастыру жұмыстарын жүргізу және қалдық
мұнай қорларын тиімді өндіру жүйелерін жетілдіру міндеті алға қойылды
{[}7{]}.

Аталған технологияларды қолдану әрбір нақты жағдайда мұнайды барынша
өндіру және экономикалық тұрғыдан тиімділік қағидаларына негізделуі
тиіс. Осыған орай, әрбір технологияны енгізбестен бұрын, кен орнын
игерудің жан-жақты геологиялық-технологиялық талдауы, геологиялық және
гидродинамикалық модельдеу, сондай-ақ нақты геологиялық-физикалық
жағдайларға сәйкес технологияны қолданудың көп критерийлі бағалауы
жүргізілуі қажет.

Мұнай қорларын ұтымды өндіру үшін үңғыманың түп маңы аймағына тиімді
әсер ету әдістерін әзірлеу және қолдану -- қазіргі таңда маңызды
міндеттердің бірі болып отыр. Бұл бағыттағы зерттеулер қалдық қорларды
өндірудің тиімділігін арттырып, кен орындарын игерудің соңғы сатыларында
мұнай бергіштікті жақсартуға мүмкіншілік береді.

Соңғы онжылдықта ШСҚ диагностикасы саласында цифрлық технологиялардың
енуі айтарлықтай жетістіктерге қол жеткізді. Заманауи динамографтар мен
эхолокациялық жүйелер нақты уақыт режимінде деректерді жинақтауға және
өңдеуге мүмкіндік береді. Сонымен қатар, арнайы бағдарламалық
жасақтамалар динамограммаларды автоматты түрде талдауға және ақауларды
диагностикалауға мүмкіндік береді. Мысалы, искусственный интеллект
негізіндегі жүйелер ШСҚ жұмысының аномалияларын алдын ала анықтай алады
{[}8, 9{]}.
\end{multicols}

\fig[0.5\textwidth]{g/image7}[3-сурет. Арқан штангаларымен жабдықталған үңғымалық сорап қондырғысының типтік схемасы:\\\normalfont{\emph{1-тербелмелі станок; 2-сағалық жабдық; 3-жылтырлатылған шток; 4-сағалық сальник; 5-ұңғымалық сорғы; 6-арқан; 7-арқанның жоғарғы бітеуі; 8-арқанның төменгі бітеуі; 9-сорғы штангалары.}}]

\begin{multicols}{2}
Халықаралық тәжірибеде ШСҚ мониторингінің жаңа әдістері қолданыс табуда.
АҚШ және Канадада бұл саладағы зерттеулер негізінен предиктивтік
аналитикаға бағытталған. Осы мақсатта машиналық үйрену әдістері,
сенсорлық желілер және бульттық деректерді талдау қолданылады. Осы
технологиялардың тиімділігі бірқатар ірі мұнай компанияларымен расталған
{[}10{]}.

Қазақстандық зерттеулер негізінен жергілікті геологиялық-өндірістік
ерекшеліктерге бағытталған. Соңғы жылдары гидродинамикалық модельдеудің
жаңа әдістері әзірленді, олар ШСҚ жұмысын оңтайландыруға мүмкіндік
береді. Дегенмен, әлі күнге дейін заманауи халықаралық тәжірибені
желіктеу қажеттілігі сақталуда.

Осылайша, әдебиеттерді талдау көрсеткендей, ШСҚ диагностикасы
саласындағы заманауи зерттеулер негізінен цифрлық технологияларды
енгізуге бағытталған. Бұл бағыттағы жұмыстарды одан әрі жалғастыру,
әсіресе Қазақстан жағдайына бейімделген әдістемелер әзірлеу өзекті
мәселе болып табылады.

{\bfseries Материалдар мен әдістер.} Бұл зерттеуде штангалы сорап
қондырғыларының жұмысын диагностикалау үшін стандартты әдістер кешені
қолданылды. Әдістемелік тұрғыда олар жеке аналитикалық блоктарға
бөлініп, әрқайсысы нақты кезеңдер мен міндеттерді қамтыды.

\emph{Бақылау кезеңі.}

Динамометрлік тексеру - плунжер жүрісінің күш-жүктеме диаграммаларын алу
үшін жер бетіндегі жетек пен штангаларға орнатылған динамометрлер
қолданылды.

Эхолокацияны өлшеу - плунжер қозғалысының нақты ұзындығы мен сорап
цилиндрінің толу дәрежесін бағалау үшін эхолокациялық құрылғылар
пайдаланылды.

\emph{Талдау кезеңі.}

Алынған динамограммалар визуалды және сандық әдістермен талданды:
қалыпты режим, газ әсері, сулану және механикалық ақаулардың белгілері
сараланды.

Эхограммалар арқылы плунжер қозғалысының тұрақтылығы, жүріс
ұзындығындағы өзгерістер және цилиндр толу коэффициентінің динамикасы
анықталды.

\emph{Интерпретация кезеңі.}

Динамометрлік және эхолокациялық деректер салыстырылып, газ құрамының
артуы мен суланудың сорап тиімділігіне ықпалы есептелді.

Қолданылған әдістердің нәтижелері жүйелік талдау негізінде біріктіріліп,
сорап қондырғыларының жұмысындағы әлсіз тұстар анықталды.

Зерттеу негізінен нақты өндірістік жағдайларда алынған деректерге
сүйенді. Эмпирикалық материалды жинау үшін ай сайынғы динамограммалар,
ұңғымадағы сұйық деңгейін өлшеу нәтижелері, сондай-ақ газ факторы
бойынша сынамалар пайдаланылды. Бұл деректер нәтижелердің сенімділігін
қамтамасыз етті.

Ұңғымадағы гидроимпульстік қондырғының жұмысын зерттеу және бақылау,
сондай-ақ ақаулардың себептерін анықтау үшін штангалы тереңдік
сораптармен жабдықталған ұңғымаларды зерттеуге арналған стандартты
жабдық қолданылады. Себебі, технологиялық шектеулерге байланысты мұндай
ұңғымаларда жалпы өндірістік өлшеу жабдықтарын пайдалану мүмкін емес
{[}11{]}.

Гидробұрылыс шамасын есептеу үшін тұрақталған жұмыс режиміне сәйкес
келетін түптік қысымды тікелей немесе жанама әдістермен анықтау қажет.

Түптік қысымды тікелей өлшеу үшін диаметрі 22-25 мм болатын шағын
габаритті ұңғымалық манометрлер қолданылады. Мұндай аспаптар штангалы
сорапты қондырғының ұңғыма сағасындағы сорап-компрессорлық құбырларды
эксцентрикалық ілу жағдайында планшайбадағы тесіктер арқылы болат сыммен
аралық кеңістікке түсіріледі. Осы тәсілмен алынған түптік қысым
мәліметтері ең сенімді болып табылады. Алайда, терең және иілген
ұңғымаларда немесе құбырлар арасындағы саңылау тым тар болған жағдайда
манометрдің қысылып қалуы немесе сымның үзілуі орын алуы мүмкін. Мұндай
жағдайлардың алдын алу мақсатында гидроимпульстік қондырғының қабылдау
патрубкасына бекітіліп, сорап-компрессорлық құбырлармен (СКҚ) бірге
ұңғымаға түсірілетін, лифтілік ұңғымалық манометрлер деп аталатын арнайы
аспаптар қолданылады. Бұл әдіс зерттеу нәтижелерінің жеткілікті
сенімділігін қамтамасыз етеді, дегенмен лифтілік манометрді ұңғымаға
түсіру және оны қайта шығару үшін көтеріп-түсіру операцияларын орындау
қажеттілігімен байланысты. Осы себепті мұндай өлшеулер әдетте, ұңғымада
кезекті жөндеу жұмыстары немес қондырғыны ауыстыру барысында жүргізіледі
{[}12{]}.

Нәтижелердің жоғары дәлдігіне қарамастан, бұл әдістердің еңбек
сыйымдылығы мен технологиялық тиімсіздігіне байланысты олар кеңінен
қолданыс таппаған. Осыған орай, түптік қысымды анықтаудың анағұрлым
технологиялық тұрғыдан тиімді жолы - аралық кеңістіктегі сұйықтықтың
динамикалық деңгейінің тереңдігін эхолоттар немесе толқын өлшегіштер
арқылы өлшеу негізінде жүзеге асырылатын жанама әдістері болып табылады.

Аспаптың жұмыс принципі келесідей, аралық кеңістікке дыбыстық импульс
жіберіледі, ол сұйықтық деңгейінен шығылысып, ұңғыма сағасына қайта
оралады және қабылдаған сигналды тіркейтін құрылғымен байланысқан
микрофон арқылы қабылданады. Бұл микрофон күшейткіш арқылы сигналдарды
тіркеуші аспаппен қосылады. Динамикалық деңгейдің тереңдігі -
диаграммада бастапқы импульс пен сұйықтық деңгейінен шағылған сигналға
сәйкес келетін екі шыңның арақашықтығын өлшеу арқылы анықталады.
Дыбыстық сигнал ұңғыма сағасынан сұйықтық деңгейіне дейінгі аралықты екі
рет (бару және қайту) жүріп өтетіндіктен және газды ортада дыбыстың
таралу жылдамдығы белгілі болған жағдайда, сұйық деңгейінің тереңдігі
келесі қатынас бойынша есептеледі {[}13{]}:

\begin{equation}
S = \vartheta \cdot \frac{t}{2}
\end{equation}

мұнда:

\(S\) - сұйықтық деңгейінің тереңдігі;

\(t\) - импульс берілген сәттен бастап шағылған сигналдың қайтып келу
уақыты, бұл уақыт ішінде сигнал \(2S\) жол жүреді;

\(\vartheta\) - аралық кеңістіктегі газ ортасында дыбыстың таралу
жылдамдығы.

Мұндай әдіспен сұйық деңгейін анықтаудың бірқатар кемшіліктері бар.

Аралық кеңістіктегі дыбыс жылдамдығы \(\vartheta\) газдың қысымына,
температурасына және тығыздығына тәуелді. \(\vartheta\) мәнін
анықтаудағы қателік сұйық деңгейінің тереңдігін \(S\) анықтаудағы
нәтижеге тікелей әсер етеді.

Бір ұңғымада сұйықтықты өндірудің әртүрлі режимдеріне сәйкес бірнеше
\(S\) мәні өлшеніп, олардың орташа шамалары есептелген жағдайда,
қателіктер азаяды, өйткені \(\vartheta\) мәніндегі жүйелі қате барлық
өлшенген \(S\) шамаларына бірдей әсер етеді.

Ұңғымааралық кеңістікте көбіктенген сұйықтықтың болуы деңгейден шағылған
сигналдың анық қабылдануын қиындатады және бұл эхолот арқылы өлшеу
әдісіне тән жалпы кемшілік болып табылады. Сондықтан өлшеу жүргізер
алдында аралық кеңістіктен газды шығару жұмыстарын жүргізбеу аса
маңызды, өйткені бұл сұйықтықтың көбіктенуіне алып келуі мүмкін. Алайда
бұл талапты әрдайым сақтау мүмкін емес, себебі кейбір хлопушка түріндегі
сағалық жабдықтарда арнайы бұрандалы тығынмен жабылатын ұңғыма сағасы
плитасындағы тесіктер арқылы газ шығатын арналар қарастырылған.
Сондай-ақ, сұйықтық алу режиміне сәйкес келетін түптік қысымды деңгей
арқылы анықтау үшін, сұйық деңгейі мен ұңғыма түбі арасындағы сұйық
бағанының орташа тығыздығын білу қажет екенін атап өткен жөн. Алайда бұл
тығыздық сұйықтықтың сулану дәрежесіне және газ құрамына тәуелді
болғандықтан, оны дәл анықтау едеәуір қиындық туғызады {[}14{]}.

Қазіргі заманғы жоғары смезімтал эхолоттар СКҚ колоннасының әрбір
муфтасынан шағылған сигналдарды тіркей алады. Мұндай жағдайда сұйықтық
деңгейінің тереңдігі эхограммада сұйық деңгейіне сәйкес келетін сигналға
дейінгі пик саны арқылы анықталады және бұл сан бір құбырдың ұзындығына
көбейтіледі. Аспап муфталардың үстіңгі бөліктерінен шағылған сигналдарды
электрлік сүзгілер көмегімен бөлуге, үлкен тереңдікте орналасқан
муфталардан шағылған сигналдарды айқындауға, сондай-ақ үлкен тереңдіктер
жағдайында сұйықтық деңгейінен шағылған негізгі сиганлды анықтауға
мүмкіндік береді.

Хлопушкада немес оның бүйірлік тармағында шағылған сигналдарды
қабылдауға арналған кварцты сезімтал микрофон орнатылады. Эхолоттардың
кейбір конструкцияларында микрофонның орнына термофондар қолданылады.
Микрофон дыбыстық сигналдарды электрлік сигналдарға түрлендіріп, оларды
күшейткішке жібереді. Күшейткіште өлшенетін сигналды бөліп көрсету және
кедергілерді басу үшін үш арналы сүзгі жүйесі қарастырылған. Сонымен
қатар, күшейткіште сезімталдықты реттейтін құрылғы мен сигналды жазуға
арналған цифрлық блок бар.

Хлопушка патрубогы аралық кеңістіктегі ысырманың фланецңне газды
шығарусыз қосылады және 2,5 МПа-ға дейінгі қысымда өлшеулер жүргізуге
мүмкіндік береді.

Сондай-ақ, сұйықтықты алу режиміне сәйкес келетін түптік қысымды деңгей
бойынша анықтау үшін, сұйық деңгейі мен ұңғыма түбі арасындағы сұйық
бағанының орташа тығыздығын білу қажет екенін атап өткен жөн. Бұл
тығыздық сұйықтықтың сулану дәрежесіне және газ құрамына тәуелді
болғандықтан, оны дәл анықтау едәуір қиындық туғызады {[}15,16{]}.

Сипатталған кемшіліктерге қарамастан, заманауи технологиялар
эхолокациялық өлшеулердің сенімділігін едәуір арттырды. Жоғары сезімтал
эхолоттар күрделі сандық сүзгілерді және сигналдарды жинақтау
алгоритмдерін қолданады, бұл сұйықтық деңгейінен алынған әлсіз пайдалы
сигналды СКҚ муфталарынан немесе көбіктен туындаған күшті кедергілер
аясынан бөліп алуға мүмкіндік береді. Негізгі жүйелі қателікті - дыбыс
жылдамдығын (ϑ) дәл анықтамаудан туындайтын ауытқуды - азайту үшін озық
әдістер белгілі тереңдіктегі шағылдырғышқа (мысалы, муфта немесе пакер)
қатысты аспапты калибрлеуді ұсынады. Бұл нақты ұңғы жағдайларына сәйкес
дыбыс жылдамдығын есептеуге және соңғы нәтиженің дәлдігін айтарлықтай
арттыруға мүмкіндік береді.

Маңыздысы - эхолот деректерін жеке-дара қарастыруға болмайды. Ең дәл
нәтижелер кешенді интерпретация кезінде алынады, яғни аспап
көрсеткіштері ұңғы жұмысына қатысты басқа деректермен:
динамограммалармен, дебитпен, сулану деңгейімен және газ факторы
көрсеткіштерімен салыстырылады. Бұдан бөлек, аса маңызды ұңғыларда
немесе эхолот деректерін мерзімді тексеру үшін эталондық өлшеу әдісі
қолданылады, ол үшін түпке тереңдік манометрлері түсіріледі. Бұл әдіс
сұйық бағанының деңгейі мен тығыздығы арқылы есептелуге байланысты
қателіктерді айналып өтіп, түптік қысымды тікелей өлшеуге мүмкіндік
береді.

{\bfseries Нәтижелер және талқылау.} Бұл әдіс қондырғы жұмысының
эхограммасын тіркеуге мүмкіндік береді және ол пластқа берілетін
импульстік әсерлердің толық циклін бақылау мен талдау үшін де
қолданылады. Қабатқа әсер ету цикліне имплозиялық және гидроимпульстік
соққы түріндегі импульстардың белгілі бір тізбегі кіреді. Эхолотпен
сигнал жазу уақыты 32 секундпен шектелген, бұл станок-тербелгіштің
жылдамдығы 5 тербеліс/мин болған жағдайда қондырғы жұмысының толық
циклін жазуға толық жеткілікті.

Қондырғы жұмысының бақылау динамометрия әдісі арқылы жүзеге асырылады.
Динамограмма штангаларды орын ауысуына байланысты жылтыр штоктағы
жүктеменің өзгеруін көрсетеді және осы мәліметтерді математикалық өңдеу
нәтижесінде плунжердің нақты орын ауысуы анықталады (4-сурет).
Динамограммаларды тіркеу үшін күшті өлшейтін тіркеуші аспаптар --
динамометрлер қолданылады. Қолдануға ыңғайлы болу үшін олар эхолоттармен
біріктірілген түрде шығарылады.
\end{multicols}

\fig{g/image8}[4-сурет. ШСҚ қалыпты жұмысы бойынша динамограмма көрсеткіші]

\begin{multicols}{2}
Динамограмма арқылы анықталатын ШСҚ-дың ақаулықтары: қабылдау
клапанындағы ағып кету; айдау клапанындағы ағып кету; газдың әсері;
газдың әсерінен берілудің үзілуі; плунжердің тығыз орналасуы немесе
қысыла қозғалуы; плунжердің төмен орналасуы; плунжердің жоғары
орналасуы; плунжердің төмен жүріс соңында қысылы қозғалысы; плунжер
жұбының тозуы; плунжердің цилиндрен толық шығуы; айдау клапанының
жабылуының кешеуілдеуі; қабылдау клапанының жабылуының кешеуілдеуі;
қабылдау және айдау клапан -

дарының жабылуының бір мезгілде кешеуілдеуі; қабылдау және айдау
клапандарының бітеліп қалуы; құбырлардағы ағып кету; қабылдау
клапанындағы ағып кетулер; айдау клапаны ашық, бірақ сұйықтық
берілмейді; айдау клапанындағы ағып кету салдарынан берілудің үзілуі;

штангалардың плунжердің немесе айдау клапанының бұралып босап кетуі;
плунжердің тығыз орналасуы.

Көп жағдайда сорғы жұмысына газдың әсері нәтижесінде ақаулықтар
туындайды, бұл цилиндрдің толу коэффициентінің төмендеуіне алып келеді
(5-сурет). Сорғы штангалы колоннасы арқылы көтерілетін сұйықтықтың
құрамындағы бос газдың болуы скважина өнімділігі мен дебитінің
төмендеуіне себеп болуы мүмкін, өйткені газ сорғы цилиндріндегі сұйықтық
алуы тиіс көлемді толықтай немесе ішінара алып қояды.

Еркін газдың әсері кезінде динамограмма пішіні көбінесе, сорғы берілісі
ұңғымадағы өнім ағынынан асып түсіп, сорғы қабылдауындағы қысым күрт
төмендеген жағдайдағы динамограммаға ұқсас болады. Мұндай жағдайда
ұңғыманы қысқа мерзімді тоқтатқаннан кейін қатарынан тіркелетін
динамограммалардың пішініне мұқият талдау жасалады. Егер сорғының беру
мөлшері ұңғыманың өнім ағысынан артық болса, онда ұңғыманы тоқтатқаннан
кейінгі алғашқы динамограмма (6-сурет, динамограмма 1) сорғы цилиндрінің
сору жүрісінде толық толуын көрсетеді. Ал кейіннен тіркелетін
динамограммалар 2,3 және 4 біртіндеп еркін газдың әсеріне тән пішінді
қабылдай бастайды. Пунктир сызықпен жоғарыда көрсетілген жағдайларға
сәйкес келетін теориялық динамограмма белгіленген.
\end{multicols}
\vspace{2em}
\begin{figs}[5-сурет. Бос газдың әсері жағдайындағы динамограммалар\\\normalfont{\emph{а - төмен қысымда газ қарқынды түрде бөлініп шығып, цилиндр көлемінің бір бөлігін алып қояды. Бұл жағдайда динамограмма «ішке қарай майысқан» және жұмыс аймағының көлемі азайған - бұл цилиндрдің толық толмауын және сұйықтық берілісінің төмендеуін көрсетеді; б - жоғары қысымда газ еріген күйде қалып, аздап ғана бөлінеді, сондықтан динамограмма формасы салыстырмалы түрде тұрақты, бірақ әлі де толымсыздық байқалады. Плунжер жүрісінің соңғы кезеңінде сұйықтық орнына газ толып, сорғы жұмысына кедергі келтіреді.}}]
  \fig[0.45\textwidth]{g/image9}[а)]
  \fig[0.45\textwidth]{g/image10}[б)]
\end{figs}

Мұндай мәселелерді шешудің бір жолы - газ сепараторларын қолдану болып
табылады, бұл ұңғыма өнімділігін арттырып, сорғының бұзылу қаупін
азайтады.

\fig{g/image11}[6-сурет. Сорғы берілісі ұңғыма өнімі ағыныан асып түсетін жағдайдағы типтік динамограммалар\\\normalfont{\emph{1 - ұңғыманы тоқтатқаннан кейін сорудың жұмыс жүрісінде цилиндрдің қалыпты толуы. Ұңғыманы қысқа мерзімді тоқтатқаннан кейінгі бірінші сорап айналымы. Бұл кезде түптен сұйықтық толық көлемде келіп, цилиндр толық толады. Динамограммада бұл қалыпты теориялыққа жақын пішінмен сипатталады; 2,3,4 - бірінші динамограмма алынғаннан кейін белгілі уақыт аралығында тіркелген динамограммалар. Келесі айналымдарда сұйықтық көлемі азайып, оның орынна еркін газ көтеріле бастайды. Цилиндр іші ішінара сұйықтықпенғ ішінара газбен толады.2→3 динамограмма пішіні біртіндеп тарылып, ішке қарай майысады, толымсыздық көрсетеді.3→4 бұл газлифттік эффект немесе сору кезінде газдың көлем алып қоюы салдарынан пайда болатын құбылыс.}}]

\begin{multicols}{2}
Сонымен қатар, эхограммалар мен динамограммаларды кешенді талдау тек
қана типтік ақауларды анықтап қана қоймай, қолданылып жатқан
технологиялық шешімдердің тиімділігін де бағалауға мүмкіндік береді.
Мысалы, газ бөлгіштерді қолдану немесе станок-качалканың жұмыс режимін
өзгерту динамограммада айқын өзгерістерге әкеледі: цилиндрдің толу
коэффициенті артады, қисықтың формасы бірқалыпты болып, теориялық
қисыққа жақындайды. Бұл еркін газдың әсерінің азайғанын және сұйықтық
алудың тиімділігінің артқанын көрсетеді. Осылайша, қондырғы жұмысына
мониторинг жүргізу әдістері ұңғыманы пайдалану технологиялық режимін
оңтайландыруға және жүргізілген шаралардың нәтижелілігін бағалауға құнды
ақпарат береді.

Ұсынылған әдістеменің тиімділігі өндірістік сынақтар кезінде расталды.
Газ сепараторларын қолдану цилиндр толу коэффициентін 0,75-тен 0,92-ге
дейін арттыруға мүмкіндік берді, бұл өнімділікті 18-22\%-ға арттырды.
Гидроимпульстік әсерлеу тәсілі қабаттың өткізгіштігін жақсартып,
сұйықтық алу деңгейін 15\%-ға дейін арттырды. Алынған нәтижелер ШСҚ
диагностикасында интеграцияланған тәсілдің тиімділігін растайды.

{\bfseries Қорытынды.} Қазіргі уақытта штангалы сорғы қондырғыларын (ШСҚ)
пайдалану мұнай өндірудің басқа әдістерімен салыстырғанда әлемдік
тәжірибеде ұңғымалар саны бойынша бірінші орында тұрады. Мысалы, АҚШ-та
ұңғымалардың 86\%-ы, ал Батыс Еуропада 90\%-ы ШСҚ-мен жабдықталған.
Дегенмен, практикада ШСҚ-ны пайдалану кезінде ұңғыма дебиті мен
өнімділігінің төмендеуіне, жабдық элементтерінің тозуының артуына және
экономикалық тиімділіктің азаюына әкелетін күтпеген жағдайларды
болдырмай қалу мүмкін емес.

Жүргізілген зерттеу нәтижелері ШСҚ жұмысын диагностикалауда динамограмма
мен эхограммаларды біріктіретін кешенді тәсілдің тиімділігін растады.
Газ сепараторларын қолдану цилиндр толу коэффициентін 15-20\% арттырып,
ал гидроимпульстік әсерлеу сұйықтық алу деңгейін 15\%-ға дейін
жоғарылатты. Динамограммаларды талдау арқылы 30-дан астам түрлі
ақауларды, соның ішінде клапандардың ағып кетуін, газ әсерін және
механикалық бұзылуларды анықтауға болатындығы дәлелденді.

Зерттеу барысында әзірленген әдістемелерді өндіріске енгізу ШСҚ-ның
жөндеу аралық мерзімін 25-30\%-ға ұзартуға, апаттық тоқтауларды азайтуға
және өндірістік шығындарды төмендетуге мүмкіндік береді. Алынған
нәтижелер мұнай өндірудің соңғы кезеңінде ШСҚ-ны тиімді пайдаланудың
жаңа мүмкіндіктерін ашады және оларды одан әрі жетілдіру бағыттарын
анықтайды.
\end{multicols}

\begin{center}
{\bfseries Әдебиеттер}
\end{center}

\begin{refs}
1. Ананьев В.П., Потапов А.Д., Филькин Н.А. Специальная инженерная
геология: Учебник / М.: Инфра-М. - 2016. -С.111-125. ISBN
978-5-16-010407-2.

2. Galeev A.S., ArslanovR.I., Suleymanov R.N., Filimonov O.V.
Development of a method for wireless information transmission in wells
equipped with sucker rod pump units //Journal of Physics: Conference
Series. -2020. -Vol.1661: 012003. DOI
\href{http://dx.doi.org/10.1088/1742-6596/1661/1/012003}{10.1088/1742-6596/1661/1/012003}.

3. Добров Э.М. Инженерная геология: Учебник / М.: Издательский центр
«Академия». - 2008. -С.151-178. ISBN 978-5-7695-2890-3.

4. Коробейников А.Ф. Геология прогнозирование и поиск месторождений
полезных ископаемых: Учебник для бакалавриата и магистратуры // Люберцы:
Юрайт. -2016. -С.10-98. ISBN~978-5-4387-0175-0.

5. Кочеков М.А. Повышение эффективности эксплуатации штанговых насосных
установок в высокообводненных скважинах. Автореферат дис..канд.тех.
наук: 25.00.17.- Уфа,2014.- 132 с.

6. Мухаметзянов А.К., Чернышов И. Н., Липерт А. И., Ишемгужин С. Б.
Добыча нефти штанговыми насосами/ - М.: Недра. -1993. -350 c.
ISBN~5-247-02488-5.

7. Уразаков К.Р., Богомольный Е.И., Сейтпагамбетов Ж.С., Газаров А.Г.
Насосная добыча высоковязкой нефти из наклонных обводненных скважин /
М.: Недра. - 2003. -303 c. ISBN 5-8365-0121-1.

8. Ковшов В.Д., Сидоров М.Е., Светлакова С.В. Динамометрирование,
моделирование и диагностирование состояния глубинной штанговой насосной
установки // Известия вузов. Нефть и газ. -2011.- № 3. - С.26-30.

9. Fakher S., Khlaifat A., Hossain M.E. et al. A comprehensive review of
sucker rod pumps' components, diagnostics, mathematical models, and
common failures and mitigations// Journal of Petroleum Exploration and
Production Technology.-2021.-Vol.11(10).-P.3815--3839.
\href{https://doi.org/10.1007/s13202-021-01270-7}{DOI
10.1007/s13202-021-01270-7}.

10. Гилаев Г.Г., Бахтизин Р.Н., Уразаков К.Р. Современные методы
насосной добычи нефти: монография // Уфа: Восточная печать. -2016. - 410
с. ISBN 978-5-905220-72-1.

11. Ковшов В.Д., Сидоров М.Е., Светлакова С.В. Моделирование динамограмм
на квазистационарных режимах работы ГШН //Известия вузов. Нефть и газ.
-2015. -№ 4.- С.51-56.

12. Валовский В.М., Валовский К.В. Цепные приводы скважинных штанговых
насосов/ М.: ОАО «ВНИИОЭНГ». -2004. -С.350-415. ISBN~5-88595-140-1.

13. Попов А.Л., Вихарев А.Н., Абанов А.Э., Теселкин М.В., Штанговые
скважинные насосные установки: конструирование и расчет: учебное
пособие/ Северный (Арктический) федеральный университет им. М.В.
Ломоносова: САФУ. -2016.-С.37-72. ISBN978-5-261-01172-9.

14. Досжанов М.Ж., Жубанов О.Н., Султанбеккызы А. Вопросы
математического моделирования процессов разработки трудоизвлекаемых и
высоковязких запасов нефти// The Way of Science №1(1). -2014.
-С.100-102. ISSN 2311-2158.

15. Досжанов М.Ж., Қойлыбаев Б.Н., КаражановаМ.К., Ахметов Д.А.,
Тасболат Г.Ж., Анализ развития трещин автогидроразрыва на нагнетательных
скважинах//Научный журнал «Нефть и газ».-2021 -№1(121). -С.69-76.
\href{https://doi.org/10.37878/2708-0080/2021-1.05}{DOI
10.37878/2708-0080/2021-1.05}.

16. Сейтжанов С.С., Тасболат Г.Ж., Сүлейменов Н.С., Танжариков П.А.,
ДосжановМ.Ж. Кен орнының сарқылуы жағдайында мұнай ұңғымасының көлденең
учаскесінің ағымдағы ұзындығын анықтау әдісі//«ҚазТБУ
хабаршысы».-2024.-~№4(25).- Б.289-297.
\href{https://doi.org/10.58805/kazutb.v.4.25-637}{DOI
10.58805/kazutb.v.4.25-637}.
\end{refs}

\begin{center}
{\bfseries References}
\end{center}

\begin{refs}
1. Anan' ev V.P., Potapov A.D., Fil' kin
N.A. Special' naja inzhenernaja geologija: Uchebnik / M.:
Infra-M. - 2016. -S.111-125. ISBN 978-5-16-010407-2. {[}in Russian{]}

2. Galeev A.S., ArslanovR.I., Suleymanov R.N., Filimonov O.V.
Development of a method for wireless information transmission in wells
equipped with sucker rod pump units //Journal of Physics: Conference
Series. -2020. -Vol.1661: 012003. DOI
\href{http://dx.doi.org/10.1088/1742-6596/1661/1/012003}{10.1088/1742-6596/1661/1/012003}.

3. Dobrov Je.M. Inzhenernaja geologija: Uchebnik / M.:
Izdatel' skij centr «Akademija». - 2008. -S.151-178. ISBN
978-5-7695-2890-3. {[}in Russian{]}

4. Korobejnikov A.F. Geologija prognozirovanie i poisk mestorozhdenij
poleznyh iskopaemyh: Uchebnik dlja bakalavriata i magistratury //
Ljubercy: Jurajt. -2016. -S.10-98. ISBN 978-5-4387-0175-0. {[}in
Russian{]}

5. Kochekov M.A. Povyshenie jeffektivnosti jekspluatacii shtangovyh
nasosnyh ustanovok v vysokoobvodnennyh skvazhinah. Avtoreferat
dis..kand.teh. nauk: 25.00.17.- Ufa,2014.- 132 s. {[}in Russian{]}

6. Muhametzjanov A.K., Chernyshov I. N., Lipert A. I., Ishemguzhin S. B.
Dobycha nefti shtangovymi nasosami/ - M.: Nedra. -1993. -350 c. ISBN
5-247-02488-5. {[}in Russian{]}

7. Urazakov K.R., Bogomol' nyj E.I., Sejtpagambetov
Zh.S., Gazarov A.G. Nasosnaja dobycha vysokovjazkoj nefti iz naklonnyh
obvodnennyh skvazhin / M.: Nedra. - 2003. -303 c. ISBN 5-8365-0121-1.
{[}in Russian{]}

8. Kovshov V.D., Sidorov M.E., Svetlakova S.V. Dinamometrirovanie,
modelirovanie i diagnostirovanie sostojanija glubinnoj shtangovoj
nasosnoj ustanovki // Izvestija vuzov. Neft'{} i gaz.
-2011.- № 3. - S.26-30. {[}in Russian{]}

9. Fakher S., Khlaifat A., Hossain M.E. et al. A comprehensive review of
sucker rod pumps' components, diagnostics, mathematical models, and
common failures and mitigations// Journal of Petroleum Exploration and
Production Technology.-2021.-Vol.11(10).-P.3815--3839.
\href{https://doi.org/10.1007/s13202-021-01270-7}{DOI
10.1007/s13202-021-01270-7}.

10. Gilaev G.G., Bahtizin R.N., Urazakov K.R. Sovremennye metody
nasosnoj dobychi nefti: monografija // Ufa: Vostochnaja
pechat'. -2016. - 410 s. ISBN 978-5-905220-72-1. {[}in
Russian{]}

11. Kovshov V.D., Sidorov M.E., Svetlakova S.V. Modelirovanie
dinamogramm na kvazistacionarnyh rezhimah raboty GShN //Izvestija vuzov.
Neft'{} i gaz. -2015. -№ 4.- S.51-56. {[}in Russian{]}

12. Valovskij V.M., Valovskij K.V. Cepnye privody skvazhinnyh shtangovyh
nasosov/ M.: OAO «VNIIOJeNG». -2004. -S.350-415. ISBN 5-88595-140-1.
{[}in Russian{]}

13. Popov A.L., Viharev A.N., Abanov A.Je., Teselkin M.V., Shtangovye
skvazhinnye nasosnye ustanovki: konstruirovanie i raschet: uchebnoe
posobie/ Severnyj (Arkticheskij) federal' nyj universitet
im. M.V. Lomonosova: SAFU. -2016.-S.37-72. ISBN978-5-261-01172-9. {[}in
Russian{]}

14. Doszhanov M.Zh., Zhubanov O.N., Sultanbekkyzy A. Voprosy
matematicheskogo modelirovanija processov razrabotki trudoizvlekaemyh i
vysokovjazkih zapasov nefti// The Way of Science №1(1). -2014.
-S.100-102. ISSN 2311-2158. {[}in Russian{]}

15. Doszhanov M.Zh., Қojlybaev B.N., KarazhanovaM.K., Ahmetov D.A.,
Tasbolat G.Zh., Analiz razvitija treshhin avtogidrorazryva na
nagnetatel' nyh skvazhinah//Nauchnyj zhurnal
«Neft'{} i gaz».-2021 -№1(121). -S.69-76. DOI
10.37878/2708-0080/2021-1.05. {[}in Russian{]}

16. Sejtzhanov S.S., Tasbolat G.Zh., Sүlejmenov N.S., Tanzharikov P.A.,
DoszhanovM.Zh. Ken ornynyң sarқyluy zhaғdajynda mұnaj ұңғymasynyң
kөldeneң uchaskesіnің aғymdaғy ұzyndyғyn anyқtau әdіsі//«ҚazTBU
habarshysy».-2024.- №4(25).- B.289-297. DOI 10.58805/kazutb.v.4.25-637.
{[}in Russian{]}
\end{refs}

\begin{info}
\hspace{1em}\emph{{\bfseries Авторлар туралы мәліметтер}}

Досжанов М.Ж. - техникалық ғылымдар докторы, профессор, Қызылорда
«Болашақ» университеті, Қызылорда, Қазақстан, е-mail:
doszhanov55@mail.ru;

Нифонтов Ю.А. - техника ғылымдарының докторы, профессор,
Санкт-Петербург мемлекеттік теңіз техникалық университеті,
Санкт-Петербург, Ресей, е-mail: nifontov@yandex.ru;

Тасболат Ғ.Ж. - техника ғылымдарының магистрі, сеньор-лектор, Қ.
Құлажанов атындағы Қазақ технология және бизнес университеті, Астана,
Қазақстан, е-mail: galymzhan\_zh@mail.ru;

Қодар Е.Т. - техника ғылымдарының кандидаты, доцент, Қызылорда
«Болашақ» университеті, Қызылорда, Қазақстан, e-mail:
erdenkodar@inbox.ru;

Сүлейменов Н.С. - техника ғылымдарының кандидаты, Қорқыт Ата атындағы
Қызылорда университеті, Қызылорда, Қазақстан, е-mail:
nurzhan\_suleymen@mail.ru;

Бисенбайқызы Л. - аға оқытушы, Қызылорда «Болашақ» университеті,
Қызылорда, Қазақстан, e-mail: lyazzat84\_bu@mail.ru;

Төлегенова А.М. - техника ғылымдарының магистрі, аға оқытушы,
Қызылорда «Болашақ» университеті

\hspace{1em}\emph{{\bfseries Information about the authors}}

Doszhanov M. - Doctor of Technical Sciences, Professor, Kyzylorda
Bolashak University, Kyzylorda, Kazakhstan, е-mail:
doszhanov55@mail.ru;

Nifontov I. - Doctor of Technical Sciences, Professor,
Saint-Petersburg state marine technical University, Saint-Petersburg,
Russia, е-mail: nifontov@yandex.ru;

Tasbolat G. - Master of Technical Sciences, Senior Lecturer, K.
Kulazhanov Kazakh University of Technology and Business, Astana,
Kazakhstan, е-mail: galymzhan\_zh@mail.ru;

Kodar E. - Candidate of Technical Sciences, Associate Professor,
Kyzylorda Bolashak University, Kyzylorda, Kazakhstan, е-mail:
erdenkodar@inbox.ru;

Suleimenov N.S. - Candidate of Technical Sciences, Korkyt Ata
Kyzylorda University, Kyzylorda, Kazakhstan, e-mail:
nurzhan\_suleymen@mail.ru;

Bisenbaikyzy L. - Senior, Kyzylorda Bolashak University, Kyzylorda,
Kazakhstan, е-mail: lyazzat84\_bu@mail.ru;

Tolegenova A. - Master of Technical Sciences, Senior Lecturer,
Kyzylorda Bolashak University, Kyzylorda, Kazakhstan, е-mail:
smk\_bolashak@mail.ru.
\end{info}
