\id{ҒТАМР 52.47.19}{}

\begin{header}
\swa{}{МҰНАЙГАЗ КЕН ОРЫНДАРЫН ИГЕРУДІҢ ТИІМДІ ИННОВАЦИЯЛЫҚ ӘДІСТЕРІН ЗЕРТТЕУГЕ АНАЛИТИКАЛЫҚ ШОЛУ}

Б.М. Нұранбаева\envelope,
А.М. Жұматай
\end{header}

\begin{affil}
Caspian University, Алматы, Казахстан

\corrauthor{Корреспондент-автор: bulbulmold@gmail.com}
\end{affil}

Мақалада мұнай-газ өнеркәсібінде цифрландыру шешімдерінің қарқынды және
кеңінен қолданысқа иә болуына байланысты құрылымдық трансформация
сатысында кен орындарын игерудің заманауи әдістеріне арналған ғылыми
материалдарға аналитикалық шолу жүргізілді.

Зерттеу барысында «цифрлық кен орны» тұжырымдамасының негізгі
элементтерін, өндірістік деректерді интеграцияланған басқару,
өндірістегі үрдістерді автоматтандыру, сонымен қатар оны оңтайландыру
рөлін, оның ішінде мұнай бергіштік коэффициентін арттыруға бағытталған
технологиялық шешімдерді талданды. AVIST цифрлық платформасының
функционалдық мүмкіндіктері және оның интеллектуалды жүйелерді,
модельдеу құралдарын және деректерді өңдеу архитектурасын біріктіру
маңыздылығы Қарақұдық кен орнында жүзеге асырылған пилоттық жобаның
материалдары негізінде ашылып көрсетілді.

Мақалада жүргізілген шолу нәтижелерінің нәтижесіні көрсеткендей
мұнай-газ саласындағы ци\-фрландырудың өзекті бағыттарын айқындап,
инновациялық технологияларды өнеркәсіптік деңгейде енгізу арқылы оның
тиімділігін бағалауға мүмкіндік береді. Алынған тұжырымдар жоғары
қолданбалы мәнге ие және қазіргі өндірістік процестерді жаңғыртуға
бағытталған тәжірибелік шешімдер үшін әдістемелік негіз бола алады.

{\bfseries Түйін сөздер}: кен орын, игеру, цифрландыру, зерттеу,
аналитикалық шолу.

\begin{header}
АНАЛИТИЧЕСКИЙ ОБЗОР ИССЛЕДОВАНИЙ ЭФФЕКТИВНЫХ ИННОВАЦИОННЫХ МЕТОДОВ РАЗРАБОТКИ МЕСТОРОЖДЕНИЙ НЕФТИ И ГАЗА

Б.М. Нуранбаева\envelope,
А.М. Жұматай
\end{header}

\begin{affil}
Caspian University, Алматы, Казахстан,

e-mail: bulbulmold@gmail.com
\end{affil}

В статье проведен аналитический обзор научных материалов, посвященных
современным инновационным методам разработки месторождений на стадии
структурной трансформации, в которые решения цифровизации связаны с
интенсивным и широким применением в нефтегазовой отрасли.

Исследование направлено на анализ ключевых элементов концепции
«цифрового месторождения», интегрированного управления производственными
данными, роли автоматизации и оптимизации процессов, а также
технологических решений, направленных на повышение коэффициента
нефтеотдачи. Раскрыты функциональные возможности цифровой платформы
AVIST и ее значение в интеграции интеллектуальных систем, средств
моделирования и архитектуры обработки данных на основе материалов
пилотного проекта, реализованного на месторождении Каракудык.

Результаты проведенного обзора позволяют оценить эффективность внедрения
инновационных технологий на промышленном уровне с определением
актуальных направлений цифровизации нефтегазовой отрасли. Полученные
выводы имеют высокую прикладную ценность и могут служить
методологической основой для практических решений, направленных на
модернизацию современных производственных процессов.

{\bfseries Ключевые слова:} месторождение, разработка, цифровизация,
исследования, аналитический обзор.

\begin{header}
ANALYTICAL REVIEW OF RESEARCH ON EFFECTIVE INNOVATIVE METHODS FOR DEVELOPING OIL AND GAS FIELDS

B. Nuranbaeva\envelope,
A. Zhumatai
\end{header}

\begin{affil}
Caspian University, Алматы, Казахстан,

e-mail: bulbulmold@gmail.com
\end{affil}

The article provides an analytical review of scientific materials
devoted to modern innovative methods of field development at the stage
of structural transformation, in which digitalization solutions are
associa\-ted with intensive and widespread use in the oil and gas
industry.

The article provides an analytical review of scientific materials
devoted to modern innovative methods of oil and gas field development.
The study focuses on analyzing the key elements of the ``digital field''
concept, integrated production data management, the role of automation
and process optimization, as well as technological solutions aimed at
increasing oil recovery rates. The functional capabilities of the AVIST
digital platform and its importance in the integration of intelligent
systems, simulation tools, and data processing architecture are revealed
based on materials from a pilot project implemented at the Karakudyk
field.

The results of the review allow us to assess the effectiveness of the
implementation of innovative technologies at the industrial level and
identify the current trends in the digitalization of the oil and gas
industry. The conclusions drawn are of high practical value and can
serve as a methodological basis for practical solutions aimed at
modernizing contemporary production processes.

{\bfseries Keywords}: field, development, digitalization, research,
analytical review.

\begin{multicols}{2}
{\bfseries Кіріспе.} Жыл сайын дәстүрлі мұнай қоры таусылып, оның сапасы
нашарлайды. Бұл, ең алдымен, кен орындарының жоғары өндірілуіне және
өндірілетін өнімнің жоғары сулануына байланысты. Бұл факт қазіргі
өнеркәсіп үшін қиын алынатын қорларды дамытуға жаңа міндеттер қояды,
оларды қолданыстағы өндіріс технологияларымен дамыту тиімсіз. Қазіргі
уақытта тұтқырлығы жоғары және өте тұтқырлы мұнайды өндіруге көбірек
көңіл бөлу өзекті мәселелердің бірі.

\emph{Дәстүрлі қорлардың сарқылуы}. Жыл сайын көмірсутектердің күрделі,
өткізгіштігі төмен және алынуы қиын қорлары игерілуде. Бұл мұнай мен газ
өндіруді арттырудың жаңа тәсілдерін іздеуді талап етеді.

\emph{Тиімділік және оны арттыру}. Дәстүрлі әдістер енді әрқашан табысты
өндірісті қамтамасыз ете бермейді. Инновациялық технологиялар шығындарды
оңтайландыруға, ұңғыма өнімділігін арттыруға және өндіріс шығындарын
азайтуға мүмкіндік береді.

\emph{Цифрландыру мен автоматтандыру}. Қазіргі мұнай-газ өнеркәсібі
жинақталған ғылыми деректерді жан-жақты талдау мен жалпылауды қажет
ететін цифрлық егіздерді, зияткерлік кен орындарының жүйелерін, заттар
интернетін (IoT) және жасанды интеллектті белсенді түрде енгізуде.

\emph{Экологиялық мәселелер және тұрақты даму}. Ең маңызды міндет -
инновациялық және экологиялық таза технологиялардың көмегімен мүмкін
болатын кен орындарын пайдалану кезінде қоршаған ортаға әсерді барынша
азайту.

\emph{Әлемдік энергияның ауысуы}. Таза энергияға жаһандық көшу
жағдайында мұнай-газ саласы бәсекеге қабілеттілігін арттыру үшін
көмірсутектерді өндіру және өңдеу технологияларын жетілдіру арқылы
бейімделуі керек.

Зерттеудің мақсаты - мұнай-газ секторы қызметінің негізгі аспектілеріне,
атап айтқанда, кен орындарын игеру және игеру саласындағы заманауи
технологиялық шешімдердің әсерін талдау. Өнеркәсіптік тәжірибеде
қолданылатын озық тәжірибелер мен олардың кең таралуына кедергі
келтіретін кедергілер қарастырылады.

Көмірсутек ресурстары көптеген ондаған жылдар бойы жаһандық энергия
балансының негізгі элементі болып қала берді. Дегенмен, қазіргі заманғы
мұнай-газ өнеркәсібі қол жетімді қорларды азайтуды, операциялық
шығындарды арттыруды және қоршаған ортаны реттеу талаптарын күшейтуді
қоса алғанда, бірқатар маңызды міндеттерді талап етеді {[}1{]}. Мұндай
жағдайларда қоршаған ортаға әсерді азайта отырып, көмірсутектерді барлау
және өндіру процестерінің тиімділігін арттыруға бағытталған инновациялық
технологияларды енгізудің маңыздылығы артады {[}2{]}.

Авторлардың пікірі {[}3{]} саланың тұрақты және бәсекеге қабілетті
дамуын қамтамасыз етудегі цифрландыру, автоматтандыру және
интеллектуалды жүйелердің рөлін атап көрсетеді.

Инновациялық технологиялық шешімдерді пайдалану өндіріс шығындарын
айтарлықтай азайтуға және экологиялық тәуекелдерді азайтуға көмектеседі.
Осылайша, нақты уақыт режимінде деректерді өңдеу арқылы сенсорлық
бақылау жүйелері мен аналитикалық жүйелерді енгізу жабдықтың жұмыс
параметрлерін жылдам бақылауға, төтенше жағдайлардың алдын алуға және
технологиялық процестерді оңтайландыруға мүмкіндік береді {[}4{]}.

Жұмыста {[}5{]} авторлар автоматтандырылған жүйелер мен роботтық
қондырғыларды пайдалану, әсіресе шалғай немесе экстремалды нысандарда,
соның ішінде теңіз платформаларында жұмыс қауіпсіздігін арттырады.

Өнеркәсіптің дамуындағы заманауи тенденциялар көмірсутектердің дәстүрлі
көздерінен дәстүрлі емес көздеріне көшуді болжайды. Маңызды назар
аударатын нысандардың бірі - ауыр мұнай, ол барланған әлемдік қорлардың
шамамен 70\% құрайды {[}6{]}. Дегенмен, оны алу техникалық күрделіліктің
жоғарылауымен байланысты. Қазіргі уақытта буды ынталандыру
технологиялары кеңінен қолданылады, олардың тиімділігі 30\% -дан
аспайды, ал мұнайдың айтарлықтай көлемі жер қойнауында қалады. Осыған
байланысты тиімділігі 55\% жетуі мүмкін ауаны айдау {[}7{]} сияқты
балама әдістерге қызығушылық артып келеді.

Мұнай қабаттарына ауа айдау - бу стимуляциясынан бірқатар
артықшылықтарға ие перспективалы және экологиялық таза технология. Ол
мұнай өндіру құнын төмендетуге көмектеседі, су мен табиғи газды тұтынуды
азайтады және көміртегі ізін азайтады. Мысалы, Шыңжаң кен орнында ауаны
айдау құны буды пайдаланатын ұқсас операциялар құнының шамамен 60\%
құрайды. Бұл технология жобалардың экономикалық табыстылығын арттыруға,
энергетикалық қауіпсіздікті нығайтуға және көмірсутегі ресурстарын
дамытуда төмен көміртекті шешімдерді ілгерілетуге көмектеседі. Дегенмен,
мұндай жобаларды жүзеге асыру өндірісті басқаруға кешенді көзқарасты
және геологиялық, техникалық және экологиялық параметрлерді қатаң
ескеруді талап етеді. Тәжірибе {[}8{]} көрсеткендей, мұндай
бастамалардың табыстылығы 20\%-дан аспайды, бұл көптеген факторларға,
соның ішінде процестердің күрделілігіне, тәжірибелік базаның болмауына
және деректердің толық болмауына байланысты.

Мұнайдың {[}9{]} ауыр фракцияларының тотығуымен бірге жүретін
термохимиялық процестерді зерттеу үш температуралық режимді анықтауға
мүмкіндік береді - төмен, орташа және жоғары. Коксты жоғары
температурада жағуға назар аударылғанымен, төмен және орташа
температурадағы тотығу процестері әлі де аз зерттелген. Мұндай
реакциялар кезінде пайда болатын жылу әсерлерін және олардың қабаттағы
температураны жоғарылату әлеуетін зерттеу әсіресе өзекті болып көрінеді.

{\bfseries Материалдар мен әдістер.} Тұтқырлығы жоғары мұнайдың тотығу
үрдісі жүретін температура диапазоны үш негізгі кезеңді ажыратуға
мүмкіндік береді: төмен, орташа және жоғары температура тотығуы. Отандық
және шетелдік зерттеушілер жүргізген жұмыстардың көпшілігі жоғары
температурада болатын процестерге, негізінен алынған коксты жағуға
байланысты. Сонымен қатар, негізінен сапалық сипатта болатын төмен және
орташа температура жағдайында шикі мұнай компоненттерінің тотығу
реакцияларына қатысты зерттеулер оттегі шығынын және тотығу өнімдерінің
құрамын зерттеумен шектеледі. Сонымен қатар, зерттеулер мен әдебиеттерде
төмен және орташа температуралық тотығудың жылулық сипаттамаларын
объективті бағалауға мүмкіндік беретін жүйелі эксперименттік деректер
жетіспейді. Бұл осы жағдайларда бөлінетін энергия өнімді қабатқа
термиялық әсер ету үшін жеткілікті болуы мүмкін бе деген белгісіздікке
әкеледі. Сонымен қатар, геологиялық және стратиграфиялық факторлардың
(қабат қысымы, қабат суының болуы, кеуектілік, сазды минералдардың
мөлшері және т.б.) термиялық тотығу процестеріне және ауыр мұнайдың
тотығу кезіндегі жылу бөліну дәрежесіне әсері туралы мәселе ашық күйінде
қалып отыр.

Осыған байланысты {[}3{]} қабат жағдайларының (қысым, қанықтылық,
кеуектілік және тау жыныстарының құрамы) термиялық тотығу әдістерінің
тиімділігіне әсері маңызды болып қала береді. Жылдам технологиялық
трансформация жағдайында мұнай-газ саласы прогрессивті шешімдерді үнемі
бақылауды және енгізуді талап етеді. Жаңа тенденцияларды терең түсіну
және өзгермелі энергия нарығына бейімделу ғана тұрақты даму
стратегияларын қалыптастыруға мүмкіндік береді.

«Ақылды» кен орындарының {[}4{]} тұжырымдамаларын құру үшін цифрлық
модельдерді пайдалану ерекше маңызға ие. Математикалық модельдеу
өндірістік процестердің әртүрлі элементтерін бір жүйеге біріктіреді, бұл
технологиялық тізбектерді тереңірек талдауға ғана емес, сонымен қатар
басқару шешімдерін жылдам қабылдауға мүмкіндік береді. Қазіргі жағдайда
деректерді талдау, процестерді визуализациялау және жүйенің мінез-құлқын
болжау тиімді кен орындарын басқарудың ажырамас құрамдас бөліктеріне
айналуда.

Schlumberger, Roxar және Petroleum Experts ұсынатын бағдарламалық
жүйелер бұрғылауды жоспарлау мен жабдықты таңдаудан бастап өндіру мен
ұңғымаларды жоюды оңтайландыруға дейінгі кен орнын {[}5{]} игерудің
өмірлік циклінің барлық кезеңдерін қамтиды. Олар өндіріс параметрлерін
үздіксіз бақылауға, қабат қысымын басқаруға және техникалық қызмет
көрсетуді уақтылы жоспарлауға мүмкіндік береді, осылайша операциялық
және ұзақ мерзімді тиімділікті қамтамасыз етеді.

BP жобаларындағы өндірістің 2-3\%-ға артуы және Shell-дегі 5 миллиард
долларлық экономикалық тиімділік сияқты практикалық мысалдар мұнай мен
газды цифрландырудың тиімділігін көрсетеді. Дегенмен, үлкен көлемдегі
деректерді талдау және түсіндіру, үлгілерді үнемі жаңарту және
нәтижелердің сенімділігін қамтамасыз ету тек техникалық шешімдерді ғана
емес, сонымен қатар ұйымдастырушылық күш-жігерді де қажет етеді. Кәдімгі
операцияларды автоматтандыру терең талдау үшін ресурстарды босатады, бұл
өз кезегінде стратегиялық негізделген басқару шешімдеріне ықпал етеді.

{\bfseries Нәтижелер және} {\bfseries талқылау.} Еліміздегі мұнай-газ
саласының дамуына қазіргі кезеңіне стратегиялық даму бағыттарын
айқындайтын инновациялық және цифрлық технологиялардың елеулі әсерімен
сипатталады. Технологиялық прогресті интенсификациялау жағдайында
инновациялық тенденцияларды үнемі бақылау, оларды аналитикалық түсіндіру
және басқару шешімдеріне айналдыру негізгі міндеттерге айналады. Бұл
процестерді {[}10,11{]} жүйелі қабылдау ресурс тиімділігін, экологиялық
тұрақтылықты арттыру және энергия нарығының өзгеретін талаптарына
бейімделу үшін қажет.

Саланы цифрландырудың маңызды бағыттарының бірі мұнай-газ орындарын
кешенді модельдеуді дамыту болып табылады. Бұл тәсіл геологиялық,
геофизикалық, гидродинамикалық және өндірістік үлгілерді қабаттар мен
ұңғымалардың әрекетін кешенді бағалауға және болжауға мүмкіндік беретін
бір цифрлық жүйеге біріктіруді қамтамасыз етеді. Бұл тәсілді {[}12,
13{]} жүзеге асыру ауыстыру өмірлік циклінің барлық кезеңдерінде -
жобалаудан бастап жабуға дейін негізделген операциялық және стратегиялық
шешімдер қабылдауға ықпал етеді.

Іс жүзінде жетекші халықаралық компаниялар интеграцияланған модельдер
мен цифрлық қосарлану негізіне белсенді түрде Petrel, Roxar RMS және IPM
Suite сияқты бағдарламалық жүйелер көп параметрлі оңтайландыруға,
есептеулерді автоматтандыруға және күрделі процестерді визуализациялауға
мүмкіндік береді және кен орындарға енгізуге, сонымен қатар бұл оның
өнімділікті арттыруға, шығындарды азайтуға және инвестицияның
қайтарымдылығын жақсартуға көмектесетіндігін атап көрсетеді.

Атап айтқанда, BP жүзеге асырған цифрлық трансформация цифрлық
қосарлануға негізделген икемді басқару жүйесін енгізу есебінен
күнделікті өндірістің 2-3\%-ға артуына әкелді. Аталған модельдеу
платформаларының тиімділігі нақты жағдайлармен расталады. Авторлардың
{[}7, 14{]} Shell өзінің цифрлық кен орнының архитектурасын енгізгеннен
кейін кірістің 5 миллиард АҚШ долларына өскенін атап өтті. Мұндай
нәтижелер цифрлық инвестициялардың экономикалық орындылығын және
интеграцияланған шешімдердің өзектілігін көрсетеді. Әлемдегі жетекші
мұнай-газ өндіруші компаниялар интеграцияланған модельдеу әдістерін
тәжірибеде белсенді қолдана бастады және мұндай тәсілдің жоғары
тиімділігін ескере отырып, жарияланған есептері бойынша «цифрлық кен
орны» бағдарламасын енгізу мұнай өндірудің тәуліктік көлемін 2-3 \%
арттырған. Осындай құралдарды Shell компаниясы да қолданған, нәтижесінде
шамамен 5 млрд доллар қосымша табыс алған. Бірақ та интеграцияланған
модельдеуді іс жүзінде жүзеге асыру бірқатар қиындықтарды тудырады.
Негізгі мәселелердің бірі деректердің үлкен көлемін түсіндірудің
күрделілігі және өндіріс көрсеткіштерінің динамикасына байланысты
үлгілерді үнемі жаңарту қажеттілігі туындайды. Бұл қиындықтарды жою үшін
{[}15,16{]} адам факторын барынша азайтатын және болжамның дәлдігін
арттыратын деректерді жинау, алдын ала өңдеу және тексерудің
автоматтандырылған жүйелерін енгізу орынды.

Перспективалы бағыттардың бірі сыни параметрлерді жылдам талдауға,
ауытқуларды анықтауға және адаптивті басқару сценарийлерін ұсынуға
қабілетті интеллектуалды шешімдерді қолдау жүйелерін дамыту болып
табылады. Мұндай жүйелерді модельдеумен {[}11, 13{]} біріктіру баламалы
стратегияларды бағалауға мүмкіндіктер ашады (сценарийлік талдау), бұл
өндірістің тұрақтылығын арттыру және тәуекелдерді азайту үшін өте
маңызды.

2012-2014 жылы пилоттық жоба аясында Қарақұдық кен орнында AVIST
пайдалану айтарлықтай экономикалық нәтиже берді {[}17{]}.3,7 миллион
доллар инвестиция арқылы шамамен 17 миллион доллар болжамды пайда
алынды, бұл мұнай өндіруді басқарудағы цифрлық технологиялардың жоғары
табыстылығын растайды.

Аймақтық тәжірибеде цифрлық шешімдерді интеграциялаудың мысалы ретінде
ЛУКОЙЛ Оверсиз үшін ITPS компаниялар тобы әзірлеген Қазақстандағы
Қарақұдық кен орнында AVIST платформасын енгізуді келтіруге болады. Бұл
жүйе нақты уақыт режимінде телеметриялық ақпаратты жинауды, оны {[}15{]}
цифрлық модельге түрлендіруді және бір басқару порталының интерфейсі
арқылы кейінгі визуализацияны қамтамасыз етеді.

Бұл мәселені шешудің тиімді жолы - жылдам әрекет ететін жүйесі құру.
Мұндай жүйенің негізгі функционалдығы кен орнының жай-күйін және оны
игеру үрдістерін сипаттайтын негізгі көрсеткіштерді бақылауды, сондай-ақ
ықтимал мәселелер туралы проактивті (алдын ала) сигналдарды беруді
қамтамасыз етеді. Соның ішінде «ЛУКОЙЛ Оверсиз» болып табылатын
Қазақстандағы «Қарақұдық» кен орнында «Parma-Telecom» (ITPS Group) жеке
әзірлеген AVIST платформасы қолданылады. Бұл платформаның көмегімен
өндірістік жүйелерден телеме-триа деректерін жылдам жүктеу үшін
автоматтандырылған интерфейстер жасауға мүмкіндік берді,
интеграцияланған модель және оның жеке құрамдас бөліктері бойынша
есептеулердің нәтижелік таттарының визуализациясы және оларды бір
басқару порталында жариялау конфигурацияланды.

AVIST интеграциялық платформасы арқылы кен орнынды цифрлық модельдеу
және басқаруға мүмкіндік береді: бұл жүйенің негізгі мақсаты - мұнай
өндірудің тиімділігін арттырып, технологиялық үдерістерді оңтайландыру
және нақты уақыт режимінде шешім қабылдау.

AVIST платформасы арқылы цифрлық кен орны архитектурасында келесі
стратегиялық рөлдерді атқарады:

1. \emph{Деректерді бірыңғай орталықта біріктіру.} Кен орнындағы
геологиялық, гидродинамикалық, технологиялық және өндірістік мәліметтер
платформаның бірыңғай базасында жинақталады, бұл ақпараттық оқшаулықты
жояды.

2. \emph{Өндірісті басқарудың интеграцияланған контуры}. Платформа
ұңғылардың, сорғы-жабдықтардың, мұнай жинау жүйелерінің жұмысын бақылау
және басқару процестерін автоматтандырады.

3. \emph{Операциялық тиімділікті арттыру}. Жасанды интеллект элементтері
мен аналитикалық алгоритмдер мұнай бергіштің коэффициентін арттыруға,
су-мұнай қатынасын (WOR) төмендетуге, жоспарсыз тоқтаулар санын азайтуға
мүмкіндік береді.

Сонымен қатар. AVIST негізіндегі аналитика ұңғы қорының жұмысын
болжауға, өндіру жоспарын түзетуге және мұнай бергіштікіті арттырудың
(EOR) әдістерін оңтайландыру кезінде таңдауға мүмкіндік береді.

Еліміздегі мұнайгаз саласының ерекшеліктеріне байланысты, AVIST
платформасының бірнеше артықшылықтары айқындауға болады: сулануы жоғары
ұңғыларды тиімді басқару; ұңғы жұмысының күрделі режимдерін модельдеу;
тепе-теңдік және динамикалық модельдер негізіндегі мұнай бергіштік
әдістерінің тиімділігін болжау; өндірістік тәуекелдерді алдын ала
анықтау; жоғары деректер көлемімен (Big Data) жұмыс істеу қабілеті. Бұл
платформа сондай-ақ қазақстандық мұнай кәсіпорындарының цифрлық
трансформация стратегияларымен үйлесімді, бірнеше кен орындарында
пилоттық және өнеркәсіптік деңгейде қолданылып жүр (1-кесте).
\end{multicols}

\tcap{Кесте 1 - Қолданылатын цифрлық платформалар}
\begin{longtblr}[
  label = none,
  entry = none,
]{
  width = \linewidth,
  colspec = {Q[167]Q[410]Q[363]},
  cells = {c},
  cells = {font = \small},
  hlines,
  vlines,
}
\textbf{Платформа}                     & \textbf{Қызметтері}                                                              & \textbf{Шектеулері}                                                     \\
{AVIST\\(ҚР жобаларында кең таралған)} & Кен орнын моделдеу, өндірісті оңтайландыру, ұңғыны нақты уақыт режимінде бақылау & Деректердің сапасына тәуелді; кейбір күрделі EOR сценарийлері шектелген \\
Schlumberger Delfi / Petrel            & Геомодельдеу, динамикалық симуляция, EOR сценарийлерін болжау                    & Лицензия құны жоғары; мамандарды оқыту қажет                            \\
Honeywell, Aspen HYSYS                 & Қондырғыларды симуляциялау, үрдістерді цифрландыру                               & Кен орын емес, технологиялық процестерге бағытталған                    \\
Cognite Data Fusion                    & Үлкен деректерді интеграциялау, IoT, цифрлық қосарлану (Digital Twin)            & Толық енгізу үшін инфрақұрылым қажет                                    \\
Oniq / Sensorless жүйелері             & Жабдықтарды болжау диагностикасы, жөндеу интервалын оңтайландыру                 & Ұңғымалық деңгейдегі модельдеу шектелуі мүмкін                          
\end{longtblr}

\begin{multicols}{2}
AVIST интеллектуалды платформасы өндіріс деректерінің дәстүрлі
интеграциясынан тыс әлеуетті көрсетеді. Оның негізгі функционалдығы
мұнай өндіруші активтерді белсенді басқару тұжырымдамасын жүзеге асыруды
қолдайтын болжамды аналитика мен сценарийлік модельдеуді қамтиды. Бұл өз
кезегінде жабдықтың тоқтап қалуын азайтуға, пайдалану тиімділігін
арттыруға және саладағы экологиялық мақсаттарға қол жеткізуге
көмектеседі. Жүйемен қамтамасыз етілген {[}11-13{]} өндірістік
процестерді жан-жақты шолу энергия саласындағы цифрлық трансформацияның
маңыздылығын атап көрсете отырып, негізделген басқару шешімдерін
қабылдауға мүмкіндік береді (2-кесте).

Атап айтқанда, «LUKOIL Oversiz» компаниясында AVIST енгізу ұңғымалардың
жағдайын нақты уақыт режимінде бақылауға мүмкіндік берді. Бұл
триггерлердің жұмысын, қабат қысымының өзгеруін және жұмыс режимдерінің
ауысуын қоса алғанда, өзара тәуелді параметрлерді ескеретін статикалық
модельдерден динамикалық модельдеуге көшуді бастады. Жетілдірілген
аналитика әртүрлі жұмыс сценарийлері үшін негізгі параметрлерді
есептеуге, инфрақұрылымдағы «кедергілерді анықтауға және өнімді
тасымалдау және дайындау процестерін оңтайландыруға мүмкіндік берді.
Платформа сонымен қатар {[}14-18{]} өндірісті интенсификациялау және
өндіріс көлемін қысқа мерзімді болжау үшін оңтайлы шешімдерді таңдау
үшін қолданылады.
\end{multicols}

\tcap{2-кесте. Цифрлық платформалардың қызметтері мен шектеулері}
\begin{longtblr}[
  label = none,
  entry = none,
]{
  width = \linewidth,
  colspec = {Q[198]Q[415]Q[325]},
  cells = {c},
  cells = {font = \small},
  hlines,
  vlines,
}
\textbf{Платформа}        & \textbf{Мүмкіндіктері}                                                    & \textbf{Шектеулері}                       \\
CMG (IMEX, GEM, STARS)    & {Химиялық, жылулық, газ EOR симуляциясы;\\көпфазалы ағынның нақты моделі} & Құны жоғары; қуатты серверді талап етеді  \\
Eclipse/ INTERSECT        & Резервуарлық модельдеу, CO₂ EOR, көпкомпонентті модель                    & {Пайдалану күрделі;\\оқытуды қажет етеді} \\
Petrel                    & Геомодельдеу, ұңғы траекториясы, интеграция                               & Real-time мониторинг шектеулі             \\
{AVIST\\(ITPS Group)}     & Операциялық тиімділікті бақылау, өндірісті оңтайландыру                   & Күрделі EOR симуляциясына арналмаған      \\
{Kappa\\(Saphir, Topaze)} & Ұңғыманы талдау, қысым қайта түсуі, well-test                             & Резервуардың толық модельдерін жасамайды  
\end{longtblr}

\begin{multicols}{2}
«Теңізшевройл» (ТШО) компаниясы жасанды интеллектті өндірістік
процестерге біріктіруді көздейтін цифрлық трансформация стратегиясын
жүзеге асыруда. Бұл {[}19{]} бағыт Солтүстік Каспий «Oil and Gas Atyra»
21-ші көрмесі аясында ұсынылды, онда цифрлық шешімдер тиімділікті
арттыруға ғана емес, сонымен қатар персоналдың қауіпсіздігін арттыруға
және экологиялық тәуекелдерді барынша азайтуға арналғаны баса айтылды.

Бұл стратегия цифрлық мәдениетті дамытуға және қызметкерлерді оқытуға
ерекше көңіл бөледі. Цифрландыру бұрғылау мен өндіруден бастап
көмірсутектерді өңдеу мен экспорттауға дейінгі бүкіл өндірістік
{[}20-23{]} циклді қамтиды, яғни басымдықтардың қатарында жабдыққа
болжамды техникалық қызмет көрсету, цифрлық қосарлану және
автоматтандырылған басқару жүйелері бар.

2020 жылдан бастап ТШО интеллектуалды өндіріс жүйелерін, ұшқышсыз
платформаларды және толықтырылған сандық технологияларын белсенді түрде
енгізуде. Бұл {[}24-25{]} шешімдер операциялық тұрақтылықты арттыруға
бағытталған «Future Rescent» және «Supply Design Management жобаларында
қолданылады.

Негізгі жетістіктердің бірі Теңіз кен орнында 2023 жылы аяқталған
технологиялық кешеннің цифрлық қосарлану құру болды. Бұл құрал
шығындарды азайтуды, табыстылықты арттыруды және қоршаған ортаға әсерді
азайтуды қамтамасыз ете отырып, сценарийлерді талдауға және процестерді
оңтайландыруға мүмкіндік береді.

Су ресурстарын басқару саласында да инновациялар енгізілді: су өлшегіш
көрсеткіштерін оқуға арналған автоматтандырылған жүйелер деректерді
қолмен жинаудан бас тартуға, операциялардың дәлдігі мен қауіпсіздігін
арттыруға мүмкіндік берді. Сонымен қатар, жұмысқа рұқсат берудің (ePTW)
электронды жүйесіне көшу ішкі процестерді жеделдетіп, операциялардың
ашықтығын арттыруға мүмкіндік берді. Жаңа жүйенің тиімділігінің мысалы
ретінде КТЛ-5 технологиялық желісін күрделі жөндеу болды, оның барысында
33 мыңнан астам электронды рұқсаттар берілді {[}26{]}.

Авторларда {[}23-26{]} машиналық оқыту және үлкен деректерді талдау
әдістерін қолдану ТШО-ға нарықтық және технологиялық жағдайлардың
өзгеруіне бейімделудің дәлірек стратегияларын әзірлеуге, шығындарды
азайтуға және өндірістік қызметтің тиімділігін арттыруға мүмкіндік
берді.

Осылайша, интеграцияланған цифрлық шешімдерді қолдану тек ағымдағы үрдіс
қана емес, сонымен қатар жаһандық трансформациялар жағдайында мұнай-газ
саласының тұрақтылығын қамтамасыз етудің маңызды элементі, сондай-ақ
заманауи технологияларды енгізу арқылы мұнай-газ саласын жаңғырту
тұрақты дамудың негізгі бағыты болып табылады, шектеулі ресурстар,
экологиялық тәуекелдер және өсіп келе жатқан тиімділік талаптары
жағдайында өзекті мәселелерді шешуге ықпал етеді.

{\bfseries Қорытынды.} Мұнай-газ салаасындағы өндірістік процестерді
цифрландыру саласындағы заманауи жетістіктер кен орындарын пайдалану
тәсілдерінде сапалы өзгерістерге әкелді. Заттар интернеті (IoT)
технологиялары, интеграцияланған модельдеу және жасанды интеллект
алгоритмдері арқылы жұмыс істейтін «интеллектуалды» деп аталатын кен
орындарының қалыптасуы саладағы елеулі трансформациялық өзгерістерді
көрсетеді. Бұл шешімдер операциялық тиімділікті арттырып қана қоймайды,
сонымен қатар шығындарды азайтуға, өнеркәсіптік қауіпсіздік
көрсеткіштерін жақсартуға және технологиялық және экологиялық
тәуекелдерді азайтуға көмектеседі.

Қарақұдық кен орындарында цифрлық платформалар мен интеллектуалды
жүйелерді енгізу мысалында қол жеткізілген оң нәтижелер мұндай
бастамалардың жоғары практикалық маңыздылығын растайды. Өндірістік
процестерді оңтайландыру, жабдықтың сенімділігін арттыру және тоқтап
қалуды азайту арқылы экономикалық қайтарымды және өндірістік қызметтің
тұрақтылығын арттыруға, экологиялық мәселелерді шешуге болады.

Жүргізілген зерттеулерге талдау бойынша еліміздегі кен орындарындың
мұнай бергіштікті арттыру (EOR) технологияларын таңдауды оңтайландыру
үшін бірқатар негізгі нәтижелерді айқындауға мүмкіндік болады және
негізгі параметрлер бойынша талданған, зерттелген статистикаға сүйеніп
жасалынды, осының нәтижесінде операторларға нақты кен орнынға EOR әдісін
дәл таңдауға мүмкіндік береді. AVIST платформасы еліміздегі кейбір кен
орындарында қолданылады, нәтижесінде операциялық тиімділікті бақылауда
жақсы, алайда күрделі EOR симуляцияларын орындау үшін қосымша модульдер
қажет етеді.

Жаһандық энергетика саласының жұмыс істеуінің жылдам өзгеретін
жағдайларын ескере отырып, бизнес-процестердің цифрлық трансформациясы
стратегиялық басымдық мәртебесіне ие болуда. Цифрлық шешімдерді
біріктіру компанияларға өзгермелі нарық талаптарына икемді бейімделуге
мүмкіндік береді, тұрақтылықты, технологияны және қоршаған ортаны
қорғауды жан-жақты жақсартуды қамтамасыз етеді. Мұндай инновациялық
тәсілдерді енгізуді жалғастыру мұнай-газ саласының орта және ұзақ
мерзімді перспективада даму векторын айқындайтыны сөзсіз, оны жаңғырту
мен экологиялық бейімделуге ықпал етеді. Жаңа технологиялардың бірегей
интеграциясы және мұнай-газ инжинирингін цифрландыру саласында ұзақ
мерзімді перспективада негізділігін арттыру және болашақтың цифрлық
экономикасын арттыру үшін цифрлық технологиялардың болашағы бар.
\end{multicols}

\begin{center}
{\bfseries Әдебиеттер}
\end{center}

\begin{refs}
1. Boukelia T.E., Bouraoui A., Laouafi A., Djimli S., Kabar Y.3E
(Energy-Exergy-Economic) comparative study of integrating wet and dry
cooling systems in solar tower power plants // Energy. -2020. -Vol.
200: 117567. \href{https://doi.org/10.1016/j.energy.2020.117567}{DOI
10.1016/j.energy.2020.117567}.

2. Khalili Y., Ahmadi M. Reservoir Modeling \& Simulation: Advancements,
Challenges, and Future Perspectives // Journal of Chemical and
Petroleum Engineering (JChPE). -2023. - Vol.57(2). - P.343-364. DOI
10.22059/jchpe.2023.363392.1447.

3. Кондратьев А.А. Цифровая трансформация в нефтегазовой сфере //
Международный научный журнал «Вестник науки». -2023. -№11(68). -T.2.
-C.804-821.

4. \href{https://www.researchgate.net/profile/Sharon-Campbell-Phillips?_tp=eyJjb250ZXh0Ijp7ImZpcnN0UGFnZSI6InB1YmxpY2F0aW9uIiwicGFnZSI6InB1YmxpY2F0aW9uIn19}{Campbell-Phillips}
Sh. A Critical Assessment of Digital Oilfield Implementations in the
Middle-East // International Journal of Recent Engineering Science.
-2020. - Vol.7(3). -~P.71-84. DOI 10.14445/23497157/IJRES-V7I3P114.

5. Moore G., Mehta R., Ursenbach M., Ursenbach M. Air injection for oil
recovery // Journal of Canadian Petroleum Technology. - 2002. - Vol.
41(8) - P.16-19.

6. Ник Л. Блокчейн в нефтегазовой отрасли России: неизбежен/Нефтянка.
-2017. URL:
\url{http://neftianka.ru/blokchejn-v-neftegazovoj-otrasli-rossii-neizbezhen/}.
-Қаралған күні: 26.08.2025.

7. Сулоева С.Б., Мартынатов В.С. Особенности цифровой трансформации
предприятий нефтегазового комплекса // Организатор производства. -
2019. - Т.27. № 2. - С.27-38. DOI 10.25987/VSTU.2019.26.70.003.

8. Liu X., Wang Y., Zhang J. AI-Generated Content in Digital Marketing:
Opportunities and Challenges//Journal of Marketing. -2023. -Vol.87.
-P.30-48.

9. Петров А.А. Сериков И.Д. Краткий обзор мирового опыта применения
теплового метода увеличения нефтеотдачи // Евразийский научный журнал.
- С.278-279.

10. Shuai Zh., Wanfen P., Mikhail A.V. Thermal behavior and kinetics of
heavy crude oil during combustion by high pressure differential
scanning calorimetry and accelerating rate calorimetry // Journal of
Petroleum Science and Engineering. -2019. -Vol.181(1): 106225. DOI
10.1016/j.petrol.2019.106225.

11. Айдарбаев A. Как «КазМунайГаз» совершает свою четвёртую промышленную
революцию // Forbes Kazakhstan. -2021. -С.22--28. URL:
\url{https://forbes.kz/articles/kak_kmg_provodit_svoyu_chetvertuyu_promyishlennuyu_revolyutsiyu}.
-Қаралған күні: 26.08.2025.

12. Mahdieh Z. Data Analysis and Management in the Oil \& Gas
Industry//Inernational conference of new technologies in oil, gas and
petrochemical engineering in Iran. -2024. DOI
10.13140/RG.2.2.28184.92165.

13. Schlumberger, Chevron и Microsoft объединятся для цифровизации
нефтяной отрасли/TACC. -2019. URL:
\url{https://tass.ru/ekonomika/6900650}. -Қаралған күні: 26.08.2025.

14. Фастович В.В. Влияние искусственного интеллекта на повышение
эффективности управления в нефтегазовой отрасли // Финансы и
управление. -2025. -№ 2. -С.157-173. DOI
10.25136/2409-7802.2025.2.74575.

15. Федосимов C. Универсальная платформа AVIST -- эффективное решение для
нефтегазодобывающих предприятий/ITPS. -2015. URL:
\url{https://itps.com/press/article/universalnaya_platforma_avist_effektivnoe_reshenie_dlya_neftegazodobyvayushchikh_predpriyatiy/?utm_source=chatgpt.com}.
-Қаралған күні: 29.01.2015.

16. Anshu S., Gurdeep S.R., Diem H.B., Parvinder P.S. Aging Midstream
Supply Chain in the Oil and Gas Industry: Issues and AI-Based
Solutions // Journal of Energy and Development. -2024. -Vol.49(2).
DOI
\href{https://doi.org/10.56476/jed.v49i2.25}{10.56476/jed.v49i2.25}.

17. Karakudukmunay hits one record after another: article in Oil\&Gas of
Kazakhstan magazine/ITPS. -2014. URL:
\url{https://itps.com/en/press/publications/2014_08_20-karakudukmunay_hits_one_record_after_another/}.
-Қаралған күні: 26.08.2025.

18. Akpe T.A., Shuaibu I.N., Bolarinwa S., Henry O.I. Balancing plant
safety and efficiency through innovative engineering practices in oil
and gas operations // Global Journal of Advanced Research and Reviews.
- 2024. -Vol.2(1). - P.023--046. DOI 10.58175/gjarr.2024.2.1.0029.

19. В Тенгизшевройл рассказали, как используется искусственный интеллект.
URL:\url{https://www.tengizchevroil.com/ru/operations/2024/04/06/how-artificial-intelligence-is-used-in-tengizchevroil}.
-Қаралған күні: 26.08.2025.

20. Mechergui A., Onumaegbu I., Ekpenyong D. Optimizing gas fields using
integrated asset modeling: OML58 Upgrade case study // SPE Nigeria
Annual International Conference and Exhibition. - 2017:~SPE-189071-MS.
DOI 10.2118/189071-MS.

21. Красеньков С.В. Развитие цифровых технологий в нефтегазодобывающей
отрасли / Специальные системы и технологии. URL:
\url{https://sst.ru/press/expert-articles/the-development-of-digital-technology-in-the-oil-and-gas-industry/}.
-Қаралған күні: 26.08.2025.

22. Азиева Р.Х. Мониторинг результатов цифровой трансформации в
нефтегазовой отрасли // Естественно-гуманитарные исследования. - 2022.
- № 40(2). - С.\,21-29.

23. Как «КазМунайГаз» совершает свою четвёртую промышленную революцию.
-2021. URL:
\url{https://sknews.kz/news/view/forbes.kz-kazmunaygaz-voshel-v-epohu-chetvertoy-promyshlennoy-revolyucii}.
-Қаралған күні: 26.08.2025.

24. Шахназарян А. Возраст «цифре» не помеха: старейшее месторождение
«ЭмбаМунайГаз» переходит на новые технологии/Ulys Media. -2023. URL:
\href{https://ulysmedia.kz/analitika/23556-vozrast-tsifre-ne-pomekha-stareishee-mestorozhdenie-embamunaigaz-perekhodit-na-novye-tekhnologii/}{https://ulysmedia.kz}.
-Қаралған күні: 26.08.2025.

25. Презентация проекта «Интеллектуальное месторождение»/КазГерМунай.
-2018. URL:
\url{https://kgm.kz/ru/news/prezentaciya-proekta-intellektualnoe-mestorozhdenie}.
-Қаралған күні: 26.08.2025.

26. Халбашкеев А. Месторождения Казахстана уходят в «цифру» // Добывающая
промышленность: Центральная Азия. -2023. URL:
\url{https://dprom.kz/trendy/myestorozhdyeneya-kazahstana-uhodyat-v-tsefru/}.
-Қаралған күні: 26.08.2025.
\end{refs}

\begin{center}
{\bfseries References}
\end{center}

\begin{refs}
1. Boukelia T.E., Bouraoui A., Laouafi A., Djimli S., Kabar Y.3E
(Energy-Exergy-Economic) comparative study of integrating wet and dry
cooling systems in solar tower power plants // Energy. -2020. -Vol.
200: 117567. \href{https://doi.org/10.1016/j.energy.2020.117567}{DOI
10.1016/j.energy.2020.117567}.

2. Khalili Y., Ahmadi M. Reservoir Modeling \& Simulation: Advancements,
Challenges, and Future Perspectives // Journal of Chemical and
Petroleum Engineering (JChPE). -2023. - Vol.57(2). - P.343-364. DOI
10.22059/jchpe.2023.363392.1447.

3. Kondrat' ev A.A. Tsifrovaya transformatsiya v
neftegazovoi sfere // Mezhdunarodnyi nauchnyi zhurnal «Vestnik nauki»
- 2023. - №11(68). -T.2. -S.804-821. {[}in Russian{]}

4. \href{https://www.researchgate.net/profile/Sharon-Campbell-Phillips?_tp=eyJjb250ZXh0Ijp7ImZpcnN0UGFnZSI6InB1YmxpY2F0aW9uIiwicGFnZSI6InB1YmxpY2F0aW9uIn19}{Campbell-Phillips}
Sh. A Critical Assessment of Digital Oilfield Implementations in the
Middle-East // International Journal of Recent Engineering Science.
-2020. - Vol.7(3). -~P.71-84. DOI 10.14445/23497157/IJRES-V7I3P114.

5. Moore G., Mehta R., Ursenbach M., Ursenbach M. Air injection for oil
recovery // Journal of Canadian Petroleum Technology. - 2002. - Vol.
41(8) - P.16-19.

6. Nik. L. Blokchein v neftegazovoi otrasli Rossii: neizbezhen/Neftyanka.
-2017. URL:
\url{http://neftianka.ru/blokchejn-v-neftegazovoj-otrasli-rossii-neizbezhen/}.
-Date of access: 21.09.2023. {[}in Russian{]}

7. Suloeva S.B. Martynatov V.S. Osobennosti tsifrovoi transformatsii
predpriyatii neftegazovogo kompleksa //Organizator proizvodstva. -
2019. -T.27.- № 2.-S.27-38. DOI 10.25987/VSTU.2019.26.70.003.
{[}in Russian{]}

8. Liu X., Wang Y., Zhang J. AI-Generated Content in Digital Marketing:
Opportunities and Challenges//Journal of Marketing. -2023. -Vol.87.
-P.30-48.

9. Петров А.А. Сериков И.Д. Краткий обзор мирового опыта применения
теплового метода увеличения нефтеотдачи // Евразийский научный журнал.
- С.278-279.

10. Shuai Zh., Wanfen P., Mikhail A.V. Thermal behavior and kinetics of
heavy crude oil during combustion by high pressure differential
scanning calorimetry and accelerating rate calorimetry // Journal of
Petroleum Science and Engineering. -2019. -Vol.181(1): 106225. DOI
10.1016/j.petrol.2019.106225.

11. Aidarbayev A. Kak «KazMunaiGaz» sovershaet svoyu chetvertuyu
promyshlennuyu revolyutsiyu (data obrashcheniya 29.09.2021) // Forbes
Kazakhstan. - 2021. - S.22--28. URL:
\url{https://forbes.kz/articles/kak_kmg_provodit_svoyu_chetvertuyu_promyishlennuyu_revolyutsiyu}.
-Date of access: 26.08.2025. {[}in Russian{]}

12. Mahdieh Z. Data Analysis and Management in the Oil \& Gas Industry//
Inernational conference of new technologies in oil, gas and
petrochemical engineering in Iran. -2024. DOI
10.13140/RG.2.2.28184.92165.

13. Schlumberger, Chevron i Microsoft ob"edinyatsya dlya tsifrovizatsii
neftyanoi otrasli/ TACC. -2019. URL:
\url{https://tass.ru/ekonomika/6900650}. -Date of access: 20.03.2024.
{[}in Russian{]}

14. Fastovich V.V. Vliyanie iskusstvennogo intellekta na povyshenie
effektivnosti upravleniya v neftegazovoi otrasli // Finansy i
upravlenie. -2025. -№ 2. -S.157-173. DOI
10.25136/2409-7802.2025.2.74575. {[}in Russian{]}

15. Fedosimov S. Universal' naya platforma AVIST
-effektivnoe reshenie dlya neftegazodobyvayushchih predpriyatij/ITPS.
-2015. URL:
\url{https://itps.com/press/article/universalnaya_platforma_avist_effektivnoe_reshenie_dlya_neftegazodobyvayushchikh_predpriyatiy/?utm_source=chatgpt.com}.
-Date of access: 29.01.2015. {[}in Russian{]}

16. Anshu S., Gurdeep S.R., Diem H.B., Parvinder P.S. Aging Midstream
Supply Chain in the Oil and Gas Industry: Issues and AI-Based
Solutions // Journal of Energy and Development. -2024. -Vol.49(2).
DOI \href{https://doi.org/10.56476/jed.v49i2.25}{10.56476/jed.v49i2.25}.

17. Karakudukmunay hits one record after another: article in Oil\&Gas of
Kazakhstan magazine/ITPS. -2014. URL:
\url{https://itps.com/en/press/publications/2014_08_20-karakudukmunay_hits_one_record_after_another/}.
-Date of access: 26.08.2025.

18. Akpe T.A., Shuaibu I.N., Bolarinwa S., Henry O.I. Balancing plant
safety and efficiency through innovative engineering practices in oil
and gas operations//Global Journal of Advanced Research and Reviews. -
2024. -Vol.2(1). - P.023--046. DOI 10.58175/gjarr.2024.2.1.0029.

19. V Tengizshevroil rasskazali, kak ispol' zuetsya
iskustvennyi intellekt. URL:
\url{https://www.tengizchevroil.com/ru/operations/2024/04/06/how-artificial-intelligence-is-used-in-tengizchevroil}.
-Date of access: 06.04.2024. {[}in Russian{]}

20. Mechergui A., Onumaegbu I., Ekpenyong D. Optimizing gas fields using
integrated asset modeling: OML58 Upgrade case study // SPE Nigeria
Annual International Conference and Exhibition. - 2017:~SPE-189071-MS.
DOI 10.2118/189071-MS.

21. Krasen' kov S.V. Razvitie tsifrovykh tekhnologii v
neftegazodobyvayushchei otrasli // Spetsial' nye
sistemy i tekhnologii. URL:
\url{https://sst.ru/press/expert-articles/the-development-of-digital-technology-in-the-oil-and-gas-industry/}.
-Date of access: 20.03.2024. {[}in Russian{]}

22. Azieva R.Kh. Monitoring rezul' tatov tsifrovoi
transformatsii v neftegazovoi otrasli // Estestvenno-gumanitarnye
issledovaniya. -2022. -№ 40(2). - S.\,21-29. {[}in Russian{]}

23. Kak «KazMunaiGaz» sovershaet svoyu chetvertuyu promyshlennuyu
revolyutsiyu. -2021.
URL: \url{https://sknews.kz/news/view/forbes.kz-kazmunaygaz-voshel-v-epohu-chetvertoy-promyshlennoy-revolyucii}.
-Date of access: 29.09.2021. {[}in Russian{]}

24. Shahnazryan A. Vozrast «tsifre» ne pomekha: stareishee mestorozhdenie
«EmbaMunaiGaz» perekhodit na novye tekhnologii/ Ulys Media.-2013 URL:
\href{https://ulysmedia.kz/analitika/23556-vozrast-tsifre-ne-pomekha-stareishee-mestorozhdenie-embamunaigaz-perekhodit-na-novye-tekhnologii/}{https://ulysmedia.kz}.
-Date of access: 28.11.2023. {[}in Russian{]}

25. Prezentatsiya proekta «Intellektual' noe
mestorozhdenie». URL:
\url{https://kgm.kz/ru/news/prezentaciya-proekta-intellektualnoe-mestorozhdenie}.
--Date of access: 26.10.2018. {[}in Russian{]}

26. Halbashkeev A. Mestorozhdeniya Kazakhstana ukhodyat v «tsifru» //
Dobyvayushchaya promyshlennost':
Tsentral' naya Aziya. -2023. URL:
\url{https://dprom.kz/trendy/myestorozhdyeneya-kazahstana-uhodyat-v-tsefru/}.
-Date of access: 16.05.2023. {[}in Russian{]}
\end{refs}

\begin{info}
\hspace{1em}\emph{{\bfseries Авторлар туралы мәліметтер}}

Нұранбаева Б.М. - химия ғылымдарының кандидаты, қауымдастырылған
профессор, Caspian University, Алматы, Қазақстан, e-mail:
bulbulmold@gmail.com;

Жұматай А.М. - магистрант, Caspian University, Алматы, Қазақстан,
e-mail: zhumatay.ansar7@gmail.com.

\hspace{1em}\emph{{\bfseries Information about the authors}}

Nuranbaeva B.M. - Ph.D. in Chemistry, Associate Professor, Caspian
University, Almaty, Kazakhstan, e-mail: bulbulmold@gmail.com;

Zhumatai A.M. - Master' s student, Caspian University, Almaty,
Kazakhstan, e-mail: zhumatay.ansar7@gmail.com.
\end{info}
