\id{ҒТАМР 52.31.01}{}

{\bfseries ПАЙДАЛЫ ҚАЗБАЛАР КЕН ОРЫНДАРЫНДАҒЫ КҮРДЕЛІ ҚҰРЫЛЫМДЫ КЕН
БЛОКТАРЫН ИГЕРУ ҮДЕРІСІНДЕ ШИКІЗАТ САПАСЫН БАСҚАРУ}

{\bfseries \tsp{1}Т.С.
Ибырханов}{\bfseries \envelope ,
\tsp{2}А.И.
Ибырханова}

\emph{\tsp{1}КеАҚ «Қ.И. Сәтбаев атындағы Қазақ ұлттық
техникалық зерттеу университеті», Алматы, Қазақстан,}

\emph{\tsp{2} Әбілқас Сағынов атындағы Қарағанды техникалық
университеті, Қарағанды, Қазақстан}

\envelope Корреспондент-автор: ibir.tem@mail.ru

Соңғы жылдары кен өндіру өнеркәсібінде айрықша маңызды жаңартулар жүзеге
асты. Заманауи ғылыми-техникалық прогресс жер қойнауын пайдалану саласын
да айналып кеткен жоқ. Инновациялық көп функцияналды ақпараттық
жүйелерді белсенді енгізу пайдалы қазбаларды өндіру мен өңдеуге тікелей
қатысты секторларға да, сондай-ақ кен орындарын жобалау және модельдеу
салаларына да әсер етті.

Қазақстан халқына Жолдауында Қасым-Жомарт Тоқаев елімізде геологиялық
барлау мен пайдалы қазбаларды өндіруді дамыту үшін кен өндіру
өнеркәсібіне қатысты бірқатар маңызды шараларды ұсынды.

Сондықтан қорларды ұтымды әрі үнемді пайдалану үстінен қатаң бақылау
орнату, сондай-ақ түпкілікті қаржылық нәтижеге бағдарланған тау-кен
жұмыстарын үздіксіз және икемді жоспарлау - дамыған елдердің әлемдік
нарығындағы қатаң бәсеке мен өзгермелі конъюнктура жағдайында табысты
қызметтің негізі. Бүгінде кен өндіруші кәсіпорындарға технологиялық
дамудың әлемдік деңгейіне жылдам шығу қажет. Әлемдік экономикадағы терең
өзгерістер геология-барлау және кен өндіру компанияларының жұмысына
көптеген жаңашылдықтарды әкелуде. Қазіргі күрделі жағдайда
тау-геологиялық бейіндегі ұйымдар үшін оңтайлы стратегия - тиімділікті
арттыру: шығындарды жан-жақты қысқарту әрі жөнелтілетін кеннің сапасын
бір мезгілде өсіру.

Кен өндіру өнеркәсібінің аса маңызды міндеті - тауарлық өнімнің талап
етілетін сапасына қол жеткізуге жағдай жасау, жоғалтуларды болдырмау
немесе барынша азайту және өңдеуге түсетін материалдағы бағалы
компонентті толық алуды қамтамасыз ету. Күрделі құрылымды кен блоктары
бар кен орындарын игеру жағдайында бұл міндет түбегейлі мәнге ие,
өйткені мұндай кен орындарының минералдық шикізатын өңдеу едәуір
материалдық шығындарды талап етеді; өндірілетін массаның сапалық
көрсеткіштерінің тұтынушы талаптарынан болар-болмас ауытқуының өзі де
кен орнын игерудің жалпы теріс экономикалық нәтижелеріне алып келуі
мүмкін. Сондықтан күрделі құрылымды кен блоктарынан өндірілетін шикізат
сапасын басқару - өзекті ғылыми-практикалық міндет.

{\bfseries Түйін сөздер:} кен орны, күрделі құрылым, пайдалы қазба, кен
массасы, сапаны басқару жүйесі, тауарлық өнімнің сапасы, жер қойнауын
ұтымды және кешенді игеру.

{\bfseries УПРАВЛЕНИЕ КАЧЕСТВОМ СЫРЬЯ ПРИ РАЗРАБОТКЕ СЛОЖНОСТРУКТУРНЫХ
РУДНЫХ БЛОКОВ МЕСТОРОЖДЕНИИ ПОЛЕЗНЫХ ИСКОПАЕМЫХ}

{\bfseries \tsp{1}Т.С.Ибырханов\envelope 
,\tsp{2}А.И. Ибырханова}

\emph{\tsp{1}НАО «Казахский национальный исследовательский
технический университет им. К.И. Сатпаева», Алматы, Казахстан,}

\emph{\tsp{2}Карагандинский технический университет им.
Абылкаса Сагинова, Караганда, Казахстан,}

e-mail: ibir.tem@mail.ru

За последние годы горнодобывающая промышленность претерпела особенно
значимые обновления. Современный научно-технический прогресс не обошел и
сферу недропользования. Активное внедрение инновационных
многофункциональных информационных систем затронуло как сектора,
непосредственно связанные с добычей и переработкой полезных ископаемых,
так и сферы проектирования и моделирования месторождении полезных
ископаемых.

В послании народу Казахстана Касым-Жомарт Токаев предложил ряд значимых
мероприятий в горнодобывающей промышленности для развития
геологоразведки и добычи полезных ископаемых в нашей стране.

Поэтому строгий контроль за рациональным и экономным использованием
запасов, а также непрерывное и гибкое планирование горных работ,
ориентированное на конечный финансовый результат, это основа прибыльной
деятельности в условиях жесткой конкуренции и изменчивой конъюнктуры
мирового рынка развитых стран. Сегодня горно добывающим предприятиям
необходим стремительный выход на мировой уровень технологического
развития. Глубокие изменения в мировой экономике приносят в работу
геологических и горнодобывающих компаний множество нововведений. В
текущих сложных условиях оптимальная стратегия для организаций
горногеологического профиля повышение эффективности: всестороннее
сокращение издержек при одновременном росте качества отгружаемой руды.

Важнейшей задачей горнодобывающей промышленности является обеспечение
условий достижения необходимого качества товарной продукции, исключить
или минимизировать потери и обеспечить полное извлечение полезного
компонента поступающей на переработку. Применительно к условиям
разработки месторождений сложноструктурных рудных блоков эта задача
будет принципиальной, так как переработка минерального сырья таких
месторождений связана со значительными материальными затратами, даже
малейшее отклонение качественных характеристик добываемой массы от
требований потребителя может привести к отрицательным экономическим
результатам разработки месторождения в целом, поэтому управление
качеством добываемого сырья в сложноструктурных рудных блоков является
актуальной научно-практической задачей.

{\bfseries Ключевые слова:} месторождение, сложноструктурные, полезное
ископаемое, рудная масса, система управления качеством, качество
товарной продукции, рациональное и комплексное освоение недр.

{\bfseries ORE QUALITY MANAGEMENT IN THE MINING OF COMPLEX-STRUCTURED}

{\bfseries ORE BLOCKS AT MINERAL DEPOSITS}

{\bfseries \tsp{1}T.S. Ibyrkhanov\envelope ,
\tsp{2}A.I.Ibyrkhanova}

\emph{\tsp{1}K.I.Satpayev Kazakh National Research Technical
University, Almaty, Kazakhstan,}

\emph{\tsp{2}Abylkas Saginov Karaganda Technical University,
Karaganda, Kazakhstan,}

e-mail: ibir.tem@mail.ru

Over the past few years, the mining industry has undergone especially
significant updates. Modern scientific and technological progress has
not bypassed the field of subsoil use (the mineral resources sector).
The active introduction of innovative, multifunctional information
systems has affected both the sectors directly involved in the
extraction and processing of mineral resources and the areas concerned
with the design and modeling of mineral deposits.

In his Address to the People of Kazakhstan, Kassym-Jomart Tokayev
proposed a number of important measures in the mining industry to
advance geological exploration and mineral extraction in our country.

Therefore, strict oversight of the rational and economical use of
reserves, as well as continuous and flexible mine planning oriented
toward the ultimate financial result, form the foundation of profitable
operations amid fierce competition and the volatile conditions of global
markets in developed economies. Today, mining enterprises must make a
rapid leap to a world-class level of technological development. Profound
changes in the global economy are bringing numerous innovations into the
work of geological and mining companies. In the current challenging
environment, the optimal strategy for mining-geological organizations is
to boost efficiency: comprehensive cost reduction while simultaneously
improving the quality of ore delivered.

A paramount task for the mining industry is to create the conditions
necessary to achieve the required quality of marketable products,
eliminate or minimize losses, and ensure full recovery of the valuable
component in the material fed to processing. In the context of
developing deposits composed of complex-structured ore blocks, this task
is fundamental, since processing mineral raw materials from such
deposits entails substantial costs; even the slightest deviation in the
quality characteristics of the mined material from customer requirements
can lead to negative economic outcomes for the deposit's development as
a whole. Therefore, managing the quality of raw materials extracted from
complex-structured ore blocks is a pressing scientific and practical
challenge.

{\bfseries Keywords:} deposit, complex structure, mineral resource, ore
mass, quality management system, quality of commercial products,
rational and comprehensive development of subsoil.

{\bfseries Кіріспе.} Минералды шикізаттың талап етілетін сапасын қамтамасыз
ету-тау-кен өндірісінің шешуші міндеті, әсіресе күрделі құрылымды кен
орындарын игеру кезінде.

Мұндай жағдайда өндірілген масса көрсеткіштерінің тұтынушы талаптарынан
шамалы да ауытқуы өндіру мен өңдеудің экономикалық тиімділігін жойып
жіберуі мүмкін. Осындай жағдайларда сапаны басқару стратегиялық сипат
алып, бір уақытта руда массасындағы құнды компоненттердің үлесін
арттыруға және байыту фабрикасына тұрақты сападағы ағын қалыптастыруға
бағытталған жүйелі тәсілді талап етеді. Сонымен қатар, пайдалы
қазбаларды игеру кезеңіндегі жалпы ағынды тауарлық өнім сапасын арттыру,
ілеспе компоненттерді пайдалануға ендіру, сондай-ақ қалдықтар мен
аумақпен ұтымды жұмыс істеу арқылы өсіруге болады.

Сондықтан, берілген жұмыстың мақсаты-күрделі құрылымды кен орындары үшін
ағын сапасының тұрақтылығына, өндірудің экономикалық тиімділігін
арттыруға бағытталған минералды шикізат сапасын басқару жүйесінің
құрылымы мен құралдарын негіздеу және сипаттау болып табылады.

{\bfseries Материалдар мен әдістер.} Жер қойнауындағы пайдалы қазбалардың
шығын мөлшерін азайтуы арқылы толық өндіру мәселесі пайдалы қазба кен
орнын игеру объектілерінің цифрлық модельдерін жетілдіру міндеттерін
одан әрі өзектендіреді.

Зерттеудің әдіснамалық негіздер және зерттеу әдістері.

Зерттеу әдіснамасы күрделі геологиялық-құрылымдық әртектілік жағдайында
минералдық шикізат сапасын басқару тұжырымдамасына негізделеді.

Ол төмендегі қағидалар мен тәсілдерді қамтиды:

- геостатистикалық модельдеу қағидаттары;

- селективті игеру және табиғи-технологиялық аймақтарға бөлу
тұжырымдамасы;

- сапаны онлайн-бақылау және жүк ағындарын жедел басқару әдістері;

- геоақпараттық модельдеу және кенорындардың блоктық 3D-модельдері
әдістері;

- жер қойнауын кешенді игеру және сапаға бағытталған өндіруді басқару
қағидаттары.

Қорлардың қозғалысын есепке алуды геологиялық және маркшейдерлік
қызметтер өндіру барысында үздіксіз жүргізеді. Кен өндіруде, өндірістік,
эксплуатациялық барлау жұмыстары кезінде туындайтын, сондай-ақ
кондицияларды түзетуге байланысты қорларды қайта есептеу нәтижесінде
пайда болатын өзгерістердің барлығын цифрлық модельге енгізеді
{[}1,2{]}.

Геометаллургиялық тәсіл пайдалы қазба сапасын басқару жүйесінің
«қаңқасы» ретінде көрінеді: ол геологиялық әртектілікті, технологиялық
көрсеткіштерді және өндіру/араластыруды жоспарлауды өзара ұштастырып,
байыту фабрикасын шикізатпен тұрақсыз қамтамасыз ету тәуекелін
төмендетеді; жақындағы шолулар геометаллургиялық карталаудың және оны
гранулометриялық бақылау модельдерімен ықпалдастырудың шешуші рөлін атап
өтеді {[}3,4{]}.

Карьердің шекараларын айқындайтын негізгі фактор - өнеркәсіптік
санаттағы барланған руда қорларының кеңістіктегі орналасуы болып
табылады. Қазіргі кондициялық бақылау сынама алудың сапасына, жарылыстан
кейін жыныс массивінің ығысуын ескеруге және іріктеу шекараларын дәл
анықтауға шоғырланады. Іріктеу кезіндегі қателері мен бағалаудың шартты
ығысуы - «сапа жоғалтуларының» басты көздері болып саналады {[}5{]}.

Жерасты және қуаты аз (жұқа) күрделі құрылымды кен денелері үшін негізгі
тренд - шағын габаритті өздігінен жүретін техника қолданылатын таңдамалы
қазып алу және қабаттап іріктеу; бұл кеннің ыдыратуын азайтып,
кондициялық кеннің шығымын арттырады {[}6,7{]}.

Пайдалы қазбалардың сенсорлық сұрыптау (XRT/XRF/HIS және т.б.) алдын ала
байыту мен фабрикаға берілетін шикізат сапасын тұрақтандырудың тиімді
құралы ретінде орнықты. Шолулар мен қолданбалы зерттеулер құрамның
айтарлықтай жақсарғанын және фабрикаға түсетін жүктеменің азайғанын
көрсетеді. Бірқатар жағдайларда алынымның/өнімділіктің
\textasciitilde5-20\% деңгейінде артқаны тіркелген {[}8-10{]}.

Пайдалы қазбаларды араластыруды оңтайландыру кезінде "сапа терезелерін"
бір уақытта ұстап тұру, ауытқуларды азайту және шығындарды тұрақтандыру
үшін көп өлшемді data-driven модельдеріне (NSGA-II, кеңейтілген мат
элементтері, цифрлық егіздерге) ауысады {[}11,12{]}.

Машиналық оқыту сапаны бағалау мен болжауды күшейтеді: ML-регрессорлары
және қашықтан зондтау (ДЗЗ) әдістері блок-модельдерді нақтылауға және
«кен/қоршаушы жыныс» бойынша жедел жіктеуді жүзеге асыруға көмектеседі.
ML әдістерін классикалық геостатистикалық модельдермен салыстыру
бірқатар кен орындарында MAPE/RMSE көрсеткіштері бойынша артықшылық
беретінін көрсетеді {[}13-15{]}.

Жерасты өндіруде ресурстық және құрамдық бақылау модельдерін жедел («on
the fly») жаңарту (model updating) тәжірибеге айналып келеді. Жедел
сынама деректері, өндірістік метрикалармен және байыту фабрикасының кері
байланысын біріктіреді {[}16{]}.

Орыс тіліндегі әдебиеттерде көбіне алдын ала аудандастыруға, кен
дайындауды жетілдіруге және сапаны басқарудың кешенді жүйелеріне
(блоктық модельдер, сорттар бойынша іріктеу, шығын азайту) басымдық
беріледі {[}17-19{]}.

Кен орны карьерді қалыптастыру селективтіліктің және жөнелтілетін кен
сапасының қажетті деңгейін қамтамасыз ететін кезеңділікті
(кезектер/фазалар) әрі уақытша контурларды көздейді. Қабылданған
шешімдер маркшейдерлік және геомониторинг жұмыстары барысында, пайдалану
кезіндегі кері байланыс нәтижелері негізінде, сондай-ақ
тау-кен-геологиялық жағдайлар өзгерген кезде нақтыланады.

Кен орны карьерлерінің 3D блок моделінің фрагментінде төмендегі шартты
белгілер (1-суретте) қолданылған:

- кондициялық блок - сапалық көрсеткіштері талапқа сай келетін, негізгі
өндіруге жататын руда бөлігі;

-автосамосвалға тиелетін кондициялық құрамды блок - өндіру барысында
тікелей қазылып, автосамосвалдарға тиеп фабрикаға жөнелтілетін
кондициялық руда бөлігі;

\fig{g2/image41}{}-
кондициялық емес құрамды қабат - сапалық көрсеткіштері нормативтен
төмен, бірақ кондициялық рудамен реттелген түрде араластыру арқылы жалпы
рудалық массаның талап етілетін сапасын қамтамасыз ету үшін
пайдаланылатын қабат.

{\bfseries 1-сурет. Күрделі құрылымды кен блоктарын игеру кезінде шикізат
сапасын басқарудың ұсынылған үдерісінің схемасы}

\emph{Кен орны моделінің 3D блок моделінің фрагменті:}

\fig{g2/image42}{}

- кондициялық блок,

\fig{g2/image43}{}-
автосамосвалға тиелетін кондициялық құрамды блок,

\fig{g2/image44}{}-
кондициялық емес құрамды қабат.

Карьерді қалыптастыру селективтіліктің және жөнелтілетін кен сапасының
қажетті деңгейін қамтамасыз ететін кезеңділікті (кезектер/фазалар) әрі
уақытша контурларды көздейді. Қабылданған шешімдер маркшейдерлік және
геомониторинг жұмыстары барысында, пайдалану кезіндегі кері байланыс
нәтижелері негізінде, сондай-ақ тау-кен-геологиялық жағдайлар өзгерген
кезде нақтыланады.

Карьерді қалыптастыру селективтіліктің және жөнелтілетін кен сапасының
қажетті деңгейін қамтамасыз ететін кезеңділікті фазалар әрі уақытша
контурларды көздейді. Қабылданған шешімдер маркшейдерлік және
геомониторинг жұмыстары барысында, пайдалану кезіндегі кері байланыс
нәтижелері негізінде, сондай-ақ тау-кен-геологиялық жағдайлар өзгерген
кезде нақтыланады.

Карьердің қалыптасуын тиелетін кеннің селективтілігі мен сапасының
қажетті деңгейін қамтамасыз ететін кезеңділікті, фазаларды және уақыт
контурларын көздейді. Қабылданған шешімдер маркшейдерлік және
геомониторинг барысында, пайдаланудың кері байланысының нәтижелері
бойынша және тау-кен-геологиялық жағдайлары өзгерген кезде нақтылауға
жатады.

Өндірілетін шикізаттың сапасын басқарудың негізгі мақсаты өзара
байланысты және өзара тәуелді екі міндетті шешу болып табылады:

- өндірілетін және өңдеуге берілетін кен массасындағы бағалы
компоненттердің құрамын арттыру;

- байыту фабрикасына кеннің түсетін сапасы бойынша тұрақты пайдалы
қазбаның ағынын қалыптастыру.

Осы міндеттерді бірлікте шешу кен орны массивінде жүргізілген
егжей-тегжейлі геологиялық зерттеулердің негізінде минералдық шикізаттың
әртүрлі технологиялық түрлерін сұрыптау арқылы мүмкін. Қатты пайдалы
қазбалардың бірқатар кен орындары бойынша жүргізілген зерттеулер
нәтижелері көрсеткендей, технологиялық операцияларды орындауға және
минералдық шикізат сапасын басқару үдерістеріне байланысты шығындардың
өсуі тау-кен кәсіпорнының шикізат өнімінің құнының артуымен толық
өтеледі.

Идеяны іске асырудың негізгі шарты лицензиялық жер қойнауындағы
учаскесін игеру кезеңінде шикізат өнімінің құнының өсуі есебінен жиынтық
дисконтталған ақша ағынының едәуір артуы; оған, соның ішінде, тауарлық
өнім сапасының жоғары болуы, өндірілген шикізатты кешенді пайдалануы,
ілеспе пайдалы қазбаларды өндіру, жинақталған және ағымдағы қалдықтарды
пайдаланымға тартуы, аумақтарды рекультивациялау және жер қойнауын әрі
қоршаған ортаны қорғау шараларын қамтамасыз етуі жатады {[}20-22{]}.

Бір типтегі қатты пайдалы қазбалардың жер қойнауында пайда болу
жағдайларының айтарлықтай айырмашылығына байланысты, сондай-ақ кен орны
шегінде олардың сапасының үлкен өзгергіштігіне байланысты сапаны
басқарудың бірыңғай жүйелерін құру орынсыз екенін есте ұстаған жөн.

Мұндай жүйелерді ұқсас белгілері бойынша (пайдалы қазбалардың түрі,
құнды компоненттердің үлесі, минералдар мен негізгі жыныстардың
минералдануы, генезисі және т.б.) біріктірілген кен орындары топтары
бойынша дифференциалды түрде игеру тиімдірек. Ал содан кейін, осыны
ескере отырып, негізінен физикалық-техникалық және физикалық-химиялық
геотехнологияларды біріктіруге негізделген кен орнын игеру үйлесіміне
сүйене отырып әдісін анықтау қажет.

Мысал ретінде алынған рудалы кен орнының блоктық геологиялық-құрылымдық
3D моделінің фрагментінен (2-сурет) жыныс массивінде пайдалы
компоненттің барлық жерде кездесетіні, бірақ оның үлесі бойынша біртекті
емес екені байқалады.

\fig{g2/image45}{}

{\bfseries 2 - сурет. Кен орны моделінің 3D блок моделінің фрагменті:}

\fig{g2/image46}{}
\emph{- құрамы А\%-дан жоғары сапалы кен},
\fig{g2/image47}{}-
\emph{құрамында А\% \textgreater{} PI \textgreater B\% бар жақсы сапалы
кен,}
\fig{g2/image48}{}
\emph{- құрамында B\% \textgreater{} PI \textgreater{} C\% бар жақсы
сапалы кен,}
\fig{g2/image49}{}
\emph{- құрамында <{} С\% және бос жынысы бар}

\emph{стандартты емес кен,}
\fig{g2/image50}{}
, \emph{тақыр бөгеттер,}
\fig{g2/image51}{}
- \emph{геологиялық бұзылулар}

{\bfseries Нәтижелер және талқылау.} Осы зерттеулердің негізінде кен орнын
игеру үшін келесі сапаны басқару жүйесі ұсынылады.

Карьер алаңын табиғи-технологиялық аймақтарға бөлу (рудалы, руда-жынысты
және жынысты), бұл өндірілетін минералдық шикізаттың сапа жөніндегі
жобалық көрсеткіштерін дифференцияланған түрде айқындауға және
минералдануы жоғары рудалы аймақтарда тау-кен жұмыстарын қарқынды
жүргізуге мүмкіндік береді. Сонымен қатар, талап етілетін орташа сапа
көрсеткіштеріне қол жеткізу үшін белгілі бір рудалану аймағында
қопарылған рудалық массаның көлемін икемді басқару мүмкіндігі туындайды.
Бұдан бөлек, тау-кен жұмыстарының бағытын жоспарлы басқару қазу
техникасының біртекті ортада жұмыс істеуіне жағдай жасап, қазу
тереңдігін оңтайландыруға және, тиісінше, жабдықтың өнімділігін
арттыруға мүмкіндік береді.

Рудалық массаның бірлік бөлігінің сапалық сипаттамаларын анықтау
нәтижелерінің сенімділігіне қойылатын талап жоғары, өйткені кондициялық
және некондициялық пайдалы қазбалардың құрамы арасындағы айырма
салыстырмалы түрде шағын. Сондықтан қозғалыстағы тау-кен массасының беті
бойынша тікелей өлшеу жүргізетін радиометриялық бақылау-өлшеу кешенін
орнату қажет; бұл автосамосвалға тиеу кезінде руданың пайдалы
компонентінің құрамын тікелей анықтауға мүмкіндік береді {[}23, 24{]}.

Бір автосамосвалды тиеу барысында арнайы (XRT/XRF/HIS және т.б.)
жабдықтың көмегімен өлшеулер жүргізіліп, әр автосамосвал кузовындағы
құнды компоненттің орташа құрамы айқындалады; бұл көрсеткіш мониторда
бейнеленеді, содан кейін автосамосвалға түсіру орны беріледі:

- рудалық массадағы кондициялық құнды компоненттің құрамы борттық
құрамнан жоғары болса - байыту фабрикасының орташалау қоймаларына;

- құрамы борттық кондициялық құрамнан төмен, бірақ өнеркәсіптік құрамнан
жоғары болса - кейінгі перспективалық пайдалану үшін қоймаға жинақтауға;

- құрамы борттық кондициялық құрамнан төмен рудалық масса арнайы
үйіндіге жіберіледі {[}25{]}.

Осылайша, пайдалы кен орнын игеру кезінде тау-кен массасы ағындарының
сапалық сипаттамаларын басқарудың ұсынылған схемасы 3-суретте
келтірілген және осындай мүмкіндіктер береді:

- карьер алаңын табиғи - технологиялық аймақтарға (рудалы, руда-жынысты,
жынысты) бөлу негізінде жер қойнауынан пайдалы қазбаларды өндірудің
жобалық көрсеткіштерін дифференцияланған түрде айқындау және тау-кен
жұмыстары фронтының дамуының ұтымды бағытын жоспарлау;

- тау-кен жұмыстарының даму жоспарын жасағанда технологиялық
көрсеткіштер мен пайдалы қазбалардың алыну көрсеткіштерін нақтылау;

- нақты өндірілген тау-кен массасы ағындарын, құрамындағы құнды
компоненттерге қарай, жекелеген технологиялық түрлерге бөлу;

- нақты уақыт режимінде технологиялық сорттар бойынша және жалпы карьер
көлемінде рудалық масса өндіру көлемдері туралы техникалық есептер алу;

- тау-кен жұмыстары фронтының даму бағытын, тау-кен-көлік кешенінің
жұмыс режимін және жүк ағындарының сипаттарын өзгерту жөніндегі
басқарушылық шешімдерді дер кезінде қабылдау.

\fig{g2/image52}{}

{\bfseries 3 - сурет. Кен орнын игеру кезіндегі минералды шикізат
ағындарының сапалық сипаттамаларын бақылау блок - схемасы}

{\bfseries Қорытынды.} Осылайша, жаһандық сын-қатерлер мен тәуекелдер
жағдайында тау-кен техникалық жүйелердің тұрақты дамуын қамтамасыз ету
және күрделі құрылымды қатты пайдалы қазбалар кен орнын игеру кезінде
жоғары экономикалық нәтижелерге қол жеткізу үшін, осы мақалада
баяндалған қағидаттар негізінде минералдық шикізат сапасын басқару
мәселесін шешу жер қойнауын пайдаланудың, сондай-ақ жер қойнауындағы
ресурстарды (георесурстарды) игерудің барлық кезеңдеріндегі маңызды
міндет болып табылады.

Тереңде жатқан күрделі құрылымды кен орындарының қорларын кешенді игеру
әдіснамасының одан арғы дамуы кәсіпорынның тау-кен-техникалық жүйесін
тау-кен жұмыстарының даму үрдісі жағдайында қызмет ету шарттарының
өзгеруіне бейімдеуге бағытталған жаңашыл шешімдерді әзірлеу және іске
асыру барысында негізгі технологиялық үдерістердің өзара байланысын және
оларды есепке алу тәсілдерін зерттеу қажеттілігімен байланысты.

{\bfseries Әдебиеттер}

1. Ракишев Б.Р. Полное извлечение кондиционных руд из сложноструктурных
блоков за счет частичного примешивания некондиционных руд // Записки
Горного института. - 2024. - Т.270. - С.919-930. ISSN 2411-3336,
2541-9404.

2. Ракишев Б.Р. Новые определения естественного и трансформированного
месторождения полезного ископаемого и этапы их эксплуатации // Горный
информационно-аналитический бюллетень. -2023. -№ 8. - С.165-177. DOI
10.25018/ 0236\_1493\_2023\_8\_0\_165.

3. Frenzel M., Baumgartner R., Tolosana-Delgado R., Gutzmer J.
Geometallurgy: Present and Future // Elements. -2023. -Vol.19 (6).
-P.345-351.
DOI~\href{https://doi.org/10.2138/gselements.19.6.345}{10.2138/gselements.19.6.345}.

4. Anvari K., Benndorf J., A Review of Developments Within the Last
Decade // Mining. -2025. -Vol.5(3): 38. DOI
\href{https://doi.org/10.3390/mining5030038}{10.3390/mining5030038}.

5. Potakey N. E., Ortiz J. M., A review of grade control methods in open
cast mining // Predictive Geometallurgy and Geostatistics Lab, Queen's
University, Annual Report. -2022. -Vol.04. - P.59-70.

6. Рогизный В.Ф., Хромов В.М., Карпухина М.В. Технологии селективной
выемки маломощных рудных тел с применением малогабаритного самоходного
оборудования // Горная промышленность. - 2020. - № 1. - С.34-41. ISSN
1609-9192, 2587-9138.

7. Желябовский Ю.Г. Система подземной разработки с селективной выемкой
руды // Золото и технологии. -2020. - №4 (46).- С.36-40.

8. Peukert D., Xu C., Dowd P. A Review of Sensor-Based Sorting in Mineral
Processing: The Potential Benefits of Sensor Fusion // Minerals.
-2022. -Vol.12 (11): 1364. DOI 10.3390/min12111364.

9. Owais A. Improving Sustainability and Efficiency in Ore Sorting with
Sensors // AZoMining. -2023.

10. dos Santos E.G., de Brum I.A.S., Ambrós W.M. Sensor-based Sorting
using De-XRT Sensor Applied to a Greenfield Copper Ore Project in
Southern Brazil // IgMin Research a BioMed \& Engineering. -2025. -Vol
3(4). - P.201-205. DOI 10.61927/igmin299.

11. Xiang J., Chen J., Zhang A., Zhao X., Zhuo S., Yang S. Multi-Objective
Ore Blending Optimization for Polymetallic Open-Pit Mines Based on
Improved Matter - Element Extension Model and NSGA-II // Mathematics.
-2025. -Vol.13 (11):1843.
\href{https://doi.org/10.3390/math13111843}{DOI 10.3390/math13111843}.

12. Luo J., Gu Q., Chen L., Li X., Li P. Multi-objective optimization for
ore blending schemes in the open-pit phosphate mine using an improved
NSGA-II algorithm // Green and Smart Mining Engineering. -2025. -Vol.
2 (1). -P.42-56.
\href{https://doi.org/10.1016/j.gsme.2024.12.004}{DOI
10.1016/j.gsme.2024.12.004}.

13. Maniteja M., Samanta G., Gebretsadik A., Tsae N.B., Rai S.S., Fissha
Y., Okada N., Kawamura Y. Advancing Iron Ore Grade Estimation: A
Comparative Study of Machine Learning and Ordinary Kriging //
Minerals. -2025. -Vol.15(2):131.
\href{https://doi.org/10.3390/min15020131}{DOI 10.3390/min15020131}

14. Thiruchittampalam S., Banerjee B. P., Glenn N. F., Raval S. A
systematic review of machine learning-based remote sensing data
analysis for geological and mined materials characterization//European
Journal of Remote Sensing. -2025. -Vol.58 (1): 2524622.
\href{https://doi.org/10.1080/22797254.2025.2524622}{DOI
\href{https://doi.org/10.1080/22797254.2025.2524622}{10.1080/22797254.2025.2524622}.}

15. Marquina-Araujo J., Cotrina-Teatino M., Mamani-Quispe J.,
Noriega-Vidal E., Vega-Gonzales J., Cruz-Galvez J. Copper Ore Grade
Prediction using Machine Learning Techniques in a Copper Deposit //
Journal of Mining and Environment. -2024. -Vol.15 (3). -P.1011-1027.
DOI 10.22044/jme.2024.14032.2617.

16. Prior Á., Jorg B. Mueller U. Resource and Grade Control Model Updating
for Underground Mining Production Settings // Mathematical
Geosciences. -2021. -Vol.53. P.757-779. DOI
10.1007/s11004-020-09881-2.

17. Пыталев И.А., Доможиров Д.В., Симаков Д.Б., Борисенко Е.В. Управление
качеством минерального сырья путем обоснования технологии и параметров
подготовки к выемке пород природных массивов при открытой
геотехнологии // Известия Тульского государственного университета.
Науки о земле. -2023. -№ 4. -С.472-483.

18. Кантемиров В.Д., Яковлев А.М., Титов Р.С., Тимохин А.В.
Совершенствование методов рудоподготовки минерального сырья при
освоении сложноструктурных месторождений // Горная Промышленность.
-2022. -№1S. С.63-70. DOI
\href{https://doi.org/10.30686/1609-9192-2022-1S-63-70}{10.30686/1609-9192-2022-1S-63-70}.

19. Мальцев Е. Н., Комплексная система управления качеством руды в
условиях реального горнодобывающего предприятия // Золотодобыча для
профессионалов: специалистов, руководителей, инвесторов. -2018. URL:
\url{https://zolotodb.ru/article/11888/}.

20. Трубецкой К. Н., Захаров В.~Н., Каплунов Д. Р., Рыльникова М. В.
Эффективные технологии использования техногенных георесурсов - основа
экологической безопасности освоения недр // Горный журнал. -2016. -№
5. - С.34-40. DOI 10.17580/gzh.2016.05.03.

21. Трубецкой К.Н. Развитие ресурсосберегающих и ресурсовоспроизводящих
геотехнологий комплексного освоения месторождений полезных ископаемых
// ИПКОН РАН. -2014. -196 c. ISBN 978-5-91177-082-2.

22. Каплунов Д.Р., Комплексное освоение недр - основное направление
проектирования разработки рудных месторождений // Горный
информационно-аналитический бюллетень (научно-технический журнал).
-2014, -№ 1. -С.347-357.

23. Федянин С. Н. О возможности разделения пород и руд месторождения
Мурунтау рентгенорадиометрическим способом // Сборник
научно-технических статей, Ташкент, Фан. - 1997. - С.135-141.

24. Еремин А.М. Научное обоснование возможности предварительного
обогащения золотосульфидных руд и разработка технологии
рентгенорадиометрической сортировки (на примере месторождения
Кокпатас): дис. канд. техн. наук: 25.00.22./ ВНИПИ промтехнологии,
Москва. - 2010.138 c.

25. Рыльникова М. В., Швабенланд Е.Е. Особенности управления качеством
рудной массы при разработке сложноструктурных месторождений апатитовых
руд с применением комбайновой выемки // Рациональное освоение недр. -
2019. -№ 2(3). - С.80-86.

{\bfseries References}

1. Rakishev B.R. Polnoe izvlechenie kondicionnyh rud iz
slozhnostrukturnyh blokov za schet chastichnogo primeshivanija
nekondicionnyh rud // Zapiski Gornogo instituta. - 2024. - T.270. - S.
919-930. ISSN 2411-3336, 2541-9404. {[}in Russian{]}

2. Rakishev B.R. Novye opredelenija estestvennogo i transformirovannogo
mestorozhdenija poleznogo iskopaemogo i jetapy ih jekspluatacii //
Gornyj informacionno-analiticheskij bjulleten'. -2023. -№
8. - S.165-177. DOI 10.25018/ 0236\_1493\_2023\_8\_0\_165. {[}in
Russian{]}

3. Frenzel M., Baumgartner R., Tolosana-Delgado R., Gutzmer J.
Geometallurgy: Present and Future // Elements. -2023. -Vol.19 (6). -P.
345-351.
DOI~\href{https://doi.org/10.2138/gselements.19.6.345}{10.2138/gselements.19.6.345}.

4. Anvari K., Benndorf J., A Review of Developments Within the Last
Decade // Mining. -2025. -Vol.5(3): 38. DOI
\href{https://doi.org/10.3390/mining5030038}{10.3390/mining5030038}.

5. Potakey N. E., Ortiz J. M., A review of grade control methods in open
cast mining // Predictive Geometallurgy and Geostatistics Lab, Queen's
University, Annual Report. -2022. -Vol.04. - P.59-70.

6. Rogiznyj V.F., Hromov V.M., Karpuhina M.V. Tehnologii selektivnoj
vyemki malomoshhnyh rudnyh tel s primeneniem malogabaritnogo samohodnogo
oborudovanija // Gornaja promyshlennost'. - 2020. - № 1.
- S.34-41. ISSN 1609-9192, 2587-9138. {[}in Russian{]}

7. Zheljabovskij Ju.G. Sistema podzemnoj razrabotki s selektivnoj
vyemkoj rudy // Zoloto i tehnologii. -2020. - №4 (46).- S.36-40. {[}in
Russian{]}

8. Peukert D., Xu C., Dowd P. A Review of Sensor-Based Sorting in Mineral
Processing: The Potential Benefits of Sensor Fusion // Minerals. -2022.
-Vol.12 (11): 1364. DOI 10.3390/min12111364.

9. Owais A. Improving Sustainability and Efficiency in Ore Sorting with
Sensors // AZoMining. -2023.

10. dos Santos E.G., de Brum I.A.S., Ambrós W.M. Sensor-based Sorting
using De-XRT Sensor Applied to a Greenfield Copper Ore Project in
Southern Brazil // IgMin Research a BioMed \& Engineering. -2025. -Vol
3(4). - P.201-205. DOI 10.61927/igmin299.

11. Xiang J., Chen J., Zhang A., Zhao X., Zhuo S., Yang S.
Multi-Objective Ore Blending Optimization for Polymetallic Open-Pit
Mines Based on Improved Matter - Element Extension Model and NSGA-II //
Mathematics. -2025. -Vol.13 (11):1843.
\href{https://doi.org/10.3390/math13111843}{DOI 10.3390/math13111843}.

12. Luo J., Gu Q., Chen L., Li X., Li P. Multi-objective optimization
for ore blending schemes in the open-pit phosphate mine using an
improved NSGA-II algorithm // Green and Smart Mining Engineering. -2025.
-Vol.2 (1). -P.42-56.
\href{https://doi.org/10.1016/j.gsme.2024.12.004}{DOI
10.1016/j.gsme.2024.12.004}.

13. Maniteja M., Samanta G., Gebretsadik A., Tsae N.B., Rai S.S., Fissha
Y., Okada N., Kawamura Y. Advancing Iron Ore Grade Estimation: A
Comparative Study of Machine Learning and Ordinary Kriging // Minerals.
-2025. -Vol.15(2):131. \href{https://doi.org/10.3390/min15020131}{DOI
10.3390/min15020131}

14. Thiruchittampalam S., Banerjee B. P., Glenn N. F., Raval S. A
systematic review of machine learning-based remote sensing data analysis
for geological and mined materials characterization//European Journal of
Remote Sensing. -2025. -Vol.58 (1): 2524622.
\href{https://doi.org/10.1080/22797254.2025.2524622}{DOI
\href{https://doi.org/10.1080/22797254.2025.2524622}{10.1080/22797254.2025.2524622}.}

15. Marquina-Araujo J., Cotrina-Teatino M., Mamani-Quispe J.,
Noriega-Vidal E., Vega-Gonzales J., Cruz-Galvez J. Copper Ore Grade
Prediction using Machine Learning Techniques in a Copper Deposit //
Journal of Mining and Environment. -2024. -Vol.15 (3). -P.1011-1027.
DOI 10.22044/jme.2024.14032.2617.

16. Prior Á., Jorg B. Mueller U. Resource and Grade Control Model
Updating for Underground Mining Production Settings //Mathematical
Geosciences. -2021. -Vol.53. P.757-779. DOI 10.1007/s11004-020-09881-2.

17. Pytalev I.A., Domozhirov D.V., Simakov D.B., Borisenko E.V.
Upravlenie kachestvom mineral' nogo
syr' ja putem obosnovanija tehnologii i parametrov
podgotovki k vyemke porod prirodnyh massivov pri otkrytoj geotehnologii
// Izvestija Tul' skogo gosudarstvennogo universiteta.
Nauki o zemle. -2023. -№ 4. -S.472-483. {[}in Russian{]}

18. Kantemirov V.D., Jakovlev A.M., Titov R.S., Timohin A.V.
Sovershenstvovanie metodov rudopodgotovki mineral' nogo
syr' ja pri osvoenii slozhnostrukturnyh mestorozhdenij //
Gornaja Promyshlennost'. -2022. -№1S. S.63-70. DOI
10.30686/1609-9192-2022-1S-63-70. {[}in Russian{]}

19. Mal' cev E. N., Kompleksnaja sistema upravlenija
kachestvom rudy v uslovijah real' nogo
gornodobyvajushhego predprijatija // Zolotodobycha dlja professionalov:
specialistov, rukovoditelej, investorov. -2018. URL:
https://zolotodb.ru/article/11888/.{[}in Russian{]}

20. Trubeckoj K. N., Zaharov V. N., Kaplunov D. R.,
Ryl' nikova M. V. Jeffektivnye tehnologii
ispol' zovanija tehnogennyh georesursov - osnova
jekologicheskoj bezopasnosti osvoenija nedr // Gornyj zhurnal. -2016. -№
5. - S.34-40. DOI 10.17580/gzh.2016.05.03. {[}in Russian{]}

21. Trubeckoj K.N. Razvitie resursosberegajushhih i
resursovosproizvodjashhih geotehnologij kompleksnogo osvoenija
mestorozhdenij poleznyh iskopaemyh // IPKON RAN. -2014. -196 c. ISBN
978-5-91177-082-2. {[}in Russian{]}

22. Kaplunov D.R., Kompleksnoe osvoenie nedr - osnovnoe napravlenie
proektirovanija razrabotki rudnyh mestorozhdenij // Gornyj
informacionno-analiticheskij bjulleten'{}
(nauchno-tehnicheskij zhurnal). -2014, -№ 1. -S.347-357. {[}in
Russian{]}

23. Fedjanin S. N. O vozmozhnosti razdelenija porod i rud
mestorozhdenija Muruntau rentgenoradiometricheskim sposobom // Sbornik
nauchno-tehnicheskih statej, Tashkent, Fan. - 1997. - S.135-141. {[}in
Russian{]}

24. Eremin A.M. Nauchnoe obosnovanie vozmozhnosti
predvaritel' nogo obogashhenija
zolotosul' fidnyh rud i razrabotka tehnologii
rentgenoradiometricheskoj sortirovki (na primere mestorozhdenija
Kokpatas): dis. kand. tehn. nauk: 25.00.22./ VNIPI promtehnologii,
Moskva. - 2010.138 c. {[}in Russian{]}

25. Ryl' nikova M. V., Shvabenland E.E. Osobennosti
upravlenija kachestvom rudnoj massy pri razrabotke slozhnostrukturnyh
mestorozhdenij apatitovyh rud s primeneniem kombajnovoj vyemki //
Racional' noe osvoenie nedr. - 2019. -№ 2(3). - S.80-86.
{[}in Russian{]}

\emph{{\bfseries Авторлар туралы мәліметтер}}

%% \begin{longtable}[]{@{}
%%   >{\raggedright\arraybackslash}p{(\linewidth - 0\tabcolsep) * \real{1.0000}}@{}}
%% \toprule\noalign{}
%% \begin{minipage}[b]{\linewidth}\raggedright
%% Ибырханов Т.С. - докторант, КеАҚ «Қ.И. Сәтбаев атындағы Қазақ ұлттық
%% техникалық зерттеу университеті», Алматы, Қазақстан, e-mail:
%% ibir.tem@mail.ru;
%% \end{minipage} \\
%% \midrule\noalign{}
%% \endhead
%% \bottomrule\noalign{}
%% \endlastfoot
%% Ибырханова А.И. - докторант, Әбілқас Сағынов атындағы Қарағанды
%% техникалық университеті, Қарағанды, Қазақстан,
%% e-mail:~i.altynay33@gmail.com; \\
%% \end{longtable}

\emph{{\bfseries Information about the authors}}

%% \begin{longtable}[]{@{}
%%   >{\raggedright\arraybackslash}p{(\linewidth - 0\tabcolsep) * \real{1.0000}}@{}}
%% \toprule\noalign{}
%% \begin{minipage}[b]{\linewidth}\raggedright
%% Ibyrkhanov T.S. - doctoral student, the NJSC «Kazakh National Research
%% Technical University named after K.I. Satpayev», Almaty, Kazakhstan,
%% e-mail: ibir.tem@mail.ru;
%% \end{minipage} \\
%% \midrule\noalign{}
%% \endhead
%% \bottomrule\noalign{}
%% \endlastfoot
%% Ibyrkhanova A.I.- doctoral student, Abylkas Saginov Karaganda Technical
%% University, Karaganda, Kazakhstan,
%% e-mail:~i.altynay33@gmail.com. \\
%% \end{longtable}
