\id{ҒТАМР 52.47.17}{}

\begin{header}
\swa{}{АНИЗОТРОПИЯ МЕН ҚАБАТТЫҢ ДЕФОРМАЦИЯЛАНУЫ ЖАҒДАЙЫНДАҒЫ КӨЛДЕНЕҢ ҰҢҒЫМАНЫҢ ДЕБИТІ}

М.Д. Бисенгалиев\envelope,
Ж.К. Зайдемова,
Ж.Б. Шаяхметова,
А.С. Каримова,
Г.Е.~Имангалиева,
Ж.У.~Икласова
\end{header}

\begin{affil}
С.Өтебаев атындағы Атырау мұнай және газ университеті, Атырау, Қазахстан

\corrauthor{Корреспондент-автор: e-mail: maks\_bisengali@mail.ru}
\end{affil}

Мақалада деформацияланатын коллекторы бар шекті изотропты айналмалы
қабатта орналасқан көлденең ұңғымаға стационарлық ағынның аналитикалық
және сандық талдауы ұсынылған. Өткізгіштіктің өзгеруіне жыныстардың
деформация әсерін ескеретін модель жасалды, бұл ұңғыма ағынының дебитін
нақтырақ бағалауға мүмкіндік берді Көлденең ұңғыма оқпанының
ұзындығының, оның шатыр мен түбіне қатысты орналасуының, қабат
қалыңдығың, сондай-ақ ұңғымалар батареясының жұмысындағы
интерференцияның әсері зерттелді.

Оқпан ұзындығының қабат қалыңдығына белгілі бір қатынасы кезінде
көлденең ұңғымаға ағын тік көрсеткіштерден асып түсетіні анықталды.
Оңтайлы әдіс -- бұл ағынға кедергі минималды болатын қабаттың
симметриялы ашылуы.

Жыныстардың түріне байланысты коллектордың деформациялануы дебитті
21,6\%-ға дейін төмендете алатыны көрсетілген. Алынған тәуелділіктер мен
графоталдау арақатынастары көлденең ұңғымалардың жобалық параметрлерін
оңтайландыру және пайдаланудың әртүрлі кезеңдерінде кен орындарын
игеруге техникалық-экономикалық бағалау жүргізу үшін пайдаланылуы
мүмкін.

{\bfseries Түйінді сөздер:} көлденең ұңғыма, деформацияланатын коллектор,
сұйықтық ағыны, өткізгіштік, гидродинамикалық модель, интерференция,
параметрлерді оңтайландыру.

\begin{header}
ДЕБИТ ГОРИЗОНТАЛЬНОЙ СКВАЖИНЫ С УЧЁТОМ АНИЗОТРОПИИ И ДЕФОРМИРУЕМОСТИ ПЛАСТА

М.Д. Бисенгалиев\envelope,
Ж.К. Зайдемова,
Ж.Б. Шаяхметова,
А.С. Каримова,
Г.Е.~Имангалиева,
Ж.У.~Икласова
\end{header}

\begin{affil}
Атырауский университет нефти и газа им. Сафи Утебаева, Атырау, Казахстан,

e-mail: maks\_bisengali@mail.ru
\end{affil}

В статье представлен аналитический и численный анализ стационарного
притока к горизонтальной скважине, расположенной в конечном изотропном
круговом пласте с деформируемым коллектором. Разработана модель,
учитывающая влияние деформации породы на изменение проницаемости, что
позволило получить более реалистичные оценки дебита скважин. Исследовано
влияние длины горизонтального ствола, его положения относительно кровли
и подошвы, мощности пласта, а также интерференции при работе батареи
скважин.У становлено, что при определённом соотношении длины ствола к
мощности пласта приток к горизонтальной скважине превышает показатели
вертикальной. Оптимальным является симметричное вскрытие пласта, при
котором сопротивление притоку минимально.

Показано, что деформируемость коллектора способна снижать дебит до
21,6\% в зависимости от типа породы. Полученные зависимости и
графоаналитические соотношения могут использоваться для
технико-экономической оценки и оптимизации параметров горизонтальных
скважин при проектировании и эксплуатации месторождений нефти и газа.

{\bfseries Ключевые слова:} горизонтальная скважина, деформируемый
коллектор, приток флюида, проницаемость, гидродинамическая модель,
интерференция, оптимизация параметров.

\begin{header}
FLOW RATE OF A HORIZONTAL WELL TAKING INTO ACCOUNT ANISOTROPY AND DEFORMABILITY OF THE FORMATION

M.D. Bissengaliyev\envelope,
Zh.K. Zaidemova,
Zh.B. Shayakhmetova,
A.S. Karimova,
G.T.~Imangalieva,
Zh.U.~Iklassova
\end{header}

\begin{affil}
Safi Utebaev Atyrau University of Oil and Gas, Atyrau, Kazakhstan,

e-mail: maks\_bisengali@mail.ru
\end{affil}

The article presents an analytical and numerical analysis of
steady-state inflow to a horizontal well located in a finite isotropic
circular formation with a deformable reservoir. A model has been
developed that takes into account the effect of rock deformation on
permeability changes, which has made it possible to obtain more
realistic estimates of well flow rates. The influence of the length of
the horizontal wellbore, its position relative to the roof and floor,
the thickness of the formation, and interference during the operation of
a well battery are investigated.

It has been established that at a certain ratio of the length of the
wellbore to the thickness of the formation, the inflow to a horizontal
well exceeds that of a vertical well. Symmetrical opening of the
formation is optimal, as it minimizes resistance to inflow.

It has been shown that the deformability of the reservoir can reduce the
flow rate by up to 21.6\%, depending on the type of rock. The obtained
dependencies and graphical-analytical ratios can be used for technical
and economic evaluation and optimization of horizontal well parameters
in the design and operation of oil and gas fields.

{\bfseries Keywords:} horizontal well, deformable reservoir, fluid inflow,
permeability, hydrodynamic model, interference, parameter optimization.

{\bfseries Кіріспе.} Әлемдік ғылыми басылымдарда көлденең ұңғымалардың
өнімділігін анықтауға арналған формулалар белгілі, олар {[}1- 4{]} және
т.б. еңбектерде ұсынылған. Бұл формулалардың барлығы сұйықтықты сүзудің
тұрақты режимі үшін жазық есептерді шешуде және қабаттың
деформациялануын есепке алмай, тігінен (к\tsb{z}) және
көлденеңінен (к\tsb{r}) өткізгіштік теңдігі жағдайында
шығарылды. Көптеген зерттеушілер {[}5-9{]} қабат бойындағы бойлық
ұңғымаларға қолданған кезде қабаттың анизотропиясын ескеру міндетті деп
санайды.

Біртекті анизотропты қабаттар жағдайлары үшін белгілі ағын
формулаларының модификацияларын қабаттың тиімді өткізгіштік
коэффициентін енгізу арқылы алуға
болады\fig{g3/image36}{}=\fig{g3/image37}{}
{[}5,7,8{]}.

Дөңгелек біртекті анизотропты деформацияланатын қабатта көлденең
ұңғымаға ағынды анықтау кезінде, анизотропия мен қабаттың
деформациялануы көлденең ұңғыма ағынына жеке және біріктірілген әсерінің
сандық бағасы беріледі.

Егер
\fig{g3/image38}{}және\fig{g3/image39}{}-
бастапқы қабат қысымындағы қабаттың өткізгіштігінің мәндері радиалды
және тік бағытта болса, және қысымға байланысты өткізгіштіктің өзгеру
заңы әрбір координата үшін бірдей болса, біз мынаны аламыз:

\fig{g3/image40}{}
(1)

мұндағы,
\fig{g3/image41}{};
\fig{g3/image42}{};\fig{g3/image43}{}--
өткізгіштік, бұл
\fig{g3/image44}{}
және
\fig{g3/image45}{}қатынасының
ұңғымаға түсетін суға әсерін өлшемсіз бағалауға мүмкіндік береді;
\fig{g3/image46}{}
және
\fig{g3/image47}{}--
сұйықтықтың тығыздығы мен тұтқырлығы;
\fig{g3/image48}{}
және
\fig{g3/image49}{} --
қабат пен ұңғыма контурының радиустары;
\fig{g3/image50}{}
және
\fig{g3/image51}{} --
қабат контурындағы және ұңғыма түбіндегі қысым функциясының мәндері;
\fig{g3/image52}{} --
қабат қалыңдығы;
\fig{g3/image53}{}--
көлденең оқпанның ұзындығы;
\fig{g3/image54}{} --
қабаттың контурындағы қысым
\fig{g3/image50}{}
кезіндегі өткізгіштік;
\fig{g3/image55}{}--
ұңғыма оқпанынан қабаттың түбіне дейінгі
қашықтық;\fig{g3/image56}{}
және
\fig{g3/image57}{}--
жұмыстан белгілі қабат жыныстарының деформациясы процесінде
өткізгіштіктің өзгеруін сипаттайтын коэффициенттер {[}5{]}.

{\bfseries Материалдар мен әдістер.} Қабат жыныстарының анизотропиясы мен
деформациялануының ағынға әсерін бағалау үшін (1) формула бойынша
есептеулер орташа және нашар сұрыпталған құмтастардан, жұқа және ұсақ
түйіршікті, далалық шпатты мен кварцты (кварц 25-60\%) алевриттермен,
орташа кеуекті (10-20\%) саз мөлшері 20-35\% болатын, қалыңдығы 7,5 м
және 60 м болатын қабаттардағы көлденең оқпанның ұзындығы 3000 м дейін
болатын ұңғымалар үшін жүргізілді, бұл үшін
\fig{g3/image56}{}=
0,958899 и
\fig{g3/image57}{}=0,0096465.

Есептеу нәтижелерін талдау қабаттың өткізгіштік анизотропиясы көлденең
ұңғымалардың өнімділігіне теріс және оң әсер етуі мүмкін екенін
көрсетеді.

Изотропты қабаттың өткізгіштігіне тең көлденең
өткізгіштігі\fig{g3/image58}{}бар
қабатта көлденең ұңғыманың дебиті
\fig{g3/image59}{}кезінде
біртекті қабатта бірдей ұңғыманың дебитінен үлкен және
\fig{g3/image60}{}кезінде
аз болады (1-кесте: қабаттың қуаты h=7,5 м және 1-суреттегі 1-5 қисықтар
h=60 м кезінде).

{\bfseries 1-кесте. Қалыңдығы h=7,5 м болатын қабаттағы тік өткізгіштіктің
көлденең ұңғымаға қатынасының әртүрлі мәндері кезінде көлденең оқпанның
ұзындығына байланысты ұңғыма өнімділігінің өзгеруі}

%% \begin{longtable}[]{@{}
%%   >{\raggedleft\arraybackslash}p{(\linewidth - 10\tabcolsep) * \real{0.1905}}
%%   >{\centering\arraybackslash}p{(\linewidth - 10\tabcolsep) * \real{0.1300}}
%%   >{\centering\arraybackslash}p{(\linewidth - 10\tabcolsep) * \real{0.1633}}
%%   >{\centering\arraybackslash}p{(\linewidth - 10\tabcolsep) * \real{0.1635}}
%%   >{\centering\arraybackslash}p{(\linewidth - 10\tabcolsep) * \real{0.1633}}
%%   >{\centering\arraybackslash}p{(\linewidth - 10\tabcolsep) * \real{0.1895}}@{}}
%% \toprule\noalign{}
%% \endhead
%% \bottomrule\noalign{}
%% \endlastfoot
%% Ұңғыма
%% 
%% оқпаны ұзындығы\emph{,} \emph{l}, м & 0,2 & 0,5 & 1,0 & 2,0 & 5,0 \\
%% 50 & 35,35 & 38,39 & 39,94 & 40,87 & 41,88 \\
%% 100 & 43,95 & 46,26 & 47,37 & 48,11 & 48,7 \\
%% 300 & 61,73 & 63,19 & 63,86 & 64,22 & 64,67 \\
%% 500 & 73,37 & 74,8 & 75,37 & 75,7 & 76,04 \\
%% 1000 & 97,52 & 98,96 & 99,09 & 99,31 & 99,66 \\
%% 1500 & 119,57 & 120,68 & 121,14 & 121,37 & 121,71 \\
%% 2000 & 142,06 & 143,19 & 143,71 & 144,06 & 144,31 \\
%% 2500 & 166,14 & 167,38 & 167,95 & 167,31 & 168,6 \\
%% 3000 & 192,7 & 194,09 & 194,72 & 195,12 & 195,46 \\
%% \end{longtable}

\fig{g3/image62}{}

{\bfseries 1-сурет. Ұңғыманың өнімділігі көлденең оқпан ұзындығының
функциясы ретінде әртүрлі тік өткізгіштік мәндерінде:}

\emph{1-5 -- сәйкесінше χ=0.2; 0.5; 1,0; 2.0; 5,0 и h=60 м; 6 -- χ=2;
h=60 м;}

\emph{7 -- χ=5; h=60 м; жыныстардың деформациясын ескермегенде}

Кез келген қалыңдықтағы қабаттарда анизотропияның ұңғыма өнімділігіне
әсері бірдей заңдылыққа бағынады және тік және көлденең
өткізгіштіктердің арақатынасына қарамастан, көлденең ұңғыма оқпанының
ұзындығының артуымен азаяды (2-сурет).

\fig{g3/image63}{}

{\bfseries 2-сурет. Анизотропты және изотропты қабаттардағы дебиттердің
көлденең ұңғыма оқпанының ұзындығына қатынасы:}

\begin{quote}
\emph{1; 2 -- χ=0.2; 1а; 2а -- χ=5; -\/-\/-\/-\/- қабат қалыңдығы h=60
м; h=7.5 м}
\end{quote}

Алайда, анизотропияның ағынға әсер ету дәрежесі қабаттың қалыңдығына
айтарлықтай байланысты. Жұқа қабаттарда бұл әсер соншалықты маңызды
емес, сондықтан оны іс жүзінде ескермеу мүмкін. Осылайша, қалыңдығы 7,5
м болатын қабатта, есептеулер үшін қабылданған шарттарда, 0,2-ден 5-ке
дейінгі диапазонда, көлденең оқпан ұзындығы 50 м болатын ұңғыманың
дебиті изотропты қабаттағы ұқсас ұңғыманың дебитімен салыстырғанда тек
1,49-4,86\%-ға өзгереді, ал көлденең оқпан ұзындығы 3000 м-ге дейін
ұлғаюымен ол одан да көп 0,04-

0,38\% мәніне дейін төмендейді (1 және 1a қисықтары).

Үлкен қалыңдықтағы қабаттарда, әсіресе көлденең оқпанның ұзындығының аз
мәндерінде ұңғыманың өнімділігі анизотропияға тәуелді, сондықтан оның
әсерін елемеуге болмайды. Қабат қалыңдығы 60 м және көлденең оқпанның
ұзындығы 50 м болғанда, ұңғыма өнімділігі
\fig{g3/image64}{}
мәнге байланысты 45,0-47,0\%-ға өзгереді. Көлденең ұңғыма оқпанының
ұзындығы 3000 м-ге дейін ұлғаюымен бұл өзгеріс 4,35--7,30\%-ға дейін
төмендейді, абсолютті мәнде ұңғыма ағынының дебиті өзгеруі үлкен болады
(2 және 2a қисықтары).

Тік өткізгіштігі төмен қабаттардағы көлденең ұңғыманың дебитінің
төмендеуі көлденең оқпан ұзындығының ұлғаюына байланысты болуы мүмкін.
Осылайша, қалыңдығы бірдей 60 м және
\fig{g3/image64}{}=0,2
болатын біртекті және анизотропты қабаттардағы ұңғымалардың дебиті, егер
анизотропты қабаттағы көлденең ұңғыма оқпаны изотропты қабаттағы
ұзындығы 1500 м ұңғымадан 340 м ұзын болса, тең болады.

Сондай-ақ, гидравликалық жаруды қолдана отырып, ұңғыма оқпанына
перпендикуляр жарықтар жасау арқылы тік өткізгіштігі төмен қабаттардағы
көлденең ұңғымалардың дебитін арттыруға болады.

Сондай-ақ, біз қарастырған ұңғыма оқпанының дөңгелек қабатта орналасу
сұлбасында көлденең өткізгіштіктің төмендеуі көлденең ұңғыманың
өнімділігіне тік өткізгіштіктің ұқсас өзгеруіне қарағанда әлдеқайда көп
әсер ететінін атап өткен жөн. Біркелкі қабаттағы ұңғыманың өнімділік
кестесін сәйкестендіру арқылы бұған көз жеткізу оңай (1-суреттегі 3
қисық;
\fig{g3/image65}{})
анизотропты қабаттардағы ұқсас графиктермен
\fig{g3/image65}{}және\fig{g3/image66}{}
(2 қисық)
және\fig{g3/image67}{};
\fig{g3/image68}{};
(6 қисық).

Жыныстардың деформациялануының көлденең ұңғыманың өнімділігіне әсерін
бағалау үшін деформацияланбайтын жыныстармен қабаттарда бұрғыланған
әртүрлі\fig{g3/image64}{}мәндері
бар ұңғымалар үшін жоғарыда сипатталғандарға ұқсас есептеулер
жүргізілді. Алынған нәтижелерді 1-суретте көрсету үшін (7-қисық)
қалыңдығы 60 м,
\fig{g3/image65}{}
және
\fig{g3/image64}{}=5
деформацияланбайтын қабаттағы ұңғымаға тәуелділік
\fig{g3/image69}{}
салынған. Ұқсас деформацияланатын қабат үшін 7-қисықты 5-қисықпен
салыстыру көлденең ұңғымалардың өнімділігін есептеу қабаттың
жыныстарының деформациялануын ескермей, жоғары нәтиже беретіндігін
көрсетеді.

Ақырында, көлденең және жетілген тік ұңғымалардың дебиттері
салыстырылды, олар ұқсас қабаттардың барлық қалыңдығын көрсетеді.
Зерттеу барысында көлденең ұңғымалар үшін өнімділік коэффициенттерінің
\fig{g3/image70}{}қатынасында
кез-келген тік және көлденең өткізгіштік коэффициенттері жұқа қабаттарда
әлдеқайда жоғары екендігі белгілі.

Қарапайым мысалда анизотропия тензорының осьтері
координаталық\fig{g3/image71}{},
\fig{g3/image72}{}
және
\fig{g3/image73}{}
осьтерге сәйкес келеді делік (3-сурет). Сонда мәселе келесі теңдеуді
шешу үшін:

\fig{g3/image74}{}, (2)

мұндағы,
\fig{g3/image75}{} --
\fig{g3/image71}{},
\fig{g3/image72}{}
және
\fig{g3/image73}{}осьтер
бағытына сәйкес өткізгіштік мәндері бастапқы қабаттық қысым кезінде;
\fig{g3/image76}{} --
Дирак функциясы;
\fig{g3/image77}{} --
Хевисайдтың бірлік функциясы.

\fig{g3/image78}{}

{\bfseries 3-сурет. Көлденең сызықтық ағыны бар қабат элементі}

Соңғы шегінде Фурье түрлендіру косинусын екі рет қолдану және өңдеу
формулалары келесідей: Есепті шешу үшін ұсынылған алгоритмді қолдана
отырып, кезіндегі сұйықтық
ағынына\fig{g3/image79}{},
\fig{g3/image80}{}
қабат өткізгіштігінің анизотропия мәндерінің өзгеруінің әсерін анықтау
үшін көптүрлі есептеулер жүргізілді. Тікбұрышты пішінді қабат
элементіндегі көлденең ұңғыманың орналасуының әртүрлі нұсқалары үшін
зерттеулер жүргізілді.

{\bfseries Нәтежелер жене талқылау.} Біз әртүрлі факторлардың сүзу
кедергілеріне әсер етуінің салыстырмалы тәуелділіктерін берілген
параметр үшін ерікті кедергінің ең кішіге қатынасы түрінде ұсындық, бұл
координатасы
\fig{g3/image81}{}қабат
элементіне көлденең ұңғыма сүзгісін орналастыру кезіндегі фильтрация
кедергісінің бірдей кедергіге қатынасын бейнелейді, бірақ сүзгінің
шатырға және негізге қатысты симметриялы орналасуымен, яғни
\fig{g3/image82}{}кезінде,
яғни, сүзгі
\fig{g3/image83}{}
координатасынан басталғанда, сүзгінің ортасы қабат элементінің тік
симметрия осіне сәйкес келген кезде сүзгіні симметриялы орналастыру. Бұл
жағдайда нұсқалар үшін өткізгіштік коэффициенті келесідей болды:
\fig{g3/image84}{}.
Осылайша, қарастырылған нұсқалардың үшіншісі біртекті қабатты, біріншісі
және екіншісі тік өткізгіштігі нашарлаған нұсқаларды, ал төртіншісі
жақсартылған тік өткізгіштікті білдіреді. Көріп отырғанымыздай, қабатты
түбінде ашу көлденең ұңғымаға сұйықтық ағынына сүзілу кедергісін
айтарлықтай арттырады. Сүзілуге кедергінің
абсолютті өсуі ашылу дәрежесіне және тік өткізгіштікке айтарлықтай
байланысты. Қабаттың ашылу дәрежесі неғұрлым жоғары болса, сүзгінің тік
жазықтықтағы орналасуы мен тік өткізгіштігінің әсері соғұрлым аз болады.
Сонымен, қарастырылған нұсқалардағы
\fig{g3/image85}{}
және
\fig{g3/image86}{}
максималды айырмашылықтар сәйкесінше 60\% және 18,5\%-дан асты. Тік
өткізгіштіктің әсері біртекті қабатпен салыстырғанда оңай орнатылады.
Тігінен өткізгіштіктің нашарлауы әрқашан сүзу кедергісінің өсуіне оның
біркелкі өсуіне қарағанда көбірек әсер ететіні анықталды.

Бұл симметриялы ашылу кезіндегі
\fig{g3/image87}{}
кедергі минималды болуына және оның төмендеуі асимметриялық ашылу
кезіндегі төмендеу жылдамдығынан асып түсуіне байланысты. Сонымен қатар,
бірдей тәуелділіктер анықталды, бірақ өткізгіштіктің әртүрлі мәндері
үшін. Бұл жағдайда нұсқалардың
өткізгіштіктері\fig{g3/image88}{}бір-бірімен
келесідей байланысты болды:
\fig{g3/image89}{}.
Зерттеу барысында қабаттың өткізгіштігінің нашарлауы және бұл жағдайда
оның біркелкі өсуіне қарағанда айтарлықтай үлкен әсер етеді. Жалпы
алғанда, бұл жағдайда бұрын қарастырылған нұсқаның сапалық сипаттамалары
сақталады. Сонымен қатар, зерттелген көлденең ұңғымаға ағынға
салыстырмалы кедергіге
\fig{g3/image90}{}
сандық әсері
\fig{g3/image91}{}-ға
қарағанда айтарлықтай төмен.

Жүргізілген жұмыс барысында көлденең ұңғымаға ағын кедергісі басқа тең
жағдайларда біртекті қабаттың кедергісіне берілген кедергінің қатынасы
\fig{g3/image91}{}
ретінде салыстырмалы түрде ұсынылатыны анықталды. Көріп отырғаныңыздай,
\fig{g3/image91}{}әсер
айтарлықтай және неғұрлым көп болса, қабаттың салыстырмалы соғұрлым
ашылуы аз болады
(\fig{g3/image92}{}).
Қарастырылып отырған процеске қабат қимасындағы көлденең ұңғының
орналасуы да
\fig{g3/image93}{}.айтарлықтай
әсер етеді . Осылайша, қарастырылып отырған мысалдағы минималды ашылу
\fig{g3/image94}{}және
максималды анизотропия
\fig{g3/image95}{}
болғанда,
\fig{g3/image71}{}
осі бойымен өткізгіштіктің нашарлауы
\fig{g3/image96}{}
болатын біртекті қабатпен салыстырғанда сүзілу кедергісін 93\%-ға, ал
\fig{g3/image97}{}болатын
қабатпен салыстырғанда 51\%-ға арттырады.
\fig{g3/image98}{}және
сол алдыңғы анизотропия кезінде көлденең ұңғымаға сұйықтық ағынына
кедергінің байқалған жоғарылауы сәйкесінше 47\% және 20\% құрады.

Берілген қисықтарды талдау
\fig{g3/image71}{}
бойындағы анизотропияның сүзілу кедергісіне айтарлықтай әсерін
көрсетеді. Сонымен,
\fig{g3/image99}{}
және
\fig{g3/image100}{}
кезінде сүзілу кедергісі жоғары болады:
\fig{g3/image95}{},
z\tsb{1}=0 кезінде 68.7\%-ға,
\fig{g3/image101}{}кезінде
60,5\%-ға жоғары. Бұрынғыдай, өткізгіштіктің төмендеуінің зерттелетін
мөлшеріне оның біркелкі өсуіне қарағанда біршама үлкен әсер бар. Бұл
жағдайда әсер ашылу дәрежесіне де
\fig{g3/image92}{}
тән.

Біріншісі
\fig{g3/image102}{}аралығын
қамтиды, мұндағы өзгеріс кедергінің күрт өзгеруіне әкеледі. Екінші бөлім
\fig{g3/image103}{}аралығын
қамтиды, мұнда әсер айтарлықтай. Соңында,
\fig{g3/image104}{}аралығында,
ашылу\fig{g3/image71}{}осі
бойымен анизотропты өткізгіштігі бар қабаттағы көлденең ұңғымаға
сұйықтық ағынының мөлшеріне аз әсер етеді.

Әрі қарай, жоғарыда қарастырылған процесс зерттеледі, бірақ ось бойымен
өткізгіштігі анизотропты
``\fig{g3/image72}{}''қабатта.
Мұнда сүзу кедергісі салыстырмалы түрде,
\fig{g3/image105}{}кезіндегі
кедергінің біртекті түзілімнің кедергісіне қатынасы ретінде беріледі
(\fig{g3/image106}{}).
Зерттеу барысында. бір сипатты жағдай, қабаттың толық ашылуымен,
\fig{g3/image107}{}қарамастан,
\fig{g3/image105}{}
өзгергіштігінің әсері жоғалады. Жалпы, ізденіс кезінде деректерін
салыстыра отырып,
\fig{g3/image105}{}қарастырылып
отырған процеске әсері өткізгіштігі бойынша анизотропияның басқа
түрлеріне қарағанда әлдеқайда аз екенін көруге болады. Жүргізілген
зерттеулер барысында қабаттың оның бүйірлік шекараларына қатысты
симметриялы ашылуы бар көлденең ұңғыманың жұмысына анизотропияның әсері
зерттелді. Бұл мақалада сол тапсырма қарастырылады, бірақ қабаттың
асимметриялық ашылуы үшін. Бұл зерттеуде асимметрияның шекті дәрежесін
талданды, онда сүзгі ағынының басталуы координаттар жүйесінің
басталуымен сәйкес келеді, яғни x=0x = 0x=0 координаты бар қабат
элементінің бүйір беті. Алынған нәтижелерді симметриялы ашуға арналған
есептеу деректерімен салыстыру симметриялы емес ашу кезінде
анизотропияның фильтрация процесіне әсері симметриялық жағдайға
қарағанда әлдеқайда күштірек көрінетінін көрсетті.

Жалпы,
\fig{g3/image108}{}
әсері, бұрынғыдай, басқа екі бағыттағы анизотропияларға қарағанда аз.

Жүргізілген зерттеулер барысында қабаттың ашылуының симметрия
дәрежесінің көлденең ұңғыма ағынының сүзілу кедергісіне әсері анықталды.
Алынған тәуелділіктер симметриялық кедергіге асимметриялық ашу кезіндегі
қарсылықтың қатынасы ретінде берілген. Осылайша, есептелген мәндердің
әрқайсысы симметриялық жағдаймен салыстырғанда асимметриялық ашу кезінде
көлденең ұңғымаға ағын гидродинамикалық кедергісінің жоғарылау дәрежесін
көрсетеді. Жалпы алғанда, анизотропияның әртүрлі дәрежелерінде,
сондай-ақ осы параметрлердің шекті мәндерінде қарастырылған шарттарда
асимметриялық ашылу әсері +36\%-дан -10\% аралығында өзгерді. Оң мәндер
көлденең көлденең ұңғыма оқпанының қабат шатырына қарай жылжыған кезде
кедергі жоғарылауын сипаттайды, ал теріс мәндер түбіне қарай жылжу
кезінде оның төмендеуін сипаттайды. Бұл ағынды желілерінің қайта
бөлінуіне және қабаттың қабырғаларына жақын қысым градиентінің өзгеруіне
байланысты, бұл дренаж аймағының пішініне және ұңғыманың тиімділігіне
айтарлықтай әсер етеді.

{\bfseries Қорытынды:} Анизотропты өткізгіштігі бар жолақ тәрізді қабатқа
енген көлденең ұңғыманың өнімділігін сұйықтықты үш өлшемді сүзгілеу
кезінде оның өту дәрежесін ескере отырып анықтау үшін есептеу әдісі
ұсынылады. Көлденең ұңғыманың өнімділігі ашу дәрежесі мен симметриясына,
сондай-ақ қабат өткізгіштігінің анизотропиясына айтарлықтай тәуелді
екені анықталды. Ақырғы изотропты және анизотропты деформацияланатын
қабаттағы горизонталь ұңғымаға ағынның әзірленген математикалық моделі
сүзу процестерін неғұрлым дәл сипаттауды қамтамасыз ететін
гидродинамикалық және геомеханикалық факторлардың өзара әсерін ескереді.
Жыныстардың деформациялануы дебитке айтарлықтай әсер ететіні
көрсетілген: коллектордың қысылуының жоғарылауымен ұңғыманың өнімділігі
қабаттың қасиеттеріне байланысты 21,6\%-ға дейін төмендеуі мүмкін.
Көлденең ұңғыма оқпанының оңтайлы гидродинамикалық жағдайы - бұл ағынға
кедергі ең аз және ең жоғары дебит қамтамасыз етілетін қабаттың
симметриялы ашылуы. Анизотропия мен деформацияның бірлескен әсері
дебиттің өткізгіштікке сызықтық емес тәуелділігіне әкеледі, бірақ
ағынның негізгі сапалық заңдылықтары сақталады. Алынған аналитикалық
тәуелділіктер мен графоталдау қатынастары көлденең ұңғымалардың жобалық
параметрлерін оңтайландыру және пайдаланудың әртүрлі кезеңдерінде кен
орындарын игеруге техникалық-экономикалық бағалау жүргізу үшін
пайдаланылуы мүмкін.

{\bfseries Әдебиеттер}

1. Sun E., Yang W., Peng Q., Meng P., Mu S. The Productivity Model of
Horizontal Well Considering Acidification Effect in Anisotropic
Reservoirs//of Engineering and Technology.- 2020.-Vol.8(1).-P.19-32.
DOI 10.4236/wjet.2020.81003.

2. \href{https://pubmed.ncbi.nlm.nih.gov/?term=\%22Nwonodi\%20RI\%22\%5BAuthor\%5D}{Roland
I Nwonodi}. A novel model for predicting the productivity index of
horizontal/vertical wells based on Darcy' s law,
drainage radius, and flow convergence // Heliyon.
-2024.-Vil.10(13): e25073. DOI
\href{https://doi.org/10.1016/j.heliyon.2024.e25073}{10.1016/j.heliyon.2024.e25073}.

3. Xinyang Guo, Shiging Cheng, Wenpreng Bai, Dingning Cai, Zexuan Xu,
Yang Wang Productivity Prediction of Multilateral Horizontal Wells in
Low-Permeability Gas Reservoirs Under Two-Phase Flow Conditions~//
Available to Purchase. Paper presented at the 58th U.S. Rock
Mechanics/Geomechanics Symposium, Golden, Colorado, USA, June
2024. Paper
Number:~ARMA-P.0185\href{https://doi.org/10.56952/ARMA-2024-0185}{.
DOI 10.56952/ARMA-2024-0185}.

4. Корепанов С.К. Прогнозирование размещения горизантальных и наклонно
направленных скважин в зависимости от особенности геологического
строения низкопрницаемого пласта ЮС-2// Недропользование. - 2024. -
Т.24(4).- С.186-193. DOI 10.15593/2712-8008.2024.4.2.

5. Мухаметгалиев И.Д. Развитие технологий и технических средств бурения
наклонно-направленных и горизонтальных скважин\emph{//}Инфра-Инженерия.
- 2024.-108 с. ISBN 978-5-9729-1737-2.

6. Wenqi Ke et al. A Transient Productivity Prediction Model for
Horizontal Wells Coupled with Oil and Gas Two-Phase Seepage and Wellbore
Flow//Processes.-2023.-Vol.11(7).-P.1-22. DOI 10.3390/pr11112012.

7. Попов И.П.,Томилов А.А, Казанцев Г.В., Инновационные технологии
разработки нефтяных и газовых месторождений // Тюменский государственный
нефтегазовый университет // Нефтепромысловое дело. - 2015. - №7. -
С.19-22.

8. Li L., Xie M., Liu W., Dai J., Feng S., Luo D., Wang K., Gao Y.,
Huang R. A New Productivity Evaluation Method for Horizontal Wells in
Offshore Low-Permeability Reservoir Based on Modified Theoretical
Model\emph{//}Processes. -2024.- Vol.12(12):2830. DOI
10.3390/pr12122830.

9. Шаяхметова Ж.Б., Бисенгалиев М.Д., Тулегенова О.Ш., Каримова А.С.
Исследование стационарного притока к горизонтальной скважине в
деформируемом коллекторе // Нефть и газ. - 2025. - № 3(147).-
С.188-199.DOI: 10.37878/2708-0080/2025-3.19.

\subsubsection{}

\subsubsection{References}

1. Sun E., Yang W., Peng Q., Meng P., Mu S. The Productivity Model of
Horizontal Well Considering Acidification Effect in Anisotropic
Reservoirs//of Engineering and Technology.- 2020.-Vol.8(1).-P.19-32. DOI
10.4236/wjet.2020.81003.

2. \href{https://pubmed.ncbi.nlm.nih.gov/?term=\%22Nwonodi\%20RI\%22\%5BAuthor\%5D}{Roland
I Nwonodi}. A novel model for predicting the productivity index of
horizontal/vertical wells based on Darcy' s law, drainage
radius, and flow convergence // Heliyon. -2024.-Vil.10(13): e25073.
DOI
\href{https://doi.org/10.1016/j.heliyon.2024.e25073}{10.1016/j.heliyon.2024.e25073}.

3. Xinyang Guo, Shiging Cheng, Wenpreng Bai, Dingning Cai, Zexuan Xu,
Yang Wang Productivity Prediction of Multilateral Horizontal Wells in
Low-Permeability Gas Reservoirs Under Two-Phase Flow Conditions~//
Available to Purchase. Paper presented at the 58th U.S. Rock
Mechanics/Geomechanics Symposium, Golden, Colorado, USA, June 2024.Paper
Number:~ARMA-P.0185\href{https://doi.org/10.56952/ARMA-2024-0185}{. DOI
10.56952/ARMA-2024-0185}.

4. Korepanov S.K. Prognozirovanie razmeshhenija
gorizantal' nyh i naklonno napravlennyh skvazhin v
zavisimosti ot osobennosti geologicheskogo stroenija nizkoprnicaemogo
plasta JuS-2// Nedropol' zovanie. - 2024. - T.24(4).-
S.186-193. DOI 10.15593/2712-8008.2024.4.2. {[}in Russian{]}

5. Muhametgaliev I.D. Razvitie tehnologij i tehnicheskih sredstv burenija
naklonno-napravlennyh i gorizontal' nyh
skvazhin//Infra-Inzhenerija. - 2024.-108 s. ISBN 978-5-9729-1737-2.
{[}in Russian{]}

6. Wenqi Ke et al. A Transient Productivity Prediction Model for
Horizontal Wells Coupled with Oil and Gas Two-Phase Seepage and Wellbore
Flow//Processes.-2023.-Vol.11(7).-P.1-22. DOI 10.3390/pr11112012.

7. Popov I.P., Tomilov A.A., Kazancev G.V., Innovacionnye tehnologii
razrabotki neftjanyh i gazovyh mestorozhdenij // Tjumenskij
gosudarstvennyj neftegazovyj universitet // Neftepromyslovoe delo. -
2015. - №7. - S.19-22. {[}in Russian{]}

8. Li L., Xie M., Liu W., Dai J., Feng S., Luo D., Wang K., Gao Y.,
Huang R. A New Productivity Evaluation Method for Horizontal Wells in
Offshore Low-Permeability Reservoir Based on Modified Theoretical
Model\emph{//}Processes. -2024.- Vol.12(12):2830. DOI
10.3390/pr12122830.

9. Shajahmetova Zh.B., Bisengaliev M.D., Tulegenova O.Sh., Karimova A.S.
Issledovanie stacionarnogo pritoka k gorizontal' noj
skvazhine v deformiruemom kollektore // Neft'{} i gaz. -
2025. - № 3(147).- S.188-199.DOI: 10.37878/2708-0080/2025-3.19. {[}in
Russian{]}

\emph{{\bfseries Авторлар туралы мәлметтер}}

Бисенгалиев М.Д. - техника ғылымының кандидаты, профессор, С.Өтебаев
атындағы Атырау мұнай және газ университеті, Атырау, Казахстан, e-mail:
maks\_bisengali@mail.ru;

Зайдемова Ж.К. - кандидат технических наук, профессор, С.Өтебаев
атындағы Атырау мұнай және газ университеті, Атырау, Казахстан,
e-mail:b.n.m.99@list.ru;

Шаяхметова Ж.Б. - техника ғылымының кандидаты, қауымдастырылған
профессор, С.Өтебаев атындағы Атырау мұнай және газ университеті,
Атырау, Казахстан, e-mail:
zhanar6688@mail.ru;

Каримова А.С. - техника ғылымының кандидаты, қауымдастырылған профессор,
С.Өтебаев атындағы Атырау мұнай және газ университеті, Атырау,
Казахстан, e-mail:
akmaral0167@mail.ru;

Имангалиева Г.Е. - техника ғылымының кандидаты, қауымдастырылған
профессор, С.Өтебаев атындағы Атырау мұнай және газ университеті,
Атырау, Казахстан, e-mail:
gulnar-imangalieva@mail.ru;

Икласова Ж. - техника ғылымының кандидаты, ассоциированный профессор,
С.Өтебаев атындағы Атырау мұнай және газ университеті, Атырау,
Казахстан, e-mail:
janna\_ua@mail.ru.

\emph{{\bfseries Information about the authors}}

Bissengaliyev M.D. - Ph., professor, S. Ute󠀁baev Atyrau Uni󠀁versity of Oil
and Gas, Aty󠀁rau, Kaz󠀁akhstan, e-mail:
maks\_bisengali@mail.ru;

Zaidemova Zh.R. -Ph., Professor, S. Ute󠀁baev Atyrau Uni󠀁versity of Oil and
Gas, Aty󠀁rau, Kaz󠀁akhstan, e-mail:
b.n.m.99@list.ru;

Shayakhmetova Zh.B.- Ph., Associate Professor, S.Ute󠀁baev Atyrau
Uni󠀁versity of Oil and Gas, Aty󠀁rau, Kaz󠀁akhstan, e-mail:
zhanar6688@mail.ru;

Karimova A.S. - PhD, Associate Professor, S.Ute󠀁baev Atyrau
Uni󠀁versity of Oil and Gas, e-mail:
akmaral0167@mail.ru;

Imangalieva G.E.- Ph., Associate Professor, S.Ute󠀁baev Atyrau Uni󠀁versity
of Oil and Gas, Aty󠀁rau, Kaz󠀁akhstan, e-mail:
gulnar-imangalieva@mail.ru;

Iklassova Zh.U. \emph{-} Ph. Sci, Associate Professor, S.Ute󠀁baev Atyrau
Uni󠀁versity of Oil and Gas, Aty󠀁rau, Kaz󠀁akhstan, e-mail:
janna\_ua@mail.ru.
