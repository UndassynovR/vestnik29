\id{МРНТИ 61.51.37}{}

\begin{header}
\swa{}{СОВЕРШЕНСТВОВАНИЕ ТЕХНОЛОГИИ ОКИСЛЕНИЯ ТЯЖЕЛЫХ НЕФТЯНЫХ ОСТАТКОВ КАРАЖАНБАССКОЙ НЕФТИ}

\tsp{1}С.К. Буканова\envelope,
\tsp{1,2}Г.К. Шамбилов,
\tsp{1}А.И. Абилхайров,
\tsp{1}А.С. Буканова,
\tsp{1}Ф.Б. Кайрлиева,
\tsp{3}Н.У.~Нургалиев\envelope
\end{header}

\begin{affil}
\tsp{1}НАО «Атырауский университет нефти и газа имени Сафи Утебаева», Атырау,Казахстан,

\tsp{2}НАО «Атырауский университет имени Х. Досмухамедова», Атырац,Казахстан,

\tsp{3}Казахский университет технологии и бизнеса им. К.Кулажанова, Астана, Казахстан

\corrauthor{Корреспондент-автор: sauleshik81@mail.ru, nurgaliev\_nao@mail.ru}
\end{affil}

Современные тенденции развития нефтеперерабатывающей и
дорожно-строительной отраслей Казахстана требуют внедрения
энергоэффективных и экологически рациональных технологий получения
битумов. В условиях растущего спроса на качественные вяжущие материалы
особую актуальность приобретают каталитические процессы окисления,
позволяющие интенсифицировать реакции при мягких температурных режимах и
снизить нагрузку на оборудование.

Настоящая работа направлена на экспериментальное обоснование
возможности замены традиционной схемы «переокисление-разбавление»
упрощённой технологией прямого каталитического окисления тяжёлых
нефтяных остатков Каражанбасской нефти с использованием
перманганата калия (KMnO₄) в качестве активного катализатора.

Основной задачей являлось получение дорожных битумов марки БНД
70/100, соответствующих требованиям СТ РК 1373-2013, при
одновременном снижении энергозатрат и исключении стадии
компаундирования.

В исследовании использованы гудрон и затемнённая вакуумная
фракция, полученные из Каражанбасской нефти. Окисление проводилось
в лабораторном реакторе периодического действия при температурах
140-260°С с регулируемой подачей воздуха (1 л/кг∙мин). В качестве
катализатора применялась 0,5 \% калиевая соль марганцевой кислоты,
обеспечивающая ускорение окислительных процессов и формирование
асфальтосмолистой структуры битума. Отобранные пробы
анализировались по стандартам СТ РК и ГОСТ: определялись
пенетрация, температура размягчения, растяжимость и температура
хрупкости.

Установлено, что введение KMnO₄ обеспечивает устойчивое протекание
реакции окисления при пониженных температурах, что приводит к
формированию битума с высокими эксплуатационными свойствами.
Полученный продукт характеризуется пенетрацией 111 (0,1 мм, 25
°С), температурой размягчения 45 °С, дуктильностью выше 150 см и
температурой хрупкости -19 °С.

Эти показатели полностью соответствуют требованиям к битумам марки
БНД 70/100. Проведённые исследования подтвердили эффективность
использования перманганата калия как катализатора в процессе
окисления тяжёлых нефтяных остатков. Предложенная методика
отличается упрощённой технологической схемой, сниженным
энергопотреблением и повышенной экологической безопасностью.
Реализация данного подхода на промышленных установках Казахстана
позволит оптимизировать производство дорожных битумов, повысить
ресурсосбережение и обеспечить стабильное качество продукции
отечественных нефтеперерабатывающих предприятий.

{\bfseries Ключевые слова:} тяж\emph{}ёлые нефтяные остатки,
Каражанбасская нефть, дорожные битумы, каталитическое окисление,
энергоэффективность, дорожное строительство.

\begin{header}
ҚАРАЖАНБАС МҰНАЙЫНЫҢ АУЫР МҰНАЙ ҚАЛДЫҚТАРЫН ТОТЫҚТЫРУ ТЕХНОЛОГИЯСЫН ЖЕТІЛДІРУ

\tsp{1}С.К. Буканова\envelope,
\tsp{1,2}Г.К. Шамбилова,
\tsp{1}А.И. Абилхайров,
\tsp{1}А.С. Буканова,
\tsp{1}Ф.Б. Кайрлиева,
\tsp{3}Н.У.~Нургалиев\envelope
\end{header}

\begin{affil}
\tsp{1}«Сафи Өтебаев атындағы Атырау мұнай және газ университеті» КЕАҚ, Атырау, Қазақстан,

\tsp{2}«Х. Досмұхамедов атындағы Атырау университеті» КЕАҚ, Атырау, Қазақстан,

\tsp{3}Қ.Құлажанов атындағы технология және бизнес университеті, Астана, Қазақстан,

e-mail: sauleshik81@mail.ru, nurgaliev\_nao@mail.ru
\end{affil}

Қазақстанның мұнай өңдеу және жол-құрылыс салаларын
дамытудың қазіргі заманғы үрдістері битумдарды алудың энергия
тиімді және экологиялық ұтымды технологияларын енгізуді талап
етеді. Жоғары сапалы байланыстырғыш материалдарға сұраныстың
артуы жағдайында каталитикалық тотығу процестері ерекше өзекті
болып табылады, бұл жұмсақ температуралық режимдердегі
реакцияларды күшейтуге және жабдыққа жүктемені азайтуға мүмкіндік
береді.

Бұл жұмыс Белсенді катализатор ретінде калий перманганатын (KMnO₄)
пайдалана отырып Қаражанбас мұнайының ауыр мұнай қалдықтарын
оңайлатылған тікелей каталитикалық тотығу технологиясымен «артық
тотығу-сұйылту» дәстүрлі сызбасын ауыстыру мүмкіндігін
эксперименттік негіздеуге бағытталған.

Негізгі міндет энергия шығынын азайтып, араластыру сатысын алып
тастағанда ҚР СТ 1373-2013 талаптарына сәйкес келетін 70/100
БНД маркалы жол битумдарын алу.

Зерттеуде Қаражанбас мұнайынан алынған гудрон мен күңгірттенген
вакуумдық фракция пайдаланылды. Тотығу зертханалық реакторда
140--260°C температурада реттелетін ауа берумен (1 л/кг∙мин)
жүргізілді. Тотығу процестерін жеделдетіп, битумның
асфальтенді-шайырлы құрылымын қалыптастыратын катализатор ретінде
0,5\% марганец қышқылының калий тұзы қолданылды. Сынамалар ену,
жұмсарту, ұзару және сынғыштық нүктелерін анықтай отырып, ҚР СТ
және МЕМСТ стандарттарына сәйкес талданды.

KMnO₄ енгізу төмен температурада тұрақты тотығу реакциясын
қамтамасыз ететіні анықталды, бұл жоғары өнімділік қасиеттері
бар битумның түзілуіне әкеледі. Алынған өнім 111 (0,1 мм,
25°С) енуімен, жұмсарту температурасы 45°С, созылғыштығы 150
см-ден жоғары және сынғыштық температурасы -19°С сипатталады.

Бұл көрсеткіштер БНД 70/100 маркалы битумдарға қойылатын талаптарға
толық сәйкес келеді. Жүргізілген зерттеулер ауыр мұнай қалдықтарының
тотығуында катализатор ретінде калий перманганатын қолданудың
тиімділігін растады. Ұсынылған әдіс жеңілдетілген процесс ағынын,
энергияны тұтынуды азайтуды және жақсартылған экологиялық
қауіпсіздікті сипаттайды. Бұл тәсілді Қазақстанның өнеркәсіптік
нысандарында енгізу жол битумдарын өндіруді оңтайландыруға,
ресурстарды үнемдеуді арттыруға және отандық мұнай өңдеу
зауыттарында тұрақты өнім сапасын қамтамасыз етуге мүмкіндік
береді.

{\bfseries Түйін сөздер:} ауы\emph{}р мұнай қалдықтары,
Қаражанбас мұнайы, жол битумдары, каталитикалық тотығу, энергия
тиімділігі, жол құрылысы.

\begin{header}
IMPROVEMENT OF THE OXIDATION TECHNOLOGY FOR HEAVY OIL RESIDUES FROM KARAZHANBASS OIL

\tsp{1}S.K. Bukanova\envelope,
\tsp{1,2}G.K. Shambilova,
\tsp{1}A.I. Abilkhairov,
\tsp{1}A.S. Bukanova,
\tsp{1}F.B. Kairliyeva,
\tsp{3}N.U.~Nurgaliyev\envelope
\end{header}

\begin{affil}
\tsp{1}Safi Utebaev Atyrau University of oil and gas, Atyrau, Kazakhstan,

\tsp{2}Kh. Dosmukhamedov Atyrau University, Atyrau, Kazakhstan,

\tsp{3}K. Kulazhanov Kazakh University of Technology and Business,Astana, Kazakhstan,

e-mail: sauleshik81@mail.ru, nurgaliev\_nao@mail.ru
\end{affil}

Current trends in the development of
Kazakhstan' s oil refining and road construction
industries require the introduction of energy-efficient and
environmentally sound technologies for bitumen production. In the
context of growing demand for high-quality binding materials,
catalytic oxidation processes are beco\-ming particularly relevant,
as they allow reactions to be intensified at mild temperatures
and reduce the load on equipment.

This work is aimed at experimentally proving the possibility of
replacing the traditional `re-oxidation-dilution' scheme with a
simplified technology of direct catalytic oxidation of heavy oil
residues from Karazhanbas oil using potassium permanganate
(KMnO₄) as an active catalyst. The main task was to obtain BND
70/100 road bitumen that meets the requirements of ST RK
1373-2013, while reducing energy costs and eliminating the
compounding stage.

The study used tar and darkened vacuum fraction obtained from
Karazhanbas oil. Oxidation was carried out in a batch laboratory
reactor at temperatures of 140--260°C with a controlled air
supply (1 l/kg∙min). A 0.5\% potassium salt of manganic acid was
used as a catalyst to accelerate the oxidation processes and
form an asphalt-resinous bitumen structure. The selected samples
were analysed according to ST RK and GOST standards: penetration,
softening temperature, ductility and brittleness temperature were
determined.

It was found that the introduction of KMnO₄ ensures a stable
oxidation reaction at low temperatures, which leads to the
formation of bitumen with high performance properties. The
resulting product is characterised by a penetration of 111 (0.1
mm, 25°C), a softening point of 45°C, ductility above 150 cm and
a brittleness temperature of -19°C. These indicators fully comply
with the requirements for BND 70/100 bitumen. Materials and
methods. The study used tar and darkened vacuum fraction
obtained from Karazhanbas oil. Oxidation was carried out in a
batch laboratory reactor at temperatures of 140--260°C with a
controlled air supply (1 l/kg∙min). A 0.5\% potassium salt of
manganic acid was used as a catalyst to accelerate the oxidation
processes and form an asphalt-resinous bitumen structure. The
selected samples were analysed according to ST RK and GOST
standards: penetration, softening temperature, ductility and
brittleness temperature were determined.

The studies conducted confirmed the effectiveness of potassium
permanganate as a catalyst in the oxidation of heavy oil
residues. The proposed method is characterised by a simplified
technological scheme, reduced energy consumption and increased
environmental safety. The implementation of this approach at
industrial facilities in Kazakhstan will optimise the production
of road bitumen, increase resource conservation and ensure the
stable quality of products from domestic oil refineries.

{\bfseries Keywords:} heavy oil residues, Karazhanbas oil, road
bitumen, catalytic oxidation, energy efficiency, road
construction.

\begin{multicols}{2}
{\bfseries Введение.} Развитие дорожной инфраструктуры
Казахстана требует стабильного обеспечения отрасли качественными
дорожно-строительными материалами. Одной из ключевых проблем
остаётся дефицит и нестабильное качество битумов, от которых
зависят надёжность и долговечность дорожных покрытий. В условиях
сезонных колебаний спроса дефицит битума усиливается, особенно в
летние месяцы, когда объём потребления достигает 150 тыс. тонн
в месяц при совокупной мощности заводов менее 100 тыс. тонн.
Данный дисбаланс между производством и потреблением не только
вызывает перебои в поставках, но и способствует росту цен на
битум в строительный сезон {[}1{]}.

Основными производителями дорожного битума в Казахстане являются
ТОО «СП Caspi Bitum», Павлодарский НХЗ, ТОО «Qazaq Bitum» и
другие предприятия, суммарно обеспечивающие внутренний рынок.
Однако для покрытия растущего спроса требуется совершенствование
технологий и повышение эффективности переработки тяжёлых нефтяных
остатков.

Целью настоящего исследования является оценка эффективности
каталитического окисления тяжёлых нефтяных остатков Каражанбасской
нефти с применением KMnO₄ для получения битумов, соответствующих
требованиям стандарта СТ РК 1373-2013.

ТОО «Caspi Bitum» является одним из крупнейших производителей,
вяжущих в Казахстане. Производство дорожных битумов состоит из
двух основных установок: установки ЭЛОУ-АВТ, где нефть
месторождения Каражанбас очищается от воды и солей, а затем
подвергается многоступенчатой ректификации при атмосферном и
вакуумном давлении. Полученный гудрон превращается в дорожный
битум двух видов: обычный окисленный и модифицированный, с
применением современных стирол-бутадиен-стирольных полимеров
{[}2{]}.

Существует три основных способа производства нефтяных битумов:

- глубоковакуумная перегонка мазутов высокосернистых
высокосмолистых нефтей (остаточных битумов) {[}3{]};

- окисление кислородом воздуха различных нефтяных остатков
(гудронов, полугудронов, мазутов, экстрактов селективной очистки
масел, асфальтов деасфальтизации, крекинг-остатков или их смесей)
при температуре 180-300°С {[}4{]};

- компаундирование (смешение) различных нефтяных остатков с
дистиллятами и с остаточными или окисленными битумами и др.
{[}5{]}.

Возможны и сочетания указанных выше способов.

Прямое окисление нефтяных остатков (гудронов) и смесевых
композиций гудронов с другими остаточными продуктами переработки -
наиболее распространенная технология из всех вышеперечисленных
{[}6{]}.

Гудрон - это остаток глубоковакуумной перегонки мазута, последний
получают после атмосферной перегонки нефтей в атмосферной колонне
с одновременным выделением бензиновых и дизельных фракций
углеводородов. Гудрон окисляют в специальных аппаратах (реакторах)
продувая через него воздух. Процесс окисления проводят при
повышенных температурах 240-260°С {[}7{]}. В результате получают
дорожные битумы, которые должны соответствовать по своим
качественным показателям требованиям СТ РК 1373-2013.

Для производства улучшенных дорожных битумов также возможно
применение апробированной в промышленности и хорошо
зарекомендовавшей себя технологии «переокисление-разбавление». Суть
ee заключается в том, что исходное сырье (гудрон) окисляют до
значения показателя «температура размягчения», равного 90-105°С.
Далее этот стабильный асфальтосмолистый «структурный каркас»
битума - концентрат смол и асфальтенов пластифицируют путем
добавления к нему исходного стабильного неокисленного прямогонного
гудрона.
\end{multicols}

\fig{c4/image35}[Рис.1 - Схема переработки нефти месторождения Каражанбас с получением дорожных битумов]

\begin{multicols}{2}
Такая технология применяется в ТОО «CASPI BITUM» (рисунок 1).
Однако данная схема энергоёмка и требует отдельного блока
компаундирования.

На стадии компаундирования при возникновении необходимости
повышение пластичности и улучшения низкотемпературных характеристик
битума в смесь окисленного и неокисленного компонентов
дополнительно вводят пластифицирующие добавки: тяжелый вакуумный
газойль, затемненную фракцию. Отличительной особенностью данной
технологии является получение гудрона требуемой плотности и
вязкости, а также обеспечение производства битумов пластификатором
-- тяжелым вакуумным газойлем и затемненной фракцией. Кроме того,
для уменьшения потребления пара для создания глубокого вакуума в
вакуумной колонне используется гидроэжекторная вакуумсоздающая
система {[}8{]}.

{\bfseries Материалы и методы.} Одним из путей
совершенствования процессов получения битумов является введение
различных активирующих добавок или катализаторов, которые
позволяют увеличить производительность битумной установки,
улучшить свойства получаемых битумов и сократить удельные
энергетические затраты {[}9{]}.

Использование катализаторов, таких как хлорид железа (III)
{[}1000{]}, ортофосфорная кислота и их композиции {[}11111{]},
позволяет ускорить процесс окисления, снизить рабочую температуру
и повысить качество получаемых битумов.

В ходе исследований установлено, что добавление КМnО4 в нефтяные
остатки способствует проведение процесса в более мягких условиях
(140-150°С). Смесь, содержащую 90\% каражанбасского гудрона и
10\% окисленной затемнённой вакуумной фракции окисляют при
температуре 240-260°С {[}12222{]}.

Нами были исследованы гудрон и затемненная вакуумная фракция
Каражанбасской нефти. Фракционный состав затемненной вакуумной
фракции следующий: начало кипения - 409°С; 5\% выкипает при
453°С; 10\% - 462°С; 20\% - 472°С; 30\% - 478°С; 40\% - 485°С;
50\% - 492°С; 60\% - 500°С; 70\% - 509°С; 80\% - 523°С; 90\% - 545;
конец кипения - 548°С.

Окисление нефтяных остатков в лабораторных условиях осуществляли в
лабораторном кубике периодического действия. В общем, процесс
окисления сырья в лабораторных условиях аналогичен промышленному
процессу производства битумов в окислительных кубах.

Лабораторный кубик (рисунок 2) представляет собой
термоизолированный аппарат с регулируемым электрообогревом. Во
внутреннюю часть аппарата вставляется стальной стакан, в котором
происходит собственно окисление. Стакан оборудован крышкой, на
которой крепятся: штуцер для подачи воздуха на окисление,
барботер, штуцер для вывода отгона и газов окисления и
термометр.

Затемненную вакуумную фракцию каражанбаской нефти (410-550°С)
наливают в стакан. Рабочий объем продукта в стакане не должен
превышать 70\% его геометрического объема. Стакан закрывают
крышкой, оборудованной перечисленными приспособлениями, и включают
электрообогрев.

В данном эксперименте в качестве катализатора селективного
окисления парафиновых углеводородов добавлена калийная соль 0,5\%
марганцевой кислоты в расчете на двуокись марганца.

При достижении температуры продукта в стакане 140-150ºC вводится
воздух. Оптимальный рабочий расход воздуха 1 л/кг сырья в
минуту. Расход, подаваемого на окисление, регулируют ротаметром. С
момента подачи воздуха на окисление начинают отсчет времени
течения процесса, продолжительность реакции 4 часа.

Газы окисления и отгон через верхний штуцер, сообщающийся с зоной
сепарации аппарата, выводят в сепаратор, охлаждаемый водой.
Сконденсированный отгон выводят с низа сепаратора, а
несконденсировавшиеся пары и газы выводят в вытяжную вентиляцию.

Далее окисленную затемненную вакуумную фракцию (10\%) и 90\%
каражанбаского гудрона окисляли в лабораторном кубике при
240-260°С при скорости подачи воздуха 1 л/кг∙мин.

По истечении первых двух часов окисления отбирают специальным
пробоотборником пробу для определения температуры размягчения
окисляемого продукта. Следующий отбор пробы для анализа отбирают
также через 2 часа. В зависимости от интенсивности процесса
окисления, пробу для анализа следует отбирать через иные
промежутки времени, позволяющие достаточно достоверно построить
кинетику течения процесса. Если значение показателя «температура
размягчения» приближается к требуемому значению, отбор проб для
испытаний проводят через каждые 15-20 минут.
\end{multicols}

\fig[0.5\textwidth]{c4/image36}[Рис.2 - Лабораторный окислительный кубик периодического действия]

\begin{multicols}{2}
При значении показателя «температура размягчения» на 2-3ºC ниже
требуемого заданием на работу значения, подачу воздуха на
окисление прекращают, электрообогрев кубика отключают.

После остывания битума в кубике до 100-120ºC битум из кубика
сливают и разливают в специальные формы для проведения полного
комплекса испытаний.

{\bfseries Результаты и обсуждение.} При окислении были
отобраны пробы и проведен качественный анализ полученного битума
(таблица 1).
\end{multicols}

\tcap{Таблица 1 - Результаты испытаний образца нефтяного битума, полученного в условиях лаборатории}
\begin{longtblr}[
  label = none,
  entry = none,
]{
  width = \linewidth,
  colspec = {Q[481]Q[121]Q[135]Q[202]},
  cells = {c},
  cells = {font = \small},
  hlines,
  vlines,
}
\textbf{Наименование показателей, единицы измерения} & \textbf{СТ РК 1373} & \textbf{Образец} & \textbf{Методы испытания} \\
Глубина проникания иглы, 0,1 мм при 25°С0ºC & 71-10022 & 11122 & СТ РК 1226 \\
Температура размягчения по кольцу и шара, ºC & 45 & 43 & СТ РК 1227 \\
Растяжимость, см прине менее 25ºC0ºC & 753.8 & более 1504,3 & СТ РК 1374 \\
Температура хрупкости по Фраасу, ºC & не выше-18 & - 19 & СТ РК 1229 
\end{longtblr}

\begin{multicols}{2}
Из данных таблицы 1 следует, что в результате данной работы был
получен дорожный битум co следующими технологическими
характеристиками: пенетрацией при 25°С-111 мм; температурой
размягчения 45°С; дуктильностью при 25°С более 150 см и
температурой хрупкости минус 19°С, что соответствует СТ РК 1373
на марку битума БНД 70/100.

На основании проведенных исследований, рассмотрена возможность
использования данного метода {[}1222222{]} с целью получения
требуемых эксплуатационных характеристик битумов (рисунок 3).
\end{multicols}

\fig{c4/image37}[Рис.3 - Предлагаемая схема замены традиционного компаундирования каталитическим окислением]

В таблице 2 приведена сравнительная характеристика технологических
схем получения битума БНД 70/100.

\tcap{Таблица 2 - Сравнительная характеристика схем}
\begin{longtblr}[
  label = none,
  entry = none,
]{
  width = \linewidth,
  colspec = {Q[285]Q[329]Q[300]},
  cells = {c},
  cells = {font = \small},
  hlines,
  vlines,
}
\textbf{Показатель}            & \textbf{Схема «переокисление-разбавление»} & \textbf{Каталитическое окисление с KMnO₄} \\
Стадии                         & окисление + компаундирование               & Каталитическое окисление                  \\
Температурный режим            & 240-260°C + последующее разбавление        & 140-260°C в присутствии катализатора      \\
Катализаторы                   & отсутствуют                                & KMnO₄ (0,5\%)                             \\
Энергозатраты                  & высокие                                    & ниже на 10–15\%                           \\
Продолжительность цикла        & 5–6 часов                                  & 4 часа                                    \\
Необходимость компаундирования & Да                                         & Нет                                       
\end{longtblr}

\begin{multicols}{2}
Полученные экспериментальные данные показывают, что каталитическая
добавка перманганата калия (0,05-0,1 \%) существенно влияет на
направление и скорость процессов окисления тяжёлых нефтяных
остатков Каражанбасской нефти. Анализ показателей качества
готовых битумов (таблицы 1-2) подтверждает, что применение KMnO₄
позволяет оптимизировать условия окисления и сформировать
сбалансированное соотношение смолистых и асфальтеновых компонентов,
что проявляется в повышении температуры размягчения, снижении
температуры хрупкости и стабилизации показателей растяжимости.

В традиционном термоокислительном процессе без катализатора
наблюдаются параллельные реакции деструкции и поликонденсации, что
приводит к образованию избыточного количества твёрдых смол и
ухудшению коллоидной стабильности битума
{[}4, 8{]}.

Введение небольшого количества перманганата калия обеспечивает
контролируемое протекание окислительных реакций. Ионы Mn⁷⁺,
переходя в состояние Mn⁴⁺, ускоряют стадии дегидрирования и
образования полярных функциональных групп (карбоксильных,
гидроксильных, карбонильных), способствуя формированию устойчивой
асфальто-смоло-асфальтеновой структуры.

Динамика изменения физико-механических свойств подтверждает влияние
катализатора на механизм структурообразования. На начальных стадиях
окисления ионы Mn⁷⁺ выполняют роль инициаторов цепных радикальных
реакций, а образующийся MnO₂ формирует каталитически активную
поверхность, на которой протекают реакции конденсации смол в
асфальтены. Таким образом, катализатор одновременно выполняет
функции ускорителя и структурообразующего агента.

Сравнение с литературными данными показывает, что каталитическая
активность перманганата калия сопоставима с FeCl₃, H₃PO₄ и NiO,
однако его применение обеспечивает улучшение качества продуктов
окисления. При этом достигается высокая степень окисления без
побочных реакций, таких как образование кокса, что подтверждает
высокую селективность катализатора по отношению к насыщенным
углеводородным фрагментам. Это является значительным преимуществом
для тяжёлых нефтяных остатков Каражанбасской нефти, богатых
парафиновыми структурами.

Каталитическая схема окисления не только улучшает физико-химические
свойства битума, но и обеспечивает существенный ресурсосберегающий
эффект.

В целом анализ показывает, что каталитическое окисление с
использованием небольшого количества KMnO₄ приводит к формированию
стабильной коллоидной структуры битума с оптимальным соотношением
асфальтеновой и смолистой фаз. Полученные результаты подтверждают
эффективность применения перманганата калия в качестве
катализатора окисления тяжёлых нефтяных остатков и обеспечивают
стабильные эксплуатационные характеристики готового продукта.

{\bfseries Выводы.} В ходе проведённых исследований изучено влияние
предварительного каталитического окисления парафинсодержащей
затемнённой фракции Каражанбасской нефти и последующего
совместного окисления с гудроном Каражанбасской нефти на
формирование структурно-механических и физико-химических свойств
полученного окисленного битума.

В качестве катализатора на стадии предварительной обработки сырья
до основного процесса окисления использовался перманганат калия
(KMnO₄). Такой подход позволил целенаправленно регулировать
соотношение смолисто-асфальтеновых соединений и исключить
необходимость стадии «избыточного окисления-разбавления»,
характерной для традиционных схем.

Использование KMnO₄ в качестве катализатора способствует протеканию
реакций частичного окисления парафиновых соединений, что, в свою
очередь, обеспечивает образование устойчивых асфальтено-смолистых
структур. В результате достигается оптимальное соотношение между
коллоидно-дисперсной и гелеобразной фазами, повышающее термическую
стабильность и пластичность готового продукта.

Физико-химические показатели полученных битумов-глубина проникания
иглы, температура размягчения, растяжимость и температура
хрупкости-находятся в оптимальном диапазоне и соответствуют
требованиям СТ РК 1373-2013 для битумов марки БНД 70/100.
Результаты лабораторных испытаний подтверждают, что применение
каталитической стадии обеспечивает более высокий уровень
структурной однородности и стабильности битумов.

Установлено, что предварительное каталитическое окисление
затемнённой фракции повышает технологическую гибкость процесса,
что позволяет регулировать структуру и свойства битума в широких
пределах без существенных изменений температуры и
продолжительности окисления. Таким образом, предложенный подход
обеспечивает стабильность качества готового продукта.

Практическая значимость результатов заключается в том, что
предложенный метод может быть внедрён на действующих битумных
заводах без значительных изменений технологического оборудования,
что делает процесс привлекательным для модернизации
отечественного производства дорожных битумов и повышения его
ресурсной эффективности.

Полученные битумы характеризуются улучшенными эксплуатационными
свойствами, что особенно важно для континентального климата
Казахстана и его дорожных условий. Применение каталитически
окисленных битумов может способствовать увеличению прочности
асфальтобетонных покрытий и снижению затрат на их ремонт и
содержание.

Разработка научно-технических основ промышленного внедрения
каталитических технологий окисления гудронов на отечественных
нефтеперерабатывающих заводах представляется целесообразной и
позволит повысить конкурентоспособность и качество казахстанских
дорожных материалов.
\end{multicols}

\begin{center}
{\bfseries Литература}
\end{center}

\begin{refs}
1. Как в Казахстане планируют решать проблему дефицита битума.URL:
URL:
\href{https://www.inform.kz/ru/kak-v-kazahstane-planiruyut-reshat-problemu-defitsita-bituma-6a3972}{https://www.inform.kz}.-
Дата обращения: 07.09.2025.

2. Аналитический отчет по рынку битума в Республике Казахстан
{[}Электронный ресурс{]}. URL:
\href{https://ccx.kz/analytical-report-bitumen-market}{https://ccx.kz/analytical-report-bitumen-market}.-
Дата обращения: 16.09.2025.

3. Djimasbe R., Galiullin E.A., Varfolomeev M.A., et.al. Experimental
study of non-oxidized and oxidized bitumen obtained from heavy oil//Sci.
Rep. -2021.-Vol.11(1).-P.1-10. DOI 10.1038/s41598-021-87398-2.

4. Тлехусеж М.А., Дубиненко Н.А. Влияние химических процессов
производства строительных битумов на их свойства//Научные труды КубГТУ.
-2024. -№ 2.- C.50-58. DOI
\href{http://dx.doi.org/10.26297/2312-9409.2024.2.4}{10.26297/2312-9409.2024.2.4}.

5. Khusnutdinov I.,~ Goncharova I.,~ Safiulina A. Extractive
deasphalting as a method of obtaining asphalt binders and low-viscosity
deasphalted hydrocarbon feedstock from natural
bitumen//\href{https://www.sciencedirect.com/journal/egyptian-journal-of-petroleum}{Egyptian
Journal of Petroleum}.-2021.-Vol.30(2).-P.69-73.
\href{https://doi.org/10.1016/j.ejpe.2021.03.002}{DOI
10.1016/j.ejpe.2021.03.002}.

6. Кутьин Ю.А., Теляшев Э.Г. Битумы и битумные материалы. Нормативы,
качество, технологии / Ю.А. Кутьин, Э.Г. Теляшев. ГУП ИНХП РБ. -2018.-
272 c. ISBN 978-5-902159-56-8.

7. Abbasov V.M., Efendieva L.M., Cherepnova Yu.P., Nasibova G.G., et al.
Commercial Production of Bitumen via Oxidation of Vacuum Residue over
Iron Catalysts/Petroleum Chemistry. -2025.-Vol.65(3). -
P.298-303. DOI 10.1134/S0965544125600596.

8. Грудников И.Б. Нефтяные битумы. Процессы и технологии производства /
И.Б. Грудников.-Уфа: ГУП ИНХП РБ. -2015.- 288 c. ISBN 978-5-902159-51-3.

9. Евдокимова Н.Г., Прозорова О.Б., Буканова С.К. Совершенствование
технологии производства окисленных битумов как способ повышения
эффективности энергосбережения // Нефть и газ. - 2015. - №1. - С.43-52.

10. Лю Инчжоу,~МалгаждароваН.С.,~Онгарбаев Е.К.,~Тлеуберды Е., Акказин
Е.А.,Умбеткалиев К.А. Исследование влияния хлористого железа на свойства
окисленных битумов. - 2016.
URL:
\url{https://www.rusnauka.com/18_NPN_2016/Chimia/8_213557.doc.htm}.-
Дата обращения: 16.09.2025.

11. Онгарбаев Е.К., Жамболова А.Б., Тилеуберди Е. и др. Окисление
тяжелых нефтяных остатков в присутствии катализаторов и модификаторов //
Горение и плазмохимия.-2019. -№17.-С.47-56.
DOI \href{https://doi.org/10.18321/cpc286}{10.18321/cpc286}

12. Абилхайров А.И., Шамбилова Г.К., Буканова С.К., Карабасова Н.А.,
Кайрлиева Ф.Б., Наурызбаева А.Д., Сакипова Л.Б., Имангалиева Ф.Б. Способ
получения окисленного битума: патент на полезную модель №10821.
Казахстан, опубл.04.07.2025.
\end{refs}

\begin{center}
{\bfseries References}
\end{center}

\begin{refs}
1. Kak v Kazahstane planirujut reshat'{} problemu
deficita bituma.URL: URL:
\href{https://www.inform.kz/ru/kak-v-kazahstane-planiruyut-reshat-problemu-defitsita-bituma-6a3972}{https://www.inform.kz}.-
Data obrashhenija: 07.09.2025.{[}in Russian{]}

2. Analiticheskij otchet po rynku bituma v Respublike Kazahstan
{[}Jelektronnyj resurs{]}. URL:
https://ccx.kz/analytical-report-bitumen-market.- Data obrashhenija:
16.09.2025. {[}in Russian{]}

3. Djimasbe R., Galiullin E.A., Varfolomeev M.A., et.al. Experimental
study of non-oxidized and oxidized bitumen obtained from heavy oil//Sci.
Rep. -2021.-Vol.11(1).-P.1-10. DOI 10.1038/s41598-021-87398-2.

4. Tlehusezh M.A., Dubinenko N.A. Vlijanie himicheskih processov
proizvodstva stroitel' nyh bitumov na ih
svojstva//Nauchnye trudy KubGTU. -2024. -№ 2.- C.50-58. DOI
10.26297/2312-9409.2024.2.4.{[}in Russian{]}

5. Khusnutdinov I.,~ Goncharova I.,~ Safiulina A. Extractive
deasphalting as a method of obtaining asphalt binders and low-viscosity
deasphalted hydrocarbon feedstock from natural
bitumen//\href{https://www.sciencedirect.com/journal/egyptian-journal-of-petroleum}{Egyptian
Journal of Petroleum}.-2021.-Vol.30(2).-P.69-73.
\href{https://doi.org/10.1016/j.ejpe.2021.03.002}{DOI
10.1016/j.ejpe.2021.03.002}.

6. Kut' in Ju.A., Teljashev Je.G. Bitumy i bitumnye
materialy. Normativy, kachestvo, tehnologii / Ju.A.
Kut' in, Je.G. Teljashev. GUP INHP RB. -2018.- 272 c.
ISBN 978-5-902159-56-8.{[}in Russian{]}

7. Abbasov V.M., Efendieva L.M., Cherepnova Yu.P., Nasibova G.G., et al.
Commercial Production of Bitumen via Oxidation of Vacuum Residue over
Iron Catalysts/Petroleum Chemistry. -2025.-Vol.65(3). -
P.298-303. DOI 10.1134/S0965544125600596.

8. Grudnikov I.B. Neftjanye bitumy. Processy i tehnologii proizvodstva /
I.B. Grudnikov.-Ufa: GUP INHP RB. -2015.- 288 c. ISBN 978-5-902159-51-3.
{[}in Russian{]}

9. Evdokimova N.G., Prozorova O.B., Bukanova S.K. Sovershenstvovanie
tehnologii proizvodstva okislennyh bitumov kak sposob povyshenija
jeffektivnosti jenergosberezhenija //Neft'{} i gaz. -
2015. - №1. - S.43-52. {[}in Russian{]}

10. Lju Inchzhou, MalgazhdarovaN.S., Ongarbaev E.K., Tleuberdy E.,
Akkazin E.A.,Umbetkaliev K.A. Issledovanie vlijanija hloristogo zheleza
na svojstva okislennyh bitumov. - 2016.
URL: https://www.rusnauka.com/18\_NPN\_2016/Chimia/8\_213557.doc.htm.-
Data obrashhenija: 16.09.2025. {[}in Russian{]}

11. Ongarbaev E.K., Zhambolova A.B., Tileuberdi E. i dr. Okislenie
tjazhelyh neftjanyh ostatkov v prisutstvii kataliz{[}in Russian{]}atorov
i modifikatorov // Gorenie i plazmohimija.-2019. -№17.-S.47-56. DOI
10.18321/cpc286.

12. Abilhajrov A.I., Shambilova G.K., Bukanova S.K., Karabasova N.A.,
Kajrlieva F.B., Nauryzbaeva A.D., Sakipova L.B., Imangalieva F.B. Sposob
poluchenija okislennogo bituma: patent na poleznuju
model'{} №10821. Kazahstan, opubl.04.07.2025. {[}in
Russian{]}
\end{refs}

\begin{info}
\hspace{1em}\emph{{\bfseries Сведения об авторах}}

Буканова СК.- старший преподаватель Атырауского университета нефти и
газа им. Сафи Утебаева, Атырау, Казахстан, e-mail:
sauleshik81@mail.ru;

Шамбилова Г.К.- д.х.н., профессор Атырауского университета нефти и
газа им. Сафи Утебаева и Атырауского университета им. Х.
Досмухамедова, Атырау, Казахстан, e-mail: shambilova\_gulba@mail.ru;

Абилхайров А.И.- к.х.н., ассоциированный профессор Атырауского
университета нефти и газа им. Сафи Утебаева, Атырау, Казахстан,
e-mail: a.abilkhayrov@mail.ru;

БукановаА.С.- к.т.н., профессор Атырауского университета нефти и газа
им. Сафи Утебаева,Атырау, Казахстан, e-mail: bukanova66@mail.ru;

Кайрлиева Ф.Б.- к.т.н., старший преподаватель Атырауского университета
нефти и газа им. Сафи Утебаева,

Атырау, Казахстан, e-mail: kairliyeva.fazi@mail.ru;

Нургалиев Н.У.-к.х.н., ассоциированный профессор, Казахский
университет технологии и бизнеса им. К.Кулажанова, Астана, Казахстан,
e-mail: nurgaliev\_nao@mail.ru.

\hspace{1em}\emph{{\bfseries Information about the authors}}

Bukanova S.R.- Senior Lecturer, Safi Utebayev Atyrau oil and gas
university, Atyrau, Kazakhstan, e-mail:
\href{mailto:sauleshik81@mail.ru}{sauleshik81@mail.ru};

Shambilova G.K.- Doctor of Chemical Sciences, Professor, Safi Utebayev
of Atyrau oil and gas university and H. Dosmukhamedov University,
Atyrau, Kazakhstan, e-mail: shambilova\_gulba@mail.ru;

Abilkhaiev A.I.- Candidate of Chemical Sciences, Associate Professor,
Safi Utebayev Atyrau oil and gas university, Atyrau, Kazakhstan,
e-mail: a.abilkhayrov@mail.ru;

Bukanova A.S.- Candidate of technical Sciences, Professor, Safi
Utebayev Atyrau oil and gas university, Atyrau, Kazakhstan,
e-mail: bukanova66@mail.ru;

Kairlieva F.B.- Candidate of technical Sciences, Senior Lecturer, Safi
Utebayev Atyrau oil and gas university, Atyrau, Kazakhstan,
e-mail: kairliyeva.fazi@mail.ru;

Nurgaliyev N.U.- Candidate of Chemical Sciences, Associate Professor,
K.Kulazhanov Kazakh University of Technology and Business, Astana„
Kazakhstan e-mail: nurgaliev\_nao@mail.ru.
\end{info}
