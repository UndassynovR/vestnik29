\id{МРНТИ 61.51.37}{}

{\bfseries СОВЕРШЕНСТВОВАНИЕ ТЕХНОЛОГИИ ОКИСЛЕНИЯ ТЯЖЕЛЫХ НЕФТЯНЫХ ОСТАТКОВ
КАРАЖАНБАССКОЙ НЕФТИ}

{\bfseries \tsp{1}С.К Буканова}
{\bfseries \envelope ,
\tsp{1,2}Г.К.} {\bfseries Шамбилов}
{\bfseries ,
\tsp{1}А.И.}
{\bfseries Абилхайров}{\bfseries ,
\tsp{1}А.С.}
{\bfseries Буканова},

{\bfseries \tsp{1}Ф.Б.}
{\bfseries Кайрлиева}{\bfseries ,
\tsp{3}Н.У.} {\bfseries Нургалиев}
\envelope 

\tsp{1}\emph{НАО «Атырауский университет нефти и газа имени
Сафи Утебаева», Атырау,Казахстан,}

\tsp{2}\emph{НАО «Атырауский университет имени Х.
Досмухамедова», Атырац,Казахстан,}

\tsp{3}\emph{Казахский университет технологии и бизнеса им.
К.Кулажанова, Астана, Казахстан}

\envelope Корреспондент- автор:
sauleshik81@mail.ru,
nurgaliev\_nao@mail.ru

Современные тенденции развития нефтеперерабатывающей и
дорожно-строительной отраслей Казахстана требуют внедрения
энергоэффективных и экологически рациональных технологий получения
битумов. В условиях растущего спроса на качественные вяжущие материалы
особую актуальность приобретают каталитические процессы окисления,
позволяющие интенсифицировать реакции при мягких температурных режимах и
снизить нагрузку на оборудование.

Настоящая работа направлена на экспериментальное обоснование
возможности замены традиционной схемы «переокисление-разбавление»
упрощённой технологией прямого каталитического окисления тяжёлых
нефтяных остатков Каражанбасской нефти с использованием
перманганата калия (KMnO₄) в качестве активного катализатора.

Основной задачей являлось получение дорожных битумов марки БНД
70/100, соответствующих требованиям СТ РК 1373-2013, при
одновременном снижении энергозатрат и исключении стадии
компаундирования.

В исследовании использованы гудрон и затемнённая вакуумная
фракция, полученные из Каражанбасской нефти. Окисление проводилось
в лабораторном реакторе периодического действия при температурах
140-260°С с регулируемой подачей воздуха (1 л/кг∙мин). В качестве
катализатора применялась 0,5 \% калиевая соль марганцевой кислоты,
обеспечивающая ускорение окислительных процессов и формирование
асфальтосмолистой структуры битума. Отобранные пробы
анализировались по стандартам СТ РК и ГОСТ: определялись
пенетрация, температура размягчения, растяжимость и температура
хрупкости.

Установлено, что введение KMnO₄ обеспечивает устойчивое протекание
реакции окисления при пониженных температурах, что приводит к
формированию битума с высокими эксплуатационными свойствами.
Полученный продукт характеризуется пенетрацией 111 (0,1 мм, 25
°С), температурой размягчения 45 °С, дуктильностью выше 150 см и
температурой хрупкости -19 °С.

Эти показатели полностью соответствуют требованиям к битумам марки
БНД 70/100. Проведённые исследования подтвердили эффективность
использования перманганата калия как катализатора в процессе
окисления тяжёлых нефтяных остатков. Предложенная методика
отличается упрощённой технологической схемой, сниженным
энергопотреблением и повышенной экологической безопасностью.
Реализация данного подхода на промышленных установках Казахстана
позволит оптимизировать производство дорожных битумов, повысить
ресурсосбережение и обеспечить стабильное качество продукции
отечественных нефтеперерабатывающих предприятий.

{\bfseries Ключевые слова:} тяж\emph{}ёлые нефтяные остатки,
Каражанбасская нефть, дорожные битумы, каталитическое окисление,
энергоэффективность, дорожное строительство.

{\bfseries ҚАРАЖАНБАС МҰНАЙЫНЫҢ АУЫР МҰНАЙ ҚАЛДЫҚТАРЫН ТОТЫҚТЫРУ
ТЕХНОЛОГИЯСЫН ЖЕТІЛДІРУ}

{\bfseries \tsp{1}С.К.Буканова\tsp{\envelope },
\tsp{1,2}Г.К.Шамбилова\tsp{},
\tsp{1}А.И.Абилхайров\tsp{},
\tsp{1}А.С.Буканова\tsp{},}

{\bfseries \tsp{1}Ф.Б.Кайрлиева\tsp{}},
{\bfseries \tsp{3}Н.У. Нургалиев\envelope }

\tsp{1}

\tsp{}\emph{«Сафи Өтебаев атындағы Атырау мұнай және
газ университеті» КЕАҚ, Атырау, Қазақстан,}

\tsp{2\emph{}}\emph{«Х. Досмұхамедов атындағы Атырау
университеті» КЕАҚ, Атырау, Қазақстан,}

\tsp{3\emph{}}\emph{Қ.Құлажанов атындағы технология және
бизнес университеті,Астана, Қазақстан,}

e-mail: \href{mailto:sauleshik81@mail.ru}{sauleshik81@mail.ru},
nurgaliev\_nao@mail.ru

Қазақстанның мұнай өңдеу және жол-құрылыс салаларын
дамытудың қазіргі заманғы үрдістері битумдарды алудың энергия
тиімді және экологиялық ұтымды технологияларын енгізуді талап
етеді. Жоғары сапалы байланыстырғыш материалдарға сұраныстың
артуы жағдайында каталитикалық тотығу процестері ерекше өзекті
болып табылады, бұл жұмсақ температуралық режимдердегі
реакцияларды күшейтуге және жабдыққа жүктемені азайтуға мүмкіндік
береді.

Бұл жұмыс Белсенді катализатор ретінде калий перманганатын (KMnO₄)
пайдалана отырып Қаражанбас мұнайының ауыр мұнай қалдықтарын
оңайлатылған тікелей каталитикалық тотығу технологиясымен «артық
тотығу-сұйылту» дәстүрлі сызбасын ауыстыру мүмкіндігін
эксперименттік негіздеуге бағытталған.

Негізгі міндет энергия шығынын азайтып, араластыру сатысын алып
тастағанда ҚР СТ 1373-2013 талаптарына сәйкес келетін 70/100
БНД маркалы жол битумдарын алу.

Зерттеуде Қаражанбас мұнайынан алынған гудрон мен күңгірттенген
вакуумдық фракция пайдаланылды. Тотығу зертханалық реакторда
140--260°C температурада реттелетін ауа берумен (1 л/кг∙мин)
жүргізілді. Тотығу процестерін жеделдетіп, битумның
асфальтенді-шайырлы құрылымын қалыптастыратын катализатор ретінде
0,5\% марганец қышқылының калий тұзы қолданылды. Сынамалар ену,
жұмсарту, ұзару және сынғыштық нүктелерін анықтай отырып, ҚР СТ
және МЕМСТ стандарттарына сәйкес талданды.

KMnO₄ енгізу төмен температурада тұрақты тотығу реакциясын
қамтамасыз ететіні анықталды, бұл жоғары өнімділік қасиеттері
бар битумның түзілуіне әкеледі. Алынған өнім 111 (0,1 мм,
25°С) енуімен, жұмсарту температурасы 45°С, созылғыштығы 150
см-ден жоғары және сынғыштық температурасы -19°С сипатталады.

Бұл көрсеткіштер БНД 70/100 маркалы битумдарға қойылатын талаптарға
толық сәйкес келеді. Жүргізілген зерттеулер ауыр мұнай қалдықтарының
тотығуында катализатор ретінде калий перманганатын қолданудың
тиімділігін растады. Ұсынылған әдіс жеңілдетілген процесс ағынын,
энергияны тұтынуды азайтуды және жақсартылған экологиялық
қауіпсіздікті сипаттайды. Бұл тәсілді Қазақстанның өнеркәсіптік
нысандарында енгізу жол битумдарын өндіруді оңтайландыруға,
ресурстарды үнемдеуді арттыруға және отандық мұнай өңдеу
зауыттарында тұрақты өнім сапасын қамтамасыз етуге мүмкіндік
береді.

{\bfseries Түйін сөздер:} ауы\emph{}р мұнай қалдықтары,
Қаражанбас мұнайы, жол битумдары, каталитикалық тотығу, энергия
тиімділігі, жол құрылысы.

{\bfseries IMPROVEMENT OF THE OXIDATION TECHNOLOGY FOR HEAVY OIL
RESIDUES FROM KARAZHANBASS OIL}

{\bfseries \tsp{1}}

{\bfseries S.K.Bukanova\tsp{1\envelope }, \tsp{1,2}G.
K.Shambilova\tsp{},
\tsp{1}A.I.Abilkhairov\tsp{},
\tsp{1}A.S.Bukanova\tsp{},}

{\bfseries \tsp{1}F.B.Kairliyeva\tsp{} ,
\tsp{3}N.U.Nurgaliyev\envelope }

{\bfseries \tsp{}}

\tsp{1}\emph{Safi Utebaev}

\emph{Aty\tsp{}rau University of oil and gas,
Aty\tsp{}rau, Kazakhstan,}

\tsp{2}\emph{Kh. Dosmukhamedov}

\emph{Aty\tsp{}rau University, Aty\tsp{}rau,
Kazakhstan,}

\tsp{3}\emph{K. Kulazhanov}

\emph{\tsp{}Kazakh University of Technology and
Business,Astana, Kazakhstan,}

e-mail: \href{mailto:sauleshik81@mail.ru}{sauleshik81@mail.ru},
\href{mailto:nurgaliev_nao@mail.ru}{nurgaliev\_nao@mail.ru}

Current trends in the development of
Kazakhstan' s oil refining and road construction
industries require the introduction of energy-efficient and
environmentally sound technologies for bitumen production. In the
context of growing demand for high-quality binding materials,
catalytic oxidation processes are becoming particularly relevant,
as they allow reactions to be intensified at mild temperatures
and reduce the load on equipment.

This work is aimed at experimentally proving the possibility of
replacing the traditional `re-oxidation-dilution' scheme with a
simplified technology of direct catalytic oxidation of heavy oil
residues from Karazhanbas oil using potassium permanganate
(KMnO₄) as an active catalyst. The main task was to obtain BND
70/100 road bitumen that meets the requirements of ST RK
1373-2013, while reducing energy costs and eliminating the
compounding stage.

The study used tar and darkened vacuum fraction obtained from
Karazhanbas oil. Oxidation was carried out in a batch laboratory
reactor at temperatures of 140--260°C with a controlled air
supply (1 l/kg∙min). A 0.5\% potassium salt of manganic acid was
used as a catalyst to accelerate the oxidation processes and
form an asphalt-resinous bitumen structure. The selected samples
were analysed according to ST RK and GOST standards: penetration,
softening temperature, ductility and brittleness temperature were
determined.

It was found that the introduction of KMnO₄ ensures a stable
oxidation reaction at low temperatures, which leads to the
formation of bitumen with high performance properties. The
resulting product is characterised by a penetration of 111 (0.1
mm, 25°C), a softening point of 45°C, ductility above 150 cm and
a brittleness temperature of -19°C. These indicators fully comply
with the requirements for BND 70/100 bitumen. Materials and
methods. The study used tar and darkened vacuum fraction
obtained from Karazhanbas oil. Oxidation was carried out in a
batch laboratory reactor at temperatures of 140--260°C with a
controlled air supply (1 l/kg∙min). A 0.5\% potassium salt of
manganic acid was used as a catalyst to accelerate the oxidation
processes and form an asphalt-resinous bitumen structure. The
selected samples were analysed according to ST RK and GOST
standards: penetration, softening temperature, ductility and
brittleness temperature were determined.

The studies conducted confirmed the effectiveness of potassium
permanganate as a catalyst in the oxidation of heavy oil
residues. The proposed method is characterised by a simplified
technological scheme, reduced energy consumption and increased
environmental safety. The implementation of this approach at
industrial facilities in Kazakhstan will optimise the production
of road bitumen, increase resource conservation and ensure the
stable quality of products from domestic oil refineries.

{\bfseries Keywords:} heavy oil residues, Karazhanbas oil, road
bitumen, catalytic oxidation, energy efficiency, road
construction.

{\bfseries Введение.} Развитие дорожной инфраструктуры
Казахстана требует стабильного обеспечения отрасли качественными
дорожно-строительными материалами. Одной из ключевых проблем
остаётся дефицит и нестабильное качество битумов, от которых
зависят надёжность и долговечность дорожных покрытий. В условиях
сезонных колебаний спроса дефицит битума усиливается, особенно в
летние месяцы, когда объём потребления достигает 150 тыс. тонн
в месяц при совокупной мощности заводов менее 100 тыс. тонн.
Данный дисбаланс между производством и потреблением не только
вызывает перебои в поставках, но и способствует росту цен на
битум в строительный сезон {[}1{]}.

Основными производителями дорожного битума в Казахстане являются
ТОО «СП Caspi Bitum», Павлодарский НХЗ, ТОО «Qazaq Bitum» и
другие предприятия, суммарно обеспечивающие внутренний рынок.
Однако для покрытия растущего спроса требуется совершенствование
технологий и повышение эффективности переработки тяжёлых нефтяных
остатков.

Целью настоящего исследования является оценка эффективности
каталитического окисления тяжёлых нефтяных остатков Каражанбасской
нефти с применением KMnO₄ для получения битумов, соответствующих
требованиям стандарта СТ РК 1373-2013.

ТОО «Caspi Bitum» является одним из крупнейших производителей,
вяжущих в Казахстане. Производство дорожных битумов состоит из
двух основных установок: установки ЭЛОУ-АВТ, где нефть
месторождения Каражанбас очищается от воды и солей, а затем
подвергается многоступенчатой ректификации при атмосферном и
вакуумном давлении. Полученный гудрон превращается в дорожный
битум двух видов: обычный окисленный и модифицированный, с
применением современных стирол-бутадиен-стирольных полимеров
{[}2{]}.

Существует три основных способа производства нефтяных битумов:

- глубоковакуумная перегонка мазутов высокосернистых
высокосмолистых нефтей (остаточных битумов) {[}3{]};

- окисление кислородом воздуха различных нефтяных остатков
(гудронов, полугудронов, мазутов, экстрактов селективной очистки
масел, асфальтов деасфальтизации, крекинг-остатков или их смесей)
при температуре 180-300°С {[}4{]};

- компаундирование (смешение) различных нефтяных остатков с
дистиллятами и с остаточными или окисленными битумами и др.
{[}5{]}.

Возможны и сочетания указанных выше способов.

Прямое окисление нефтяных остатков (гудронов) и смесевых
композиций гудронов с другими остаточными продуктами переработки -
наиболее распространенная технология из всех вышеперечисленных
{[}6{]}.

Гудрон - это остаток глубоковакуумной перегонки мазута, последний
получают после атмосферной перегонки нефтей в атмосферной колонне
с одновременным выделением бензиновых и дизельных фракций
углеводородов. Гудрон окисляют в специальных аппаратах (реакторах)
продувая через него воздух. Процесс окисления проводят при
повышенных температурах 240-260°С {[}7{]}. В результате получают
дорожные битумы, которые должны соответствовать по своим
качественным показателям требованиям СТ РК 1373-2013.

Для производства улучшенных дорожных битумов также возможно
применение апробированной в промышленности и хорошо
зарекомендовавшей себя технологии «переокисление-разбавление». Суть
ee заключается в том, что исходное сырье (гудрон) окисляют до
значения показателя «температура размягчения», равного 90-105°С.
Далее этот стабильный асфальтосмолистый «структурный каркас»
битума - концентрат смол и асфальтенов пластифицируют путем
добавления к нему исходного стабильного неокисленного прямогонного
гудрона.

На стадии компаундирования при возникновении необходимости
повышение пластичности и улучшения низкотемпературных характеристик
битума в смесь окисленного и неокисленного компонентов
дополнительно вводят пластифицирующие добавки: тяжелый вакуумный
газойль, затемненную фракцию. Отличительной особенностью данной
технологии является получение гудрона требуемой плотности и
вязкости, а также обеспечение производства битумов пластификатором
-- тяжелым вакуумным газойлем и затемненной фракцией. Кроме того,
для уменьшения потребления пара для создания глубокого вакуума в
вакуумной колонне используется гидроэжекторная вакуумсоздающая
система {[}8{]}.

Такая технология применяется в ТОО «CASPI BITUM» (рисунок 1).
Однако данная схема энергоёмка и требует отдельного блока
компаундирования.

\fig{c4/image35}{}{\bfseries Рис.1
- Схема переработки нефти месторождения Каражанбас с получением}

{\bfseries дорожных битумов}

{\bfseries Материалы и методы.} Одним из путей
совершенствования процессов получения битумов является введение
различных активирующих добавок или катализаторов, которые
позволяют увеличить производительность битумной установки,
улучшить свойства получаемых битумов и сократить удельные
энергетические затраты {[}9{]}.

Использование катализаторов, таких как хлорид железа (III)
{[}1000{]}, ортофосфорная кислота и их композиции {[}11111{]},
позволяет ускорить процесс окисления, снизить рабочую температуру
и повысить качество получаемых битумов.

В ходе исследований установлено, что добавление КМnО4 в нефтяные
остатки способствует проведение процесса в более мягких условиях
(140-150°С). Смесь, содержащую 90\% каражанбасского гудрона и
10\% окисленной затемнённой вакуумной фракции окисляют при
температуре 240-260°С {[}12222{]}.

Нами были исследованы гудрон и затемненная вакуумная фракция
Каражанбасской нефти. Фракционный состав затемненной вакуумной
фракции следующий: начало кипения - 409°С; 5\% выкипает при
453°С; 10\% - 462°С; 20\% - 472°С; 30\% - 478°С; 40\% - 485°С;
50\% - 492°С; 60\% - 500°С; 70\% - 509°С; 80\% - 523°С; 90\% - 545;
конец кипения - 548°С.

Окисление нефтяных остатков в лабораторных условиях осуществляли в
лабораторном кубике периодического действия. В общем, процесс
окисления сырья в лабораторных условиях аналогичен промышленному
процессу производства битумов в окислительных кубах.

Лабораторный кубик (рисунок 2) представляет собой
термоизолированный аппарат с регулируемым электрообогревом. Во
внутреннюю часть аппарата вставляется стальной стакан, в котором
происходит собственно окисление. Стакан оборудован крышкой, на
которой крепятся: штуцер для подачи воздуха на окисление,
барботер, штуцер для вывода отгона и газов окисления и
термометр.

Затемненную вакуумную фракцию каражанбаской нефти (410-550°С)
наливают в стакан. Рабочий объем продукта в стакане не должен
превышать 70\% его геометрического объема. Стакан закрывают
крышкой, оборудованной перечисленными приспособлениями, и включают
электрообогрев.

В данном эксперименте в качестве катализатора селективного
окисления парафиновых углеводородов добавлена калийная соль 0,5\%
марганцевой кислоты в расчете на двуокись марганца.

При достижении температуры продукта в стакане 140-150ºC вводится
воздух. Оптимальный рабочий расход воздуха 1 л/кг сырья в
минуту. Расход, подаваемого на окисление, регулируют ротаметром. С
момента подачи воздуха на окисление начинают отсчет времени
течения процесса, продолжительность реакции 4 часа.

Газы окисления и отгон через верхний штуцер, сообщающийся с зоной
сепарации аппарата, выводят в сепаратор, охлаждаемый водой.
Сконденсированный отгон выводят с низа сепаратора, а
несконденсировавшиеся пары и газы выводят в вытяжную вентиляцию.

\fig{c4/image36}{}

{\bfseries Рис.2- Лабораторный окислительный кубик периодического
действия}

Далее окисленную затемненную вакуумную фракцию (10\%) и 90\%
каражанбаского гудрона окисляли в лабораторном кубике при
240-260°С при скорости подачи воздуха 1 л/кг∙мин.

По истечении первых двух часов окисления отбирают специальным
пробоотборником пробу для определения температуры размягчения
окисляемого продукта. Следующий отбор пробы для анализа отбирают
также через 2 часа. В зависимости от интенсивности процесса
окисления, пробу для анализа следует отбирать через иные
промежутки времени, позволяющие достаточно достоверно построить
кинетику течения процесса. Если значение показателя «температура
размягчения» приближается к требуемому значению, отбор проб для
испытаний проводят через каждые 15-20 минут.

При значении показателя «температура размягчения» на 2-3ºC ниже
требуемого заданием на работу значения, подачу воздуха на
окисление прекращают, электрообогрев кубика отключают.

После остывания битума в кубике до 100-120ºC битум из кубика
сливают и разливают в специальные формы для проведения полного
комплекса испытаний.

{\bfseries Результаты и обсуждение.} При окислении были
отобраны пробы и проведен качественный анализ полученного битума
(таблица 1).

{\bfseries Таблица 1 - Результаты испытаний образца нефтяного битума,
полученного}

{\bfseries в условиях лаборатории}

%% \begin{longtable}[]{@{}
%%   >{\raggedright\arraybackslash}p{(\linewidth - 6\tabcolsep) * \real{0.3735}}
%%   >{\centering\arraybackslash}p{(\linewidth - 6\tabcolsep) * \real{0.2226}}
%%   >{\centering\arraybackslash}p{(\linewidth - 6\tabcolsep) * \real{0.1812}}
%%   >{\centering\arraybackslash}p{(\linewidth - 6\tabcolsep) * \real{0.2227}}@{}}
%% \toprule\noalign{}
%% \begin{minipage}[b]{\linewidth}\centering
%% {\bfseries Наименование показателей, единицы измерения}
%% \end{minipage} & \begin{minipage}[b]{\linewidth}\centering
%% {\bfseries СТ РК 1373}
%% \end{minipage} & \begin{minipage}[b]{\linewidth}\centering
%% {\bfseries Образец}
%% \end{minipage} & \begin{minipage}[b]{\linewidth}\centering
%% {\bfseries Методы испытания}
%% \end{minipage} \\
%% \midrule\noalign{}
%% \endhead
%% \bottomrule\noalign{}
%% \endlastfoot
%% Глубина проникания иглы, 0,1 мм при 25°С
%% 
%% 0ºC & 71-100
%% 
%% 22 & 111
%% 
%% 22 & СТ РК 1226 \\
%% Температура размягчения по кольцу и шара, ºC & 45 & 43 & СТ РК
%% 1227 \\
%% Растяжимость, см при
%% 
%% не менее 25ºC
%% 
%% 0ºC & 75
%% 
%% 3.8 & более 150
%% 
%% 4,3 & СТ РК 1374 \\
%% Температура хрупкости по Фраасу, ºC & не выше
%% 
%% -18 & - 19 & СТ РК 1229 \\
%% \end{longtable}

Из данных таблицы 1 следует, что в результате данной работы был
получен дорожный битум co следующими технологическими
характеристиками: пенетрацией при 25°С-111 мм; температурой
размягчения 45°С; дуктильностью при 25°С более 150 см и
температурой хрупкости минус 19°С, что соответствует СТ РК 1373
на марку битума БНД 70/100.

На основании проведенных исследований, рассмотрена возможность
использования данного метода {[}1222222{]} с целью получения
требуемых эксплуатационных характеристик битумов (рисунок 3).

\fig{c4/image37}{}

{\bfseries Рис.3 - Предлагаемая схема замены традиционного
компаундирования каталитическим окислением}

В таблице 2 приведена сравнительная характеристика технологических
схем получения битума БНД 70/100.

{\bfseries Таблица 2 - Сравнительная характеристика схем}

%% \begin{longtable}[]{@{}
%%   >{\centering\arraybackslash}p{(\linewidth - 4\tabcolsep) * \real{0.2685}}
%%   >{\centering\arraybackslash}p{(\linewidth - 4\tabcolsep) * \real{0.3581}}
%%   >{\centering\arraybackslash}p{(\linewidth - 4\tabcolsep) * \real{0.3734}}@{}}
%% \toprule\noalign{}
%% \begin{minipage}[b]{\linewidth}\centering
%% {\bfseries Показатель}
%% \end{minipage} & \begin{minipage}[b]{\linewidth}\centering
%% {\bfseries Схема «переокисление-разбавление»}
%% \end{minipage} & \begin{minipage}[b]{\linewidth}\centering
%% {\bfseries Каталитическое окисление с KMnO₄}
%% \end{minipage} \\
%% \midrule\noalign{}
%% \endhead
%% \bottomrule\noalign{}
%% \endlastfoot
%% Стадии & окисление + компаундирование & Каталитическое окисление \\
%% Температурный режим & 240-260°C + последующее разбавление & 140-260°C в
%% присутствии катализатора \\
%% Катализаторы & отсутствуют & KMnO₄ (0,5\%) \\
%% Энергозатраты & высокие & ниже на 10--15\% \\
%% Продолжительность цикла & 5--6 часов & 4 часа \\
%% Необходимость компаундирования & Да & Нет \\
%% \end{longtable}

Полученные экспериментальные данные показывают, что каталитическая
добавка перманганата калия (0,05-0,1 \%) существенно влияет на
направление и скорость процессов окисления тяжёлых нефтяных
остатков Каражанбасской нефти. Анализ показателей качества
готовых битумов (таблицы 1-2) подтверждает, что применение KMnO₄
позволяет оптимизировать условия окисления и сформировать
сбалансированное соотношение смолистых и асфальтеновых компонентов,
что проявляется в повышении температуры размягчения, снижении
температуры хрупкости и стабилизации показателей растяжимости.

В традиционном термоокислительном процессе без катализатора
наблюдаются параллельные реакции деструкции и поликонденсации, что
приводит к образованию избыточного количества твёрдых смол и
ухудшению коллоидной стабильности битума\\
{[}4, 8{]}.

Введение небольшого количества перманганата калия обеспечивает
контролируемое протекание окислительных реакций. Ионы Mn⁷⁺,
переходя в состояние Mn⁴⁺, ускоряют стадии дегидрирования и
образования полярных функциональных групп (карбоксильных,
гидроксильных, карбонильных), способствуя формированию устойчивой
асфальто-смоло-асфальтеновой структуры.

Динамика изменения физико-механических свойств подтверждает влияние
катализатора на механизм структурообразования. На начальных стадиях
окисления ионы Mn⁷⁺ выполняют роль инициаторов цепных радикальных
реакций, а образующийся MnO₂ формирует каталитически активную
поверхность, на которой протекают реакции конденсации смол в
асфальтены. Таким образом, катализатор одновременно выполняет
функции ускорителя и структурообразующего агента.

Сравнение с литературными данными показывает, что каталитическая
активность перманганата калия сопоставима с FeCl₃, H₃PO₄ и NiO,
однако его применение обеспечивает улучшение качества продуктов
окисления. При этом достигается высокая степень окисления без
побочных реакций, таких как образование кокса, что подтверждает
высокую селективность катализатора по отношению к насыщенным
углеводородным фрагментам. Это является значительным преимуществом
для тяжёлых нефтяных остатков Каражанбасской нефти, богатых
парафиновыми структурами.

Каталитическая схема окисления не только улучшает физико-химические
свойства битума, но и обеспечивает существенный ресурсосберегающий
эффект.

В целом анализ показывает, что каталитическое окисление с
использованием небольшого количества KMnO₄ приводит к формированию
стабильной коллоидной структуры битума с оптимальным соотношением
асфальтеновой и смолистой фаз. Полученные результаты подтверждают
эффективность применения перманганата калия в качестве
катализатора окисления тяжёлых нефтяных остатков и обеспечивают
стабильные эксплуатационные характеристики готового продукта.

{\bfseries Выводы.} В ходе проведённых исследований изучено влияние
предварительного каталитического окисления парафинсодержащей
затемнённой фракции Каражанбасской нефти и последующего
совместного окисления с гудроном Каражанбасской нефти на
формирование структурно-механических и физико-химических свойств
полученного окисленного битума.

В качестве катализатора на стадии предварительной обработки сырья
до основного процесса окисления использовался перманганат калия
(KMnO₄). Такой подход позволил целенаправленно регулировать
соотношение смолисто-асфальтеновых соединений и исключить
необходимость стадии «избыточного окисления-разбавления»,
характерной для традиционных схем.

Использование KMnO₄ в качестве катализатора способствует протеканию
реакций частичного окисления парафиновых соединений, что, в свою
очередь, обеспечивает образование устойчивых асфальтено-смолистых
структур. В результате достигается оптимальное соотношение между
коллоидно-дисперсной и гелеобразной фазами, повышающее термическую
стабильность и пластичность готового продукта.

Физико-химические показатели полученных битумов-глубина проникания
иглы, температура размягчения, растяжимость и температура
хрупкости-находятся в оптимальном диапазоне и соответствуют
требованиям СТ РК 1373-2013 для битумов марки БНД 70/100.
Результаты лабораторных испытаний подтверждают, что применение
каталитической стадии обеспечивает более высокий уровень
структурной однородности и стабильности битумов.

Установлено, что предварительное каталитическое окисление
затемнённой фракции повышает технологическую гибкость процесса,
что позволяет регулировать структуру и свойства битума в широких
пределах без существенных изменений температуры и
продолжительности окисления. Таким образом, предложенный подход
обеспечивает стабильность качества готового продукта.

Практическая значимость результатов заключается в том, что
предложенный метод может быть внедрён на действующих битумных
заводах без значительных изменений технологического оборудования,
что делает процесс привлекательным для модернизации
отечественного производства дорожных битумов и повышения его
ресурсной эффективности.

Полученные битумы характеризуются улучшенными эксплуатационными
свойствами, что особенно важно для континентального климата
Казахстана и его дорожных условий. Применение каталитически
окисленных битумов может способствовать увеличению прочности
асфальтобетонных покрытий и снижению затрат на их ремонт и
содержание.

Разработка научно-технических основ промышленного внедрения
каталитических технологий окисления гудронов на отечественных
нефтеперерабатывающих заводах представляется целесообразной и
позволит повысить конкурентоспособность и качество казахстанских
дорожных материалов.

{\bfseries Литература}

1. Как в Казахстане планируют решать проблему дефицита битума.URL:

URL:
\href{https://www.inform.kz/ru/kak-v-kazahstane-planiruyut-reshat-problemu-defitsita-bituma-6a3972}{\ul{https://www.inform.kz/ru/kak-v-kazahstane-planiruyut-reshat-problemu-defitsita-bituma-6a3972}}.-
Дата обращения: 07.09.2025.

2. Аналитический отчет по рынку битума в Республике Казахстан
{[}Электронный ресурс{]}. URL:
\href{https://ccx.kz/analytical-report-bitumen-market}{\ul{https://ccx.kz/analytical-report-bitumen-market}}.-
Дата обращения: 16.09.2025.

3. Djimasbe R., Galiullin E.A., Varfolomeev M.A., et.al. Experimental
study of non-oxidized and oxidized bitumen obtained from heavy oil//Sci.
Rep. -2021.-Vol.11(1).-P.1-10. DOI 10.1038/s41598-021-87398-2.

4. Тлехусеж М.А., Дубиненко Н.А. Влияние химических процессов
производства строительных битумов на их свойства//Научные труды КубГТУ.
-2024. -№ 2.- C.50-58. DOI
\href{http://dx.doi.org/10.26297/2312-9409.2024.2.4}{\ul{10.26297/2312-9409.2024.2.4}}.

5. Khusnutdinov I.,~ Goncharova I.,~ Safiulina A. Extractive
deasphalting as a method of obtaining asphalt binders and low-viscosity
deasphalted hydrocarbon feedstock from natural
bitumen//\href{https://www.sciencedirect.com/journal/egyptian-journal-of-petroleum}{Egyptian
Journal of Petroleum}.-2021.-Vol.30(2).-P.69-73.
\href{https://doi.org/10.1016/j.ejpe.2021.03.002}{\ul{DOI
10.1016/j.ejpe.2021.03.002}}.

6. Кутьин Ю.А., Теляшев Э.Г. Битумы и битумные материалы. Нормативы,
качество, технологии / Ю.А. Кутьин, Э.Г. Теляшев. ГУП ИНХП РБ. -2018.-
272 c. ISBN 978-5-902159-56-8.

7. Abbasov V.M., Efendieva L.M., Cherepnova Yu.P., Nasibova G.G., et al.
Commercial Production of Bitumen via Oxidation of Vacuum Residue over
Iron Catalysts/Petroleum Chemistry. -2025.-Vol.65(3). -
P.298-303. DOI 10.1134/S0965544125600596.

8. Грудников И.Б. Нефтяные битумы. Процессы и технологии производства /
И.Б. Грудников.-Уфа: ГУП ИНХП РБ. -2015.- 288 c. ISBN 978-5-902159-51-3.

9. Евдокимова Н.Г., Прозорова О.Б., Буканова С.К. Совершенствование
технологии производства окисленных битумов как способ повышения
эффективности энергосбережения //Нефть и газ. - 2015. - №1. - С.43-52.

10. Лю Инчжоу,~МалгаждароваН.С.,~Онгарбаев Е.К.,~Тлеуберды Е., Акказин
Е.А.,Умбеткалиев К.А. Исследование влияния хлористого железа на свойства
окисленных битумов. - 2016.

URL:
\ul{\url{https://www.rusnauka.com/18_NPN_2016/Chimia/8_213557.doc.htm}.-}
Дата обращения: 16.09.2025.

11. Онгарбаев Е.К., Жамболова А.Б., Тилеуберди Е. и др. Окисление
тяжелых нефтяных остатков в присутствии катализаторов и модификаторов //
Горение и плазмохимия.-2019. -№17.-С.47-56.
DOI~\href{https://doi.org/10.18321/cpc286}{\ul{10.18321/cpc286}}

12. Абилхайров А.И., Шамбилова Г.К., Буканова С.К., Карабасова Н.А.,
Кайрлиева Ф.Б., Наурызбаева А.Д., Сакипова Л.Б., Имангалиева Ф.Б. Способ
получения окисленного битума: патент на полезную модель №10821.
Казахстан, опубл.04.07.2025.

{\bfseries References}

1. Kak v Kazahstane planirujut reshat'{} problemu
deficita bituma.URL:

URL:
https://www.inform.kz/ru/kak-v-kazahstane-planiruyut-reshat-problemu-defitsita-bituma-6a3972.-
Data obrashhenija: 07.09.2025.{[}in Russian{]}

2. Analiticheskij otchet po rynku bituma v Respublike Kazahstan
{[}Jelektronnyj resurs{]}. URL:
https://ccx.kz/analytical-report-bitumen-market.- Data obrashhenija:
16.09.2025. {[}in Russian{]}

3. Djimasbe R., Galiullin E.A., Varfolomeev M.A., et.al. Experimental
study of non-oxidized and oxidized bitumen obtained from heavy oil//Sci.
Rep. -2021.-Vol.11(1).-P.1-10. DOI 10.1038/s41598-021-87398-2.

4. Tlehusezh M.A., Dubinenko N.A. Vlijanie himicheskih processov
proizvodstva stroitel' nyh bitumov na ih
svojstva//Nauchnye trudy KubGTU. -2024. -№ 2.- C.50-58. DOI
10.26297/2312-9409.2024.2.4.{[}in Russian{]}

5. Khusnutdinov I.,~ Goncharova I.,~ Safiulina A. Extractive
deasphalting as a method of obtaining asphalt binders and low-viscosity
deasphalted hydrocarbon feedstock from natural
bitumen//\href{https://www.sciencedirect.com/journal/egyptian-journal-of-petroleum}{Egyptian
Journal of Petroleum}.-2021.-Vol.30(2).-P.69-73.
\href{https://doi.org/10.1016/j.ejpe.2021.03.002}{\ul{DOI
10.1016/j.ejpe.2021.03.002}}.

6. Kut' in Ju.A., Teljashev Je.G. Bitumy i bitumnye
materialy. Normativy, kachestvo, tehnologii / Ju.A.
Kut' in, Je.G. Teljashev. GUP INHP RB. -2018.- 272 c.
ISBN 978-5-902159-56-8.{[}in Russian{]}

7. Abbasov V.M., Efendieva L.M., Cherepnova Yu.P., Nasibova G.G., et al.
Commercial Production of Bitumen via Oxidation of Vacuum Residue over
Iron Catalysts/Petroleum Chemistry. -2025.-Vol.65(3). -
P.298-303. DOI 10.1134/S0965544125600596.

8. Grudnikov I.B. Neftjanye bitumy. Processy i tehnologii proizvodstva /
I.B. Grudnikov.-Ufa: GUP INHP RB. -2015.- 288 c. ISBN 978-5-902159-51-3.
{[}in Russian{]}

9. Evdokimova N.G., Prozorova O.B., Bukanova S.K. Sovershenstvovanie
tehnologii proizvodstva okislennyh bitumov kak sposob povyshenija
jeffektivnosti jenergosberezhenija //Neft'{} i gaz. -
2015. - №1. - S.43-52. {[}in Russian{]}

10. Lju Inchzhou, MalgazhdarovaN.S., Ongarbaev E.K., Tleuberdy E.,
Akkazin E.A.,Umbetkaliev K.A. Issledovanie vlijanija hloristogo zheleza
na svojstva okislennyh bitumov. - 2016.

URL: https://www.rusnauka.com/18\_NPN\_2016/Chimia/8\_213557.doc.htm.-
Data obrashhenija: 16.09.2025. {[}in Russian{]}

11. Ongarbaev E.K., Zhambolova A.B., Tileuberdi E. i dr. Okislenie
tjazhelyh neftjanyh ostatkov v prisutstvii kataliz{[}in Russian{]}atorov
i modifikatorov // Gorenie i plazmohimija.-2019. -№17.-S.47-56. DOI
10.18321/cpc286.

12. Abilhajrov A.I., Shambilova G.K., Bukanova S.K., Karabasova N.A.,
Kajrlieva F.B., Nauryzbaeva A.D., Sakipova L.B., Imangalieva F.B. Sposob
poluchenija okislennogo bituma: patent na poleznuju
model'{} №10821. Kazahstan, opubl.04.07.2025. {[}in
Russian{]}

\emph{{\bfseries Сведения об авторах}}

Буканова СК.- старший преподаватель Атырауского университета нефти
и газа им. Сафи Утебаева, Атырау, Казахстан, e-mail:
\href{mailto:sauleshik81@mail.ru}{sauleshik81@mail.ru};

Шамбилова Г.К.- д.х.н., профессор Атырауского университета нефти и
газа им. Сафи Утебаева и Атырауского университета им. Х.
Досмухамедова, Атырау, Казахстан, e-mail:
shambilova\_gulba@mail.ru;

Абилхайров А.И.- к.х.н., ассоциированный профессор Атырауского
университета нефти и газа им. Сафи Утебаева, Атырау, Казахстан,
e-mail:
a.abilkhayrov@mail.ru;

БукановаА.С.- к.т.н., профессор Атырауского университета нефти и
газа им. Сафи Утебаева,Атырау, Казахстан, e-mail:
bukanova66@mail.ru;

Кайрлиева Ф.Б.- к.т.н., старший преподаватель Атырауского
университета нефти и газа им. Сафи Утебаева,

Атырау, Казахстан, e-mail:
kairliyeva.fazi@mail.ru;

Нургалиев Н.У.-к.х.н., ассоциированный профессор, Казахский
университет технологии и бизнеса им. К.Кулажанова, Астана,
Казахстан, e-mail: nurgaliev\_nao@mail.ru.

\emph{{\bfseries Information about the authors}}

Bukanova S.R.- Senior Lecturer, Safi Utebayev Atyrau oil and gas
university, Aty\tsp{}rau, Kazakhstan, e-mail:
\href{mailto:sauleshik81@mail.ru}{sauleshik81@mail.ru};

Shambilova G.K.- Doctor of Chemical Sciences, Professor, Safi
Utebayev of Atyrau oil and gas university and H. Dosmukhamedov
University, Aty\tsp{}rau, Kazakhstan, e-mail:
shambilova\_gulba@mail.ru;

Abilkhaiev A.I.- Candidate of Chemical Sciences, Associate
Professor, Safi Utebayev Atyrau oil and gas university,
Aty\tsp{}rau, Kazakhstan, e-mail:
a.abilkhayrov@mail.ru;

Bukanova A.S.- Candidate of technical Sciences, Professor, Safi
Utebayev Atyrau oil and gas university, Aty\tsp{}rau,
Kazakhstan, e-mail:
bukanova66@mail.ru;

Kairlieva F.B.- Candidate of technical Sciences, Senior Lecturer,
Safi Utebayev Atyrau oil and gas university,
Aty\tsp{}rau, Kazakhstan, e-mail:
kairliyeva.fazi@mail.ru;

Nurgaliyev N.U.- Candidate of Chemical Sciences, Associate
Professor, K.Kulazhanov Kazakh University of Technology and
Business, Astana„ Kazakhstan e-mail: nurgaliev\_nao@mail.ru.\
