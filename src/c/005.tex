\id{IRSTI 31.21.01}{}

\begin{header}
\swa{}{STUDY OF PHYSICOCHEMICAL PROPERTIES OF ADSORBENTS OBTAINED BASED ON TEXTILE CORD}

\tsp{1}M.K. Kazankapova\envelope,
\tsp{1}B.T. Yermagambet,
\tsp{1,2}U.M. Kozhamuratova,
\tsp{2}Zh.E. Jakupova\envelope
\end{header}

\begin{affil}
\tsp{1}«Institute of Coal Chemistry and Technology» LLP, Astana, Kazakhstan,

\tsp{2}«L.N. Gumilyov Eurasian National University, Astana, Kazakhstan

\corrauthor{Corresponding author: kozhamuratova.u@mail.ru, zhanereke@mail.ru}
\end{affil}

This paper examines the physicochemical principles and principles of
textile cord recycling to produce adsorbents. The accumulation of used
automobile tires and their use as secondary products, coupled with the
lack of suitable recycling processes and technologies, has prompted the
search for effective ways to recycle worn tires and use them as a
promising raw material to produce carbon-containing materials. Textile
cord, a byproduct of organic rubber waste recycling, was selected as a
substandard raw material to produce carbon-containing materials. The
structure formation processes of composites modified with textile wire
have been described inconsistently in terms of the mechanism of
formation, volume, type, size, and differential porosity. Therefore,
this paper presents a comprehensive study of the influence of additives
on the properties and structure of composites with the goal of producing
porous carbon adsorbents for water and gas purification from textile
cord from waste automobile tires.

{\bfseries Keywords:} automobile tire waste, activation, organic waste,
porous adsorbents, nanomaterials, carboni\-zation.

\begin{header}
ТОҚЫМА СЫМЫ НЕГІЗІНДЕ АЛЫНҒАН АДСОРБЕНТТЕРДІҢ ФИЗИКА-ХИМИЯЛЫҚ ҚАСИЕТТЕРІН ЗЕРТТЕУ

\tsp{1}М.Қ. Қазанқапова\envelope,
\tsp{1}Б.Т. Ермағамбет,
\tsp{1,2}Ұ.М. Қожамұратова,
\tsp{2}Ж.Е. Джакупова\envelope
\end{header}

\begin{affil}
\tsp{1}«Көмір химиясы және технология институты» ЖШС, Астана, Қазақстан,

\tsp{2}«Л.Н.Гумилев атындағы Еуразия ұлттық университеті» Астана, Қазақстан,

е-mail: kozhamuratova.u@mail.ru, zhanereke@mail.ru
\end{affil}

Жұмыста адсорбенттер алу үшін тоқыма сымдарын қайта өңдеудің
физика-химиялық принцип\-тері мен принциптері қарастырылады.\hfill Пайдаланылған\hfill
автомобиль\hfill шиналарының\hfill жинақталуы \\және оларды қайталама өнім ретінде
пайдалану, сәйкес қайта өңдеу процестері мен технологияларының жоқтығы
тозған шиналарды қайта өңдеудің тиімді жолдарын іздеуге және оларды
көміртегі бар материалдарды өндіру үшін перспективалық шикізат ретінде
пайдалануға түрткі болды. Құрамында көміртегі бар материалдарды өндіру
үшін сапасыз шикізат ретінде органикалық резеңке қалдықтарын қайта
өңдеудің жанама өнімі болып табылатын тоқыма шнуры таңдалды. Тоқыма
сымымен модификацияланған композиттердің құрылымды қалыптастыру
процестері түзілу механизмі, көлемі, түрі, өлшемі және дифференциалды
кеуектілігі тұрғысынан сәйкес келмейтін сипатталған. Сондықтан бұл
жұмыста автомобиль шиналарының қалдықтарынан тоқыма шнурларынан су мен
газды тазарту үшін кеуекті көміртекті адсорбенттерді алу мақсатымен
композиттердің қасиеттері мен құрылымына қоспалардың әсерін кешенді
зерттеу ұсынылады.

{\bfseries Түйін сөздер:} автомобиль шиналарының қалдықтары, активация,
органикалық қалдықтар, кеуекті адсорбенттер, наноматериалдар,
карбонизация.

\begin{header}
ИЗУЧЕНИЕ ФИЗИКО-ХИМИЧЕСКИХ СВОЙСТВ АДСОРБЕНТОВ, ПОЛУЧЕННЫХ НА ОСНОВЕ ТЕКСТИЛЬНОГО КОРДА

\tsp{1}М.К. Казанкапова\envelope,
\tsp{1}Б.Т. Ермагамбет,
\tsp{1,2}У.М. Кожамуратова,
\tsp{2}Ж.Е. Джакупова\envelope
\end{header}

\begin{affil}
\tsp{1}ТОО «Институт химии угля и технологии», Астана, Казахстан,

\tsp{2}«Евразийский национальный университет имени Л.Н. Гумилева» Астана, Казахстан,

е-mail: kozhamuratova.u@mail.ru, zhanereke@mail.ru
\end{affil}

В работе рассмотрены физико-химические основы и закономерности способа
переработки текстильного корда с целью получения адсорбентов. Проблема
накопления использованных автомобильных шин и применение их в качестве
вторичной продукции, отсутствие подходящих процессов и технологий
переработки явилось основой поиска эффективных способов переработки
изношенных шин, применение их в качестве перспективного сырья для
производства углеродсодержащих материалов. В качестве некондиционного
сырья для производства углеродсодержащих материалов выбран продукт
переработки органических отходов резины -- текстильный корд. Процессы
формирования структуры композитов, модифицированных текстильной
проволокой неоднозначно описаны по механизму возникновения
новообразований, объему, типу, размерам и дифференциальной пористости. В
связи с этим, в работе проведены комплексные исследования влияния
добавок на свойства и структуру композитов с целью получения пористых
углеродных адсорбентов для очистки воды и газов из текстильного корда
отходов автомобильных покрышек.

{\bfseries Ключевые слова:} отходы автомобильных шин, активация,
текстильный корд, пористые адсорбенты, наноматериалы, карбонизация.

\begin{multicols}{2}
{\bfseries Introduction.} Today, one of the fastest-growing industries
globally is tire manufacturing. The global tire market reached 2,268
million units in 2021 and is projected to reach 2,665 million units by
2027. According to the European Tire and Rubber Manufacturers
Association, the number of end-of-life tires amounted to 3.56 million
tons in 2019, which is approximately 1\% less than the previous year
{[}1-3{]}. Overall, 65\% of rubber products are used in automotive
applications, and the recycling and reuse of partially worn tires in the
domestic market remain at a very low level {[}4,5{]}.

Porous carbon-based materials are obtained through thermal treatment or
activation using various oxidizing agents. These materials act as
sorbents, enabling the efficient separation of gas and liquid mixtures,
while their sorption properties enhance separation capabilities. Such
carbon materials are widely used as sorbents, catalyst carriers,
nanocomposite materials, and substrates in next-generation energy
storage devices such as lithium-ion batteries, supercapacitors,
ionistors, and fuel cells {[}6-9{]}.

Currently, the development of carbon adsorbents derived from automotive
tire waste is rapidly advancing. This direction helps address waste
recycling challenges while producing efficient materials.

For example, in studies on Turkic peoples, textile cord was processed
using the pyrolysis method, followed by chemical activation with
potassium hydroxide (KOH). It was found that these materials effectively
remove organic pollutants, particularly methylene blue, from aqueous
solutions. The best results were observed at a pH of 6.5, highlighting
the importance of controlling acidity in adsorption processes {[}10{]}.
Additionally, other researchers have investigated the role of pyrolysis
temperature and duration. It was shown that increasing the temperature
improves pore structure and increases the specific surface area, which
is a crucial factor for gas storage or water purification applications
{[}11{]}.

Our activation method for textile cord stands out from other approaches,
as it provides a high carbon content (86.53\%) along with enrichment in
mineral elements (Si, Ca, Fe, V, P) that create additional active sites.
Moreover, the liquid product is rich in aromatic compounds (toluene,
ethylbenzene, etc.), giving it added value. Unlike many studies focused
only on surface area, our approach combines high activation efficiency
with the practical significance of both solid and liquid products.

{\bfseries Materials and methods.} The following equipment was used to
carry out this research: a laboratory quartz reactor, a BR-12NRT rotary
tube furnace, a "Thermostep Eltra" thermogravimetric analyzer, a
Kristallux-4000M gas chromatograph, an ES-20/60 shaker-incubator, a
PD-303 spectrophotometer, a 150-MI pH meter, a C-2204 centrifuge, a
"GRAD" ultrasonic bath, and a Quanta 3D 200i EDAX scanning electron
microscope.

To obtain carbon adsorbents, the activation process was carried out in a
laboratory setting at a temperature of 800°C in an inert atmosphere
(nitrogen gas). The physicochemical properties were analyzed using a
thermogravimetric analyzer, while the adsorption properties were
determined in the presence of methyl orange. The technological process
was conducted in two stages: carbonization (at 700°C) to remove highly
volatile components and achieve a uniformly distributed large-pore
structure, followed by activation (at 900°C) to develop a microporous
structure. The experiments were performed in a laboratory-scale
steam-gas activation unit.

The methods for determining the properties included moisture content,
ash content, volatility, pH value of the aqueous extract, bulk density,
adsorption activity with methyl orange, pore volume in water, elemental
composition, and specific surface area (BET method).

Textile cord consists of several primary materials, each influencing its
mechanical properties and strength. The main component is polyester,
accounting for 50--70\% of the composition, providing tensile strength.
Polyamide (nylon) constitutes 10--20\%, enhancing mechanical properties
and wear resistance. Steel wires make up 10--15\%, adding extra hardness
and durability.

{\bfseries Results and discussion.} The technological process was carried
out in two stages: carbonization (to remove volatile components and
produce a large-porous structure uniformly distributed throughout the
volume) and activation (to produce a microporous structure). Experiments
were conducted on a laboratory steam-gas activation unit (Fig.1).
\end{multicols}

\fig[0.7\textwidth]{c2/image31}[Fig.1 - Schematic diagram of a laboratory setup for
steam-gas activation:\\\normalfont{\emph{1 - gas cylinder (nitrogen),
2 - steam generator, 3 -- reactor, 4 -- LATR, 5 - temperature sensor,
6 - direct condenser, 7 - flask for cleaning gas from resins, 8 - gas
outlet}}]

\begin{multicols}{2}
Carbonization was carried out in an inert argon atmosphere at
temperatures ranging from 400 to 900°C for 60 minutes. Argon was
supplied from a cylinder (1) to the reactor at a set flow rate of 20
ml/min, which was adjusted using a flow meter. After leaving the
reactor, the gas was directed to a condenser (6), from which a portion
of the condensed gas flowed into a flask (7) to remove resinous
substances. Unreacted gases were discharged to the ventilation system
(8).

The next step in adsorbent preparation was activation with water vapor
(flow rate of 10 ml/min) at the maximum temperature for 60 minutes to
improve its adsorption properties. Table 1 shows the temperature
dependences of the components of the gas obtained as a result of
carbonization and activation of nylon cord. The formation of combustible
gas components (CO, H\tsb{2}, CH\tsb{4}) occurs
according to the following main chemical reactions:

\begin{equation}
2\mathrm{H}_2\mathrm{O} \rightarrow 2\mathrm{H}_2 + \mathrm{O}_2 -115700~\text{kcal}
\end{equation}

\begin{equation}
\mathrm{C} + \mathrm{H}_2\mathrm{O} \rightarrow \mathrm{CO} + \mathrm{H}_2 -28150~\text{kcal}
\end{equation}

\begin{equation}
\mathrm{C} + \mathrm{CO}_2 \rightarrow 2\mathrm{CO} -38400~\text{kcal}
\end{equation}

\begin{equation}
\mathrm{C} + 2\mathrm{H}_2 \rightarrow \mathrm{CH}_4 +18600~\text{kcal}
\end{equation}

When heated above 200℃, textile cord begins to decompose, forming a
flammable gas containing hydrogen, carbon monoxide, alkanes and alkenes.
\end{multicols}

\tcap{Table 1 - Gas composition of carbonization and activation of textile cord}
\begin{longtblr}[
  label = none,
  entry = none,
]{
  width = \linewidth,
  colspec = {Q[162]Q[71]Q[79]Q[67]Q[67]Q[79]Q[67]Q[67]Q[69]Q[69]Q[69]},
  cells = {c},
  cell{1}{1} = {r=2}{},
  cell{1}{2} = {r=2}{},
  cell{1}{3} = {c=9}{0.708\linewidth},
  cell{3}{1} = {r=5}{},
  cell{8}{1} = {r=3}{},
  cells = {font = \small},
  vlines,
  hline{1,3,8,11} = {-}{},
  hline{2} = {3-11}{},
  hline{4-7,9-10} = {2-11}{},
}
Process       & Т1 С° & \textbf{Gas composition (volume, \%)} &       &       &        &       &       &       &       &             \\
              &       & $\mathrm{O}_2$ & $\mathrm{H}_2$ & $\mathrm{CO}_2$ & $\mathrm{N}_2$ & $\mathrm{CH}_4$ & $\mathrm{CO}$ & $\mathrm{C}_2\mathrm{H}_6$ & $\mathrm{C}_2\mathrm{H}_4$ & $\mathrm{C}_3\mathrm{H}_8 + \mathrm{C}_3\mathrm{H}_6$ \\
Carbonization & 200   & 37.003                                & 0.046 & 0.126 & 32.911 & 0.166 & 0.040 & 0.005 & -     & 0.589       \\
              & 300   & 37.874                                & 0.259 & 1.139 & 68.089 & 0.083 & -     & 0.009 & 0.335 & 0.020       \\
              & 400   & 43.355                                & 0.237 & 2.219 & 66.944 & 0.045 & -     & 0.036 & 0.680 & 0.054       \\
              & 500   & 34.788                                & 3.380 & 1.874 & 58.749 & 1.082 & -     & 1.095 & 0.705 & 1.012       \\
              & 600   & 34.526                                & 3.091 & 1.671 & 61.023 & 2.570 & -     & 0.918 & 0.565 & 1.012       \\
Activation    & 700   & 30.469                                & 3.873 & 0.411 & 76.123 & 0.746 & -     & 0.137 & 0.036 & 0.162       \\
              & 800   & 28.204                                & 5.608 & 0.302 & 75.943 & 0.813 & 0.436 & 0.066 & 0.080 & 0.082       \\
              & 900   & 27.532                                & 3.304 & 0.157 & 79.477 & 0.841 & 0.511 & 0.038 & 0.058 & 0.006       
\end{longtblr}

\tcap{Table 2 - Material Balance of Textile Cord after activation}
\begin{longtblr}[
  label = none,
  entry = none,
]{
  cells = {c},
  cells = {font = \small},
  cell{2}{1} = {r=3}{},
  cell{5}{1} = {r=4}{},
  vlines,
  hline{1-2,5,9} = {-}{},
  hline{3-4,6-8} = {2-4}{},
}
                 & \textbf{Name}              & \textbf{Massa, g} & \textbf{Amount, \%} \\
Initial Material & Textile cord               & 200.00            & 66.67               \\
                 & Water                      & 100.00            & 33.33               \\
                 & Total                      & 300.00            & 100.00              \\
Product          & Solid Residue (adsorbent)  & 68.40             & 22.80               \\
                 & Gas                        & 201.66            & 67.22               \\
                 & Liquid product (water+tar) & 29.94             & 9.98                \\
                 & Total                      & 300.00            & 100.00              
\end{longtblr}

\begin{multicols}{2}
The analysis of gaseous products shows that during carbonization an
increase in temperature leads to higher hydrogen release (up to 3.091\%
at 600 °C) and formation of light hydrocarbons (CH₄, C₂--C₃), while
oxygen decreases. At the same time, nitrogen remains the dominant
component of the mixture. During activation (700-900 °C), the gas phase
is characterized by high nitrogen content (over 75\%) along with
noticeable hydrogen formation (up to 5.608\% at 800 °C) and small
amounts of CO and hydrocarbons. These results indicate that thermal
treatment intensifies decomposition processes and promotes gas
evolution, enriching the system with hydrogen and light hydrocarbons.

As a model system, automotive tire waste was crushed and sieved to
separate the rubber granules and textile cord. The obtained textile cord
underwent carbonization and activation by being processed in a pyrolysis
furnace at 900°C for 4 hours. The material balance was then calculated
(Table 2).

The developed system boasts a simple and reliable design, as well as an
environmentally friendly technology. The gas phase and solid residue are
used as adsorbents for water and gas purification, and various
commercial products can be obtained from the liquid fraction, which is a
mixture of hydrocarbons. No wastewater is generated during operation.

During the carbonization and activation of the textile cord, the
vapor-gas mixture was diverted to a refrigerator, where the hydrocarbon
vapors condensed, forming a resin. The liquid products obtained in this
process at temperatures ranging from 400 to 900°C were distilled on a
rotary evaporator (at temperatures of T = 55-61°C under vacuum), with
the hydrocarbon fractions collected. Figure 2 shows the chromatograms of
the liquid products obtained from the carbonization and activation of
the textile cord.
\end{multicols}

\fig[0.7\textwidth]{c2/image32}[Fig.2 - Chromatogram of liquid products from carbonization and activation of nylon cord]

\begin{multicols}{2}
According to the analysis results, the component composition of the
liquid hydrocarbon fractions formed during the carbonization and
activation of the textile cord is predominantly toluene with a retention
time of 30.414 min and a concentration of 10.525\%, indicating its
dominant content in the mixture. The second most abundant component is
2-methyl-3-ethylpentane (3.329\%, retention time 32.108 min). Aromatic
hydrocarbons were detected in smaller quantities: ethylbenzene (2.093\%,
47.718 min) and 1-methyl-3-n-butylbenzene (1.018\%, 80.714 min), as well
as other compounds in trace amounts. The obtained results indicate the
predominance of aromatic compounds, among which toluene is the main
representative, which can have a significant impact on the
physicochemical properties of the studied sample.

The elemental composition of the initial product and the adsorbent based
on textile cord were studied using a portable X-ray fluorescence
analyzer (Table 3).
\end{multicols}

\tcap{Table 3 - Elemental composition obtained using a portable X-ray fluorescence instrument}
\begin{longtblr}[
  label = none,
  entry = none,
]{
  cells = {c},
  cells = {font = \small},
  cell{1}{1} = {r=2}{},
  cell{1}{2} = {c=2}{},
  cell{1}{4} = {c=2}{},
  vlines,
  hline{1,3-14} = {-}{},
  hline{2} = {2-5}{},
}
\textbf{Elemental composition} & \textbf{Initial textile cord} &              & \textbf{Activated textile cord} &              \\
                               & \textbf{\%}                   & \textbf{$\pm 2\sigma$} & \textbf{\%}                     & \textbf{$\pm 2\sigma$} \\
С                              & 77.89                         & 0.57         & 86.53                           & 0.36         \\
Zn                             & 4.74                          & 0.14         & -                               & -            \\
S                              & 5.30                          & 0.15         & 6.01                            & 0.15         \\
Si                             & 0.636                         & 0.040        & 1.95                            & 0.07         \\
Cl                             & 0.260                         & 0.010        & 0.311                           & 0.010        \\
K                              & 0.310                         & 0.20         & 0.476                           & 0.020        \\
Ca                             & 0.721                         & 0.031        & 1.74                            & 0.06         \\
Fe                             & 0.603                         & 0.028        & 3.06                            & 0.08         \\
Ti                             & 0.153                         & 0.067        & 0.183                           & 0.064        \\
V                              & -                             & -            & 0.054                           & 0.033        \\
P                              & -                             & -            & 0.055                           & 0.031        
\end{longtblr}

\begin{multicols}{2}
The elemental composition analysis of the original and activated textile
cord revealed significant changes after activation. The carbon content
increased from 77.89\% to 86.53\%, indicating enrichment of the carbon
matrix due to the removal of volatile compounds and inorganic
impurities. Sulfur slightly increased (5.30\% to 6.01\%), while zinc,
initially present at 4.74\%, completely disappeared, suggesting
volatilization during thermal treatment. At the same time, mineral
components became more concentrated: silicon rose from 0.636\% to
1.95\%, calcium from 0.721\% to 1.74\%, and iron from 0.603\% to 3.06\%,
with minor increases in chlorine, potassium, and titanium. Notably,
vanadium (0.054\%) and phosphorus (0.055\%) appeared only in the
activated material, likely due to redistribution of trace elements or
impurities introduced during activation. Overall, the activation process
not only enriched the carbon structure but also significantly
transformed the mineral profile of the textile cord, which may strongly
influence its adsorption and catalytic properties.

Figures 3 and 4 present microscopic images of the initial textile cord
and the activated adsorbent, respectively.

In Figure 3, fiber particles with a diameter of 10--25 µm are clearly
visible, with structural elements forming fibrils in the form of
thread-like formations.

As shown in Figure 4, the analysis of micrographs reveals that after
thermal treatment, the surface structure changes, particle sizes
decrease (\textasciitilde145 nm), and fine-dispersed carbon
nanoparticles with diameters ranging from approximately 70 to 600 nm are
formed. This may be associated with the interaction of reactive radicals
resulting from carbonization and activation, leading to the formation of
new compounds. The most likely cause of nanoparticle formation on the
surface is the synthesis process from the gas phase.

During thermal treatment, the formation and growth of ordered carbon in
the textile cord may occur through the self-organization of carbon
nanoparticles without the involvement of a mesophase. However, further
studies are needed to clarify this issue.
\end{multicols}

\begin{figs}[Fig.3 - SEM results of the Initial Textile Cord: а-х500; b-х5000]
\fig[0.45\textwidth][0.75\textwidth]{c2/image33}[a]
\fig[0.45\textwidth][0.75\textwidth]{c2/image34}[b]
\end{figs}
\vspace{-1em}
\begin{figs}[Fig.4 - SEM results of the Activated Textile Cord: а-х500; b-х5000]
\fig[0.45\textwidth][0.75\textwidth]{c2/image35}[a]
\fig[0.45\textwidth][0.75\textwidth]{c2/image36}[b]
\end{figs}

\begin{multicols}{2}
Thus, the method of processing textile cords through carbonization and
activation in the recycling of automobile tires not only ensures the
disposal of worn-out tires but also enables the production of materials
that can be used in various industries.

{\bfseries Conclusion.} Traditional methods for purifying natural objects
are not always effective and may harm the environment. While using
organic waste for wastewater and air purification is economically and
environmentally beneficial, such materials generally lack the required
sorption capacity, necessitating carbonization and activation. As a
result of this process, new sorbents are obtained, combining the
beneficial properties of the original material with the characteristics
of synthetic sorbents, and having a surface different from the initial
mineral.

Thus, the proposed method for processing textile cords derived from
automobile tire recycling through carbonization and activation not only
facilitates the disposal of worn-out tires but also enables the
production of materials that can be used in various sectors of the
economy.

\emph{{\bfseries Acknowledgement.} This research has been funded by the
Science Committee of the Ministry of Science and Higher Education of the
Republic of Kazakhstan (Grant No. AP19577512 "Development of scientific
and technical bases for obtaining microporous carbon nanomaterials for
hydrogen separation and storage").}
\end{multicols}

\begin{center}
{\bfseries References}
\end{center}

\begin{refs}
1. Aylón E., Murillo R., Navarro M.V., García T., Mastral A.M.
Valorisation of waste tyre by pyrolysis in a moving bed reactor//Waste
Management.-2010.-Vol.30.- P.1220 - 1224. DOI
10.1016/j.wasman.2009.10.001.

2. Williams P.T. Pyrolysis of waste tyres: A review//Waste
Management.~-2013.-Vol.-33. -P.1714-1728. DOI
10.1016/j.wasman.2013.05.003.

3. Choi G., Oh S., Kim J. Clean pyrolysis oil from a continuous two-stage
pyrolysis of scrap tires using in-situ and ex-situ
desulfurization//~Energy.-2017.-Vol.141 (C).- P.2234 - 2241. DOI
10.1016/j.energy.2017.12.015.

4. Murugan S., Ramaswamy M.C., Nagarajan G. A comparative study on the
performance, emission and combustion studies of a DI diesel engine using
distilled tyre pyrolysis oil--diesel blends //Fuel. -2008. -Vol.87.- P.
2111-2121. DOI 10.1016/j.fuel.2008.01.008.

5. Jahirul M.I., Hossain F.M., Rasul M.G., Chowdhury A.A. A Review on the
Thermochemical Recycling of Waste Tyres to Oil for Automobile Engine
Application//Energies. -2021. -Vol.14(13): 3837. DOI
10.3390/en14133837.

6. Danon B., van der Gryp P., Schwarz C.E., Görgens J.F. A review of
dipentene (DL-limonene) production from waste tire pyrolysis//~Journal
of Analytical and Applied Pyrolysis.- 2015.-Vol.112.- P.1-13. DOI
10.1016/j.jaap.2014.12.025.

7. Yazdani E., Hashemabadi S.H., Taghizadeh A. Study of waste tire
pyrolysis in a rotary kiln reactor in a wide range of pyrolysis
temperature//~Waste Management. - 2019.- Vol.85.- P.195-201. DOI
10.1016/j.wasman.2018.12.020.

8. Mudhoo A., Sillanpää M. Magnetic nanoadsorbents for micropollutant
removal in real water treatment: A review//Environmental Chemistry
letters.-2021.-Vol.19(6). -P.1145-1168. DOI 10.1007/s10311-021-01289-6.

9. Sadegh H., Ghoshekandi R.S., Masjedi A., Mahmoodi Z., Kazemi M. A
review on Carbon nanotubes adsorbents for the removal of pollutants from
aqueous solutions// International Journal of Nano Dimension.
-2016.-Vol.7(2).- P.109-120. DOI 10.7508/ijnd.2016.02.002.

10. Islam M. T. et al. Conversion of waste tire rubber into a
high-capacity adsorbent for the removal of methylene blue, methyl
orange, and tetracycline from water //Journal of Environmental Chemical
Engineering.-2018.-Vol.6(2). - P.3070-3082.
\href{https://doi.org/10.1016/j.jece.2018.04.058}{DOI
10.1016/j.jece.2018.04.058}.

11. \href{https://link.springer.com/article/10.1134/S1070427218100063\#auth-S_-Chouaya-Aff1}{Chouaya}
S.,~\href{https://link.springer.com/article/10.1134/S1070427218100063\#auth-M__A_-Abbassi-Aff2}{Abbassi}
M.
A.,~\href{https://link.springer.com/article/10.1134/S1070427218100063\#auth-R__B_-Younes-Aff1}{Younes}
R. B.,~
\href{https://link.springer.com/article/10.1134/S1070427218100063\#auth-A_-Zoulalian-Aff3}{Zoulalian}
A. Scrap tires pyrolysis: Product yields, properties and chemical
compositions of pyrolytic oil //Russian Journal of Applied Chemistry. -
2018. - Vol.91(10). - P.1603-1611. DOI 10.1134/S1070427218100063.
\end{refs}

\begin{info}
\hspace{1em}\emph{{\bfseries Сведения об авторах}}

Казанкапова М.К. - PhD, асс. профессор, чл.-корр. КазНАЕН, ведущий
научный сотрудник, ТОО «Институт химии и технологии угля», Астана,
Казахстан, e-mail: maira\_1986@mail.ru;

Ермагамбет Б.Т.- доктор химических наук, профессор, академик КазНАЕН,
ТОО «Институт химии и технологии угля», Астана, Казахстан, e-mail:
bake.yer@mail.ru;

Қожамұратова Ұ.М. - магистрант, Евразийский национальний университет
им.  Л.Н.Гумилева, Астана, Казахстан, e-mail: kozhamuratova.u@mail.ru;

Джакупова Ж.Е. - кандидат химических наук, доцент, Евразийский
национальний университет им. Л.Н. Гумилёва, Астана, Казахстан, e-mail:
zhanereke@mail.ru.

\hspace{1em}\emph{{\bfseries Information about the authors}}

Kazankapova M.K.- PhD, Associate Professor, Corresponding Member of
KazNAEN, LLP "Institute of Coal Chemistry and Technology", Astana,
Kazakhstan, e-mail: maira\_1986@mail.ru;

Yermagambet B.T. - Doctor of Chemical Sciences, Professor, Academician
of KazNAEN, LLP "Institute of Coal Chemistry and Technology", Astana,
Kazakhstan, e-mail: bake.yer@mail.ru;

Kozhamuratova U.M. - master student, Eurasian National University of
L.N. Gumilyov, Astana, Kazakhstan, e-mail: kozhamuratova.u@mail.ru;

Jakupova Zh.E. - Candidate of Chemical Sciences, Associate Professor
Eurasian National University of L.N. Gumilyov, Astana, Kazakhstan,
e-mail: zhanereke@mail.ru.
\end{info}
