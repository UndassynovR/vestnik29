\id{МРНТИ 61.35.29}{}

{\bfseries АНАЛИЗ ПЕРСПЕКТИВЫ ПРИМЕНЕНИЯ ЗОЛЫ-УНОСА В ПОЛУЧЕНИИ
КОМПОЗИЦИОННЫХ МАТЕРИАЛОВ НА ПРИМЕРЕ ОТХОДОВ ЭНЕРГЕТИКИ РЕСПУБЛИКИ
КАЗАХСТАН}

{\bfseries \tsp{1}С.С.
Маусумбаев}{\bfseries ,\tsp{1}М.А.
Елубай}{\bfseries ,\tsp{1}Д.Т.
Толегенов}{\bfseries \envelope ,\tsp{1}С.Р.Масакбаева},

{\bfseries \tsp{2}Г.С.
Айткалиева}

{\bfseries \tsp{1}}\emph{Торайгыров университет, Казахстан,}

{\bfseries \tsp{2}}\emph{Сатпаев университет,Казахстан}

\envelope Корреспондент-автор:
\href{http://www.dika-92@mail.ru}{www.dika-92@mail.ru}

В данной статье авторами рассматриваются современные методы анализы и
перспективы применения золы-уноса в получении керамических изделий.
Авторы исследовали золошлаковые отходы (ЗШО) ТЭЦ и ГРЭС, их
физико-химические и физико-механические свойства, а также полученных на
основе них образцов керамических материалов.

Как известно, в настоящее время более 80 \% всех работающих
теплоэлектроцентралей (ТЭЦ) и гидроэлектростанций (ГРЭС) в мире работают
все еще на угле. Применение угля как источника топлива на ТЭЦ и ГРЭС
связано все еще большими запасами угольных месторождений. Это говорит о
том, что ежегодно увеличиваются объемы складируемых золошлаковых отходов
(ЗШО). С увеличением объемов шламохранилищ увеличиваются территории и
бассейны для складирования образуемых золы-уноса и ЗШО. Данная проблема
касается и для нашей страны, где основная проблема состоит в их
утилизации и вторичном использовании.

В ходе выполнения научно-исследовательской работы авторами были получены
результаты физико-химических и физико-механических свойств образцов
керамических изделий и сделаны выводы о перспективах использования
золы-уноса в качестве добавочного агента в состав керамической массы. В
ходе исследований были подобраны оптимальные составы сырьевых материалов
на основе золы-уноса для получения образцов композиционных изделий.

{\bfseries Ключевые слова:} керамические материалы, зола-уноса,
теплоэлектроцентрали, золошлаковые отходы, гранулометрический анализ,
сканирующая электронная микроскопия, керамическая масса.

{\bfseries ҚАЗАҚСТАН РЕСПУБЛИКАСЫНЫҢ ЭНЕРГЕТИКА ҚАЛДЫҚТАРЫ МЫСАЛЫНДА
КОМПОЗИЦИЯЛЫҚ МАТЕРИАЛДАРДЫ АЛУДА КҮЛ-ТАСЫҒЫШТЫ ҚОЛДАНУ ПЕРСПЕКТИВАСЫН
ТАЛДАУ}

{\bfseries \tsp{1}С.С. Маусумбаев, \tsp{1}М.А.
Елубай, \tsp{1}Д.Т. Төлегенов\envelope ,
\tsp{1}С.Р. Масакбаева, \tsp{2}Г.С. Айткалиева}

\tsp{1}\emph{Торайғыров университеті, Павлодар, Қазақстан,}

\tsp{2}\emph{Сәтпаев университеті, Алматы, Қазақстан,}

e-mail: www.dika-92@mail.ru

Бұл мақалада авторлар керамикалық бұйымдарды өндіруде күлді жоюдың
қазіргі заманғы әдістері мен перспективаларын қарастырады. Авторлар ЖЭО
мен ГРЭС күл-қож қалдықтарын, олардың физика-химиялық және
физика-механикалық қасиеттерін, сондай-ақ олардан алынған керамикалық
материалдардың үлгілерін зерттеді.

Қазіргі уақытта әлемдегі барлық жұмыс істейтін жылу электр
орталықтарының (ЖЭО) және су электр станцияларының (СЭС) 80\% - дан
астамы әлі де көмірмен жұмыс істейтіні белгілі. ЖЭО мен СЭС-те көмірді
отын көзі ретінде пайдалану көмір кен орындарының әлі де үлкен
қорларымен байланысты. Бұл жыл сайын жиналатын күл-қож қалдықтарының
көлемі артып келе жатқанын көрсетеді. Шлам қоймаларының көлемінің
ұлғаюымен пайда болған күл-тасығыштарды және күл-қож қалдықтарын
сақтауға арналған аумақтар мен бассейндер ұлғаяды. Бұл мәселе біздің
еліміз үшін де қатысты, мұнда негізгі мәселе оларды кәдеге жарату және
қайталама пайдалану болып табылады.

Ғылыми-зерттеу жұмыстарын орындау барысында авторлар керамикалық
бұйымдар үлгілерінің физика-химиялық және физика-механикалық
қасиеттерінің нәтижелерін алды және күл-қоқысты керамикалық массаның
құрамына қосымша агент ретінде пайдалану перспективалары туралы
қорытынды жасады. Зерттеу барысында композициялық бұйымдардың үлгілерін
алу үшін күлге негізделген шикізаттың оңтайлы құрамы таңдалды.

{\bfseries Негізгі сөздер:} керамикалық материалдар, күл-тасы, жылу электр
орталықтары, күл-қож қалдықтары, гранулометриялық талдау, сканерлеуші
электронды микроскопия, керамикалық масса.

{\bfseries ANALYSIS OF THE PROSPECTS FOR THE USE OF FLY ASH IN THE
PRODUCTION OF COMPOSITE MATERIALS USING THE EXAMPLE OF ENERGY WASTE IN
THE REPUBLIC OF KAZAKHSTAN}

{\bfseries \tsp{1}S.S. Maussumbayev, \tsp{1}M.A.
Yelubay, \tsp{1}D.T. Tolegenov\envelope ,
\tsp{1}S.R. Masakbayeva, \tsp{2}G.S.
Aitkaliyeva}

{\bfseries \tsp{1}}\emph{Toraigyrov University, Pavlodar,
Kazakhstan,}

{\bfseries \tsp{2}}\emph{Satbayev University, Almaty,
Kazakhstan,}

e-mail: www.dika-92@mail.ru

In this article, the authors consider modern methods of analysis and
prospects for the use of fly ash in the production of ceramic products.
The authors investigated ash and slag waste Thermal power plants and
GRES, their physico-chemical and physico-mechanical properties, as well
as samples of ceramic materials obtained on the basis of them.

As you know, currently more than 80\% of all operating thermal power
plants (CHP) and hydroelectric power plants (GRES) in the world are
still coal-fired. The use of coal as a fuel source at thermal power
plants and GRES is still associated with large reserves of coal
deposits. This indicates that the volume of stored ash and slag waste is
increasing annually. With an increase in the volume of sludge storage
facilities, territories and basins for storing the generated fly ash and
waste water are increasing. This problem also applies to our country,
where the main problem is their disposal and recycling.

In the course of the research, the authors obtained the results of the
physico-chemical and physico-mechanical properties of ceramic product
samples and drew conclusions about the prospects of using fly ash as an
additive agent in the composition of ceramic mass. During the research,
optimal compositions of raw materials based on fly ash were selected to
obtain samples of composite products.

{\bfseries Keywords:} ceramic materials, fly ash, thermal power plants, ash
and slag waste, granulometric analysis, scanning electron microscopy,
ceramic mass.

{\bfseries Введение.} Несмотря на стремительное развитие современных
технологий в энергетической отрасли, уголь все же остается во всем мире
одним из основных источников энергии. Разумеется, как и любой другой вид
энергетических ресурсов, добыча и использование угля наносит вред
окружающей среде. Как известно, процесс сжигания угля сопровождается
производством, так называемых, шлаковых отходов {[}1{]}.

В настоящее время, развитие отрасли по производству керамических изделий
напрямую связано с непрерывной модернизацией существующих технологий
(процессов), освоением производства новых материалов, использованием
отходов местных производств в качестве сырьевых и добавочных материалов,
повышением экономичности и снижением энергоемкости действующих
производств. Тенденция к истощению природных ресурсов для производства
стеновой керамических изделий способствовала выявлению качественного
сырья из отходов теплоэнергетических производств. Поэтому одним из
перспективных направлений, как утверждают авторы {[}2{]}, является
использование золы гидроудаления от сжигания каменного угля на тепловых
электростанциях (ТЭС) в качестве одного из основного топливосодержащего
сырья. По мнению авторов, наработанная практика научных исследований и
промышленного использования золошлаковых смесей из отвалов ТЭС в
производстве эффективных стеновых глинозольных изделий показала не
только пригодность химико-минералогического состава сырья, но и
возможность энергосбережения при термической обработке {[}2{]}.

{\bfseries Материалы и методы.} В последние года в Казахстане резко
уменьшилась доля переработки и утилизации промышленных отходов. По
информации Бюро национальной статистики АСПиР РК (рисунок 1), в 2022
году в стране было образовано 888,1 млн тонн различного
производственного мусора - на 20,5\% больше, чем в 2017-м. В целом в
стране накоплено около 32 млрд тонн промышленных «хвостов» {[}3{]}.

\fig{c4/image38}{}

{\bfseries Рис.1 - Анализ образования, переработки и утилизации
промышленных отходов в Казахстане {[}3{]}}

Как известно, в Казахстане имеется значительное количество тепловых
электрических станций. Каждый год увеличивается количество золошлаковых
отходов (ЗШО), образующихся на ТЭЦ, ГРЭС, в котельных.
Топливно-электроэнергетический комплекс является одним из основных
«загрязнителей» окружающей природной среды. Ухудшение экологической
обстановки небезосновательно связано с загрязнением атмосферы. Ежегодный
выход золы и золошлаковых смесей в Казахстане при сжигании углей
составляет более 17 млн т, а в золоотвалах накоплено более 300 млн тонн
отходов золы (рисунок 2) {[}4{]}.

\fig{c4/image39}{}

{\bfseries Рис.2 - Образование золошлаковых отходов в областях Республики
Казахстан {[}4{]}}

Как утверждают авторы {[}4{]}, из золошлаковых отходов, вырабатываемых
ТЭЦ, в Казахстане перерабатывается около 8\% золы (менее 1,9 млн тонн)
на исследовательско-производственном уровне. Если использование ЗШО
останется на этом уровне, то к 2030 году объём накопленных отходов
достигнет 1 млрд тонн {[}3{]}. Конечно же, это очень низкий показатель
переработки для накопленных энергетических отходов. Поэтому переработка
ЗШО в экологически безопасные строительные материалы остается весьма
актуальной проблемой и для нашей страны.

Так, авторы {[}5{]} изучили экологические аспекты применения золы от
сжигания осадка сточных вод в производстве стеновой керамики, где
авторами изучены влияние золы от сжигания осадка сточных вод на
химический состав керамического кирпича. Также учеными {[}5{]} проведена
оценка эффективности методов снижения концентраций тяжелых металлов в
строительной керамике.

Немаловажным фактором и критерием для керамических изделий является еще
одно их свойство -- морозостойкость. В работе {[}6{]} авторы показывают
возможность применения пылей газоочистки производства ферросплавов
(ПГПФ) и жидкого стекла в качестве основных компонентов сырьевой смеси
для изготовления материала полусухого прессования, обладающего
повышенной морозостойкостью. Как отмечают авторы, важно, что с
повышением температуры до 850 \tsp{0}С существенно
возрастает количество промежуточных пор (рисунок 3). Таким образом,
развитая микропористость черепка предопределяет не только снижение
теплопроводности, но и повышение морозостойкости стеновых изделий
{[}6{]}.

\fig{c4/image40}{}

{\bfseries Рис.3 - Зависимость объема пор диаметром 0,5-10,0 мкм от
температуры тепловой обработки материала на основе ПГПФ и жидкого стекла
{[}6{]}}

Авторы {[}7{]} отмечают, что работа тепловых электрических станций
Казахстана характеризуется значительным количеством накопленных
золошлаковых отходов. Установлено, что хранение золошлаков вызывает
весьма существенное воздействие на окружающую среду в зоне их
расположения. Поэтому весьма актуальным является решение вопросов
снижения нагрузки на окружающую среду путем разработки технологий
утилизации золошлаков и использования их в дорожном строительстве.
Учеными разработана технология получения битумо - золошлакового вяжущего
для покрытия оснований автодорог и определены ее основные параметры.
Определено также, что применение золошлаков в качестве вяжущего для
асфальтового покрытия позволит повысить эксплуатационные характеристики
дорожного покрытия {[}7{]}.

Конечно же, применение промышленных отходов в самой материалоемкой
отрасли, например -- в производстве керамического кирпича может решить
многие задачи. К их числу можно отнести: экономия природного сырья,
утилизация многотоннажных отходов, снижение экологической напряженности,
а самое главное - расширение номенклатуры керамических изделий.
Известно, что качество керамических изделий определяются свойствами
применяемого сырья. В работе {[}8{]} авторами приведены сведения об
изменение свойств природного сырья для производства керамического
кирпича. Установлено, что разнообразное техногенное сырья может являться
хорошим дополнением к природному сырью при производстве керамического
кирпича.

В одном из исследований учеными был определён зерновой состав и
возможность использования золошлаковых отходов тепловых станций г.
Павлодара, г. Екибастуза, ТЭС №4 и Амгалан г. Уланбатыр (Монголия)
{[}8{]}. Зерновой состав определён методом лазерной дифракции (LSM24)
при помощи прибора сит электрического виброгрохота ВГ 028М (рисунок 4).

\fig{c4/image41}{}

{\bfseries Рис.4 - Интеграл и дифференциал график распределение зернового
состава золы ТЭЦ Казахстана и Монголии методом лазерной дифракций
{[}9{]}}

\fig{c4/image42}{}

{\bfseries Рис.5 - Интеграл график распределения зернового состава золы
ТЭС Казахстана и Монголии методом сит виброгрохот ВГ 028М {[}9{]}}

\fig{c4/image43}{}

{\bfseries Рис.6 - Интеграл график распределения зернового состава золы
ТЭС Казахстана и Монголии методом сит виброгрохот ВГ 028М {[}9{]}}

В результате опытов учеными установлено, что максимальная прочность
конструктивных ячеистых бетонов с содержанием менее 0,05 мм - 84,7 \% и
менее 0,2 мм - 93,8 \% мелким зерновым составом летучей золы
теплоэлектростанций №4 и Амгалан г. Уланбатыр. Летучая зола
теплоэлектрической станций Экибастуза имеет более крупный зерновой
состав и меньшую плотность, соответствует требованиям стандарта
газобетона (рисунки 5- 6) {[}9{]}.

Согласно ГОСТ 25818-91 {[}10{]} золы по виду сжигаемого угля
подразделяют на:

- антрацитовые, образующиеся при сжигании антрацита, полуантрацита и
тощего каменного угля (А);

- каменноугольные, образующиеся при сжигании каменного, кроме тощего,
угля (КУ);

- буроугольные, образующиеся при сжигании бурого угля (Б).

Золу-уноса в зависимости от химического состава подразделяют на типы:

- кислые (К) - антрацитовые, каменноугольные и буроугольные, содержащие
оксид кальция до 10 \%;

- основные (О) - буроугольные, содержащие оксид кальция более 10 \% по
массе {[}10{]}.

Применяемая в данной исследовательской работе зола-уноса представляет
собой отход от сжигания Экибастузских углей Павлодарских тепловых
электростанций {[}11{]}.

Исследование золы-уноса с Павлодарской ТЭЦ и Аксуской ГРЭС состояло из
следующих стадий:

- подготовка образцов;

- обжиг материалов;

- определение физико-механических свойств обожженных композиционных
материалов;

- определение физико-химических свойств обожженных композиционных
материалов.

Анализ исходной золы-уноса с Павлодарской ТЭЦ и Аксуской ГРЭС
осуществлялось следующими методами:

- пробоподготовка;

- рентгенофлуоресцентный анализ;

- инфракрасная Фурье-спектроскопия;

- рентгенофазовый анализ;

- гранулометрический анализ;

- сканирующая электронная микроскопия (CЭМ).

\emph{Пробоподготовка}

Перед проведением лабораторных исследований образцы были предварительно
высушены до постоянной массы в сушильном шкафу UT-4603 (Xieli
International Trading Co., Китай) при температуре 100 °C. Для проведения
фазового анализа образцы измельчались до фракции ≤ 0,5 мм, тогда как для
проведения силикатного анализа до фракции ≤ 15 мкм.

\emph{Рентгенофлуоресцентный анализ}

Перед проведением анализа проба высушивалась при температуре 105 °С в
течение 1 часа, затем прокаливалась при 1000 °С в течение 2,5 часов,
после чего смешивалась с флюсом (66,67 масс.\% тетрабората лития; 32,83
масс.\% метабората лития и 0,5 \% лития бромистого) в соотношении 1:9
(общий вес смеси составляет 5 г.). Смесь плавилась в платиновых тиглях в
индукционной печи Lifumat-2,0-Ox (Linn High Therm, Германия). Полученные
таким образом стекла анализировались на рентгенофлуоресцентном
спектрометре ARL-9900-XP (Thermo Electron Corporation, США). Для
построения градуировочных графиков использовались следующие стандартные
образцы состава горных пород: 313, Му-1, Му-3, Счт-1, Сду-1, Сг-1а,
Сг-2, Сг-3. Сгд-1, Сгх-1, Сгхм-2, Сгхм-3, Си-1, Снс-1, Снс-2, Ст-1, а
также химреактивы MgO (осч), Al\tsb{2}O\tsb{3}
(чда), SiO\tsb{2} (чда), CaSO\tsb{4} (чда), CaO
(чда), TiO\tsb{2} (чда), Cr\tsb{2}O\tsb{3}
(чда), Fe\tsb{2}O\tsb{3} (осч). Погрешность
определения не превышает таковую для второй категории точности по ОСТ
41-08-212-82.

\emph{Инфракрасная Фурье-спектроскопия}

Инфракрасные (ИК) спектры были получены на ИК-Фурье спектрометре FT-801
(Simex, Россия) с использованием приставки нарушенного полного
внутреннего отражения (НПВО) и подложки с кристаллом ZnSe при разрешении
4 см\tsp{-1} и количестве сканов, равным 50. Образцы
исследовались в виде порошка массой около 250 мг, предварительно
измельчённые до 50 мкм на дисковом истирателе ИД-65.

\emph{Рентгенофазовый анализ}

Рентгеновские дифрактограммы получены на дифрактометре ДРОН-8 (НПП
«Буревестник», Россия). При проведении съёмки использовалась
рентгеновская трубка с медным анодом. После проведения съемки были
получены цифровые рентгенограммы с изображением пиков, показывающих
интенсивность отражения рентгеновских лучей от плоских сеток структур
минералов под различными углами (2Θ\tsp{о}).

\emph{Гранулометрический анализ}

Гранулометрический (зерновой) анализ производился на лазерном
анализаторе частиц Ласка-ТД (Биомедицинские системы, Россия). Было
выполнено измерение трех образцов подготовленных суспензий (λ от 650 до
670нм, Р <10мВт.).

\emph{Сканирующая электронная микроскопия (CЭМ)}

СЭМ-изображения были получены на аналитическом сканирующем электронном
микроскопе TESCAN MIRA с энергодисперсионным микрозондом (TESCAN, Чехия)
в широком диапазоне увеличений от 2 до 1000000 крат. Измерения
проводились при ускоряющем напряжении 20 кВ, токе пучка~1400 pA и
рабочем расстоянии 10 мм в условиях высокого вакуума (10⁻⁴ Па).

Исходными сырьевыми материалами для получения образцов служили
тугоплавкая глина Кемертузского месторождения, что находится в 120 км от
г. Павлодара и зола-уноса с Павлодарской ТЭЦ-1 и Аксуской ГРЭС-1.
Исходное глинистое сырье по огнеупорности и по содержанию оксида
\({Al}_{2}O_{3}\) (таблица 1), согласно ГОСТ 9169--2021 {[}12{]},
относится к группе тугоплавких и полукислых видов глин. Зола-уноса,
согласно ГОСТ 25818--2017 {[}10{]}, относится к типу кислых зол, что
подтверждается содержанием основного оксида кальция, содержание которого
в золе составляет до 10 \% масс. (таблица 1).

Начальным этапом работы была стадия подготовки сырьевых материалов, где
исходное глинистое сырье и зола-уноса проходили стадии измельчения на
планетарной мельнице YXQM-1L, где глина измельчалась до прохода через
сито размером 40 мкм, а исходная зола-уноса -- 20 мкм. После стадии
измельчения и прохождения через ситовый анализатор ZDS-200 исходные
сырьевые материалы смешивались в различных пропорциях в зависимости от
общей массы сырья. Аналогом служила проба из 100\% массы глинистого
сырья. Далее глина отбиралась в количестве от 90 до 70\%, которая
служила связующим агентом в растворе. Содержание исходной золы-уноса же,
которая является добавкой в растворе, составляла от 10 до 30\% от общей
массы загружаемого сырья.

Подготовленная масса увлажнялась водой, объем которой составлял 15-20
мл. Образцы прессовали по полусухому методу прессования. Композиционные
материалы получали в виде таблеток и цилиндров, которые прессовались на
гидравлическом прессе AutoMeister 20т T0901F, сила прессования которой
составляла 1,5 т. После процесса прессования полученные образцы
проходили стадии естественной сушки, если была необходимость, то образцы
проходили стадии искусственной сушки в сушильном шкафу до 100
\tsp{0}\(С\).

Следующим этапом данной научной работы был обжиг композиционных
материалов. Полученные образцы в виде таблеток и цилиндров разделили на
3 части и, поместив в капсели, проводили процесс обжига при 1000--1100
\tsp{0}С с шагом в 50 \tsp{0}С. Процесс обжига
проводили 10 часов с выдержкой 2 часа при максимальной температуре
обжига материалов.

После процесса обжига обожженные образцы проходили через следующую
стадию определения физико-механических свойств обожженных композиционных
материалов, где в лабораторных условиях определялись основные
физико-механические свойства керамических материалов:

- воздушная и огневая усадки;

- водопоглощение;

- огнеупорность;

- прочность на сжатие.

{\bfseries Результаты и обсуждение.} Результаты полученных
физико-механических свойств обожженных образцов керамических материалов
представлены в таблице 1. Как видно из таблицы 1, глина с содержанием
100\% служит эталоном для образцов с добавкой золы-уноса.

{\bfseries Таблица 1 -- Результаты физико-механических свойств керамических
материалов}

%% \begin{longtable}[]{@{}
%%   >{\raggedright\arraybackslash}p{(\linewidth - 10\tabcolsep) * \real{0.1193}}
%%   >{\centering\arraybackslash}p{(\linewidth - 10\tabcolsep) * \real{0.1224}}
%%   >{\centering\arraybackslash}p{(\linewidth - 10\tabcolsep) * \real{0.1674}}
%%   >{\centering\arraybackslash}p{(\linewidth - 10\tabcolsep) * \real{0.1970}}
%%   >{\centering\arraybackslash}p{(\linewidth - 10\tabcolsep) * \real{0.2413}}
%%   >{\centering\arraybackslash}p{(\linewidth - 10\tabcolsep) * \real{0.1525}}@{}}
%% \toprule\noalign{}
%% \begin{minipage}[b]{\linewidth}\centering
%% {\bfseries Шифр состава}
%% \end{minipage} &
%% \multicolumn{2}{>{\centering\arraybackslash}p{(\linewidth - 10\tabcolsep) * \real{0.2898} + 2\tabcolsep}}{%
%% \begin{minipage}[b]{\linewidth}\centering
%% {\bfseries Планируемый фазовый состав, \%}
%% \end{minipage}} & \begin{minipage}[b]{\linewidth}\centering
%% {\bfseries Температура обжига, \tsp{о}С}
%% \end{minipage} &
%% \multicolumn{2}{>{\centering\arraybackslash}p{(\linewidth - 10\tabcolsep) * \real{0.3938} + 2\tabcolsep}@{}}{%
%% \begin{minipage}[b]{\linewidth}\centering
%% {\bfseries Свойства}
%% \end{minipage}} \\
%% \midrule\noalign{}
%% \endhead
%% \bottomrule\noalign{}
%% \endlastfoot
%% ~ & глина & добавки & & водопоглощение, \% & прочность на сжатие, МПа \\
%% 1 & 100 & 0 & 1000 & 12,8 & 74,4 \\
%% ~ & ~ & ~ & 1050 & 12,0 & 65,7 \\
%% ~ & ~ & ~ & 1100 & 10,6 & 103,1 \\
%% 3 & {\bfseries 90} & {\bfseries 10 (зола)} & 1000 & 13,0 & 61,7 \\
%% & ~ & ~ & 1050 & 12,0 & 90,2 \\
%% & ~ & ~ & 1100 & 11,6 & 125,6 \\
%% 4 & {\bfseries 80} & {\bfseries 20 (зола)} & 1000 & 13,8 & 51,8 \\
%% ~ & & & 1050 & 12,1 & 68,3 \\
%% & & & 1100 & 11,6 & 81,6 \\
%% 5 & {\bfseries 70} & {\bfseries 30 (зола)} & 1000 & 14,4 & 59,2 \\
%% & & & 1050 & 12,7 & 63,7 \\
%% ~ & & & 1100 & 13,2 & 75,5 \\
%% \end{longtable}

На рисунках 7-8 показаны графические результаты показателей
водопоглощения и прочностные свойства обожженных образцов керамических
материалов на основе золы-уноса. Как видно из рисунка 7 показатели
водопоглощения при увеличении температуры обжига образцов уменьшаются.

Однако, стоит отметить, что при увеличении концентраций золы-уноса в
составах от 10 до 30\%, показатели водопоглощения заметно увеличиваются.
Выполнив анализ графиков изменения прочностных свойств образцов
керамических материалов при концентрациях от 10 до 30\% из рисунка 8
видно, что показатели прочности при сжатии образцов значительно
уменьшаются, что в конечном итоге влияет на качество получаемого
керамического материала. Поэтому можно сделать вывод, что керамическая
масса состава «глина-зола» концентрации 90:10\% является оптимальным
составом для приготовления образцов керамических материалов.

{\bfseries Рис.7 - Показатели водопоглощения обожженных образцов на основе
золы-уноса}

{\bfseries Рис.8 - Показатели прочности при сжатии образцов на основе
золы-уноса}

Результаты анализа химического состава образцов используемого
минерального сырья представлены в таблице 2. Соответствующие образцам
спектры инфракрасного нарушенного полного внутреннего отражения
представлены на рисунке 9. Как видно из таблицы 2, образцы изученной
золы ГРЭС, золы ТЭЦ 1 и глины имеют преимущественно алюмосиликатный
состав.

Условные обозначения: ППП -- потери при прокаливании.

{\bfseries Таблица 2 - Химический состав образцов используемого
минерального}

{\bfseries сырья (в мас.\%)}

%% \begin{longtable}[]{@{}
%%   >{\raggedright\arraybackslash}p{(\linewidth - 24\tabcolsep) * \real{0.0753}}
%%   >{\raggedright\arraybackslash}p{(\linewidth - 24\tabcolsep) * \real{0.0760}}
%%   >{\raggedright\arraybackslash}p{(\linewidth - 24\tabcolsep) * \real{0.0792}}
%%   >{\raggedright\arraybackslash}p{(\linewidth - 24\tabcolsep) * \real{0.0898}}
%%   >{\raggedright\arraybackslash}p{(\linewidth - 24\tabcolsep) * \real{0.0756}}
%%   >{\raggedright\arraybackslash}p{(\linewidth - 24\tabcolsep) * \real{0.0757}}
%%   >{\raggedright\arraybackslash}p{(\linewidth - 24\tabcolsep) * \real{0.0742}}
%%   >{\raggedright\arraybackslash}p{(\linewidth - 24\tabcolsep) * \real{0.0761}}
%%   >{\raggedright\arraybackslash}p{(\linewidth - 24\tabcolsep) * \real{0.0759}}
%%   >{\raggedright\arraybackslash}p{(\linewidth - 24\tabcolsep) * \real{0.0760}}
%%   >{\raggedright\arraybackslash}p{(\linewidth - 24\tabcolsep) * \real{0.0759}}
%%   >{\raggedright\arraybackslash}p{(\linewidth - 24\tabcolsep) * \real{0.0605}}
%%   >{\raggedright\arraybackslash}p{(\linewidth - 24\tabcolsep) * \real{0.0900}}@{}}
%% \toprule\noalign{}
%% \begin{minipage}[b]{\linewidth}\raggedright
%% {\bfseries Наименование образца}
%% \end{minipage} & \begin{minipage}[b]{\linewidth}\raggedright
%% {\bfseries SiO\tsb{2}}
%% \end{minipage} & \begin{minipage}[b]{\linewidth}\raggedright
%% {\bfseries TiO\tsb{2}}
%% \end{minipage} & \begin{minipage}[b]{\linewidth}\raggedright
%% {\bfseries Al\tsb{2}O\tsb{3}}
%% \end{minipage} & \begin{minipage}[b]{\linewidth}\raggedright
%% {\bfseries Fe\tsb{2}O\tsb{3}}
%% \end{minipage} & \begin{minipage}[b]{\linewidth}\raggedright
%% {\bfseries MnO}
%% \end{minipage} & \begin{minipage}[b]{\linewidth}\raggedright
%% {\bfseries MgO}
%% \end{minipage} & \begin{minipage}[b]{\linewidth}\raggedright
%% {\bfseries CaO}
%% \end{minipage} & \begin{minipage}[b]{\linewidth}\raggedright
%% {\bfseries Na\tsb{2}O}
%% \end{minipage} & \begin{minipage}[b]{\linewidth}\raggedright
%% {\bfseries K\tsb{2}O}
%% \end{minipage} & \begin{minipage}[b]{\linewidth}\raggedright
%% {\bfseries P\tsb{2}O\tsb{5}}
%% \end{minipage} & \begin{minipage}[b]{\linewidth}\raggedright
%% {\bfseries NiO}
%% \end{minipage} & \begin{minipage}[b]{\linewidth}\raggedright
%% {\bfseries ППП\tsp{1}}
%% \end{minipage} \\
%% \midrule\noalign{}
%% \endhead
%% \bottomrule\noalign{}
%% \endlastfoot
%% Зола ГРЭС & 58,62 & 1,19 & 27,92 & 3,86 & 0,07 & 0,37 & 1,04 & 0,40 &
%% 0,55 & 0,37 & 0,01 & 4,00 \\
%% Зола ТЭЦ 1 & 51,93 & 1,14 & 25,98 & 11,84 & 0,23 & 0,77 & 2,15 & 0,55 &
%% 0,44 & 0,46 & <0,01 & 2,72 \\
%% Глина & 58,71 & 1,46 & 21,91 & 3,72 & 0,02 & 0,46 & 1,28 & 0,26 & 0,92 &
%% 0,05 & 0,01 & 8,80 \\
%% \end{longtable}

На полученных ИК спектрах исследуемых образцов (рисунок 9) отмечается
широкая интенсивная полоса поглощения 950-1100 см\tsp{-1},
которая обусловлена валентными колебаниями Al--O--Si и Si--O--Si
{[}13-15{]}, тогда как пики в интервале 670-690 см\tsp{−1}
соответствуют симметричному валентному колебанию Si--O {[}16{]}.

ИК-спектры проб регистрировали на ИК-Фурье спектрометре ФТ-805
производства Симекс. Спектрометр был оборудован приставкой нарушенного
полного внутреннего отражения (НПВО) с кристаллом селенида цинка.
Измерения проводили в режиме накопления с числом сканов не менее 50.
Спектральный диапазон составлял от 500 до 4000~см\tsp{--1} с
шагом 4 см\tsp{--1}. Перед каждым измерением образцов
проводили оценку фонового спектра приставки. Затем на поверхность
кристалла помещали порошковую пробу под высоким давлением. После каждого
снятия спектра образца кристалл очищали смоченной в ацетоне ватой.
Интерпретация наблюдаемых полос ИК-Фурье спектроскопии проводилась на
основе представленных данных {[}17, 18, 19{]}.

В случае смесей «тугоплавкая глина+зола-уноса» составов 90:10\%,
80:20\%, 70:30\% (рисунок 9-11) наблюдаются следующие максимумы
поглощения в диапазонах от 650 до 1200 см\tsp{-1} и в
диапазоне от 3600 до 4000 см\tsp{-1}. В области 3600-4000
см\tsp{-1}, наблюдаются невыраженные полосы на уровне шума,
что говорит о практически полной дегидратации кристаллической структуры
исходной глины в ходе температурной обработки. В диапазонах от 650 до
1200 см\tsp{-1} наблюдаются полосы поглощения в зависимости
от состава смесей, которые можно отнести согласно литературным данным к
характерным для колебания связи Si-O в тетраэдрах SiO\tsb{4} и
связи Al-O (как в тетраэдрах AlO\tsb{4} , так и в октаэдрах
AlO\tsb{6}). В силу схожего физико-химического состава золы к
составу глины, общий характер изменений полос поглощения в целом
соответствует чистой обожжённой глине, также наблюдается сдвиг полос
1057-1087 см\tsp{-1} в область 1058-1080 или 1058-1083 в
зависимости от добавки золы. В целом, в диапазоне от 650-1200
см\tsp{-1}, по мере повышения температуры обжига,
наблюдается снижение интенсивности ряда полос наблюдается снижение
интенсивности ряда полос 693, 893, 954 см\tsp{-1}, а также
сдвиг полос поглощения 1058-1087, что свидетельствует об изменении
кристаллической структуры и частичной аморфизации оксида кремния.

\fig{c4/image44}{}

{\bfseries Рис.9 - Ик-спектры смеси «глина-зола» состава 90/10\%}

\fig{c4/image45}{}

{\bfseries Рис.10 - Ик-спектры смеси «глина-зола» состава 80/20\%}

\fig{c4/image46}{}

{\bfseries Рис.11 - Ик-спектры смеси «глина-зола» состава 70/30\%}

{\bfseries Выводы.} Выполнив физико-химический анализ сырьевых материалов
образцов керамических материалов глинистого сырья и золы-уноса с ТЭЦ и
ГРЭС выявлено, что высокое содержание SiO\tsb{2} в двух видах
золы указывает на возможность ее использования в производстве силикатных
строительных материалов, а также строительной керамики. Помимо высокого
содержания оксидов Al\tsb{2}O\tsb{3} и
SiO\tsb{2}, образцы золы отличаются низким содержанием CaO
(менее 2,2 \%). Золу ТЭЦ-1 отличает сравнительно большое содержание
Fe\tsb{2}O\tsb{3} (11,84 \%) в сравнении с золой
ГРЭС, что обусловлено различиями процессов горения, источников и
составов сжигаемого угля. Результаты физико-механических свойств
образцов керамических материалов, полученных на основе золы-уноса с
различными комбинациями показали возможность определиться с оптимальными
составами керамических масс. На основе анализов полученных
физико-механических свойств составы смеси «глина-зола-уноса» в
соотношении 90:10\% являются оптимальным составом керамической массы для
получения образцов керамических изделий.

\emph{{\bfseries Финансирование.} Данное исследование финансируется
Комитетом науки Министерства науки и высшего образования Республики
Казахстан (грант АР23486826 «Разработка композиционных керамических
материалов на основе техногенных отходов»).}

{\bfseries Литература}

1. Тлеуов А. А., Нишанова З. Н., Колесников А. С., Макулбекова Г. О.
Золошлаки тепло-электро станций как опасные отходы техносферы //
Качественное экологическое образование и инновационная деятельность -
основа прогресса и устойчивого развития: Сборник статей VII
международной научно-практической конференции, Саратов, 28-30 марта 2024
года. -- Саратов: Саратовский государственный университет генетики,
биотехнологии и инженерии им. Н.И. Вавилова, 2024. - С.238-242

2. Гусев Е.В., Сергиенкова А.А.,Таламаев Д. С. Использование золоотвалов
ТЭС в производстве стеновой глинозольной керамики // Развитие методов
прикладной математики для решения междисциплинарных проблем энергетики :
Материалы I Всероссийской научно-технической конференции с международным
участием, Ульяновск, 06-07 октября 2021 года. - Ульяновск: Ульяновский
государственный технический университет, 2021. - С.124-129.

%% \begin{longtable}[]{@{}
%%   >{\raggedright\arraybackslash}p{(\linewidth - 6\tabcolsep) * \real{0.2958}}
%%   >{\raggedright\arraybackslash}p{(\linewidth - 6\tabcolsep) * \real{0.1130}}
%%   >{\raggedright\arraybackslash}p{(\linewidth - 6\tabcolsep) * \real{0.2958}}
%%   >{\raggedright\arraybackslash}p{(\linewidth - 6\tabcolsep) * \real{0.2954}}@{}}
%% \toprule\noalign{}
%% \begin{minipage}[b]{\linewidth}\raggedright
%% 3. Казахстан откатился назад в вопросах переработки промышленных
%% отходов.2024
%% URLhttps://energyprom.kz/articles-ru/industries-ru/kazahstan-otkatilsya-nazad-v-voprosah-pererabotki-promyshlennyh-othodov/.
%% - Дата обращения: 15.07.2025
%% 
%% 4. Нурпеисова М. Б., Естемесов З. А., Бек А. А., Жунусова Г. Е.
%% Использование вторичного сырья в производстве строительных материалов //
%% Труды университета. - 2024. - № 2(95). - С.114-122.
%% DOI~\href{https://doi.org/10.52209/1609-1825_2024_2_114}{10.52209/1609-1825\_2024\_2\_114}.
%% \end{minipage} & \begin{minipage}[b]{\linewidth}\raggedright
%% \end{minipage} & \begin{minipage}[b]{\linewidth}\raggedright
%% \end{minipage} & \begin{minipage}[b]{\linewidth}\raggedright
%% \end{minipage} \\
%% \midrule\noalign{}
%% \endhead
%% \bottomrule\noalign{}
%% \endlastfoot
%% \end{longtable}

5. Голдаева А.В., Дидоренко А.А., Шахов С.А. Экологические аспекты
применения золы от сжигания осадка сточных вод в производстве стеновой
керамики // Химия и жизнь: Сборник XVIII Международной
научно-практической студенческой конференции, Новосибирск, 16 мая 2019
года. - Новосибирск: Издательский Центр "Золотой колос", 2019. - С.
234-237.

6. Лохова Н.А., Н. Боева В.С., Либеровская В. Микропоризованные
керамические стеновые изделия на основе пыли газоочистки производства
ферросплавов / Н.А. Лохова, // Системы. Методы.Технологии. -2012. - №
3(15). - С.114 -118.

7. Нурпеисова М. Б., Естемесов З. А., Федотенко Н. А., Габбасов С. Г.
Технология утилизации и использования золошлаковых отходов для
обеспечения экологической безопасности региона // Устойчивое развитие
горных территорий. - 2023.-Т.15(3) 3(57).-С.631-639.

\href{https://doi.org/10.21177/1998-4502-2023-15-3-631-639}{DOI
10.21177/1998-4502-2023-15-3-631-639}.

8. Беккалиев, Н. М., Шакешев, Б. Т., Чумаченко, Н. Г. Использование
вторичных ресурсов для производства строительных материалов // Вестник
Западно-Казахстанского инновационно-технологического университета. -
2021. - Т.19(3). - С.10-13.

9. Такибай Ш. Т., Саканов К.Т., Данзандорж С. Влияния разновидной
зерновой состав золы ТЭС для формирования структуры и прочности
газобетона // Наука и техника Казахстана. - 2019. - № 4. - С.42-49.

10. ГОСТ 25818--2017 Золы-уноса тепловых электростанций для бетонов.
Технические условия. Взамен ГОСТ 25818-91. Дата введения в действие:
01.03.2018. - M.:Межгосударствен-

ный стандарт. Стандартинформ, 2017.
\url{https://www.status-grunt.ru/upload/bases/26.pdf.-} Дата обращения:
15.07.2025

11. Толегенов Д.Т., Елубай М.А., Кулумбаев, Н. К. {[}и др.{]}
Определение технологических свойств техногенных отходов предприятий
энергетики и металлургии Павлодарского региона // Наука и техника
Казахстана. - 2022. - № 1. - С.208-219.
DOI~\href{https://doi.org/10.48081/IHOZ7105}{10.48081/IHOZ7105}.

12. ГОСТ 9169-2021. Сырье глинистое для керамической промышленности.
Классификация. -- Взамен ГОСТ 9169-75: Дата введения в действие:
01.04.2022. - М.: Межгосударственный стандарт. Российский институт
стандартизации. - 2021.

\url{https://internet-law.ru/gosts/gost/75778/}.- Дата обращения:
15.07.2025.13. Zhang W., Dong C., Huang P., Sun Q., Li M., Chai J. Experimental
study on the characteristics of activated coal gangue and coal
gangue-based geopolymer// Energies. - 2020.- Vol.13(10):2504.

DOI
~\href{https://www.mdpi.com/1996-1073/13/10/2504}{10.3390/en13102504}.

14 Kasprzhitskii A., Lazorenko G., Khater A., Yavna V. Mid-Infrared
Spectroscopic Assessment of Plasticity Characteristics of Clay Soils//
Minerals. - 2018.-Vol.8(5):184.
\href{https://doi.org/10.3390/min8050184}{DOI 10.3390/min8050184}.

15. Lazorenko G., Kasprzhitskii A., Yavna V. Comparative Study of the
Hydrophobicity of Organo-Montmorillonite Modified with Cationic
Amphoteric and Nonionic Surfactants//Minerals. - 2020.- Vol.10(9): 732.
DOI \href{https://doi.org/10.3390/min10090732}{10.3390/min10090732}.

16. Sun Z., Wang X., Jia S., Liang J., Ning X., Li C. Fabrication of
pollution-free coal gangue-based catalytic material utilizing ferrous
chloride as activator for efficient peroxymonosulfate activation //Int.
J. Coal Sci. Technol. -2024.- 11(1).
\href{https://doi.org/10.1007/s40789-023-00659-5}{DOI
10.1007/s40789-023-00659-5}.

17. Chukanov, N. V., Chervonnyi, A. D. Infrared spectroscopy of minerals
and related compounds// Springer. - 2016. -
DOI:\href{https://doi.org/10.1007/978-3-319-25349-7}{10.1007/978-3-319-25349-7}.

18 Madejová, J., Gates, W. P., Petit, S. Chapter 5-~IR spectra
of clay minerals // Developments in clay science. - 2017.-Vol.8.- P.
107-149. DOI
10.1016/B978-0-08-100355-8.00005-9.

19. База данных ПО «Zair 3.5.», Симекс, Новосибирск. --
\url{https://old.simex-ftir.ru/product_1.html}

{\bfseries References}

1. Tleuov A. A., Nishanova Z. N., Kolesnikov A. S., Makulbekova G. O.
Zoloshlaki teplo-jelektro stancij kak opasnye othody tehnosfery //
Kachestvennoe jekologicheskoe obrazovanie i innovacionnaja
dejatel' nost'{} - osnova progressa i
ustojchivogo razvitija: Sbornik statej VII mezhdunarodnoj
nauchno-prakticheskoj konferencii, Saratov, 28-30 marta 2024 goda. --
Saratov: Saratovskij gosudarstvennyj universitet genetiki, biotehnologii
i inzhenerii im. N.I. Vavilova, 2024. - S.238-242. {[}in Russian{]}

2. Gusev E.V., Sergienkova A.A.,Talamaev D. S.
Ispol' zovanie zolootvalov TJeS v proizvodstve stenovoj
glinozol' noj keramiki // Razvitie metodov prikladnoj
matematiki dlja reshenija mezhdisciplinarnyh problem jenergetiki :
Materialy I Vserossijskoj nauchno-tehnicheskoj konferencii s
mezhdunarodnym uchastiem, Ul' janovsk, 06-07 oktjabrja
2021 goda. - Ul' janovsk: Ul' janovskij
gosudarstvennyj tehnicheskij universitet, 2021. - S.124-129. {[}in
Russian{]}

3. Kazahstan otkatilsja nazad v voprosah pererabotki promyshlennyh
othodov.2024
URLhttps://energyprom.kz/articles-ru/industries-ru/kazahstan-otkatilsya-nazad-v-voprosah-pererabotki-promyshlennyh-othodov/.
- Data obrashhenija: 15.07.2025. {[}in Russian{]}

4. Nurpeisova M. B., Estemesov Z. A., Bek A. A., Zhunusova G. E.
Ispol' zovanie vtorichnogo syr' ja v
proizvodstve stroitel' nyh materialov // Trudy
universiteta. - 2024. - № 2(95). - S.114-122. DOI
10.52209/1609-1825\_2024\_2\_114. {[}in Russian{]}

5. Goldaeva A.V., Didorenko A.A., Shahov S.A. Jekologicheskie aspekty
primenenija zoly ot szhiganija osadka stochnyh vod v proizvodstve
stenovoj keramiki // Himija i zhizn': Sbornik XVIII
Mezhdunarodnoj nauchno-prakticheskoj studencheskoj konferencii,
Novosibirsk, 16 maja 2019 goda. - Novosibirsk:
Izdatel' skij Centr "Zolotoj kolos", 2019. - S.234-237.
{[}in Russian{]}

6. Lohova N.A., N. Boeva V.S., Liberovskaja V. Mikroporizovannye
keramicheskie stenovye izdelija na osnove pyli gazoochistki proizvodstva
ferrosplavov / N.A. Lohova, // Sistemy. Metody.Tehnologii. -2012. - №
3(15). - S.114 -118. {[}in Russian{]}

7. Nurpeisova M. B., Estemesov Z. A., Fedotenko N. A., Gabbasov S. G.
Tehnologija utilizacii i ispol' zovanija zoloshlakovyh
othodov dlja obespechenija jekologicheskoj bezopasnosti regiona //
Ustojchivoe razvitie gornyh territorij. -2023.-T.15(3) 3(57) .-S.
631-639.

DOI 10.21177/1998-4502-2023-15-3-631-639. {[}in Russian{]}

8. Bekkaliev, N. M., Shakeshev, B. T., Chumachenko, N. G.
Ispol' zovanie vtorichnyh resursov dlja proizvodstva
stroitel' nyh materialov // Vestnik
Zapadno-Kazahstanskogo innovacionno-tehnologicheskogo universiteta. -
2021. - T.19(3). -S.10-13. {[}in Russian{]}

9. Takibaj Sh. T., Sakanov K.T., Danzandorzh S. Vlijanija raznovidnoj
zernovoj sostav zoly TJeS dlja formirovanija struktury i prochnosti
gazobetona // Nauka i tehnika Kazahstana. - 2019. - № 4. - S.42-49.
{[}in Russian{]}

10. GOST 25818--2017 Zoly-unosa teplovyh jelektrostancij dlja betonov.
Tehnicheskie uslovija. Vzamen GOST 25818-91. Data vvedenija v dejstvie:
01.03.2018. - M.: Mezhgosudarstven-

nyj standart. Standartinform, 2017.
https://www.status-grunt.ru/upload/bases/26.pdf.- Data obrashhenija:
15.07.2025. {[}in Russian{]}

11. Tolegenov D.T., Elubaj M.A., Kulumbaev, N. K. {[}i dr.{]}
Opredelenie tehnologicheskih svojstv tehnogennyh othodov predprijatij
jenergetiki i metallurgii Pavlodarskogo regiona // Nauka i tehnika
Kazahstana. - 2022. - № 1. - S.208-219. DOI 10.48081/IHOZ7105. {[}in
Russian{]}

12. GOST 9169-2021. Syr' e glinistoe dlja keramicheskoj
promyshlennosti. Klassifikacija. -- Vzamen GOST 9169-75: Data vvedenija
v dejstvie: 01.04.2022. - M.: Mezhgosudarstvennyj standart. Rossijskij
institut standartizacii. - 2021. {[}in Russian{]}

13. Zhang W., Dong C., Huang P., Sun Q., Li M., Chai J. Experimental
study on the characteristics of activated coal gangue and coal
gangue-based geopolymer// Energies. - 2020.- Vol.13(10):2504.

DOI
~\href{https://www.mdpi.com/1996-1073/13/10/2504}{10.3390/en13102504}.

14 Kasprzhitskii A., Lazorenko G., Khater A., Yavna V. Mid-Infrared
Spectroscopic Assessment of Plasticity Characteristics of Clay Soils//
Minerals. - 2018.-Vol.8(5):184.
\href{https://doi.org/10.3390/min8050184}{DOI 10.3390/min8050184}.

15. Lazorenko G., Kasprzhitskii A., Yavna V. Comparative Study of the
Hydrophobicity of Organo-Montmorillonite Modified with Cationic
Amphoteric and Nonionic Surfactants//Minerals. - 2020.- Vol.10(9): 732.
DOI \href{https://doi.org/10.3390/min10090732}{10.3390/min10090732}.

16. Sun Z., Wang X., Jia S., Liang J., Ning X., Li C. Fabrication of
pollution-free coal gangue-based catalytic material utilizing ferrous
chloride as activator for efficient peroxymonosulfate activation //Int.
J. Coal Sci. Technol. -2024.- 11(1).
\href{https://doi.org/10.1007/s40789-023-00659-5}{DOI
10.1007/s40789-023-00659-5}.

17. Chukanov, N. V., Chervonnyi, A. D. Infrared spectroscopy of minerals
and related compounds// Springer. - 2016. -
DOI:\href{https://doi.org/10.1007/978-3-319-25349-7}{10.1007/978-3-319-25349-7}.

18 Madejová, J., Gates, W. P., Petit, S. Chapter 5-~IR spectra
of clay minerals // Developments in clay science. - 2017.-Vol.8.- P.
107-149. DOI
10.1016/B978-0-08-100355-8.00005-9.

19. Baza dannyh PO «Zair 3.5.», Simeks, Novosibirsk. --
\url{https://old.simex-ftir.ru/product_1.html}. {[}in Russian{]}

\emph{{\bfseries Сведения об авторах}}

Маусумбаев С.- докторант, НАО «Торайгыров университет», Павлодар,
Казахстан,

e-mail:sabit\_mausumbaev@mail.ru;

Елубай М.- кандидат химических наук, профессор, НАО «Торайгыров
университет», Павлодар, Казахстан, e-mail:
yelubay.m@tou.edu.kz;

Толегенов Д.- научный сотрудник, НАО «Торайгыров университет», Павлодар,
Казахстан, e-mail:
\href{http://www.dika-92@mail.ru}{www.dika-92@mail.ru};

Масакбаева С.- кандидат химических наук, НАО «Торайгыров университет»,
Павлодар, Казахстан, e-mail:
sofochka184@mail.ru;

Айткалиева Г.- доктор PhD, ассоциированный профессор НАО «Сатпаев
университет», Алматы, Казахстан, e-mail:
gulzat\_slyashevna@mail.ru.

\begin{quote}
\emph{{\bfseries Information about the authors}}
\end{quote}

Maussumbayev S.- Doctoral student, "Toraigyrov University" NCJSC,
Pavlodar, Kazakhstan, е-
mail:sabit\_mausumbaev@mail.ru;

Yelubay M. - Candidate of Chemical Science, Professor «Toraighyrov
University» NCJSC, Pavlodar, of Kazakhstan, е-mail:
yelubay.m@tou.edu.kz;

Tolegenov D. - researcher, «Toraighyrov University» NCJSC, Pavlodar, of
Kazakhstan, е-mail:
\href{http://www.dika-92@mail.ru}{www.dika-92@mail.ru}

Masakbayeva S.- Candidate of Chemical Sciences, НАО «Торайгыров
университет», Павлодар, Казахстан, e-mail:
sofochka184@mail.ru;

Aitkaliyeva G.- doctor PhD, associate professor NAO"Satpayev
University", Almaty, Kazakhstan, e-mail:
gulzat\_slyashevna@mail.ru.
