\id{МРНТИ 61.33.39}{}

\begin{header}
\swa{}{ВЛИЯНИЕ АМОРФНОГО КРЕМНИЯ НА ФИЗИЧЕСКОЕ И ХИМИЧЕСКОЕ СОСТОЯНИЕ ПОЧВЫ}

\tsp{1}М.Ш.Сулейменова,
\tsp{1}С.Т. Дәуметова\envelope,
\tsp{2}E. Varol,
\tsp{1}Г.О. Бугубаева,
\tsp{1}М.С. Жетенова,
\end{header}

\begin{affil}
\tsp{1}Алматинский технологический университет, Алматы, Казахстан,

\tsp{2}Эскишехирский технический университет, Эскишехир, Турция

\corrauthor{Корреспондент-автор: daumetova83@mail.ru}
\end{affil}

Физико-химическая активность соединений SO\tsb{2} в почве -
это их способность взаимодействовать с другими веществами, менять
физические и химические свойства почвы, например, образовывать
устойчивые соединения, которые влияют на структуру и плодородие почвы.
Биологическая активность связана с тем, насколько эти соединения
участвуют в жизни микроорганизмов, стимулируя их развитие, способствуя
обменным процессам и повышая устойчивость растений к болезням. Что
касается стабильности влажности и плотности в почве - это показатели,
отражающие, как хорошо почва удерживает влагу и насколько она устойчива
к сжатию или разрыхлению. Высокая стабильность влажности обеспечивает
постоянный источник воды для растений, а оптимальная плотность - хорошую
аэрацию и проникновение корней.

SO\tsb{2} биопрепарат - это настоящее инновационное средство в
науке, сочетающий в себе природные свойства SO\tsb{2} и
биологические компоненты, стимулируя рост растений, укрепляя их
иммунитет и повышая их устойчивость к стрессам и болезням. Благодаря
своему уникальному составу, препарат способствует улучшению структуры
почвы, увеличивает её влагосодержание и стабильность плотности, что
создает оптимальные условия для развития корневых систем и повышения
урожайности. Использование данного биопрепарата позволяет снизить
использование химических пестицидов и удобрений, делая сельское
хозяйство более экологичным и устойчивым.

В данной работе описана инновационная технология получения биопрепарата
на основе SO\tsb{2} для растений. Метод основан на термической
обработке рисовой шелухи в лабораторных условиях, что позволяет
эффективно извлекать активные компоненты и увеличивать их усвояемость.
Использование такого подхода способствует повышению устойчивости
растений к стрессам и стимулирует их рост, делая технологию
перспективной для экологически чистого аграрного производства.

В статье представлен обзор основных аспектов биогеохимии аморфного
кремнезема в почве. Кремнийорганические соединения в почве изучены
слабо. Также в статье описан опыт термической обработки рисовой шелухи и
влияние температуры на свойства полученного SO\tsb{2}.
Представленные данные основаны на методике определения влажности и
плотности в почве. Физико-химической и биологической активности
соединений SO\tsb{2} и стабильности влажности и плотности в
почве.

{\bfseries Ключевые слова:} диоксид кремния, биопрепарат, почва, рисовая
шелуха, гуминовая кислота, влажность, плотность.

\begin{header}
EFFECT OF AMORPHOUS SILICON ON THE PHYSICAL AND CHEMICAL STATE OF SOIL

\tsp{1}M. Suleimenova,
\tsp{1}S. Daumetova\envelope,
\tsp{2}E. Varol,
\tsp{1}G. Bugubaeva,
\tsp{1}М. Zhetenova
\end{header}

\begin{affil}
Almaty Technological University, Almaty, Kazakhstan,

Eskisehir Technical University, Eskisehir, Turkey,

e-mail: daumetova83@mail.ru
\end{affil}

The physicochemical activity of SO\tsb{2} in the soil is their
ability to interact with other substances, change the physical and
chemical properties of the soil, for example, form stable compounds that
affect the structure and fertility of the soil. Biological activity is
related to the extent to which these compounds participate in the life
of microorganisms, stimulating their development, promoting metabolic
processes and increasing plant resistance to diseases. As for the
stability of moisture and density in the soil, these are indicators that
reflect how well the soil retains moisture and how resistant it is to
compression or loosening. High moisture stability provides a constant
source of water for plants, and optimal density -- good aeration and
root penetration.

The SO\tsb{2} biopreparation is a truly innovative tool in
science, combining the natural properties of SO\tsb{2} and
biological components, stimulating plant growth, strengthening their
immunity and increasing their resistance to stress and disease. Due to
its unique composition, the preparation helps improve the soil
structure, increases its moisture content and density stability, which
creates optimal conditions for the development of root systems and
increased yields. The use of this biological product allows reducing the
use of chemical pesticides and fertilizers, making agriculture more
environmentally friendly and sustainable.

This paper describes an innovative technology for obtaining a
SO\tsb{2} biopreparation for plants. The method is based on
thermal treatment of rice husks in laboratory conditions, which allows
for the effective extraction of active components and increases their
digestibility. The use of this approach helps to increase plant
resistance to stress and stimulates their growth, making the technology
promising for environmentally friendly agricultural production.

The article presents an overview of the main aspects of the
biogeochemistry of amorphous silica in soil. Organosilicon compounds in
soil have been poorly studied. The article also describes the experience
of thermal treatment of rice husk and the effect of temperature on the
properties of the resulting SO\tsb{2}. The presented data are
based on the method of determining the moisture and density in the soil.
Physicochemical and biological activity of SO\tsb{2} compounds
and the stability of moisture and density in the soil.

{\bfseries Keywords:} silicon dioxide, biopreparation, soil, rice husks,
humic acid, humidity, density.

\begin{header}
АМОРФТЫ КРЕМНИЙДІҢ ТОПЫРАҚТЫҢ ФИЗИКАЛЫҚ ЖӘНЕ ХИМИЯЛЫҚ КҮЙІНЕ ӘСЕРІ

\tsp{1}М.Ш. Сулейменова,
\tsp{1}С.Т. Дәуметова\envelope,
\tsp{2}E. Varol,
\tsp{1}Г. Бугубаева,
\tsp{1}М.С. Жетенова
\end{header}
\vspace{0.1em}
\begin{affil}
\tsp{1}Алматы технологиялық университеті, Алматы, Қазахстан,

\tsp{2}Ескішехир техникалық университеті, Ескішехир, Түркия,

e-mail: daumetova83@mail.ru
\end{affil}

Топырақтағы SO\tsb{2} қосылыстарының физика-химиялық
белсенділігі олардың басқа заттармен әрекеттесуі, топырақтың
физикалық-химиялық қасиеттерін өзгертуі, мысалы, топырақтың құрылымы мен
құнарлылығына әсер ететін тұрақты қосылыстар түзуі. Биологиялық
белсенділік бұл қосылыстардың микроорганизмдердің тіршілігіне қатысу
дәрежесіне байланысты, олардың дамуын ынталандырады, зат алмасу
процестерін жылдамдатады және өсімдіктердің ауруларға төзімділігін
арттырады. Топырақтағы ылғал мен тығыздықтың тұрақтылығына келетін
болсақ, бұл топырақтың ылғалды қаншалықты жақсы сақтайтынын және оның
қысылу немесе қопсытуға қаншалықты төзімді екенін көрсететін
көрсеткіштер. Ылғалдылықтың тұрақтылығы өсімдіктер үшін тұрақты су көзін
қамтамасыз ету, ал оңтайлы тығыздық жақсы аэрация мен тамырдың
беріктігін қамтамасыз етеді.

Құрамында SO\tsb{2} бар биопрепарат - бұл SO\tsb{2}
мен биологиялық компоненттердің табиғи қасиеттерін біріктіретін,
өсімдіктердің өсуін ынталандыратын, олардың иммунитетін күшейтетін және
стресс пен ауруға төзімділігін арттыратын ғылымдағы нағыз инновациялық
құрал. Бірегей құрамының арқасында препарат топырақ құрылымын жақсартуға
көмектеседі, оның ылғалдылығы мен тығыздығының тұрақтылығын арттырады,
бұл тамыр жүйесін дамыту және өнімділікті арттыру үшін оңтайлы жағдай
жасайды. Бұл биопрепаратты пайдалану химиялық пестицидтер мен
тыңайтқыштарды пайдалануды азайтуға, ауыл шаруашылығын экологиялық таза
және тұрақты етуге мүмкіндік береді.

Бұл мақалада өсімдіктерге арналған SO\tsb{2} негізіндегі
биопрепарат алудың инновациялық технологиясы сипатталған. Әдіс белсенді
компоненттерді тиімді бөліп алуға және олардың сіңімділігін арттыруға
мүмкіндік беретін зертханалық жағдайларда күріш қабығын термиялық
өңдеуге негізделген. Бұл тәсілді қолдану өсімдіктердің күйзеліске
төзімділігін арттыруға көмектеседі және олардың өсуін ынталандырады, бұл
технологияны экологиялық таза ауылшаруашылық өндірісі үшін перспективалы
етеді.

Мақалада топырақтағы аморфты кремнеземнің биогеохимиясының негізгі
аспектілеріне шолу берілген. Топырақтағы кремнийорганикалық қосылыстар
нашар зерттелген. Мақалада күріш қабығын термиялық өңдеу тәжірибесі және
алынған SO\tsb{2} қасиеттеріне температураның әсері де
сипатталған. Ұсынылған мәліметтер топырақтағы ылғалдылық пен тығыздықты,
SO\tsb{2} қосылыстарының физикалық, химиялық және биологиялық
белсенділігін және топырақтағы ылғал мен тығыздықтың тұрақтылығын
анықтау әдісіне негізделген.

{\bfseries Түйін сөздер:} кремний диоксиді, биопрепарат, топырақ, күріш
қауызы, гумин қышқылы, ылғалдылық, тығыздық.

\begin{multicols}{2}
{\bfseries Введение.} Кремнийорганические соединения в почве - это
неусваимаемые частицы, формирующие её структуру и влиящие на рост
растений. Они образуются в результате взаимодействия кремния с
органическими веществами и играют важную роль в укреплении тканей
растений и повышении их устойчивости к неблагоприятным условиям
окружающей среды.

Установлено, что общее количество кремнезема в литосфере составляет
58,3\% (для сравнения: лунный грунт - 41\%, каменные метеориты - в
среднем 21\%). Количество кремния в почве зависит от ее
гранулометрического состава: в глинистой почве оно составляет примерно
20-35\%, в песчаной - 45-49\% {[}1,2{]}.

Фактически для поддержания стабильности растительных ресурсов в
сельскохозяйственном производстве очень важна разработка технологии
получения биологического препарата из химического элемента кремния. Это
связано с тем, что внесение кремния (Si) в растительные ресурсы
способствует повышению устойчивости ко многим стрессовым, инфекционным,
вирусным заболеваниям и активно защищает растения от грибов и травоядных
насекомых {[}3,4{]}. Поскольку растения, обработанные биопрепаратом
кремния, не оставляют остатков пестицидов в продуктах питания и
окружающей среде, они не наносят вреда плодородию почвы, как удобрения.
Кремнезем поглощается из почвы корнями растений и транспортируется к
надземным частям растения транспирационным потоком. Кремний (Si),
перенесенный в ткани растений, снижает абиотический стресс и придает
растению твердость и силу. Кремний содержится в стенках растительных
клеток, трихомах, внутриклеточных пространствах и репродуктивных
органах. В целом органы растений содержат 0,1-10\% кремния по массе
{[}5,6{]}.

Известно, что кремний - второй по распространенности элемент на нашей
планете после кислорода (рисунок - 1). Определено, что общее количество
кремнезема в литосфере составляет 58,3\% (для сравнения: лунный грунт -
41\%, каменные метеориты - в среднем 21\%) {[}7,8{]}.

Кремний впервые показано в работах Дэви в 1814 году как важный элемент
питания растений. Он предположил, что кремний накапливается в
эпидермальных тканях растений, создавая защитный барьер от
насекомых-вредителей и болезней {[}9,10{]}. В 1840 г. на основании
сведений об элементном составе растений Ю. Либих пришел к выводу о
необходимости применения кремниевых удобрений. Он провел первый опыт с
силикатом натрия на сахарной свекле. Помимо увеличения массы
корнеплодов, Ю.Либих зафиксировал увеличение содержания сахара при
использовании кремниевых удобрений {[}11,12{]}. В 1856 году Лоуз начал
эксперимент «Травяной парк» с использованием силиката натрия на станции
Ротамстед в Англии. Эта практика продолжается до сих пор, и кремниевые
удобрения позволяют увеличить максимальную урожайность в таблице 1
{[}13,14{]}.
\end{multicols}

\fig[0.95\textwidth]{c3/image37}[Рис.1 - Распределение элементов в земной коре]

\tcap{Таблица 1 - Содержание кремния в некоторых растениях.}
\begin{longtblr}[
  label = none,
  entry = none,
]{
  cells = {c},
  cells = {font = \small},
  hlines,
  vlines,
}
\textbf{Растение}        & \textbf{Si по сухому весу \%} & \textbf{Растение}    & \textbf{Si по сухому весу \%} \\
Equisetum                & 0,7–8,99                      & Avena sativa         & 0,65–3,74                     \\
Picea excelsa            & 0,31–1,75                     & Nicotina tabacum     & 0,16–0,65                     \\
Beta vulgaris            & 0,70                          & Theobroma cacao      & 2,08–2,90                     \\
Helianthus annuus, hulls & 1,23–2,27                     & Gossypium barbadence & 0,28–0,71                     \\
Lactuca sativa           & 1,32                          & Hordeum vulgare      & 0,42–4,70                     \\
Oryza sativa             & 2,72–8,40                     & Secale cereale       & 0,46–1,23                     \\
Triticum aestivum        & 0,16–3,11                     & Zea mays             & 0,32–0,78                     
\end{longtblr}

\begin{multicols}{2}
В 1870 году великий русский химик Д.И.Менделеев предложил использовать
аморфный диоксид кремния в качестве кремниевого удобрения. Он предложил
провести первые в России полевые агрохимические опыты с этим
соединением. Спустя несколько лет Максвелл, американский почвовед и
агрохимик, провел первые исследования по наличию кремния в различных
почвах для растений {[}15,16{]}.

{\bfseries Материалы и методы.} Способ получения биопрепарата кремния из
золы растений очень эффективен. Кремний -- макроэлемент, питающий
растения зольного типа, а его соединения относятся к группе неотъемлемых
компонентов любого растительного организма. Количество кремния в золе
культурных растений в среднем составляет от 0,16 до 8,4\%. Наибольшее
количество Si содержится в злаках, его количество достигает 8-16\%, а в
рисе - 15-20\% SiO\tsb{2}. Применение кремнийсодержащих
веществ в зерновых культурах способствует увеличению листовой
поверхности растений, стимулирует общий рост, ускоряет начало фаз
созревания зерна {[}17{]}.
\end{multicols}

\begin{figs}
  \fig[0.45\textwidth][0.75\textwidth]{c/image26}[Рис.2 - Рисовая шелуха]
  \fig[0.45\textwidth][0.75\textwidth]{c/image27}[Рис.3 - Цвета сгоревшей рисовой шелухи при разных температурах]
\end{figs}

В качестве источника активного кремния в селекции растений
рассматривалось свойства аморфный диоксид кремния из рисовой шелухи
(таблица 2)

\tcap{Таблица 2 - Свойства аморфного диоксида кремния из рисовой шелухи.}
\begin{longtblr}[
  label = none,
  entry = none,
]{
  width = \linewidth,
  colspec = {Q[300]Q[71]Q[54]Q[56]Q[69]Q[62]Q[65]Q[56]Q[124]},
  cells = {c},
  cells = {font = \small},
  hlines,
  vlines,
}
\textbf{Материал}                         & {\textbf{pH}\\\textbf{(H2O)}\\\textbf{\%}} & {\textbf{CaO}\\\textbf{\%}} & {\textbf{MgO}\\\textbf{\%}} & {\textbf{Fe2O3}\\\textbf{\%}} & {\textbf{Р2О5}\\\textbf{\%}} & {\textbf{Al2O3}\\\textbf{\%}} & {\textbf{SiO2}\\\textbf{\%}} & {\textbf{SiO2 аморфный}\\\textbf{\%}} \\
аморфный диоксид кремния (рисовая шелуха) & 6,2-8,5                                    & 0                           & 0,31                        & 0,57                          & 0                            & 0,23                          & 93,1                         & 90                                    
\end{longtblr}

\fig[0.9\textwidth]{c/image28}[{\normalfont Температура, °С}\\Рис.4 - Масса золы рисовой шелухи после термообработки, м/г]

\begin{multicols}{2}
К. Аскарулы и др. в статье рисовая шелуха - это остаток
сельскохозяйственного производства, из тонн отходов рисоводства ежегодно
в лабораторных условиях можно получить важный биопрепарат кремния для
растений {[}18{]}. Прежде всего рисовую шелуху несколько раз промывают
проточной водой, а затем сушат в течение 12 часов при температуре 120°С.
50 г сушеной рисовой шелухи взвешивают и нагревают в муфельной печи при
500, 550, 600, 650, 700, 750, 800 и 850°С в течение 4 часов. В процессе
нагрева были выявлены следующие условия. При 650-700°С оксид кремния
начинает образовывать кристаллическую структуру; При 800°С углерод
сгорает полностью (остается чистый кремний); рисовая шелуха меняет цвет
в зависимости от температуры, например: 500-550°С цвет желто-серый, 600-
650°С цвет белый, выше 700°С имеет сероватый оттенок, как показано на
рисунке - 4.

Полученный диоксид кремния химическим способом растворяют и положить в
стеклянный посуду, к нему был добавлен гидроксид натрия (NaOH) получили
силикагель (Na\tsb{2}SiO\tsb{3}) и перемешивали при
95°С в течение 2 ч на магнитной мешалке (метод непрерывного
перемешивания) {[}19,20{]}.

Химическая реакция:

\begin{equation}
\mathrm{NaOH} + \mathrm{SiO_2} \rightarrow \mathrm{Na_2SiO_3} + \mathrm{H_2}
\end{equation}

Затем к полученной белой жидкости (силикагелю) добавляли соляную кислоту
(концентрированная HCl) для уменьшения содержания твердого кремнезема.
После этого оксид кремния фильтровали. В результате образуется чистый
кремнезем, который несколько раз промывают горячей дистиллированной
водой для удаления хлора (Cl) {[}21{]}.

Химическая реакция:
\vspace{-0.3em}
\begin{equation}
\mathrm{Na_2SiO_3} + 2\mathrm{HCl} \rightarrow \mathrm{H_2SiO_3}\downarrow + 2\mathrm{NaCl}
\end{equation}
\vspace{-1em}
\begin{equation}
\mathrm{H_2SiO_3}\,(\text{кремниевая кислота}) \rightarrow \mathrm{SiO_2} + \mathrm{H_2O}
\end{equation}

Вышеуказанные факторы побудили нас рассмотреть альтернативный материал -
инновационную технологию - для производства кремниевых биопрепаратов.
Для этого мы сначала изучили состав почвы. Поскольку кремний является
одним из наиболее распространенных элементов земной коры, трудно оценить
его роль в формировании плодородия почвы и многообразии процессов,
происходящих в почве.

{\bfseries Обсуждение и результаты.} В ряде отечественных и зарубежных
работ отмечается особое значение кремния в формировании различных
агрохимических и агрофизических свойств почвы, а также в контроле многих
геохимических и почвенных процессов. Влажность почвы -- это процентное
содержание воды в почве (диаграмма-3,4 ). Этот показатель не является
морфологическим признаком, от него зависит проявление почти всех
морфологических свойств. Это объясняется морфогенетическими
особенностями почвенных участков при лабораторном определении влажности
почвы.

Процент влажности почвы без удобрений составляет 21,1\% на глубине 0-20
см; 12,9\% на глубине 20-31 см; 7,6\% на глубине 31-62 см; 26,1\% на
глубине 62-85 см; 20,3\% на глубине 85-92 см;

В некоторых горизонтах слоев почвы, подвергающихся деградации без
первоначального внесения удобрений, снижение влажности почвы происходит
при содержании каменистых и щебнистых плотных пород, где влага не
удерживается и происходит процесс фильтрации.

На рисунке - 6 показано, что при выращивании растений с использованием
биопрепарата кремния влажность почвы стабилизируется.

Процент влажности почвы с удобрением составляет 26\% на глубине 0-20 см;
23,7\% на глубине 20-31 см; 15,2\% на глубине 31-62 см; 12,9\% на
глубине 62-85 см; 12,1\% на глубине 85-92 см.

Значительное наличие почвенной влаги в верхних горизонтах объясняется
накоплением влаги в плотном плодородном слое почвы и обильной подстилкой
растений, что снижает процесс испарения.

Степень и направление влияния кремниевых удобрений и почвенных
мелиорантов на физические свойства почвы зависит от свойств почвы и
вносимых удобрений {[}22,23{]}. Мунк сообщает об улучшении физических
свойств почвы при дозах кремнезема 200--800 кг/га в год {[}24{]}.
Поликремниевые кислоты, образующиеся при применении кремниевых
препаратов, способны связывать частицы почвы и способствуют улучшению
структуры почвы за счет образования кремниевых мостиков между зернами
осадка. При этом повышаются агрегативная, влагоемкость, обменная и
буферная способность легких почв. Общие физические свойства включают
плотность почвы, плотность твердых частиц и пористость. Объемная масса
почвы - это масса единицы объема абсолютно сухой пробы почвы, взятой в
ненарушенном поле. Насыпная плотность 1,5 г/см\tsp{3} и
более свидетельствует о чрезмерной плотности почвы, что создает
неблагоприятные условия для растений. Для большинства культурных
растений оптимальная плотность полевого горизонта составляет 1,0-1,2
г/см³ {[}25,26{]}. Оценки плотности верхнего слоя почвы приведены в
табл.3.
\end{multicols}

\fig[0.8\textwidth]{c3/image38}[Рис.5 - Процент влажности почвы без удобрений]

\fig[0.8\textwidth]{c3/image39}[Рис.6 - Процент влажности почвы после внесения биопрепарата кремния]

\begin{multicols}{2}
Теперь мы можем увидеть плотность деградированной почвы без удобрений на
нашем объекте исследования (рисунок 7). Данные показывают, что объемная
плотность деградированных почв в верхнем горизонте составляет 1,2
г/см\tsp{3}, а в почвенном слое наблюдается постепенное
уплотнение почвы от 10 до 50 см - 1,4 г/см\tsp{3}.

Для анализа исследовали неповрежденную почву и определили объемную
массу. Объемная плотность в верхнем горизонте составляет 0,9
г/см\tsp{3}, а в нижних - 1,2 г/см\tsp{3}
(рисунок 8).
\end{multicols}

\tcap{Таблица 3 - Оценка плотности грунта}
\begin{longtblr}[
  label = none,
  entry = none,
]{
  width = \linewidth,
  colspec = {Q[110]Q[322]Q[110]Q[346]},
  cells = {c},
  cells = {font = \small},
  hlines,
  vlines,
}
{\textbf{Плотность,}\\\textbf{г/см³}} & \textbf{Оценка}                                    & {\textbf{Плотность,}\\\textbf{г/см³}} & \textbf{Оценка}                                          \\
1,0                                   & Почва рыхлая или богатая органическими веществами. & 1,3 – 1,4                             & Почва очень уплотненная                                  \\
1,0 – 1,1                             & Свежевспаханная почва                              & 1,4 - 1,6                             & Типичные значения орошаемых горизонтов (кроме чернозема) 
\end{longtblr}

{\bfseries Рис.7 - Плотность почвы без удобрений}

{\bfseries Рис.9 - Плотность грунта с кремниевым биопрепаратом}

\begin{multicols}{2}
Из-за гибели растения свойства почвы, особенно физические, существенно
изменяются. Уплотнение почвы, вызванное копытами домашнего скота,
является неизбежным следствием чрезмерного выпаса скота. Негативное
воздействие человека на природную среду проявляется не только в утрате
биологического разнообразия и устойчивости экосистем, но и в
значительном снижении продуктивности природных и антропогенных
ландшафтов из-за утраты плодородия почв в результате прогрессирующего
землетрясения. разработка. процессы их разложения (эрозия, дефляция,
дегумификация, уплотнение, засоление и др.).

Для сравнения из такого же количества исследовали почву с удобрением и
определили объемную массу. Объемная плотность в верхнем горизонте
составляет 1,2 г/см\tsp{3}, а в нижних - 0,81
г/см\tsp{3}.

{\bfseries Выводы}. Технология получения важного для растения биопрепарата
кремния создана в лабораторных условиях с использованием технологии
термической обработки рисовой шелухи К. Аскарулы.

Влажность почвы определяли относительно деградированной почвы и почвы,
обработанной биопрепаратом кремния. Значительное количество влаги в
верхнем горизонте объясняется накоплением влаги в плотном плодородном
слое почвы и обильной подстилкой растений, что снижает процесс
испарения. Уменьшение влажности почвы связано с тем, что влага не
сохраняется в плотных каменисто-гравийных породах и происходит процесс
фильтрации. Процент влажности почвы без удобрений составляет 21,1\% на
глубине 0-20 см; 12,9\% на глубине 20-31 см; 7,6\% на глубине 31-62 см;
26,1\% на глубине 62-85 см; 20,3\% на глубине 85-92 см. Процент
влажности почвы с удобрением составляет 26\% на глубине 0-20 см; 23,7\%
на глубине 20-31 см; 15,2\% на глубине 31-62 см; 12,9\% на глубине 62-85
см; 12,1\% на глубине 85-92 см.

Плотность почвы определяли в сравнении с деградированной почвой и почвой
с биопрепаратом кремния. Данные показали, что объемная масса выветрелой
почвы представляет собой постепенное уплотнение почвы в почвенном слое.
В почвах, обработанных биопрепаратом кремния, процесс уплотнения
протекает слабо, поскольку кремний обеспечивает циркуляцию воды и других
минеральных веществ в почве. Для анализа исследовали неповрежденную
почву и определили объемную массу. Объемная плотность в верхнем
горизонте составляет 0,9 г/см\tsp{3}, а в нижних - 1,2
г/см\tsp{3}.

Для сравнения исследовали почву с удобрением и определили объемную
массу. Объемная плотность в верхнем горизонте составляет 1,2
г/см\tsp{3}, а в нижних - 0,81 г/см\tsp{3}

Разработанная технология получения кремниевого биопрепарата из рисовой
шелухи демонстрирует высокую эффективность, экологическую безопасность и
перспективность для внедрения в агропромышленный комплекс. Использование
термической обработки удается максимально сохранить и активировать
биологически ценную составляющую сырья.

Внедрение системы применения инновационных органических удобрений и
многофункциональных биологических препаратов в технологии выращивания
растений в рамках современного органического земледелия обеспечивает
экологическую защиту полевых культур. А также поможет обеспечить баланс
элементов минерального питания и органического вещества в почве без
дефицита и получать экологически чистую растительную продукцию для
производства продовольствия.
\end{multicols}

\begin{center}
{\bfseries Литература}
\end{center}

\begin{refs}
1. Mehrabanjoubani P., Abdolzadeh A., Sadeghipour H. R.. Impacts of
silicon nutrition on growth and nutrient status of rice plants grown
under varying zinc regimes // Theoretical and Experimental Plant
Physiology. - 2014.- Vol.27. - P.19-29.DOI
\href{http://dx.doi.org/10.1007/s40626-014-0028-9}{10.1007/s40626-014-0028-9}.

2. Etesami, H., Adl S. M. Can interaction between silicon and
non-rhizobial bacteria help in improving nodulation and nitrogen
fixation in salinity-stressed legumes? A review // Rhizosphere. - 2020.
- Vol.15(22).- Р.100 - 229.
\href{https://doi.org/10.1016/j.rhisph.2020.100229}{DOI
10.1016/j.rhisph.2020.100229}.

3. Fatema K. The effect of silicon on strawberry plants and it's role in
reducing infection by Podosphaera aphanis// Doctoral thesis, University
of Hertfordshire. - 2014. DOI
\href{https://doi.org/10.18745/th.14445}{10.18745/th.14445}.

4. Sakr N. The role of silicon (Si) in increasing plant resistance
against fungal diseases // Hellenic Plant Protection Journal. - 2016. -
Vol.9 (1). - P.1-15. DOI 10.1515/hppj-2016-0001.

5. Anggraeni L. W. {[}et al.{]} Effect of biostimulant and silica
application on sugarcane (Saccharum officinarum L.) production // IOP
Conference Series: Earth and Environmental Science. - 2022. - Vol.
974(1). - P.12-17. DOI
\href{http://dx.doi.org/10.1088/1755-1315/974/1/012077}{10.1088/1755-1315/974/1/012077}.

6. Shen X. F. {[}et al.{]} Effects of intercropping with peanut and
silicon application on sugarcane growth, yield and quality // Sugar
Tech. - 2019. - Vol.21(3). - P.437-443. DOI
\href{http://dx.doi.org/10.1007/s12355-018-0667-2}{10.1007/s12355-018-0667-2}.

7. Atencio R. {[}et al.{]} Effect of silicon and nitrogen on Diatraea
tabernella Dyar in sugarcane in Panama // Sugar Tech. - 2019. - Vol.
21(1). - P.113--121.
DOI~\href{https://doi.org/10.1007/s12355-018-0634-y}{10.1007/s12355-018-0634-y}.

8. Etienne P. {[}et al.{]} Root silicon treatment modulates the shoot
transcriptome in Brassica napus L. and in particular upregulates genes
related to ribosomes and photosynthesis // Silicon. -- 2020. -- Vol.13.
- Р.4047-4055. DOI 10.1007/s12633-020-00710-z.

9. Mahmoud L.M. {[}et al.{]} Silicon nanoparticles mitigate oxidative
stress of in vitro-derived banana (Musa acuminata `Grand Nain') under
simulated water deficit or salinity stress // South African Journal of
Botany. - 2020. - Vol.132. - P.155-163.
\href{https://doi.org/10.1016/j.sajb.2020.04.027}{DOI
10.1016/j.sajb.2020.04.027}.

10. Teixeira G.C.M. {[}et al.{]} Silicon increases leaf chlorophyll
content and iron nutritional efficiency and reduces iron deficiency in
sorghum plants // Journal of Soil Science and Plant Nutrition. - 2020. -
Vol.20. - P.1311-1320.
\href{https://doi.org/10.1007/s42729-020-00214-0}{DOI
10.1007/s42729-020-00214-0}.

11. Etesami H., Jeong B.R. Silicon (Si): Review and future prospects on
the action mechanisms in alleviating biotic and abiotic stresses in
plants // Ecotoxicology and environmental safety. - 2018. - Vol.147. -
P.881-896. \href{https://doi.org/10.1016/j.ecoenv.2017.09.063}{DOI
10.1016/j.ecoenv.2017.09.063}.

12. Bhat J.A. {[}et al.{]} Role of silicon in mitigation of heavy metal
stresses in crop plants // Plants. - 2019. - Vol.8(3). - P.71-78.
\href{https://doi.org/10.3390/plants8030071}{DOI 10.3390/plants8030071}.

13. Li Z. {[}et al.{]} Combined silicon-phosphorus fertilization affects
the biomass and phytolith stock of rice plants//Frontiers in plant
science.- 2020. - Vol.11.-P.67.
\href{https://doi.org/10.3389/fpls.2020.00067}{DOI
10.3389/fpls.2020.00067}.

14. Laîné P. {[}et al.{]} Silicon promotes agronomic performance in
Brassica napus cultivated under field conditions with two nitrogen
fertilizer inputs//Plants.-2019.-Vol.8(5):137.
\href{https://doi.org/10.3390/plants8050137}{DOI 10.3390/plants8050137}.

15. Wang L. {[}et al.{]} Effects of silicon and phosphatic fertilization
on rice yield and soil fertility // Journal of Soil Science and Plant
Nutrition.- 2020.-Vol.20(5).-P.557-565.
\href{https://ui.adsabs.harvard.edu/link_gateway/2020JSSPN..20..557W/doi:10.1007/s42729-019-00145-5}{DOI
10.1007/s42729-019-00145-5}.

16. Z. Ahmad {[}et al.{]} Enhancing drought tolerance in wheat through
improving morpho-physiological and antioxidants activities of plants by
the supplementation of foliar silicon // Phyton-International Journal of
Experimental Botany.-2020.-Vol 89 (3). - P.529-539. DOI
10.32604/phyton.2020.09143.

17. Araujo L., Paschoalino R., Rodrigues F. Microscopic. Aspects of
Silicon--Mediated Rice Resistance to Leaf Scald // Phytopathology. -
2015. - Vol.106(2). - P.132-141.
\href{https://doi.org/10.1094/PHYTO-04-15-0109-R}{DOI
10.1094/PHYTO-04-15-0109-R}.

18. Askaruly K., Azat S., Eleuov M., Kerimkulova A.R., Zhantikeev U.N.,
Berdihanov A.E. Poluchenie oksida kremnija iz risovoj sheluhi metodom
termicheskoj obrabotki // Gorenie i plazmohimija.-2019.-№17(3).- S.
183-188. \href{https://doi.org/10.18321/cpc324}{DOI 10.18321/cpc324}.
{[}in Russian{]}

19. Anggraeni L. W. {[}et al.{]} Effect of biostimulant and silica
application on sugarcane (Saccharum officinarum L.) production // IOP
Conference Series: Earth and Environmental Science. - IOP
Publishing.-2022.-Vol.974(1). - P.12-17. DOI
10.1088/1755-1315/974/1/012077.

20. Moradtalab N. {[}et al.{]} Silicon improves chilling tolerance
during early growth of maize by effects on micronutrient homeostasis and
hormonal balances // Frontiers in plant science. - 2018.-Vol 9: 420.
\href{https://doi.org/10.3389/fpls.2018.00420}{DOI
10.3389/fpls.2018.00420}.

21. Kim Y. H. {[}et al.{]} Silicon regulates antioxidant activities of
crop plants under abiotic-induced oxidative stress: a review//Frontiers
in Plant Science.-2017.-Vol.8:510.
\href{https://doi.org/10.3389/fpls.2017.00510}{DOI
10.3389/fpls.2017.00510}.

22. Luyckx M. {[}et al.{]} Silicon and plants: current knowledge and
technological perspectives // Frontiers in Plant Science.-2017.- Vol.
8. -P.411-418. \href{https://doi.org/10.3389/fpls.2017.00411}{DOI
10.3389/fpls.2017.00411}.

23. Bocharnikova, E. A. Matichenkov V. V. Influence of plant
associations on the silicon cycle in the soil-plant ecosystem // Applied
Ecology and Environmental Research.-2012. -Vol.10(4). -P.547-560.
DOI 10.15666/AEER\%2F1004\_547560.

24. Zargar S. M. {[}et al.{]} Role of silicon in plant stress tolerance:
opportunities to achieve a sustainable cropping system // 3
Biotech.-2019. -Vol.9: 73.
\href{https://doi.org/10.1007/s13205-019-1613-z}{DOI
10.1007/s13205-019-1613-z}.

25. Hawerroth С., Araujo L., Bermúdez-Cardona M. Silicon-mediated maize
resistance to macrospora leaf spot // The journal of Tropical Plant
Pathology.- 2018.-Vol.44.- P.192-196.
\href{https://ui.adsabs.harvard.edu/link_gateway/2019TroPP..44..192H/doi:10.1007/s40858-018-0247-8}{DOI
10.1007/s40858-018-0247-8}.

26. Ren Z., Gao H., Zhang H., Liu X.. Effects of fluxes on the structure
and filtration properties of diatomite filter aids // International
Journal of Mineral Processing. - 2014. - Vol.130. - P.28--33.
\href{https://doi.org/10.1016/j.minpro.2014.05.011}{DOI
10.1016/j.minpro.2014.05.011}.
\end{refs}

\begin{info}
\hspace{1em}\emph{{\bfseries Сведения об авторах}}

Сулейменова М.Ш. - к.хн., доцент, Алматинский технологический
университет, Алматы, Казахстан, e-mail: s.mariyash@mail.ru;

Дәуметова С.Т. - магистр, Алматинский технологический университет,
Алматы, Казахстан, e-mail: daumetova83@mail.ru;

Esin Varol - профессор Др, Эскишехирский технический университет,
Эскишехир, Турция, e-mail: eapaydin@eskisehir.edu.tr;

Бугубаева Г. - к.х.н., ассистент професора Алматинский технологический
университет, Алматы, Казахстан, e-mail: bugub@mail.ru;

Жетенова М.С. - докторант, Алматинский технологический университет,
Алматы, Казахстан, e-mail: zhetenova\_madina@mail.ru.

\hspace{1em}\emph{{\bfseries Information about the authors}}

Suleimenova M.-Associate Professor, Almaty Technological University,
Almaty, Kazakhstan, e-mail: s.mariyash@mail.ru;

Daumetova S.- Master, Almaty Technological University, Almaty,
Kazakhstan, e-mail: daumetova83@mail.ru;

Esin Varol - Professor Dr., Eskisehir Technical University, Eskisehir,
Turkey, e-mail: eapaydin@eskisehir.edu.tr;

Bugubaeva G. - Assistant Professor, Almaty Technological University,
Almaty, Kazakhstan, e-mail: bugub@mail.ru;

Zhetenova М. - PhD student, Almaty Technological University, Almaty,
Kazakhstan, e-mail: zhetenova\_madina@mail.ru.
\end{info}
