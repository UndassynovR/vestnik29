\id{ҒТАМР 31.21.25}{}

\begin{header}
\swa{}{КӨРІНЕТІН СПЕКТРДЕ БЕЛСЕНДІ СІҢІРУ ЖӘНЕ ШЫҒАРУЫ БАР ДОНОРЛЫ АЛМАСТЫРҒАН АЗУЛЕН ҚОСЫЛАСТАРЫ}

\tsp{1}А.Н. Искандеров\envelope,
\tsp{1}Н. Мерхатұлы,
\tsp{2}С.К. Жокижанова,
\tsp{1}А.Н. Искандеров,
\tsp{3}Х.Б. Омаров,
\tsp{1}А.О.~Булумбаева
\end{header}

\begin{affil}
\tsp{1}Е.А.Бөкетов атындағы Қарағанды Мемлекеттік университеті, Қарағанды, Қазақстан,

\tsp{2}С.Сейфуллин атындағы Қазақ агротехникалық зерттеу университеті, Астана, Қазақстан

\tsp{3}Қ.Құлажанов атындағы технология және бизнес университеті, Астана, Қазақстан

\corrauthor{Корреспондент-автор: e-mail:aby93@yandex.kz}
\end{affil}

Өзектілігі. Бірегей құрылымы бар ароматты көмірсутек азулен жаңа
органикалық функционалды материалдарды жасау үшін перспективалы
модульдік компонент ретінде көбірек қарастырылуда. Оның қызығушылығы
ерекше поляризацияланған құрылымымен, сондай-ақ ерекше электрондық және
оптикалық қасиеттерімен байланысты. Азуленнің көгілдір түсі бар, ол
тыйым салынған π→π* электронды ауысудың салдары болып табылады, сонымен
қатар аномальды анти-Каша флуоресценциясын көрсетеді. Осы зерттеуде
донор-алмастырылған топтары бар жаңа азулен қосылыстары,
2,6-бис(9Н-карбазол)-азулен және 2,6-бис(N,N-дифениланилин)-азулен
палладий катализаторын пайдаланып дибромоазуленді карбазолмен және
бориланилинмен айқастыру арқылы алынды.

Әдістер. Негізгі синтетикалық әдістер ретінде ирридий және палладий
катализаторлары қатысатын қазіргі кросс-қосу реакциялары қолданылды.
Синтезделген қосылыстардың химиялық құрылымы мен тазалығы ядролық
магниттік (\tsp{1}Н және \tsp{13}С) және
инфрақызыл спектрометрия және жоғары ажыратымдылықтағы
масс-спектрометрия (HRMS) арқылы дәлелденді.

Негізгі тұжырымдар. Алынған қосылыстар көрінетін жарықтың қарқынды
жұтылуымен және шығарылуымен (толқын ұзындығы 400-ден 600 нм-ге дейін)
сипатталатыны тәжірибе жүзінде анықталды. Зерттеулер көрсеткендей,
фотолюминесценцияның көк-жасыл аймағында көрінетін жарықтың
флуоресцентті сәулеленуін қоса алғанда, бірегей фотофизикалық
сипаттамалар азуленнің 2 және 6 позицияларына электрон беретін карбазол
мен дифениланилин топтарының қосылуы есебінен алынған. Зерттеулер
көрсеткендей, алынған қосылыстардың электронды құрылғыларда қолданылатын
олиготиофендермен салыстырғанда жоғары HOMO (ең жоғары орналасқан
молекулалық орбиталь) орбитальдары бар және толқын ұзындығының кең
спектрінде күшті сіңіру қабілетін көрсетеді.

Практикалық құндылық. Алынған мәліметтер оптоэлектронды және фотоникалық
құрылғыларда қолдануға арналған азулен негізіндегі жаңа конъюгацияланған
қосылыстарды ұтымды жобалауға негіз болады.

{\bfseries Түйін сөздер}: азулендер, карбазол, фениланилин, бірігу,
электрондардың жұтылуы және эмиссиясы.

\begin{header}
ДОНОРНО-ЗАМЕЩЕННЫЕ АЗУЛЕНОВЫЕ СОЕДИНЕНИЯ С ИНТЕНСИВНЫМ ПОГЛОЩЕНИЕМ И ИЗЛУЧЕНИЕМ В ВИДИМОМ СПЕКТРЕ

\tsp{1}А.Н. Искандеров\envelope,
\tsp{1}Н. Мерхатұлы,
\tsp{2}С.К. Жокижанова,
\tsp{1}А.Н. Искандеров,
\tsp{3}Х.Б. Омаров,
\tsp{1}А.О.~Булумбаева
\end{header}

\begin{affil}
\tsp{1}Карагандинский университет им. Е.А. Букетова, Караганда, Казахстан,

\tsp{2}Казахский агротехнический исследовательский университет им. С.Сейфуллина, Астана, Казахстан,

\tsp{3}Казахский университет технологии и бизнеса им. К.Кулажанова, Астана, Казахстан,

e-mail: aby93@yandex.kz
\end{affil}

Актуальность. Азулен, неальтернантный ароматический углеводород с
уникальной структурой, все чаще рассматривается как перспективный
модульный компонент для создания новых органических функциональных
материалов. Его интерес обусловлен специфическим поляризованным
строением, а также исключительными электронными и оптическими
характеристиками. Азулен имеет синий цвет, являющейся следствием
запрещенного π→π* электронного перехода, а также демонстрирует
аномальную анти-Каша флуоресценцию. В ходе настоящего исследования путем
кросс-сочетания дибромазулена с карбазолом и бориланилином при
использовании палладиевого катализатора были получены новые азуленовые
соединения с донорно-замещенными группами: 2,6-бис(9H-карбазолил)-азулен
{\bfseries 5} и 2,6-бис(N,N-дифениланилинил)-азулен {\bfseries 7}.

Методы. Для синтеза ключевых соединений были применены современные
реакции кросс-сочетания, опосредованные катализаторами на основе иридия
и палладия. Структурная идентичность и чистота полученных соединений
подтверждены посредством ЯМР-спектроскопии (\tsp{1}H и
\tsp{13}C) и ИК-спектроскопии, а также масс-спектрометрии
высокого разрешения (HRMS).

Основные выводы. Научные исследования подтвердили, что синтезированные
соединения демонстрируют интенсивное поглощение и излучение света
видимого спектра (в диапазоне волн от 400 до 600 нм). Анализ показал,
что введение электронодонорных карбазольных и дифениланилиновых групп в
2 и 6 позиции азулена привело к уникальным фотофизическим свойствам,
среди которых особое внимание заслуживает флуоресцентное излучение
сине-зеленого спектра при фотолюминесценции. Анализы выявили, что
созданные соединения характеризуются более высокими орбиталями HOMO по
отношению к олиготиофенам, используемым в электронных приборах, и
обладают выраженной способностью поглощать свет в широком диапазоне длин
волн.

Практическая значимость. Полученные результаты предоставляют основу для
разработки новых азуленовых сопряженных соединений, которые могут быть
эффективно использованы в оптоэлектронных и фотонных устройствах.

{\bfseries Ключевые слова:} азулены, карбазол, фениланилин, сочетание,
электронное поглощение и эмиссия.

{\bfseries DONOR-SUBSTITUTED AZULENE COMPOUNDS WITH INTENSE ABSORPTION AND
EMISSION IN THE VISIBLE SPECTRUM}

{\bfseries \tsp{1}A.N. Iskanderov\envelope  ,
\tsp{1}N.Merkhatuly, \tsp{2}S.K.Zhokizhanova,
\tsp{1}A.N.Iskanderov,}

{\bfseries \tsp{3}Kh.B.Omarov, \tsp{1}A.O.
Bulumbaeva}

\emph{\tsp{1} Karaganda Buketov University, Karaganda,
Kazakhstan,}

\emph{\tsp{2}S.Seifullin Kazakh Agrotechnical Research
University Astana, Kazakhstan,}

\emph{\tsp{3} K.Kulazhanov Kazakh University of Technology
and Business, Astana, Kazakhstan,}

e-mail: aby93@yandex.kz

Relevance. Azulene, an aromatic hydrocarbon with a unique structure, is
increasingly being considered as a promising modular component for the
creation of new organic functional materials. Its interest is due to its
specific polarised structure, as well as its exceptional electronic and
optical characteristics. In this study, new azulenes with
donor-substituted groups were obtained by cross-coupling dibromazulenes
with carbazole and borylaniline using a palladium catalyst:
2,6-bis(9H-carbazole)-azulene 5 and 2,6-bis(N,N-diphenylaniline)-azulene
7. Methods. Modern cross-coupling reactions involving iridium and palladium
catalysts were used as key synthetic methods. The structure and purity
of the synthesized compounds were confirmed by NMR (\tsp{1}H
and \tsp{13}C) and IR spectrometry and HRMS.

Main conclusions. Experiments have shown that the compounds obtained are
characterised by intense absorption and emission of visible light (in
the wavelength range from 400 to 600 nm). Studies have shown that unique
photophysical characteristics, including visible light fluorescence in
the blue-green region of photoluminescence, were obtained due to the
inclusion of electron-donating carbazole and diphenylaniline groups in
the 2 and 6 positions of azulene. Studies have shown that the resulting
compounds have higher HOMO orbitals compared to oligothiophenes used in
electronic devices and demonstrate strong absorption capacity across a
wide range of wavelengths.

Practical significance. The data obtained provides a basis for the
rational design of new azulen-based conjugated compounds intended for
use in optoelectronic and photonic devices.

{\bfseries Keywords:} azulenes, carbazol, phenylanilin, coupling,
electronic absorption and emission.

{\bfseries Кіріспе}. Кеңейтілген π-электрондық байланыс жүйелері бар
ароматтық қосылыстардың кеңінен қолданылуы олардың органикалық
оптоэлектроника саласындағы функционалды материалдар ретінде маңызды рөл
атқаруымен негізделеді.\\
Соңғы уақытта ароматтық және гетероароматтық қосылыстарға қызығушылық
артып келеді, әсіресе оларды электронды акцепторлық және/немесе донорлық
қасиеттерге ие арил топтарымен түрлендіруге баса назар аударылуда.
Мысалы, диариламино-замещенген нафталиндер мен антрацендер айтарлықтай
электрондонорлық қасиеттерімен белгілі және оларды әртүрлі салаларда
қолдану мүмкіндігіне байланысты зерттеушілердің назарын өзіне аударуда,
мысалы, органикалық жартылай өткізгіштерде, тотығу-тотықсыздану
реакцияларына қатысатын материалдарда, металлорганикалық қаңқаларда
(MOFs) және органикалық жарық шығаратын диодтарда (OLEDs) (Freudenberg,
et.al., 2018; Roy, et.al., 2022; Zhang, et.al., 2016; Mayer, et.al.,
2019; Wu, et.al., 2009; Taniguchi, et.al., 2021; Noto, et.al., 2019)
{[}1-3{]}.

Осы контексте нафталиннің құрылымдық изомері -азуленге (Tsuchiya,
et.al., 2019; Xin, et.al., 2021) ерекше назар аудару қажет. Азулен
өзінің бірегей ароматтық көмірсутектік құрылымының арқасында ерекше
электрондық және спектралдық қасиеттерді көрсетеді. Оның қасиеттерінің
ішінде дипольдік моменті 1,08 D болатын полярланған құрылым мен Каши
заңына бағынбайтын, яғни S2→S0 ауысуын көрсететін ерекше флуоресценцияны
атап өтуге болады (Dunlop, et.al., 2023) {[}4,5{]}. Азуленнің
молекулалық құрылымын тропилий катионы мен циклопентадиенил анионының
конденсация өнімі ретінде интерпретациялауға болады, бұл 1-суретте
көрсетілген. Мұндай ерекше архитектура типтік ароматтық көмірсутектермен
салыстырғанда HOMO деңгейінің жоғарылауына және LUMO деңгейінің
төмендеуіне алып келеді (Tsuchiya, et.al., 2019; Xin, et.al., 2021).
Азуленнің 1 және 3 позицияларында HOMO-де үлкен атомдық орбитальдық
коэффициенттер байқалса, ал 2 және 6 позицияларында -HOMO-1 және LUMO-де
жоғары коэффициенттер анықталады (Tsuchiya, et.al., 2019; Xin, et.al.,
2021). Түссіз нафталиннен айырмашылығы, азулен қою көк реңкке ие, бұл
шамамен 580 нм аймағындағы λmax мәнімен сипатталатын S0-S1 ауысуына
байланысты жұтылу қасиетін көрсетеді, дегенмен π--π* электрондық ауысу
үшін мольдік сіңіру коэффициенті аз ғана -бар болғаны 350 М⁻¹см⁻¹
(Shevyakov, et.al., 2003) {[}6,7{]}.

Осылайша, азулен ядросының белгілі бір позицияларына донорлық
карбазолдық және дифениланилиндік топтарды енгізу азулен жүйесінің
электрондық құрылымын айтарлықтай өзгерте алады, бұл оның оптикалық
қасиеттерінің елеулі түзетулеріне алып келеді. Азулен ядросындағы
орынбасарлардың орналасуын реттеу материалдардың оптикалық қасиеттерін
жоғары дәлдікпен өзгертуге мүмкіндік береді {[}8{]}. Осы зерттеуде біз
2- және 6-позицияларында электрондонорлық карбазолдық және
дифениланилиндік топтармен модификацияланған 5 және 7 азуленді
қосылыстарды сәтті алуды көрсетеміз. Сонымен қатар, олардың
фотофизикалық қасиеттері жан-жақты қарастырылады {[}9{]}. Мұндай
молекулалық архитектуралар азуленнің электрондық құрылымына айтарлықтай
әсер етіп, HOMO және LUMO орбитальдарының энергия деңгейлері мен олардың
арасындағы энергетикалық саңылауларды өзгертеді, бұл өз кезегінде
көрінетін спектр аймағында жарықтың жұтылуы мен сәулеленуінің күшеюіне
әкеледі {[}10{]}.



{\bfseries 1-сурет. Азуленнің полярланған резонанстық құрылымы 1}

{\bfseries Материалдар мен әдістер.} ¹H NMR және ¹³C NMR спектрлері JNM-ECA
500 спектрометрінде (500 МГц және 126 МГц, DMSO-D₆ ерітіндісінде)
тіркелді. ИҚ-спектрлер Avatar-360 ИҚ-спектрометрінде алынды. Жоғары
ажыратымдылықтағы масс-спектрлер (HRMS) Thermo Electron Corporation DFS
масс-спектрометрінде өлшенді. УФ-көрінетін спектрлер Shimadzu UV-1800
спектрофотометрінде өлшенді. Флуоресценция спектрлері Agilent Cary
Eclipse флуориметрі арқылы тіркелді. Балқу нүктелері Buchi M-560
аспабында анықталды.

Жұмыста келесі реактивтер мен еріткіштер пайдаланылды: азулен 1
(99,5\%), карбазол 4 (99,8\%),
N,N-дифенил-6-(4,4,5,5-тетраметил-1,3,2-диоксаборолан-2-ил)азулен-2-амин
6 (99,6\%), (Bpin)₂ (99\%), {[}IrCl(cod){]}₂ (98\%), Pd(OAc)₂ (98\%),
Pd(PPh₃)₂Cl₂ (98\%), 2,2′-bpy (98,5\%), CuBr (98,8\%), t-BuOK (99,2\%),
t-Bu₃PHBF₄ (98,8\%), K₂CO₃ (99,5\%), диметилформамид (ДМФ, 99,8\%),
толуол (99,6\%), ТГФ (99,7\%), дихлорметан (ДСМ, 99,8\%), сондай-ақ
басқа да «Sigma-Aldrich» компаниясының реактивтері.

\emph{2,6-бис(9H-карбазол)-азулен(5).} Дибромазулен 3 (286 мг, 1,00
ммоль), карбазол 4 (462 мг, 3,00 ммоль), палладий(II) ацетаты (12 мг,
0,05 ммоль), калийдің трет-бутоксиді (1,57 г, 14,0 ммоль) және
три-трет-бутилфосфоний тетрафторбораты (3,0 мг, 0,01 ммоль) 30 мл
толуолда ерітіліп, аргон атмосферасында 3 сағат бойы қайнатылды.

Содан кейін қоспа толуол және H₂O қосылып сұйылтылды және целит арқылы
сүзілді. Толуол қабаты натрий сульфатымен кептіріліп, роторлы
буландырғышта концентрленді. Алынған шикі өнім SiO₂ бағанасында
колонкалық хроматография әдісімен (еріткіш қоспасы C₆H₁₄/DCM, қатынасы
9:1) және DCM-нен қайта кристалдану арқылы тазартылды.\\
Нәтижесінде 367 мг қою қоңыр түсті қатты зат алынды (шығым 80\%).\\
Б.т (153--154°C). ИК-спектр (ν, см⁻¹): 2993, 2855, 1574, 1442, 1392,
1338, 1226, 1161, 748.\\
¹H ЯМР спектрі (δ, м.д.): 8.24 (д, J = 7.6 Гц, 2H), 8.06 (д, J = 7.6 Гц,
2H), 8.00 (д, J = 8.2 Гц, 2H), 7.76 (с, 2H), 7.50--7.32 (м, 14H). ¹³C
ЯМР спектрі (δ, м.д.): 140.21, 139.93, 139.85, 136.85, 127.17, 125.96,
124.18, 122.89, 118.95, 111.42. Жоғары ажыратымдылықтағы масс-спектр
(ESI) m/z: C₃₄H₂₃N₂ {[}M+H{]}⁺: есептелген 459.0374, табылған
459.0349. \\
Элементтік талдау (C₃₄H₂₂N₂): есептелген -C 89.05, H 4.84, N 6.11;
табылған -C 88.86, H 4.71, N 6.18.

\emph{2,6-бис(N,N-дифениланилин)-азулен (7}). Дибромазулен 3 (286 мг,
1,00 ммоль) және бориланилин 6 (462 мг, 3,00 ммоль) 10 мл
дегазацияланған ТГФ/H₂O (4:1) қоспасында ерітіліп, Pd(PPh₃)₂Cl₂ (16 мг,
0,02 ммоль) және K₂CO₃ (60 мг, 0,43 ммоль) аргон атмосферасында қосылды.
Қоспа 75°C температурада 18 сағат бойы араластырылды, содан кейін бөлме
температурасына дейін салқындатылып, ДСМ-мен (3 × 20 мл)
экстракцияланды. Біріктірілген экстракттар MgSO₄ үстінде кептіріліп,
вакуумда буландырылды. Алынған өнім SiO₂ үстінде колонкалық
хроматография арқылы (еріткіш қоспасы C₆H₁₄/ДСМ, 9:1) және ДСМ-нен қайта
кристалдану арқылы тазартылды.\\
Нәтижесінде 225 мг қоңыр түсті қатты зат алынды (шығым 83\%).\\
Б.т.214--216°C. ИК-спектр (ν, см⁻¹): 2921, 2850, 1580, 1481, 1421,
1342, 1275, 1161, 1033.\\
¹H ЯМР спектрі (δ, м.д.): 8.23 (д, J = 10.0 Гц, 2H), 7.82 (д, J = 8.5
Гц, 2H), 7.55 (с, 1H), 7.50 (с, 1H), 7.37 (дд, J = 9.8, 3.3 Гц, 4H),
7.37--7.35 (м, 4H), 7.30--7.25 (м, 8H), 7.15--7.12 (м, 16H). ¹³C ЯМР
спектрі (δ, м.д.): 149.60, 148.11, 147.90, 147.51, 147.44, 147.15,
141.51, 135.71, 137.25, 132.22, 130.23, 129.46, 129.30, 128.50, 125.82,
125.51, 125.22, 124.83, 124.24, 123.34, 123.24, 123.86, 123.42, 122.66,
114.05, 113.35. Жоғары ажыратымдылықтағы масс-спектр (ESI) m/z: C₄₆H₃₅N₂
{[}M+H{]}⁺: есептелген 614.0256, табылған 614.0231.\\
Элементтік талдау (C₄₆H₃₄N₂): есептелген -C 89.86, H 5.56, N 4.57;
табылған -C 89.56, H 5.42, N 4.64.

{\bfseries Нәтижелер мен талқылау.} 1-сызба азуленді қосылыстар
-2,6-бис(9H-карбазол)-азулен (5) және 2,6-бис(N,N-дифениланилин)-азулен
(7) синтезінің жолдарын көрсетеді.1-сызбада көрсетілгендей, борилазулен
2 азулен 1 мен (Bpin)₂ қосылыстарының қатысуымен Ir-катализатордың
көмегімен борилдену реакциясы арқылы алынды, бұл әдебиетте сипатталған
әдістемелерге сәйкес жүргізілді (Kurotobi, et al., 2003; Fujinaga, et
al., 2009) {[}11{]}. Кейіннен 2 қосылысының CuBr-пен әрекеттесуі
нәтижесінде негізгі аралық өнім -дибромазулен 3 70\% шығыммен алынды
(Narita, et al., 2018). Одан әрі 3 және 4 қосылыстарының палладий
ацетатының қатысуымен жүретін қосылу реакциясы соңғы өнім 5-ті 80\%
шығыммен алуға мүмкіндік берді (1-сызба). Дифениланилин-азулен 7
қосылысы да ұқсас жолмен, яғни 3 және 6 қосылыстарының ТГФ ерітіндісінде
Pd(PPh₃)₂Cl₂ катализаторының қатысуымен әрекеттесуі арқылы 83\% жоғары
шығыммен синтезделді (1-сызба). Донорлық орынбасарлары бар азулендердің
5 және 7 құрылымдары заманауи физика-химиялық әдістер арқылы анықталды
(материалдар мен әдістер бөлімі) {[}12{]}.



{\bfseries 1-сызба.5 және 7 қосылыстарын алу синтетикалық жолдары}

2-сурет пен 1-кестеде 5 және 7 қосылыстарының УФ--көрінетін аймақтағы
электрондық жұтылу спектрлері мен сипаттамалары келтірілген.5
қосылысының жұтылу спектрі 428 нм-де максимумға ие жаңа жұтылу жолағының
пайда болуын көрсетеді, молярлық жұтылу коэффициенті 43 486 М⁻¹ см⁻¹
(1-кесте).7 қосылысы да π--π* электрондық ауысуға сәйкес келетін 472
нм-дегі қарқынды жұтылу жолағын көрсетеді, молярлық жұтылу коэффициенті
36 253 М⁻¹ см⁻¹ (1-кесте) {[}13{]}. Көрінетін аймақтағы жоғары молярлық
жұтылу коэффициенттері донорлық фрагменттер мен азулен ядросы арасындағы
күшті өзара әрекеттесуді көрсетеді, бұл өз кезегінде жарық энергиясын
жұтудың тиімділігін арттырады.7 қосылысының көрінетін жұтылу максимумы
5 қосылысына қарағанда 44 нм-ге батохромды ығысқан.7-де
π-конъюгациясының кеңеюі мен электрондардың делокализациясының артуы
(5-сурет) оның HOMO--LUMO энергетикалық аралығының азаюына (4-сурет)
әкеледі, бұл моноалмастырылған азулен 5-пен салыстырғанда жұтылу
спектрлеріндегі батохромды ығысуды түсіндіреді{[}14{]}. Мұндай
құрылымдық модификация электрондық қасиеттердің айқынырақ өзгеруіне
себеп болып, нәтижесінде 2,6-бис-алмастырылған азулен жүйесінің
фотофизикалық сипаттамаларын жақсартады.5 және 7 қосылыстарының
электрондық жұтылуы бастапқы азулен 1-мен салыстырғанда әлдеқайда жоғары
(ε = 350 М⁻¹ см⁻¹) (Shevyakov және т.б., 2003) {[}8,9{]}.

{\bfseries 1-кесте.5 және 7a қосылыстарының УФ--көрінетін және
флуоресценция}

{\bfseries спектрлерінің деректері}

%% \begin{longtable}[]{@{}
%%   >{\raggedright\arraybackslash}p{(\linewidth - 10\tabcolsep) * \real{0.1560}}
%%   >{\raggedright\arraybackslash}p{(\linewidth - 10\tabcolsep) * \real{0.1587}}
%%   >{\raggedright\arraybackslash}p{(\linewidth - 10\tabcolsep) * \real{0.2034}}
%%   >{\raggedright\arraybackslash}p{(\linewidth - 10\tabcolsep) * \real{0.1747}}
%%   >{\raggedright\arraybackslash}p{(\linewidth - 10\tabcolsep) * \real{0.1263}}
%%   >{\raggedright\arraybackslash}p{(\linewidth - 10\tabcolsep) * \real{0.1809}}@{}}
%% \toprule\noalign{}
%% \multirow{2}{=}{\begin{minipage}[b]{\linewidth}\raggedright
%% {\bfseries Қосылыс}
%% \end{minipage}} &
%% \multirow{2}{=}{\begin{minipage}[b]{\linewidth}\raggedright
%% {\bfseries Еріткіш}
%% \end{minipage}} &
%% \multicolumn{2}{>{\centering\arraybackslash}p{(\linewidth - 10\tabcolsep) * \real{0.3781} + 2\tabcolsep}}{%
%% \begin{minipage}[b]{\linewidth}\centering
%% {\bfseries Сіңіру}
%% \end{minipage}} &
%% \multicolumn{2}{>{\centering\arraybackslash}p{(\linewidth - 10\tabcolsep) * \real{0.3072} + 2\tabcolsep}@{}}{%
%% \begin{minipage}[b]{\linewidth}\centering
%% {\bfseries Флуоресценция \emph{\tsp{b}}}
%% \end{minipage}} \\
%% & & \begin{minipage}[b]{\linewidth}\centering
%% {\bfseries λabs, нм}
%% \end{minipage} & \begin{minipage}[b]{\linewidth}\centering
%% {\bfseries ɛ, М\tsp{-1}cm\tsp{-1}}
%% \end{minipage} & \begin{minipage}[b]{\linewidth}\centering
%% {\bfseries λem,нм}
%% \end{minipage} & \begin{minipage}[b]{\linewidth}\centering
%% {\bfseries Интенсив-тілік,у.е.}
%% \end{minipage} \\
%% \midrule\noalign{}
%% \endhead
%% \bottomrule\noalign{}
%% \endlastfoot
%% \multirow{3}{=}{{\bfseries 5}} & \multirow{3}{=}{Диметил-формамид} & 296 &
%% 86 665 & \multirow{3}{=}{468} & \multirow{3}{=}{1 101} \\
%% & & 341 & 32 002 \\
%% & & 428 & 43 486 \\
%% \multirow{3}{=}{{\bfseries 7}} & \multirow{3}{=}{Диметил-формамид} & 296 &
%% 86 667 & \multirow{3}{=}{530} & \multirow{3}{=}{321} \\
%% & & 315 & 87 886 \\
%% & & 472 & 36 253 \\
%% \end{longtable}

\emph{\tsp{a} Диметилформамид, бөлме температурасы ( с, 1 ×
10\tsp{-4} M);}

\emph{\tsp{b} λ\tsb{ex} (қозу толқынының ұзындығы)
370 нм {\bfseries 5} үшін и 450 нм {\bfseries 7} үшін}

\fig{c/image21}{}

{\bfseries 2-сурет.5 және 7 қосылыстарының УФ - көрінетін спектрлері}

Донорлық орынбасқан азулендердің 5 және 7 флуоресценция спектрлері
3-суретте көрсетілген.5 қосылысының спектрі (3-сурет) 370 нм толқын
ұзындығымен қоздырғанда 468 нм-де максимумға ие жаңа, қарқынды көрінетін
сәулеленуді көрсетеді (1-кесте). {[}12{]} 7 молекуласының спектрі де
(3-сурет, 1-кесте) 450 нм-де қоздырғанда 530 нм максимумымен кең
эмиссиялық жолағын байқатады. Көріп тұрғандай, 7 қосылысының эмиссия
максимумы 62 нм-ге қызыл аймаққа қарай ығысқан (1-кесте) {[}14,15{]}.

Осы қосылыстардың (стандартты әдістемеге сәйкес өлшенген; стандартты
материал ретінде этанолдағы 1,4-бис-(5-фенилоксазол-2-ил) бензол
ерітіндісі пайдаланылған {[}15{]}) флуоресценциялық кванттық шығымдары
(PLQY) сәйкесінше 50\% және 60\% құрайды.

Маңыздысы - донорлық орынбасқан азулендердің 5 және 7 көк және жасыл
аймақта қарқынды фотолю-минесценциялық сәулелену қабілеті бар, ал
бастапқы 1 қосылысы мұндай қасиетке ие емес. Нәтижесінде, азуленнің C2
және C6 орындарында электрондонорлық карбазолды және дифениланилинді
фрагменттерді енгізу 1 қосылысының көрінетін аймақтағы ерекше қарқынды
жұтылу (ε = 43 486 M⁻¹cm⁻¹ және ε = 36 253 M⁻¹cm⁻¹) және
флуоресценциялық сипаттамаларына (интенсивтілік 1 100 б.ө. және 320
б.ө.) алып келеді.

\fig{c/image22}{}

{\bfseries 3-сурет.5 және 7 қосылыстарының флуоресценция спектрлері}

Донорлы-орынбасқан 5 және 7 азулендердің электрондық құрылымын және
олардың құрылымдық дизайны мен оптикалық қасиеттері арасындағы
байланысты түсіну үшін DFT (B3LYP/6-31G*) есептеулері жүргізілді
(4-сурет) {[}14,15{]}. Есептеу нәтижелері 5 және 7 қосылыстарында ең
жоғарғы толтырылған молекулалық орбитальдардың (HOMO) азулен құрылымында
және карбазолды/дифениланилинді орынбасарларда біркелкі таралғанын
көрсетті. HOMO орбитальдарының мұндай таралуы карбазол 4 (немесе
дифениланилин 6) қосылыстарының ең жоғарғы толтырылған орбитальдары мен
бастапқы азулен 1-дің HOMO-1 орбиталі арасындағы өзара әрекеттесуден
туындауы мүмкін (Tsuchiya және т.б., 2023). Негізгі азуленнің HOMO
орбитальдарында 2 және 6 көміртек атомдары түйіндік жазықтықта
орналасқан, ал HOMO-1 орбитальдарында бұл атомдар түйіндік жазықтықта
болмай, айқын атомдық-орбитальдық коэффициенттерге ие (Dunlop және т.б.,
2023) {[}10{]}.

Сонымен қатар, DFT есептеулері көрсеткендей, 5 қосылысының HOMO деңгейі
--5,25 eV, ал 7 қосылысының HOMO деңгейі -4,74 eV, бұл олардың бастапқы
азулен 1-мен салыстырғанда сәйкесінше төмен және жоғары мәндер екенін
білдіреді. Сонымен бірге, осы қосылыстардың фронтальды орбитальдары
арасындағы энергия айырмашылықтары сәйкесінше 0,3 және 0,58 eV-ға дейін
азаяды (4-сурет).

Бұл құбылыс, көрінгендей, бастапқы 1 қосылысы мен оның 5 және 7
туындылары арасындағы молекулалық орбитальдар энергия деңгейлерінің
реттілігінің инверсиясымен түсіндіріледі. Нәтижесінде, азулендегі бұрын
тыйым салынған π-π* электрондық ауысу рұқсат етіледі, бұл 5 және 7
қосылыстары үшін көрінетін аймақта байқалатын сипаттық жұтылу және
эмиссия қасиеттеріне әкеледі (2 және 3-суреттер, 1-кесте) {[}10{]}.

\fig{c/image23}{}

{\bfseries 4-сурет.1 азуленмен салыстырғандағы 5 және 7 қосылыстарының
HOMO және LUMO}

{\bfseries молекулалық орбитальдарының таралуы}





{\bfseries 5-сурет.5 және 7 π-конъюгацияланған азулендерінің резонанстық
құрылымдарының сызбасы.}

{\bfseries π-электрондардың делокализациясы}

{\bfseries Қорытынды.} 5 және 7 жаңа азулен туындыларын синтездеу мен
зерттеу олардың бірегей оптикалық қасиеттерін анықтады. Азуленнің С2
және С6 орындарындағы донорлық топтардың (карбазол және дифениланилин)
қосылуы 400-600 нм диапазонында айқын жұтылу және сәулелену қасиеттерін
қамтамасыз етеді. DFT есебі бұл деректерді растап, электрондық
құрылымдағы өзгерістерді көрсетті, нәтижесінде бұрын тыйым салынған π→π*
электрондық өтулер рұқсат етіледі. Эксперименттік анықталған HOMO
энергия деңгейлері есептік мәндермен сәйкес келеді, бұл теориялық
талдауды қосымша дәлелдейді. Алынған қосылыстар құрылымдық өзгерістер
арқылы олардың қасиеттерін дәл баптау мүмкіндігіне байланысты
органикалық электроника мен фотоникада қолдануға зор әлеуетке ие.

\emph{{\bfseries Қаржыландыру.} Зерттеу жұмысы Қазақстан Республикасы Ғылым
және жоғары білім министрлігі Ғылым комитетінің бағдарламалық-мақсатты
қаржыландыру (BR21882309) аясында жүзеге асырылды.}

{\bfseries Әдебиеттер}

1. Freudenberg J., Jänsch D., Hinkel F.Immobilization Strategies for
Organic Semiconducting Conjugated Polymers//Chem.
Rev.-2018.-Vol.118(11)-P.5598-5689.

DOI 10.1021/acs.chemrev.8b00063.

2. Roy M., Walton J.H., Fettinger J.C., Balch A.L. Direct
Crystallization of Diamine Radical Cations: Carbon-Nitrogen Bond
Formation from the Reaction of Triphenylamine with TiCl4, TiBr4, or
SnCl4 versus Carbon-Carbon Bond Formation with SbCl5// Chem. Eur. J. -
2022. - Vol.28(17): e202104631. DOI 10.1002/chem.202104631.



3. Zhang J., Chen Z., Yang L., Pan F.F., Yu G.A., Yin, J., Liu S.H.
Elaborately Tuning Intramolecular Electron Transfer Through Varying
Oligoacene Linkers in the Bis(diarylamino) Systems// Sci. Rep. - 2016. -
Vol.6: 36310. DOI
\href{https://doi.org/10.1038/srep36310}{10.1038/srep36310}.

4. Wang X., Zhang Z., Song Y., Su Y., Wang X. Bis(phenothiazine)arene
diradicaloids: Isolation, characterization and crystal structures //
Chem. Commun. - 2015.- Vol.51(59). - P.11822-11825. DOI
10.1039/C5CC03573B.

5. Wu C., Djurovich P. I., Thompson M. E. Study of energy transfer and
triplet exciton diffusion in hole‐transporting host materials. // Adv.
Funct. Mater. - 2009. - Vol.19(19). - P.3157-3164. DOI
\href{https://doi.org/10.1002/adfm.200900357}{10.1002/adfm.200900357}.

6. Taniguchi R., Noto N., Tanaka S., Takahashi K., Sarkar S. K., Oyama
R., Abe M., Koike T., Akita M. Simple generation of various
α-monofluoroalkyl radicals by organic photoredox catalysis: modular
synthesis of β-monofluoroketones. // Chem. Commun. - 2021. - Vol.57(21).
- P.2609-2612. DOI
\href{https://doi.org/10.1039/D0CC08060H}{10.1039/D0CC08060H}.

7. Noto N., Koike T., Akita M. Visible-Light-Triggered
Monofluoromethylation of Alkenes by Strongly Reducing
1,4-Bis(diphenylamino)naphthalene Photoredox Catalysis. // ACS Catal. -
2019. -Vol.9(5). - P.4382--4387.
\href{https://doi.org/10.1021/acscatal.9b00473}{DOI
10.1021/acscatal.9b00473}.

8. Tsuchiya T., Katsuoka Y., Yoza K., Sato H., Mazaki Y.
Stereochemistry, Stereodynamics, and Redox and Complexation Behaviors of
2,2′-Diaryl-1,1′-Biazulenes // Chem.Plus Chem. - 2019. - Vol.84(11). -
P.1659-1667. DOI
\href{https://doi.org/10.1002/cplu.201900262}{10.1002/cplu.201900262}.

9. Xin H., Hou B., Gao X. Azulene-Based π-Functional Materials: Design,
Synthesis, and Applications.//Acc Chem.Res.-2021.-Vol.54(7).-
P.1737-1753. DOI
\href{https://doi.org/10.1021/acs.accounts.0c00893}{10.1021/acs.accounts.0c00893}.

10. Dunlop D., Ludvikova L., Banerjee A., Ottosson H., Slanina T.
Excited-State (Anti)Aromaticity Explains Why Azulene Disobeys Kasha's
Rule. // J. Am. Chem. Soc. - 2023. - Vol.145(39). - P.21569-21575. DOI
\href{https://doi.org/10.1021/jacs.3c07625}{10.1021/jacs.3c07625}.

11. Shevyakov S.V., Li H., Muthyala R., Asato A.E., Croney J.C., Jameson
D.M., Liu R.S. Orbital control of the color and excited state properties
of formylated and fluorinated derivatives of azulene. // J. Phys. Chem.
A. - 2003. - Vol.107(18). - P.3295-3299.
\href{https://doi.org/10.1021/jp021605f}{DOI 10.1021/jp021605f}.

12. Kurotobi K., Miyauchi M., Takakura K., Murafuji T., Sugihara Y.
Direct Introduction of a Boryl Substituent into the 2-Position of
Azulene: Application of the Miyaura and Smith Methods to Azulene. //
Eur. J. Org. Chem. - 2003.- Vol.2003(18). - P.3663-3665.
\href{https://doi.org/10.1002/ejoc.200300001}{DOI
10.1002/ejoc.200300001}.

13. Fujinaga M., Murafuji T., Kurotobi K., Sugihara Y. Polyborylation of
azulenes. // Tetrahedron. - 2009. - Vol.65(34). - P.7115-7121. DOI
\href{https://doi.org/10.1016/j.tet.2009.06.053}{10.1016/j.tet.2009.06.053}.

14. Narita M., Toshihiro Y., Saki F., Masayuki H. Synthesis of
2-Iodoazulenes by the Iododeboronation of Azulen-2-ylboronic Acid
Pinacol Esters with Copper(I) Iodide // J. Org.Chem. - 2018.- Vol.83(3).
- P.1298--1303. DOI 10.1021/acs.joc.7b02820.

15. Tsuchiya T., Hamano T., Inoue M., Nakamura T., Wakamiya A., Mazaki
Y. Intense absorption of azulene realized by molecular orbital
inversion. // Chem. Commun. - 2023. - Vol.59(71). - P.10604-10607.
\href{https://doi.org/10.1039/D3CC02311G}{DOI 10.1039/D3CC02311G}.

\emph{{\bfseries Авторлар туралы мәліметтер}}

Искандеров А. Н. - академик, Е.А. Бөкетов атындағы Қарағанды ұлттық
зерттеу университетінің аға оқытушысы, Қарағанды, Қазақстан, e-mail:
aby93@yandex.kz;

Мерхатұлы Н. - химия ғылымдарының докторы, академик, Е.А. Бөкетов
атындағы Қарағанды ұлттық зерттеу университетінің профессоры, Қарағанды,
Қазақстан, e-mail:
merhatuly@ya.ru;

Жокижанова С. К.- химия ғылымдарының кандидаты, С.Сейфуллин атындағы
Қазақ агротехникалық зерттеу университетінің қауымдастырылған
профессоры, Астана, Қазақстан, e-mail:
saltanat\_zh75@mail.ru;

Искандеров А. Н.- академик, Е.А. Бөкетов атындағы Қарағанды ұлттық
зерттеу университетінің ғылыми қызметкері, Қарағанды, Қазақстан, e-mail:
dr-amantay@ya.ru;

Омаров Х.Б. - ҚР ҰҒА академигі, техника ғылымдарының докторы,
Қ.Құлажанов атындағы Қазақ технология және бизнес университетінің
профессоры, Астана, Қазақстан, e-mail:
homarov@mail.ru;

A.O. Булумбаева - академик, Е. А. Бөкетов атындағы Қарағанды ұлттық
зерттеу университетінің магистранты, Қарағанды, Қазақстан, e-mail:
bulumbaevaa02@mail.ru.

\emph{{\bfseries Information about the authors}}

Iskanderov A. N. - senior lecturer, Karaganda National Research
university named after academician E. A. Buketov, Karaganda, Kazakhstan,
e-mail: aby93@yandex.kz;

Merkhatuly N.- doctor of chemical sciences, professor at Karaganda
National Research university named after academician E.A. Buketov,
Karaganda, Kazakhstan, e-mail:
merhatuly@ya.ru;

Zhokizhanova S. K. - candidate of chemical sciences, associate
professor, S. Seifullin Kazakh agrotechnical research university,
Astana, Kazakhstan, e-mail:
saltanat\_zh75@mail.ru;

Iskanderov A. N. -research associate, Karaganda National Research
university named after academician E. A. Buketov, Karaganda, Kazakhstan,
e-mail: dr-amantay@ya.ru;

Omarov Kh.B. - academician of the National Academy of Sciences of the
Republic of Kazakhstan, doctor of technical sciences, professor at the
Kazakh University of Technology and Business named after K. Kulazhanov,
Astana, Kazakhstan, e-mail:
homarov@mail.ru;

A.O. Bulumbaeva - master' s student of Karaganda National
Research university named after academician E. A. Buketov, Karaganda,
Kazakhstan, e-mail:
bulumbaevaa02@mail.ru.\
