\id{IRSTI 20.23.17}{}; 20.53.21

{\bfseries AUTOMATION OF THE PROCESSES BASED ON MACHINE LEARNING FOR
TELEMEDICINE INFORMATION SYSTEM}

{\bfseries \tsp{1,2}A.S.
Seitenov}{\bfseries \envelope ,
\tsp{1}T.K.
Zhukabayeva}

\emph{\tsp{1}L.N. Gumilyov Eurasian National University,
Astana, Kazakhstan,}

\emph{\tsp{2}Astana IT University, Astana, Kazakhstan}

\corrauthor{Corresponding author: altynbekss@gmail.com}

The integration of Machine Learning into Telemedicine Information
Systems has revolutionised remote healthcare delivery, optimising
patient monitoring, diagnosis, and treatment processes. However,
challenges persist in automating these systems, particularly in areas of
data interoperability, security, and scalability. The research explores
the role of ML-driven automation in overcoming described challenges,
analysing current trends, obstacles, and research gaps. The study
notices technologies~ predictive analytics and image recognition are
advancing telemedicine capabilities, while issues like fragmented data,
lack of standard protocols, and privacy concerns remain. It emphasises
the need for robust frameworks that integrate ML techniques to enhance
the reliability and efficiency of TIS, especially in
resource-constrained settings. By combining the latest findings, the
work demonstrates the importance of adopting complex automation
strategies enhanced via ML tools for developing future innovation in the
telemedicine field.

{\bfseries Keywords}: telemedicine, medical information system, machine
learning, healthcare information system, remote patient monitoring.

{\bfseries АВТОМАТИЗАЦИЯ ПРОЦЕССОВ НА ОСНОВЕ МАШИННОГО ОБУЧЕНИЯ ДЛЯ
ТЕЛЕМЕДИЦИНСКОЙ ИНФОРМАЦИОННОЙ СИСТЕМЫ}

{\bfseries \tsp{1, 2}А.С. Сейтенов\envelope ,
\tsp{1}Т.К. Жукабаева}

\emph{\tsp{1} Евразийский национальный университет имени
Л.Н. Гумилева, Астана, Казахстан,}

\emph{\tsp{2}Astana IT University, Астана, Казахстан,}

e-mail:
altynbekss@gmail.com

Интеграция методов машинного обучения в информационные системы
телемедицины произвела революцию в области дистанционного оказания
медицинской помощи, оптимизируя процессы мониторинга пациентов,
диагностики и лечения. Тем не менее, автоматизация таких систем
по-прежнему сталкивается с рядом проблем, особенно в сферах
совместимости данных, безопасности и масштабируемости. В исследовании
рассматривается роль автоматизации, основанной на машинном обучении, в
преодолении указанных вызовов, проводится анализ современных тенденций,
препятствий и существующих исследовательских пробелов. Отмечается, что
технологии прогнозной аналитики и распознавания изображений значительно
продвигают возможности телемедицины, однако сохраняются такие проблемы,
как фрагментированность данных, отсутствие единых стандартов и вопросы
конфиденциальности. Особое внимание уделяется необходимости создания
надежных архитектур, интегрирующих методы машинного обучения для
повышения надежности и эффективности информационных систем телемедицины,
особенно в условиях ограниченных ресурсов.На основе анализа последних
достижений работа подчеркивает важность внедрения комплексных стратегий
автоматизации, усиленных инструментами машинного обучения, для
стимулирования инноваций в области телемедицины.

{\bfseries Ключевые слова:} телемедицина, медицинская информационная
система, машинное обучение, система медицинской информации,
дистанционный мониторинг пациентов

{\bfseries ТЕЛЕМЕДИЦИНАЛЫҚ АҚПАРАТТЫҚ ЖҮЙЕ ҮШІН МАШИНАЛЫҚ ОҚЫТУҒА
НЕГІЗДЕЛГЕН ҮДЕРІСТЕРДІ АВТОМАТТАНДЫРУ}

{\bfseries \tsp{1, 2}А.С. Сейтенов\envelope ,
\tsp{1}Т.К. Жукабаева}

\emph{\tsp{1}Л.Н. Гумилев атындағы Еуразия ұлттық
университеті, Астана, Қазақстан,}

\emph{\tsp{2} Astana IT University, Астана, Қазақстан,}

e-mail:
altynbekss@gmail.com

Машиналық оқыту әдістерінің телемедицина ақпараттық жүйелеріне
интеграциялануы қашықтықтан медициналық көмек көрсетуді түбегейлі
өзгертті, бұл пациенттерді бақылау, диагноз қою және емдеу үдерістерін
оңтайландырды. Дегенмен, мұндай жүйелерді автоматтандыруда деректердің
үйлесімділігі, қауіпсіздік және ауқымдылық секілді салаларда әлі де
бірқатар қиындықтар бар. Зерттеу машиналық оқытуға негізделген
автоматтандырудың осы қиындықтарды шешудегі рөлін қарастырып, қазіргі
үрдістерді, тосқауылдарды және зерттеу алшақтықтарын талдайды. Болжамдық
талдау мен кескіндерді тану технологиялары телемедицинаның
мүмкіндіктерін кеңейтуде, бірақ деректердің шашыраңқылығы, бірыңғай
стандарттардың болмауы және дербес деректердің құпиялылығы сияқты
мәселелер өзекті болып отыр. Зерттеуде әсіресе ресурс шектеулі
жағдайларда телемедицина ақпараттық жүйелерінің сенімділігі мен
тиімділігін арттыру үшін машиналық оқыту әдістерін біріктіретін мықты
архитектуралардың қажеттілігі атап өтіледі. Соңғы жетістіктерді талдай
отырып, бұл жұмыс телемедицина саласындағы болашақ инновацияларды дамыту
үшін машиналық оқыту құралдарымен күшейтілген кешенді автоматтандыру
стратегияларын енгізудің маңыздылығын көрсетеді.

{\bfseries Түйін сөздер}: телемедицина, медициналық ақпараттық жүйе,
машиналық оқыту, денсаулық сақтау ақпараттық жүйесі, болжамдық
аналитика, қашықтан пациентті бақылау.

{\bfseries Introduction.} Over the last decades, telemedicine
information systems (TIS) have noticeably transformed the healthcare
field via the deployment of remote doctors' consultation, medical
diagnosis, and patient management services {[}1{]}. Integration of
machine learning (ML) technologies presents further opportunities to
enforce~ TIS systems through predictive analytics, natural language
processing (NLP), and patient image recognition. These features can
enhance the telemedicine usage performance efficiency and adaptability
to user needs {[}2,3{]}. However, critical barriers remain in automating
core processes of TIS, especially in ensuring seamless interoperability,
robust data security, and adaptability across different healthcare
settings {[}4,5{]}. These challenges become a crucial issue due to the
heterogeneous nature of healthcare data and the absence of standardised
data exchange protocols {[}6{]}.

The research work shows a review of ML
technologies'{} role in automation TIS system processes.
It consists of an empirical investigation based on healthcare
professionals'{} thoughts, analysed via mixed-method
approaches. The work adopts a broader perspective, synthesising insights
from diverse studies to identify trends and challenges in the
telemedicine domain. This study focuses on how machine learning-based
automation can be applied to address the primary challenges of the
telemedicine platform, with particular attention to developing secure,
scalable, and user-centred design strategies. Understanding of these
aspects is essential for effectively integrating ML features into TIS
systems and for informing future developments in the ehealth field.
Moreover, the integration of ML technologies with other innovations,
namely blockchain network, offers promising avenues for improving data
integrity and patient privacy in telemedicine applications {[}7-9{]}.
While prior works have examined individual ML and blockchain
applications, the research proposes an architecture that is able to
combine the noted technologies in a modular framework for low-resource
settings.

Additionally, the study proposes an enhanced system architecture by
ML tools, blockchain technology, and secure interoperability standards.
The results highlight major concerns, namely, data fragmentation, lack
of standardisation, and security risks, which the proposed model aims to
mitigate. The outcome of the article informs possible future design,
implementation, and policy development for telemedicine systems. What
differentiates this study from prior literature is the integration of
technical and user-centred perspectives through a mixed-methods study,
which supports the design of a new modular system architecture. Rather
than solely reviewing trends, this work formulates a forward-looking
implementation framework rooted in both literature and stakeholder
insights.

{\bfseries Materials and methods.} In this section, a set of articles
relevant to the research topic was systematically reviewed and analysed.
The primary sources for the literature search included academic
databases, namely, Google Scholar, Scopus, and ScienceDirect. The
inclusion criteria required that the publication has to reference
specific keywords such as telemedicine, medical information system,
healthcare remote technologies, medical information systems, and related
digital health infrastructures. Studies that did not directly engage
with these themes were excluded. Applying the inclusion criteria, the
research review aims to offer a comprehensive overview of the current
state of research, emphasising major advancements, emerging trends, and
persistent challenges in the field. This approach ensures a conceptual
foundation for understanding the broader implications of TIS in the
scope of the modern healthcare service.

Analysis of publications related to the TIS system shows diverse
significant outcomes depending on the context of implementation.
Firstly, the adoption of TIS platforms has been associated with
improvements in clinical decision-making, reducing medical errors and
optimising treatment plans {[}10,11{]}. As an illustraion, the
integration of Electronic Health Records (EHR) illustrated error
reduction in the medical process {[}12,13{]}. Additionally, the
implementation of Clinical Decision Support Systems (CDSS) within TIS
frameworks helps medical staff by offering evidence-based treatment
recommendations and computer diagnostic accuracy {[}14{]}.

Contrastingly, other works present that the effectiveness of using
TIS systems often faces foundational challenges. There are low levels of
infrastructure, limited internet connectivity, unqualified training for
healthcare workers, and financial constraints. These factors reflect the
scalability and sustainability of telemedicine integration {[}15{]}.
Moreover, a systematic review emphasised the absence of clear guidelines
and investment can increase the possibilities of successful TIS
integration {[}16{]}.~

Also, existing studies provide valuable advantages of integration of
TIS~ systems in the medical field, they often overlook non-clinical
applications, including public health monitoring and policy development.
Additionally, there is research addressing the socio-economic factors
influencing TIS adoption in resource-limited settings. It underlines the
necessity of deep studies in the telemedicine area {[}17{]}.

CDSS has demonstrated the potential to enhance patient outcomes by
reducing medical errors, particularly those related to drug interactions
and diagnostic inaccuracies {[}18,19{]}. Institutions have successfully
implemented CDSS to bolster patient safety measures {[}20{]}. This
indicates that while CDSS offers potential benefits, its success is
contingent on clinician engagement and the system' s
alignment with the workflow.

Various methods and technologies have been deployed in TIS for the
improvement of business processes. The innovative approaches in
telemedicine systems, mobile health (mHealth), and medical information
platforms have successfully improved many healthcare settings {[}21{]}.
These systems have helped in healthcare operations, reduced operational
costs, and increased the efficiency of medical service delivery.

Recent advances and integration of blockchain technology in
telemedicine have able to get higher security standards and improved
data-sharing capabilities. Other methods, namely, machine learning
algorithms could be used for predictive analytics in healthcare,
assisting in decision-making and early disease diagnosis
{[}4,5,8{]}.

CDSS can be categorised into knowledge-based and non-knowledge-based
systems. Knowledge-based systems rely on rules and guidelines derived
from medical literature, while non-knowledge-based systems use machine
learning and statistical pattern recognition. Most of
today's CDSS are knowledge-based, utilising if-then rules to trigger
alerts for issues such as duplicate test orders or drug interactions.
However, advancements in artificial intelligence have led to more
sophisticated CDSS, which can learn from past data and provide
supportive clinical support {[}1{]}. While the integration
of these IT solutions gives a chance of successful usage, the whole
concerns about transparency and trust have limited their adoption
{[}20-22{]}.

\def\labelenumi{\Alph{enumi}.}
\setcounter{enumi}{1}

1. \emph{Identifying} Research Gaps

Several issues remain in current TIS research. The one of the
significant areas is the under-researching of TIS adaptability in
low-resource settings. Although TIS has been established sufficiently in
healthcare systems, there is still incomplete research on how
telemedicine system can be fully optimized in developing territories
{[}21{]}. While studies have highlighted the importance of management
skills within TIS, there is limited research on the integration of
artificial intelligence tools in the telemedicine ecosystem for
real-time knowledge processing.

Despite the proven benefits of CDSS, research presents application
limitations in some critical areas. One of them is the lack of
large-scale, systematic studies examining CDSS effectiveness in
real-world settings outside of controlled environments excluding
academic medical centres. Additionally, while much attention has been
given to CDSS for healthcare management, less research has focused on
their use in diagnostic decision-making, particularly in general
practice {[}19,22{]}.

The current research is able to address these gaps by focusing on
how TIS can be better tailored for resource-constrained environments and
identifying the best practices for integrating cutting-edge technologies
with existing systems or by focusing on the real-world implementation of
CDSS in diverse healthcare environments, examining factors such as user
interaction and integration with existing workflows.

The connection diagram illustrates how a set of research papers
addresses the core issues of interoperability and data security within
the TIS framework. It shows how studies join/differ from one another in
their approaches to solving the challenges {[}23-25{]}.

- Concentration on challenges related to interoperability and
heterogeneous data formats.

- Exploration of the influence on healthcare expenses and patient
care.

- Discussion of the benefit of AI and blockchain for providing
security aspects.

The figure (Figure 1) presents pointers between the reports to
spotlight the advancement of ideas over the term. It shows where the
last studies build on older findings.

\fig{i/image7}{}

{\bfseries Fig.1 - Graphical representation of source}

The above diagram presents how interoperability barriers have
evolved from early discussions of data formats and communication
protocols {[}25{]}. They lead to modern solutions involving AI and
blockchain {[}21{]}. Each study concerns previous findings, gradually
introducing more advanced technological keys to solve essence
difficulties in Medical Information Systems

Table 1 is a comparative table with specific columns. The content of
the table consists of methods, advantages, and limitations of different
studies on TIS. It helps summarise how different authors approached
common challenges and what results they reported.~

{\bfseries Table 1 - Description for TIS approaches}

%% \begin{longtable}[]{@{}
%%   >{\raggedright\arraybackslash}p{(\linewidth - 6\tabcolsep) * \real{0.1967}}
%%   >{\raggedright\arraybackslash}p{(\linewidth - 6\tabcolsep) * \real{0.1331}}
%%   >{\raggedright\arraybackslash}p{(\linewidth - 6\tabcolsep) * \real{0.2854}}
%%   >{\raggedright\arraybackslash}p{(\linewidth - 6\tabcolsep) * \real{0.3848}}@{}}
%% \toprule\noalign{}
%% \endhead
%% \bottomrule\noalign{}
%% \endlastfoot
%% {\bfseries Study} & {\bfseries Year} & {\bfseries Method Used} &
%% {\bfseries Advantages} \\
%% Häyrinen et al.{[}24{]} & 2008 & Data format
%% standardisation & Data format standardisation \\
%% DesRoches et al.{[}23{]} & 2008 & TIS implementation &
%% Reduced patient wait times \\
%% Epizitone et al.{[}21{]} & 2023 & AI and blockchain
%% integration & Enhanced security, automated decision-making \\
%% Liu et al. {[}5{]} & 2020 & Blockchain for security &
%% Strong encryption and distributed access control \\
%% Ahmad et al. {[}7{]} & 2021 & Blockchain in telehealth &
%% Tamper-proof logging and improved patient privacy \\
%% \end{longtable}

Graph represents a progression of research publications on TIS
syste, implementation and AI in Healthcare from 2000 to 2020 (Figure 2).
The blue line represents the increasing number of publications focused
on TIS implementation. The green line represents the rise in
publications related to AI tools in Healthcare.

\fig{i/image8}{}

\begin{quote}
{\bfseries Fig.2 - Progression of research publications on TIS
implementation}
\end{quote}

The review set highlights that EHR-s are not only repositories of
structured clinical and administrative information but also function as
tools for decision support, communication, and quality improvement. The
authors emphasise that while EHR implementation has the potential to
enhance healthcare efficiency and patient safety, success depends
heavily on standardization, user training, and system integrations.

Figure presents (Figure 3) a bar chart displaying the primary
challenges associated with the realisation of TIS technologies. It was
identified across multiple studies. The most frequently reported issues
are the following:

- interoperability. A notable amount of healthcare institutions
emphasise difficulties in blending heterogeneous systems and data
formats;

- data security. It emerges as another vital consideration with more
than half of the surveyed institutions expressing suspicions about
safeguarding patient information;

- high implementation costs. It poses a considerable barrier,
particularly in low-resource settings, where budget regulations limit
the scalability of digital health solutions.

\fig{i/image9}{}

\begin{quote}
{\bfseries Fig.3 - Key challenges in TIS implementation}
\end{quote}

The visual representation actually underlines these key challenges,
strengthening prevalence and impacting the existing literature.The
charts and visual materials in this paper are based on reliable sources
and real-world studies related to EHR and TIS technologies. These
visuals help explain the main challenges and trends in telemedicine in a
clear and accessible way.

- Interoperability: The data demonstrates a different researcher
attempts for solving the intercommunication and data exchange among
different programes. There are even a some progress has been made. A
number healthcare systems still struggle to share data smoothly. The
evidence from studies merges and diverges in work approaches to
embarking on the complex problem.

- Data Security: Information breakings and the use of blockchain
make it clear why protecting patient information is so important. As
more additional, healthcare systems go digital, consequently, it is
necessary to develop more reliable and secure tools. These visuals
highlight how new technologies are helping to keep patient data safe.

- System Adoption: The review presents how TIS and CDSS are used to
reveal success relies on the local context. Factors, namely, staff
training, technical readiness, and budget all play a role. These
insights show why it' s important to adapt systems to
each specific environment.

{\bfseries Results and discussion.} The study uses mixed (qualitative
and quantitative) research methods. The methods demonstrate a
description of the usage of TIS applications and challenges in TIS
implementation. Approaches are made to examine both clear and measurable
patterns. Additionally, it is modest and experience-based insights. The
work' s main focus area includes a user' s
satisfaction with TIS deploying. Particularly, it touches on
assessments, namely, data protection, system integrity in daily
routines, and efficiency of usage. Participants are chosen randomly from
healthcare institutions. The medical centres have already used medical
information systems with any telemedicine functionalities. The work is
aimed to gather feedback from a variety of medical professionals from
different departments, and levels of experience. It helps to capture a
full picture of the analysis and avoid biased insights. The result has
different viewpoints included in the analysis.

The analysis focuses on three main aspects of research:

- User Satisfaction. Measurement of using a scale (from 1 to 5), and
through yes/no responses.

- Data Security. Evaluation of users'{} safety
feelings about patient' s data is.

- Workflow Integration. Assessment of the usability of TIS system
fitted into daily work processes.

Responses are gathered via an online survey using Google Forms. The
survey was accomplished remotely and participants were able to fill it
when it was most convenient for them. It was made to diminish of
interrupting participant' s work. The questions covered
key areas, namely, a system usability, security consideration, and TIS
integration in their institutions.

Each questionnaire question was carefully developed to match the
objectives of the study. It used rankings scores and open-ended
questions to collect a broad range of data. Some questions are requested
by parties to rate TIS usage via their satisfaction. Others are asked
their thoughts about data security.

This metric captures the perceived efficiency and usability of the
TIS platforms and was analyzed alongside data security and workflow
integration variables to identify correlations. Limitations of the
study, such as potential biases in questionnaire analysis, were noted
and reduced its risks. The assessment steps were taken to mitigate
through anonymising responses and conducting the survey remotely.

The research work meets ethical requirements. It was obtained by
informing all participants as well as their private information was
secured. The participants were informed about the purpose of the
research, confidentiality, and the anonymity of their responses.

Data sharing is one of the primary keys of the study. It does not
occur seamlessly across systems, which the majority of organisations
mentioned in the survey. The proposed architecture can be framed in an
integration engine and API gateway for standardisation of the
communication channels between the internal modules (TIS, CDSS,
Analytics) and external systems (laboratory, pharmacy platforms).
Elimination of data fragmentation helps ensure continuous patient care
by removing repeated activities in the medical workflow process. Users
have expressed in a survey about dissatisfaction with fragmented
workflows in domestic medical information systems. It means that
ensuring the data sharing process in real-time between all parties TIS
system might improve the overall efficiency of the system work.

According to the findings, 72\% of users determined data breaches
and unauthorised access as the most crucial threats in existing systems.
To address this, the proposed architecture includes a multi-layered
security approach using OAuth 2.0, encryption, anomaly detection, and
audit logs to protect sensitive patient data from login to storage.

Additionally, integrating a blockchain ledger will strengthen the
system by improving transparency and keeping with MIS standards. The
current security measures directly align with the survey findings. The
users will feel confident in system security, resulting in higher levels
of satisfaction.

The research indicates one of the main obstacles for deploying
telemedicine systems is incapacity for scaling, especially in
low-resource environments. To overcome the current issue, the proposed
infrastructure should contain load balancer components and auto-scaling
tools. A such technologies allow the system to automatically adapt to
changing workloads while maintaining strong performance.

This is especially important during periods of high demand,
especially in health emergency situations. In this case, the system must
remain available and responsive. This approach directly addresses a
common weakness found in many existing systems, as noted in previous
studies.

The survey outcomes reveal current systems are commonly adequate,
but many users still face difficulties due to poor levels of usability
and unproductive workflows. For eliminatation this case, the proposed
model should possess interactive dashboards for different user roles.
Particularly, it could be a patient portal, a clinical dashboard and an
administrative dashboard. This role-based approach ensures a more
personalized and user-friendly experience for each user type.
Furthermore, the dashboards is able to allow users do rapid data-driven
decisions.~

The findings display that poor and fragmented data management
remains a major issue. One of the possible ways for improvement is
designing a specific data layer. The layer can be proposed for handling
different types of medical data in a well-organised and efficient way.
For instance:

- database will store structured data such as patient records;

- data Warehouses will gather long-term information for analytics
and reporting;

- file Storage will manage unstructured data (e.g. medical images
with a variety of resolutions).

Combining all the above aspects into a unified system, healthcare
providers will gain quick and reliable access to medical information.
This will help reduce delays and support more accurate, personalised,
and timely patient care.

{\bfseries Conclusion.} The findings of this study confirm that
Telemedicine Information Systems (TIS) play an important role in
improving healthcare processes and outcomes. Nevertheless, key
challenges continue to restrict its full potential scaling. Quantitative
data revealed that the majority of users identified interoperability
issues, while others expressed to anxieties over data security. The
results were sustained by qualitative feedback demonstrating that user
satisfaction with existing systems remains moderate level. Due to there
were usability limitations and security shortcomings.

For overcoming the issue, the study presents an improved system
architecture description that highlights data exchange through
integration engines and API gateways. They can have a capability for
reducing workflow fragmentation. It also introduces robust security
mechanisms, including OAuth 2.0, encryption, blockchain integration, and
anomaly detection to sustain user trust and ensure data protection
principles. Furthermore, the architecture presented mechanisms for
scalable applications during periods of high demand.

Through implementation-specific dashboards, administrators, and
patients are able to reveal the significant distinguish in usability and
align with the diverse their needs. Lastly, the architecture could adopt
a structured data management model that unifies databases, data
warehouses, and file storage for developing clinical decision-making and
long-term analytics.

In contrast to traditional literature reviews, the article
contributes new approaches for system architecture that synthesise
cutting-edge ML and blockchain technologies with empirical survey
findings. This mixed method maintains the practical relevance of the
design for ensuring alignment among technical requirements and user
satisfaction.

The proposed architecture not only mitigates the well-documented
challenges in TIS implementation but also introduces an innovative,
integrative model combining ML, blockchain, and modular design---unlike
current fragmented solutions. Planned validation and comparative testing
will further substantiate its real-world applicability, paving the way
for future adoption in scalable digital healthcare deliver services.

{\bfseries References}

1. Di Cerbo A., Morales-Medina J.C., Palmieri B., Iannitti T. Narrative
review of telemedicine consultation in medical practice // Patient
Preference and Adherence. -2015. -Vol.9. -P.65--75. DOI
10.2147/PPA.S61617.

2. Jiang F., Jiang Y., Zhi H., Dong Y., Li H., Ma S., Wang Y. Artificial
intelligence in healthcare: Past, present and future // Stroke and
Vascular Neurology. -2017. -Vol.2(4). -P.230--243. DOI
10.1136/svn-2017-000101.

3. Albahri O.S., Zaidan A.A., Albahri A.S., Zaidan B.B., Abdulkareem
K.H., Al-Qaysi Z.T., Rashid N.A. Systematic review of artificial
intelligence techniques in the detection and classification of COVID-19
medical images // Journal of Infection and Public Health. -2020. -Vol.
13(10). -P.1381-1396. DOI
\href{https://doi.org/10.1016/j.jiph.2020.06.028}{10.1016/j.jiph.2020.06.028}.

4. Adeghe E.P., Okolo C.A., Ojeyinka O.T. Evaluating the impact of
blockchain technology in healthcare data management: A review of
security, privacy, and patient outcomes // Open Access Research Journal
of Science and Technology. -2024. -Vol.10(2). -P.013--020. DOI
\href{https://doi.org/10.53022/oarjst.2024.10.2.0044}{10.53022/oarjst.2024.10.2.0044}.

5. Liu H., Crespo R.G., Martínez O.S. Enhancing privacy and data
security across healthcare applications using blockchain and distributed
ledger concepts // Healthcare. -2020. -Vol.8(3): 243. DOI
10.3390/healthcare8030243.

6. Ferreira J.C., Elvas L.B., Correia R., Mascarenhas M. Enhancing EHR
interoperability and security through distributed ledger technology: A
review // Healthcare. -2024. -Vol.12(19): 1967. DOI
10.3390/healthcare12191967.

7. Ahmad R.W., Salah K., Jayaraman R., Yaqoob I., Ellahham S., Omar M.
The role of blockchain technology in telehealth and telemedicine //
International Journal of Medical Informatics. -2021. -Vol.148:104399.
DOI 10.1016/j.ijmedinf.2021.104399.

8. Shynar Y., Seitenov A., Kenzhegarina A., Kenzhetayev A., Kemel A.,
Ualiyev N., Sakhipov A., Myrzakerimova A., Mursakimova G., Orynbek A.
Comprehensive analysis of blockchain technology in the healthcare sector
and its security implications // International Journal of E-Health and
Medical Communications. -2025. - Vol.16(1). -- P.1--45. DOI
10.4018/IJEHMC.372423.

9. Seitenov A., Smagulova G. Distribution of Ethereum blockchain
addresses // Scientific Journal of Astana IT University. -2020. -Vol.4
(4). -P.41-48. DOI 10.37943/aitu.2020.36.57.005.

10. Steinman M., Morbeck R.A., Pires P.V., Abreu Filho C.A.C., Andrade
A.H.V., Terra J.C.C., Kanamura A.H. Impact of telemedicine in hospital
culture and its consequences on quality of care and safety // Einstein
(São Paulo). -2015. -Vol.13(4). -P.580--586. DOI
10.1590/S1679-45082015GS2893.

11. Gajarawala S.N., Pelkowski J.N. Telehealth benefits and barriers //
The Journal for Nurse Practitioners. -2021. -Vol.17(2). -P.218-221.
DOI 10.1016/j.nurpra.2020.09.013.

12. Adeyemi C., Adegoke B.O., Odugbose T. The impact of healthcare
information technology on reducing medication errors // International
Journal of Frontiers in Medical and Surgical Research. -2024. -Vol.
5(2). -P.20--29. DOI 10.53294/ijfmsr.2024.5.2.0034.

13. Seitenov A., Zhukabayeva T., Sansyzbay K., Kalpakov Y. Design
development of medicine information system for telemedicine field // The
Bulletin of KazATC. -2023. -4 (127). - P.241-251. DOI
10.52167/1609-1817-2023-127-4-241-251.

14. Mars M., Scott R. Telemedicine service use: A new metric // Journal
of Medical Internet Research. -2012. -Vol.14(6): e178. DOI
10.2196/jmir.1938.

15. Dorsey E.R., Topol E.J. State of telehealth // The New England
Journal of Medicine. -2016. -Vol.375(2). -P.154-161. DOI
10.1056/NEJMra160170.

16. Kruse C.S., Karem P., Shifflett K., Vegi L., Ravi K., Brooks M.
Evaluating barriers to adopting telemedicine worldwide: A systematic
review // Journal of Telemedicine and Telecare. - 2018. - Vol.24(1). -
P.4-12. DOI 10.1177/1357633X16674087.

17. Whitten P., Holtz B., Laplante C. Telemedicine What have we learned?
// Applied Clinical Informatics. - 2010. -- Vol.1(2). -P.132--141. DOI
10.4338/ACI-2009-12-R-0020.

18. Sutton R.T., Pincock D., Baumgart D.C., Sadowski D.C., Fedorak R.N.,
Kroeker K.I. An overview of clinical decision support systems: benefits,
risks, and strategies for success // NPJ Digital Medicine. -2020. -Vol.
3:17. DOI 10.1038/s41746-020-0221-y.

19. Kaushal R., Shojania K.G., Bates D.W. Effects of computerized
physician order entry and clinical decision support systems on
medication safety: A systematic review // Archives of Internal Medicine.
-2003. -Vol.163(12). -P.1409-1416. DOI 10.1001/archinte.163.12.1409.

20. Osheroff J.A., Teich J., Levick D., Saldana L., Velasco F., Sittig
D., Rogers K., Jenders R. Improving Outcomes with Clinical Decision
Support: An Implementer's Guide, Second Edition/Chicago: HIMSS
Publishing. -2012. -48p. ISBN 978-0984457731; 978-1000396416.

21. Epizitone A., Moyane S.P., Agbehadji I.E. A systematic literature
review of health information systems for healthcare // Healthcare.
-2023. -Vol.11(7): 959. DOI 10.3390/healthcare11070959.

22. Berner E.S., La Lande T.J. Overview of clinical decision support
systems //Clinical Decision Support Systems. -2016. -P.1-17. DOI
10.1007/978-3-319-31913-1\_1.

23. DesRoches C.M., Campbell E.G., Rao S.R., et al. Electronic health
records in ambulatory care -A national survey of physicians // The New
England Journal of Medicine. -2008. - Vol.359(1). -P.50--60. DOI
10.1056/NEJMsa0802005.

24. Häyrinen K., Saranto K., Nykänen P. Definition, structure, content,
use and impacts of electronic health records: A review of the research
literature // International Journal of Medical Informatics. -2008. -Vol.
77(5). - P.291-304. DOI 10.1016/j.ijmedinf.2007.09.001.

25. Becker S.H., Arenson R.L. Costs and benefits of picture archiving
and communication systems // Journal of the American Medical Informatics
Association. - 1994. -Vol.1(5). -P.361-371. DOI
~\href{https://doi.org/10.1136/jamia.1994.95153424}{10.1136/jamia.1994.95153424}

\emph{{\bfseries Information about the authors}}

Seitenov A.S. - PhD student, L.N. Gumilyov Eurasian National
University, Astana, Kazakhstan, e-mail: altynbekss@gmail.com;

Zhukabayeva T.K.- PhD, Professor, L.N. Gumilyov Eurasian National
University, Astana, Kazakhstan, e-mail:
tamara.kokenovna@gmail.com.

\emph{{\bfseries Сведения об авторах}}

Сейтенов А.С. - докторант, Евразийский национальный университет им.
Л.Н.Гумилёва, Астана, Казахстан, e-mail: altynbekss@gmail.com;

Жукабаева Т.К. - PhD, профессор, Евразийский национальный
университет им. Л.Н.Гумилёва, Астана, Казахстан, e-mail:
tamara.kokenovna@gmail.com.\
