\id{МРНТИ 55.30.31}{}

\begin{header}
\swa{}{УСТОЙЧИВОСТЬ ДИНАМИКИ БЕСПИЛОТНОГО НАДВОДНОГО СУДНА}

\tsp{1}И.С. Тажибай,
\tsp{1,2}Т.Ж. Мазаков,
\tsp{2}Ш.А. Джомартова,
\tsp{2}Ә.Т. Мазакова,
\tsp{3}А.Д. Майлыбаева,
\tsp{1}М.С.~Алиаскар\envelope,
\tsp{4}Б.М. Мазакова
\end{header}

\begin{affil}
\tsp{1}Международный инженерно-технологический университет, Алматы, Казахстан,

\tsp{2}Казахский национальный университет имени аль-Фараби, Алматы, Казахстан,

\tsp{3}Атырауский университет им. Х. Досмухамедова, Атырау, Казахстан,

\tsp{4}Международный университет, Астана, Казахстан

\corrauthor{Корреспондент-автор: m.alyasqar@gmail.ru}
\end{affil}

Современное развитие автономных морских робототехнических систем
обусловлено глобальными задачами в области экологии, охраны акваторий,
морской безопасности, навигации и ресурсного мониторинга. БНС активно
внедряются в такие сферы, как картографирование, мониторинг загрязнений,
контроль территориальных вод, научные исследования, обслуживание морской
инфраструктуры и поиск-спасение.

При проектировании автономных судов критически важным является наличие
адекватной математической модели, обеспечивающей предсказуемость
поведения платформы. Без такой модели невозможно гарантировать
безопасную эксплуатацию, корректную стабилизацию курса и эффективную
адаптацию к морским возмущениям.

Таким образом, развитие методов моделирования динамики БНС представляет
собой актуальную научно-техническую задачу.

В работе рассматривается устойчивость динамики беспилотного надводного
судна. Показана как на основе линеаризации возможен вывод моделей Номота
первого и второго порядка.

Для линеаризованной системы сформулирован критерий интервальной
устойчивости, который учитывает неточность задания параметров
беспилотного надводного аппарата.

{\bfseries Ключевые слова:} беспилотное надводное судно (БНС), кинематика,
линеаризация, математическая модель, модель Номото, управление,
устойчивость.

\begin{header}
ПІСІРІМСІЗ БЕТКІ КЕМЕНІҢ ДИНАМИКАСЫНЫҢ ТҰРАҚТЫЛЫҒЫ

\tsp{1}И.С. Тәжібай,
\tsp{1,2}Т.Ж. Мазаков,
\tsp{2}Ш.А. Джомартова,
\tsp{2}Ә.Т. Мазақова,
\tsp{3}А.Д. Майлыбаева,
\tsp{1}М.С.~Әлиасқар\envelope,
\tsp{4}Б.М. Мазакова
\end{header}

\begin{affil}
\tsp{1}Халықаралық инженерлік-технологиялық университет, Алматы, Қазақстан,

\tsp{2}Әл-Фараби атындағы Қазақ Ұлттық Университеті, Алматы, Қазақстан,

\tsp{3}Kh. Досмұхамедов атындағы Атырау университеті, Атырау, Қазақстан,

\tsp{3}Астана халықаралық университеті, Астана, Қазақстан,

e-mail: m.alyasqar@gmail.ru
\end{affil}

Автономды теңіз роботтық жүйелерінің қазіргі дамуы экология, суды
қорғау, теңіз қауіпсіздігі, навигация және ресурстарды бақылау
саласындағы жаһандық қиындықтармен байланысты. Ұш\-қышсыз ұшу аппараттары
картаға түсіру, ластануды бақылау, аумақтық суды бақылау, ғылыми
зерттеулер, теңіз инфрақұрылымын күтіп ұстау және іздеу-құтқару сияқты
салаларда белсенді түрде енгізілуде.

Автономды кемелерді жобалау кезінде платформаның мінез-құлқының
болжамдылығын қамтамасыз ететін тиісті математикалық модельдің болуы өте
маңызды. Мұндай модельсіз қауіпсіз жұмысты, бағытты дұрыс тұрақтандыруды
және теңіздегі толқуларға тиімді бейімделуді қамтамасыз ету мүмкін емес.

Сондықтан, Ұшқышсыз ұшу аппараттарының динамикасын модельдеу әдістерін
әзірлеу өзекті ғылыми және техникалық міндет болып табылады.

Бұл мақалада ұшқышсыз жер үсті кемесінің динамикалық тұрақтылығы
қарастырылады. Онда бірінші және екінші ретті Nomota модельдерін
сызықтықтандыруды қолдану арқылы қалай алуға болатыны көрсетілген.

Сызықтық жүйе үшін ұшқышсыз жер үсті көлігінің параметрлеріндегі
белгісіздікті ескеретін аралық тұрақтылық критерийі тұжырымдалған.

{\bfseries Түйін сөздер:} ұшқышсыз су үсті кемесі (ҰСК), кинематика,
сызықтық өңдеу, математикалық модель, Номото моделі, басқару,
тұрақтылық.

\begin{header}
STABILITY OF THE DYNAMICS OF AN UNMANNED SURFACE VESSEL

\tsp{1}I. Tazhibay,
\tsp{1,2}T. Mazakov,
\tsp{2}Sh. Dzhomartova,
\tsp{2}A. Mazakova,
\tsp{3}A. Mailybaeva,
\tsp{1}M. Aliaskar\envelope,
\tsp{4}B.~Mazakova
\end{header}

\begin{affil}
\tsp{1}International Engineering Technological University, Almaty, Kazakhstan,

\tsp{2}al-Farabi Kazakh National University, Almaty, Kazakhstan,

\tsp{3}Kh. Dosmukhamedov Atyrau University, Atyrau, Kazakhstan,

\tsp{4}Astana International University, Astana, Kazakhstan,

e-mail: m.alyasqar@gmail.ru
\end{affil}

The current development of autonomous marine robotic systems is driven
by global challenges in ecology, water protection, maritime security,
navigation, and resource monitoring. UAVs are actively being implemented
in areas such as mapping, pollution monitoring, territorial water
control, scientific research, marine infrastructure maintenance, and
search and rescue.

When designing autonomous vessels, the availability of an adequate
mathematical model that ensures the predictability of the
platform' s behavior is critical. Without such a model,
it is impossible to guarantee safe operation, correct course
stabilization, and effective adaptation to sea disturbances.

Therefore, the development of methods for modeling UAV dynamics is a
pressing scientific and technical challenge.

This paper examines the dynamic stability of an unmanned surface vessel.
It demonstrates how first- and second-order Nomota models can be derived
using linearization.

For a linearized system, an interval stability criterion is formulated
that takes into account the uncertainty in the parameters of the
unmanned surface vehicle.

{\bfseries Keywords:} unmanned surface vessel (USV), kinematics,
linearization, mathematical model, Nomoto model, control, stability.

\begin{multicols}{2}
{\bfseries Введение.} Математическое моделирование движения судов является
классической задачей морской гидродинамики, однако переход к беспилотным
платформам предъявляет новые требования. Автономные системы требуют
алгоритмов реального времени, что делает необходимым использование
упрощённых, но достаточно точных моделей {[}1{]}.

Полная нелинейная модель движения судна в координатах является точной,
но сложной для непосредственного применения в системах управления.
Поэтому в инженерной практике широко применяют упрощенные линейные
модели, Модель Номото позволяет эффективно проектировать регуляторы
курса. Одной из центральных научно-технических проблем при
проектировании подобных систем остаётся разработка адекватной
математической модели динамики движения, обеспечивающей
корректный синтез алгоритмов управления, оценку устойчивости курса,
оптимизацию траекторий и адаптацию к внешним возмущениям (ветер, волны,
течение). Без строгого динамического описания невозможно обеспечить
высокую точность позиционирования, надёжность автономной навигации и
безопасность работы в сложных условиях акватории {[}2-3{]}.

Целью данной статьи является разработка, анализ и упрощение
математической модели динамики беспилотного надводного судна, а также
вывод линейной курсовой модели, пригодной для задач стабилизации и
автономной навигации.

{\bfseries Материалы и методы.} В статье рассматривается математическая
модель динамики беспилотного надводного судна, которая описывается
системой обыкновенных дифференциальных уравнений.

Введем следующие обозначения:
\end{multicols}

\begin{equation}
x = \begin{pmatrix} x_{c} \\ y_{c} \\ \psi \\ u \\ \upsilon \\ r \end{pmatrix}, \quad u_{c} = \begin{pmatrix} X \\ Y \\ N \end{pmatrix}, \quad \eta = \begin{pmatrix} x_{c} \\ y_{c} \\ \psi \end{pmatrix}, \quad \nu = \begin{pmatrix} u \\ \upsilon \\ r \end{pmatrix},
\end{equation}

\begin{multicols}{2}
где

\(x_{c},y_{c}\). - координаты центра судна в инерциальной системе,

\(\psi\)- курсовое направление,

\(u,\upsilon\)- продольная и поперечная скорость судна,

\(r\)- угловая скорость рыскания,

\(X\) - суммарная продольная управляющая сила (тяга винтов, водомётов и
т.п.),

\(Y\) - суммарная поперечная управляющая сила (при наличии боковых
движителей),

\(N\) - управляющий момент по курсу {[}4, 5{]}.

Кинематическая связь между введенными выше параметрами удовлетворяет
следующей системе обыкновенных дифференциальных уравнений

\begin{equation}
\frac{d\eta}{\text{dt}} = J(\psi)\eta,
\end{equation}

где

\[J(\psi) = \begin{bmatrix}
\text{cos}(\psi) & - \text{sin}(\psi) & 0 \\
\text{sin}(\psi) & \text{cos}(\psi) & 0 \\
0 & 0 & 1
\end{bmatrix}\]

Динамическая связь удовлетворяет следующей системе обыкновенных
дифференциальных уравнений

\[M\frac{d\nu}{\text{dt}} + C(\nu)\nu + D(\nu)\nu = u_{c}\]

или

\begin{equation}
\frac{d\nu}{\text{dt}} = M^{- 1}\left( u_{c} - C(\nu)\nu - D(\nu)\nu \right),
\end{equation}

где \(М\) - 3*3- матрица, учитывающая массовые характеристики судна,

\(C(\nu)\)- 3*3- матрица, учитывающая влияние кориолисовых и
центробежных сил,

\(D(\nu)\)- 3*3- матрица, учитывающая влияние демпфирующих сил.

Перепишем систему уравнений (2)-(3) в виде следующей системы нелинейных
дифференциальных уравнений

\begin{equation}
\frac{\text{dx}}{\text{dt}} = f(x,u_{c})
\end{equation}

{\bfseries Обсуждение и результаты. \emph{Критерий интервальной
устойчивости}.} Для решения поставленной задачи введем следующие
обозначения:
\end{multicols}

\begin{equation}
n = 6, \quad m = 3, \quad x = (x_{1},\ldots,x_{n}), \quad x_{0} = (x_{1}(0),\ldots,x_{n}(0)), \quad c = \begin{pmatrix} 1 & 0 & 0 & 0 & 0 & 0 \\ 0 & 1 & 0 & 0 & 0 & 0 \end{pmatrix}
\end{equation}

Здесь \(x,\text{.}x_{0}\) - n-мерные векторы, c - 2*n-матрица.

\begin{multicols}{2}
Тогда добавим к системе уравнений (4) начальные условия

\begin{equation}
x(0) = x_{0}
\end{equation}

и конечные условия в момент времени Т

\begin{equation}
с^{}x(T) = z
\end{equation}

Линеаризуем правую часть системы уравнений (5) при следующих допущениях:
1) судно идёт прямолинейно с постоянной продольной скоростью
\(U_{0} > 0\), 2) малые отклонения курса и руля {[}6{]}.

Теперь исходная система (5) описывается линейными дифференциальными
уравнениями

\begin{equation}
\frac{\text{dx}}{\text{dt}} = \text{Ax} + \text{Bu}_{c}
\end{equation}

где \(А\) - n*n-матрица, B n*m-матрица.

Рассмотрим далее частные случаи системы (8).

{\bfseries \emph{1. Модель Номото 1-го порядка}}

Обозначим через

\[x = \begin{pmatrix} \psi \\ r \end{pmatrix}, u_{c} = \delta\]

\(T>0\) - эффективную постояннуя времени курсовой динамики,
\(K_{\delta}\) - усиление по рулю, \(\delta\)-управляющий
сигнал (угол поворота руля).

Тогда из линейной связи «руль \(\delta \) → угловая скорость по курсу \(r\)»:

\(T\frac{\text{dr}}{\text{dt}} + r = K_{\delta}\delta\) и соотношения
\(\frac{d\psi}{\text{dt}} = r\), получим модель Номота первого порядка в
виде (8):

\begin{equation}
\frac{\text{dx}}{\text{dt}} = \begin{pmatrix}
\frac{d\psi}{\text{dt}} \\
\frac{\text{dr}}{\text{dt}}
\end{pmatrix} = \begin{bmatrix}
0 & 1 \\
0 & - \frac{1}{T}
\end{bmatrix}\begin{bmatrix}
\psi \\
r
\end{bmatrix} + \begin{bmatrix}
0 \\
\frac{K_{\delta}}{T}
\end{bmatrix}\delta
\end{equation}

2. \emph{{\bfseries Модель Номото 2-го порядка}}

Для учета влияния поперечного движения и сложной гидродинамики часто
используют модель Номота 2-го порядка, которая выводится из следующих
уравнений:

\begin{equation}
T_{1}T_{2}\frac{d^{2}r}{\text{dr}^{2}} + (T_{1} + T_{2})\frac{\text{dr}}{\text{dt}} + r = K_{\delta}\delta, \frac{d\psi}{\text{dt}} = r
\end{equation}

где \(T_{1},T_{2} > 0\) - постоянные времени быстрого и медленного
режимов.

Обозначим через

\[x = \begin{pmatrix} \psi \\ r \\ p \end{pmatrix}, p = \frac{\text{dr}}{\text{dt}}u_{c} = \delta\]

Тогда из (10) получим модель Номота второго порядка:
\end{multicols}

\begin{equation}
\frac{\text{dx}}{\text{dt}} = \begin{pmatrix}
d\psi \\
\text{dr} \\
\text{dp}
\end{pmatrix} = \begin{bmatrix}
0 & 1 & 0 \\
0 & 0 & 1 \\
0 & - \frac{1}{T_{1}T} & - \frac{T_{1} + T}{T_{1}T}
\end{bmatrix}\begin{bmatrix}
\psi \\
r
\end{bmatrix} + \begin{bmatrix}
0 \\
\frac{K_{\delta}}{T_{1}T}
\end{bmatrix}\delta
\end{equation}

\begin{multicols}{2}
\emph{Примечание.} В моделях Номота (9) и (11) параметры
\(T,T_{1},T_{2},K_{\delta}\) выражаются через коэффициенты матриц M, C,
D системы уравнений (3).

Далее исследуется проблема устойчивости линеаризованной модели вида /8/:

\begin{equation}
\dot{x} = Ax
\end{equation}

где через коэффициенты матрицы \(А\) заданы параметры,
характеризующие механические параметры (такие как вес, метрические
характеристики, инерционность и т.п.). При этом предполагается, что
матрица \(А\) является интервальной, т.е. элементы ее
представляют собой интервальные числа.

Для системы (12) в случае, когда элементы матрицы \(А\)
являются «точечными» числами разработаны критерии устойчивости,
выражающиеся через элементы матрицы \(А\) -- критерий
Рауса-Гурвица и др. {[}7{]}. Однако, при этом не учитываются то
обстоятельство, что указанные физические параметры измеряются с
некоторой погрешностью. Исследователи часто, делая вывод об устойчивости
системы (12), забывают, что коэффициенты и корни характеристического
полинома могут быть очень чувствительны к малым погрешностям матричных
элементов.

Интервальный анализ дает возможность автоматически учитывать погрешности
в задании исходных данных и погрешности, вызываемые машинным
округлением.

Использование интервального анализа при решении задачи устойчивости
динамики механических систем позволяет получить критерий гарантированной
устойчивости.

Для определения устойчивости интервальной матрицы строится
характеристический полином с интервальными коэффициентами:
\end{multicols}
\vspace{-1em}
\begin{equation}
\phi_{A}(\lambda) = \text{det}(\lambda E - A) = p_{n}\lambda^{n} + p_{n - 1}\lambda^{n - 1} + \ldots + p_{0}
\end{equation}

\begin{multicols}{2}
где \(p_{i},i = \overline{0,n}\) - интервальные числа.

\emph{{\bfseries Определение.}} Интервальный характеристический полином
(13) называется устойчивым, если интервал, составленный из вещественных
частей интервальных собственных значений, не содержит 0 и находится
полностью в отрицательной области.

\emph{{\bfseries Необходимое условие устойчивости:}} все коэффициенты
характеристического полинома (13) должны находиться в положительной
области и не содержать 0: т.е. для

\[p_{i} = \lbrack\overline{p_{i}} - \varepsilon_{i}^{p},\overline{p_{i}} - \varepsilon_{i}^{p}\rbrack,i = \overline{0,n}\]

должно выполняться

\(0 \notin p_{i},\overline{p_{i}} - \varepsilon_{i}^{p} > 0,i = \overline{0,n}\)
{[}8{]}.

Составим матрицу Гурвица

\[M = \begin{bmatrix}
p_{1} & p_{0} & 0 & 0 & \text{.}\text{.}\text{.} & 0 \\
p_{3} & p_{2} & p_{1} & p_{0} & \text{.}\text{.}\text{.} & 0 \\
\text{.}\text{.}\text{.} & \text{.}\text{.}\text{.} & \text{.}\text{.}\text{.} & \text{.}\text{.}\text{.} & \text{.}\text{.}\text{.} & \text{.}\text{.}\text{.} \\
p_{2n - 1} & p_{2n - 2} & p_{2n - 3} & p_{2n - 4} & \text{.}\text{.}\text{.} & p_{n}
\end{bmatrix}\]

где принято \(p_{j} = 0\) при \(j < 0\) и \(j > n\).

Обозначим через \(\Delta_{1},\Delta_{2},\text{...},\Delta_{n}\)главные
диагональные миноры матрицы \(M\):

\[\begin{array}{r}
\Delta_{1} = p_{1}, \\
\Delta_{2} = \left| \begin{matrix}
p_{1} & p_{0} \\
p_{3} & p_{2}
\end{matrix} \right|, \\
\text{.}\text{.}\text{.}\text{.}\text{.}\text{.}\text{.}\text{.}\text{.} \\
\Delta_{n} = |M| = p_{n}\Delta_{n - 1},
\end{array}\]

которые в свою очередь являются интервальными числами.

\emph{{\bfseries Критерий интервальной устойчивости Гурвица:}} для того
чтобы \(\text{Re}\lambda_{j}(A) < 0,j = \overline{1,n}\) необходимо и
достаточно, чтобы главные диагональные миноры
\(\Delta_{1},\Delta_{2},\text{...},\Delta_{n}\) матрицы \(M\) находились
в правой полуплоскости, т.е.
\(\Delta_{j} \in (0,\infty),j = \overline{1,n}\).

\emph{{\bfseries Критерий интервальной устойчивости Льенара-Шипара:}} Если
все интервальные коэффициенты \(p_{0},p_{1},\text{...},p_{n}\)
характеристического полинома (13) находятся в положительной области,
т.е.
\(0 \notin p_{i},\overline{p_{i}} - \varepsilon_{i}^{p} > 0,i = \overline{0,n}\)тогда
для того чтобы
\(\text{Re}\lambda_{j}(A) < 0,j = \overline{1,n}\)необходимо и
достаточно, чтобы для главных диагональных миноров
\(\Delta_{1},\Delta_{2},\text{...},\Delta_{n}\) матрицы \(M\)
выполнялись условия

\(\Delta_{3} \in (0,\infty),\Delta_{5} \in (0,\infty),\Delta_{7} \in (0,\infty),\text{...}\).

или

\(\Delta_{2} \in (0,\infty),\Delta_{4} \in (0,\infty),\Delta_{6} \in (0,\infty),\text{...}\)..

Для характеристического уравнения

\[a_{0}\lambda^{n} + a_{1}\lambda^{n - 1} + \text{...} + a_{n} = 0\]

составим таблицу Рауса (матрица размерности (\emph{n}+1)*\emph{m}, где
\emph{m}={[}\emph{n}/2{]}+2)

\[C = \begin{bmatrix}
c_{\text{11}} = a_{0} & c_{\text{12}} = a_{2} & \text{.}\text{.}\text{.} \\
c_{\text{21} +} = a_{1} & c_{\text{22}} = a_{3} & \text{.}\text{.}\text{.} \\
c_{\text{31}} & c_{\text{32}} & \text{.}\text{.}\text{.} \\
c_{\text{41}} & c_{\text{42}} & \text{.}\text{.}\text{.} \\
\text{.}\text{.}\text{.} & \text{.}\text{.}\text{.} & \text{.}\text{.}\text{.}
\end{bmatrix}\]

в первую строку записываются коэффициенты исходного характеристического
уравнения с четными индексами, во вторую -- с нечетными. Элементы
остальных строк рекуррентно вычисляются по формуле

\[\begin{array}{r}
c_{1k} = \left\{ \begin{matrix}
a_{2(k - 1)} & \text{если} & 2(k - 1) \leq n \\
0 & \text{если} & 2(k - 1) > n
\end{matrix} \right.\ ; \\
c_{2k} = \left\{ \begin{matrix}
a_{2k - 1} & \text{если} & 2k - 1 \leq n \\
0 & \text{если} & 2k - 1 > n
\end{matrix} \right.\ ; \\
c_{\text{ik}} = c_{i - 2,k + 1} - d_{i}c_{i - 1,k + 1}; \\
d_{i} = c\frac{}{c_{i - 1,1}};i = \overline{3,n + 1};k = \overline{1,m}\text{.}
\end{array}\]

\emph{{\bfseries Критерий интервальной устойчивости Рауса:}} для
устойчивости системы необходимо и достаточно, чтобы коэффициенты первого
столбца матрицы С находились в правой полуплоскости, т.е.

\[c_{1j} \in (0,\infty),j = \overline{1,n + 1}\]

Для автоматизированного построения интервального характеристического
полинома реализованы алгоритмы Крылова, Леверье, Леверье-Данилевского,
основанные как на классической, так и введенной интервальной математике.

Вышеперечисленные критерии интервальной устойчивости также реализованы в
разработанном пакете интервальных вычислений {[}9{]}.

{\bfseries Выводы.} В работе рассмотрены теоретические основы
математического моделирования динамики беспилотного надводного судна и
выполнен вывод упрощённой курсовой модели, основанной на предположении о
постоянной продольной скорости. Представление модели в линейной форме
позволяет применять современные методы анализа устойчивости, оптимизации
траекторий и синтеза алгоритмов управления.

Упрощённая модель Номото 1-го и 2-го порядка является эффективным
инструментом для проектирования систем курсовой стабилизации, а также
служит основой для разработки более сложных регуляторов.

Для линейной системы общего вида получен критерий интервальной
устойчивости, допускающий небольшие погрешности измерений параметров
беспилотного надводного аппарата.

Полученный критерий устойчивости может быть использован в системах
автоматического управления БНС, моделировании морских сценарием,
кооперативном движении флотилий и разработке цифровых двойников
беспилотных платформ.

Полученные результаты создают основу для последующей инженерной
реализации автономных надводных платформ и их применения в задачах
мониторинга, навигации, патрулирования и экологического контроля.
\end{multicols}

\begin{center}
{\bfseries Литература}
\end{center}

\begin{refs}
1. Маслюк Е.В., Меркулов А.А. Беспилотные морские дроны. - Калининград:
КГТУ, 2018. - 200 с. ISBN 978-5-94826-501-8.

2. Донцов С.В. Основы теории судна. - Одесса: Феникс, 2007. -142 с.
ISBN: 966-8631-93-5.

3. Жинкин В.Б. Теория и устройство корабля. - СПб.: Судостроение, 2002.
-336 с. ISBN 5 -7355-0629-3.

4. Лебедева М.П., Лебедев А.О. Управляемость и моделирования движения
водоизмещающего судна. - Москва, Вологда: Инфра-Инженерия, 2023. - 200
с. ISBN: 978-5-9729-1478-4.

5. Liu R., Dou X., Wang X., Li Q., Mazakov T. Marine moving target
detection algorithm based on improved YOLOv8 //Proc. SPIE 13780, Second
International Conference on Image Processing and Artificial Intelligence
(ICIPAI 2025), 137801M (11 August 2025). DOI
\href{https://doi.org/10.1117/12.3072475}{10.1117/12.3072475}.

6. Mazakova A.,Jomartova Sh.,Wójcik W., Mazakov T., Ziyatbekova G.
Automated Linearization of a System of Nonlinear Ordinary Differential
Equations // International Journal of Electronics and
Telecommunications. - 2023.-Vol.69(4).-P.655-660. DOI
\href{https://doi.org/10.24425/ijet.2023.147684}{10.24425/ijet.2023.147684}

7. Mazakov T.Zh., Wójcik W., Jomartova Sh., Karymsakova N., Ziyatbekova
G., Tursynbai A. The Stability Interval of the Set of Linear System //
International Journal of Electronics and Telecommunications. -2021. -
Vol.67(2).- P.155-161. Manuscript received August 17, 2020; revised
April, 2021, DOI 10.24425/ijet.2021.13595.

8. Мазакова А.Т., Джомартова Ш.А., Мазаков Т.Ж., Зиятбекова Г.З.,
Досаналиева А.Т. Автоматизация процесса линеаризации нелинейной модели,
описываемой обыкновенными дифференциальными уравнениями // Вестник
КазУТБ.-2023. -№1(18).- C.7-23.
DOI 10.58805/kazutb.v.1.18-69.

9. А.с. №7576, 2020, 17 января. Мазаков Т.Ж., Зиятбекова Г.З., Мазакова
А.Т., Джомартова ША Карымсакова НТ, Амирханов Б.С Жолмагамбетова БР.
Библиотека интервальных функций. Программа для ЭВМ.
\end{refs}

\begin{center}
{\bfseries References}
\end{center}

\begin{refs}
1. Masljuk E.V., Merkulov A.A. Bespilotnye morskie drony. -
Kaliningrad: KGTU, 2018. - 200 s. ISBN 978-5-94826-501-8.{[}in
Russian{]}

2. Doncov S.V. Osnovy teorii sudna. - Odessa: Feniks, 2007. -142 s.
ISBN: 966-8631-93-5. {[}in Russian{]}

3. Zhinkin V.B. Teorija i ustrojstvo korablja. - SPb.: Sudostroenie,
2002. -336 s. ISBN 5 -7355-0629-3. {[}in Russian{]}

4. Lebedeva M.P., Lebedev A.O. Upravljaemost'{} i
modelirovanija dvizhenija vodoizmeshhajushhego sudna. - Moskva, Vologda:
Infra-Inzhenerija, 2023. - 200 s. ISBN: 978-5-9729-1478-4. {[}in
Russian{]}

5. Liu R., Dou X., Wang X., Li Q., Mazakov T. Marine moving target
detection algorithm based on improved YOLOv8 //Proc. SPIE 13780, Second
International Conference on Image Processing and Artificial Intelligence
(ICIPAI 2025), 137801M (11 August 2025). DOI
\href{https://doi.org/10.1117/12.3072475}{10.1117/12.3072475}.

6. Mazakova A.,Jomartova Sh.,Wójcik W., Mazakov T., Ziyatbekova G.
Automated Linearization of a System of Nonlinear Ordinary Differential
Equations // International Journal of Electronics and
Telecommunications. - 2023.-Vol.69(4).-P.655-660. DOI
\href{https://doi.org/10.24425/ijet.2023.147684}{10.24425/ijet.2023.147684}

7. Mazakov T.Zh., Wójcik W., Jomartova Sh., Karymsakova N., Ziyatbekova
G., Tursynbai A. The Stability Interval of the Set of Linear System //
International Journal of Electronics and Telecommunications. -2021. -
Vol.67(2).- P.155-161. Manuscript received August 17, 2020; revised
April, 2021, DOI 10.24425/ijet.2021.13595.

8. Mazakova A.T., Dzhomartova Sh.A., Mazakov T.Zh., Zijatbekova G.Z.,
Dosanalieva A.T. Avtomatizacija processa linearizacii nelinejnoj modeli,
opisyvaemoj obyknovennymi differencial' nymi uravnenijami
// Vestnik KazUTB.-2023. -№1(18).- C.7-23.
DOI 10.58805/kazutb.v.1.18-69. {[}in Russian{]}

9. A.s. №7576, 2020, 17 janvarja. Mazakov T.Zh., Zijatbekova G.Z.,
Mazakova A.T., Dzhomartova ShA Karymsakova NT, Amirhanov B.S
Zholmagambetova BR. Biblioteka interval' nyh funkcij.
Programma dlja JeVM. {[}in Russian{]}
\end{refs}

\begin{info}
\hspace{1em}\emph{{\bfseries Сведения об авторах}}

Тажибай И.С. - докторант МИТУ, Алматы, Казахстан, e-mail:
\href{mailto:Talgat.duzbaev@mail.ru}{isataitazhibay@mail.ru};

Мазаков Т.Ж. - д.ф.-м.н., профессор МИТУ, Алматы, Казахстан, e-mail:
tmazakov@mail.ru;

Джомартова Ш.А. - д.т.н., профессор КазНУ им. аль-Фараби, Алматы,
Казахстан, e-mail: jomartova@mail.ru;

Мазакова А.Т. - PhD, старший преподаватель КазНУ им. аль-Фараби,
Алматы, Казахстан, e-mail: aigerym97@mail.ru;

Майлыбаева А.Д. - к.ф.-м.н., ассоциированный профессор, Атырауского
университета им. Х. Досмухамедова, Атырау, Казахстан, e-mail:
a.maylibayeva@asu.edu.kz;

Алиаскар М.С. - лектор МИТУ, Алматы, Казахстан, e-mail:
m.alyasqar@gmail.ru;

Мазакова Б.М. - старший преподаватель, Международный университет
Астана, e-mail: bayan7080@mail.ru.

\hspace{1em}\emph{{\bfseries Information about the authors}}

Tazhibay I. - Doctoral student at IETU, Almaty, Kazakhstan, e-mail:
isataitazhibay@mail.ru,

Mazakov T. - Doctor of Physical and Mathematical Sciences, Professor
at IETU, Almaty, Kazakhstan, e-mail: tmazakov@mail.ru;

Jomartova Sh.- Doctor of Technical Sciences, Professor, Al-Farabi
Kazakh National University, Almaty, Kazakhstan, e-mail:
jomartova@mail.ru;

Mazakova A. - PhD, senior lecturer at KazNU al-Farabi, Almaty,
Kazakhstan, e-mail: aigerym97@mail.ru;

Mailybayeva A. - Candidate of Physical and Mathematical Sciences,
Associate Professor, Kh. Dosmukhamedov Atyrau University, Atyrau,
Kazakhstan, e-mail: a.maylibayeva@asu.edu.kz;

Aliaskar M. - Lecturer at IETU, Almaty, Kazakhstan, e-mail:
m.alyasqar@gmail.ru;

Mazakova B. - senior lecturer, Astana International University, Astana
Kazakhstan, e-mail: bayan7080@mail.ru.
\end{info}
