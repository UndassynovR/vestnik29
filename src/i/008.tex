\id{МРНТИ }{}\href{https://grnti.ru/?p1=28&p2=17&p3=31\#31}{28.17.31}

{\bfseries ГИБРИДНАЯ МОДЕЛЬ МАССОВОГО ОБСЛУЖИВАНИЯ С ДИНАМИЧЕСКИМИ
ПАРАМЕТРАМИ ДЛЯ АВТОМАТИЗИРОВАННОЙ СИСТЕМЫ}

{\bfseries \tsp{1}И.
Масырова}{\bfseries ,
\tsp{1}С.К.
Джолдасбаев}{\bfseries ,
\tsp{1}А.Р.
Гизатуллина}{\bfseries ,
\tsp{1}М.Ж.
Жанадилов}{\bfseries ,
\tsp{2}О.К.~Джолдасбаев}{\bfseries ,
\tsp{3}А.А.
Орманбекова}{\bfseries ,
\tsp{3,4}С.Т.
Мамбетов}\envelope 

\emph{\tsp{1}Международный университет информационных
технологий, Алматы, Казахстан,}

\emph{\tsp{2}Филиал академии государственного управления при
Президенте РК по Алматинский области,}

\emph{Конаев, Казахстан,}

\emph{\tsp{3}Алматинский технологический университет,
Алматы, Казахстан,}

\emph{\tsp{4}Университет «Туран», Алматы, Казахстан}

\corrauthor{Корреспондент-автор:mambetov.saken@gmail.com}

Цифровизация ключевых процессов требует внедрения научно обоснованных
подходов к проектированию автоматизированных систем, способных
адаптироваться к изменяющимся условиям внешней среды. В данной работе
предлагается гибридная модель массового обслуживания с динамическими
параметрами, предназначенная для применения в автоматизированных
системах, ориентированных на распределение ресурсов или заявок в
условиях переменной нагрузки. Модель объединяет элементы марковских
процессов с немарковскими задержками и механизмами динамического
перераспределения потоков, что позволяет учитывать временные колебания,
приоритеты и ограничения системы.

Проведён анализ вероятностных характеристик модели, включая
распределение времени ожидания и степень загруженности обслуживающих
каналов. Разработаны методы адаптивной оптимизации, обеспечивающие
устойчивое функционирование системы при изменяющихся параметрах входного
потока. Предложенная модель может быть использована в различных
прикладных задачах, включая автоматизацию процессов в образовании,
логистике и производственной сфере, где требуется гибкое и эффективное
управление распределением ресурсов.

Представлены результаты имитационного моделирования, демонстрирующие
устойчивость и адаптивность модели при различных сценариях нагрузки.
Полученные данные могут служить основой для разработки
автоматизированных систем управления, обеспечивающих высокую
эффективность и надёжность функционирования.

{\bfseries Ключевые слова:} гибридная модель, массовое обслуживание,
динамические параметры, автоматизация, адаптивная оптимизация,
имитационное моделирование.

{\bfseries АВТОМАТТАНДЫРЫЛҒАН ЖҮЙЕ ҮШІН ДИНАМИКАЛЫҚ ПАРАМЕТРЛЕРІ БАР
ГИБРИДТІ КЕЗЕК ҮЛГІСІ}

{\bfseries \tsp{1}И. Масырова, \tsp{1}С.К.
Джолдасбаев, \tsp{1}А.Р. Гизатуллина,
\tsp{1}М.Ж. Жанадилов, \tsp{2}О.К. Джолдасбаев,
\tsp{3}А.А. Орманбекова, \tsp{3,4}С.Т.
Мамбетов\envelope }

\emph{\tsp{1}Халықаралық ақпараттық технологиялар
университеті, Алматы, Қазақстан,}

\emph{\tsp{2}ҚР Президентінің жанындағы мемлекеттік басқару
академиясының Алматы облысы бойынша филиалы, Қонаев, Қазақстан,}

\emph{\tsp{3}Алматы технологиялық университеті, Алматы,
Қазақстан,}

\emph{\tsp{4}«Тұран» университеті, Алматы, Қазақстан,}

e-mail:
mambetov.saken@gmail.com

Негізгі процестерді цифрландыру қоршаған ортаның өзгеретін жағдайларына
бейімделуге қабілетті автоматтандырылған жүйелерді жобалаудың ғылыми
негізделген тәсілдерін енгізуді талап етеді. Бұл құжат айнымалы жүктеме
жағдайында ресурсты немесе сұрауды бөлуге бағытталған автоматтандырылған
жүйелерде пайдалануға арналған динамикалық параметрлері бар гибридті
кезек үлгісін ұсынады. Модель Марков процестерінің элементтерін
Марковтық емес кідірістермен және динамикалық ағынды қайта бөлу
механизмдерімен біріктіреді, бұл уақыт ауытқуларын, басымдықтарды және
жүйе шектеулерін қарастыруға мүмкіндік береді.

Күту уақытының бөлінуін және қызмет көрсету арналарын пайдалану
дәрежесін қоса алғанда, модельдің ықтималдық сипаттамаларына талдау
жүргізіледі. Кіріс ағынының өзгеретін параметрлері кезінде жүйенің
тұрақты жұмысын қамтамасыз ету үшін бейімделген оңтайландыру әдістері
әзірленген. Ұсынылған модель әртүрлі қосымшаларда, соның ішінде білім
беру, логистика және өндірістегі процестерді автоматтандыруда,
ресурстарды бөлуді икемді және тиімді басқару қажет болған жағдайда
қолданылуы мүмкін.

Модельдің әртүрлі жүктеме сценарийлерінде беріктігі мен бейімделуін
көрсететін модельдеу нәтижелері ұсынылған. Алынған мәліметтер жоғары
тиімділік пен жұмыс сенімділігін қамтамасыз ететін автоматтандырылған
басқару жүйелерін әзірлеуге негіз бола алады.

{\bfseries Түйін сөздер:} гибридті модель, кезекке тұру, динамикалық
параметрлер, автоматтандыру, адаптивті оңтайландыру, имитациялық
модельдеу.

{\bfseries A HYBRID QUEUEING MODEL WITH DYNAMIC PARAMETERS FOR AN AUTOMATED
SYSTEM}

{\bfseries \tsp{1}I. Massyrova, \tsp{1}S.K.
Joldasbayev, \tsp{1}А.R. Gizatullina,
\tsp{1}М.Zh. Zhanadilov, \tsp{2}О.К.
Joldasbayev, \tsp{3}А.А. Ormanbekova,
\tsp{3,4}S.Т. Mambetov\envelope }

\emph{\tsp{1}International IT University, Almaty,
Kazakhstan,}

\emph{\tsp{2}Branch of the Academy of Public Administration
under the president of the Republic of Kazakhstan in Almaty region,
Konayev, Kazakhstan,}

\emph{\tsp{3}Almaty Technological University, Almaty,
Kazakhstan,}

\emph{\tsp{4}Turan University, Almaty, Kazakhstan,}

e-mail:
mambetov.saken@gmail.com

Digitalization of key processes requires the implementation of
scientifically sound approaches to the design of automated systems
capable of adapting to changing environmental conditions. This paper
proposes a hybrid queuing model with dynamic parameters, designed for
use in automated systems focused on resource or request allocation under
variable load conditions. The model combines elements of Markov
processes with non-Markov delays and dynamic flow redistribution
mechanisms, allowing for consideration of time fluctuations, priorities,
and system constraints.

An analysis of the model' s probabilistic
characteristics, including the distribution of waiting times and the
degree of utilisation of service channels, is conducted. Adaptive
optimization methods are developed to ensure stable system operation
under changing input flow parameters. The proposed model can be used in
various applications, including process automation in education,
logistics, and manufacturing, where flexible and efficient resource
allocation management is required.

Simulation results are presented, demonstrating the
model' s robustness and adaptability under various load
scenarios. The obtained data can serve as a basis for the development of
automated control systems that ensure high efficiency and reliability of
operation.

{\bfseries Keywords:} hybrid model; queuing, dynamic parameters,
automation, adaptive optimization, simulation modeling.

{\bfseries Введение.} Современная система высшего образования в большей
части автоматизирована, и тем не менее, все еще требует внедрение и
изменения многих, еще не автоматизированных учебных и административных
процессов с целью упрощения массивных и трудоемких задач, сокращая
затраты времени и большого внимания операторов с учетом удобств
пользователей. Учебные заведения, где применяются передовые технологии
автоматизации учебно-административных процессов преуспевает по обработке
данных, чем мотивирует студентов, предоставляя удобства при
взаимодействии с администрацией и преподавателями, тем самым влияет на
продуктивность организации в целом {[}1{]}. Одним из таких процессов,
который мы можем автоматизировать и цифровизовать является прохождение
производственной практики студентов в разных организациях. В данном
исследовании мы на основе Теории массового обслуживания (ТМО) попытались
смоделировать распределение студентов по предприятиям с учетом
загруженности практикантами, а также специфики направлений подготовок и
динамики поступления заявок на рассмотрение кандидатуры и предложили
гибридную модель, где учитываются многие детали, которые не берут во
внимание во многих случаях.

Многоканальные системы типа M/M/N рассматривают марковские процессы,
которые не соответствуют распределениям студентов по разным организациям
с учетом условий {[}2{]}. Задержки, при распределении студентов по
организациям, адаптация студентов, административные процессы,
согласование, подписание документов и другая специфика имеют модель
немарковского характера. При данных условиях оптимальным решением будет
разработка новой модели объединяющий марковские процессы обслуживания
очереди с немарковскими задержками и динамическим перераспределением
заявок. Разработанная гибридная модель будет применяться в
информационной системе для автоматизации производственной практики
студентов, исследования которой были опубликованы нами в работе {[}3{]}.
Основной акцент данной работы направлен на создание математической
модели системы распределения с учетом вероятности процессов, а также
механизмов балансировки нагрузки на сервере и анализа отказов. Также
проводится анализ вероятностных характеристик системы и оптимизация
функционирования системы во внедренной информационной системе. В
дальнейшем на основе данных исследований мы планируем внедрение
интеллектуальных систем в платформу по организации и управлению
производственной практики студентов, что позволит повысить эффективность
распределения ресурсов и сократить время ожидания.

{\bfseries Литературный обзор.} При моделировании процессов обработки
запросов в системе массового обслуживания (СМО) необходимо учитывать
многие параметры и условия поставленных задач {[}4{]}. Моделирование
системы обслуживания заявок основано на использовании математического
аппарата теории исследования операций, теории транспортных сетей и
графов, а также теории массового обслуживания. Главной характеристикой
качества работы Интернет-портала для учебных процессов является
возможность быстрого поиска необходимой информации и быстрые обработки
запросов {[}5{]}. Модель позволяет оценить и выставить возможные пределы
изменения отдельных параметров системы {[}6{]}. Управление ресурсами в
облачных и периферийных вычислениях, управление и распределение
ресурсов, планирование, мониторинг, оркестровка в распределенных
вычислительных системах, управление ресурсами с учетом задержек,
энергоэффективное управление ресурсами, совместимость и переносимость,
безопасности конфиденциальность в управлении ресурсами, надежное
управление ресурсами, отказоустойчивость в управлении и моделирование,
связанные с управлением являются важными задачами при проектировании
системы массового обслуживания {[}7{]}. Эффективность системы
предоставления услуг учитывает выгодное позиционирование ресурсов с
алгоритмической балансировкой нагрузки на серверах с максимальной
выгодой, как для пользователя, так и для стороны, предоставляющей услуги
{[}8{]}. Cистема рекомендаций курсов для студентов высших учебных
заведений на основе отбора функций и машинного обучения немаловажная
разработка, требующая особого внимания {[}9{]}. В {[}10{]} было
рассмотрены история и современные подходы автоматизации в образовании,
тесно связанные с проектами по разработке образовательных программ
сотрудничества производственного, образовательного и государственных
секторов для подготовки кадров. В работе {[}11{]} рассмотрены
современные системы управления обучением, и проведен сравнительный
анализ. В {[}12{]} отмечается, использование качественных и
функциональных платформ способствует активному вовлечению студентов в
образовательный процесс, что, в свою очередь, положительно влияет на
динамику успеваемости и качество образования. Исследования поведений
учащихся в системе управления обучением и закономерности, взаимодействия
конкретных субъектов образовательного процесса излагаются в {[}13{]}.
Также в {[}14{]} рассмотрены инновационные решения в области обучения. В
работе {[}15{]} выявлены недостатки существующих бизнес-процессов и
предложен способ решения проблемы, позволяющий повысить эффективность
использования трудовых ресурсов. В {[}16{]} обсуждаются изменения,
необходимые в управлении учебными курсами для создания системы, которая
будет достаточно гибкой и эффективной, чтобы справиться с большим
количеством студентов без потери качества. Ценность статьи {[}17{]}
заключается в представлении концептуальной модели, которая могла бы
реализовать «массовую кастомизацию» в университетах путем интеграции
человеческих ресурсов, операционных и функциональных измерений в
систематическую разработку для предоставления индивидуальных услуг
студентам как отдельным лицам. Эффективное управление человеческими
ресурсами является критически важным для успеха строительных проектов. В
данной работе {[}18{]} предложена динамическая модель, позволяющая
эффективно распределять рабочую силу с использованием подхода системной
динамики. В работе {[}19{]} введены и изучены динамические игры
распределения ресурсов, позволяющие игрокам выбирать ресурсы в
итеративной и несимультанной манере. В данной статье {[}20{]} предложено
решение для улучшения качества платформ электронного обучения в
виртуальных и традиционных университетах путем динамического
распределения ресурсов. В {[}21{]} представлено стохастическое
моделирование и симуляция динамического распределения ресурсов, что
способствует реалистичному моделированию бизнес-процессов.

Рассмотренные исследования охватывают различные аспекты динамического
распределения ресурсов в различных контекстах: проектные работы,
академические инвестиции, виртуальные образовательные платформы,
дистанционное обучение и стохастические моделирования бизнес-процессов.
Большинство акцентируются на управление ресурсами в проектах с жесткой
структурой, разработку игровых моделей, которые не применимы в реальной
разработке. Также в работах рассмотрены агрегированные или
стохастические модели, в которых не учитываются индивидуальные параметры
объектов. Работы с виртуальными платформами не учитывают реальное
взаимодействие между вузами и предприятиями.

Как мы видим большинство научных подходов были исследованы на тему СМО
для автоматизации учебных процессов, но нет конкретного применения для
системы автоматизации производственной практики, где учитывались этапы
производственной практики в качестве Марковских процессов обслуживания с
немарковскими задержками и динамическим перераспределением заявок, тем
самым подчеркивая актуальность данного исследования.

Таким образом, остается недостаточно изученной сфера прикладного
моделирования динамического распределения студентов на практику с учетом
индивидуальных параметров обучающегося: успеваемость, навыки, пройденные
предметы, освоение пройденных предметов; предпочтений и критериев со
стороны предприятий (навыки, владения и т д); наличия ограниченности
времени и вместимости; необходимости автоматизации процессов через
имитационные и математические модели.

{\bfseries Материалы и методы.} В рамках данного исследования применяются
методы ТМО, математического моделирования, а также имитационного
моделирования для анализа оптимизации процессов автоматизированного
распределения агентов в системе. В качестве агентов в нашей задаче
выступают студенты, которые поступают в систему как заявки - поток
поступающих, а места практики - обслуживающее предприятие. Время
обслуживания зависит от случайных факторов: гибкость узлов, требования,
задержки. Возможные отказы: перегруженность узла, несоответствие
квалификационных требований.

Процесс распределения студентов по предприятиям чаще всего производится
исходя от интересов студента, соответствующих параметров предприятия:
студент выбирает предприятие, предприятие также оценивает необходимые
или скорее, соответствующие параметры. После выбора места идет
согласование с вузом и предприятием, оформляется трехсторонний договор и
письмо-согласие от предприятия.

{\bfseries Математическая модель}

Обозначим следующие параметры:

λ - интенсивность поступления заявок (студенты проходящие практику);

\(M\ \)- количество предприятий;

\(w_{s}\) - вес студента, определяемый по успеваемости и компетентности,
оцениваемой вузом;

\(p_{ij}\) - вероятность распределения студента \(i\ \)на предприятие
\(j\ \);

\(t\ \) - текущее время;

\(N_{i}\) - количество мест на предприятии \(i\ \);

\(\mu_{i}\) - интенсивность обслуживания на определенном предприятий (с
учетом разного времени прохождения практики);

\(P_{fail}\) - вероятность отказа (при перегруженности всех
предприятий);

\(T_{s}\) - среднее время обслуживания;

\(T_{w}\) - время ожидания студента до начала практики;

\(T_{t}\) - полное время нахождения заявки в системе (ожидание +
практика).

Обозначив данные параметры, мы можем дать описание потоку заявок и
распределение студентов. Студенты подают заявки в разное время и их
поступление можно описать как пуассоновский процесс:

%% \begin{longtable}[]{@{}
%%   >{\raggedright\arraybackslash}p{(\linewidth - 2\tabcolsep) * \real{0.9116}}
%%   >{\raggedright\arraybackslash}p{(\linewidth - 2\tabcolsep) * \real{0.0884}}@{}}
%% \toprule\noalign{}
%% \begin{minipage}[b]{\linewidth}\raggedright
%% \[P_{n}(t) = \frac{(\lambda t)^{n}e^{\lambda t}}{n!}\]
%% \end{minipage} & \begin{minipage}[b]{\linewidth}\raggedleft
%% (1)
%% \end{minipage} \\
%% \midrule\noalign{}
%% \endhead
%% \bottomrule\noalign{}
%% \endlastfoot
%% \end{longtable}

где \(P_{n}(t)\) - вероятность того, что за время \(t\ \) поступит
\(n\ \) заявок. Переменная \emph{n} означает количество заявок
студентов, поступивших за время \emph{t.} Зависимость вероятности
распределения студента на конкретное предприятие можно описать как:

%% \begin{longtable}[]{@{}
%%   >{\raggedright\arraybackslash}p{(\linewidth - 2\tabcolsep) * \real{0.9116}}
%%   >{\raggedright\arraybackslash}p{(\linewidth - 2\tabcolsep) * \real{0.0884}}@{}}
%% \toprule\noalign{}
%% \begin{minipage}[b]{\linewidth}\raggedright
%% \[p_{ij} = \frac{w_{s}^{i}w_{e}^{j}}{\sum_{k = 1}^{M}w_{s}^{i}w_{e}^{k}}\]
%% \end{minipage} & \begin{minipage}[b]{\linewidth}\raggedleft
%% (2)
%% \end{minipage} \\
%% \midrule\noalign{}
%% \endhead
%% \bottomrule\noalign{}
%% \endlastfoot
%% \end{longtable}

где \(w_{s}^{i}\) - вес студента \(i\ \), \(w_{e}^{j}\) - вес
предприятия \(j\ \), который учитывает престиж, условия практики и
другие положительные параметры, \(k\ \)- индекс предприятия, по кторому
идет суммирование в знаменателе. Данное уравнение показывает вероятность
распределения студента зависит от его характеристик и рейтинга
предприятия, чем выше значения \(w_{s}^{i}\,,\, w_{e}^{j}\) , тем больше
вероятность попадания в это предприятие.

Каждое предприятие имеет ограничение по количеству приема студентов.
Вероятность отказа для многоканальной системы массового обслуживания
рассчитывается по формуле:

%% \begin{longtable}[]{@{}
%%   >{\raggedright\arraybackslash}p{(\linewidth - 2\tabcolsep) * \real{0.9116}}
%%   >{\raggedright\arraybackslash}p{(\linewidth - 2\tabcolsep) * \real{0.0884}}@{}}
%% \toprule\noalign{}
%% \begin{minipage}[b]{\linewidth}\raggedright
%% \[P_{rej} = \frac{\frac{\left( \rho_{j} \right)^{N_{j}}}{N_{j}!}}{\sum_{k = 0}^{N_{j}}\frac{\left( \rho_{j} \right)^{k}}{k!}}\]
%% \end{minipage} & \begin{minipage}[b]{\linewidth}\raggedleft
%% (3)
%% \end{minipage} \\
%% \midrule\noalign{}
%% \endhead
%% \bottomrule\noalign{}
%% \endlastfoot
%% \end{longtable}

где \(\rho_{j} = \ \frac{\lambda p_{ij}}{\mu_{j}}\) - коэффициент
загрузки предприятия. Если вероятность отказа превышает допустимое
значение \(\theta\), студент

- либо находится в ожидании освобождения места,

- либо перенаправляется в свободное место другого предприятия.

Если предприятие перегружено \(P_{rej} > \theta\), поток заявок можно
перераспределить в другие предприятия:

%% \begin{longtable}[]{@{}
%%   >{\raggedright\arraybackslash}p{(\linewidth - 2\tabcolsep) * \real{0.9116}}
%%   >{\raggedright\arraybackslash}p{(\linewidth - 2\tabcolsep) * \real{0.0884}}@{}}
%% \toprule\noalign{}
%% \begin{minipage}[b]{\linewidth}\raggedright
%% \[\lambda_{j}' = \lambda_{j}\left( 1 - P_{rej} \right) + \lambda_{alt}\]
%% \end{minipage} & \begin{minipage}[b]{\linewidth}\raggedleft
%% (4)
%% \end{minipage} \\
%% \midrule\noalign{}
%% \endhead
%% \bottomrule\noalign{}
%% \endlastfoot
%% \end{longtable}

где \(\lambda_{alt}\) - поток студентов, направленных в альтернативные
предприятия.

Для временных параметров модели введем следующие определения:

Время согласования заявки на предприятии описывается экспоненциальным
распределением:

%% \begin{longtable}[]{@{}
%%   >{\raggedright\arraybackslash}p{(\linewidth - 2\tabcolsep) * \real{0.9116}}
%%   >{\raggedright\arraybackslash}p{(\linewidth - 2\tabcolsep) * \real{0.0884}}@{}}
%% \toprule\noalign{}
%% \begin{minipage}[b]{\linewidth}\raggedright
%% \[P\left( T_{s} < t \right) = 1 - e^{- \mu_{j}t}\]
%% \end{minipage} & \begin{minipage}[b]{\linewidth}\raggedleft
%% (5)
%% \end{minipage} \\
%% \midrule\noalign{}
%% \endhead
%% \bottomrule\noalign{}
%% \endlastfoot
%% \end{longtable}

Среднее время обслуживания будет выражена как:

%% \begin{longtable}[]{@{}
%%   >{\raggedright\arraybackslash}p{(\linewidth - 2\tabcolsep) * \real{0.9116}}
%%   >{\raggedright\arraybackslash}p{(\linewidth - 2\tabcolsep) * \real{0.0884}}@{}}
%% \toprule\noalign{}
%% \begin{minipage}[b]{\linewidth}\raggedright
%% \[T_{s} = \frac{1}{\mu}\]
%% \end{minipage} & \begin{minipage}[b]{\linewidth}\raggedleft
%% (6)
%% \end{minipage} \\
%% \midrule\noalign{}
%% \endhead
%% \bottomrule\noalign{}
%% \endlastfoot
%% \end{longtable}

Время ожидания до начала практики выражается:

%% \begin{longtable}[]{@{}
%%   >{\raggedright\arraybackslash}p{(\linewidth - 2\tabcolsep) * \real{0.9116}}
%%   >{\raggedright\arraybackslash}p{(\linewidth - 2\tabcolsep) * \real{0.0884}}@{}}
%% \toprule\noalign{}
%% \begin{minipage}[b]{\linewidth}\raggedright
%% \[T_{w} = \frac{\lambda}{\mu(\mu - \lambda)}\]
%% \end{minipage} & \begin{minipage}[b]{\linewidth}\raggedleft
%% (7)
%% \end{minipage} \\
%% \midrule\noalign{}
%% \endhead
%% \bottomrule\noalign{}
%% \endlastfoot
%% \end{longtable}

Полное время будет

%% \begin{longtable}[]{@{}
%%   >{\raggedright\arraybackslash}p{(\linewidth - 2\tabcolsep) * \real{0.9116}}
%%   >{\raggedright\arraybackslash}p{(\linewidth - 2\tabcolsep) * \real{0.0884}}@{}}
%% \toprule\noalign{}
%% \begin{minipage}[b]{\linewidth}\raggedright
%% \[T_{t} = T_{w} + T_{s}\]
%% \end{minipage} & \begin{minipage}[b]{\linewidth}\raggedleft
%% (8)
%% \end{minipage} \\
%% \midrule\noalign{}
%% \endhead
%% \bottomrule\noalign{}
%% \endlastfoot
%% \end{longtable}

И среднее время перераспределения при заполнении мест:

%% \begin{longtable}[]{@{}
%%   >{\raggedright\arraybackslash}p{(\linewidth - 2\tabcolsep) * \real{0.9116}}
%%   >{\raggedright\arraybackslash}p{(\linewidth - 2\tabcolsep) * \real{0.0884}}@{}}
%% \toprule\noalign{}
%% \begin{minipage}[b]{\linewidth}\raggedright
%% \[T_{r} = \min\left( T_{\max},\frac{1}{\lambda} \right)\]
%% \end{minipage} & \begin{minipage}[b]{\linewidth}\raggedleft
%% (9)
%% \end{minipage} \\
%% \midrule\noalign{}
%% \endhead
%% \bottomrule\noalign{}
%% \endlastfoot
%% \end{longtable}

где \(T_{\max}\) - максимальное время ожидания перераспределения.

На основе этих математических подходов для сравнения предложенной модели
с классической моделью было проведено имитационное моделирование
системы. На первом этапе было выявлено соотношение количества агентов на
распределение времени нахождения заявки в системе. Ожидалось, что в
классической модели будет больше необработанных заявок в очереди, с
большим количеством времени ожидания, так как остаются в режиме ожидания
либо выходит из очереди, не поступая в обработку, а в гибридной модели
среднее время нахождения агентов в системе уменьшиться,
перераспределяясь при перезагрузке. Для эксперимента было взято
количество 1000 студентов и 100 предприятий. В классической модели
среднее время ожидания было чуть меньше, чем в предложенной модели, что
на первый взгляд может показаться как отрицательное нововведение. Если
мест в предприятиях недостаточно, вероятность отказа \(P_{rej}\) будет
выше. Однако, перераспределение в предложенной модели поможет уменьшить
перегрузку отдельных предприятий. Также классическая модель показывает
более высокие время нахождения в системе.

\fig{i/image50}{}

{\bfseries Рис.1 - Сравнение моделей с учетом улучшений}

\fig{i/image51}{}

{\bfseries Рис.2 - Влияние количества студентов на отказы}

\fig{i/image52}{}

{\bfseries Рис.3 - Влияние числа студентов на среднее время нахождения
запроса в системе}

\fig{i/image53}{}

{\bfseries Рис.4 - Зависимость плотности вероятности от времени пребывания
запроса в системе}

{\bfseries Обсуждение и результаты.} Проведенное исследование позволило
оценить эффективность предложенной гибридной модели в системе
распределения студентов по предприятиям по производственной практике в
сравнении с классическим подходом. Анализ распределения времени
нахождения заявок в системе в классической модели определяют
предоставления места для студентов либо покидают систему, тогда как в
гибридной модели предусмотрено динамическое перераспределение.

Среднее значение времени ожидания обработки заявки для классической
модели составило 10.29, тогда как для предложенной модели составило
10.09, что определяет гибридную модель более применимой для системы
распределения при программной разработке. Данный подход демонстрирует
положительный эффект при небольшом количестве студентов и снижает
вероятность отказов. При увеличении числа студентов все же она приводит
к дополнительным нагрузкам, увеличивая время ожидания.

Дополнительно была проанализирована нагрузка на обслуживающие элементы
системы, включая предприятия с разной пропускной способностью. Выявлено,
что гибридная модель более эффективно перераспределяет поток заявок,
снижая перегрузку на наиболее популярных предприятиях и равномернее
распределяя студентов. Это способствует повышению общего коэффициента
использования ресурсов системы и уменьшению количества незаполненных
мест. Кроме того, внедрение динамических параметров позволяет системе
адаптироваться к изменениям во входном потоке заявок в реальном времени.
Важно отметить, что предложенная модель обеспечивает большую
устойчивость при пиковых нагрузках и снижает вероятность полного отказа
в распределении. Моделирование различных сценариев показало, что при
оптимальной настройке параметров перераспределения возможно достичь
улучшения ключевых метрик - таких как среднее время пребывания заявки в
системе и уровень удовлетворенности студентов. Таким образом, результаты
подтверждают целесообразность использования гибридного подхода для
автоматизации процессов организации производственной практики с учетом
современных требований цифровой трансформации.

{\bfseries Выводы}. Исследование было направлено на построение и
сравнительный анализ моделей распределение заявок по множеству ресурсов
для оптимизации работ системы по обслуживанию. Классическая модель,
основанная на случайном распределении, была дополнена улучшенной
гибридной моделью, в которой учитывается текущая загрузка ресурсов при
принятии решении распределения. Проведенное исследование показывает, что
внедрение даже простейшего механизмов динамической адаптации позволяет
существенно снизить среднее время пребывания заявок в системе. Анализ
плотностей распределения времени пребывания в системе подтвердил
преимущества предложенной модели - повышением эффективности
использования ресурсов.

В данной работе реализована модель, имитирующая работу реальной
распределяющей системы с учетом массовости процессов, динамических
ограничений и реалистичного взаимодействия участников. Имитация
реального административного процесса: согласование, отказ,
перераспределение может быть легко встроена в интеграцию с цифровыми
платформами вуза. Используемый подход позволяет учитывать не только
логическую сторону распределения, но и аспекты оптимальности выбора с
обеих сторон - студента и организации. Это особенно актуально в условиях
цифровой трансформации образования и растущей потребности вузов к
интеграции с внешними профессиональными структурами. Представленная
модель будет применена как основа для разработки цифровой платформы
автоматизированного распределения студентов по организациям для
прохождения производственной практики.

Представленные результаты будут применены для при разработке и
оптимизации информационной системы автоматизированного распределения, в
том числе, в задачах управления потоками в образовательной и
производственной сферах и их взаимодействия. Перспективным направлением
дальнейших исследований будет интеграция адаптивных методов и методов
машинного обучения в рассматриваемую модель, а также масштабирование
решения для обработки больших данных в реальном времени.

\emph{{\bfseries Примечание}: В процессе подготовки статьи использовались
инструменты искусственного интеллекта, в том числе языковая модель
GPT-4, предоставляемая OpenAI. Модель использовалась исключительно для
помощи в редактировании текста и структурировании научного изложения}
{[}22{]}\emph{.}

{\bfseries Литература}

1. Balakayeva G.T., Ezhichelvan P., Tursynkozha M.K. Analysis, research
and development of an innovative enterprise digitalization system for
remote work // International Journal of Mathematics and Physics. - 2022.
- Vol.13(1). - P.19-29. DOI
\href{https://doi.org/10.26577/ijmph.2022.v13.i1.02}{10.26577/ijmph.2022.v13.i1.02}.

2. Дворяткина С.Н., Прокуратова О.Н. Марковские процессы и простейшие
модели теории массового обслуживания. - М.: ФЛИНТА, 2019. - 80 с. ISBN
978-5-9765-4828-2.

3. Massyrova I., Joldasbayev O., Joldasbayev S., Bolysbek A., Mambetov
S. Automation of the system for industrial practice and internships for
students in organizations outside of the university // News of the
National Academy of Sciences of the Republic of Kazakhstan.
Physico-Mathematical Series. -- 2025. -- № 1. -- P.168--184. DOI
\href{https://doi.org/10.32014/2025.2518-1726.332}{10.32014/2025.2518-1726.332}.

4. Самусевич Г. Моделирование процессов функционирования СМО. Учебное
пособие для вузов. - Litres, 2021. - 112 c. ISBN: 9785534142556.

5. Кузнецова И.А. Дистанционное обучение как система массового
обслуживания // Вестник евразийской
науки.-2011.-№2(7).\href{http://naukovedenie.ru/}{http://naukovedenie.ru}.-
Дата обращения: 03.03.2025).

6. Лобашев В.Д. Элементы системы массового обслуживания в управлении
учебным процессом // Проблемы современного педагогического образования.
- 2023. - № 78-1.-C.226-229.

7. Mukherjee A., De D., Buyya R. Cloud Computing Resource Management //
In: book:Resource Management in Distributed Systems.-- Singapore:
Springer, 2024.-P.17-37. DOI
\href{https://doi.org/10.1007/978-981-97-2644-8_2}{10.1007/978-981-97-2644-8\_2}.

8. Joldasbayev S., Balakayeva G., Joldasbayev O. Application of load
balancing algorithms to improve the quality of service delivery using
modifications of the least connections algorithm // Journal of
Theoretical and Applied Information Technology. - 2020. - Vol.98(12). -
P.2063--2077. ISSN 1992-8645.

9. Arcinas M.M., Meenakshi M., Bahalkar P.S., Bhaturkar D., Lalar S.,
Rane K.P., Raghuvanshi A. An efficient course recommendation system for
higher education students using machine learning techniques // Bulletin
of Electrical Engineering and Informatics.- 2025.-Vol.14(2). - P.
1468--1475. DOI
\href{https://doi.org/10.11591/eei.v14i2.7711}{10.11591/eei.v14i2.7711}.

10. Ishii K., Tamaki K. Automation in Education/Learning Systems // In:
Nof S. (ed.) Springer Handbook of Automation. Springer Handbooks. --
Berlin, Heidelberg: Springer, 2009. -- P.1503--1527. DOI
\href{https://doi.org/10.1007/978-3-540-78831-7_85}{10.1007/978-3-540-78831-7\_85}.

11. Aldiab A., Chowdhury H., Kootsookos A., Alam F., Allhibi H.
Utilization of Learning Management Systems (LMSs) in higher education
system: A case review for Saudi Arabia // Energy Procedia.- 2019. -Vol.
160. -P.731-737. DOI
\href{https://doi.org/10.1016/j.egypro.2019.02.186}{10.1016/j.egypro.2019.02.186}.

12. Rabiman R., Nurtanto M., Kholifah N. Design and Development
E-Learning System by Learning Management System (LMS) in Vocational
Education // Online Submission. -- 2020. -- Vol.9(1). - P.1059-1063.

13. Juhaňák L., Zounek J., Rohlíková L. Using process mining to analyze
students'{} quiz-taking behavior patterns in a learning
management system // Computers in Human Behavior.- 2019. - Vol.92. - P.
496-506. DOI
\href{https://doi.org/10.1016/j.chb.2017.12.015}{10.1016/j.chb.2017.12.015}.

14. Chick R.C., Clifton G.T., Peace K.M., Propper B.W., Hale D.F.,
Alseidi A.A., Vreeland T.J. Using technology to maintain the education
of residents during the COVID-19 pandemic // Journal of Surgical
Education. - 2020. - Vol.77(4). - P.729-732. DOI
\href{https://doi.org/10.1016/j.jsurg.2020.03.018}{10.1016/j.jsurg.2020.03.018}.

15. Frolova M.A., Razumova T.A. The use of process approach to base the
need of automation of business processes in educational institutions //
AIP Conference Proceedings. - 2017. - Vol.1797(1). - P.
040004-1--040004-8. DOI
\href{https://doi.org/10.1063/1.4972460}{10.1063/1.4972460}.

16. Eriksen S.D. TQM and the transformation from an élite to a mass
system of higher education in the UK//Quality Assurance in
Education.-1995.-Vol.3(1).-P.14--29.DOI
\href{https://doi.org/10.1108/09684889510146795}{10.1108/09684889510146795}.

17. Pham D.T., Jaaron A.A.M. Design for Mass Customisation in Higher
Education: a Systems-Thinking Approach // Systemic Practice and Action
Research. - 2018. -Vol.31. - P.293-310. DOI
\href{https://doi.org/10.1007/s11213-017-9426-7}{10.1007/s11213-017-9426-7}.

18. Dabirian S., Abbaspour S., Khanzadi M., Ahmadi M. Dynamic modelling
of human resource allocation in construction projects // International
Journal of Construction Management. - 2019. - Vol.22(2). - P.182-191.
DOI
\href{https://doi.org/10.1080/15623599.2019.1616411}{10.1080/15623599.2019.1616411}.

19. Brudner A., Gavious A. A dynamic model of investment in research and
teaching facilities in academic institutions // Annals of Operations
Research. - 2024.-Vol.343.-P.67-85. DOI
\href{https://doi.org/10.1007/s10479-024-06232-w}{10.1007/s10479-024-06232-w}.

20. Ly O.R., Boko U.H.S., Gueye K., Ouya S. Proposal of a Dynamic
Resource Allocation Solution for Virtual Classroom Platforms // In: Auer
M., Hortsch H., Sethakul P. (eds) The Impact of the 4th Industrial
Revolution on Engineering Education. ICL 2019. Advances in Intelligent
Systems and Computing. -- Cham: Springer, 2020. -- Vol.1135. - P.
59-68. DOI
\href{https://doi.org/10.1007/978-3-030-40271-6_7}{10.1007/978-3-030-40271-6\_7}.

21. Donyina A. Stochastic Modelling and Simulation of Dynamic Resource
Allocation // In: Ehrig H., Rensink A., Rozenberg G., Schürr A. (eds)
Graph Transformations. ICGT 2010. Lecture Notes in Computer Science. --
Berlin, Heidelberg: Springer.- 2010.-Vol.6372. - P.388-390. DOI
\href{https://doi.org/10.1007/978-3-642-15928-2_28}{10.1007/978-3-642-15928-2\_28}.

22. OpenAI. ChatGPT (GPT-4) Language Model. OpenAI. - 2023. - URL:
\url{https://chat.openai.com}. Intended use: language editing and
grammar checking. GPT-4. Модель использовалась исключительно для помощи
в редактировании текста и структурирова-

нии научного изложения.-Дата обращения: 06.10.2025.

{\bfseries References}

1. Balakayeva G.T., Ezhichelvan P., Tursynkozha M.K. Analysis, research
and development of an innovative enterprise digitalization system for
remote work // International Journal of Mathematics and Physics. - 2022.
- Vol.13(1). - P.19-29. DOI
\href{https://doi.org/10.26577/ijmph.2022.v13.i1.02}{10.26577/ijmph.2022.v13.i1.02}.

2. Dvorjatkina S.N., Prokuratova O.N. Markovskie processy i prostejshie
modeli teorii massovogo obsluzhivanija. - M.: FLINTA, 2019. - 80 s. ISBN
978-5-9765-4828-2. {[}in Russian{]}

3. Massyrova I., Joldasbayev O., Joldasbayev S., Bolysbek A., Mambetov
S. Automation of the system for industrial practice and internships for
students in organizations outside of the university // News of the
National Academy of Sciences of the Republic of Kazakhstan.
Physico-Mathematical Series. -- 2025. -- № 1. -- P.168--184. DOI
\href{https://doi.org/10.32014/2025.2518-1726.332}{10.32014/2025.2518-1726.332}.

4. Samusevich G. Modelirovanie processov funkcionirovanija SMO. Uchebnoe
posobie dlja vuzov. - Litres, 2021. - 112 c. ISBN: 9785534142556. {[}in
Russian{]}

5. Kuznecova I.A. Distancionnoe obuchenie kak sistema massovogo
obsluzhivanija // Vestnik evrazijskoj
nauki.-2011.-№2(7).http://naukovedenie.ru.- Data obrashhenija:
03.03.2025). {[}in Russian{]}

6. Lobashev V.D. Jelementy sistemy massovogo obsluzhivanija v upravlenii
uchebnym processom // Problemy sovremennogo pedagogicheskogo
obrazovanija. - 2023. - № 78-1.-C.226-229. {[}in Russian{]}

7. Mukherjee A., De D., Buyya R. Cloud Computing Resource Management //
In: book:Resource Management in Distributed Systems.-- Singapore:
Springer, 2024.-P.17-37. DOI
\href{https://doi.org/10.1007/978-981-97-2644-8_2}{10.1007/978-981-97-2644-8\_2}.

8. Joldasbayev S., Balakayeva G., Joldasbayev O. Application of load
balancing algorithms to improve the quality of service delivery using
modifications of the least connections algorithm // Journal of
Theoretical and Applied Information Technology. - 2020. - Vol.98(12). -
P.2063--2077. ISSN 1992-8645.

9. Arcinas M.M., Meenakshi M., Bahalkar P.S., Bhaturkar D., Lalar S.,
Rane K.P., Raghuvanshi A. An efficient course recommendation system for
higher education students using machine learning techniques // Bulletin
of Electrical Engineering and Informatics.- 2025.-Vol.14(2). - P.
1468--1475. DOI
\href{https://doi.org/10.11591/eei.v14i2.7711}{10.11591/eei.v14i2.7711}.

10. Ishii K., Tamaki K. Automation in Education/Learning Systems // In:
Nof S. (ed.) Springer Handbook of Automation. Springer Handbooks. --
Berlin, Heidelberg: Springer, 2009. -- P.1503--1527. DOI
\href{https://doi.org/10.1007/978-3-540-78831-7_85}{10.1007/978-3-540-78831-7\_85}.

11. Aldiab A., Chowdhury H., Kootsookos A., Alam F., Allhibi H.
Utilization of Learning Management Systems (LMSs) in higher education
system: A case review for Saudi Arabia // Energy Procedia.- 2019. -Vol.
160. -P.731-737. DOI
\href{https://doi.org/10.1016/j.egypro.2019.02.186}{10.1016/j.egypro.2019.02.186}.

12. Rabiman R., Nurtanto M., Kholifah N. Design and Development
E-Learning System by Learning Management System (LMS) in Vocational
Education // Online Submission. -- 2020. -- Vol.9(1). - P.1059-1063.

13. Juhaňák L., Zounek J., Rohlíková L. Using process mining to analyze
students'{} quiz-taking behavior patterns in a learning
management system // Computers in Human Behavior.- 2019. - Vol.92. - P.
496-506. DOI
\href{https://doi.org/10.1016/j.chb.2017.12.015}{10.1016/j.chb.2017.12.015}.

14. Chick R.C., Clifton G.T., Peace K.M., Propper B.W., Hale D.F.,
Alseidi A.A., Vreeland T.J. Using technology to maintain the education
of residents during the COVID-19 pandemic // Journal of Surgical
Education. - 2020. - Vol.77(4). - P.729-732. DOI
\href{https://doi.org/10.1016/j.jsurg.2020.03.018}{10.1016/j.jsurg.2020.03.018}.

15. Frolova M.A., Razumova T.A. The use of process approach to base the
need of automation of business processes in educational institutions //
AIP Conference Proceedings. - 2017. - Vol.1797(1). - P.
040004-1--040004-8. DOI
\href{https://doi.org/10.1063/1.4972460}{10.1063/1.4972460}.

16. Eriksen S.D. TQM and the transformation from an élite to a mass
system of higher education in the UK//Quality Assurance in
Education.-1995.-Vol.3(1).-P.14--29.DOI
\href{https://doi.org/10.1108/09684889510146795}{10.1108/09684889510146795}.

17. Pham D.T., Jaaron A.A.M. Design for Mass Customisation in Higher
Education: a Systems-Thinking Approach // Systemic Practice and Action
Research. - 2018. -Vol.31. - P.293-310. DOI
\href{https://doi.org/10.1007/s11213-017-9426-7}{10.1007/s11213-017-9426-7}.

18. Dabirian S., Abbaspour S., Khanzadi M., Ahmadi M. Dynamic modelling
of human resource allocation in construction projects // International
Journal of Construction Management. - 2019. - Vol.22(2). - P.182-191.
DOI
\href{https://doi.org/10.1080/15623599.2019.1616411}{10.1080/15623599.2019.1616411}.

19. Brudner A., Gavious A. A dynamic model of investment in research and
teaching facilities in academic institutions // Annals of Operations
Research. - 2024.-Vol.343.-P.67-85. DOI
\href{https://doi.org/10.1007/s10479-024-06232-w}{10.1007/s10479-024-06232-w}.

20. Ly O.R., Boko U.H.S., Gueye K., Ouya S. Proposal of a Dynamic
Resource Allocation Solution for Virtual Classroom Platforms // In: Auer
M., Hortsch H., Sethakul P. (eds) The Impact of the 4th Industrial
Revolution on Engineering Education. ICL 2019. Advances in Intelligent
Systems and Computing. -- Cham: Springer, 2020. -- Vol.1135. - P.
59-68. DOI
\href{https://doi.org/10.1007/978-3-030-40271-6_7}{10.1007/978-3-030-40271-6\_7}.

21. Donyina A. Stochastic Modelling and Simulation of Dynamic Resource
Allocation // In: Ehrig H., Rensink A., Rozenberg G., Schürr A. (eds)
Graph Transformations. ICGT 2010. Lecture Notes in Computer Science. --
Berlin, Heidelberg: Springer.- 2010.-Vol.6372. - P.388-390. DOI
\href{https://doi.org/10.1007/978-3-642-15928-2_28}{10.1007/978-3-642-15928-2\_28}.

22. OpenAI. ChatGPT (GPT-4) Language Model. OpenAI. - 2023. * URL:
\url{https://chat.openai.com}. Intended use: language editing and
grammar checking. \emph{GPT-4} The model was used exclusively to assist
in editing the text and structuring the scientific presentation
(accessed: 06.10.2025).

\emph{{\bfseries Сведения об авторах}}

Масырова И. - магистр, ассистент-профессор, Международный университет
информационных технологий, Алматы, Казахстан, e-mail:
i.massyrova@iitu.edu.kz;

Джолдасбаев С.К. - магистр, ассистент-профессор, Международный
университет информационных технологий, Алматы, Казахстан, e-mail:
s.joldasbayev@iitu.edu.kz;

Гизатуллина А.Р. - cтудент 4-курса, Международный университет
информационных технологий, Алматы, Казахстан, e-mail:
gizatullina.alyana@mail.ru;

Жанадилов М.Ж. - cтудент 4-курса, Международный университет
информационных технологий, Алматы, Казахстан, e-mail:
madiyar20032009@gmail.com;

Джолдасбаев О.К. PhD, старший преподаватель филиала академии
государственного управления при Президенте РК по Алматинской области,
Конаев, Казахстан, e-mail:
orynbassarjoldasbayev@gmail.com;

Орманбекова А.А. -- PhD, Алматинский технологический университет,
Алматы, Казахстан, e-mail:
a.ormanbekova@atu.edu.kz;

Мамбетов С.Т. -- магистр технических наук,Университет «Туран», Алматы,
Казахстан, e-mail:
s.mambetov@turan-edu.kz;

\emph{{\bfseries Information about the authors}}

Massyrova I. - Master, Assistant Professor, International IT University,
Almaty, Kazakhstan, e-mail:
i.massyrova@iitu.edu.kz;

Joldasbayev S. - MSc, Assistant Professor, International IT University,
Almaty, Kazakhstan, e-mail:
s.joldasbayev@iitu.edu.kz;

Gizatullina A. - Student, International IT University, Almaty,
Kazakhstan, e-mail:
gizatullina.alyana@mail.ru;

Zhanadilov M. - Student, International IT University, Almaty,
Kazakhstan, e-mail:
madiyar20032009@gmail.com;

Joldasbayev O. - PhD, Senior Lecturer of the branch of the Academy of
Public Administration under the president of the Republic of Kazakhstan
in Almaty region, Konayev, Kazakhstan, e-mail:
orynbassarjoldasbayev@gmail.com;

Ormanbekova A. - PhD, Almaty Technological University, Almaty,
Kazakhstan, e-mail:
a.ormanbekova@atu.edu.kz;

Mambetov S.- Master of Technical Sciences, , Turan University, Almaty,
Kazakhstan, e-mail:
s.mambetov@turan-edu.kz.\
