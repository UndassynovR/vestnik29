\id{МРНТИ 20.23.17}{}

\begin{header}
\swa{}{METHODS OF MODELING PHYSICAL PROCESSES IN MATHCAD}

G.B. Turebaeva\envelope,
Zh.B. Doshakova
\end{header}

\begin{affil}
Karaganda Technical University named after Abylkas Saginov, Karaganda, Kazakhstan

\corrauthor{Corresponding author: gulnara\_83.06.12@mail.ru}
\end{affil}

In the context of modern educational realities, characterized by strict
criteria for assessing students'{} knowledge, high
academic workload, reduced hours of study, as well as outdated and
insufficient equip\-ment, the problem of improving the quality of
education is becoming particularly acute. In this regard, the training
of specialists within the framework of the credit education system
highlights the need to develop new conceptual approaches to the
organization of educational and methodological activities and the active
introduction of modern information and communication technologies into
the educational process. This implies the widespread use of information
technology and personal computers for modelling a variety of physical
phenomena, both in the framework of training and in the implementation
of ongoing knowledge control. This paper demonstrates that the use of
modern applied software packages in the educational process can
significantly transform the methodology of studying certain sections of
the physics course, providing visual visualization of the results of
solving problems using these applied tools.

For a clearer demonstration of physical phenomena, this article presents
modelling of complex systems described by second-order differential
equations using the MathCad software environment. Using the odesolve
function built into Mathcad, a graph of the addition of two harmonic
oscillations was obtained. The method of solving differential equations
presented in this paper can be used to model various other physical
processes considered in the physics course. The main purpose of this
work is to master the methods of modelling non-stationary physical
processes in the Mathcad environment using the odesolve function, using
the example of oscillatory movements. In addition, the article
highlights the importance of using physical models in the educational
process. This allows not only to visualize complex processes, but also
contributes to a deeper understanding of the principles underlying
physical phenomena.

{\bfseries Keywords}: odesolve function, physical processes and models,
complex systems, ordinary differential equations of the first and second
orders, addition of two harmonic oscillations, modeling of physical
processes.

\begin{header}
MATHCAD-ТАҒЫ ФИЗИКАЛЫҚ ПРОЦЕСТЕРДІ МОДЕЛЬДЕУ ӘДІСТЕРІ

Г.Б. Туребаева\envelope,
Ж.Б. Дошакова
\end{header}

\begin{affil}
Карағанды техникалық университеті Абылқас Сағынов атындағы, Қарағанды, Қазақстан,

e-mail: gulnara\_83.06.12@mail.ru
\end{affil}

Студенттердің білімін бағалаудың қатаң критерийлерімен, жоғары оқу
жүктемесімен, оқу сағаттарының қысқартылған көлемімен, сондай-ақ
ескірген және жеткіліксіз жабдықтармен сипатталатын қазіргі білім беру
шындықтары жағдайында білім беру сапасын арттыру мәселесі ерекше өткір
болып отыр. Осыған байланысты, кредиттік оқыту жүйесі шеңберінде
мамандарды даярлау Оқу-әдістемелік қызметті ұйымдастырудың жаңа
тұжырымдамалық тәсілдерін әзірлеу және білім беру процесіне заманауи
ақпараттық-коммуникациялық технологияларды белсенді енгізу қажеттілігін
бірінші орынға қояды. Бұл оқыту шеңберінде де, білімнің ағымдағы
бақылауын жүзеге асыру кезінде де әртүрлі физикалық құбылыстарды
модельдеу үшін ақпараттық технологиялар мен дербес компьютерлерді
кеңінен қолдануды білдіреді. Бұл жұмыста оқу процесінде заманауи
қолданбалы бағдарламалық пакеттерді қолдану физика курсының кейбір
бөлімдерін зерттеу әдістемесін айтарлықтай өзгертуге мүмкіндік
беретіндігі, осы қолданбалы құралдарды қолдана отырып, есептерді шешу
нәтижелерін көрнекі түрде визуализациялауды қамтамасыз ететіндігі
көрсетілген.

Физикалық құбылыстарды неғұрлым нақты көрсету үшін бұл мақалада Mathcad
бағдарламалық ортасын қолдана отырып, екінші ретті дифференциалдық
теңдеулермен сипатталған күрделі жүйелерді модельдеу ұсынылған.
Mathcad-қа енгізілген odesolve функциясының көмегімен екі гармоникалық
тербелісті қосу графигі алынды. Жұмыста ұсынылған дифференциалдық
теңдеулерді шешу әдісі физика курсында қарастырылатын әртүрлі басқа
физикалық процестерді модельдеу үшін пайдаланылуы мүмкін. Бұл жұмыстың
негізгі мақсаты тербелмелі қозғалыстар мысалында odesolve функциясын
қолдана отырып, Mathcad ортасындағы стационарлық емес физикалық
процестерді модельдеу әдістерін игеру болып табылады. Сонымен қатар,
мақалада білім беру процесінде физикалық модельдерді қолданудың
маңыздылығы көрсетілген. Бұл күрделі процестерді елестетуге ғана емес,
сонымен қатар физикалық құбылыстардың негізінде жатқан принциптерді
тереңірек түсінуге мүмкіндік береді.

{\bfseries Түйінді сөздер:} odesolve функциясы, физикалық процестер мен
модельдер, күрделі жүйелер, бірінші және екінші ретті қарапайым
дифференциалдық теңдеулер, екі гармоникалық тербелісті қосу, физикалық
процестерді модельдеу.

\begin{header}
МЕТОДЫ МОДЕЛИРОВАНИЯ ФИЗИЧЕСКИХ ПРОЦЕССОВ В MATHCAD

Г.Б. Туребаева\envelope,
Ж.Б. Дошакова
\end{header}

\begin{affil}
Карагандинский технический университет им. Абылкаса Сагинова, Караганда, Казахстан,

e-mail: gulnara\_83.06.12@mail.ru
\end{affil}

В условиях современных образовательных реалий, характеризующихся
строгими критериями оценки знаний студентов, высокой учебной нагрузкой,
сокращенным объемом учебных часов, а также устаревшим и недостаточным
оборудованием, проблема повышения качества образования становится
особенно острой. В связи с этим, подготовка специалистов в рамках
кредитной системы обучения выдвигает на первый план необходимость
разработки новых концептуальных подходов к организации
учебно-методической деятельности и активного внедрения современных
информационно-коммуникационных технологий в образовательный процесс. Это
подразумевает широкое применение информационных технологий и
персональных компьютеров для моделирования разнообразных физических
явлений, как в рамках обучения, так и при осуществлении текущего
контроля знаний. В настоящей работе демонстрируется, что применение
современных прикладных программных пакетов в учебном процессе позволяет
существенно трансформировать методику изучения некоторых разделов курса
физики, обеспечивая наглядную визуализацию результатов решения задач с
использованием этих прикладных средств.

Для более четкой демонстрации физических явлений, в данной статье
представлено моделирование сложных систем, описываемых дифференциальными
уравнениями второго порядка, с применением программной среды MathCad.
При помощи функции odesolve, встроенной в Mathcad, был получен график
сложения двух гармонических колебаний. Представленный в работе способ
решения дифференциальных уравнений может быть использован для
моделирования различных других физических процессов, рассматриваемых в
курсе физики. Основная цель этой работы заключается в освоении методов
моделирования нестационарных физических процессов в среде Mathcad с
использованием функции odesolve, на примере колебательных движений.
Кроме того, в статье подчеркивается важность применения физических
моделей в образовательном процессе. Это позволяет не только
визуализировать сложные процессы, но и способствует более глубокому
пониманию принципов, лежащих в основе физических явлений.

{\bfseries Ключевые слова:~}функция odesolve, физические процессы и модели,
сложные системы, обыкновенные дифференциальные уравнения первого и
второго порядков, сложения двух гармонических колебаний, моделирование
физических процессов.

\begin{multicols}{2}
{\bfseries Introduction}. Modern requirements for the training of qualified
personnel using credit technology used in higher education dictate the
need to create innovative approaches to the organization of educational
activities and the integration of modern information and communication
tools into the educational process. This implies the active use of
digital technologies and personal computers for modelling various
physical phenomena both in the framework of training and for conducting
ongoing knowledge control. The use of computer technology stimulates
students'{} interest in the discipline being studied,
simplifies and accelerates the process of mastering new knowledge and
verifying their assimilation, which ultimately contributes to improving
the quality of education and a deeper understanding of the material.
Replacing traditional lecture demonstrations with their digital
counterparts makes it possible to significantly optimize the time spent
on their presentation and explanation.

Practice shows that introducing students to computer demonstrations
before starting to study a topic, as a preliminary stage, is more
effective. The clarity, complexity and interactivity inherent in these
programs make them indispensable assistants for both students and
teachers. The variety of functions offered by the programs makes the
learning process more exciting and productive. In this regard, the study
of methods for modelling physical processes using modern application
software packages is becoming particularly relevant at the present time
{[}1{]}.

In the credit-modular learning system, an electronic summary becomes one
of the options for using modern information technologies in lectures. It
differs from an electronic textbook and is developed mainly for the
teacher, taking into account his individual teaching style. A multimedia
presentation is a key tool in the preparation of an electronic summary.
And for a clearer demonstration of certain processes using mathematical
modelling, programs such as Mathcad, Matlab and other analogues are
used.

As part of our approach, we opt for Mathcad when considering key
sections of the theoretical part of the course, as well as materials
from practical and laboratory work.

Training complexes developed using digital technologies are a
specialized type of technical training tools. Their goal is to simplify
the teacher' s work, eliminate monotonous operations, and
ensure a high level of learning and skill development.

The use of computers is inextricably linked with the solution of many
issues related to the improvement of physical education. Automated
learning systems can be used as a supplement to lectures and for a more
detailed explanation of the material, for the implementation of ongoing
knowledge control at seminars, as well as for the purpose of automating
laboratory workshops {[}2{]}.

For some areas of training, laboratory practice is a key form of
education. The main objective of the workshop is to empirically test the
theoretical concepts of the discipline under study, to create a clear
understanding of the basic principles and ways of their implementation
among students, to develop a scientific worldview among future
professionals, as well as to form practical skills and experimentation
skills.

The realization of creative abilities and professional growth are
achieved to the maximum extent through the practical application of
acquired knowledge. Laboratory work helps students understand the
relationship between theory and practice, demonstrates current trends in
the development of experimental science and stimulates interest in
independent scientific research and a creative approach. Computer-based
learning systems are widely used at all stages of laboratory activities,
including experiment planning, data processing and analysis, and the
design of research results. In cases where the computer is not the
direct object of study, its function is to support ongoing activities
{[}3{]}.

A program that simulates physical experience should be perceived as an
integrated part of an interconnected set of educational tools. Computer
visualization tools allow you to transform the results into dynamic and
understandable images, displaying the dependence of the solution on the
set parameters on a graphical screen. This brings the computational
experiment closer to real conditions. Interacting with such a model is
fascinating and helps students intuitively understand the essence of
fundamental physical equations, developing their scientific intuition.
It is important to note that numerical experiment makes it possible to
predict previously unknown phenomena and analyse systems that are
inaccessible for physical reproduction {[}4{]}.

{\bfseries Materials and methods}. The creation of computer laboratory work
dates back to the 90s of the last century, which was made possible by
the availability of microcomputers, the development of conversational
interfaces, machine graphics and animation technologies. The basis of
such educational projects were the methods of machine modelling, which
radically changed the approach to physics and gave impetus to the
emergence of a new scientific field - computational physics. "Virtual"
laboratory work is educational computer experiments that adequately
complement traditional physical experiments conducted in real conditions
{[}5{]}.

The laboratory workshop is aimed at in-depth mastering of theoretical
material, familiarization with methods of measuring various physical
quantities and development of students'{} skills of
independent experimental activity. Laboratory experiments are active and
relatively autonomous tasks: after studying the theoretical part, the
student, under the guidance of a teacher, takes measurements of
specified physical quantities, analyses the data obtained, builds graphs
and works with them, and then independently formulates conclusions based
on the work performed. Thus, laboratory practice plays an important role
in the development of research skills among students {[}6{]}.

There are many examples of phenomena and processes that can be used to
create models with varying degrees of realism, covering all areas of
physics. An important advantage is that the use of computer modelling in
the study of various branches of physics opens up huge potential
opportunities by providing modern technologies and specialized programs.
These tools not only expand the horizons of research, but also
significantly increase them.

Modern approaches to the preparation of students in technical
disciplines indicate the importance of teaching not only the basic
materials and techniques of work in the specialty, but also the use of
various methods of analysis and modelling. However, given the limited
time of classes and the physical condition of laboratory equipment, it
is necessary to reconsider traditional ideas about the pedagogy of
natural science disciplines. This is especially true of fundamental
disciplines such as physics, which is central to the development of the
engineering approach.

The need to use application software in the learning process is
associated with trends towards increasing various requirements for the
clarity and effectiveness of the disclosed material, as well as the need
for effective preparation of students to work with technology. This
article describes how the new Mathcad mathematical modelling system can
significantly improve the study of physical processes.

For example, modelling of forced oscillations without taking into
account resistance under the influence of an external periodic force is
shown. The application of /odesolve for solving second-order ordinary
differential equations is demonstrated. The constructed graphs make it
possible not only to reveal deeper and more meaningful aspects of the
very nature of the process under consideration, but also to study its
behaviour in comparison with the given initial conditions. These methods
are also important for developing students'{} skills in
interpreting mathematical models of studied physical phenomena.

The Mathcad program creates other opportunities for teachers: it becomes
possible to introduce dynamic models into educational materials, which
makes it possible to replace some traditional calculations using
computers and facilitate the study of complex systems. At the same time,
the use of such approaches leads to the formation of engineering
thinking among students, which is necessary for further professional
activity.

In addition, it is worth emphasizing that the main purpose of this work
is to explain that the use of Mathcad helps to solve a specific problem
from a physics course, as well as to prove the importance of digital
technologies in the educational process. The modelling approach used can
be applied to the study of other physical systems, such as electrical
circuits, heat transfer, and wave phenomena.

In our opinion, the Mathcad package is the most adapted for the
educational process, as it includes a text editor, a computing module,
and a graphics processor. Mathcad is a universal system that can be used
in various fields of science and technology where mathematical methods
are used. Writing commands in Mathcad in a language close to standard
mathematical formulas makes it much easier to set and solve problems
{[}7{]}.

Mathcad implements functionality for finding solutions to partial
differential equations, as well as for solving systems of such
equations. The Mathcad toolkit makes it possible to find solutions to
parabolic and hyperbolic equations in one dimension (using one spatial
and one temporal variable). Despite the apparent limitations of this set
of tasks, it covers a significant part of the problems that arise in the
physical and engineering fields.

In Mathcad, to solve ordinary differential equations using numerical
methods, you can choose between the Given/odesolve solution block and
built-in functions, for example, rkfixed, which were used in previous
versions. The use of the Given/odesolve solution block is usually
considered preferable due to the clarity of the problem formulation and
presentation of the results obtained. At the same time, the built-in
functions provide the user with more control over the settings of the
numerical solution.

Mathcad implements a specialized built-in function designed for
numerical solution of differential equations. It has the following
syntax: Odesolve (x, b{[} , steps {]} ).

To solve the Cauchy problem, it is necessary to set the initial
conditions and determine the endpoint of the integration interval. These
parameters, along with the equation, are formatted in the Given block,
after which the odesolve function is called. It is important to consider
a number of nuances when using it. If the number of step steps is set,
the calculations are performed with a constant integration step.
Otherwise, an adaptive step selection method is used.

The functional scheme of Odesolve(x, b, {[}step{]}) looks like this: it
forms a function that is the result of solving a differential equation.
This function is intended to be used inside a block starting with the
Given statement.

Here:

- x is the independent variable over which the integration takes place,
and it must be represented by a real number.

- b is the value of the endpoint of the interval where the integration
is performed.

- step is an optional parameter that defines the increment step size of
the integration variable.

MathCAD has the ability to solve differential equations directly
numerically, where the highest derivative is expressed explicitly (that
is, the equation has the form a).

\[\frac{d^2}{dt^2}x(t)-sin(\frac{d}{dt}x(t))=-x(t)\]

To solve a differential equation, you can use the odesolve function and
the specialized Given-Odesolve block. This block includes:

1. The Given keyword, which initiates the decision process.

2. The differential equation itself, written in the usual mathematical
form, but with some changes:

a) the logical equality operator is used to denote equality (entered by
pressing Ctrl - =);
\end{multicols}

b) the integrated function is always specified with an explicit argument
(for example, x(t), and not just x); c) derivatives can be denoted using
the standard operators
\fig{i3/image130}{}\fig{i3/image131}{}
and, or using the derivative symbols obtained by pressing Ctrl-F7 (for
example, x'(t), x'\,'(t)).

3. Setting the initial or boundary conditions for the integrable
function and its derivatives (except for the highest one). These values
are also set using the logical equality operator. The number of set
values must correspond to the order of the equation. For example, for a
second-order
equation\fig{i3/image132}{}
it is necessary to specify the initial values of the function and its
first derivative, for example: x(0) = 1; x'(0) = 0.5. The
derivative symbol «'» is entered using the keyboard shortcut Ctrl-F7.

The Odesolve function. The required first parameter is the name of the
independent variable. The second required parameter specifies the final
value of this variable. The optional third parameter specifies the
number of steps used for numerical integration. Odesolve returns a
function as a result of its work, which is a numerical approximation of
the solution of a differential equation in a given range of values of an
independent variable. The resulting function can be used to calculate
the values of the desired function at any point in the interval, as well
as to visualize the solution as a graph.

Let' s consider an example: let' s find a
solution to the previously considered differential equation in the range
of values of the variable t from 0 to 5. Let' s determine
the values of x for t equal to 2 and 4, and also plot the resulting
solution (see Fig.1) {[}8{]}.

\emph{{\bfseries Listing 1.}} Solving a second-order equation for values of
t=0..5

\fig{i3/image133}{}

{\bfseries Fig.1 - Graph of the second-order equation for values t=0..5}

\emph{{\bfseries Listing 2.}} The solution of this differential equation
b(a) and the derivative of the solution function is c(a)\ul{\hfill\break
}\fig{i3/image134}{}

\fig{i3/image135}{}
The differential equation is given

\fig{i3/image136}{}
Initial conditions are set

\fig{i3/image137}{}
The solution of the differential equation is given

\fig{i3/image138}{}
Calculating the derivative of b(a)

\fig{i3/image139}{}

{\bfseries Fig.2 - The graph of the solution of a given differential
equation b(a) and the derivative of the solution function - c(a)}

\emph{{\bfseries Remarks:}}

1. The equation must be linear with respect to the highest derivative.

2. The number of specified initial or boundary conditions within the
block must be equal to the order of the equation.

3. When writing an equation to indicate the derivatives of a function,
use the special buttons from the Math panel or '{}
(stroke) - {[}Ctrl+F7{]}, for the equal sign = {[}Ctrl+={]} (including
for additional conditions).

4. The end point must be larger than the start point.

5. Initial and boundary conditions of mixed type are not allowed
(f'(a)+f(a)=5).

6. The desired function in the block must necessarily have an argument
(f(x)).

Based on the above illustrations, this article demonstrates various
approaches to solving ordinary differential equations used to model
oscillatory phenomena in physics. In particular, using the odesolve
function in the Mathcad program, a graph of the addition of two harmonic
oscillations was obtained. This process is described by equation
\(y\text{''} + \text{qy} = a\text{sinwt}\), which is studied in the
section of the physics course ``Vibrations and waves".

\emph{{\bfseries Listing 3.}} Solution of the Cauchy problem for
second-order ordinary differential equations

\fig{i3/image140}{}

\fig{i3/image141}{}

{\bfseries Fig.3 - The graph obtained by adding two harmonic oscillations}

\begin{multicols}{2}
The novelty of the work is that the considered methods for solving
differential equations differ from those previously proposed in that it
is focused on modelling physical processes, which is characterized by a
simple methodology and is suitable for use by students without special
training.

{\bfseries Results and discussion.} The presented methods for solving
differential equations in Mathcad can be used to model oscillatory
physical processes, as well as an example of solving other physical
problems. As a result of these modeling works, the user receives a
ready-made model of the system and can only arbitrarily set the initial
conditions and control all the parameters of the model during the
numerical experiment. The resulting nonlinear model is widely used in
the study of branches of physics: mechanics, electricity and magnetism,
vibrations and waves. It can also be used during lectures, practical,
laboratory classes, as an element of information technology. Why was
Mathcad chosen because it has a simple methodology and is suitable for
students to use without special training. {[}9{]}

This development aimed to deepen students'{}
understanding of fundamental physical principles, their interdependence
and logical structure; to facilitate the understanding of the
relationship between different physical quantities, to compare the real
behaviour of an object, analytical expressions and their visual
representation. An interactive platform is being created for the user,
allowing them to freely experiment with mathematical models describing
physical objects, phenomena, and processes.

Students are able to use simple object models as building blocks to
develop complex systems. Instead of simply performing laboratory work
according to ready-made instructions, they can create new circuits using
existing models, and even adapt these models to their needs. By
understanding the principles of interaction of elements, establishing
information links between them and analysing the
system' s response to external influences, the user
learns how to manage complex systems, organically combining the study of
physics with computer science. At the same time, what is especially
important, computer science becomes a tool for students with obvious
practical applications.

Modern software packages stand out because they allow you to enter
mathematical expressions or functions to perform numerical calculations,
set various values for the variables used, create graphical
representations of simulation results, generate random numbers (to
simulate random processes), perform logical operations, which makes it
possible to apply a variety of numerical methods. When using Mathcad,
the student is freed from the need to write code for a computational
algorithm and create auxiliary program blocks. Mathcad significantly
reduces the amount of monotonous computing work. Mastering Mathcad is
easy and fast, without requiring in-depth study of voluminous manuals,
writing notes, or memorizing complex rules for study and practical
application. The simplicity of Mathcad lies in the fact that you can get
a solution to the problem in a short time {[}10{]}.

The integration of modern applied software tools into educational
practice opens up new horizons for teaching physics and computer
science. This allows us to review traditional approaches to the study of
topics involving large, routine calculations, solving differential
equations, graphical representation of data and visualization of
results. Thanks to the software packages, complex tasks that were
previously available only through analytical calculations can now be
solved using numerical computer modelling methods that offer significant
advantages. This approach allows not only to simplify the learning
process, but also to make it more visual and understandable for
students. The use of numerical methods expands the boundaries of
cognition, allowing us to study physical systems in dynamics and from
various points of view, which was inaccessible with an exclusively
analytical approach. Thus, modern application packages are becoming an
indispensable tool in education, increasing the effectiveness of
learning and stimulating interest in the study of physics and computer
science. {[}11{]}.

Currently, computer modeling and computational experiments are a
sought-after approach to the study of physical processes. This method is
characterized by unique features, advantages, and limitations that
distinguish it from other methods of analyzing physical systems. It is
important that university students have knowledge of computer models,
numerical methods used to study various objects, and confidently use
modern software. Thanks to modern application programs, complex systems
of equations are solved instantly, graphs of the studied dependencies
are plotted, and experiments that are difficult to reproduce in reality
are modelled.

{\bfseries Conclusion.} Consequently, the use of applied software in the
educational process makes it possible to draw conclusions based on
theoretical principles, compare them with experimental data and make
changes to the original model. To display the real system and acquire
skills in the field of modeling, an algorithm was created and a program
for adding two harmonic vibrations was written. In addition, the user
can engage in design and research work, which makes it possible to
coordinate the creation, testing and improvement of innovative systems,
for example, technical devices, devices and mechanisms. In this context,
the Mathcad software package is a valuable learning tool, allowing you
to teach various subjects (numerical methods, computer modeling of
physical phenomena, mathematical modeling, physics, and others) at a
more in-depth level {[}12{]}.
\end{multicols}

\begin{center}
{\bfseries Литература}
\end{center}

\begin{refs}
1. Дьяконов В. П. Энциклопедия компьютерной алгебры /ДМК Пресс. -2016.
-1264 с. ISBN 978-5-94074-490-0.

2. Голанова А. В., Голикова Е. И. Готовность преподавателя к
использованию компьютерных математических систем в образовательном
процессе // Вестник Череповецкого государственного университета. -2019.
-Т.1 (88). -С.144-153. DOI 10.23859/1994-0637-2019-1-88-14.

3. Волков Е. А. Численные методы/Санкт-Петербург: Лань. -2004. -248 с.
ISBN 5‑8114‑0538‑3.

4. Isrokatun I., Haryani~C.S., Rahmi N.I. Analysis of mathematical
problem-posing ability// Journal of Physics: Conference Series. -2021.
-Т.1869(1): 012122. DOI 10.1088/1742-6596/1869/1/012122.

5. Воскобойников В. Е., Задорожный А. В. Основы вычислений и
программирования в пакете MathCAD PRIME: учебное пособие/Лань. - 2016. -
224 с. ISBN 978‑5‑8114‑2052‑0.

6. \href{https://www.amazon.com/David-Randolph-Martin-II/e/B0745YJCLP/ref=dp_byline_cont_book_1}{Randolph
Martin II} D. Engineering Calculations with Creo Parametric and PTC
Mathcad Prime (Creo Power Users)/ Independently published. - 2020. -210
p. ISBN 979-8649196673.

7. Воскобойников Ю. Е., Задорожный А. Ф. Изучение и программирование в
пакете Mathcadprime 2.0: учебное пособие.- метод. пособие для студентов/
Новосибирск: НГАСУ (Сибстрин). - 2021. - 224 с.

8. Maxfield B. An Introduction to PTC® Mathcad Prime® 3.0.// Academic
Press. -2014. -P.3-34. DOI 10.1016/B978-0-12-410410-5.00001-5.

9. Кондратьев А. С., Ляпцев А.В. Физика. Задачи на компьютере /
Физматлит. - 2008. - 400 с. ISBN 978-5-9221-0917-8.

10. Левицкий А.А. Информатика. Основы численных методов: Лабораторный
практикум/Красноярск: ИПЦ КГТУ. -2005. -111 с. ISBN 5-7636-0745-7.

11. Мукушев Б.А., Нурбакова Г.С., Жаугашева С.А., Исимов Н.Т.Бейсызық
математикалық маятниктің тербелісін MathCAD қолданбалы программалар
пакеті көмегімен зерттеу // ҚазҰТЗУ Хабаршысы. -2016. - №3 (115). -С.
592-597.

12. Мукушев Б.А., Нурбакова Г.С., Исимов Н.Т. Математикалық маятник
тербелісін MathCad пакеті көмегімен зерттеу // ҚазҰТЗУ Хабаршысы -2016.
-№6 (118). -С.175-179.
\end{refs}

\begin{center}
{\bfseries References}
\end{center}

\begin{refs}
1. D' yakonov V. P. Enciklopediya
komp' yuternoj algebry/DMK Press. - 2016. - 1264 s. ISBN
978-5-94074-490-0. {[}in Russian{]}

2. Golanova A. V., Golikova E. I. Gotovnost'{}
prepodavatelya k ispol' zovaniyu
komp' yuternyh matematicheskih sistem v
obrazovatel' nom processe // Vestnik CHerepoveckogo
gosudarstvennogo universiteta. -2019. -T.1 (88). -S.144-153. DOI
10.23859/1994-0637-2019-1-88-14. {[}in Russian{]}

3. Volkov E. A. CHislennye metody/Sankt-Peterburg: Lan'.
- 2004. - 248 s. ISBN 5811405383. {[}in Russian{]}

4. Isrokatun I., Haryani~C.S., Rahmi N.I. Analysis of mathematical
problem-posing ability// Journal of Physics: Conference Series. -2021.
-Т.1869(1): 012122. DOI 10.1088/1742-6596/1869/1/012122.

5. Voskobojnikov V. E., Zadorozhnyj A. V. Osnovy vychisleniy i
programmirovaniya v pakete MathCAD PRIME: uchebnoe posobie/
Lan'. -2016. -224 s. ISBN 97858114 20520. {[}in
Russian{]}

6. \href{https://www.amazon.com/David-Randolph-Martin-II/e/B0745YJCLP/ref=dp_byline_cont_book_1}{Randolph
Martin II} D. Engineering Calculations with Creo Parametric and PTC
Mathcad Prime (Creo Power Users)/ Independently published. -2020. -210
p. ISBN 979-8649196673.

7. Voskobojnikov YU. E., Zadorozhnyj A. F. Izuchenie i programmirovanie
v pakete Mathcadprime 2.0: uchebnoe posobie.- metod. posobie dlya
studentov/ Novosibirsk: NGASU (Sibstrin). -2013. -224 s. ISBN
9785811420520. {[}in Russian{]}

8. Maxfield B. An Introduction to PTC® Mathcad Prime® 3.0.// Academic
Press. -2014. -P.3-34. DOI 10.1016/B978-0-12-410410-5.00001-5. .

9. Kondrat' ev A.S., Lyaptsev A.V. Fizika. zadachi na
komp' yutere / Fizmatlit. -2015. -105 s. ISBN
978-5-9221-0917-8. {[}in Russian{]}

10. Levickij A.A. Informatika. Osnovy chislennyh metodov: Laboratornyj
praktikum/Krasnojarsk: IPC KGTU. -- 2005. - 111 s. ISBN
5-7636-0745-7.{[}in Russian{]}

11. Mukushev B.A., Nurbakova G.S. Zhaugasheva S.A., Isimov N.T.Beisyzyk
matematikalyk mayatniktіn terbelіsіn MathCAD koldanbaly programmalar
paketі komegіmen zertteu // KaZUTZU Khabarshysy. -2016. -№3 (115). -S.
592-597.

12. Mukushev B.A., Nurbakova G.S. Isimov N.T. Matematikalyқ mayatnik
terbelіsіn MathCad paketі kөmegіmen zertteu // ҚaZҰTZU Khabarshysy -
2016. - №6 (118). --S.175-179.
\end{refs}

\begin{info}
\hspace{1em}\emph{{\bfseries Information about the authors}}

Turebaeva G.B. - Master of Physics, Senior Lecturer, Abylkas Saginov
Karaganda Technical University, Karaganda, Kazakhstan, e-mail:
gulnara\_83.06.12@mail.ru;

Doshakova Zh. B. - Master of Physics, Abylkas Saginov Karaganda
Technical University, Karaganda, Kazakhstan, e-mail:
m29kt@mail.ru.

\hspace{1em}\emph{{\bfseries Сведения об авторах}}

Туребаева Г.Б. - магистр физики, КарТУ имени Абылкаса Сагинова,
Караганда, Казахстан, e- mail: gulnara\_83.06.12@mail.ru;

Дошакова Ж.Б.- магистр физики, КарТУ имени Абылкаса Сагинова,
Караганда, Казахстан, e- mail: m29kt@mail.ru.
\end{info}
