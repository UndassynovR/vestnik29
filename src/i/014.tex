\id{МРНТИ 20.23.25}{}

\begin{header}
\swa{}{ПРОЕКТИРОВАНИЕ МОДЕЛИ ПИД-РЕГУЛЯТОРА ДЛЯ НЕЛИНЕЙНОЙ СИСТЕМЫ}

\tsp{1}Г.Б. Бахадирова,
\tsp{2}Н. Тасболатұлы\envelope,
\tsp{2}Д.Б. Баумуратова,
\tsp{1}А.С. Аканова,
\tsp{3}Ә.Қ. Әлімжан
\end{header}

\begin{affil}
\tsp{1}Казахский агротехнический исследовательский университет им. С. Сейфуллина, Астана, Казахстан,

\tsp{2}Международный университет Астана, Астана, Казахстан,

\tsp{3}Astana IT University, Астана, Казахстан

\corrauthor{Корреспондент-автор: tasbolatuly@gmail.com}
\end{affil}

В данной статье подробно рассматриваются теория ПИД-контроля,
современные методы настройки регуляторов, а также подход к моделированию
ПИД-регулятор для нелинейных систем в среде MATLAB/Simulink и с
использованием языка программирования Python. В качестве примера
представлено компьютерное моделирование ПИД-регулятор на Python для
нелинейной системы, основанной на модели химического реактора с
охлаждающей оболочкой. С целью проверки достоверности результатов
моделирования были выполнены графические симуляции примера в средах
Simulink и Python. В перспективе полученные результаты могут служить
справочным материалом для более глубокого понимания характеристик
ПИД-регуляторов, особенно применительно к нелинейным системам.

В статье также представлена разработка модели ПИД-регулятора для
управления работой системы химического реактора. С целью анализа
динамики нелинейных систем предложено прикладное программное обеспечение
с графическим интерфейсом, реализующее процесс моделирования. В
программе результаты моделирования - изменения концентрации, температур
реактора и охлаждающей оболочки, а также расхода охлаждающей жидкости -
визуализируются в виде четырёх отдельных графиков. Воздействие
параметров ПИД-регулятора, введённых пользователем, демонстрируется
посредством наглядной анимации. Инструмент оснащён функциями изменения
параметров системы, очистки результатов и возврата к исходным данным.
Проведённое в среде Python моделирование подтвердило корректность и
достоверность полученных результатов. Представленная разработка может
эффективно использоваться в образовательных целях для визуального
освоения принципов управления.

{\bfseries Ключевые слова:} нелинейные системы, контроллер, коэффициент
усиления, замкнутый цикл, обратная связь.

\begin{header}
СЫЗЫҚТЫҚ ЕМЕС ЖҮЙЕГЕ АРНАЛҒАН ПИД-РЕТТЕГІШІНІҢ МОДЕЛІН ЖОБАЛАУ

\tsp{1}Г.Б. Бахадирова,
\tsp{2}Н. Тасболатұлы\envelope,
\tsp{2}Д.Б. Баумуратова,
\tsp{1}А.С. Аканова,
\tsp{3}Ә.Қ. Әлімжан
\end{header}

\begin{affil}
\tsp{1}С. Сейфуллин атындағы Қазақ агротехникалық зерттеу университеті, Астана, Қазақстан,

\tsp{2}Астана халықаралық университеті, Астана, Қазақстан,

\tsp{3}Astana IT University, Астана, Казақстан,

\envelope e-mail: tasbolatuly@gmail.com
\end{affil}

Бұл мақалада MATLAB/Simulink, \emph{Python} бағдарламалау тілдерінде
сызықтық емес жүйелерді басқару үшін ПИД контроллерінің моделін жобалау
әдісі ПИД теориясы, ПИД-реттегіштерін баптаудың заманауи әдістері
егжей-тегжейлі қарастырылады. Сонымен қатар, химиялық реактордың
салқындатқыш қаптама моделіне негізделген сызықтық емес жүйе үшін Python
тілінде компьютерлік модельдеу арқылы ПИД-реттегіш мысалы келтірілген.
Модельдеу нәтижелерінің дұрыстығына көз жеткізу мақсатында Simulink,
\emph{Python} ортасында мысалды график түріндегі модельдеу жүзеге
асырылды. Болашақта осы зерттеу нәтижелері ПИД-реттегіштердің, әсіресе
сызықтық емес жүйелер үшін, сипаттамаларын түсінуге арналған анықтамалық
ретінде пайдаланылуы мүмкін.

Сонымен қатар мақалада химиялық реактор жүйесінің жұмысын басқаруға
арналған ПИД-рет\-тегіштері моделінің жобалануы қарастырылады. Сызықтық
емес жүйелердің динамикасын зерттеу мақсатында графикалық интерфейс
арқылы модельдеуді жүзеге асыратын қолданбалы бағдарлама ұсынылған.
Бағдарламада модельдеу нәтижелері - концентрацияның, реактор мен
салқындатқыш қаптаманың температурасының және салқындатушы сұйықтық
ағынының уақыт бойынша өзгерісі -- төрт түрлі график түрінде
бейнеленген. Пайдаланушы енгізген ПИД параметрлерінің әсері нақты
визуализация арқылы көрсетіледі. Құралда жүйе параметрлерін өзгерту,
нәтижелерді тазарту және бастапқы мәндерге қайта оралу функцияларымен
жабдықталған. Жасалған модельдеу \emph{Python} ортасында енгізілген
мәндер негізінде алынған нәтижелері көрсетіліп, оның дұрыстығы мен
сенімділігі тексерілген. Бұл модельдеуді басқару жүйелерін басқаруды
оқытуда тиімді визуалды құрал ретінде қолдануға болады.

{\bfseries Түйін сөздер:} сызықтық емес жүйелер, реттегіш, күшею
коэффициенті, тұйық цикл, кері байланыс.

\begin{header}
MODELING AND DESIGN OF A PID-CONTROLLER FOR NONLINEAR SYSTEMS

\tsp{1}G.B. Bakhadirova,
\tsp{2}H. Tasbolatuly\envelope,
\tsp{2}D.B. Baumuratova,
\tsp{1}A.S. Akanova,
\tsp{3}A. Alimzhan
\end{header}

\begin{affil}
\tsp{1}S. Seifullin Kazakh agrotechnical research university, Astana, Kazakhstan,

\tsp{2}Astana International University, Astana, Kazakhstan,

\tsp{3}Astana IT University, Astana, Kazakhstan,

\envelope e-mail: tasbolatuly@gmail.com
\end{affil}

This paper presents a method for designing a PID-controller model for
the control of nonlinear systems using MATLAB/Simulink and Python
programming environments. It provides a detailed overview of PID-control
theory and modern tuning techniques for PID-controllers. Additionally, a
practical example is presented through computer-based simulation in
Python, applied to a nonlinear system modeled on a chemical
reactor' s cooling jacket. To verify the accuracy of the
modeling results, graphical simulations were conducted in both Simulink
and Python environments. In the future, the outcomes of this research
may serve as a reference for understanding the characteristics of
PID-controllers, particularly in the context of nonlinear
systems.

In addition, the article presents the design of a PID-controller model
for the control of a chemical reactor system. To study the dynamics of
nonlinear systems, an application with a graphical user interface was
developed for simulation purposes. The simulation results-including the
temporal evolution of concentra\-tion, reactor and coolant jacket
temperatures, and coolant flow rate---are displayed in the form of three
separate graphs. The impact of user-defined PID- parameters is
visualized in real-time, providing intuitive feedback. The tool includes
functionalities for modifying system parameters, clearing results, and
resetting to initial values. The developed simulation, implemented in
the Python environment, demonstrates output consistency based on the
entered values, confirming its accuracy and reliability. This model can
serve as an effective visual aid for teaching control of dynamic
systems.

{\bfseries Keywords:} nonlinear systems, controller, gain, closed loop, feedback

\begin{multicols}{2}
{\bfseries Введение.} ПИД-регулятор является одним из наиболее широко
используемых элементов управления, применяемым примерно в 90\%
промышленных систем. ПИД-регулятор
(пропорционально-интегрально-дифференцирующий) представляет собой
устойчивый регулятор, основанный на механизме обратной связи в контуре
управления, и является наиболее распространённым типом регуляторов в
автоматизированных системах управления {[}1{]}.

ПИД-регулятор является наиболее распространённым типом управления,
основанным на принципе обратной связи. Впервые он был применён в
регуляторах как ключевой элемент и с 1940-х годов стал стандартным
инструментом в системах управления технологическими процессами. В
настоящее время более 95\% контуров управления в промышленных системах
реализованы на основе ПИД-регуляторов, причём значительная их часть
использует ПИД-регулирование. На сегодняшний день ПИД-регуляторы находят
применение практически во всех областях, где требуется управление
{[}2{]}. ПИД-регуляторы широко применяются во многих сферах, включая
автомобильную промышленность, управление климатом в зданиях, медицину,
авиацию и бытовую электронику. Они считаются наиболее распространённым
типом управления в реальных системах благодаря своей универсальности и
эффективности. В частности, ПИД-регуляторы находят применение в
робототехнике, промышленности и управлении технологическими процессами
на химических предприятиях -- например, при регулировании температуры,
управления скоростью вращения двигателей, контроле положения
манипулятора робота. В задачах, где важно поддержание заданного
значения, например, для обеспечения полу-глобальной практической
экспоненциальной устойчивости динамики робота, ПИД-регуляторы
оказываются незаменимыми. Кроме того, они используются для управления
беспилотными летательными аппаратами с целью стабилизации на заданной
высоте, удержания ориентации и позиции в пространстве {[}3, 4{]}.

В работе Li Z., Liu F., Luo Y., Chen Y. {[}5{]} предложен дискретный
дробно-порядковый ПИД-регулятор для нелинейных систем. На первом этапе
применяется U-модель, с помощью которой нелинейная система преобразуется
в модель, приближенную к линейной, что упрощает структуру и снижает
сложность модели, облегчая тем самым проектирование а. Далее
разрабатывается новый закон управления в виде разностного уравнения с
двумя дополнительными настраиваемыми параметрами. В заключение
проводится анализ робастной устойчивости предложенного закона
управления. Для демонстрации эффективности и преимуществ предложенного
контроллера приводятся как детерминированные, так и стохастические
примеры.

ПИД-регулятор применяется в процессорных системах управления с целью
обеспечения устойчивости технологических процессов, улучшения переходных
характеристик и повышения точности управления. Он позволяет точно
согласовать выход системы с заданным значением, играя ключевую роль в
обеспечении качества продукции и повышении эффективности производства
{[}6{]}. В исследовании Kang Y. L., Shrestha G. B. {[}7{]} для повышения
переходной устойчивости используется специальный алгоритм нелинейного
управления. В 2005 году Su Y. X., Sun D., Duan Y. {[}8{]} предложил
нелинейный регулятор, улучшающий характеристики стандартного линейного
ПИД-регулятора.

{\bfseries Материалы и методы.} Настройка регулятора - давний и важный
аспект контроллеров обратной связи. С развитием ПИД-регулятора появилось
много интереса к методам настройки, которые обеспечивают отличную
производительность ПИД-регуляторов. ПИД-регулятор определяет сумму
пропорциональных, интегральных и производных слов трех переменных
следующим образом и обозначается как \(K_p, K_I\) и \(K_D\).

Это зависит от параметров \(K_p, K_I\) и \(K_D\) и \(e(t)\) ошибки
между входными и выходными параметрами. Правильная установка этих
переменных улучшит динамическую реакцию системы, уменьшит чрезмерную
настройку, устранит ошибку постоянного состояния и повысит стабильность
процесса. Это соотношение описано в уравнении (1).

\begin{equation}
output=K_p\times e(t)+K\times \int_0^te(t)dt+K_D\times \frac{de}{dt}
\end{equation}

\(K_p, K_I\) и \(K_D\) могут быть предоставлены параметрами \(P, I\) и \(D\).
\(K_p, K_I\) и \(K_D\) соответсвенно могут быть предоставлены следующим уравнением:

\begin{equation}
K_I=K_p\times T_i,\quad K_d=K_p\times \frac{1}{T_d}
\end{equation}

Соответственно, время восстановления (\(T_i\)) и производное время
(\(T_d\)).  Отвечает за время восстановления и тип реакции производной
системы.  П, И и Д контроллера описываются, то есть кратко
рассматриваются как:

Пропорциональный термин (\emph{П}): пропорциональный параметр облегчает
реакцию, поскольку постоянная времени замкнутого цикла уменьшается
пропорциональным параметром, хотя и не меняет порядок системы, поскольку
выход пропорционален входу. Пропорциональный параметр уменьшает, но не
устраняет постоянную ошибку состояния или смещение. Основная задача
пропорционального элемента-пропорционально изменить величину ошибки и
реакцию ПИД-контроллера. Пропорциональный элемент (Р) обозначается
уравнением:

\begin{equation}
P_{out}=K_pe,
\end{equation}

где \(P_{out}\) - пропорциональная часть выхода контроллера,
\(K_pe\) - пропорциональная прибыль, \(e\)-неправильная часть.

Интегральный термин (\emph{И}): поскольку этот параметр увеличивает
характер и порядок системы на 1, он устраняет смещение. Этот параметр
также увеличивает скорость реакции системы, но является оценкой
постоянных колебаний. Интегральное управление стремится упростить задачи
пропорционального управления. Он регулирует (умножает) ошибку с течением
времени. Скорость восстановления, которая является фактором времени,
используется для представления интегрального параметра управления.

\begin{equation}
I_{out}=\frac{1}{T_i}\int edt=K_I\int edt,
\end{equation}

где \(I_{out}\) - интегральная часть выхода контроллера, \(T_i\) -
время восстановления или интегральное время, \(K_I\) - интегральное
усиление, \(e\) термин ошибки.

Дифференциальный термин (Д): этот параметр в основном снижает
колебательную реакцию системы. Он не влияет на смещение и не влияет на
характер и порядок системы. Он определяет скорость изменения сигнала
ошибки. Сильная реакция системы на быстрые темпы изменения вызвана
производной. Производная со временем корректируется. Слишком большое
значение производной может привести к нестабильному управлению или
проскальзыванию. Он обозначается уравнением:

\begin{equation}
D_{out}=T_d\frac{d}{dt}e=K_D\frac{d}{dt}e,
\end{equation}

Термины обозначаются следующим
образом: \(D_{out}\) - производная часть вывода контроллера, \(T_d\) -
производное время, \(K_D\) - коэффициент усиления процесса, \(e(t)\) -
термин ошибки. На 2-рисунке показана стандартная конфигурация системы
ПИД-регулирования {[}9, 10{]}.
\end{multicols}

\fig[0.7\textwidth]{i2/image94}[Рис.1 - Базовая структура управления PID]

\begin{multicols}{2}
Когда значение \(r(t)\) изменяется, неправильный термин \(e(t)\)
пересчитывается между заданным значением и фактическим выходом. Если
сигнал ошибки \(e(t)\) используется для создания, то \(K_p,K_I\) и
\(K_D\) используется для расчета параметров. Действия параметров
\(K_p,K_I\) и \(K_D\), при получении сигналов измеряются и суммируются
с управляющим сигналом \(u(t)\), используемым в установке. В
результате получается выходной \(y(t)\) сигнал.  Контроллер получает
этот обновленный истинный сигнал, и сигнал ошибки
пересчитывается. Агрегат получает вновь принятый управляющий сигнал
\(u(t)\).  Этот процесс выполняется бесконечно, пока вы не столкнетесь
с постоянной ошибкой. Когда заданное значение изменяется, член ошибки
пересчитывается между заданным значением и фактическим выходом.

Общая формула ПИД - регулятор будет выглядеть следующим образом:

\begin{equation*}
u(t)=K\left[e(t)+\frac{1}{T_i}\int_0^te(\tau)d\tau + T_d\frac{de(t)}{dt}\right],
\end{equation*}

\begin{equation}
u(t)=Ke(t)+K_i\int_0^t e(\tau)d\tau +K_d\frac{de(t)}{dt},
\end{equation}

где \(K_i=\frac{K}{T_i}\) - усиление (сброс) неотъемлемой части
контроллера, \(K_d=KT_d\) - преимущества производной части
контроллера.

{\bfseries Обсуждение и результаты.} Систему управления с обратной связью
часто называют «системой управления с замкнутым контуром». С
практической точки зрения, термины «управление обратной связью» и
«управление замкнутым контуром» могут быть взаимозаменяемыми. Сигналы
ошибок, которые работают в системе управления замкнутым контуром, то
есть разница между входным сигналом и сигналом обратной связи,
передаются контроллеру, чтобы уменьшить ошибку и довести вывод системы
до желаемого значения. Термин «управление по замкнутому циклу» всегда
означает использование эффекта управления обратной связью для уменьшения
ошибок системы.

\emph{{\bfseries Пример 1}}. Моделирование пропорционально-интегрального
контроллера с помощью Simulink.

Полиномиальные уравнения для этого моделирования

\[G(s)=\frac{s^2+12.s+15}{s^3+3s^2+7.s+5}\]

\(H(s)=1; K_p=1; K_i=0.5;\) Вход = 1; Время = 4 с. Процесс ввода
данных можно увидеть на рисунке:
\end{multicols}

\fig[0.6\textwidth]{i2/image112}[Рис.2 - Блок-схема для управления пропорционально интегральным ПИД-контроллером]
\fig[0.6\textwidth]{i2/image113}[Рис.3 - \(K_p=1\) и \(K_i=0.5\) выходной ответ для значений]

\begin{multicols}{2}
Вы можете видеть, что пропорциональный интегральный контроллер имеет
тенденцию уменьшать время роста, добавлять время перегрузки и падения и
устранять постоянное состояние.

Система, которую мы регулируем ПИД, работает намного лучше, чем система
с открытым циклом. Это сравнение с системой с открытым циклом часто
бывает эффективным, поскольку оно показывает случай, когда используется
самая простая форма управления. Если система ПИД работает одинаково, ее
использование будет неэффективным {[}11{]}.

Тепловые процессы с хорошей теплоизоляцией ведут себя как интеграторы,
поскольку из-за малых теплопотерь система аккумулирует энергию. Это
придаёт процессу большую инерционность и медленную динамику. Поэтому для
управления такими процессами необходим производный режим, а
интегральный, напротив, лишь дополнительно замедляет реакцию системы.
\end{multicols}

\fig[0.5\textwidth]{i2/image116}[Рис.4 - Генерация управляющего сигнала ПИД-регулятора]
\fig[0.7\textwidth]{i2/image118}[Рис.5 - P-диапазон 60\% и 40\%]

\begin{multicols}{2}
На рисунке 5 показано, как получить выходной сигнал и сигнал управления
ПИД. Также важно интерпретировать пропорциональную область, так как во
многих промышленных контроллерах вместо коэффициента усиления
используется именно пропорциональная область для настройки системы.

\[PB=\frac{100}{K}[\%]\]

Обычно пропорциональный диапазон устанавливается на уровне PB = 100\%.
Графическое представление работы контроллера P показано на рисунках 4
(а) и 4 (b), которые наглядно демонстрируют концепции для диапазонов
60\% и 40\%. Если у контроллера узкая полоса пропорциональности,
выходной сигнал будет увеличиваться быстрее, чем в случае более широкой
полосы. Это важно при настройке контроллера, особенно в процессе
интеграции с регуляторами И и Д, что касается контроллеров ПИ, ПД и ПИД
{[}12{]}.
\end{multicols}

\fig{i2/image119}[Рис.6 - Передаточная функция интегратора]

\begin{multicols}{2}
{\bfseries \emph{Пример 2}.} Температура системы регулируется с помощью
скорости потока воды и температуры охлаждающей воды. Уровень температуры
оказывает влияние на химическую реакцию внутри охлаждающей оболочки.
Химический материал является важным продуктом. Таким образом, скорость
потока и температура охлаждающей воды напрямую воздействуют на качество
продукции. Это, в свою очередь, означает, что ПИД-регулятор влияет на
качество продукта. Далее приведена настройка охлаждающего слоя. Для
демонстрации эффективной настройки ПИД-регулятора можно рассмотреть
динамическое уравнение. Модель этой системы охлаждения может быть
описана через баланс энергии охлаждающей воды и системы химического
реактора. На рисунке 7 представлена блочная схема рубашки охлаждения,
используемой для моделирования управления нелинейной системой в данной
работе. Температура данной системы регулируется с помощью расхода воды и
температуры охлаждающей жидкости. Температурный режим оказывает влияние
на химическую реакцию, происходящую внутри упаковки. Химическое
вещество, в свою очередь, является важным конечным продуктом.
Следовательно, расход и температура охлаждающей воды напрямую влияют на
качество продукции. Таким образом, настройка ПИД-регулятора оказывает
влияние на качество производственного процесса. Параметры охлаждающей
рубашки приведены ниже. Для эффективного объяснения процесса настройки
ПИД-регулятора преподавателю следует использовать динамическое
уравнение. Это уравнение должно быть представлено в упрощённой форме,
так как основное внимание в статье уделяется именно настройке
ПИД-регулятора. Модель системы охлаждающей упаковки может быть выражена
с помощью энергетического баланса между охлаждающей водой и химическим
реактором. Таким образом, уравнения энергетического баланса представлены
в формулах (2) и (3).

На рисунке 7 представлена блочная схема охлаждающей упаковки,
использованной в данной работе для моделирования управления нелинейной
системой.
\end{multicols}

\fig[0.4\textwidth]{i2/image120}[Рис.7 - Нелинейная система охлаждающей упаковки]

\begin{multicols}{2}
Уравнение энергетического баланса может быть представлено с помощью
уравнений (7) и (8).

\begin{equation}
\frac{dc_a}{dt}=q(c_{ai}-c_a)-Vkc_a,
\end{equation}

\begin{equation}
V\rho c_p\frac{dT}{dt}=\omega c_p(T_i-T)+(-\sigma H_rVkc_a+UA(T_c-T))
\end{equation}

Затем эта система должна быть нормализована для проектирования
контроллера проектирования.

\begin{equation}
\frac{dc_a}{dt}=q(c_{ai}-c_a)-kc_a
\end{equation}

\begin{equation}
\frac{dT}{dt}=\frac{q}{V}(T_i-T)+\frac{-\delta H_r}{\rho c_p}kc_a+\frac{UA}{V\rho c_p}(T_c-T)
\end{equation}

Эта охлаждающая упаковка контролирует химическую реакцию через
температуру охлаждающей воды в упаковке. Поэтому результаты реакции
зависят от скорости и температуры охлаждающей воды. На рисунке 8 показан
график начального состояния упаковки {[}12{]}.
\end{multicols}

\tcap{Таблица 1 -- Проектные параметры охлаждающей упаковки}
\begin{longtblr}[
  label = none,
  entry = none,
]{
  cells = {c},
  cells = {font = \small},
  hlines,
  vlines,
}
№  & \textbf{Описание}                        & \textbf{Обозначение} & \textbf{Значения}            & \textbf{Единица измерения} \\
1  & Энергия активации                        & \textit{E\_a}        & 72,750                       & Дж/моль                    \\
2  & Предэкспоненциальный множитель Аррениуса & \textit{k0}          & 7.2 × 10\textsuperscript{10} & 1/мин                      \\
3  & Газовая постоянная                       & \textit{R}           & 8.314                        & Дж/моль/К                  \\
4  & Объём реактора                           & \textit{V}           & 100                          & литр                       \\
5  & Плотность                                & \textit{ρ}           & 1000                         & г/литр                     \\
6  & Теплоёмкость                             & \textit{Cp}          & 0.239                        & Дж/г/К                     \\
7  & Энтальпия реакции                        &                      & -50,000                      & Дж/моль                    \\
8  & Коэффициент теплообмена                  & \textit{UA}          & 50,000                       & Дж/мин/К                   \\
9  & Расход питающего потока                  & \textit{q}           & 100                          & литр/мин                   \\
10 & Концентрация реагента на входе           & \textit{Ca}          & 1                            & моль/литр                  \\
11 & Температура на входе                     & \textit{Tf}          & 350                          & К                          \\
12 & Начальная концентрация реагента          & \textit{cA,0}        & 0.5                          & моль/литр                  
\end{longtblr}

\begin{multicols}{2}
В языке Python для различных задач требуются специализированные
программные пакеты, что отражает модульную принцип проектирования,
характерную для программного обеспечения с открытым исходным кодом.
Такой подход позволяет пользователям разрабатывать и распространять
пакеты Python, адаптированные под конкретные потребности. В результате
доступен обширный набор Python-библиотек, охватывающий широкий спектр
областей - от научных вычислений до компьютерного моделирования, от
интеллектуального анализа данных до систем искусственного интеллекта.
Пользователи могут искать, загружать и устанавливать данные программные
компоненты в соответствии со своими требованиями.

Библиотека Python-Control особенно эффективна для анализа и
проектирования систем автоматического управления с обратной связью. С
использованием данной библиотеки численные расчёты автоматических систем
управления могут выполняться в сочетании с такими инструментами, как
\emph{NumPy, SciPy и Matplotlib}. По своей структуре и командам
Python-Control во многом аналогична средствам, представленным в
инструментарии MATLAB {[}13{]}.

Эта охлаждающая рубашка предназначена для контроля химической реакции
посредством температуры охлаждающей воды. Поэтому результаты реакции
результаты реакции зависят от скорости и температуры охлаждающей воды
температуры. На рисунке 8 показан код Python и график для начального
состояния рубашки.

ПИД-регулирование нелинейной системы охлаждающей упаковки в Python На
рисунке 8 а), б), в) показаны концентрация, температура и расход
концентрацию, температуру и расход с помощью P, I, D ПИД-регулятора.
\end{multicols}

\fig[0.7\textwidth]{i2/image126}[Рис.8 - а) График отклика системы при настройках ПИД-регулятора K = 50, I = 70, D = 0]

\fig[0.7\textwidth]{i2/image127}[Рис.8 - б) График отклика системы при настройках ПИД-регулятора K = 50, I = 70, D = 5]

\fig[0.7\textwidth]{i2/image128}[Рис.8 - в) График отклика системы при настройках ПИД-регулятора K = 50, I = 70, D = 7]

\begin{multicols}{2}
Из этих рисунков видно, что влияние D-функции больше, чем у других. Это
означает, что D-функция увеличивается резко возрастает в процессе шума.

На основе приведённых выше изображений можно заметить, что влияние
D-функции является более значительным по сравнению с другими
параметрами. Это указывает на то, что D-функция резко увеличивается при
наличии шумового процесса или играет важную роль в управлении внезапными
изменениями с целью подавления шума.

{\bfseries Выводы.} В данной статье исследованы методы настройки
ПИД-регулятора и его эффективность в управлении нелинейными системами.
ПИД-регулятор обеспечивает точное следование системы заданному значению
во времени, сокращает время установления, устраняет установившуюся
ошибку и улучшает переходный процесс.

Результаты моделирования в среде Simulink показали, что при правильном
выборе пропорционального (P), интегрального (I) и дифференциального (D)
параметров ПИД-регулятора можно повысить устойчивость и точность
выходного сигнала системы. Особенно чётко проявилось влияние D-параметра
на снижение колебательности системы.

Доказано, что язык программирования Python и его библиотеки
предоставляют удобную среду для моделирования систем управления,
аналогичную MATLAB. Результаты моделирования с использованием
ПИД-регулятора, выполненного на Python, оказались сопоставимыми с
результатами, полученными в среде Simulink. Кроме того, разработка
системы управления с графическим интерфейсом на базе Python является
удобной и эффективной для пользователя. В заключение следует отметить,
что моделирование ПИД-регулятора в средах MATLAB и Python позволяет
точно оценить динамику системы. Язык Python, как инструмент с открытым
исходным кодом и высокой гибкостью, предоставляет широкие возможности
для исследования и применения ПИД-регуляторов в инженерных и научных
целях.
\end{multicols}

\begin{center}
{\bfseries Литература}
\end{center}

\begin{refs}
1. Joseph S. B. et al. Metaheuristic algorithms for PID controller
parameters tuning: Review, approaches and open problems//Heliyon.-2022.
-Vol.8(5): e09399. DOI 10.1016/j.heliyon.2022.e09399.

2. Åström K. J., Hägglund T. Advanced PID Control//ISA - The
Instrumentation, Systems and Automation Society. -2006. - 460 p. ISBN:
978-1-55617-942-6.

3. Vinagre B. M., Monje C. A., Calderón A. J., Suárez J. I. Fractional
PID controllers for industry application. A brief introduction //
Journal of Vibration and Control. -2007. -Vol.13 (9--10). -P.1419 -
1429. DOI 10.1177/1077546307077498.

4. Pan Y., Li X., Yu H. Efficient PID tracking control of robotic
manipulators driven by compliant actuators //IEEE Transactions on
Control Systems Technology. -2019. -Vol.27(2). -P.915--922. DOI
10.1109/TCST.2017.2783339.

5. Li Z., Liu F., Luo Y., Chen Y. Discrete fractional order PID
controller design for nonlinear systems //International Journal of
Systems Science. -2021. -Vol.52(15). -P.3206--3213. DOI
10.1080/00207721.2021.1924307.

6. Dubey V., Goud H., Sharma P. C. Role of PID control techniques in
process control system: a review //Data Engineering for Smart Systems:
Proceedings of SSIC 2021. -2022. -P.659--670. DOI
10.1007/978-981-16-2641-8\_62.

7. Kang Y. L., Shrestha G. B., Lie T. T. Application of an NLPID
controller on a UPFC to improve transient stability of a power system
//IEE Proceedings-Generation, Transmission and Distribution. -2001.
-Vol.148 (6). -P.523-529. DOI
\href{https://doi.org/10.1049/ip-gtd:20010526}{10.1049/ip-gtd:20010526}.

8. Su Y. X., Sun D., Duan Y. Design of an enhanced nonlinear PID
controller // Mechatronics. -2005. -Vol.15 (8). -P.1005-1024. DOI
\href{https://doi.org/10.1016/j.mechatronics.2005.03.003}{10.1016/j.mechatronics.2005.03.003}.

9. Sundström E. et al. Reference Implementation of the PID Controller
//IFAC-PapersOnLine. -2024. -Vol.58(7). -Р.370-375. DOI
\href{https://doi.org/10.1016/j.ifacol.2024.08.090}{10.1016/j.ifacol.2024.08.090}.

10. Kim D. H. Advanced Lecture for PID Controller of Non-linear systems
in Python //International Journal of Recent Engineering and Technology.
-2021. -Vol.9 (6). -P.20-29. DOI
\href{http://dx.doi.org/10.35940/ijrte.F5375.039621}{10.35940/ijrte.F5375.039621}.

11. Senen A., Ratnasari T., Simamora Y. PID controller simulator design
for polynomials transfer function//~E3S Web of Conferences. -2020. -Vol.
202: 15006. DOI
\href{https://doi.org/10.1051/e3sconf/202020215006}{10.1051/e3sconf/202020215006}.

12. Kim D.H., Alemayehu H. A study on teaching method of control
engineering by using Python based PID // International Advanced Research
Journal in Science, Engineering and Technology. -2020. -Vol.7(9). -P.
1-9. DOI
\href{http://dx.doi.org/10.17148/IARJSET.2020.7901}{10.17148/IARJSET.2020.7901}.

13. Chongquan G. The application of state feedback principle in turbine
adjustment system with Python control library //Journal of Physics:
Conference Series. -2024. - Vol.2782: 012072.
DOI 10.1088/1742-6596/2782/1/012072.
\end{refs}

\begin{info}
\hspace{1em}\emph{{\bfseries Cведения об авторах}}

Бахадирова Г.Б - PhD студент, Международный университет Астана, Астана,
Казахстан, e-mail: gulnaz.bahadirova.84@mail.ru;

Тасболатұлы Н.- PhD, ассоциированный профессор, Международный
университет Астана, Астана, Казахстан, e-mail: tasbolatuly@gmail.com;

Баумуратова Д.Б - PhD, Международный университет Астана, Астана,
Казахстан, е-mail: Baumuratova.d@gmail.com;

Аканова А.С. - PhD, Казахский агротехнический исследовательский
университет им. С. Сейфуллина, е-mail: akerkegansaj@mail.ru;

Әлімжан Ә.Қ. - Магистр, Astana IT University, Астана, Казахстан, е-mail:
Assel.Alimzhan@astanait.edu.kz.

\hspace{1em}\emph{{\bfseries Information about the authors}}

Bakhadirova G.B.-PhD student, Astana International University, Astana
Kazakhstan, е-mail: gulnaz.bahadirova.84@mail.ru;

Tasbolatuly N.- PhD, Associate Professor, Astana International
University, Astana, Kazakhstan, е-mail: tasbolatuly@gmail.com;

Baumuratova D.B. - PhD, Astana International University,
Astana, Kazakhstan, е-mail: Baumuratova.d@gmail.com;

Akanova A.S. - PhD, S. Seifullin Kazakh agrotechnical research
university, Astana, Kazakhstan, е-mail: akerkegansaj@mail.ru;

Alimzhan A. - MSc in Computer Science and Engineering, Astana IT
University, Astana, Kazakhstan, mail: Assel.Alimzhan@astanait.edu.kz.
\end{info}
