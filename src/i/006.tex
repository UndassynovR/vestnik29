\id{МРНТИ 50.43.17; 68.29.15}{}

\begin{header}
\swa{А. Алтыбаев, С. Жұмағали, Е. Конысбаев, Б. Бекмухамедов К. Акишев}{НАУЧНО-МЕТОДОЛОГИЧЕСКИЕ ОСНОВЫ ЦИФРОВОЙ ТРАНСФОРМАЦИИ В СИСТЕМЕ ТОЧНОГО ЗЕМЛЕДЕЛИЯ}

\tsp{1}А. Алтыбаев,
\tsp{1}С. Жұмағали,
\tsp{1}Е. Конысбаев,
\tsp{1}Б. Бекмухамедов,
\tsp{2}К. Акишев\envelope
\end{header}

\begin{affil}
\tsp{1}ТОО «Научно-производственный центр агроинженерии», Алматы, Казахстан,

\tsp{2}Казахский университет технологии и бизнеса им. К.Кулажанова, Астана, Казахстан

\corrauthor{Корреспондент - автор:akmail04cx@mail.ru}
\end{affil}

По данным из открытых источников за 2023год, площадь внедрения элементов
точного земледелия в Казахстане, превысила 3 млн га, тем не менее
возможность, оценивается более чем в 20 млн га.

В этой связи актуальность разработки научно-методических основ,
обеспечивающих применение цифровых решений точного земледелия достаточно
высока.

В настоящей статье представлены и обоснованы принципы функционирования
автоматизированной системы дифференцированного дозирования сыпучих
материалов (АСДДСП). Разработан и апробирован алгоритм расчёта
управляющей информации (УИ) на входе исполнительного механизма дозатора,
который служит теоретической основой программного обеспечения точности и
оперативности дифференцированного внесения удобрений, позволяющий
достичь повышения точности внесения удобрений до 10--15\%, а также
сокращение их перерасхода на 12-18\% по сравнению с традиционными
методами.

Теоретическая значимость исследования заключается в создании основы
методически корректного математического обеспечения вычислительных
экспериментов.

Практическая значимость исследований, подтверждается результатами
эксплуатационно-техно\-логических тестов на базе Казахского
научно-исследовательского института земледелия и растениеводства
(КазНИИЗиР), показавшими достаточно высокие результаты работы АСДДСП с
ростом производительности операций до 8--10\%.

Научная новизна исследования заключена в методологической основе
разработки алгоритма определения управляющей (командной) информации для
оперативного управления исполнительным механизмом дозатора сыпучих
материалов.

Полученные результаты, подтверждают высокий уровень эффективности
выполнения работ мобильными агрегатами в пространственно-временных
координатах, а также могут использоваться при проектировании и
модернизации сельскохозяйственных агрегатов, в образовательных
программах по цифровой агроинженерии.

Разработанные научно-методологические основы исследования создают
предпосылки повышения эффективности, устойчивости сельского хозяйства,
ресурсосбережения, а также снижения себестоимости сельскохозяйственной
продукции, повышения конкурентоспособности аграрного сектора Республики
Казахстан.

{\bfseries Ключевые слова}: цифровая трансформация, автоматизированная
система, алгоритм, управляющая информация, модель, блок схема,
земледелие.

\begin{header}
SCIENTIFIC AND METHODOLOGICAL BASES OF DIGITAL TRANSFORMATION IN THE PRECISION AGRICULTURE SYSTEM

\tsp{1}A. Altybayev,
\tsp{1}C. Zhumagali,
\tsp{1}E. Konysbayev,
\tsp{1}B. Bekmukhamedov,
\tsp{2}K. Akishev\envelope
\end{header}

\begin{affil}
\tsp{1}Scientific and Production Centre for Agricultural Engineering LLP, Almaty, Kazakhstan,

\tsp{2}Kazakh University of Technology and Business named after K.Kulazhanov, Astana, Kazakhstan,

e-mail: akmail04cx@mail.ru
\end{affil}

According to open source data for 2023, the area of introduction of
precision farming elements in Kazakhstan has exceeded 3 million
hectares, however, the possibility is estimated at more than 20 million
hectares.

In this regard, the relevance of developing scientific and
methodological foundations for the application of digital precision
farming solutions is quite high.

This article presents and substantiates the principles of functioning of
the automated system of different\-iated dosing of bulk materials
(ASDDSP). An algorithm for calculating the control information (CI) at
the input of the dispenser actuator has been developed and tested, which
serves as the theoretical basis for software for the accuracy and
efficiency of differentiated fertilizer application, allowing for an
increase in the accuracy of fertilizer application to 10-15\%, as well
as reducing their overspending by 12-18\% compared with traditional
methods.

The theoretical significance of the research lies in creating the basis
for methodically correct mathemati\-cal support for computational
experiments.

The practical significance of the research is confirmed by the results
of operational and technological tests on the basis of the Kazakh
Scientific Research Institute of Agriculture and Horticulture
(KazNIIZiR), which showed fairly high results of ASDDSP with an increase
in the productivity of operations up to 8-10\%.

The scientific novelty of the research lies in the methodological basis
for the development of an algo\-rithm for determining the control
(command) information for the operational control of the actuator of the
dispenser of bulk materials.

The results obtained confirm the high level of efficiency of work
performed by mobile units in space-time coordinates, and can also be
used in the design and modernization of agricultural units, in
educational programs on digital agroengineering.

The developed scientific and methodological foundations of the research
create prerequisites for increa\-sing the efficiency, sustainability of
agriculture, resource conservation, as well as reducing the cost of
agricultural products, increasing the competitiveness of the
agricultural sector of the Republic of Kazakh\-stan.

{\bfseries Keywords}: digital transformation, automated system, algorithm,
control information, model, flowchart, agriculture.

\begin{header}
ҒЫЛЫМИ-ӘДІСТЕМЕЛІК НЕГІЗДЕРІ ДӘЛ ЕГІНШІЛІК ЖҮЙЕСІНДЕГІ ӨЗГЕРІСТЕР

\tsp{1}А. Алтыбаев,
\tsp{1}С. Жұмағали,
\tsp{1}Е. Қонысбаев,
\tsp{1}Б. Бекмұхамедов,
\tsp{2}К. Акишев\envelope
\end{header}

\begin{affil}
\tsp{1}"Агроинженерия ғылыми-өндірістік орталығы" ЖШС, Алматы, Қазақстан,

\tsp{2}Қазақ технология және бизнес университеті. Қ. Құлажанова, Астана, Қазақстан,

e-mail: akmail04cx@mail.ru
\end{affil}

2023 жылы ашық көздерден алынған мәліметтерге сәйкес, Қазақстанда нақты
егіншілік элементтерін енгізу көлемі 3 млн гектардан асты, алайда бұл
мүмкіндік 20 млн гектардан асады деп бағалануда.

Осыған байланысты дәл егіншіліктің цифрлық шешімдерін қолдануды
қамтамасыз ететін ғылыми-әдістемелік негіздерді әзірлеудің өзектілігі
өте жоғары.

Осы бапта Сусымалы материалдарды сараланған мөлшерлеудің
автоматтандырылған жүйесінің (АСДДСП) жұмыс істеу қағидаттары ұсынылған
және негізделген. Тыңайтқыштарды қолдану дәлдігін 10-15\% - ға дейін
арттыруға, сондай-ақ олардың артық шығынын дәстүрлі әдістермен
салыстырғанда 12-18\% - ға қысқартуға қол жеткізуге мүмкіндік беретін
дифференциалды тыңайтқыштарды қолданудың дәлдігі мен жеделдігін
бағдарламалық қамтамасыз етудің теориялық негізі ретінде қызмет ететін
диспенсердің атқарушы механизмінің кірісінде басқару ақпаратын (УИ)
есептеу алгоритмі әзірленді және сыналды.

Зерттеудің теориялық маңыздылығы есептеу эксперименттерін әдістемелік
тұрғыдан дұрыс математикалық қамтамасыз етудің негізін құру болып
табылады.

Зерттеулердің практикалық маңыздылығы операциялар өнімділігінің
8-10\%-ға дейін өсуімен АСДДСП жұмысының жеткілікті жоғары нәтижелерін
көрсеткен Қазақ егіншілік және Өсімдік шаруашылығы ғылыми-зерттеу
институты(Қазғзии) базасындағы пайдалану-технологиялық сынақтардың
нәтижелерімен расталады.

Зерттеудің ғылыми жаңалығы сусымалы материалдар диспенсерінің атқарушы
механизмін жедел басқару үшін басқарушы (командалық) ақпаратты анықтау
алгоритмін әзірлеудің әдіснамалық негізіне негізделген.

Алынған нәтижелер кеңістіктік-уақыттық координаттарда МБ агрегаттарының
жұмыстарды орындау тиімділігінің жоғары деңгейін растайды, сондай-ақ
ауыл шаруашылығы агрегаттарын жобалау және жаңғырту кезінде, цифрлық
агроинженерия бойынша білім беру бағдарламаларында пайдалануға болады.

Зерттеудің әзірленген ғылыми-әдіснамалық негіздері ауыл шаруашылығының
тиімділігін, орнықтылығын арттыру, ресурс үнемдеу, сондай-ақ ауыл
шаруашылығы өнімінің өзіндік құнын төмендету, Қазақстан Республикасының
аграрлық секторының бәсекеге қабілеттілігін арттыру үшін алғышарттар
жасайды.

{\bfseries Түйін сөздер}: сандық түрлендіру, автоматтандырылған жүйе,
алгоритм, басқару ақпараты, моделі, блок-схемасы, егіншілік.

\begin{multicols}{2}
{\bfseries Введение.} Современная парадигма инновационного развития
сельскохозяйственного производства сопряжена с широким применением
информационно-цифровых технологий, электроники, автоматизированных
систем. Интеллектуальной основой для этого служат фундаментальные
инновационные решения в других сферах и отраслях, которые также успешно
используются и в сельском хозяйстве. Принято, что современная
инновационная основа сельскохозяйственного производства базируется на
системы точного земледелия работы {[}1-4{]}. Анализ возникновения и
развития системы точного земледелия представленный в работах {[}5-8{]}
позволяет сформулировать следующую ее сущность, имеющую практическое
значения для реальных производственных процессов производства полевых
работ: точное земледелие - это совокупность относительно самостоятельных
технологий (технологии GPS и GIS, точного картографирования полей и
др.), необходимость внедрения которых определяется имеющимися ресурсами
конкретного производства, а также научно-техническим потенциалом на
данном этапе производственных отношений. В то же время, следует понять,
что по существу принципы точного земледелия в большей степени направлены
на совершенствование процессов управления на всех уровнях производства
полевых механизированных работ в сельском хозяйстве. При этом сами
производственно-технологические, т.е. рабочие процессы, во многом не
претерпевают изменений принципиального порядка. Из обзора и анализа
работ отечественных и зарубежных авторов представленных в публикациях
{[}9,10{]} следует, что исследования, направленные на внедрение цифровых
технологий в реальные процессы производства полевых работ, ведутся
фрагментарно, исходя из частных физических эффектов, ориентируясь на
промежуточные результаты, не связанные требованиями оптимальности
функционирования технологических систем в целом. В основном преобладают
конструктивно-технические решения, а вопросы оперативного управления
технологическими процессами в течение рабочей смены не доведены до
эксплуатационного статуса.

В этой связи наибольшую актуальность приобретают вопросы, касающиеся
оперативного управления состоянием технологической работоспособности
рабочих органов и исполнительных механизмов рабочих машин мобильных
агрегатов сельскохозяйственного назначения обеспечивающих возможности
для точного земледелия.

Цель исследования -- установить влияние внутренних и внешних факторов
технологической системы внесения минеральных удобрений на величину
управляющей информации работой исполнительного механизма дозирующего
устройства машинно-тракторного агрегата.

Объектом исследования являются процессы взаимодействия конструктивных
параметров посевных агрегатов и технологических режимов их работы с
эксплуатационно-технологическими показателями выполнения работы.

Новизна исследования заключена в методологической основе разработки
алгоритма определения управляющей (командной) информации для
оперативного управления исполнительным механизмом дозатора сыпучих
материалов.

Практическая значимость исследования связана тем, что результаты служат
методико-аналитическим инструментарием для инженерных расчетов при
проектировании технологических операций (процессов) производства полевых
механизированных в условиях глобальной цифровой трансформации отраслей
сельского хозяйства.

{\bfseries Материалы и методы.} Методологической основой исследования
является комплекс современных научных подходов и инструментов, принципы
системного подхода и аппарат системного анализа-это позволило
рассматривать процессы цифровой трансформации в сельском хозяйстве, как
взаимосвязь технических, технологических и организационных компонентов.

При этом обеспечивалась интеграция цифровых решений с агротехническими
требованиями и нормативами.

В исследовании использовались методы теоретической и прикладной
информатики, в виде алгоритмов и моделей, методы статистического анализа
больших массивов данных, позволившее выявить закономерности изменения
параметров технологических процессов.

Динамическое моделирование выполнялось с использованием инструментов
имитационного моделирования, с реальными сценариями функционирования
систем дифференцированного внесения удобрений.

В качестве исходных материалов, использованы информационные ресурсы,
инженерно-технологического обеспечения сельскохозяйственного
производства, базы данных состава почв, справочные нормативы и
технические характеристики агромашин.

Представленные данные обеспечили проектирование, верификацию, а также
экспериментальную апробацию технологических операций по внесению
удобрений.

{\bfseries Обсуждение и результаты.} В системе точного земледелия
дифференцированное внесение удобрений осуществляется двумя основными
способами: внесение в режиме on-line (режим реального времени) и
внесение в режиме off-line (с предварительно подготовленной картой
поля).

Первый вариант (on-line) подразумевает выполнение расчетов
непосредственно в процессе внесения удобрений. Для этого используются
датчики-спектрометры, установленные на беспилотник или трактор.

Второй способ (off-line) предполагает заблаговременный расчет норм
внесения удобрений по каждой выделенной на поле зоне, техника работает в
запрограммированном режиме. С позиции проектирования автоматизированной
системы управления технологическими процессами off-line режим относится
к автоматизированной системе без обратной связи.

Обобщенная картина цифровой трансформации производства полевых
механизированных работ в растениеводстве в рамках концепции точного
земледелия, применительно к мобильным агрегатам дифференцированного
внесения удобрений, представлена на рис.1, на котором представлены 2
группы моделей - функциональную и процедурную.

Функциональная модель, отражает физическое содержание архитектуры
автоматизированной системы управления технологическими процессами,
включает в себе все физические объекты, которые непосредственно
участвуют в сфере управления, взаимосвязанная совокупность которых
составляет данную техническую систему, целевое предназначение которой
направлено на реализацию дифференцированного внесения удобрений в
соответствии с требованиями агротехнологий.

К ним относятся: программно-аппаратным комплекс (ПАК), мобильный агрегат
(серийно выпускаемый промышленностью) для внесения удобрений,
оборудование для реализации автоматизированного управления
исполнительными механизмами дозированной подачи удобрений агрегатом в
условиях оперативного времени выполнения технологической операции
внесения удобрений.

Процедурные модели описывают операционные характеристики системы, т.е.
описывают порядок действий по управлению работой технологической
операции в целом. При автоматизации производства особый интерес
представляют информационные процедурные модели, а также модели режимов и
обеспечения безопасности работы.

Работа процедурной модели автоматизированной системы дифференцированного
внесения удобрений (АСДВУ) начинается с загрузки
координатно-привязанного космического снимка требуемой территории с
использованием программы (например, SASPlanet).

Следующий шаг, создание карты-задания.

Карта-задание создается на основе точек замера проб агрохимического
состояния почвы, методами аппроксимации (Делоне, Вороного и др.) и
созданием изолиний -- границ доз внесения удобрений.

Блок-схема создания карты-задания, представляет собой относительно
самостоятельную процедуру (В).

Далее формируется карта (блок-схема Б) с векторным слоем геометрических
размеров поля с учетом неоднородностей, учитываются рельефные
особенности, водные и лесные объекты.

Данная карта может быть получена полуавтоматически путем классификации
поля с присутствующими неоднородностями, или же может быть создана
вручную с использованием разработанного графического редактора.

Разработанный программный модуль, автоматически создает векторную карту
движения агрегата с учетом ширины сеялки и радиусов разворота агрегата.

Программный модуль, также имеет возможность оптимизировать трассы
движения агрегата, как внутри поля (для сокращения трасс холостых
проходов), так и во время движения к полю.

Процедуры 4, 5, 6 оценивают координаты агрегата относительно координат
поля, требующего внесение удобрений. Если агрегат попадает на территорию
поля, где есть необходимость внесения удобрения, то посылается сигнал
сеялке -- вносить удобрения с дозой, согласно карте-заданию.

Процедура 7 обеспечивает реализацию исполнительным органом управляющей
информации «внести удобрения» согласно карте-заданию.

Процедуры 8, 9 реализуют формирование выходного отчета о внесении
удобрений, и результаты записываются в БД.

Для проектирования автоматизированной системы оперативного управления
исполнительными механизмами без обратной связи {[}11 - 13{]} наибольший
интерес представляет установление функциональной связи выходных
переменных с входными переменными, которые формируют командную
информацию, обеспечивающую точность и своевременность выполнения
технологической операции. При этом необходимо сформулировать
математическую модель физического процесса.
\end{multicols}

\fig{i/image18}[Рис. 1 - Обобщенная картина цифровой трансформации]

\begin{multicols}{2}
Для установления аналитических выражений элементов функционирования
технологической системы внесения (посева) воспользуемся абстрактным
представлением высевающего механизма (дозатора) и схемой
механизированного (машинного) выполнения данной технологической операции
(рис.2).

При этом искомую модель можно сформулировать исходя из геометрических
соотношений, первичных физических величин.

За один оборот вала высевающего аппарата объем дозирования определяется
исходя из конструктивно-технических параметров высевающего механизма,
т.е. дозатора (рис.2 \emph{б}) формула 1:

\begin{equation}
V_{\text{об}} = z \cdot V_{\text{доз}}, \text{ м\tsp{3}/об}
\end{equation}

где \(V_{\text{доз}}\) - объем дозирующего элемента, \emph{м\tsp{3}};
z - количество дозирующих элементов на валу высевающего аппарата.

Объем дозирующего элемента (\(V_{\text{доз}}\)) определяют исходя из
конструктивной особенности высевающего аппарата, как объем трехмерного
тела, т.е., формула 2:

\begin{equation}
V_{\text{доз}} = s \cdot l, \text{ см\tsp{3}}
\end{equation}

где \(s\) - площадь поперечного сечения дозирующего элемента;

\(l\) - длина дозирующего элемента.

При этом масса дозируемого материала составляет, формула 3:

\begin{equation}
m_{\text{об}} = \gamma \cdot V_{\text{об}},\ \text{кг/об}
\end{equation}

где \(\gamma\) - удельная объемная масса сыпучего материала.

Технологическая длина пути прямолинейного движения данного агрегата для
выполнения работы на 1га площади составляет, формула 4:

\begin{equation}
L_{\text{га}} = \frac{1_{\text{га}}}{B},\ \text{м}
\end{equation}

где - \(B\) рабочая ширина технологической (рабочей) машины
\end{multicols}

\fig{i/image29}[Рис. 2 - Определение управляющего параметра дозирующего механизма]

\begin{multicols}{2}
Технологическое время при заданной скорости агрегата в соответствии с
агротехническими требованиями формула 5:

\begin{equation}
t_{\text{га}} = \frac{L_{\text{га}}}{\upsilon_{\text{МТА}}},\ \text{мин}
\end{equation}

\(u\) - рабочая (рекомендуемая) скорость движения агрегата при выполнении
технологической операции, м/мин.

Технологически необходимое количество оборотов вала высевающего аппарата
для высева нормативного объема (массы) удобрений на единицу площади,
формула 6:

\begin{equation}
n_{\text{га}} = \frac{M_{\text{н}}}{i \cdot m_{\text{об}}},\ \text{об}
\end{equation}

\(M_{\text{н}}\) - масса назначенной нормы высева;
\(i\) - число рабочих органов посевного агрегата.

Число оборотов на входе дозирующего механизма, формула 7:

\begin{equation}
n_{\text{вх}} = \frac{n_{\text{га}}}{t_{\text{га}}},\;\text{об/мин}
\end{equation}

или с учетом (1) - (6), после несложных преобразований, получим, формула
8:

\begin{equation}
n_{\text{вх}}
= \prod_{j=1}^{\lambda} k_{j} \cdot 
\frac{\text{M}_{\text{н}} \cdot \upsilon_{\text{МТА}} \cdot B}
{i \cdot z \cdot s \cdot l \cdot \gamma}
,\;\text{об/мин}
\end{equation}

где \(\prod_{j=1}^{\lambda}k_{j}\) - совокупность поправочных
коэффициентов, учитывающих размерность составляющих элементов

Таким образом, формула (8) является математическим выражением
зависимости управляющей (командной) информации от
конструктивно-технических параметров применяемого средства механизации,
технологических режимов выполнения работы, физико-технических
характеристик сыпучих материалов, конкретное назначение которой
регулируются ПАК с использованием карты-задания о реальных потребностях
в пространственно-временных координатах. Таблица исходных данных
представлена в табл.1.
\end{multicols}

\tcap{Таблица 1 - Исходные данные для автоматизированных расчетов}
\begin{longtblr}[
  label = none,
  entry = none,
]{
  width = \linewidth,
  colspec = {Q[357]Q[92]Q[85]Q[140]Q[233]},
  cells = {font = \small},
  row{1} = {c},
  cell{2}{2} = {c},
  cell{2}{3} = {c},
  cell{2}{4} = {c},
  cell{2}{5} = {c},
  cell{3}{2} = {c},
  cell{3}{3} = {c},
  cell{3}{4} = {c},
  cell{3}{5} = {c},
  cell{4}{2} = {c},
  cell{4}{3} = {c},
  cell{4}{4} = {c},
  cell{4}{5} = {c},
  cell{5}{2} = {c},
  cell{5}{3} = {c},
  cell{5}{4} = {c},
  cell{5}{5} = {c},
  cell{6}{2} = {c},
  cell{6}{3} = {c},
  cell{6}{4} = {c},
  cell{6}{5} = {c},
  cell{7}{2} = {c},
  cell{7}{3} = {c},
  cell{7}{4} = {c},
  cell{7}{5} = {c},
  cell{8}{2} = {c},
  cell{8}{3} = {c},
  cell{8}{4} = {c},
  cell{8}{5} = {c},
  cell{9}{2} = {c},
  cell{9}{3} = {c},
  cell{9}{4} = {c},
  cell{9}{5} = {c},
  cell{10}{2} = {c},
  cell{10}{3} = {c},
  cell{10}{4} = {c},
  cell{10}{5} = {c},
  cell{11}{2} = {c},
  cell{11}{3} = {c},
  cell{11}{4} = {c},
  cell{11}{5} = {c},
  cell{12}{1} = {c=5}{0.937\linewidth},
  hlines,
  vlines,
}
\textbf{Параметр}                                                                   & \textbf{Обозн.}     & {\textbf{Ед.}\\\textbf{изм.}} & \textbf{Значение*} & \textbf{Примечание}                        \\
Диаметр дозирующего элемента                                                        & \textit{\textbf{d}} & см                            & 1,2                 & \textsuperscript{Конструктивный параметр}  \\
Длина дозирующего элемента                                                          & \textit{\textbf{l}} & см                            & 3,8                 & \textsuperscript{Конструктивный параметр}  \\
Ширина захвата агрегата                                                             & \textit{\textbf{B}} & м                             & 1,824               & \textsuperscript{Технический параметр}     \\
Ширина междурядий                                                                   & \textit{\textbf{b}} & см                            & 22,8                & \textsuperscript{Технический параметр}     \\
Длина пути агрегата                                                                 & \textit{\textbf{L}} & м                             & 5482,5              & \textsuperscript{Расчетный параметр}       \\
Скорость передвижения агрегата                                                      &                     & км/час                        & 7,0                 & \textsuperscript{Технологический параметр} \\
Объемный вес сыпучего вещества                                                      &                     & кг/м\textsuperscript{3}       & 810,0               & \textsuperscript{Справочный параметр}      \\
Норма внесения (высева)                                                             &                     & кг/га                         & 50,0                & \textsuperscript{Технологический параметр} \\
Число дозирующих элементов                                                          & \textit{\textbf{z}} & ед.                           & 12                  & \textsuperscript{Технический параметр}     \\
Число рабочих органов                                                               & \textit{\textbf{i}} & ед.                           & 9                   & \textsuperscript{Технологический параметр} \\
\textit{*условные значения для вычислительных экспериментов на базе сеялки СКС-2,1} &                     &                               &                     &                                            
\end{longtblr}

\begin{multicols}{2}
На рис.3, приведены результаты численных экспериментов в табличной
среде Excelпо исследованию дифференциальных закономерностей изменения
управляющей информации (\(\mathbf{n}_{\mathbf{вх}}\)) от отдельных
входных факторов.

Из анализа результатов вычислительных экспериментов следует, что
увеличение нормы внесения удобрений (рис.3\emph{а}) и скорости
прямолинейного движения агрегата (рис.3\emph{б}) сопровождается
пропорциональным ростом величины управляющего параметра, т.е. числа
оборотов (\(\mathbf{n}_{\mathbf{вх}}\)) на входе вала высевающего
аппарата. При этом интенсивность роста числа оборотов выше в зависимости
от увеличения значений нормы внесения удобрений. Изменение удельного
веса (рис.3\emph{в}) применяемого удобрения вносит изменения
управляющей информации в обратной пропорциональности. Здесь же
представлена экранная форма листинга.

Для инженерных расчетов с учетом фактических характеристик приводного
механизма следует ввести в формулу (8) коэффициент, учитывающий
передаточное число привода. Тогда формула имеет следующий вид, формула
9:
\end{multicols}

\begin{equation}
n_{\text{вх}} = \mu \cdot \prod_{j=1}^{\lambda} k_{j} \cdot \frac{M_{\text{н}} \cdot \upsilon_{\text{МТА}} \cdot B}{i \cdot z \cdot s \cdot l \cdot \gamma}
\end{equation}

где \(\mu\) - передаточное число привода.

\fig{i/image40}[Рис.3 - Результаты компьютерных экспериментов]

\begin{multicols}{2}
Эксплуатационно-технологическое тестирование работоспособности
экспериментального образца {[}14,15{]}, проведенное на опытном поле
Казахского научно-исследовательского института земледелия и
растениеводства (КазНИИЗиР) при внесении минеральных удобрений, показало
функциональной работоспособности автоматизированной системы в целом и
подтвердило корректность принятых методологических предпосылок.

{\bfseries Выводы.} Результаты исследования показали:

- цифровая трансформация производственно-технологических процессов
сельского хозяйства (ПТПСХ), представляет собой комплексную систему,
которая включает, материальные элементы (машины, агрегаты, сенсоры,
исполнительные механизмы) и нематериальные компоненты (программные
алгоритмы, цифровые модели, базы данных);

- такая интеграция, обеспечивает повышение эффективности производства,
полевых работ, мобильными агрегатами в пространственно-временных
координатах, а также способствует достижению ресурсосбережения, снижению
себестоимости, росту производительности на 8--10\% по результатам
экспериментальных испытаний;

- разработанные функциональные и процедурные модели процессов, в сфере
управления и программного обеспечения для описания последовательности
операций, алгоритмы принятия решений, управляющие воздействия, являются
основой для автоматизации технологических процессов внесения удобрений;

- разработанный алгоритм расчёта управляющей информации на входе
дозирующего механизма, обеспечивает повышение точности дозирования на
10-15\%, сокращает перерасход удобрений на 12-18\% по сравнению с
традиционными методами.

- результаты проектно-исследовательских изысканий в области создания
программно-аппаратных комплексов для системы точного земледелия, в сфере
управления и сфере программного обеспечения, дают возможность
дальнейшего практического и теоретического использования, как в
агропромышленном комплексе, так и в образовательных программах цифровой
агроинженерии.

\emph{{\bfseries Финансирование.} Материалы подготовлены в рамках
{\bfseries ИРН BR23992516} «Разработка и совершенствование технических
средств и технологического оборудования, обеспечивающих реализацию
научно-обоснованных технологий производства продукции растениеводства»}
\end{multicols}

\begin{center}
{\bfseries Литература}
\end{center}

\begin{refs}
1. Prins R. Making Precision Agriculture Work In Australia// Global Tech
Insight To Drive Agribusiness. -2017. URL:
http:precisionag.com/ag-tech-global/making-precision-agricultu-re-work-in-australia.
- Дата обращения:10.08.2025.

2. В Казахстане идет цифровая трансформация агрокомплекса. -2020. URL:
https://profit.kz. - Дата обращения:15.09.2025.

3. MacPherson J., Voglhuber-Slavinsky A., Olbrisch M., Schöbel P., Dönitz
E., Mouratiadou I., Helming K. Future agricultural systems and the
role of digitalization for achieving sustainability goals. A review//
Agronomy for Sustainable Development. -2022. -Vol.42(4): 70. DOI
\href{https://doi.org/10.1007/s13593-022-00792-6}{10.1007/s13593-022-00792-6}

4. McFadden J., Casalini F., Griffin T., Antón J. The digitalisation of
agriculture// OECD Food, Agriculture and Fisheries Papers. -2022.
-No.~176. DOI
\href{https://doi.org/10.1787/285cc27d-en}{10.1787/285cc27d-en}.

5. Celms А., et al. Global navigation satellite system as technical
solution element of farm-land processing in Latvia// Engineering for
Rural Development. -2015. -Vol.13. -P.44-50.

6. Say S.M., et al. Adoption of precision agriculture technologies in
developed and developing countries//The Online Journal of Science and
Technology. -2018. - Vol.8(1). -P.7-15.

7. Якушев В.В. Информационно-технологические основы прецизионного
производства растениеводческой продукции// Aвтореферат дис.. на док.
сельскохоз. наук: 06.01.03. -С.-Петербург. -2013. -32c.

8. Асташова Е.А., Кузнецова Н.А.,Зинич Л.В. Модель цифровой трансформации
преприятий АПК//Вопросы инновационной экономики. -2022. -Т.12 (4). -С.
2341-2356. DOI 10.18334/vinec.12.4.116890.

9. Балабанов В., Беленков А.И., Березовский Е.В. и др. Навигационные
технологии в сельском хозяйстве. Координатное земледелие. Учебное
пособие / Издательство российский государственный аграрный
университет-МСХА имени К.А. Тимирязева. - 2013. -117 с.

10. Монастырский В. А., Бабичев А.Н., Ольгаренко В.И. Алгоритм расчета доз
внесения удобрений в прецизионном земледелии // Научный журнал
Российского НИИ проблем мелиорации. -2019. -№ 1(33). -С.-26--38. DOI
10.31774/2222-1816-2019-1-26-38.

11. Шеуджен А.Х., Громова Л.И., Онищенко Л.М. Методы расчета доз
удобрений: учеб.пособие / Кубан. гос. агр. ун-т. -- Краснодар, 2010.
-- 61 с.

12. Буткевич Р.В., Клочков Ю.С., Яницкая Т.С., Ярыгин С.А. Методические
основы количественного оценивания технологических процеcсов //
Известия Самарского научного центра РАН. -2005. -T.7 (2). -С.
456--463.

13. ГОСТ 24055-2016. Межгосударственный стандарт. Техника
сельскохозяйственная Методы эксплуатационно-технологической оценки.
-2018. -23с.

14. ГОСТ 28714-2007. Межгосударственный стандарт. Машины для внесения
твердых минеральных удобрений. Методы испытаний.-2007. -44с.

15. Алтыбаев А.Н., Рзалиев А.С. и др. Цифровая трансформация системы
управления процессами подпочвенного внесения
удобрений//Научно-информационное обеспечение инно-вационного развития
АПК: материалы XVI Междунар. науч.-практ. Интернет-конф.
«ИнформАгро-2024». -2024. -805 с. URL:
\href{https://rosinformagrotech.ru/images/aspirantura/obrazproc/_конференции_8eef4.pdf}{https://rosinformagrotech.ru} .- Дата
обращения: 10.08.2025.
\end{refs}

\begin{center}
{\bfseries References}
\end{center}

\begin{refs}
1. Prins R. Making Precision Agriculture Work In Australia// Global Tech
Insight To Drive Agribusiness. -2017. URL:
http:precisionag.com/ag-tech-global/making-precision-agricultu-re-work-in-australia.
- Date of access:10.08.2025.

2. V Kazahstane idet cifrovaja transformacija agrokompleksa. -2020. URL:
https://profit.kz. - Data obrashhenija:15.09.2025. {[}in Russian{]}

3. MacPherson J., Voglhuber-Slavinsky A., Olbrisch M., Schöbel P.,
Dönitz E., Mouratiadou I., Helming K. Future agricultural systems and
the role of digitalization for achieving sustainability goals. A
review// Agronomy for Sustainable Development. -2022. -Vol.42(4): 70.
DOI
\href{https://doi.org/10.1007/s13593-022-00792-6}{10.1007/s13593-022-00792-6}

4. McFadden J., Casalini F., Griffin T., Antón J. The digitalisation of
agriculture// OECD Food, Agriculture and Fisheries Papers. -2022.
-No.~176. DOI
\href{https://doi.org/10.1787/285cc27d-en}{10.1787/285cc27d-en}.

5. Celms А., et al. Global navigation satellite system as technical
solution element of farm-land processing in Latvia// Engineering for
Rural Development. -2015. -Vol.13. -P.44-50.

6. Say S.M., et al. Adoption of precision agriculture technologies in
developed and developing countries//The Online Journal of Science and
Technology. -2018. - Vol.8(1). -P.7-15.

7. Jakushev V.V. Informacionno-tehnologicheskie osnovy precizionnogo
proizvodstva rastenievodcheskoj produkcii// Avtoreferat dis.. na dok.
sel' skohoz. nauk: 06.01.03. -S.-Peterburg. -2013. -32c.
{[}in Russian{]}

8. Astashova E.A., Kuznecova N.A.,Zinich L.V. Model'{}
cifrovoj transformacii preprijatij APK//Voprosy innovacionnoj
jekonomiki.-2022.-T.12(4).-S.2341-2356. DOI 10.18334/vinec.12.4.116890.
{[}in Russian{]}

9. Balabanov V., Belenkov A.I., Berezovskij E.V. i dr. Navigacionnye
tehnologii v sel' skom hozjajstve. Koordinatnoe
zemledelie. Uchebnoe posobie / Izdatel' stvo rossijskij
gosudarstvennyj agrarnyj universitet-MSHA imeni K.A. Timirjazeva. -
2013. -117 s. {[}in Russian{]}

10. Monastyrskij V. A., Babichev A.N., Ol' garenko V.I.
Algoritm rascheta doz vnesenija udobrenij v precizionnom zemledelii //
Nauchnyj zhurnal Rossijskogo NII problem melioracii. -2019. -№ 1(33).
-S.-26-38. DOI 10.31774/2222-1816-2019-1-26-38. {[}in Russian{]}

11. Sheudzhen A.H., Gromova L.I., Onishhenko L.M. Metody rascheta doz
udobrenij: ucheb.posobie / Kuban. gos. agr. un-t. -Krasnodar, 2010. - 61
s. {[}in Russian{]}

12. Butkevich R.V., Klochkov Ju.S., Janickaja T.S., Jarygin S.A.
Metodicheskie osnovy kolichestvennogo ocenivanija tehnologicheskih
procecsov // Izvestija Samarskogo nauchnogo centra RAN. -2005. -T.7
(2). -S.456-463. {[}in Russian{]}

13. GOST 24055-2016. Mezhgosudarstvennyj standart. Tehnika
sel' skohozjajstvennaja Metody
jekspluatacionno-tehnologicheskoj ocenki. -2018. -23s. {[}in Russian{]}

14. GOST 28714-2007. Mezhgosudarstvennyj standart. Mashiny dlja
vnesenija tverdyh mineral' nyh udobrenij. Metody
ispytanij.-2007. -44s. {[}in Russian{]}

15. Altybaev A.N., Rzaliev A.S. i dr. Cifrovaja transformacija sistemy
upravlenija processami podpochvennogo vnesenija
udobrenij//Nauchno-informacionnoe obespechenie inno-vacionnogo razvitija
APK: materialy XVI Mezhdunar. nauch.-prakt. Internet-konf.
«InformAgro-2024». -2024. -805 s. URL:
\href{https://rosinformagrotech.ru/images/aspirantura/obrazproc/_конференции_8eef4.pdf}{https://rosinformagrotech.ru}
.- Date of access: 10.08.2025. {[}in Russian{]}
\end{refs}

\begin{info}
\hspace{1em}\emph{{\bfseries Информация об авторах}}

Алтыбаев А.Н. - доктор ттехнических наук, ассоциированный профессор
(доцент), ТОО «Научно-производственный центр агроинженерии», Алматы,
Казахстан, e-mail: narikovich@yandex.ru;

Жұмағали С. Ж. - магистр технических наук, ТОО
«Научно-производственный центр агроинженерии», Алматы, Казахстан,
e-mail:Sabr.hz@mail.ru;

Конысбаев Е.К.- инженер-механик, ТОО «Научно-производственный центр
агроинженерии», Алматы, Казахстан, e-mail: erkegali@mail.ru;

Бекмухамедов Б. Э. - кандидат технических наук, ведущий научный
сотрудник,ТОО «Научно-производственный центр агроинженерии», Алматы,
Казахстан, e-mail:baur\_gis@mail.ru;

Акишев К.М.- кандидат технических наук, ассоциированный профессор
(доцент), Казахский университет технологии и бизнеса
им. К. Кулажанова, Астана, Казахстан, e-mail: akmail04cx@mail.ru.

\hspace{1em}\emph{{\bfseries Information about the authors}}

Altybaev A.N.- Doctor of Technical Sciences, associate professor,
Scientific and Production Center of Agroengineering LLP, Almaty,
Kazakhstan, e-mail: narikovich@yandex.ru;

Jumagali S. Zh.- Master of Technical Sciences, Senior Engineer,
Scientific and Production Center of Agroengineering LLP, Almaty,
Kazakhstan, e-mail: Sabr.hz@mail.ru;

Konysbaev E.K.- Mechanical engineer, Scientific and Production Center
of Agroengineering LLP, Almaty, Kazakhstan, e-mail: erkegali@mail.ru;

Bekmukhamedov B.E.- Candidate of Technical Sciences, leading
researcher, Scientific and Production Center of Agroengineering LLP,
Almaty, Kazakhstan, e-mail: baur\_gis@mail.ru;

Akishev K.M. - Candidate of Technical Sciences, Associate Professor
(docent), K. Kulazhanov Kazakh University of Technology and Business,
Astana, Republic of Kazakhstan, e-mail: akmail04cx@mail.ru.
\end{info}
