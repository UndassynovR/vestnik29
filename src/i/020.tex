\id{МРНТИ 20.23.21}{}

\begin{header}
\swa{}{МОДЕЛИРОВАНИЕ СЦЕНАРИЕВ НАРУШЕНИЙ ЦЕЛОСТНОСТИ ДАННЫХ В ИНТЕЛЛЕКТУАЛЬНЫХ СИСТЕМАХ КОММУНИКАЦИОННОГО ВЗАИМОДЕЙСТВИЯ}

\tsp{1} Л.О. Жумабаева,
\tsp{2}Mohamed Othman,
\tsp{1}М.А. Орынбасар\envelope,
\tsp{1}Б.А. Жұмажан,
\tsp{1}М.Е. Ақбердиева
\end{header}

\begin{affil}
\tsp{1}Каспийский университет технологий и инжиниринга имени Ш. Есенова, г. Актау, Казахстан

\tsp{2}Universiti Putra Malaysia, Malaysia

\corrauthor{Корреспондент автор: maksym1.orynbassar@yu.edu.kz}
\end{affil}

В статье рассматриваются интеллектуальные социальные системы (ИСС)
представляют собой интегрированные цифровые экосистемы, объединяющие
пользователей, аналитические сервисы и алгоритмы искусственного
интеллекта для управления коммуникациями, репутационными процессами и
общественным взаимодействием. К таким системам относятся социальные
сети, государственные платформы цифрового взаимодействия («Госуслуги»,
«Активный гражданин»), краудсорсинговые площадки и образовательные
онлайн-экосистемы.

Главная особенность ИСС заключается в их высокой социальной значимости
данных. Ошибки, подмены или целенаправленные манипуляции в
информационных потоках ведут не только к технологическим, но и к
социальным последствиям --- снижению уровня доверия пользователей,
распространению дезинформации, искажению общественного мнения и
результатов электронных голосований.Таким образом, обеспечение
целостности данных в ИСС становится одной из ключевых задач цифрового
общества. Необходимо не только защитить данные от внешних воздействий,
но и гарантировать корректность, достоверность и непротиворечивость их
формирования, хранения и обработки.

{\bfseries Ключевые слова:} интеллектуальные социальные системы,
имитационное моделирование, целостность данных, когнитивные атаки,
информационные искажения, агентное моделирование, машинное обучение,
нейросетевые алгоритмы, графовые нейронные сети, автоэнкодеры, LSTM,
цифровое доверие, поведенческие аномалии, анализ информационных потоков,
обеспечение безопасности данных.

\begin{header}
ИНТЕЛЛЕКТУАЛДЫ КОММУНИКАЦИЯЛЫҚ ӨЗАРА ӘРЕКЕТТЕСУ ЖҮЙЕЛЕРІНДЕГІ ДЕРЕКТЕР ТҰТАСТЫҒЫНЫҢ БҰЗЫЛУ СЦЕНАРИЙЛЕРІН МОДЕЛЬДЕУ

\tsp{1}Л.О. Жумабаева,
\tsp{2}Mohamed Othman,
\tsp{1}М.А. Орынбасар,
\tsp{1}Б.А. Жұмажан\envelope,
\tsp{1}М.Е. Ақбердиева
\end{header}

\begin{affil}
\tsp{1}Каспийский университет технологий и инжиниринга имени Ш. Есенова, г. Актау, Казахстан,

\tsp{2}Universiti Putra Malaysia, Malaysia,

e-mail: maksym1.orynbassar@yu.edu.kz
\end{affil}

Мақалада интеллектуалды әлеуметтік жүйелер (ИАЖ) интеграцияланған
цифрлық экожүйелер ретінде қарастырылады, олар пайдаланушыларды,
талдамалық сервистерді және жасанды интеллект алгоритмдерін біріктіріп,
коммуникацияларды, беделдік процестерді және қоғамдық өзара іс-қимылды
басқаруға мүмкіндік береді. Мұндай жүйелерге әлеуметтік желілер, цифрлық
өзара әрекеттесу мемлекеттік платформалары («Электрондық үкімет»,
«Белсенді азамат»), краудсорсинг алаңдары және онлайн-білім беру
экожүйелері жатады.ИАЖ-дың негізгі ерекшелігі -- деректердің жоғары
әлеуметтік маңыздылығы. Ақпараттық ағындардағы қателіктер, алмастырулар
немесе әдейі жасалған манипуляциялар тек технологиялық емес, сонымен
қатар әлеуметтік салдарға әкеледі --- пайдаланушылар сенімінің
төмендеуі, дезинформацияның таралуы, қоғамдық пікір мен электрондық
дауыс беру нәтижелерінің бұрмалануы.

Сондықтан деректердің тұтастығын қамтамасыз ету цифрлық қоғамның негізгі
міндеттерінің бірі болып табылады. Деректерді сыртқы әсерлерден қорғап
қана қоймай, олардың қалыптасуының, сақталуының және өңделуінің
дұрыстығын, сенімділігін және қайшылықсыздығын қамтамасыз ету қажет.

{\bfseries Түйін сөздер:} интеллектуалды әлеуметтік жүйелер, имитациялық
модельдеу, деректердің тұтастығы, когнитивтік шабуылдар, ақпараттық
бұрмалау, агенттік модельдеу, машиналық оқыту, нейрожелілік алгоритмдер,
графтық нейрондық желілер, автоэнкодерлер, LSTM, цифрлық сенім,
мінез-құлық аномалиялары, ақпараттық ағындарды талдау, деректер
қауіпсіздігін қамтамасыз ету.

\begin{header}
MODELING SCENARIOS OF DATA INTEGRITY VIOLATIONS IN INTELLIGENT COMMUNICATIVE INTERACTION SYSTEMS

\tsp{1}L.O. Zhumabayeva,
\tsp{2}Mohamed Othman,
\tsp{1}M.A. Orynbassar,
\tsp{1}B.A. Zhumazhan,
\tsp{1}M.E. Akberdiyeva
\end{header}

\begin{affil}
\tsp{1}Sh. Yessenov Caspian University of Technology and Engineering, Aktau, Kazakhstan,

\tsp{2}Universiti Putra Malaysia, Malaysia,

e-mail: maksym1,orynbassar@yu.edu.kz
\end{affil}

The article is examined intelligent social systems (ISS) as integrated
digital ecosystems that unite users, analytical services, and artificial
intelligence algorithms to manage communications, reputation processes,
and public interaction. Such systems include social networks, government
digital interaction platforms (``Gosuslugi'', ``Active Citizen''),
crowdsourcing platforms, and educational online ecosystems. The main
feature of ISS lies in the high social significance of their data.
Errors, substitutions, or deliberate manipulations in information flows
lead not only to technological but also to social consequences ---
reducing user trust, spreading disinformation, and distorting public
opinion and electronic voting results.

Thus, ensuring data integrity in ISS becomes one of the key challenges
of the digital society. It is necessary not only to protect data from
external influences but also to guarantee the correctness, reliability,
and consistency of their formation, storage, and processing.

{\bfseries Keywords:} intelligent social systems, simulation modeling, data
integrity, cognitive attacks, information distortion, agent-based
modeling, machine learning, neural network algorithms, graph neural
networks, autoencoders, LSTM, digital trust, behavioral anomalies,
information flow analysis, data security assurance.

\begin{multicols}{2}
{\bfseries Introduction.} The modern information space is characterized by
a high degree of interconnectedness, rapid data exchange, and the
widespread implementation of artificial intelligence (AI) technologies
in social and managerial processes. Against this backdrop, a special
role is played by intelligent social systems (ISS) - integrated digital
ecosystems that bring together users, analytical services, and AI
algorithms to manage communication, reputation processes, and public
interaction. Such systems include social networks, government digital
interaction platforms (e.g., ``Gosuslugi'', ``Active Citizen''),
crowdsourcing platforms, and online educational ecosystems, in which
vast volumes of user data continuously circulate.

In the context of the digital transformation of society, data become a
strategic resource that determines the effectiveness of decision-making,
the level of trust in digital platforms, and the resilience of social
infrastructure. Violations of data integrity - whether caused by
technical failures, deliberate manipulations, or cognitive distortions -
have a significant impact on the reliability of information, undermine
the reputation of digital services, and can lead to the destabilization
of social processes.

Data integrity in intelligent systems is not limited to preservation and
immutability; it also includes ensuring semantic, behavioral, and
temporal consistency. Traditional protection methods based on
cryptographic approaches (hashing, digital signatures, backup
mechanisms) are proving insufficient under current conditions. They do
not take into account the socio-cognitive aspects of digital
interaction, where the source of distortion may be the behavior of users
themselves, of algorithms, or of bots.

In this regard, the development of intelligent methods for analysis and
modeling capable of detecting hidden threats and forecasting scenarios
of data degradation in ISS becomes an urgent scientific challenge. One
of the most promising approaches to addressing this problem is
simulation modeling, which makes it possible to reproduce complex
socio-informational processes, investigate the dynamics of data
integrity violations, and assess the impact of various factors on their
propagation.

Simulation modeling, when combined with machine learning methods and
neural network analysis, enables the creation of experimental digital
twins of social systems, where the patterns of cognitive attacks, fake
activity, and other forms of anomalies can be studied. This opens up new
prospects for building digital trust systems in which intelligent
algorithms perform functions of diagnostics, adaptation, and restoration
of data integrity in real time.

Thus, this study is aimed at developing a methodology for simulation
modeling of data integrity violations in intelligent social systems and
at creating analytical tools capable of enhancing the reliability and
resilience of digital ecosystems.

{\bfseries Literature Review.} The problem of ensuring data integrity in
intelligent social systems (ISS) is one of the key topics in
contemporary digital science. In recent years, interest has grown
significantly in studying the relationship between the technical and
social aspects of ISS functioning, where data violations are considered
not only as the result of cyberattacks, but also as a consequence of
cognitive biases, user behavior, and algorithmic anomalies. Modern
researchers note that intelligent social systems represent complex
human--machine ecosystems in which information flows are formed
simultaneously at the technological and behavioral levels. Therefore,
classical information protection methods based solely on cryptographic
principles are becoming insufficient. There is a need for comprehensive
approaches that combine simulation modeling, machine learning, and
behavioral analysis.

According to studies in the field of digital security, data integrity
violations in ISS often have a systemic character: an error or
distortion at one level leads to cascading consequences in other
subsystems. This requires constructing models that take into account the
dynamics of interrelations and the behavior of agents within the
network. Under such conditions, simulation models make it possible to
forecast scenarios of data degradation and to assess the system's
resilience to external and internal threats {[}1{]}.

The works of foreign authors highlight the effectiveness of agent-based
modeling (ABM) for analyzing social interactions and the spread of
informational distortions {[}2{]}. Models based on graph neural networks
(GNN) make it possible to identify structural dependencies and detect
anomalies in users' network activity. In addition, domestic researchers
emphasize the role of learning analytics and digital trust in the design
of ISS. Maintaining data integrity in educational and social ecosystems
requires diagnostic and adaptive tools that are capable not only of
recording violations but also of preventing them through intelligent
self-learning mechanisms of the system.

A physiological analogue of such processes can be found in the
adaptation of living systems to stress conditions, where the system,
when facing external influences, develops resilience and compensation
mechanisms {[}3{]}. Similarly, intelligent social systems must possess
the property of digital resilience that allows them to preserve the
reliability and consistency of data in the face of failures and
information attacks.

In recent years, special attention has been paid to cognitive attacks
and informational distortions arising from manipulations in digital
communication processes {[}4{]}. These phenomena require a comprehensive
analysis that integrates psychological, technical, and behavioral
aspects. The use of simulation models makes it possible to reproduce the
process of false information dissemination and to evaluate the
effectiveness of detection algorithms.

Based on the literature analysis, several key research directions in the
field of data integrity modeling can be distinguished:

- development of multi-level modeling architectures that integrate
technical and cognitive components;

- application of machine learning and neural networks for the analysis
of network interactions and anomalies;

- investigation of digital trust models and ethical principles of ISS
functioning;

- implementation of simulation models for diagnosing and forecasting
data integrity violations in real time.

Thus, the literature review shows that simulation modeling of data
integrity violations is a relevant research area that integrates
interdisciplinary approaches from artificial intelligence,
cybersecurity, learning analytics, and socio-technical design. The
development of such models contributes to the formation of intelligent
digital trust systems, which is of strategic importance for the digital
transformation of society.

{\bfseries Materials and methods.} To test the hypothesis regarding the
possibility of predicting and detecting data integrity violations in
intelligent social systems (ISS), a comprehensive experiment was
conducted based on simulation modeling and machine learning methods. The
study included several stages: model design, computational experiments,
and analysis of the results.

The methodological framework of the study was based on the principle of
three-level modeling, which involves decomposing the system into
interrelated levels: user, informational, and infrastructural. At the
user level, the behavior of agents was modeled to imitate user
interactions in social networks, government digital services, and
crowdsourcing platforms. The informational level reflected the dynamics
of data flows, their transmission, transformation, and potential
distortions under the influence of external and cognitive factors. The
infrastructural level included network connections, server nodes, and
communication channels that ensure storage, processing, and
synchronization of information.

The simulation model was implemented using the AnyLogic software
environment and Python (SimPy and Pandas libraries). This made it
possible to combine visual modeling with analytical calculation of
indicators and statistical data processing. At the first stage,
agent-based scenarios of user interaction were created, with different
levels of activity, degrees of trust in information sources, and
probabilities of spreading distorted information.

To assess system resilience and data integrity, the following
computational indicators were used: Integrity Rate, Distortion
Propagation Factor, Resilience Index, and Entropy Shift of the
information space. These parameters made it possible to quantitatively
measure system stability under the influence of various risk factors and
cognitive attacks.

Two experimental models were used in the study. The control model
reflected the traditional architecture of ISS without intelligent data
recovery mechanisms, while the experimental version included machine
learning algorithms capable of automatically detecting anomalies and
restoring distorted data. Deep learning methods were applied for
analysis: autoencoders, recurrent neural networks (LSTM), and graph
neural networks (GNN). Autoencoders were used to detect behavioral
anomalies, LSTM networks to analyze temporal dependencies in data flows,
and GNN models to study network structures and relationships between
nodes.

In the course of the computational experiment, 10,000 virtual agents
interacting over 1,000 discrete time steps were simulated. Each agent
was characterized by individual parameters: level of trust, activity
frequency, propensity to spread information, and threshold of critical
perception {[}5{]}.

The results of the experiment showed that, in the control model, the
level of data integrity decreased on average by 28\% under the influence
of external cognitive attacks and fake messages {[}6{]}. In the
experimental model equipped with self-learning mechanisms, the Integrity
Rate remained stable and decreased by no more than 7-10\%, which
indicates the high effectiveness of the proposed approach.

A correlation analysis was also carried out between the structure of
social ties and the probability of data distortion. It was found that a
high density of links in local clusters accelerates the spread of
distortions; however, the application of adaptive trust algorithms and
content filtering can significantly reduce this effect.

To visualize the results, AnyLogic analytical dashboards were used,
while statistical data processing was performed in Python. The
constructed dependencies between attack frequency, trust coefficient,
and changes in the level of data integrity made it possible to determine
critical thresholds of system resilience.

Special attention in the methodology was paid to the pedagogical aspect
of the functioning of the intelligent system. A feedback mechanism was
implemented, whereby agents adjusted their behavior depending on
previous iterations, which modeled the process of self-learning and
adaptation to new types of threats. This approach increased the accuracy
of the model and the resilience of the system in a dynamic digital
environment.

Overall, the study confirmed that the use of simulation modeling in
combination with machine learning methods makes it possible to reproduce
complex scenarios of data degradation and to design effective mechanisms
for their automatic recovery. The developed methodology is universal and
can be used to analyze various types of digital platforms, including
social networks, educational systems, and government information
systems.

{\bfseries Results and Discussion.} The main difficulty in organizing the
experiment was that the simulated intelligent social system represented
a dynamic environment with high variability of parameters. At the
initial stages of modeling, challenges arose related to tuning the
behavioral parameters of agents and calibrating trust coefficients,
which required significant computational resources and time for training
the neural network models. In addition, the complexity of the experiment
lay in the need to synchronize data between the agent-based and
infrastructural levels, which required the introduction of an additional
mechanism for correlating network events with behavioral patterns.

Initial modeling showed that, in the baseline architecture of the
intelligent social system, the level of data integrity decreased by
25-30\% under the influence of external cognitive attacks and
disinformation flows {[}7{]}. At the same time, high sensitivity to the
parameters of social tie density was observed: the higher the number of
interactions between agents, the faster distorted messages spread. This
effect is analogous to viral propagation, which confirms the hypothesis
of a nonlinear relationship between the degree of user engagement and
the rate of loss of data reliability.

After the introduction of intelligent correction mechanisms based on
machine learning, a stable improvement in data integrity indicators was
observed. The experimental model using autoencoders and graph neural
networks demonstrated an average increase of 18\% in the Integrity Rate
compared to the control model. The Resilience Index increased by 22\%,
while the Distortion Propagation Factor almost halved {[}8{]}.

Figure 1 presents a visualization of the architecture of the three-level
simulation model, showing the relationships between agents, data flows,
and infrastructural nodes. The lower level of the illustration reflects
user activity and the density of their social ties, the middle level
shows the dynamics of information flows, and the upper level represents
the network structure that provides data transmission and storage.

Figure 2 shows the dynamics of changes in the Integrity Rate during the
simulation experiment. A clear positive trend can be observed in the
experimental model equipped with intelligent self-learning algorithms,
where the decline in data reliability occurred significantly more slowly
than in the baseline configuration.

In the course of analyzing time series and graph relationships, it was
found that the most vulnerable nodes of the system are agents with a
high degree of betweenness centrality. These nodes become critical
points for the spread of distortions and fake messages. The use of
adaptive trust mechanisms made it possible to significantly reduce the
influence of such nodes on overall data integrity.

The experiment also examined the impact of agent self-learning
parameters on system resilience. It was established that increasing the
model update frequency by 15--20\% leads to an increase in anomaly
classification accuracy up to 92\%, which confirms the effectiveness of
incorporating neural network methods into the simulation architecture
{[}9{]}.
\end{multicols}

\fig[0.45\textwidth]{i3/image151}[Fig.1 - Architecture of the simulation model of an intelligent social system]
\fig[0.5\textwidth]{i3/image152}[Fig.2 - Dynamics of the data integrity coefficient (Integrity Rate) during modeling]

\begin{multicols}{2}
The modeling results showed that the use of a hybrid approach combining
agent-based modeling and machine learning makes it possible not only to
detect data distortions but also to predict their occurrence. This is
particularly important for intelligent social systems in which
information processes occur in real time and are influenced by cognitive
and behavioral factors

At the same time, the experiment confirmed the high significance of the
pedagogical aspect in the functioning of intelligent systems. The
feedback and self-learning mechanism of agents can be regarded as an
analogue of pedagogical support, where each element of the system adapts
based on experience and accumulated data {[}10{]}. This makes the system
more flexible and resilient to external influences and contributes to
the formation of a trusted digital environment.

In conclusion, it should be noted that the experimental results
confirmed the feasibility of using simulation modeling in combination
with neural network analysis methods. This approach provides a
comprehensive understanding of the mechanisms of data integrity
violation and recovery and makes it possible to create intelligent tools
for monitoring, prevention, and automatic response to information
threats {[}11{]}. The obtained results can be used in the development of
digital governance systems, educational and governmental platforms, as
well as in solving cybersecurity and cyber-resilience tasks for social
networks.

{\bfseries Conclusion.} It has been established that the proposed
methodology, based on simulation modeling and the application of machine
learning algorithms, makes it possible to effectively detect and
forecast data integrity violations in intelligent social systems. The
conducted experiment confirmed that the use of a three-level model
architecture - with a separation into user, informational, and
infrastructural levels - provides a comprehensive understanding of the
dynamics of information processes and the mechanisms of data distortion.

The integration of neural network algorithms, including autoencoders,
graph neural networks, and recurrent networks, increased the accuracy of
anomaly detection and made it possible to stabilize the data integrity
coefficient under the influence of external cognitive attacks. The
modeling results showed that the introduction of self-learning
mechanisms and adaptive trust contributes to enhancing the digital
resilience of the system and reducing the likelihood of false
information dissemination.

Thus, the proposed approach, which combines methods of simulation
modeling, machine learning, and user behavior analysis, can be
recommended as an effective tool for the design and monitoring of
intelligent social systems. It ensures not only technical reliability
but also forms a foundation for building digital trust systems focused
on maintaining information reliability and supporting the sustainable
development of the digital society.

Prospects for further research are associated with improving the
proposed model, expanding the set of agent self-learning parameters, and
integrating cognitive and ethical factors into the modeling process.
This will make it possible to create more realistic and adaptive digital
systems capable of autonomously preventing threats and maintaining data
integrity in a dynamically changing information environment.
\end{multicols}

\begin{center}
{\bfseries Литература}
\end{center}

\begin{refs}
1. Garzón J., Patiño E., Marulanda C. Systematic Review of Artificial
Intelligence in Education: Trends, Benefits, and Challenges //
Multimodal Technologies and Interaction. - 2025.- Vol.9(8). - Art.84.
DOI 10.3390/mti9080084.

2. Wang S. et al. Artificial Intelligence in Education: A Systematic
Literature Review//Expert Systems with Applications.-
2024. -Vol.252:124167. DOI 10.1016/j.eswa.2024.124167.

3. Chen L., Chen P., Lin Z. Artificial Intelligence in Education: A
Review // IEEE Access. - 2020. -Vol.8. - P.75264-75278. DOI
10.1109/ACCESS.2020.2988510.

4. Zhai X., Chu X., Chai C.S., Liu J. A Review of Artificial
Intelligence (AI) in Education from 2010 to 2020 //Complexity.-2021.-Vol
2021(1):8812542. - DOI 10.1155/2021/8812542.

5. Yağcı M. Educational Data Mining: Prediction of Students' Academic
Performance Using Machine Learning Algorithms//Smart Learning
Environments.-2022.-Vol.9:11. DOI 10.1186/s40561-022-00192-z.

6. Staneviciene E., Gudoniene D., Punys V., Kukstys A. A Case Study on
the Data Mining-Based Prediction of Students' Performance for Effective
and Sustainable E-Learning // Sustainability. -- 2024. -Vol.
16(23):10442. DOI 10.3390/su162310442.

7. Moreno-Marcos P.M., Pong T.C., Muñoz-Merino P.J., Delgado Kloos C.
Analysis of the Factors Influencing Learners' Performance Prediction
With Learning Analytics // IEEE Access. -- 2020. -- Vol.8. - P.
5264-5282. DOI 10.1109/ACCESS.2019.2963503.

8. Ifenthaler D., Yau J.Y.-K. Utilising Learning Analytics to Support
Study Success in Higher Education: A Systematic Review // Educational
Technology Research and Development. - 2020. - Vol.68(4). - P.
1961-1990. DOI 10.1007/s11423-020-09788-z.

9. Wang X., Li L., Tan S.C., Yang L., Lei J. Preparing for AI-Enhanced
Education: Conceptualizing and Empirically Examining Teachers' AI
Readiness // Computers in Human Behavior. -2023. -Vol.146: 107798. DOI
10.1016/j.chb.2023.107798.

10. Sysoev P.V. Искусственный интеллект в образовании: осведомлённость,
готовность и практика применения преподавателями высшей школы технологий
искусственного интеллекта в профессиональной деятельности // Высшее
образование в России. - 2023. - Т.32(10) - С.9-33. DOI
10.31992/0869-3617-2023-32-10-9-33. {[}in Russian{]}.

11. Тихонова Н.В., Сабирова Д.Р. Грамотность педагога в области
искусственного интеллекта: теоретический анализ понятия // Образование и
наука. -2025. -Т.27(6). - С.180-206. DOI
10.17853/1994-5639-2025-6-180-206. {[}in Russian{]}.
\end{refs}

\begin{center}
{\bfseries References}
\end{center}

\begin{refs}
1. Garzón J., Patiño E., Marulanda C. Systematic Review of Artificial
Intelligence in Education: Trends, Benefits, and Challenges //
Multimodal Technologies and Interaction. - 2025.- Vol.9(8). - Art.84.
DOI 10.3390/mti9080084.

2. Wang S. et al. Artificial Intelligence in Education: A Systematic
Literature Review//Expert Systems with Applications.-
2024. -Vol.252:124167. DOI 10.1016/j.eswa.2024.124167.

3. Chen L., Chen P., Lin Z. Artificial Intelligence in Education: A
Review // IEEE Access. - 2020. -Vol.8. - P.75264-75278. DOI
10.1109/ACCESS.2020.2988510.

4. Zhai X., Chu X., Chai C.S., Liu J. A Review of Artificial
Intelligence (AI) in Education from 2010 to 2020 //Complexity.-2021.-Vol
2021(1):8812542. - DOI 10.1155/2021/8812542.

5. Yağcı M. Educational Data Mining: Prediction of Students' Academic
Performance Using Machine Learning Algorithms//Smart Learning
Environments.-2022.-Vol.9:11. DOI 10.1186/s40561-022-00192-z.

6. Staneviciene E., Gudoniene D., Punys V., Kukstys A. A Case Study on
the Data Mining-Based Prediction of Students' Performance for Effective
and Sustainable E-Learning // Sustainability. -- 2024. -Vol.
16(23):10442. DOI 10.3390/su162310442.

7. Moreno-Marcos P.M., Pong T.C., Muñoz-Merino P.J., Delgado Kloos C.
Analysis of the Factors Influencing Learners' Performance Prediction
With Learning Analytics // IEEE Access. -- 2020. -- Vol.8. - P.
5264-5282. DOI 10.1109/ACCESS.2019.2963503.

8. Ifenthaler D., Yau J.Y.-K. Utilising Learning Analytics to Support
Study Success in Higher Education: A Systematic Review // Educational
Technology Research and Development. - 2020. - Vol.68(4). - P.
1961-1990. DOI 10.1007/s11423-020-09788-z.

9. Wang X., Li L., Tan S.C., Yang L., Lei J. Preparing for AI-Enhanced
Education: Conceptualizing and Empirically Examining Teachers' AI
Readiness // Computers in Human Behavior. -2023. -Vol.146: 107798. DOI
10.1016/j.chb.2023.107798.

10. Sysoev P.V. Iskusstvennyj intellekt v obrazovanii:
osvedomljonnost', gotovnost'{} i praktika
primenenija prepodavateljami vysshej shkoly tehnologij iskusstvennogo
intellekta v professional' noj
dejatel' nosti // Vysshee obrazovanie v Rossii. - 2023. -
T.32(10) - S.9-33. DOI 10.31992/0869-3617-2023-32-10-9-33. {[}in
Russian{]}.

11. Tihonova N.V., Sabirova D.R. Gramotnost'{} pedagoga v
oblasti iskusstvennogo intellekta: teoreticheskij analiz ponjatija //
Obrazovanie i nauka. -2025. -T.27(6). - S.180-206. DOI
10.17853/1994-5639-2025-6-180-206. {[}in Russian{]}.
\end{refs}

\begin{info}
\hspace{1em}\emph{{\bfseries Information about the authors}}

Zhumabayeva L. O.- PhD, Associate Professor, Sh. Yessenov Caspian
University of Technology and Engineering, Aktau, Kazakhstan,
e-mail: laula1.zhumabayeva@yu.edu.kz;

Mohamed Othman - PhD, Professor, Universiti Putra Malaysia, Putrajaya,
Malaysia, e-mail: mothman@fsktm.upm.edu.my;

Orynbassar M.A. - Assistant of the Department of Computer
Technologies, Sh. Yessenov Caspian University of Technology and
Engineering, Aktau, Kazakhstan, e-mail:
maksym1.orynbassar@yu.edu.kz;

Zhumazhan B.А. - Assistant of the Department of SMART Technologies,
Sh.  Yessenov Caspian University of Technology and Engineering, Aktau,
Kazakhstan, e-mail: bekezhan1.zhumazhan@yu.edu.kz;

Akberdiyeva M.Ye.- Master's Student of the Department of Computer
Technologies, Aktau, Kazakhstan, e-mail:
meruyert1.akberdiyeva@yu.edu.kz.

\hspace{1em}\emph{{\bfseries Сведения об авторах}}

Жумабаева Л.О.- PhD, ассоцированный профессор, Каспийский Университет
технологий и инжиниринга им. Ш. Есенова, Актау,Казахстан, e-mail:
laula1.zhumabayeva@yu.edu.kz;

Mohamed Othman - PhD, профессор, Universiti Putra Malaysia, Putrajaya,
Malaysia, e-mail: mothman@fsktm.upm.edu.my;

Орынбасар М.А. - ассистент кафедры «Компьютерных технологий»,
Каспийский Университет технологий и инжиниринга им. Ш. Есенова, Актау,
Казахстан, e-mail: maksym1.orynbassar@yu.edu.kz;

Жумажан Б.А.- ассистент кафедры «SMART технологий», Каспийский
Университет технологий и инжиниринга им. Ш. Есенова, Актау, Казахстан,
e-mail: bekezhan1.zhumazhan@yu.edu.kz;

Акбердиева М. Е.- магистрант кафедры «Компьютерных технологий»,
Каспийский Университет технологий и инжиниринга им. Ш. Есенова, Актау,
Казахстан, e-mail: meruyert1.akberdiyeva@yu.edu.kz,.
\end{info}
