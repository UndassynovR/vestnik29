\id{ҒТАМР 50.13.13}{}

{\bfseries SLM ӘДІСІМЕН ҮШ ӨЛШЕМДІ ОБЪЕКТІНІ ҚАЛЫПТАСТЫРУ ПРОЦЕСІН
АВТОМАТТАНДЫРУҒА АРНАЛҒАН БАҒДАРЛАМАЛЫҚ-ТЕХНИКАЛЫҚ ҚҰРЫЛҒЫЛАР}

{\bfseries \tsp{1}А.Д.
Тулегулов}{\bfseries ,
\tsp{1}К.М.
Акишев}{\bfseries \envelope ,
\tsp{1}А.М.
Джумагалиева}{\bfseries ,
\tsp{2}М.А.
Байжарикова},

{\bfseries \tsp{2}С.М.
Исаев}

\emph{\tsp{1}Қ. Құлажанов атындағы Қазақ технология және
бизнес университеті, Астана, Қазақстан,}

\emph{\tsp{2}М. Х. Дулати атындағы Тараз университеті,
Тараз, Қазақстан}

\corrauthor{Корреспондент-автор:akmail04cx@mail.ru}

Қазіргі жағдайда автоматтандырылған бағдарламаланатын автономды
өндірістік жүйелерді құру қажеттілігі әсіресе әскери және далалық
инженерлік бөлімшелер үшін өзекті болып отыр. Дәстүрлі селективті
лазерлік балқыту қондырғылары (SLM), алынған өнімдердің жоғары дәлдігі
мен сапасына қарамастан, дизайнның күрделілігіне, компоненттердің
қымбаттығына және қоршаған орта параметрлеріне жоғары сезімталдыққа
байланысты зертханалық жағдайлардан тыс жұмыс істеуге жарамсыз. Осыған
байланысты мақалада SLM әдісімен үш өлшемді объектіні қалыптастыру
процесін автоматтандыру үшін бағдарламалық-техникалық құралдарды зерттеу
мақсаты қойылған. Бұл өз кезегінде шектеулі ресурстар мен
мамандандырылған Инфрақұрылым болмаған жағдайда металл бөлшектерін
қалпына келтіруді және өндіруді қамтамасыз етуге қабілетті далалық және
әскери жағдайларда пайдалануға бағытталған қосымша қондырғыны одан әрі
құру үшін жағдай жасайды.

{\bfseries Түйін сөздер:} бағдарламалау, аддитивті технологиялар,
Автоматтандыру, алгоритм, ұнтақ, G-code.

{\bfseries ПРОГРАММНО-ТЕХНИЧЕСКИЕ УСТРОЙСТВА ДЛЯ АВТОМАТИЗАЦИИ ПРОЦЕССА
ФОРМИРОВАНИЯ ТРЕХМЕРНОГО ОБЪЕКТА МЕТОДОМ SLM}

{\bfseries \tsp{1}А.Д. Тулегулов, \tsp{1}К.М.
Акишев\envelope , \tsp{1}А.М. Джумагалиева,
\tsp{2}М.А. Байжарикова,}

{\bfseries \tsp{2}С.М. Исаев}

\emph{\tsp{1}Казахский университет технологии и бизнеса
имени К. Кулажанова, Астана, Казахстан,}

\emph{\tsp{2}Таразский университет имени М.Х. Дулати, Тараз,
Казахстан,}

e-mail:akmail04cx@mail.ru

В современных условиях необходимость создания автоматизированных
программируемых автономных производственных систем становится особенно
актуальной для военных и полевых инженерных подразделений. Традиционные
установки селективного лазерного плавления (SLM), несмотря на высокую
точность и качество получаемых изделий, практически непригодны для
эксплуатации вне лабораторных условий из-за сложности конструкции,
дороговизны компонентов и высокой чувствительности к параметрам среды. В
связи с этим в статье поставлена цель исследовать программно-технические
средства для автоматизации процесса формирования трехмерного объекта
методом SLM. Это в свою очередь обеспечит условия для создания в
дальнейшем аддитивной установки, ориентированной на использование в
полевых и военных условиях, способную обеспечивать восстановление и
изготовление металлических деталей в условиях ограниченных ресурсов и
отсутствия специализированной инфраструктуры.

{\bfseries Ключевые слова:} программирование, аддитивные технологии,
автоматизация, алгоритм, порошок, G-code.

{\bfseries SOFTWARE AND TECHNICAL DEVICES FOR AUTOMATING THE PROCESS OF
FORMING A THREE-DIMENSIONAL OBJECT USING THE SLM METHOD}

{\bfseries \tsp{1}А.D. Тulegulov, \tsp{1}К.М.
Аkishev\envelope , \tsp{1}А.М. Jumagaliyeva,
\tsp{2}М.А. Baizharikova,}

{\bfseries \tsp{2}S.М. Issayev}

\emph{\tsp{1}K. Kulazhanov Kazakh University of Technology
and Business, Astana, Kazakhstan,}

\emph{\tsp{2}M.Kh. Dulati Taraz University, Taraz,
Kazakhstan,}

e-mail:akmail04cx@mail.ru

In modern conditions, the need to create automated, programmable, and
autonomous production systems is becoming particularly relevant for
military and field engineering units. Traditional selective laser
melting (SLM) systems, despite their high accuracy and quality, are
practically unsuitable for use outside of laboratory settings due to
their complex design, expensive components, and sensitivity to
environmental parameters. Therefore, this article aims to explore
software and hardware solutions for automating the process of creating
three-dimensional objects using SLM. This, in turn, will provide the
conditions for the further development of an additive manufacturing
system designed for use in field and military environments, capable of
restoring and manufacturing metal parts in the face of limited resources
and a lack of specialized infrastructure.

{\bfseries Key words.} programming, additive technologies, automation,
algorithm, powder, G-code.

{\bfseries Кіріспе.} Қазіргі жағдайда автоматтандырылған бағдарламаланатын
автономды өндірістік жүйелерді құру қажеттілігі әсіресе әскери және
далалық инженерлік бөлімшелер үшін өзекті болып отыр. Дәстүрлі
селективті лазерлік балқыту қондырғылары (SLM), алынған өнімдердің
жоғары дәлдігі мен сапасына қарамастан, дизайнның күрделілігіне,
компоненттердің қымбаттығына және қоршаған орта параметрлеріне жоғары
сезімталдыққа байланысты зертханалық жағдайлардан тыс жұмыс істеуге
жарамсыз {[}1{]}. Осыған байланысты мақалада SLM әдісімен үш өлшемді
объектіні қалыптастыру процесін автоматтандыру үшін
бағдарламалық-техникалық құралдарды зерттеу мақсаты қойылған. Бұл өз
кезегінде шектеулі ресурстар мен мамандандырылған Инфрақұрылым болмаған
жағдайда металл бөлшектерін қалпына келтіруді және өндіруді қамтамасыз
етуге қабілетті далалық және әскери жағдайларда пайдалануға бағытталған
қосымша қондырғыны одан әрі құру үшін жағдай жасайды {[}2{]}.

Осы мақсатқа жету үшін келесі міндеттерді шешу қажет:

1. SLM әдісімен үш өлшемді объектіні қалыптастыру процесін
эксперименттік зерттеу нәтижелеріне жүйелі талдау жүргізу.

2. SLM әдісімен үш өлшемді объектіні қалыптастыру процесін
автоматтандыру үшін бағдарламалық-техникалық құрылғыларды әзірлеу.

3. Инновациялық бағдарламалық-техникалық шешімдерді ескере отырып,
аддитивті қондырғының прототипін негіздеу

{\bfseries Материалдар мен әдістер.} Біз SLM әдісімен үш өлшемді
объектілерді қабаттастыру процесін эксперименттік зерттеу нәтижелеріне
талдау жасадық, олардың негізінде түйіршіктері еркін пішінді және 3D
басып шығару процесі түйіршіктердің қаптамасының тығыздығын бұзатын және
лазерлік энергияның фокус нүктесінде металл ұнтағының көлемдік шөгуі мен
біркелкі емес балқуы байқалады, бұл басып шығарылатын бөліктің сапасын
төмендетеді (сурет.1).

\fig{i/image54}{}

{\bfseries 1-сурет. Селективті лазерлік балқытуды орнату (SLM)}

Әдетте іс жүзінде бұл мәселені шешу ұнтақтардың екі түрін қолдану арқылы
ұсынылады:

1-еркін нысандағы металл ұнтағы (бұдан әрі-ПФФ);

2-пішіні идеалды сфераға жақын сферитизатормен қосымша өңдеуден өткен
металл ұнтағы - сфералық пішінді ұнтақ (бұдан әрі-ПСФ).

Зерттеуде 2 сатыда орындалатын 3D қабатты басып шығару әдісі қолданылды:

1 кезең-PPF-ті басып шығару бетіне тарату, содан кейін металл ұнтағын
іріктеп лазерлік балқыту.

2 кезең-PSF-ті басып шығару бетіне қазірдің өзінде жүргізілген таңдамалы
лазерлік балқытумен, қайталама таңдамалы лазерлік балқытумен тарату.

Бірінші кезеңде объектінің негізгі қабаты қалыптасады. Ұнтақ түрінің
еріктілігіне байланысты түйіршіктердің бос қаптамасы пайда болады, ал
лазерлік сәулемен өңдеу орындарында шөгу пайда болады {[}3{]}.

Екінші кезеңде PSF түсуі пайда болған орындар толтырылады. Ұнтақтың
пішініне байланысты идеалды PSF сферасына жақын, сондай-ақ PSF
түйіршіктерінің радиусы бірдей болғандықтан, шөгу орындарында тараған
кезде тығыз толтыру пайда болады. Лазермен өңдеу кезінде PFS іс жүзінде
кішіреймейді {[}4{]}.

Осылайша, металл ұнтағы формасының еріктілігі өтеледі. Екі кезеңді
орындау нәтижесінде объект ақаусыз басып шығарылады.

Сфералық және еркін пішінді қолдана отырып, селективті лазерлік балқыту
әдісімен үш өлшемді объектіні синтездеу әдісінің техникалық нәтижесіне
басып шығару бетіне ұнтақ таратқыштардың екі контейнерін қолдану арқылы
қол жеткізіледі.

Бұл әдіске Рег өнертабысына Патент алуға өтінім берілді. өтінім нөмірі
2025/0348.1, 16.04.2025 ж. сфералық және еркін пішінді металл ұнтақтарын
қолдана отырып, селективті лазерлік балқыту әдісімен үш өлшемді
объектіні синтездеу тәсілі.

{\bfseries Нәтижелер мен талқылаулар.} Нәтижесінде, осы ғылыми мақалада біз
металл ұнтағының пішінінің сфералық емес және гетерогенділігінің баспа
сапасына әсерін өтеуге арналған бағдарламалық-техникалық құрылғыларды
ұсынамыз {[}5{]}.

Бастапқыда, зерттеу бағытына сәйкес, аддитивті қондырғы Төтенше
жағдайлар мен соғыс жағдайларын қанағаттандыруы керек деп болжануда,
сондықтан біз стандартты аддитивті қондырғылардың функционалдығының
техникалық дизайнына талдау жасадық, нәтижесінде оларды далада қолдануға
кедергі келтіретін екі негізгі проблема анықталды:

1) басып шығару камерасына металл ұнтағын берудің дәл механикасына
тәуелділік;

2) Дайын бөлікті шығарудың және платформаны келесі циклге дайындаудың
қиындығы.

Бірінші мәселе дәстүрлі SLM қондырғыларының ұнтақты дәл жеткізуге
тәуелділігіне байланысты. Қабатты қалыптастыру цилиндрлер мен ұнтақты
тарату жүйесін қолдану арқылы жүреді, бұл ұнтақтың ондаған микронға
дейін біркелкі таралуын қамтамасыз етеді.

Екінші мәселе басып шығару циклін аяқтау процесінің шектеулеріне
байланысты. Өнімді қалыптастырғаннан кейін көп сатылы операция қажет:
бөлшекті байланыссыз ұнтақтан тазарту, оны платформадан ұқыпты бөлу
(көбінесе электр эрозиясын немесе гауһар кесуді қолдану), сондай-ақ
платформаның бетін міндетті түрде фрезерлеу. Себебі платформада қалдық
балқымалар, термиялық деформациялар пайда болады, бұл оны қосымша
механикалық өңдеусіз қайта пайдалануға жол бермейді {[}6{]}.

Ұнтақ қабатын қалыптастырудың балама тәсілдерін іздеу аясында сусымалы
ортаға тән физикалық принциптерді қолдану мүмкіндігі қарастырылды.
Өздігінен құю процесінде ұнтақтар табиғи беткейге тән бұрышы бар тұрақты
бетті құрайтыны белгілі. Мұндай бет қолданылатын материалдың пішініне,
массасына және динамикасына автоматты түрде бейімделіп, өзін-өзі
тегістеу және бейімделу тұрақтандыру қабілетін көрсетеді. Бұл
механикалық бөлімге, фрезерлеуге немесе негізді туралауға жүгінбестен
металды жағуға және балқытуға болатын ашық борпылдақ жастықтың пайдасына
жаппай қатты платформадан бас тартуға мүмкіндік береді {[}7{]}.

Осы негізде инновациялық шешімдер кешені ұсынылды: "толып кету және
өзін-өзі теңестіру"қағидаты бойынша мөлшерлеу. Ұсынылған шешімдер
қарапайымдылықты, жөндеуге жарамдылықты және далалық пайдалану үшін
маңызды шығындарды сақтай отырып, тығыздық пен қайталануды арттыруға
бағытталған.

Бұл тәсіл ұнтақтардың екі түрін қолдануды техникалық іске асыруды
жеңілдетеді: еркін пішінді металл ұнтағы (бұдан әрі-пфф) және сфералық
пішінді ұнтақ (бұдан әрі-ПФФ). Ашық бетті басып шығару PPF және PSF
бөлек сақтау сыйымдылықтарын ұнтақты тегістеу жүйесіне біріктіруді
айтарлықтай жеңілдетеді.

Жоғарыда сипатталған технологияны пысықтау үшін контейнерді әртүрлі
сортты металл ұнтақтарымен біріктіру мүмкіндігін анықтау үшін
зертханалық қондырғы жасалды, онда ашық көлденең бетке басып шығару
мүмкіндігі тексерілді (сурет.2).

\fig{i/image55}{}

{\bfseries 2-сурет. Қабатты механизмді өңдеуге арналған зертханалық
қондырғы}

{\bfseries ашық көлденең бетінде ұнтақ материал қалыптастыру}

1\emph{-металл ұнтағы контейнері; 2-ашық көлденең беті; 3-ұнтақты
тегістеуге арналған қырғыш}

Экструзия және лазерлік балқу процестерінің физикалық және логикалық
ұқсастығы бар микробағдарламаларды макростар арқылы өзгерту, G-code
командаларын қайта бөлу және экструдерді лазерлік модульге ауыстыру үшін
конфигурацияларды қайта конфигурациялау арқылы пайдалануға мүмкіндік
береді {[}8{]}. Бұл тәсіл дәлелденген архитектуралар мен жоспарлау
алгоритмдерін қайта пайдалануды қамтамасыз етеді (сурет.3)

;Layer count: 121

;LAYER:0

GO F3600 X66.354 Y67.539 Z0.300

;TYPE:SKIRT

G1 F1200 X99.570 Y97.539 E0.45381

G1 X99.743 Y97.444 E0.42765

G1 X100.546 Y67.324 E0.46545

G1 X101.657 Y77.674 E0.47685

G1 X102.573 Y97.564 E0.42789

G1 X103.745 Y97.235 E0.48765

G1 X104.456 Y97.546 E0.52345

G1 X105.756 Y97.436 E0.67845

G1 X106.564 Y97.476 E0.59876

G1 X107.856 Y97.657 E0.87659

G1 X108.763 Y97.239 E0.67657

G1 X109.743 Y97.876 E0.34267

G1 X110.943 Y97.435 E0.87986

G1 X111.563 Y97.438 E0.54876

G1 X112.793 Y97.476 E0.65876

G1 X113.123 Y97.487 E0.76889

G1 X114.983 Y97.498 E0.65987

{\bfseries 3 - сурет. G-Code қолдану фрагменті}

Marlin, Repetier және Smoothieware сияқты танымал микробағдарламаларды
талдау олардың селективті лазерлік балқыту сияқты күрделі технологиялық
процестерді басқаруға арналмағанын көрсетті. Бұл жүйелер бастапқыда
термопластикалық басып шығару үшін жасалған және көбінесе олардың
архитектурасымен шектелген: қатаң берілген логика, кеңейтудің минималды
мүмкіндіктері, сыртқы жетектерге әлсіз қолдау. Сонымен қатар, SLM үшін
басқарудың басқа деңгейі қажет: траекторияларды дәл есептеу, лазерлік
Қуат модуляциясымен жұмыс істеу, ұнтақтардың екі түрін беру және
тегістеу түйіндерімен синхрондау. Осыған байланысты Klipper ортасы
таңдамалы лазерлік балқытудың аддитивті қондырғысын жасау үшін негізгі
бағдарламалық платформа ретінде таңдалды. Klipper микробағдарламасы
негізінде басқарылатын селективті лазерлік балқыту қондырғысының
прототипін енгізу ұсынылған архитектураның өнімділігін растады {[}9{]}.

Klipper-бұл көптеген 3D басып шығарумен жұмыс істейтін әмбебап
микробағдарлама. Klipper қолданыстағы микробағдарламаның үстіне
орнатылмағандықтан, онымен жұмыс істеу үшін оны толығымен klipper
микробағдарламасымен ауыстыру қажет.

\emph{Желілік интерфейстер арқылы басқару}

Mainsail және Fluidd - бұл Klipper-мен жұмыс істеу үшін арнайы жасалған
ашық бастапқы веб-интерфейстер {[}10{]}. OctoPrint-ті қолдануға
болатынына қарамастан, егер ол сізге таныс болса, Mainsail және Fluidd
пайдаланушыға OctoPrint-те қол жетімді емес қосымша мүмкіндіктер
береді.Mainsail. Mainsail веб-интерфейсі осылай көрінеді (сурет.4).

).\fig{i/image56}{}

{\bfseries 4-сурет. Mainsail веб-интерфейсі}

Klipper-ді Linux отбасылық ОЖ басқаратын көптеген құрылғыларға орнатуға
болады. Орнату және жұмыс істеу үшін Raspberry Pi микрокомпьютерлерін
пайдалануға болады, бұл Klipper пайдаланушылары, сондай-ақ Orange Pi,
Beaglebone немесе тіпті жұмыс үстелі компьютерлері арасында ең танымал
таңдау {[}11{]}.

Шын мәнінде, аталған типтегі кез-келген құрылғы нарықта бар кез-келген
ашық сатылымдағы 3D принтермен жұмыс істеуі керек. Жабық жүйесі бар
меншікті құрылғылардан басқа.

{\bfseries Қорытынды.} Қорытындылай келе, қойылған мақсатқа қол
жеткізілгенін атап өтуге болады. SLM әдісімен үш өлшемді объектіні
қалыптастыру процесінің эксперименттік зерттеулерінің нәтижелеріне
жүйелі талдау жүргізу, осындай міндеттер шешілді. SLM әдісімен үш
өлшемді нысанды қалыптастыру процесін автоматтандыру үшін
бағдарламалық-техникалық құрылғылар да әзірленді. Бұл өте маңызды
инновациялық бағдарламалық-техникалық шешімдерді ескере отырып,
аддитивті қондырғының прототипін әзірлеу үшін негіздеме берілді.

\emph{{\bfseries Қаржыландыру.} Бұл зерттеуді ғылым және жоғары білім
комитеті қаржыландырады (№АР23490424 "Қорғаныс өнеркәсібі үшін металл
нысандарын жасау үшін аддитивті қондырғыны әзірлеу".)}

{\bfseries Әдебиеттер}

1. Тулегулов А.Д., Юрков Н.К. Аддитивные технологии для создания
металлических объектов для военной техники и вооружения//Вестник
Академии национальной гвардии Республики Казахстан. - 2024. - № 3(53). -
С.151-157.

2. K. Akishev, Zh. Nurtai, L. Akisheva. Automation of selection of
construction mix with additives of technogenic raw materials//Вестник
КазУТБ. - 2025. - №1 (26). - С.1-14. DOI 10.58805/kazutb.v.1.26-808.

3. Зорин В.А., Полухин Е.В. Аддитивные технологии. Перспективы
применения аддитивных технологий при производстве дорожно-строительных
машин // Строительная техника и технологии. - 2016. - №3 (119). - С.
54-57.

4. Шевченко Д.Ю. Аддитивные технологии в машиностроении // Комплексные
проблемы развития науки, образования и экономики региона:
Научно-практический журнал Коломенского института (филиала) МГМУ (МАМИ).
- 2015. - № 2 (7). - С.89-97.

5. Сироткин О.С. Современное состояние и перспективы развития аддитивных
технологий // Авиационная промышленность. - 2015. - № 2. - С.22-25.

6. Смуров И.Ю., Конов С.Г., Котобан Д.В. О внедрении аддитивных
технологий и производства в отечественную промышленность // Новости
материаловедения. Наука и техника. - 2015. - № 2. - С.11-22.

7. Советников Е.И. Оценки развития аддитивных технологий // Технология
легких сплавов. - 2015. - № 3. - С.17-31.

8. Дьячков В.Н., Баринов А.Ю., Никитин К.В. Применение аддитивных
технологий в производстве литых изделий // Литейное производство. -
2016. - № 5. - С.30-32.

9. Аббасов А.Э. Перспективы развития аддитивных технологий //
Информационные технологии. Радиоэлектроника. Телекоммуникации. - 2015. -
№ 5-1. - С.21-26.

10. Кузнецов П.А., Васильева О.В., Теленков А.И., Савин В.И., Бобырь
В.В. Аддитивные технологии на базе металлических порошковых материалов
для российской промышленности // Новости материаловедения. Наука и
техника. - 2015. - № 2. - С.4-10.

11. Чумаков Д.М. Перспективы использования аддитивных технологий при
создании авиационной и ракетно-космической техники // Труды МАИ. - 2014.
- № 78. - С.31-36.

{\bfseries References}

1. Tulegulov A.D., Jurkov N.K. Additivnye tehnologii dlja sozdanija
metallicheskih ob\#ektov dlja voennoj tehniki i vooruzhenija//Vestnik
Akademii nacional' noj gvardii Respubliki Kazahstan. -
2024. - № 3(53). - S.151-157. {[}in Russian{]}

2. K. Akishev, Zh. Nurtai, L. Akisheva. Automation of selection of
construction mix with additives of technogenic raw materials//Vestnik
KazUTB. - 2025. - №1 (26). - S.1-14. DOI 10.58805/kazutb.v.1.26-808.
{[}in Russian{]}

3. Zorin V.A., Poluhin E.V. Additivnye tehnologii. Perspektivy
primenenija additivnyh tehnologij pri proizvodstve
dorozhno-stroitel' nyh mashin //
Stroitel' naja tehnika i tehnologii. - 2016. - №3 (119).
- S.54-57. {[}in Russian{]}

4. Shevchenko D.Ju. Additivnye tehnologii v mashinostroenii //
Kompleksnye problemy razvitija nauki, obrazovanija i jekonomiki regiona:
Nauchno-prakticheskij zhurnal Kolomenskogo instituta (filiala) MGMU
(MAMI). - 2015. - № 2 (7). - S.89-97. {[}in Russian{]}

5. Sirotkin O.S. Sovremennoe sostojanie i perspektivy razvitija
additivnyh tehnologij // Aviacionnaja promyshlennost'. -
2015. - № 2. - S.22-25. {[}in Russian{]}

6. Smurov I.Ju., Konov S.G., Kotoban D.V. O vnedrenii additivnyh
tehnologij i proizvodstva v otechestvennuju
promyshlennost'{} // Novosti materialovedenija. Nauka i
tehnika. - 2015. - № 2. - S.11-22. {[}in Russian{]}

7. Sovetnikov E.I. Ocenki razvitija additivnyh tehnologij // Tehnologija
legkih splavov. - 2015. - № 3. - S.17-31. {[}in Russian{]}

8. D' jachkov V.N., Barinov A.Ju., Nikitin K.V.
Primenenie additivnyh tehnologij v proizvodstve lityh izdelij //
Litejnoe proizvodstvo. - 2016. - № 5. - S.30-32. {[}in Russian{]}

9. Abbasov A.Je. Perspektivy razvitija additivnyh tehnologij //
Informacionnye tehnologii. Radiojelektronika. Telekommunikacii. - 2015.
- № 5-1. - S.21-26. {[}in Russian{]}

10. Kuznecov P.A., Vasil' eva O.V., Telenkov A.I., Savin
V.I., Bobyr'{} V.V. Additivnye tehnologii na baze
metallicheskih poroshkovyh materialov dlja rossijskoj promyshlennosti //
Novosti materialovedenija. Nauka i tehnika. - 2015. - № 2. - S.4-10.
{[}in Russian{]}

11. Chumakov D.M. Perspektivy ispol' zovanija additivnyh
tehnologij pri sozdanii aviacionnoj i raketno-kosmicheskoj tehniki //
Trudy MAI. - 2014. - № 78. - S.31-36. {[}in Russian{]}

\emph{{\bfseries Авторлар туралы мәліметтер}}

Тулегулов А.Д.- физика-математика ғылымдарының кандидаты,
қауымдастырылған профессор, Қ. Құлажанов атындағы Қазақ технология және
бизнес университеті, Астана, Қазақстан,
e-mail:Ktad62@ya.ru;

Акишев К. М.-техника ғылымдарының кандидаты, қауымдастырылған профессор
Қ. Құлажанов атындағы Қазақ технология және бизнес университеті, Астана,
Қазақстан, e-mail:
akmail04@mail.ru;

Джұмағалиева А.М. -магистр, қауымдастырылған профессор, Қ. Құлажанов
атындағы Қазақ технология және бизнес университеті, Астана, Қазақстан,
e-mail: Dzhum@mail.ru;

Байжарикова М.А. - техника ғылымдарының кандидаты. М. Х. Дулати атындағы
Тараз университеті, Тараз, Қазақстан, e-mail:
marina.2288@mail.ru;

Исаев С. М. - техника ғылымдарының магистрі. М. Х. Дулати атындағы Тараз
университеті, Тараз, Қазақстан,e-mail: sagiissayev.1993kz@gmail.com.

\emph{{\bfseries Information~about~the~authors}}

Tulegulov A.D. - Candidate of Physical and Mathematical Sciences,
Associate Professor, K. Kulazhanov Kazakh University of Technology and
Business, Astana, Kazakhstan, e-mail: Ktad62@ya.ru;

Akishev K.M. - Candidate of Technical Sciences, Associate Professor,
K.Kulazhanov Kazakh University of Technology and Business, Astana,
Kazakhstan, e-mail:
akmail04@mail.ru;

Jumagaliyeva A.M.- Master' s degree, Associate Professor,
K. Kulazhanov Kazakh University of Technology and Business, Astana,
Kazakhstan, e-mail:
Dzhum@mail.ru;

Baizharikova M. A. - Candidate of Technical Sciences. M.Kh. Dulati Taraz
University, Taraz, Kazakhstan, e-mail: marina.2288@mail.ru;

Issayev S. M. - Master of Technical Sciences. M.Kh. Dulati Taraz
University, Taraz, Kazakhstan,

e-mail: sagiissayev.1993kz@gmail.com.
