\id{МРНТИ 20.51.23, 65.33.29}{}

\begin{header}
\swa{}{МАТЕМАТИЧЕСКОЕ МОДЕЛИРОВАНИЕ ПРОЦЕССА ВЫПЕЧКИ ХЛЕБОБУЛОЧНОЙ ПРОДУКЦИИ}

\tsp{1}Т.Т. Дузбаев,
\tsp{1,2}Т.Ж. Мазаков\envelope,
\tsp{1}Д.О. Байжанова,
\tsp{1}Ш.А. Джомартова,
\tsp{1}А.Т. Мазакова,
\tsp{1}Б.С.~Амирханов,
\tsp{1}Г.А. Амирханова,
\tsp{3}А.С. Тыныкулова
\end{header}

\begin{affil}
\tsp{1}Казахский национальный университет им. аль-Фараби, Алматы, Казахстан,

\tsp{2}Международный инженерно-технологический университет, Алматы, Казахстан,

\tsp{3}Международный университет, Астана, Казахстан

\corrauthor{Корреспондент-автор: talgat.duzbaev@mail.ru}
\end{affil}

Современная цивилизация переживает стремительный рост вычислительных
мощностей и повсеместное внедрение цифровых технологий. На этом фоне
роботы, интеллектуальные устройства и программные комплексы на базе ИИ
становятся ключевыми инструментами не только в быту и промышленности, но
и в научных исследованиях. Информатизация активно охватывает и пищевую
отрасль: исторически эмпирическая по подходам, она накопила обширные
экспериментальные данные, анализ которых сегодня невозможен без методов
компьютерного моделирования.

В работе рассматривается управляемость многостадийного процесса
изготовления хлебобулочной продукции. Показано, что математическое
моделирование способно заменить дорогостоящие и потенциально рискованные
эксперименты, а также дает критерий управляемости, реализованный
программно.

{\bfseries Ключевые слова:} кинетика, математическая модель, пищевая
промышленность, управление, хлебобулочная продукция.

\begin{header}
НАН-ТОҚАШ ӨНІМДЕРІН ПІСІРУ ПРОЦЕСІН МАТЕМАТИКАЛЫҚ МОДЕЛЬДЕУ

\tsp{1}Т.Т. Дузбаев,
\tsp{1,2}Т.Ж. Мазаков\envelope,
\tsp{1}Д.О. Байжанова,
\tsp{1}Ш.А. Джомартова,
\tsp{1}А.Т. Мазакова,
\tsp{1}Б.С.~Амирханов,
\tsp{1}Г.А. Амирханова,
\tsp{3}А.С. Тыныкулова
\end{header}

\begin{affil}
\tsp{1}Әл-Фараби атындағы Қазақ Ұлттық Университеті, Алматы, Қазақстан,

\tsp{2}Халықаралық инженерлік-технологиялық университет, Алматы, Қазақстан,

\tsp{3}Астана халықаралық университеті, Астана, Қазақстан,

e-mail: talgat.duzbaev@mail.ru
\end{affil}

Қазіргі өркениет есептеу қуатының қарқынды өсуі мен цифрлық
технологиялардың кеңінен енгізілу кезеңін бастан өткеріп отыр. Осы
жағдайда роботтар, интеллектуалды құрылғылар және жасанды интеллект
негізіндегі бағдарламалық кешендер тек тұрмыста және өндірісте ғана
емес, ғылыми зерттеулерде де негізгі құралдарға айналуда. Ақпараттандыру
үдерісі тамақ өнеркәсібін де белсенді түрде қамтып келеді: тарихи
тұрғыдан эмпирикалық әдістерге сүйенген бұл салада жиналған мол
тәжірибелік деректерді бүгінгі таңда компьютерлік модельдеу әдістерінсіз
талдау мүмкін емес.

Бұл жұмыста нан-тоқаш өнімдерін өндірудің көпсатылы процесін басқару
мәселесі қарастырылған. Математикалық модельдеу қымбат әрі тәуекелі
жоғары эксперименттердің орнын баса алатыны және басқарылу критерийін
бағдарламалық түрде іске асыруға мүмкіндік беретіні көрсетілген.

{\bfseries Түйін сөздер:} кинетика, математикалық модель, тамақ өнеркәсібі,
басқару, нан-тоқаш өнімдері, технологиялық процесс, оңтайландыру, сандық
модельдеу.

\begin{header}
MATHEMATICAL MODELING OF THE BAKING PROCESS IN BREAD AND BAKERY PRODUCTION

\tsp{1}T.T. Duzbaev,
\tsp{1,2}T.Zh. Mazakov\envelope,
\tsp{1}D.O. Baizhanova,
\tsp{1}Sh.A. Dzhomartova,
\tsp{1}A.T. Mazakova,
\tsp{1}B.S.~Amirkhanov,
\tsp{1}G.A. Amirkhanova,
\tsp{3}A.S. Tynykulova
\end{header}

\begin{affil}
\tsp{1}al-Farabi Kazakh National University, Almaty, Kazakhstan,

\tsp{2}Engineering Technological University, Almaty, Kazakhstan,

\tsp{3}Astana International University, Astana, Kazakhstan,

e-mail: talgat.duzbaev@mail.ru
\end{affil}

Modern civilization is experiencing a rapid increase in computational
power and the widespread adop\-tion of digital technologies. Against this
background, robots, intelligent devices, and AI-based software systems
are becoming key tools not only in everyday life and industry but also
in scientific research. Informatization is actively transforming the
food industry as well: historically empirical in its approaches, it has
accumu\-lated a large amount of experimental data, the analysis of which
today is impossible without computer modeling methods.

This paper examines the controllability of the multistage process of
bakery production. It is shown that mathematical modeling can replace
costly and potentially risky experiments and also provides a
controllability criterion implemented in software form.

{\bfseries Keywords:} kinetics, mathematical model, food industry, process
control, bakery products, techno\-logical process, optimization, digital
modeling

\begin{multicols}{2}
{\bfseries Введение.} Статья посвящена моделированию процессов
хлебопечения, описываемых нелинейными ОДУ, характерными для широкого
класса систем со сложной динамикой. За последние годы значительно
развилась теория управления химико-технологическими процессами и
реакторами: моделирование ускоряет разработку новых технологий, помогает
выявлять области устойчивых или неустойчивых режимов и подбирать
оптимальные управляющие параметры {[}1{]}.

В практических постановках применяют ОДУ для систем с сосредоточенными
параметрами (например, аппараты идеального вытеснения) и уравнения в
частных производных для систем с распределёнными параметрами (в
частности, кинетика реакций и диффузионный массоперенос) {[}2{]}.

Технологический цикл хлебобулочного производства включает стадии: 1)
расход сырья; 2) замес и брожение; 3) разделка; 4) выпечка; 5) выпуск
готовой продукции.

{\bfseries Материалы и методы.} В статье рассматривается математическая
модель многостадийного процесса изготовления хлебобулочной продукции,
которая описывается системой ОДУ и характеризуется следующими
параметрами в момент времени t {[}3-6{]}: x\tsb{1}(t) -
количество сырья; x\tsb{2}(t) -- количество замешанного теста,
находящегося в стадии брожения; x\tsb{3}(t) - количество теста
в стадии разделки; x\tsb{4}(t) - количество изделий в стадии
выпекания; x\tsb{5}(t)- количество готовой продукции в момент
времени t.

Переходы между стадиями характеризуются интенсивностями
k\tsb{i}(u), \(i = \overline{1,\ 4}\), зависящими от
температуры u(t), которая рассматривается как управляющее воздействие.

Введем следующие обозначения: t\tsb{0}=0 - момент времени
соответствует началу выпечки, T - время окончания выпечки.

Связь между введенными выше параметрами удовлетворяет следующей системе
ОДУ

\begin{equation}
\begin{array}{r}
\frac{\text{dx}_{1}}{\text{dt}} = - k_{1}(u)x_{1}, \\
\frac{\text{dx}_{2}}{\text{dt}} = k_{1}(u)x_{1} - k_{2}(u)x_{2}, \\
\frac{\text{dx}_{3}}{\text{dt}} = k_{2}(u)x_{2} - k_{3}(u)x_{3}, \\
\frac{\text{dx}_{4}}{\text{dt}} = k_{3}(u)x_{3} - k_{4}(u)x_{4}, \\
\frac{\text{dx}_{5}}{\text{dt}} = k_{4}(u)x_{4}, \\
\ 0 \leq t \leq T\text{.}
\end{array}
\end{equation}

Для исследования динамики хлебобулочного изготовления примем следующие
предположения, которые удовлетворяют естественным ограничениям: 1)
температура неотрицательна и не может превышать предельно допустимого
значения u\tsb{max} и 2) начальный запас сырья также
неотрицателен и не может превышать возможностей склада
p\tsb{max}.

Задание определённого режима протекания технологического процесса
позволяет целенаправленно влиять на его кинетику, а также на
количественные и качественные характеристики конечного продукта.
Управление параметрами процесса, в частности температурой, представляет
собой один из ключевых факторов обеспечения его эффективности и
стабильности. Температурный режим определяет скорость химических
реакций, энергетические затраты и степень преобразования исходных
веществ, что делает данный параметр важнейшей управляемой величиной в
системе регулирования технологического процесса.

В дальнейшем будем рассматривать случай, при котором температура
рассматривается как функция времени и обозначается u(t). Такой подход
позволяет учитывать динамический характер процесса и исследовать влияние
временных колебаний температуры на кинетические характеристики реакции.
Следует отметить, что для любого технологического процесса существует
ограниченный диапазон допустимых температурных значений. Абсолютная
температура не может опускаться ниже нуля, что обусловлено
фундаментальными законами термодинамики. Верхняя граница температурного
диапазона определяется технологическими и конструктивными параметрами
реактора, включая пределы термостойкости материалов, особенности
теплообмена и условия безопасной эксплуатации оборудования.

Таким образом, регулирование температуры в пределах допустимого
диапазона является важнейшей задачей оптимизации технологического
процесса. Корректный выбор температурного профиля обеспечивает
достижение максимальной производительности системы при сохранении её
устойчивости, надёжности и энергоэффективности.

Предполагается в момент начала процесса выпекания: 1) нулевая
концентрация промежуточного и конечного продуктов и 2) максимальная
концентрация исходного сырья. Сделанное предположение в математической
формулировке записывается в следующем виде

\begin{equation}
x_{1}(0) = p,\ x_{i}(0) = 0,\ i = \overline{2,5},
\end{equation}

где \(p\)- имеющийся запас сырья

\[p \in P = \left\{ p|0 \leq p \leq p_{\text{max}} \right\}\]

Сделанное ранее предположение, что температура в рабочей области
аппарата не может быть отрицательной и превосходить некоторого
предельного значения, в математической формулировке выглядит следующим
образом

\begin{equation}
u \in U = \left\{ u|0 \leq u(t) \leq u_{\text{max}},\ \forall t \in \lbrack 0,\ T\rbrack \right\}
\end{equation}

Математическая формулировка задачи (возможен ли выпуск требуемого
количества хлебобулочной продукции z за время Т, при ограничениях (2) -(
3)) имеет вид

\begin{equation}
x\tsb{5}(T) = z
\end{equation}

На математическом языке поставленная задача формулируется следующим
образом: существует ли управление u(t), удовлетворяющее условию (3), и
переводящее систему (1) за требуемое время T из начального состояния (2)
в желаемое состояние (4).

{\bfseries Обсуждение и результаты.} Численное решение задач при конкретных
исходных данных.

Для решения поставленной задачи введем следующие обозначения:

\(n = 5\); \(x = (x_{1},\text{...},x_{n})\);
\(x_{0} = (p,0,\text{..},0),\)\(c = (0,\text{..},0,1)\),

\begin{equation}
f(x,u) = \begin{pmatrix}
- k_{1}(u)x_{1} \\
\begin{matrix}
\begin{matrix}
\begin{matrix}
k_{1}(u)x_{1} - k_{2}(u)x_{2} \\
k_{2}(u)x_{2} - k_{3}(u)x_{3}
\end{matrix} \\
k_{3}(u)x_{3} - k_{4}(u)x_{4}
\end{matrix} \\
k_{4}(u)x_{4}
\end{matrix}
\end{pmatrix}
\end{equation}

Здесь \(x,\text{.}x_{0},c\) -n-мерные векторы, \(f(x,u)\) - n-мерная
вектор функция.

Тогда перепишем поставленную задачу в общем виде

\begin{equation}
\frac{\text{dx}}{\text{dt}} = f(x,u)
\end{equation}

с начальными условиями

\begin{equation}
x(0) = x_{0}
\end{equation}

и конечными условиями в момент времени Т

\begin{equation}
с^{}x(T) = z
\end{equation}

Вектор-функцию (5) перепишем в виде

\begin{equation}
f(x,u) = A(u)x,
\end{equation}

где элементы n*n-матрицы \(A(u)\) зависят от управления u:
\end{multicols}

\begin{equation}
A(u) = \begin{pmatrix}
- k_{1}(u) & 0 & 0 & 0 & 0 \\
k_{1}(u) & - k_{2}(u) & 0 & 0 & 0 \\
0 & k_{2}(u) & - k_{3}(u) & 0 & 0 \\
0 & 0 & k_{3}(u) & - k_{4}(u) & 0 \\
0 & 0 & 0 & k_{4}(u) & 0
\end{pmatrix}
\end{equation}

\begin{multicols}{2}
Теперь исходная система (6) описывается линейными дифференциальными
уравнениями

\(\frac{\text{dx}}{\text{dt}} = A(u)x\) (10)

Решение системы уравнений (10)-(7) запишем в виде интегрального
уравнения:

\begin{equation}
x(t,u,p) = x_{0} + \int_{0}^{t}{A(u) \ast x(\tau,u,p)d\tau},
\end{equation}

\(t \in \lbrack 0,T\rbrack\), \(u \in U\), \(p \in P\)

Введем обозначения
\(y(t) = x(t,u,p),f(t) = x_{0},\mu = 1,K(t,\tau) = A(u)\). Тогда
уравнение (11) можно записать следующим образом

\begin{equation}
y(t) = f(t) + \mu\int_{0}^{t}{K(t,\tau) \ast y(\tau)d\tau}
\end{equation}

Уравнение (12) носит название интегрального уравнения Вольтерра второго
рода {[}7, 8{]}.

{\bfseries Теорема 1.} Уравнение (11) имеет единственное непрерывное
решение при заданных фиксированных значениях параметров \(u \in U\),
\(p \in P\). Это решение может быть найдено методом последовательных
приближений.

Доказательство. Т.к. матрица \(A(u)\) и вектор начальных условий
\(x_{0}\) при фиксированных значениях параметров \(u,p\) являются
постоянными, то тем самым выполнены все условия теоремы 1
{[}7{]}. Отсюда следует справедливость утверждения теоремы.

Введем оператор Вольтерра

\begin{equation}
\text{By}(t) = \int_{0}^{t}{K(t,\tau) \ast y(\tau)d\tau}
\end{equation}

Определим повторные ядра оператора Вольтерра следующим образом:

\[B^{m}y(t) = \int_{0}^{t}{K_{m}(t,\tau) \ast y(\tau)d\tau}\]

\begin{equation}
K_{m}(t,s) = \int_{s}^{t}{K(t,\tau)K_{m}(\tau,s)d\tau}
\end{equation}

\[R(t,s) = \sum_{m = 1}^{\infty}{K_{m}(t,s)}\]

Пусть
\(M = \underset{0 \leq t,s \leq T}{\text{sup}}\left| K(t,s) \right|\).
Тогда для повторных ядер справедливо

\[\left| K_{m}(t,s) \right| \leq \frac{M^{m} \ast (t - 1)^{m}}{(n - 1)!}\]

Решение уравнения (12) можно записать в следующем виде

\begin{equation}
y(t) = f(t) + \int_{0}^{t}{R(t,\tau) \ast f(\tau)d\tau}
\end{equation}

Далее, в силу независимости матрицы \(A(u)\) от времени, подставляя
\(x(t,u,p)\) вместо \(y(t)\) из уравнения (15) , получим
\end{multicols}

\begin{equation}
x(t,u,p) = \left( E + A(u) \ast t + \frac{1}{2}A(u)^{2} \ast t^{2} + \text{.}\text{.}\text{.} + \frac{1}{k!}A(u)^{k} \ast t^{k} + \text{.}\text{.}\text{.} \right) \ast x_{0}
\end{equation}

Для достаточно больших k величина \(\frac{1}{k!}\left\| A(u)^{k} \ast
t^{k} \right\|\) становится малой и \(x(t,u,p)\) с требуемой точностью
можно записать в виде

\begin{equation}
x(t,u,p) = \left( E + A(u) \ast t + \frac{1}{2}A(u)^{2} \ast t^{2} + \text{.}\text{.}\text{.} + \frac{1}{k!}A(u)^{k} \ast t^{k} \right) \ast x_{0}
\end{equation}

\begin{multicols}{2}
{\bfseries Теорема 2.} Для \(\forall\varepsilon \geq 0\) \(\exists
\text{номер}r\), такой что для \(\forall k > r\) имеет место неравенство
\(\frac{1}{k!}\left\| A(u)^{k} \ast t^{k} \right\| \leq \varepsilon\).

Доказательство. Обозначим через \(a = T \ast \left\| A(u) \right\|\),
\(a_{k} = \frac{1}{k!}a^{k}\). Далее рассмотрим ряд \(1 + \sum_{k =
1}^{\infty}\frac{a_{k}}{k!}\). Вычислим \(\rho = \underset{k
\rightarrow \infty}{\text{lim}}\frac{a_{k + 1}}{a_{k}}\).  В силу
введенных обозначений \(\rho = \underset{k \rightarrow
\infty}{\text{lim}}\frac{a_{k + 1}}{a_{k}} = \underset{k \rightarrow
\infty}{\text{lim}}\frac{a}{k + 1} = 0\).  Рассмотренный ряд является
сходящимся, т.к.  \(\rho \leq 1\).  Известно, что для любого
\(\varepsilon \geq 0\) существует номер \(r\), такой что для всех \(k
> r\) справедливо \(\frac{a_{k + 1}}{a_{k}} \leq \rho + \varepsilon\)
{[}9{]}

Тогда в качестве \(r\) можно выбрать
\(r = \left\lbrack \frac{a}{\varepsilon} \right\rbrack\)
- целую часть от деления. Справедливость утверждения теоремы доказана.

{\bfseries Теорема 3.} Система (6)-(8) управляема если для заданных
конечных условий z и времени Т существуют параметры \(u \in U\),
\(p \in P\) такие, что выполняется равенство \(с^{}x(T) = z\).

Доказательство основано на подходе, опубликованном в работе {[}3{]}

{\bfseries Выводы.} В статье исследована математическая модель
многостадийного процесса изготовления хлебобулочной продукции,
описываемая ОДУ.

Для интегрального уравнения Вольтерра второго рода, к которому сводится
исходная модель, получен аналитический вид решения.

Получен критерий управляемости, который определяет взаимосвязь между
температурой, наличием количества сырья в начальный момент, временем
выпечки и выпуском требуемого количества хлебобулочной продукции.

Практическая ценность работы состоит в том, что разработанные в ней
технология и алгоритмы позволяют решить проблему управляемости объектов
различной природы и могут быть применены для исследования
электроэнергетических, робототехнических систем и т.д.

\emph{{\bfseries Финансирование.}} \emph{Статья опубликована при поддержке
Министерства науки и высшего образования Республики Казахстан, в рамках
проекта программно-целевого финансирования BR24992975 "Разработка
цифрового двойника предприятия пищевой промышленности с применением
искусственного интеллекта и технологий IIoT", 2024-2026 гг.}
\end{multicols}

\begin{center}
{\bfseries Литература}
\end{center}

\begin{refs}
1. Соловьев М.Е., Соловьев М.М. Компьютерная химия. -- М.: СОЛОН-Пресс,
2005. - 536 с. ISBN 5-98003-188-X.

2. Бакин И.А. Интенсификация процессов смешивания при получении
комбинированных продуктов в аппаратах центробежного типа. Автореф.дис.
доктор.тех.наук: 05.18.12.- Кемерово, 2009. - 34 с.

3. Мазақова Ә.Т., Шаймерден Б.О., Сейлхан Б.Ж., Мазаков Т.Ж., Джомартова
Ш.А. Химиялық реакторды басқару туралы // Материалы VII междунар.
научно-практ. конф. «Информатика и прикладная математика».2022, с.14-18.
ISBN 978-601-332-384-8.

4. Орлов Ю.Н., Соков С.А. Процессы и аппараты в химической технологии и
биотехнологии. Практикум. - Тольятти: ТГУ, 2021 - 95 с. ISBN
978-5-8259-1514-2.

5. Антипов С.Т., Кретов И.Т. и др. Машины и аппараты пищевых
производств. - М.:Высшая Школа, 2019. - 703с. ISBN 5-06-004168-9.

6. Медведков Е.Б. и др. Процессы и аппараты пищевых производств.-
Алматы: Гылым, 2016. - 360с. ISBN 978-601-7053-52-9.

7. Сетуха А.В. Метод интегральных уравнений в математической физике. -
М.: МГУ, 2023. -316 с. ISBN 978-5-19-011911-4.

8. Довгий С.А., Лифанов И.К. Методы решения интегральных уравнений.
Теория и приложения. - Киев: Наукова Думка, 2002. - 344 с. ISBN
966-96058-7-3.

9. Кадец В.М. Курс функционального анализа.-Харьков: ХНУ,2006.-
616с.ISBN 966-623-199-9.

10. OpenAI. ChatGPT (GPT-4) Language Model.-
OpenAI.-2023.-URL: \href{https://chat.openai.com/}{https://chat.openai.com}.
- accessed: 01.11.2025.
\end{refs}

\begin{center}
{\bfseries References}
\end{center}

\begin{refs}
1. Solov' ev M.E., Solov' ev M.M.
Komp' juternaja himija. - M.: SOLON-Press, 2005. - 536 s.
ISBN 5-98003-188-X. {[}in Russian{]}

2. Bakin I.A. Intensifikacija processov smeshivanija pri poluchenii
kombinirovannyh produktov v apparatah centrobezhnogo tipa. Avtoref.dis.
doktor.teh.nauk: 05.18.12.- Kemerovo, 2009. - 34 s. {[}in Russian{]}

3. Mazaқova Ә.T., Shajmerden B.O., Sejlhan B.Zh., Mazakov T.Zh.,
Dzhomartova Sh.A. Himijalyқ reaktordy basқaru turaly // Materialy VII
mezhdunar. nauchno-prakt. konf. «Informatika i prikladnaja
matematika».2022, s.14-18. ISBN 978-601-332-384-8. {[}in Russian{]}

4. Orlov Ju.N., Sokov S.A. Processy i apparaty v himicheskoj tehnologii
i biotehnologii. Praktikum. - Tol' jatti: TGU, 2021 - 95
s. ISBN 978-5-8259-1514-2. {[}in Russian{]}

5. Antipov S.T., Kretov I.T. i dr. Mashiny i apparaty pishhevyh
proizvodstv. - M.:Vysshaja Shkola, 2019. - 703s. ISBN 5-06-004168-9.
{[}in Russian{]}

6. Medvedkov E.B. i dr. Processy i apparaty pishhevyh proizvodstv.-
Almaty: Gylym, 2016. - 360s. ISBN 978-601-7053-52-9. {[}in Russian{]}

7. Setuha A.V. Metod integral' nyh uravnenij v
matematicheskoj fizike. - M.: MGU, 2023. -316 s. ISBN 978-5-19-011911-4.
{[}in Russian{]}

8. Dovgij S.A., Lifanov I.K. Metody reshenija
integral' nyh uravnenij. Teorija i prilozhenija. - Kiev:
Naukova Dumka, 2002. - 344 s. ISBN 966-96058-7-3. {[}in Russian{]}

9. Kadec V.M. Kurs funkcional' nogo
analiza.-Har' kov:HNU, 2006. - 616s. ISBN 966-623-199-9.

10. OpenAI. ChatGPT (GPT-4) Language Model.-
OpenAI.-2023.-URL: \href{https://chat.openai.com/}{https://chat.openai.com}.-
accessed: 01.11.2025.
\end{refs}

\begin{info}
\hspace{1em}\emph{{\bfseries Сведения об авторах}}

Дузбаев Т.Т.- преподаватель КазНУ имени аль-Фараби, Алматы, Казахстан,
e-mail: talgat.duzbaev@mail.ru;

Мазаков Т. Ж. - д.ф.-м.н., профессор МИТУ, Алматы, Казахстан, e-mail:
tmazakov@mail.ru;

Байжанова Д.О.- старший преподователь КазНУ имени аль-Фараби, Алматы,
Казахстан, e-mail: dina.baizhanova2024@gmail.com;

Джомартова Ш. А. -д.т.н., профессор КазНУ им. аль-Фараби, Алматы,
Казахстан, e-mail: jomartova@mail.ru;

Мазакова А.Т. - PhD, старший преподаватель КазНУ им. аль-Фараби,
Алматы, Казахстан, e-mail: aigerym97@mail.ru;

Амирханов Б. С.-научный сотрудник КазНУ им. аль-Фараби, Алматы,
Казахстан, e-mail: amirkhanov.b@gmail.com;

Амирханова Г.А. - PhD, старший преподаватель КазНУ им.аль-Фараби,
Алматы, Казахстан, e-mail: gulshat.aa@gmail.com;

Тыныкулова А. С. - старший преподаватель, Международный университет
Астана, Казахтан, e-mail: asem\_110981@mail.ru.

\hspace{1em}\emph{{\bfseries Information about the authors}}

Duzbaev~ T.T. - Lecturer, Al-Farabi Kazakh National University,
Almaty, Kazakhstan, e-mail: talgat.duzbaev@mail.ru;

Mazakov~T. Zh.- Doctor of Physical and Mathematical Sciences,
Professor, International Engineering and Technology University,
Almaty, Kazakhstan, e-mail: tmazakov@mail.ru;

Baizhanova~D.O. -Senior Lecturer, Al-Farabi Kazakh National
University, Almaty, Kazakhstan, e-mail: dina.baizhanova2024@gmail.com;

Zhomartova~Sh. A. - Doctor of Technical Sciences, Professor, Al-Farabi
Kazakh National University, Almaty, Kazakhstan, e-mail:
jomartova@mail.ru;

Mazakova~A.T. - PhD, Senior Lecturer, Al-Farabi Kazakh National
University, Almaty, Kazakhstan, е-mail: aigerym97@mail.ru;

Amirkhanov~B.S. - Researcher, Al-Farabi Kazakh National University,
Almaty, Kazakhstan, е-mail: amirkhanov.b@gmail.com;

Amirkhanova G.А.~-PhD, Senior Lecturer, Al-Farabi Kazakh National
University, Almaty, Kazakhstan, е-mail: gulshat.aa@gmail.com;

Tynykulova~A.S.- Senior Lecturer, Astana International University,
Astana, Kazakhstan, е-mail: asem\_110981@mail.ru.
\end{info}
