\newpage
\let\cleardoublepage\clearpage
\part{Информационно-коммуникационные и химические технологии}
\chapter{Информационно-коммуникационные технологии}
\ID{МРНТИ 50.49.31}{}

\begin{header}
\swa{А.А. Орманбекова, М.А. Джаманбаев, К.М. Жаскайратов, И.А. Берлисугиров, М.Б. Алиева}{АВТОМАТИЗИРОВАННЫЕ ГАЗОРАСПРЕДЕЛИТЕЛЬНЫЕ СТАНЦИИ}

А.А. Орманбекова,
М.А. Джаманбаев,
К.М. Жаскайратов\envelope,
И.А. Берлисугиров,
М.Б. Алиева
\end{header}

\begin{affil}
Алматинский Технологический Университет, Алматы, Казахстан

\corrauthor{Корреспондент-автор: karazhaskairatov@gmail.com}
\end{affil}

Актуальность выбранной тематики определяется тем, что развитие
информационных технологий дает возможность автоматизировать деятельность
газораспределительных станций, повышая эффективность их работы. Целью
исследования является изучение особенностей функционирования
автоматизированной газораспределительной станции в Казахстане. Для
достижения поставленной цели реализован комплекс задач, включая изучение
структуры автоматизированной газораспределительной станции, анализ
системы управления ею, а также рассмотрение проблем и перспектив ее
функционирования в условиях Казахстана. Объектом исследования выступает
деятельность автоматизированных газораспределительных станций в
Казахстане, а предметом --- подходы к организации их деятельности.
Методология исследования основана на применении анализа данных,
сравнительного анализа и системного метода. Результаты показали, что
наиболее эффективным подходом к организации деятельности станций
является двойная автоматизация, обеспечивающая не только автоматизацию
технологических процессов, но и управленческих функций. Установлено, что
использование различных подходов имеет разную эффективность, требующую
индивидуальной оптимизации. В качестве предложения по совершенствованию
организации деятельности станций обоснована необходимость материального
стимулирования сотрудников для повышения их компетенций в работе с
автоматизированными системами управления.

{\bfseries Ключевые слова:} автоматизированная газораспределительная
станция, системы учета природного газа, подход двойной автоматизации,
автоматические средства, устройства ввода и вывода,
аппаратно-технические средства.

\begin{header}
AUTOMATED GAS DISTRIBUTION STATIONS

А. Ormanbekova,
М. Dzhamanbaev,
К. Zhaskairatov\envelope,
I. Berlissugirov,
M. Aliyeva
\end{header}

\begin{affil}
Almaty Technological University, Almaty, Kazakhstan,

e-mail: karazhaskairatov@gmail.com
\end{affil}

The significance of this research lies in the transformative impact of
modern information technologies on automating gas distribution
facilities, leading to substantial improvements in their performance.
This work explores the operational dynamics of automated gas
distribution systems within Kazakhstan, aiming to uncover their
distinctive characteristics. To accomplish this, the investigation
focused on evaluating structural components, assessing control
mechanisms, and identifying both obstacles and growth opportu\-nities tied
to the local context. The analysis centers on the functioning of these
automated stations, with emphasis on methodologies for coordinating
their workflows. Utilizing data-driven evaluation, cross-comparative
techniques, and integrated analysis, the study revealed that combining
process and admini\-strative automation through dual systems yields
optimal results. Findings highlight significant performance variations
among different operational strategies, underscoring the need for
customized optimization measures. As part of improvement
recommendations, the research advocates for incentivizing workforce
development through monetary rewards tied to proficiency in operating
automated control systems, foster\-ing enhanced technical expertise.

{\bfseries Keywords:} automated gas distribution station, natural gas
metering systems, dual automation approach, automatic means, input and
output devices, hardware and technical means.

\begin{header}
АВТОМАТТАНДЫРЫЛҒАН ГАЗ ТАРАТУ СТАНЦИЯЛАРЫ

А.А. Орманбекова,
М.А. Джаманбаев,
К.М. Жаскайратов\envelope,
И.А. Берлисугиров,
М.Б. Алиева
\end{header}

\begin{affil}
Алматы Технологиялық Университеті, Алматы, Қазақстан,

e-mail: karazhaskairatov@gmail.com
\end{affil}

Зерттеу тақырыбының өзектілігі ақпараттық технологиялардың дамуы газ
тарату станцияларының қызметін автоматтандыруға мүмкіндік беріп, олардың
тиімділігін арттыруымен айқындалады. Зерттеудің мақсаты -- Қазақстандағы
автоматтандырылған газ тарату станциясының жұмыс істеу ерекшеліктерін
зерттеу. Осы мақсатқа жету үшін бірқатар міндеттер жүзеге асырылды,
соның ішінде автоматтандырылған газ тарату станциясының құрылымын
зерттеу, басқару жүйесін талдау, сондай-ақ Қазақстан жағдайында оның
қызмет ету мәселелері мен перспективаларын қарастыру. Зерттеу объектісі
ретінде Қазақстандағы автоматтандырылған газ тарату станцияларының
қызметі алынса, зерттеу пәні -- олардың қызметін ұйымдастыру тәсілдері.
Зерттеу әдістемесі деректерді талдау, салыстырмалы талдау және жүйелік
әдісті қолдануға негізделген. Нәтижелер көрсеткендей, станциялар
қызметін ұйымдастырудың ең тиімді тәсілі -- технологиялық процестерді
ғана емес, басқарушылық функцияларды да автоматтандыруды қамтамасыз
ететін қосарлы автоматтандыру әдісі. Әртүрлі тәсілдерді қолдану әртүрлі
тиімділікке ие екендігі анықталды, сондықтан олардың әрқайсысын жеке
оңтайландыру қажет. Станциялар қызметін ұйымдастыруды жетілдіру бойынша
ұсыныс ретінде қызметкерлердің автоматтандырылған басқару жүйелерімен
жұмыс істеу дағдыларын арттыруға бағытталған материалдық ынталандыру
қажеттілігі негізделді.

{\bfseries Түйін сөздер:} автоматтандырылған газ тарату станциясы, табиғи
газды есепке алу жүйесі, қосарлы автоматтандыру тәсілі, автоматты
құралдар, енгізу-шығару құрылғылары, аппараттық-техника\-лық құралдар.

\begin{multicols}{2}
{\bfseries Введение.} Автоматизация - одно из направлений
научно-технического прогресса, применение саморегулирующихся технических
средств, экономико-математических методов и систем управления,
освобождающих человека от участия в процессах получения, преобразования,
передачи и использования энергии, материалов или информации, существенно
снижая степень этого участия или трудоемкость выполняемых операций.
Именно поэтому возникает необходимость в дополнительном использовании
датчиков (сенсоров), устройств ввода, управления, исполнительных
механизмов, устройств вывода с использованием электронной техники и
вычислительных методов, порой копирующих нервно-психические функции
человека.

Автоматизация технологических процессов является одним из решающих
факторов повышения производительности труда и улучшения условий труда.
Все действующие и строящиеся промышленные объекты в той или иной степени
оснащены средствами автоматизации. Экономия энергоресурсов является
одной из важнейших задач современности. Эффективность системы учёта
природного газа играет важную роль в процессе транспортировки нефти от
скважины до потребителя. Именно поэтому целью автоматизации деятельности
газораспределительной станции (ГРС) является повышение эффективности её
работы, что и обуславливает актуальность выбранной темы.

Выбранная тема широко освещена в исследованиях отечественных и
зарубежных учёных, посвящённых автоматизации газораспределительных
станций и повышению их эффективности. В научных работах рассматриваются
различные подходы к управлению, анализируются технологические решения и
перспективы развития данного направления. Вышеуказанными учёными
проанализированы подходы к организации деятельности автоматизированной
газораспределительной станции. Несмотря на активное рассмотрение
вопросов по выбранной теме, учёным до сих пор не удалось найти наиболее
эффективный подход к организации деятельности автоматизированной
газораспределительной станции, что обуславливает необходимость
продолжения проведения исследований по выбранной теме.

Целью исследования является оценка эффективности внедрения двухсистемной
автоматизации на газораспределительных станциях Казахстана с учетом
климатических, технических и кадровых особенностей региона. Для
достижения указанной цели необходимо реализовать ряд исследовательских
задач:

- изучить описание структуры автоматизированной газораспределительной
станции;

- провести анализ системы управления автоматизированной
газораспределительной станцией;

- рассмотреть проблемы и перспективы функционирования автоматизированной
газораспределительной станции.

Объектом исследования являются автоматизированные газораспределительные
станции (АГРС) в Казахстане. Предметом исследования является подход к
организации автоматизированной газораспределительной станции (АГРС) в
Казахстане.

{\bfseries Материалы и методы.} Исследование основано на анализе научных
публикаций и технических отчётов по автоматизации газораспределительных
станций. Источники отбирались по критериям релевантности теме,
актуальности (не старше 2018 года) и наличия данных о структуре,
управлении и экономических аспектах работы станций. Материалы включают
описания трёхуровневых систем автоматизации, где нижний уровень
представлен датчиками давления, температуры и очистки газа, средний -
контроллерами и интерфейсами передачи данных, а верхний - программным
обеспечением управления технологическими процессами.

Методология исследования основывалась на системном анализе архитектуры
станции, сравнительной оценке различных подходов к автоматизации
(традиционной, частичной, полной) и синтезе данных для выявления
ключевых проблем и перспектив. Сравнение эффективности подходов
проводилось на основе показателей рентабельности и объёмов производства,
заимствованных из анализируемых источников. Особое внимание уделялось
адаптации технологий к условиям Казахстана, включая климатические
особенности и кадровые ресурсы.

Этапы работы включали сбор и верификацию данных, декомпозицию
структурных элементов станций, оценку технологических решений и
формулировку рекомендаций. Научная новизна исследования заключается в
обосновании возможности реализации концепции двойной автоматизации,
сочетающей автоматизацию технологических процессов и функций управления,
с повышением рентабельности до 35\%. Результаты указывают на
необходимость внедрения мер материального стимулирования обучения
персонала и оптимизации затрат на оборудование.

{\bfseries Обзор литературы.} Рассмотрим подходы к организации
автоматизированной газораспределительной станции (АГРС) в Казахстане.
Исследования показывают {[}1{]}, что системы «Power-to-Gas» используют
обширные и возобновляемые источники энергии, в основном солнечную
энергию, для питания электролизеров. Электролизер - это устройство,
используемое для электролиза воды, которое разлагает воду на кислород и
водород под действием электрического тока. Затем водород синтезируется
для получения метана или синтетического природного газа, или, другими
словами, синтетического природного газа, который затем может быть
закачан в инфраструктуру добычи природного газа. Такая схема соединяет
электрические и газовые системы, накапливая избыток возобновляемой
электроэнергии в виде водорода. Эти системы обеспечивают одновременное
производство водорода и природного газа в единой системе, основанной на
принципе параллельного производства.

Исследования показывают, что в схеме однопоточного газопровода
отключенный участок трубопровода может выступать источником как
закачиваемого, так и перекачиваемого природного газа {[}2{]}. При этом
снижается давление газа на всасывающей линии эжектора и центробежного
компрессора, что требует применения нескольких ступеней сжатия на
конечной стадии перекачки. Чаще ремонт участка магистрального
газопровода производится в одном технологическом коридоре с одним или
несколькими газопроводами. В схемах перекачки природного газа при
ремонте в технологическом коридоре с двумя и более газопроводами
источником закачиваемого природного газа является смежный газопровод,
давление в котором поддерживается на одном уровне.

Исследования показывают, что инновационный подход C-GOSP, основанный на
прямом нагреве и охлаждении путём смешивания горячего и холодного
потоков, позволяет значительно снизить эксплуатационные и капитальные
затраты {[}3{]}. Этот процесс обеспечивает одновременное охлаждение
природного газа, перекачиваемого компрессорами, без использования
доохладителей и нагрева поступающей сырой нефти без использования
теплообменников. Этот подход привёл к разработке новой компактной
конструкции газомасляной сепарационной установки (C-GOSP), отвечающей
требованиям экспортных поставок сырой нефти, с более низкими затратами
на технологическое оборудование и эксплуатацию, а также с более высоким
выходом нефтепродуктов по сравнению с традиционными методами GOSP.

В исследованиях отмечается, что ключевым элементом, определяющим
информационно-аналитические и управляющие функции газораспределительных
станций, является логически обоснованная структура выходных данных,
заложенная в основу концепции газораспределительных станций нового
поколения {[}4{]}. В структуре автоматизированной газораспределительной
станции выделяется следующая трёхуровневая структура, представленная на
рисунке 1.
\end{multicols}

\fig[0.7\textwidth]{i2/image1}[Рис.1 - Анализ организационной структуры автоматизированной газораспределительной станции]

\begin{multicols}{2}
Из рисунка 1 можно сделать вывод, что использование трехуровневой
структуры автоматизированной газораспределительной станции обеспечивает
эффективность ее работы, поскольку каждый уровень является связующим
элементом объектов предыдущего уровня {[}5{]}. Она состоит из следующих
компонентов, таких как:

- нижний уровень - уровень размещения контрольно-измерительных приборов
и

исполнительных механизмов, который включает в себя использование таких
приборов как:

- сейсмограф;

- датчики абсолютного и дифференциального давления;

- сигнализатор уровня природного газа;

- датчики качества природного газа;

- датчики температуры;

- кабельное и дополнительное оборудование;

- средний уровень - уровень сбора информации с нижнего уровня, выдачи
воздействий на устройства приема / передачи данных на верхний уровень,
который включает в себя интерфейсные линии связи;

- верхний уровень - уровень, включающий автоматизированное рабочее место
оператора, который включает в себя использование таких приборов как:

- персональный компьютер;

- источник бесперебойного питания мощностью не менее 450 Вт;

- принтер, в комплекте с кабелем USB;

- лицензионные и лицензионно-антивирусные аппаратно-технические
средства.

Важную роль в структуре автоматизированной газораспределительной станции
(АГРС) играет блок очистки природного газа (БОГ). Он предназначен для
очистки природного газа от механических примесей, капельной влаги и
отделения конденсата с последующим его отводом в ёмкость сбора
конденсата. Отделение капельной влаги и механических примесей
осуществляется путём закрутки потока природного газа и резкого изменения
направления его движения. В верхней части фильтра-сепаратора расположен
фильтрующий элемент, состоящий из сменных фильтрующих элементов.
Максимальный уровень конденсата в промежуточной ёмкости определяется
датчиком верхнего уровня, который подаёт сигнал на открытие/закрытие
дистанционно управляемого крана, осуществляющего сброс конденсата в
ёмкость сбора в автоматическом режиме, как показано на рисунке 2.
\end{multicols}

\fig[0.4\textwidth]{i2/image2}[Рис.2 - Анализ организационной структуры узла очистки природного газа в автоматизированной газораспределительной станции]

\begin{multicols}{2}
Из рисунка 2 следует, что процесс очистки природного газа проходит в 6
этапов, что обеспечивает высокую эффективность его добычи. Система
управления АГРС обеспечивает автоматизированное управление и
регулирование технологических процессов. Её ключевая функция ---
координация работы всех узлов станции. Её функционирование обеспечивает
контроль и управление всеми её технологическими компонентами, такими
как:

- узел переключающих устройств;

- узел редуцирования;

- узел очистки;

- узел подогрева природного газа;

- узел одоризации природного газа;

- узел подготовки импульсного природного газа;

- узел коммерческого учета природного газа;

- узел вспомогательных систем.

Исследования показывают, что традиционная система обнаружения утечек на
автоматизированных газораспределительных станциях (АГРС) основана на
использовании локальных звуковых извещателей, а информация об
обнаруженной утечке передается по беспроводной связи в группу быстрого
реагирования. Это обеспечивает немедленное принятие превентивных мер
даже при отсутствии людей на объекте.

Исследования подчёркивают, что все системы газораспределительных станций
должны быть оснащены датчиками тревоги, срабатывающими при определённых
климатических условиях {[}6{]}. Например, когда камеры закрыты для
отбора проб природного газа, они потенциально превращаются в миниатюрные
теплицы, что означает, что внутренняя температура камер может
значительно повыситься. В этом случае система тревоги прервёт отбор проб
природного газа и откроет крышки камер, чтобы обеспечить циркуляцию
воздуха для снижения температуры, гарантируя сохранность растительности
внутри камеры {[}7{]}. Также следует использовать датчик дождя, чтобы
камеры автоматически открывались во время дождя или дождевания.

Китайские исследователи отмечают {[}8{]}, что иерархически
автоматизированная система управления газораспределительной станцией
строится как двухуровневый программно-технический комплекс, что отражено
на рисунке 3.
\end{multicols}

\fig[0.8\textwidth]{i2/image3}[Рис.3 - Анализ организационной структуры автоматизированной системы управления автоматизированной газораспределительной станции]

\begin{multicols}{2}
Из рисунка 3 можно сделать вывод, что применение автоматизированной
системы управления автоматизированной газораспределительной станцией
(АГРС) позволит повысить эффективность работы данной ГРС. На нижнем
уровне АГРС задачи автоматического контроля и управления основным и
вспомогательным технологическим оборудованием выбранной системы решаются
непосредственно шкафом управления, либо с участием оператора этой
системы с использованием пульта управления и контроля или
автоматизированного рабочего места (АРМ), а на верхнем уровне АГРС
обеспечивается дистанционный контроль и управление ее специалистом.

Шкаф управления представляет собой сложное устройство, включающее в себя
19-дюймовые крейты с процессором и модулями ввода-вывода, а также
источник бесперебойного питания и вторичные источники питания, такие как
зарядное устройство и аккумуляторные батареи {[}9{]}. Пульт управления
имеет низкое энергопотребление и является эффективным средством
получения информации о текущих значениях технологических параметров и
управления оборудованием рабочей станции при отсутствии штатного
электропитания.

АРМ обеспечивает архивирование, документирование и отображение
информации, а также возможность мониторинга и управления оборудованием
выбранной системы. Система бесперебойного питания включает в себя
стабилизатор, преобразователь со встроенным зарядным устройством,
вторичные источники питания и аккумуляторные батареи. Время
бесперебойного питания составляет 2472 часа при нагрузке 150 Вт
{[}10{]}. Узел редуцирования является наиболее сложным и ответственным
узлом данной системы с точки зрения автоматизации. Централизованная
форма обслуживания автоматизированной газораспределительной станцией
(АГРС) требует её полной автоматизации {[}11{]}.

{\bfseries Обсуждение и результаты.} Применение различных подходов к
организации деятельности автоматизированной газораспределительной
станции имеет разную эффективность, показатели которой необходимо
увеличивать в каждом отдельном случае на разную величину, что отражено
на рисунке 4 и рисунке 5. Для оценки эффективности различных подходов к
организации деятельности автоматизированной газораспределительной
станции были использованы показатели добычи природного газа в миллионах
долларов США и в \%.
\end{multicols}

\fig[0.84\textwidth]{i2/image4}[Рис.4 - Анализ объемов производства природного газа при применении различных подходов к организации функционирования автоматизированной газораспределительной станции, млн. долл. США]

\begin{multicols}{2}
Из рисунка 4 можно сделать вывод, что наиболее распространенным подходом
к организации деятельности автоматизированной газораспределительной
станции в Казахстане {[}12{]} является подход двойной автоматизации,
который обеспечивает не только автоматизацию деятельности
автоматизированной газораспределительной станции, но и автоматизацию
управления деятельностью выбранной газораспределительной станции.
\end{multicols}

\fig[0.6\textwidth]{i2/image5}[Рис.5 - Анализ рентабельности производства природного газа при применении различных подходов к организации функционирования автоматизированной газораспределительной станции, \%]

\begin{multicols}{2}
Из рисунка 5 можно сделать вывод, что реализация подхода двойной
автоматизации является наиболее эффективной, поскольку его внедрение
может принести 35\% прибыли, что является самым высоким показателем по
сравнению с другими подходами {[}13{]}. Среди проблем организации работы
автоматизированной газораспределительной станции (АГРС) можно выделить
ограничение свободы разработчиков {[}14{]}, а также сокращение
численности персонала, задействованного в управлении на различных
уровнях. Кроме того, руководителям необходимо будет осваивать
современные технологии с использованием компьютеризированных систем без
посредников {[}15{]}. Именно поэтому перспективы организации работы АГРС
заключаются в использовании материального аспекта для убеждения
сотрудников дать принципиальное согласие на обучение новым
информационным технологиям и проведении периодических обучающих курсов
для сотрудников с целью ознакомления желающих с возможностями новых
устройств АГРС, внедряемых в результате развития научно-технического
прогресса.

{\bfseries Выводы.} Развитие научно-технического прогресса приводит к
появлению технологических инноваций, которые могут быть применены для
повышения эффективности современной газораспределительной станции. В
структуре построения автоматизированной газораспределительной станции
выделяются три уровня, которые позволяют объединить датчики, различные
типы контроллеров и серверов с операторами. На основании проведенных
исследований определено, что наиболее специфичным подходом к организации
деятельности автоматизированной газораспределительной станции в
Казахстане является подход двойной автоматизации, который обеспечивает
не только автоматизацию деятельности газораспределительной станции, но и
автоматизацию управления деятельностью выбранной автоматизированной
газораспределительной станции. Его использование позволяет обеспечить
наиболее высокие показатели добычи природного газа на сумму 420 млн
долларов США, обеспечивая рентабельность добычи природного газа 35\%.
Проблемами внедрения второго подхода автоматизации являются высокая
стоимость оборудования и недостаточная квалификация сотрудников для
управления автоматизированной газораспределительной станцией. Важнейшим
способом организации функционирования автоматизированной
газораспределительной станции было применение материальной
заинтересованности сотрудников для формирования интереса к приобретению
навыков внедрения автоматизированного подхода к управлению
автоматизированной газораспределительной станцией. Для обеспечения
эффективности работы автоматизированной газораспределительной станции
необходимо использовать совокупность подходов к организации ее работы,
что обуславливает необходимость продолжения проведения исследований по
выбранной теме.
\end{multicols}

\begin{center}
{\bfseries Литература}
\end{center}

\begin{refs}
1. Gondal I.A. Hydrogen Integration in Power-to-Gas Networks //
International Journal of Hydrogen Energy. -2019. -Vol.43 (11). -P.
1803-1815. DOI
\href{https://doi.org/10.1016/j.ijhydene.2018.11.164}{10.1016/j.ijhydene.2018.11.164}.

2. Buranshin A.R., Godovskiy D.A., Tokarev A.P. Regulation of Ejector
Systems under Unsteady Gas Pipage // IOP Conference Series: Earth and
Environmental Science. -2020. -Vol.459: 012034. DOI
\href{http://dx.doi.org/10.1088/1755-1315/459/2/022076}{10.1088/1755-1315/459/2/022076}.

3. Soliman M.A., et al. Innovative Integrated and Compact Gas Oil
Separation Plant for Upstream Surface Facilities// ResearchGate. -2020.
DOI \href{http://dx.doi.org/10.4043/30542-MS}{10.4043/30542-MS}.

4. Topchiev A.G., Design and monitoring of oil and gas industry
facilities based on the use of ultra-light aviation and digital
technologies // II International Scientific Conference "Advanced
Technologies in Aerospace, Mechanical and Automation Engineering" (MIST:
-2020. -Vol.734: 012005.DOI
\href{https://iopscience.iop.org/article/10.1088/1757-899X/734/1/012005}{10.1088/1757-899X/734/1/012005}.

5. Maslennikov S.G., Potapov S.I. Gas-Distribution Stations and Means of
Their Automation // Chemical and Petroleum Engineering.
-2004.-Vol.40.-P.138-142. DOI
\href{https://doi.org/10.1023/B:CAPE.0000033663.71198.0c}{10.1023/B:CAPE.0000033663.71198.0c}.

6. Grace P. R., van der Weerden T. J., Rowlings, D. W., Scheer, C.,
Brunk, C., Kiese, R., Butterbach-Bahl, K., Rees, R. M., Robertson, G.
P., Skiba, U. M. Global Research Alliance N₂O chamber methodology
guidelines: Considerations for automated flux measurement//Journal of
Environmental Quality. -2020. -Vol.49(5). -P.1126-1140. DOI
\href{https://doi.org/10.1002/jeq2.20124}{10.1002/jeq2.20124}.

7. Yao S., Zhang Y., Deng N., Yu X., \& Liu J. Performance research on a
power generation system using twin-screw expanders for energy recovery
at natural gas pressure reduction stations under off-design
conditions//Applied Energy. -2019.-Vol.236.-P.1218-1230. DOI
\href{https://doi.org/10.1016/j.apenergy.2018.12.039}{10.1016/j.apenergy.2018.12.039}.

8. Khisty V. H. SCADA Systems in Oil and Gas: Driving Innovation and
Efficiency in the Digital Age//International Journal for Research in
Applied Science and Engineering Technology. \emph{-}2024. -Vol.
12\emph{(8).} - P.96 - 107. DOI
\href{https://doi.org/10.22214/ijraset.2024.63848}{10.22214/ijraset.2024.63848}.

9. Dmitrievskiy, A. N., Eremin N. A., Stolyarov V. E. Digital
transformation of gas production//IOP Conference Series: Materials
Science and Engineering\emph{.} -2019.- Vol.700(1): 012052. DOI
\href{https://doi.org/10.1088/1757-899X/700/1/012052}{10.1088/1757-899X/700/1/012052}.

10. Petrenko Y., Velinov E., Vechkinzova E., Denisov, I., Gródek-Szostak
Z. Energy Efficiency of Kazakhstan Enterprises: Unexpected
Findings//Energies. -2020.- Vol.13(12):3100. DOI
\href{https://doi.org/10.3390/en13051055}{10.3390/en13051055}.

11. Zurkanain M,A., Subramaniam S.K.Investigation and Implementation of
IoT Based Oil and Gas Pipeline Monitoring System// International Journal
of Recent Technology and Applied Science\emph{.} -2023. -Vol.5 (1). -
P.25-30. DOI
\href{http://dx.doi.org/10.36079/lamintang.ijortas-0501.477}{10.36079/lamintang.ijortas-0501.477}.

12. ГОСТ Р 55218-2012 (EH 203-2-9:2005) «Оборудование газовое
нагревательное для предприятий общественного питания. Часть 2-9.
Специальные требования. Рассекатели пламени, мармиты и сковороды». URL:
\url{https://online.zakon.kz/Document/?doc_id=37148999}. -Дата обращения
19.03.2025.

13. Ranjbar D., Mukan S.M., Niyazgulova A.A. Central Asia - Center Gas
Pipeline System: Challenges and Opportunities for Modern Russia -
Central Asia Energy Relations//Vestnik RUDN. International
Relations.-2024.-Vol.24(2).-P.216-226.DOI\href{https://doi.org/10.22363/2313-0660-2024-24-2-216-226}{10.22363/2313-0660-2024-24-2-216-226}.
\end{refs}

\begin{center}
{\bfseries References}
\end{center}

\begin{refs}
1. Gondal I.A. Hydrogen Integration in Power-to-Gas Networks //
International Journal of Hydrogen Energy. -2019. -Vol.43 (11). -P.
1803-1815. DOI
\href{https://doi.org/10.1016/j.ijhydene.2018.11.164}{10.1016/j.ijhydene.2018.11.164}.

2. Buranshin A.R., Godovskiy D.A., Tokarev A.P. Regulation of Ejector
Systems under Unsteady Gas Pipage // IOP Conference Series: Earth and
Environmental Science. -2020. -Vol.459: 012034. DOI
\href{http://dx.doi.org/10.1088/1755-1315/459/2/022076}{10.1088/1755-1315/459/2/022076}.

3. Soliman M.A., et al. Innovative Integrated and Compact Gas Oil
Separation Plant for Upstream Surface Facilities// ResearchGate. -2020.
DOI \href{http://dx.doi.org/10.4043/30542-MS}{10.4043/30542-MS}.

4. Topchiev A.G., Design and monitoring of oil and gas industry
facilities based on the use of ultra-light aviation and digital
technologies // II International Scientific Conference "Advanced
Technologies in Aerospace, Mechanical and Automation Engineering" (MIST:
-2020. -Vol.734: 012005.DOI
\href{https://iopscience.iop.org/article/10.1088/1757-899X/734/1/012005}{10.1088/1757-899X/734/1/012005}.

5. Maslennikov S.G., Potapov S.I. Gas-Distribution Stations and Means of
Their Automation // Chemical and Petroleum Engineering.
-2004.-Vol.40.-P.138-142. DOI
\href{https://doi.org/10.1023/B:CAPE.0000033663.71198.0c}{10.1023/B:CAPE.0000033663.71198.0c}.

6. Grace P. R., van der Weerden T. J., Rowlings, D. W., Scheer, C.,
Brunk, C., Kiese, R., Butterbach-Bahl, K., Rees, R. M., Robertson, G.
P., Skiba, U. M. Global Research Alliance N₂O chamber methodology
guidelines: Considerations for automated flux measurement//Journal of
Environmental Quality. -2020. -Vol.49(5). -P.1126-1140. DOI
\href{https://doi.org/10.1002/jeq2.20124}{10.1002/jeq2.20124}.

7. Yao S., Zhang Y., Deng N., Yu X., \& Liu J. Performance research on a
power generation system using twin-screw expanders for energy recovery
at natural gas pressure reduction stations under off-design
conditions//Applied Energy. -2019.-Vol.236.-P.1218-1230. DOI
\href{https://doi.org/10.1016/j.apenergy.2018.12.039}{10.1016/j.apenergy.2018.12.039}.

8. Khisty V. H. SCADA Systems in Oil and Gas: Driving Innovation and
Efficiency in the Digital Age//International Journal for Research in
Applied Science and Engineering Technology. \emph{-}2024. -Vol.
12\emph{(8).} - P.96 - 107. DOI
\href{https://doi.org/10.22214/ijraset.2024.63848}{10.22214/ijraset.2024.63848}.

9. Dmitrievskiy, A. N., Eremin N. A., Stolyarov V. E. Digital
transformation of gas production//IOP Conference Series: Materials
Science and Engineering\emph{.} -2019.- Vol.700(1): 012052. DOI
\href{https://doi.org/10.1088/1757-899X/700/1/012052}{10.1088/1757-899X/700/1/012052}.

10. Petrenko Y., Velinov E., Vechkinzova E., Denisov, I., Gródek-Szostak
Z. Energy Efficiency of Kazakhstan Enterprises: Unexpected
Findings//Energies. -2020.- Vol.13(12):3100. DOI
\href{https://doi.org/10.3390/en13051055}{10.3390/en13051055}.

11. Zurkanain M,A., Subramaniam S.K.Investigation and Implementation of
IoT Based Oil and Gas Pipeline Monitoring System// International Journal
of Recent Technology and Applied Science\emph{.} -2023. -Vol.5 (1). -
P.25-30. DOI
\href{http://dx.doi.org/10.36079/lamintang.ijortas-0501.477}{10.36079/lamintang.ijortas-0501.477}.

12. GOST R 55218-2012 (EH 203-2-9:2005) «Oborudovanie gazovoe
nagrevatel' noe dlja predprijatij obshhestvennogo
pitanija. Chast'{} 2-9. Special' nye
trebovanija. Rassekateli plameni, marmity i skovorody». URL:
https://online.zakon.kz/Document/?doc\_id=37148999. -Data obrashhenija
19.03.2025. {[}in Russian{]}

13. Ranjbar D., Mukan S.M., Niyazgulova A.A. Central Asia - Center Gas
Pipeline System: Challenges and Opportunities for Modern Russia -
Central Asia Energy Relations//Vestnik RUDN. International
Relations.-2024.-Vol.24(2).-P.216-226. DOI
\href{https://doi.org/10.22363/2313-0660-2024-24-2-216-226}{10.22363/2313-0660-2024-24-2-216-226}.
\end{refs}

\begin{info}
\hspace{1em}\emph{{\bfseries Сведения об авторе}}

Орманбекова А.А. - доктор PhD, Алматинскоий технологический
университет, Алматы, Казахстан, e-mail: ain\_25@mail.ru;

Джаманбаев М.А.- кандидат физико-математических наук, доцент,
Алматинский технологический университет, Алматы, Казахстан,
dzhamanbaev@mail.ru;

Жаскайратов К.М. - магистрант, Алматинский технологический
университет, Алматы, Казахстан, e-mail: karazhaskairatov@gmail.com;

Берлисугиров И.А. - магистр технических наук, Алматинский
технологический университет, Алматы, Казахстан, e-mail:
ilexaba@mail.ru;

Алиева М.Б.- кандидат филологических наук, ассистент профессора,
Алматинский технологический университет, Алматы, Казахстан, e-mail:
marta.ali777@mail.ru.

\hspace{1em}\emph{{\bfseries Information about the authors}}

Ormanbekova A. - PhD, Almaty Technological University, Almaty,
Kazakhstan, e-mail: ain\_25@mail.ru;

Dzhamanbaev M. - Сandidate of physical and mathematical sciences,
associate professor, Almaty Technological University, Almaty,
Kazakhstan, e-mail: dzhamanbaev@mail.ru;

Zhaskairatov K. - Master' s Student at Almaty Technological
University, Almaty, Kazakhstan, e-mail:
\href{http://karazhaskairatov@gmail.com}{karazhaskairatov@gmail.com};

Berlisugirov I.A. - Master of Technical Sciences, Almaty Technological
University, Almaty, Kazakhstan, e-mail: ilexaba@mail.ru;

Aliyeva M. - Candidate of Philological Sciences, Assistant Professor,
Almaty Technological University, Almaty, Kazakhstan,
e-mail:marta.ali777@mail.ru.
\end{info}
