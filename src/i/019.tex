\id{ҒТАМР 20.15.05, 06.52.45}{}

\begin{header}
\swa{}{QADAM ЦИФРЛЫҚ ПЛАТФОРМАСЫ ШАҒЫН ЖӘНЕ ОРТА БИЗНЕСТІ ТРАНСФОРМАЦИЯЛАУ ҚҰРАЛЫ РЕТІНДЕ}

Б. Ғ. Әбілда
\end{header}

\begin{affil}
Astana International University, Астана қ., Қазақстан

\corrauthor{Корреспондент- автор: bauyrzhan.abilda@gmail.com}
\end{affil}

Бұл мақалада цифрлық платформа дамып келе жатқан цифрлық экономика
жағдайында шағын және орта бизнесті (ШОБ) трансформациялаудың негізгі
құралы ретінде қарастырылады. Зерттеудің мақсаты-ШОБ-тың цифрлық
трансформациясына платформалық тәсілді негіздеу және Қазақстанда
бірыңғай цифрлық бизнес экожүйесін құруға бағытталған QADAM цифрлық
платформасының тұжырымдамалық моделін әзірлеу. Зерттеудің әдіснамалық
негізі жүйелік, экожүйелік және платформалық тәсілдер, сондай-ақ ғылыми
басылымдарды жүйелі талдау әдістері, салыстырмалы талдау және
тұжырымдамалық модельдеу болып табылады. Мақалада ШОБ-тың цифрлық
трансформациясындағы халықаралық және қазақстандық тәжірибе
қорытындыланады, кәсіпорындардың цифрлық жетілуіндегі негізгі кедергілер
және қолданыстағы цифрлық шешімдердің шектеулері талданады. Нәтижесінде
автор цифрлық қызметтерді, аналитикалық және интеллектуалды құралдарды
және экожүйелік өзара әрекеттесу механизмдерін біріктіретін QADAM
цифрлық платформасының тұжырымдамалық моделін ұсынады. Платформалық
тәсілді енгізу ШОБ-тың цифрлық экономикаға кіруіне кедергілерді азайтып,
Қазақстандағы бизнес секторының тұрақты трансформациясын қамтамасыз ете
алады деген қорытындыға келді.

{\bfseries Түйінді сөздер:} сандық платформа, сандық трансформация, шағын
және орта бизнес, платформа экономикасы, сандық экожүйе, жасанды
интеллект.

\begin{header}
ЦИФРОВАЯ ПЛАТФОРМА QADAM КАК ИНСТРУМЕНТ ТРАНСФОРМАЦИИ МАЛОГО И СРЕДНЕГО БИЗНЕСА

Б. Ғ. Әбілда
\end{header}

\begin{affil}
Astana International University, Астана, Казахстан,

е-mail: bauyrzhan.abilda@gmail.com
\end{affil}

В статье рассматривается цифровая платформа как ключевой инструмент
трансформации малого и среднего бизнеса (МСБ) в условиях формирования
цифровой экономики. Цель исследования заключается в обосновании
платформенного подхода к цифровой трансформации МСБ и разработке
концептуальной модели цифровой платформы QADAM, ориентированной на
формирование единой цифровой экосистемы бизнеса в Казахстане.
Методологической основой исследования послужили системный, экосистемный
и платформенный подходы, а также методы систематического анализа научных
публикаций, сравнительного анализа и концептуального моделирования. В
работе обобщён зарубежный и казахстанский опыт цифровой трансформации
МСБ, проанализированы ключевые барьеры цифровой зрелости предприятий и
ограничения существующих цифровых решений. В результате предложена
авторская концептуальная модель цифровой платформы QADAM, интегрирующая
цифровые сервисы, аналитические и интеллектуальные инструменты, а также
механизмы экосистемного взаимодействия. Сделан вывод о том, что
реализация платформенного подхода способна снизить барьеры входа МСБ в
цифровую экономику и обеспечить устойчивую трансформацию
предпринимательского сектора Казахстана.

{\bfseries Ключевые слова}: цифровая платформа, цифровая трансформация,
малый и средний бизнес, платформенная экономика, цифровая экосистема,
искусственный интеллект.

\begin{header}
QADAM'S DIGITAL PLATFORM AS A TOOL FOR TRANSFORMING SMALL AND MEDIUM-SIZED BUSINESSES

B. G. Abilda
\end{header}

\begin{affil}
Astana International University, Astana, Kazakhstan,

е-mail: bauyrzhan.abilda@gmail.com
\end{affil}

This article examines the digital platform as a key tool for
transforming small and medium-sized businesses (SMEs) in the context of
the emerging digital economy. The aim of the study is to substantiate a
platform approach to the digital transformation of SMEs and develop a
conceptual model of the QADAM digital platform, aimed at creating a
unified digital business ecosystem in Kazakhstan. The methodological
basis of the study is the systemic, ecosystem, and platform approaches,
as well as methods of systematic analysis of scientific publications,
comparative analysis, and conceptual modeling. The paper summarizes
international and Kazakhstani experience in the digital transformation
of SMEs, analyzes key barriers to the digital maturity of enterprises,
and the limitations of existing digital solutions. As a result, the
author proposes a conceptual model of the QADAM digital platform that
integrates digital services, analytical and intelligent tools, and
ecosystem interaction mechanisms. It is concluded that the
implementation of a platform approach can reduce barriers to SME entry
into the digital economy and ensure the sustainable transformation of
the business sector in Kazakhstan.

{\bfseries Keywords}: Digital platform, digital transformation, small and
medium businesses, platform economy, digital ecosystem, artificial
intelligence.

\begin{multicols}{2}
{\bfseries Кіріспе.} Экономиканы жеделдетілген цифрландыру жағдайында
цифрлық платформалар шағын және орта бизнесті (ШОБ) трансформациялаудың,
бизнес-процестерді біріктірудің және нарықтарға, деректерге және цифрлық
қызметтерге қол жеткізудің негізгі құралына айналуда. Халықаралық
зерттеулер платформалық тәсіл ШОБ-қа ресурстардың шектеулерін жеңуге
және жоғары белгісіздік жағдайында өнімділік пен тұрақтылықты арттыруға
мүмкіндік беретінін растайды. Дегенмен, цифрлық платформалардың
тиімділігі көбінесе бизнестің цифрлық жетілу деңгейімен,
институционалдық ортамен және цифрлық экожүйелердің дамуымен анықталады.

Қазақстан үшін ШОБ-ты цифрлық трансформациялау мәселесі ерекше маңызды.
Мемлекеттік цифрлық даму бағдарламаларын іске асыруға және кәсіпкерлікті
қолдау үшін арнайы цифрлық қызметтердің қолжетімділігіне қарамастан,
ШОБ-тардың көпшілігі цифрлық технологияларды фрагменттелген түрде, ең
алдымен операциялық мақсаттарда пайдалануды жалғастыруда. ШОБ үшін
бірыңғай, интеграцияланған цифрлық экожүйенің болмауы бизнесті
масштабтау, аналитика және жасанды интеллект құралдарын енгізу
мүмкіндігін шектейді және экономиканы цифрландырудың жалпы әсерін
төмендетеді. Осыған байланысты, Қазақстанның институционалдық және
экономикалық жағдайларына бейімделген ШОБ трансформациясының кешенді
құралы ретінде цифрлық платформаны талдау міндеті өзекті болып табылады.

Зерттеудің мақсаты - шағын және орта бизнесті цифрлық трансформациялау
құралы ретіндегі цифрлық платформаның рөлін негіздеу және оның
Қазақстандағы ШОБ үшін бірыңғай цифрлық экожүйені қалыптастыру әлеуетін
бағалау.

Осы мақсатқа жету үшін зерттеу келесі мақсаттарды шешуге бағытталған:

- ШОБ-тың цифрлық трансформациясы мен платформа экономикасының теориялық
тәсілдерін талдау;

- ШОБ-ты қолдау үшін цифрлық платформаларды пайдаланудағы халықаралық
және қазақстандық тәжірибені талдау және салыстыру;

- Қазақстандағы ШОБ-тың цифрлық трансформациясына негізгі кедергілер мен
шектеулерді анықтау;

- ШОБ-қа бағытталған цифрлық платформаның функционалдық және
ұйымдастырушылық талаптарын анықтау;

- тұтас цифрлық бизнес экожүйесін қалыптастыру үшін платформа мен
жасанды интеллект шешімдерінің әлеуетін бағалау.

Зерттеудің ғылыми жаңалығы цифрлық платформаны тек технологиялық шешім
ретінде ғана емес, сонымен қатар шағын және орта бизнесті
трансформациялаудың экожүйелік құралы ретінде де кешенді зерттеуде
жатыр. \emph{Зерттеу:}

-дамып келе жатқан нарықтардың ерекшеліктерін ескере отырып, ШОБ
платформасын трансформациялаудың заманауи ғылыми тәсілдерін жүйелейді;

- Қазақстандағы ШОБ цифрлық жетілуінің негізгі факторлары мен
кедергілерін анықтайды;

- Қазақстанда бизнес қызметтерін, аналитиканы және жасанды интеллект
элементтерін қамтитын бірыңғай ШОБ экожүйесін қалыптастырудың негізі
ретінде цифрлық платформаның тұжырымдамалық моделін ұсынады.

Зерттеудің практикалық маңыздылығы зерттеу нәтижелері мен ұсыныстарын
шағын және орта бизнесті ішкі және халықаралық деңгейде қолдау үшін
цифрлық платформаларды әзірлеу және жетілдіруде пайдалану мүмкіндігінде
жатыр. Зерттеу нәтижелерін мыналар пайдалана алады:

- мемлекеттік органдар ШОБ цифрландыру бағдарламаларын әзірлеуде;

- даму институттары және кәсіпкерлік экожүйелер;

- ШОБ-қа бағытталған цифрлық платформалар мен қызметтерді әзірлеушілер;

- цифрлық экономика және кәсіпкерлік бойынша білім беру және зерттеу
жобалары.

Зерттеудің әдіснамалық негізі жалпы ғылыми және мамандандырылған зерттеу
әдістерінің үйлесімі болды. Зерттеуде сандық платформаларды бизнес
экожүйелерінің элементтері ретінде зерттеу үшін жүйелік және
құрылымдық-функционалдық талдау әдістері қолданылды. Қолданыстағы
тәсілдерді қорытындылау және зерттеудегі олқылықтарды анықтау үшін
Scopus, Web of Science және басқа да ғылыми дереккөздерде индекстелген
халықаралық және қазақстандық басылымдарға жүйелі әдебиет шолуы
қолданылды.

ШОБ цифрлық трансформациясының Қазақстандық контекстін талдау үшін
елдегі кәсіпорындардың цифрлық жетілуі бойынша эмпирикалық зерттеулердің
салыстырмалы талдауы және логикалық синтезі қолданылды. Сондай-ақ,
институционалдық талдау элементтері сандық платформалардың дамуына
реттеуші және инфрақұрылымдық ортаның әсерін бағалау үшін қолданылды.
Бұл әдістерді кешенді пайдалану тұтас және негізделген зерттеуді
қамтамасыз етті.

Осы зерттеу нәтижесінде келесілер күтілуде:

- Қазақстандағы шағын және орта бизнестің цифрлық трансформациясындағы
негізгі шектеулер мен факторларды анықтау;

- кәсіпкерлік экожүйелерді дамытудағы цифрлық платформалардың рөлін
терең түсінуді қалыптастыру;

- шағын және орта бизнесті трансформациялау құралы ретінде цифрлық
платформаларды құру және дамыту кезінде пайдаланылуы мүмкін
тұжырымдамалық қорытындылар мен модельді әзірлеу.

{\bfseries Әдебиетке шолу.} \emph{Цифрлық платформалар және шағын және орта
кәсіпорындардың цифрлық трансформациясы: жалпы тәсілдер.}

Цифрлық бизнес трансформациясы бойынша жүйелі әдебиетке шолуда шағын
және орта кәсіпорындар (ШОБ) үшін цифрлық платформа жай ғана
автоматтандыру құралы емес, бизнес моделін трансформациялаудың
архитектуралық негізі екенін атап көрсетеді. Авторлар ШОБ-ты сәтті
трансформациялау әртүрлі цифрлық шешімдерден процестерді, деректерді
және серіктестіктерді біріктіретін платформалық тәсілге көшу арқылы
мүмкін екенін көрсетеді. Дәстүрлі IT шешімдерімен салыстырғанда,
платформалар тұрақты масштаб экономикасын жасайды, бірақ жоғары
ұйымдастырушылық жетілуді талап етеді {[}1{]}.

ШОБ-ты цифрландыру бойынша аналитикалық есепте цифрлық платформалар
(бұлтты қызметтер, электрондық коммерция, цифрлық қаржылық қызметтер)
шағын бизнестің бәсекеге қабілеттілігін арттырудың негізгі факторы болып
табылатынын атап өтеді. Дамушы нарық елдері ШОБ-тың платформа
экожүйелеріне асимметриялық қол жеткізуіне тап болып, трансформация
тереңдігін шектейтіні атап өтілген. Дамыған экономикалармен
салыстырғанда, бұл елдердегі цифрлық платформалардың әсері
фрагменттелген {[}2{]}.

\emph{ШОБ үшін құндылық жасау механизмі ретіндегі платформа экожүйелері}

Бизнес, инновация және платформа экожүйелерінің салыстырмалы талдауын
жүргізіп, платформа шешімдері транзакциялық шығындардың төмендеуіне және
сыртқы ресурстарға қол жеткізуге байланысты ШОБ үшін ең үлкен әлеуетті
ұсынатынын көрсетеді. Авторлар платформаның бірнеше субъектілерді
үйлестіру құралына айналатынын, шағын фирмаларға шектеулі ішкі
ресурстарын өтеуге мүмкіндік беретінін атап өтеді. Сонымен қатар,
платформа ережелеріне тәуелділік ШОБ үшін жаңа тәуекелдер тудырады
{[}3{]}.

Сандық платформалар бүгінгі таңда барлық дерлік саланы өзгертіп
жатқандықтан, олар ақпараттық жүйелер (АЖ) туралы негізгі әдебиеттерге
біртіндеп еніп келеді. Сандық платформалар таралған сипатына және
институттармен, нарықтармен және технологиялармен өзара байланысына
байланысты күрделі зерттеу нысаны болып табылады. Платформалық
инновациялардың экспоненциалды түрде өсуі, платформа архитектураларының
күрделілігінің артуы және сандық платформалардың көптеген әртүрлі
салаларға таралуы нәтижесінде жаңа зерттеу мәселелері туындайды {[}4{]}.

Бизнес экожүйелері -- бұл жеке ресурстар мен мүмкіндіктерді ортақ
мақсатқа бағыттайтын автономды субъектілердің динамикалық желілік
құрылымдары. Бұл зерттеу бизнес экожүйелеріндегі негізгі құндылық жасау
әрекеттерін сипаттайтын модельді әзірлеу үшін дизайн теориясын ұсынады
{[}5{]}.

\emph{Жасанды интеллект ШОБ үшін сандық платформалардың құрамдас бөлігі
ретінде.}

Жасанды интеллект (ЖИ) шағын және орта бизнесті (ШОБ) трансформациялауда
айтарлықтай әлеуетке ие, бірақ оны енгізу көбінесе қаржылық және адами
ресурстардың шектеулілігі сияқты қиындықтарға байланысты кедергі
келтіреді. Бұл зерттеу ШОБ пайдаланатын негізгі ЖИ технологияларын,
олардың әртүрлі бизнес функцияларындағы кең ауқымды қолданылуын және
сәтті енгізу үшін қажетті стратегияларды зерттеу арқылы осы мәселені
шешеді. Зерттеу нәтижелері қызметкерлерді оқытудың, сенімді
технологиялық инфрақұрылымның, деректерге негізделген мәдениеттің және
ШОБ үшін стратегиялық серіктестіктің маңыздылығын көрсетеді {[}6{]}.

ЖИ-ді ШОБ-та енгізу шектеулі және эксперименталды екенін, әсіресе дамушы
нарықтарда екенін атап өтеді. Ірі компаниялармен салыстырғанда, ШОБ-та
деректер, қаржылық ресурстар және құзыреттер жетіспейді, бұл ЖИ-ге
негізделген сандық платформалардың нақты әсерін төмендетеді {[}7{]}.

\emph{Қазақстандағы шағын және орта бизнестің цифрлық жетілуінің
эмпирикалық зерттеулері.}

Технологиялық, ұйымдастырушылық және институционалдық өлшемдерді ескере
отырып, Қазақстандағы шағын және орта бизнестің цифрлық жетілу моделін
әзірледі. Зерттеу нәтижелері көптеген шағын және орта бизнес
субъектілерінің цифрландырудың алғашқы кезеңдерінде екенін, бұл олардың
платформалық шешімдерді пайдалану мүмкіндігін шектейтінін көрсетеді.
Дамыған цифрлық экономикасы бар елдерден айырмашылығы, Қазақстандағы
цифрлық платформалар көбінесе үзік-үзік қолданылады {[}8{]}.

Журналдағы зерттеулер {[}9{]} Қазақстандағы шағын және орта бизнестің
цифрлық трансформациясының негізгі кедергілері қаржылық ресурстардың
жетіспеушілігі, білікті кадрлардың жетіспеушілігі және цифрлық қызмет
интеграциясының төмен деңгейі екенін растайды. Авторлар қолданыстағы
цифрлық бастамалар шағын және орта бизнес үшін бірыңғай экожүйе
құрмайтынын атап өтеді {[}9{]}.

\emph{Қолданыстағы шешімдердің салыстырмалы талдауы.}

Халықаралық және ұлттық зерттеулердің талдауы дамыған экономикаларда ШОБ
үшін цифрлық платформалар электрондық коммерцияны, қаржыны, аналитиканы
және сервистік қолдауды біріктіретін интеграцияланған экожүйелер ретінде
құрылғанын көрсетеді. Қазақстанды қоса алғанда, дамушы нарықтарда ШОБ
үшін цифрлық шешімдер көбінесе жүйелік бизнес трансформациясын
көздемейтін әртүрлі қызметтер мен қолдау бағдарламаларымен ұсынылады
{[}10{]}.

Американдық электрондық коммерция нарығы 1990 жылдардың соңында Amazon
мен eBay арқасында пайда бола бастады. Дегенмен, нағыз секіріс 2010
жылдан кейін, Amazon FBA (Amazon арқылы орындау) моделіне көшкеннен
кейін болды, бұл кәсіпкерлерге Amazon қоймаларын, логистикасын және
тұтынушыларға қызмет көрсетуді аутсорсингке беруге мүмкіндік берді.
Walmart Marketplace, Etsy және Target Plus кейінірек нарыққа қосылды, ал
Shopify Shop қолданбасымен біріктірілген миллиондаған тікелей тұтынушыға
арналған дүкендерге негізделген өзінің «нарықтық платформаға ұқсас»
жүйесін жасады. Нәтижесінде, нарықтар іс жүзінде жаңа сандық «бөлшек
сауда желісіне» айналды, онымен ШОБ өз дүкендерін ашудан гөрі жұмыс
істеу тиімдірек деп тапты.

Қазақстандық модель басқаша дамыды. Мұнда қозғаушы күш бір ғана алып
емес, мыналардың үйлесімі болды:

- kaspi.kz, ол маркетплейсті, мобильді төлемдерді, логистиканы және
несиелерді біріктірді;

- Ресейден келген және жаңа нарық стандарттарын әкелген Wildberries және
Ozon;

- satu.kz, каталогтар мен B2B саудасына бағытталған шағын бизнес
платформасы;

- өзінің курьерлік және қойма инфрақұрылымының өсуі.

Маңыздысы, қазақстандық маркетплейс финтехпен синхрондалып дамыды. Бұл
бөліп төлеу жоспарларының, лезде төлемдердің және пайдаланушыға ыңғайлы
қолданба интерфейстерінің арқасында электрондық коммерцияны негізгі
бағытқа айналдырды. Шын мәнінде, Қазақстанда маркетплейс бұрын оффлайн
баламасы болмаған шағын бизнес үшін толыққанды жұмыс істейтін
платформаға айналды.

\emph{Қысқаша мазмұны және зерттеу мәселесінің тұжырымы.}

Кешенді әдебиеттерге шолу ШОБ цифрлық трансформациясы үшін табысты
халықаралық платформалық шешімдер мен теориялық тұрғыдан негізделген
модельдердің қолжетімділігіне қарамастан, Қазақстанда шағын және орта
бизнес (ШОБ) үшін үйлесімді цифрлық экожүйені дамытудағы негізгі
мәселелер шешілмегенін көрсетеді. ШОБ арасындағы цифрлық жетілудің төмен
деңгейі, қолданыстағы цифрлық қызметтердің фрагментациясы және платформа
мен жасанды интеллект шешімдерін шектеулі пайдалану секторда нақты
цифрлық трансформацияның болмауына ықпал етеді. Бұл зерттеудегі олқылық
Қазақстанда ШОБ трансформациясы құралы ретінде тиімді цифрлық
платформаны одан әрі талдау мен әзірлеудің өзектілігін көрсетеді.

{\bfseries Материалдар мен} {\bfseries тәсілдер.} \emph{Жалпы әдіснамалық
негіз.} Бұл зерттеу цифрлық бизнесті трансформациялау теориясының,
платформа экономикасының және шағын және орта бизнесті (ШОБ) дамытуға
экожүйелік тәсілдің қағидаттарын біріктіретін пәнаралық тәсілге
негізделген. Зерттеу әдіснамасы қолданыстағы цифрлық шешімдерді
салыстырмалы талдауға, сондай-ақ Қазақстандағы ШОБ-ты кешенді
трансформациялау құралы ретінде QADAM цифрлық платформасының авторлық
тұжырымдамалық моделін әзірлеуге және растауға бағытталған.

Зерттеу қолданыстағы сипатта және практикалық мәселені шешуге
бағытталған: бизнес-процестерді, аналитиканы, білім беруді және қаржылық
қызметтерді бірыңғай платформаға біріктіретін ШОБ үшін үйлесімді цифрлық
экожүйенің болмауы.

{\bfseries Әдістемелік тәсілдер.} Бұл зерттеуде жүйелік тәсіл, экожүйелік
тәсіл, платформалық тәсіл және процесстік тәсіл сияқты бірқатар қосымша
әдістемелік тәсілдер қолданылады.

Жүйелік тәсіл сандық платформаны технологиялық инфрақұрылымды, ШОБ
бизнес процестерін, экожүйе қатысушыларын және институционалдық ортаны
қамтитын күрделі әлеуметтік-экономикалық жүйе ретінде зерттеу үшін
қолданылады. Бұл тек жеке сандық қызметтерді ғана емес, сонымен қатар
олардың өзара байланысын және жинақталған трансформациялық әсерін
талдауға мүмкіндік береді.

Экожүйелік тәсіл QADAM сандық платформасын әртүрлі субъектілер: ШОБ,
мемлекеттік мекемелер, қаржы ұйымдары, білім беру және консалтингтік
провайдерлер және сандық технологиялар провайдерлері арасындағы өзара
әрекеттесуді үйлестіру орталығы ретінде тұжырымдау үшін қолданылады. Бұл
тәсіл құндылық пен желілік әсерлерді бірлесіп жасау механизмдерін
сипаттауға мүмкіндік береді.

Платформалық тәсіл QADAM архитектурасын дамытудың негізін қалайды және
модульдік, масштабталу және ашықтық қағидаттарын пайдаланады. Платформа
ШОБ-тарға бір терезе форматында сандық қызметтер мен деректер жиынтығына
қол жеткізуді қамтамасыз ететін инфрақұрылымдық негіз ретінде
қарастырылады. Процеске негізделген тәсіл ШОБ сандық трансформациясының
кезеңдерін - бастапқы сандық жетілуден бастап платформа экожүйесіне
интеграцияға дейін сипаттау үшін қолданылады. Бұл платформаның
функционалдығын кәсіпорындардың нақты бизнес-процестерімен
байланыстыруға мүмкіндік береді.

{\bfseries Зерттеу әдістері мен материалдар}.

Көрсетілген мақсаттар мен міндеттерге жету үшін зерттеуде сапалық және
аналитикалық әдістердің үйлесімі қолданылды, соның ішінде:

- ШОБ цифрлық трансформациясының негізгі үрдістерін, модельдерін және
шектеулерін анықтау үшін ғылыми дереккөздерді жүйелі талдау.

- цифрлық бизнес платформаларын құруға халықаралық және қазақстандық
тәсілдерді салыстыру үшін салыстырмалы талдау.

- QADAM цифрлық платформасының меншікті моделін әзірлеу және оның
құрылымдық элементтерін сипаттау үшін тұжырымдамалық модельдеу.

- платформа үшін қорытындылар мен жобалау принциптерін тұжырымдау үшін
аналитикалық синтез.

- платформаны одан әрі зерттеу және пилоттық енгізу үшін эмпирикалық
тестілеу және сандық бағалау әдістері ұсынылады.

{\bfseries QADAM сандық платформасының тұжырымдамалық моделі.} Бұл
зерттеуде Қазақстандағы шағын және орта бизнесті кешенді түрде
трансформациялауға бағытталған QADAM сандық платформасының
тұжырымдамалық моделі ұсынылады. Модель негізгі сандық қызметтер мен ШОБ
қолдау құралдарын бірыңғай платформа ортасына біріктіру қағидатына
негізделген.

Тұжырымдамалық тұрғыдан QADAM платформасы келесі өзара байланысты
қабаттарды қамтиды (1-сурет):

1) инфрақұрылым қабаты -- деректерді сақтауды, есептеу ресурстарын және
платформаның масштабталуын қамтамасыз ететін бұлтқа негізделген сандық
орта. Бұл қабат QADAM үшін технологиялық негізді құрайды.

2) сандық қызметтердің функционалдық қабаты - ШОБ үшін модульдік
шешімдер жиынтығы, оның ішінде:

- тұтынушылар мен сатуды басқару (CRM);

- негізгі ERP функциялары (бухгалтерлік есеп, қорларды басқару);

- электрондық коммерция және сандық тарату арналары;

- қаржылық және финтех қызметтері;

- басқару шешімдерін қолдауға арналған аналитикалық басқару тақталары.

3) Интеллектуалды деңгей (AI компоненттері)-AI-as-a-Service форматында
енгізілген жасанды интеллект құралдары:

- ақылды аналитика және сұранысты болжау;

- ШОБ үшін жеке ұсыныстар;

- автоматтандырылған қолдау чатботтары;

- бизнестің цифрлық жетілуін бағалауға арналған ақылды модульдер.

4) Экожүйе деңгейі-ШОБ сыртқы қатысушылармен: мемлекеттік қызметтермен,
білім беру платформаларымен, консалтингтік компаниялармен,
инвесторлармен және технологиялық серіктестермен өзара әрекеттесуге
арналған орта. Бұл деңгей желілік әсерлер мен платформаның тұрақтылығын
қамтамасыз етеді.
\end{multicols}

\fig[0.6\textwidth]{i3/image147}[1-сурет. ШОБ үшін цифрлық платформаның тұжырымдамалық моделі\\\normalfont{\emph{(Дереккөз: Автор әзірлеген)}}]

\begin{multicols}{2}
1-суретте авторлар әзірлеген және Қазақстандағы шағын және орта бизнесті
кешенді түрде өзгертуге бағытталған QADAM цифрлық платформасының
тұжырымдамалық моделі көрсетілген.

{\bfseries QADAM платформасын әзірлеу және енгізу кезеңдері}

QADAM цифрлық платформасын құру әдістемесі кезең-кезеңмен енгізуді
қамтиды (2-сурет):

- ШОБ цифрлық жетілуді бағалау - кәсіпорындарды цифрландырудың ағымдағы
деңгейі туралы деректерді жинау және талдау.

- платформа архитектурасын жобалау - қызметтер мен деректердің модульдік
құрылымын құру.

- негізгі цифрлық модульдерді әзірлеу - негізгі қызметтер мен жасанды
интеллект құралдарын енгізу.

- серіктес қызметтерді интеграциялау - сыртқы экожүйе қатысушыларын
тарту.

- пилоттық енгізу және масштабтау - платформаны сынақтан өткізу және оны
ШОБ-тың әртүрлі сегменттеріне бейімдеу.

\fig[0.15\textwidth]{i3/image148}[2-сурет. - QADAM моделін әзірлеу алгоритмі\\\normalfont{\emph{(Дереккөз: Автор әзірлеген)}}]

{\bfseries Технологиялар және материалдар}

Бұл зерттеу QADAM платформасын енгізу үшін заманауи цифрлық
технологияларды пайдалануды қарастырады:

- бұлтты есептеулер;

- микросервис архитектурасы;

- үлкен деректер технологиялары;

- машиналық оқыту және деректерді талдау;

- мемлекеттік және коммерциялық қызметтермен API интеграциясы;

- киберқауіпсіздік және деректерді қорғау құралдары.

Бұл технологияларды пайдалану платформаның икемділігін, масштабталуын
және ШОБ қажеттіліктеріне бейімделуін қамтамасыз етеді.

{\bfseries Зерттеудің әдіснамалық шектеулері.}

Зерттеудің шектеулері ұсынылған модельдің тұжырымдамалық сипаты болып
табылады, ол Қазақстандағы ШОБ үшін бірыңғай цифрлық платформаларды
енгізу бойынша ауқымды эмпирикалық деректердің болмауына байланысты.
Дегенмен, бұл шектеу кейінгі қолданбалы зерттеулер мен пилоттық
жобалардың негізін құрайды.

Осылайша, әзірленген әдіснама бізге QADAM цифрлық платформасын шағын
және орта бизнесті жүйелік трансформациялау құралы ретінде тексеруге
және оны практикалық енгізу үшін теориялық және әдіснамалық негіз
жасауға мүмкіндік береді.

{\bfseries Нәтижелер мен талқылау.} \emph{ШОБ цифрландыруға кедергілерді
азайтуды растау}

QADAM цифрлық платформасының ұсынылған тұжырымдамалық моделін талдау
платформалық тәсілді қолдану шағын және орта бизнестің цифрлық
экономикаға кіруіне кедергілерді айтарлықтай азайта алатынын көрсетеді.
Негізгі әсер әртүрлі цифрлық шешімдерден «бірыңғай цифрлық терезе»
моделіне көшу арқылы қол жеткізіледі (1-кесте).
\end{multicols}

\tcap{1 - кесте. ШОБ цифрландыруға кедергілерді салыстырмалы талдау}
\begin{longtblr}[
  label = none,
  entry = none,
]{
  width = \linewidth,
  colspec = {Q[250]Q[452]Q[238]},
  cells = {c},
  cells = {font = \small},
  hlines,
  vlines,
}
\textbf{Критерий}                      & \textbf{Дәстүрлі тәсіл}                                       & \textbf{Платформалық тәсіл (QADAM)} \\
\textbf{Бастапқы инвестиция}           & Жоғары (бағдарламалық жасақтама, серверлер, IT қызметкерлері) & Төмен (жазылу, SaaS)                \\
\textbf{Іске асырудың күрделілігі}     & Жоғары                                                        & Орташа / Төмен                      \\
\textbf{Аналитикаға қол жеткізу}       & Шектеулі                                                      & Кіріктірілген                       \\
\textbf{Жасанды интеллектті пайдалану} & Іс жүзінде жоқ                                                & Қызмет ретінде қолжетімді           \\
\textbf{Масштабталу мүмкіндігі}        & Шектеулі                                                      & Жоғары                              
\end{longtblr}

\emph{(Дереккөз: Автор әзірлеген)}

\begin{multicols}{2}
Талқылауға қатысты түсіндірмелер: 1-кесте алынған деректер платформа
моделінің қаржылық, технологиялық және кадрлық кедергілерді азайтатынын
көрсетеді, бұл әсіресе Қазақстандағы ШОБ үшін өте маңызды.

\emph{ШОБ цифрлық жетілуді жақсарту (модельдеу нәтижелері).}
Тұжырымдамалық модельдеу нәтижелері QADAM платформасы кәсіпорындардың
цифрлық жетілуінің халықаралық цифрлық трансформация модельдеріне сәйкес
біртіндеп өсуіне мүмкіндік беретінін көрсетеді.

QADAM көмегімен ШОБ цифрлық жетілуінің өсу кезеңдерінің схемалық
көрінісі төменде, 3-суретте келтірілген
\end{multicols}

\begin{multicols}{2}
\fig[0.18\textwidth]{i3/image149}[3-сурет. QADAM көмегімен шағын және орта бизнестің цифрлық жетілу кезеңдері\\\normalfont{\emph{(Дереккөз: Автор әзірлеген)}}]

Талқылауға қатысты түсіндірмелер: Осы құрылымға сәйкес, ШОБ шығындардың
күрт өсуінсіз операциялық цифрландырудан интеллектуалды бизнес басқаруға
көше алады.

\emph{Жасанды интеллект құралдарын енгізу нәтижелері (Тұжырымдамалық
бағалау).}

Зерттеудің негізгі тұжырымдарының бірі - жасанды интеллектті ШОБ цифрлық
платформаларына жасанды интеллект ретінде қызмет көрсету форматында
интеграциялаудың орындылығының дәлелі (2-кесте).
\end{multicols}

\tcap{2 - кесте. QADAM платформасындағы жасанды интеллект қолданудың әсері}
\begin{longtblr}[
  label = none,
  entry = none,
]{
  width = \linewidth,
  colspec = {Q[248]Q[288]Q[404]},
  cells = {c},
  cells = {font = \small},
  hlines,
  vlines,
}
\textbf{Жасанды интеллект}     & \textbf{Модулінің мақсаты} & \textbf{Расталған әсер}            \\
\textbf{Сұраныс болжамы}       & Сатуды талдау              & Белгісіздіктің азаюы               \\
\textbf{Ұсыныс жүйелері}       & Шешім қабылдауды қолдау    & Басқару тиімділігінің артуы        \\
\textbf{Чатботтар}             & Коммуникациялар            & Операциялық шығындардың азаюы      \\
\textbf{Жетілгендікті бағалау} & Бизнесті бағалау           & Жекешелендірілген цифрлық әзірлеме 
\end{longtblr}

\begin{multicols}{2}
\fig[0.45\textwidth]{i3/image150}[4-сурет. QADAM сандық платформа экожүйесі\\\normalfont{\emph{(Дереккөз: Автор әзірлеген)}}]

\emph{(Дереккөз: Автор әзірлеген)}

Талқылауға қатысты түсіндірмелер: Жасанды интеллект платформасының
форматы шағын және орта бизнес субъектілеріне өздерінің IT
инфрақұрылымын құру қажеттілігінсіз ақылды құралдарды пайдалануға
мүмкіндік береді.

\emph{Сандық платформаның экожүйелік әсері.} QADAM платформасы
кәсіпкерлік ортадағы негізгі қатысушыларды біріктіру арқылы экожүйелік
әсер жасайды. QADAM сандық платформа экожүйесінің схемалық көрінісі
4-суретте көрсетілген.


Талқылауға қатысты түсіндірмелер: Экожүйелік тәсіл транзакциялық
шығындарды азайтады, ШОБ-тың ресурстарға қол жеткізуін жеделдетеді және
желілік әсерді күшейтеді.

\emph{Қолданыстағы шешімдермен салыстыру.} Қолданыстағы цифрлық шешімдер
мен QADAM платформасын салыстыру, сондай-ақ оларды бағалау төменде
3-кестеде келтірілген.
\end{multicols}

\tcap{3-кесте. Қолданыстағы цифрлық шешімдер мен QADAM платформасын салыстыру}
\begin{longtblr}[
  label = none,
  entry = none,
]{
  cells = {c},
  cells = {font = \small},
  hlines,
  vlines,
}
\textbf{Критерийлер}                         & \textbf{Қолданыстағы шешімдер} & \textbf{QADAM} \\
\textbf{Қызмет интеграциясы}                 & Төмен                          & Жоғары         \\
\textbf{Экожүйе}                             & Фрагменттік                    & Бірыңғай       \\
\textbf{Жасанды интеллект компоненттері}     & Жеке                           & Кіріктірілген  \\
\textbf{Шағын және орта бизнеске бағытталға} & Ішінара                        & Толық          
\end{longtblr}

\emph{(Дереккөз: Автор әзірлеген)}

\begin{multicols}{2}
Талқылауға қатысты түсіндірмелер: QADAM оқшауланған емес, жүйелік сандық
трансформация әсерін қамтамасыз етеді.

Сонымен, алынған зерттеу және талдау нәтижелері QADAM сандық
платформасының әзірленген тұжырымдамалық моделінің:

- ШОБ-ты цифрландыруға кедергілерді азайтатынын;

- сандық жетілуге ықпал ететінін;

- жасанды интеллект құралдарына қолжетімділікті қамтамасыз ететінін;

- Қазақстанда тұрақты сандық бизнес экожүйесін құратынын растайды.

Қорытындылай келе, QADAM сандық платформасының ұсынылған тұжырымдамалық
моделі Қазақстандағы шағын және орта бизнесті трансформациялау үшін
айтарлықтай әлеуетке ие деп айтуға болады.

Жасанды интеллектті пайдалану арқылы жетілдірілген платформа және
экожүйелік тәсіл шағын және орта бизнестің фрагменттелген
сандықтандырылуынан жүйелік сандық трансформациясына көшуге мүмкіндік
береді, бұл кәсіпкерліктің және елдің сандық экономикасының тұрақты
дамуының негізін құрайды.

{\bfseries Қорытынды.} Зерттеу бүгінгі ортадағы цифрлық платформалардың тек
автоматтандыруды қолдау құралы ғана емес, сонымен қатар шағын және орта
бизнестің (ШОБ) трансформациясында жүйе құрушы фактор болып табылатынын,
олардың тұрақтылығын, бәсекеге қабілеттілігін және цифрлық экономикада
өркендеу қабілетін анықтайтынын растады. Халықаралық және қазақстандық
ғылыми басылымдарды мұқият талдау платформалық тәсіл бизнес-процестерді,
деректерді және қызметтерді интеграциялауды жеңілдететінін, ШОБ
ресурстардың шектеулерін өтей алатын және аналитикалық және
интеллектуалды құралдарды қоса алғанда, заманауи цифрлық технологияларға
қол жеткізе алатын экожүйені құратынын көрсетті.

Әдебиетке шолу нәтижелері дамыған экономикаларда ШОБ үшін цифрлық
платформалар электрондық коммерция, қаржы, оқыту, аналитика және
басқарушылық шешімдерді қолдау құралдарын біріктіретін кешенді
экожүйелерге айналғанын көрсетеді. Сонымен қатар, Қазақстан контекстін
талдау жарияланған цифрландыру мақсаттары мен ШОБ-тың цифрлық жетілуінің
нақты деңгейі арасындағы айтарлықтай алшақтықты анықтады. Фрагменттелген
цифрлық шешімдердің таралуы, қызмет көрсету интеграциясының нашарлығы,
аналитика және жасанды интеллект құралдарына шектеулі қолжетімділік және
дағдылардағы алшақтық сектордың шынайы цифрлық трансформациясына кедергі
келтіретін құрылымдық кедергілер тудырады.

Зерттеу Қазақстандағы шағын және орта бизнесті цифрлық дамытудың негізгі
қиындығы жеке цифрлық қызметтердің болмауында емес, дамудың барлық
кезеңдерінде кешенді бизнес қолдауын қамтамасыз ете алатын біртұтас,
тұтас цифрлық экожүйенің болмауында екенін көрсетеді. Бұл
институционалдық және технологиялық алшақтық арнайы IT бастамаларынан
келесі буын платформалық шешімдерге көшуді қажет етеді.

Зерттеуде әзірленген QADAM цифрлық платформасының тұжырымдамалық моделі
автордың шағын және орта бизнесті цифрлық трансформациялау теориясы мен
тәжірибесін дамытуға қосқан үлесін білдіреді. Негізінен жеке
функцияларға немесе қызметтерге бағытталған қолданыстағы шешімдерден
айырмашылығы, ұсынылған модель инфрақұрылымдық, функционалдық,
интеллектуалдық және экожүйелік деңгейлерді біріктіруге негізделген. Бұл
платформаны тек технологиялық өнім ретінде ғана емес, сонымен қатар
бизнес, үкімет және серіктес ұйымдар арасындағы бірыңғай цифрлық
кеңістіктегі өзара әрекеттесуді үйлестіру механизмі ретінде де
қарастыруға мүмкіндік береді.

Зерттеуде жасанды интеллекттің цифрлық платформалардың трансформациялық
әсерінің катализаторы ретіндегі рөліне ерекше назар аударылады. Жасанды
интеллектті платформалық қызметтер ретінде енгізу ШОБ субъектілерінің
аналитика мен интеллектуалды құралдарды пайдаланудағы кедергілерін
азайтатыны көрсетілген, бұл әсіресе ресурстары шектеулі және цифрлық
жетілудің төмен деңгейі бар компаниялар үшін маңызды. Осылайша, жасанды
интеллект компоненттері бар цифрлық платформа ШОБ субъектілерінің
операциялық цифрландырудан деректерге негізделген және білімге
негізделген басқару модельдеріне көшуінің катализаторы бола алады.

Алынған нәтижелердің практикалық маңыздылығы оларды ШОБ қолдау үшін
цифрлық платформаларды әзірлеу және енгізуде, сондай-ақ елдегі
кәсіпкерлікті цифрлық дамыту бойынша мемлекеттік және институционалдық
бағдарламаларды әзірлеуде пайдалану мүмкіндігінде жатыр. Ұсынылған
әдіснамалық тәсілдер мен тұжырымдамалық модель пилоттық жобалар,
эмпирикалық валидация және кейіннен Қазақстандағы платформалық
шешімдерді масштабтау үшін негіз бола алады.

Жалпы алғанда, зерттеу нәтижелері бірыңғай цифрлық платформаны әзірлеу
Қазақстандағы шағын және орта бизнесті нақты трансформациялаудың
алғышарты екенін растайды. Экожүйелік өзара әрекеттесуге және
интеллектуалды технологияларды пайдалануға бағытталған платформалық
тәсілді енгізу ШОБ субъектілерінің цифрлық жетілу деңгейін арттырып қана
қоймай, сонымен қатар ұлттық экономикалық даму үшін ұзақ мерзімді пайда
әкелуі мүмкін. Әрі қарайғы зерттеулер ұсынылған модельдің тиімділігін
эмпирикалық бағалауға және іс жүзінде енгізу кезінде оның
әлеуметтік-экономикалық әсерін талдауға бағытталуы керек.
\end{multicols}

\begin{center}
{\bfseries Әдебиеттер}
\end{center}

\begin{refs}
1. Sarmiento, A. G. M. (2024). A Systematic Literature Review of Digital
Transformation. International Journal of Multidisciplinary: Applied
Business and Education Research.// APPLED BISINESS AND EDUCATION
RESEARH.-2024.-Vol.5(12).-P.4974-4991.
\href{https://doi.org/10.11594/ijmaber.05.12.07}{DOI
10.11594/ijmaber.05.12.07}.

2. OECD (2021), The Digital Transformation of SMEs.// OECD.-2021.
--P.1-275. \href{https://doi.org/10.1787/bdb9256a-en}{DOI
10.1787/bdb9256a-en}.

3. Hein, A., Schreieck, M., Riasanow, T. et al. Digital platform
ecosystems.//Electron
Markets.-2020.Vol.30.-P.87-98.\href{https://doi.org/10.1007/s12525-019-00377-4}{DOI
10.1007/s12525-019-00377-4}.

4. Mark de Reuver, Carsten Sørensen, Ranul C. Basole. The Digital
Platform: A Research Agenda.// Journal of Information
Technology.-2018.Vol.33.-P.124-135.
\href{https://doi.org/10.1057/s41265-016-0033-3}{DOI
10.1057/s41265-016-0033-3}.

5. Christian Betz. Reinhard Jung. Christian and Jung, Reinhard. Value
Creation in Business Ecosystems-A Design Theory for a Reference Model.
(2021).//AIS Electronic Library. -2021.-P.1-11. URL:
\url{https://aisel.aisnet.org/amcis2021/virtual_communities/virtual_communities/4}.

6. Le Dinh, T., Vu, M.-C., Tran, G. T. C. Artificial Intelligence in
SMEs: Enhancing Business Functions Through Technologies and
Applications.//Information.-2025.-Vol.16(5).415.-P.1-22.
\href{https://doi.org/10.3390/info16050415}{DOI.10.3390/info16050415}.

7. Kramarenko A. Artificial intelligence for small and medium business:
perspectives and challenges // Journal of engineering management and
competitiveness.-2025.-Vol.15 (1). -P.43-56.
\href{https://doi.org/10.5937/JEMC2501043K}{DOI 10.5937/JEMC2501043K}.

8. A. Yezhebay, V. Sengirova, D. Igali, Y. O. Abdallah and E. Shehab.
Digital Maturity and Readiness Model for Kazakhstan SMEs.//IEEE
International Conference on Smart Information Systems and Technologies
(SIST).-2021. \href{https://doi.org/10.1109/SIST50301.2021.9465890}{DOI
10.1109/SIST50301.2021.9465890}.

9. Mohammad Fawzi Shubita (2023). The effect of human capital and
structural capital on leverage: Evidence from Jordan.//Problems and
Perspectives in Management.-2023.-Vol.21(3).-P.1-10. DOI
\href{https://doi.org/10.21511/ppm.21(3).2023.01}{10.21511/ppm.21(3).2023.01}.

10. Kazakhstan Economic Update, Winter 2024-2025: Funding the Future -
Boosting Revenues for Lasting Investments.//World Bank.-2025.-P.1-41.
\href{https://doi.org/10.1596/42791}{DOI 10.1596/42791}
\end{refs}

\begin{center}
{\bfseries References}
\end{center}

\begin{refs}
1. Sarmiento, A. G. M. (2024). A Systematic Literature Review of Digital
Transformation. International Journal of Multidisciplinary: Applied
Business and Education Research.// APPLED BISINESS AND EDUCATION
RESEARH.-2024.-Vol.5(12).-P.4974-4991.
\href{https://doi.org/10.11594/ijmaber.05.12.07}{DOI
10.11594/ijmaber.05.12.07}.

2. OECD (2021), The Digital Transformation of SMEs.// OECD.-2021.
--P.1-275. \href{https://doi.org/10.1787/bdb9256a-en}{DOI
10.1787/bdb9256a-en}.

3. Hein, A., Schreieck, M., Riasanow, T. et al. Digital platform
ecosystems.//Electron
Markets.-2020.Vol.30.-P.87-98.\href{https://doi.org/10.1007/s12525-019-00377-4}{DOI
10.1007/s12525-019-00377-4}.

4. Mark de Reuver, Carsten Sørensen, Ranul C. Basole. The Digital
Platform: A Research Agenda.// Journal of Information
Technology.-2018.Vol.33.-P.124-135.
\href{https://doi.org/10.1057/s41265-016-0033-3}{DOI
10.1057/s41265-016-0033-3}.

5. Christian Betz. Reinhard Jung. Christian and Jung, Reinhard. Value
Creation in Business Ecosystems-A Design Theory for a Reference Model.
(2021).//AIS Electronic Library. -2021.-P.1-11. URL:
\url{https://aisel.aisnet.org/amcis2021/virtual_communities/virtual_communities/4}.

6. Le Dinh, T., Vu, M.-C., Tran, G. T. C. Artificial Intelligence in
SMEs: Enhancing Business Functions Through Technologies and
Applications.//Information.-2025.-Vol.16(5).415.-P.1-22.
\href{https://doi.org/10.3390/info16050415}{DOI.10.3390/info16050415}.

7. Kramarenko A. Artificial intelligence for small and medium business:
perspectives and challenges // Journal of engineering management and
competitiveness.-2025.-Vol.15 (1). -P.43-56.
\href{https://doi.org/10.5937/JEMC2501043K}{DOI 10.5937/JEMC2501043K}.

8. A. Yezhebay, V. Sengirova, D. Igali, Y. O. Abdallah and E. Shehab.
Digital Maturity and Readiness Model for Kazakhstan SMEs.//IEEE
International Conference on Smart Information Systems and Technologies
(SIST).-2021. \href{https://doi.org/10.1109/SIST50301.2021.9465890}{DOI
10.1109/SIST50301.2021.9465890}.

9. Mohammad Fawzi Shubita (2023). The effect of human capital and
structural capital on leverage: Evidence from Jordan.//Problems and
Perspectives in Management.-2023.-Vol.21(3).-P.1-10. DOI
\href{https://doi.org/10.21511/ppm.21(3).2023.01}{10.21511/ppm.21(3).2023.01}.

10. Kazakhstan Economic Update, Winter 2024-2025: Funding the Future -
Boosting Revenues for Lasting Investments.//World Bank.-2025.-P.1-41.
\href{https://doi.org/10.1596/42791}{DOI 10.1596/42791}.
\end{refs}

\begin{info}
\hspace{1em}\emph{{\bfseries Авторлар туралы мәліметтер}}

Әбілда Б. - магистрант, Astana International University, Астана,
Қазақстан, e-mail:bauyrzhan.abilda@gmail.com

\hspace{1em}\emph{{\bfseries Information about the authors}}

Abilda B. - master student, Astana International University, Astana,
Kazakhstan, mail:bauyrzhan.abilda@gmail.com.
\end{info}
