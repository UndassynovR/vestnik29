\id{IRSTI 06.52.41}{}

\begin{header}
\swa{}{SPECIFICS OF THE TRANSITION TO A CIRCULAR ECONOMY IN KAZAKHSTAN}

\tsp{1}Y.N. Linok,
\tsp{1}Y.Y. Mubarakov\envelope,
\tsp{2}B. Zirkler,
\tsp{1}Y.Y. Sumareva,
\tsp{1}А.S. Bukhryakova
\end{header}

\begin{affil}
\tsp{1}Kazakh-American Free University, Oskemen, Kazakhstan,

\tsp{2}Westsächsische Hochschule Zwickau - University of Applied Sciences, Zwickau, Germany

\corrauthor{Corresponding-authors: mubarakovyeldar@gmail.com}
\end{affil}

This article examines the pressing issues of transition to a circular
economy in Kazakhstan. The object\-ive of the study is to identify the
specifics of this transition and determine measures to establish the
key elements of a circular economy in the country. Article highlights
the specific conditions of Kazakhstan, including the level of
socio-economic development and the country' s capabilities, the
availability, extrac\-tion, and processing of minerals, the structure
of industrial production, population density, territorial factors, and
the socio-cultural characteristics of the population. It is
demonstrated that these conditions are hindering the transition to a
circular economy in Kazakhstan. The main reason for this is the
worsening environmental problems in the country and the influence of
global experience in transitioning to a circular economy. The article
analyzes the main content and evolution of the relevant legislation
and regulations. It is noted that it is integrated into the
development of a green economy in the country and does not contain a
separate law or the necessary policy documents on a circular
economy. The article presents the key characteristics of this economy
in Kazakhstan, which are compared with positive examples from global
experience. Waste management issues in the country are also
considered. The article identifies the key steps and process features
of Kazakhstan' s transition to a circular economy. It concludes that
the country is in the initial stages of developing a circular economy,
with only a few elements already in place. Overall, the pace, scale,
and results of the transition to a circular economy in Kazakhstan have
been comparatively slow. Therefore, proposals are made for improving
the effectiveness of this transition, considering the country' s
specific characteristics and the potential for implementing global
practices.

{\bfseries Keywords:} circular economy, Kazakhstan, transition features,
conditions and main steps, waste mana\-gement, implementation.

\begin{header}
ҚАЗАҚСТАНДАҒЫ АЙНАЛМАЛЫ ЭКОНОМИКАҒА КӨШУДІҢ ЕРЕКШЕЛІКТЕРІ

\tsp{1}Я.Н. Линок,
\tsp{1}Е.Е. Мубараков\envelope,
\tsp{2}Б. Цирклер,
\tsp{1}Е.Е. Сумарева,
\tsp{1}А.С. Бухрякова
\end{header}

\begin{affil}
\tsp{1}Қазақстан-Американдық еркін университеті, Өскемен, Қазақстан,

\tsp{2}Westsächsische Hochschule Zwickau - Қолданбалы ғылымдар университеті, Цвиккау, Германия,

e-mail: mubarakovyeldar@gmail.com
\end{affil}

Бұл мақалада Қазақстандағы айналмалы экономикаға көшудің өзекті
мәселелері қарастырылады. Зерттеудің мақсаты - осы өтпелі кезеңнің
ерекшеліктерін анықтау және елдегі айналмалы экономиканың негізгі
элементтерін белгілеу шараларын анықтау. Мақалада Қазақстанның нақты
жағдайлары, соның ішінде әлеуметтік-экономикалық даму деңгейі мен елдің
мүмкіндіктері, пайдалы қазбалардың қолжетімділігі, өндірілуі және
өңделуі, өнеркәсіптік өндіріс құрылымы, халық тығыздығы, аумақтық
факторлар және халықтың әлеуметтік-мәдени ерекшеліктері көрсетілген. Бұл
жағдайлар Қазақстандағы айналмалы экономикаға көшуге кедергі келтіретіні
көрсетілген. Мұның басты себебі - елдегі экологиялық мәселелердің ушығуы
және айналмалы экономикаға көшудегі әлемдік тәжірибенің әсері. Мақалада
тиісті заңнама мен ережелердің негізгі мазмұны мен эволюциясы талданады.
Оның елдегі жасыл экономиканы дамытуға интеграцияланғаны және айналмалы
экономика бойынша жеке заң немесе қажетті саяси құжаттар жоқ екені атап
өтілген. Мақалада Қазақстандағы осы экономиканың негізгі сипаттамалары
ұсынылған, олар әлемдік тәжірибеден алынған оң мысалдармен
салыстырылады. Елдегі қалдықтарды басқару мәселелері де қарастырылады.
Мақалада Қазақстанның айналмалы экономикаға көшуінің негізгі қадамдары
мен процесінің ерекшеліктері анықталған. Онда елдің айналмалы
экономиканы дамытудың бастапқы кезеңінде екендігі, оның бірнеше элементі
ғана бар екендігі туралы қорытынды жасалған. Жалпы алғанда, Қазақстанда
айналмалы экономикаға көшудің қарқыны, ауқымы және нәтижелері
салыстырмалы түрде баяу болды. Сондықтан, елдің ерекшеліктері мен
әлемдік тәжірибелерді енгізу әлеуетін ескере отырып, осы көшудің
тиімділігін арттыру бойынша ұсыныстар жасалуда.

{\bfseries Түйін сөздер:} айналмалы экономика, Қазақстан, өтпелі кезеңнің
сипаттамалары, жағдайлары және негізгі қадамдары, қалдықтарды басқару,
енгізу.

\begin{header}
ОСОБЕННОСТИ ПЕРЕХОДА К ЦИРКУЛЯРНОЙ ЭКОНОМИКЕ В КАЗАХСТАНЕ

\tsp{1}Я.Н. Линок,
\tsp{1}Е.Е. Мубараков\envelope,
\tsp{2}Б. Цирклер,
\tsp{1}Е.Е.Сумарева,
\tsp{1}А.С.Бухрякова
\end{header}

\begin{affil}
\tsp{1}Казахстанско-Американский свободный университет, Усть-Каменогорск, Казахстан,

\tsp{2}Westsächsische Hochschule Zwickau - Университет прикладных наук, Цвиккау, Германия,

e-mail: mubarakovyeldar@gmail.com
\end{affil}

Статья посвящена актуальным вопросам перехода к циркулярной экономике в
Казахстане. Цель исследования - выявление особенностей данного перехода
и определение мер по формированию основных элементов экономики
замкнутого цикла в стране. В ней выделяются особенности условий
Казахстана: уровень социально-экономического развития и возможности
страны, наличие, добыча, переработка полезных ископаемых и структура
промышленного производства, плотность населения и территориальный
фактор, социокультурные особенности населения. Показывается, что данные
условия «сдерживают» переход к циркулярной экономике в Казахстане. Его
основная причина -- обострение экологических проблем в стране и влияние
мирового опыта перехода к экономике замкнутого цикла. Анализируются
основное содержание и эволюция соответствующего законодательства и
нормативно-правовых актов. Отмечается, что оно «встроено» в формирование
зеленой экономики в стране и не содержит отдельного закона, необходимых
программных документов по циркулярной экономике. Показываются основные
характеристики данной экономики в Казахстане, которые сравниваются с
положительными примерами из мирового опыта. Рассматриваются проблемы
управления отходами в стране. Определяются особенности процесса,
ключевых шагов по переходу к циркулярной экономике в Казахстане.
Делается вывод о том, что в стране идет начальный этап формирования
циркулярной экономики, имеются только отдельные ее элементы. В целом в
Казахстане сложились сравнительно невысокие темпы, масштабы и результаты
перехода к циркулярной экономике. В связи с этим даются предложения по
повышению эффективности ее формирования с учетом особенностей страны и
возможностей имплементации мирового опыта.~

{\bfseries Ключевые слова:} циркулярная экономика, Казахстан, особенности
перехода, условия и основные шаги, управление отходами, имплементация.

\begin{multicols}{2}
{\bfseries Introduction.} The transition to a circular economy is a current
trend in many countries worldwide. In developed countries, primarily
European ones (Germany, the Netherlands, the UK, and others), it is the
foundation of sustainable development. In Kazakhstan, following the
adoption of the Concept for the Transition to a Green Economy in 2013,
the circular economy is viewed as an important element complementing the
existing economic growth model.

The total global circular economy market was estimated at \$583.55
billion in 2023 and is projected to grow to \$2.9 trillion by 2031- an
almost fivefold increase. European countries accounted for a significant
share of this market in 2023- approximately 38.8\%, with an estimated
value of \$226.3 billion {[}1{]}.

A circular economy is an economy in which the renewal and reproduction
of resources is ensured through innovation, and mechanisms and tools for
their repeated (cyclical) involvement in the economic system are formed
and developed {[}2{]}. Production and consumption occur in a closed
loop, leading the circular economy to be called a closed-loop economy.

It offers several positive effects: reduced environmental impacts, more
efficient use of available resources, and the emergence of an additional
source of economic growth, increased employment, and more sustainable
consumption.

The transition to a circular economy in developed countries, which began
in the 1970s, was triggered by a significant increase in environmental
problems. Recently, economic benefits have also become increasingly
important. In Europe, Germany, in particular, is a recognized leader in
the transition to a circular economy, and its experience is widely
shared internationally.

Kazakhstan is a developing Eurasian country in the early stages of
developing a circular economy. Like Germany, the significant increase in
environmental problems triggered the transition to such an economy in
Kazakhstan. Global experience in the formation and development of a
green economy and a closed-loop economy, particularly in developed
countries, also played a significant role.

Theoretical research and practical implementation of the transition to a
circular economy are of great interest to us, particularly for
developing realistic and effective proposals to accelerate the
development of the core elements of such an economy in Kazakhstan.
However, the specific characteristics of our country must be taken into
account. This is precisely the focus of this article.

Accordingly, its goal is to identify the specific features of the
transition to a circular economy in Kazakhstan and determine the
necessary steps in this direction, taking into account the potential for
implementing successful international practices.

To achieve this goal, the following key objectives must be addressed:

- considering the experience of transition to a circular economy in
Kazakhstan;

- identifying the features of this transition in our country;

- providing proposals for improving the efficiency of the transition to
a closed-loop economy in Kazakhstan, considering its specific
features, as well as the possibilities for implementing international
experience.

{\bfseries Materials and methods.} The term «circular economy» appeared at
the turn of the 80-90s of the last century. Since the late 20th century,
scientists from many countries, especially developed ones, have been
actively researching the problems of reducing the negative impact of
production and consumption on the environment, zero-waste production,
and the transition to a circular economy. These include D. Pearce, A.
Nies, R. Turner, K. Bowling, P. Antunes, D. D' Amato, G.
Daly, P. Ekins, M. Geisdorfer, W. Haas, D. Kirchherr, D. Korhonen, F.
Krausmann, P. Lacy, D. Long, S. Ritzen, G. Sandström, W. Spindler, and
many others.

By now, the results of research on the transition to a circular economy
have been largely systematized, and relevant textbooks for students have
appeared. These include general descriptions of the circular economy,
its structure and infrastructure, levels, waste and secondary resource
management, circular business models, consumer behavior patterns, tax
regulation within the circular economy, and more, as well as practical
experiences in transitioning to a circular economy in various countries.

Kazakhstani researchers more frequently use the term "circular economy."
In recent years, publications by V. Amangeldiyeva, D. Ausharipova, D.
Kalmakova, L. Kulumbetova, N. Nurmuhametov, E. Orazbekov, A.
Sakhariyeva, L. Salakhatova, Zh. Temerbulatova, S. Tleuova, S. Zeynolla,
N. Zhaksybayeva, A. Zhidebekkyzy, and other domestic authors have
addressed the issues of transitioning to a circular economy in
Kazakhstan.

The literature covers practical experiences in transitioning to a
circular economy in various countries and their regional associations,
primarily in Europe. Among these are articles devoted to Kazakhstan,
which has only just begun to develop the basic elements of a circular
economy.

For example, in the article «Towards a Circular Economy: An Analysis of
the Kazakhstani Case», Zhidebekkyzy A., Temerbulatova Zh., Amangeldiyeva
B. A. and Sakhariyeva A. analyze the formation and current state of the
circular economy in Kazakhstan {[}3{]}.

Improving the waste management system - a key element of the circular
economy in Kazakhstan - is considered, for example, in the article by
Zhidebekkyzy A., Temerbulatova Zh. \& Bilan Y. {[}4{]}, a report
prepared by Orazbekov E. {[}5{]} and others.

D. Ausharipova and L. Kulumbetova examined the circular economy as a
tool for developing «green» business in Kazakhstan {[}6{]}, and, in
collaboration with E.Tankova, they examined innovative business models
for developing eco-entrepreneurship in the waste sector in Kazakhstan
{[}7{]}. Zhaksybayeva N., Serikkyzy A. and colleagues examined the
Kazakhstani circular business ecosystem {[}8{]}, etc.

There are also scientific publications devoted to the transition to a
circular economy in Kazakhstan in comparison with other Asian countries,
primarily Central Asian ones.

Interesting material on Kazakhstan is presented, for example, in the
World Bank report «Circular Economy: A Chance for Central Asia» {[}9{]},
an article by A.B. Dolgushin on the specifics of state regulation of the
transition to a circular economy in relation to waste in Asian countries
{[}10{]}, etc.

Varavin E., Kozlova M. and Makovetskiy M. analyzed the implementation of
international experience in the field of environmentally responsible
investment in Kazakhstan {[}11{]}.

The article by Zhidebekkyzy A., Moldabekova A., Amangeldiyeva B., Šanova
P. also demonstrates differences in the readiness of society, business,
government and financial institutions, science and education to develop
a circular economy in Kazakhstan {[}12{]}.

Each of the above-mentioned and many other publications on the
transition to a circular economy in Kazakhstan is relevant, presents
theoretical and practical interest, and characterizes one or another
aspect of this complex, integrated process.

In this regard, we believe that an analysis of the specifics of the
transition to a circular economy in Kazakhstan-a developing Eurasian
state with a leading economy in Central Asia-will be relevant and
productive.

This, among other things, could allow for the effective implementation
of international experience in transitioning to a circular economy in
Kazakhstan.

Understanding the specifics of the transition to a circular economy in
Kazakhstan requires, first and foremost, an analysis of the specific
conditions for its formation and development in our country.

This analysis shows, firstly, that the transition to a costly circular
economy, which requires significant investment, is determined primarily
by the country' s level of socioeconomic development.

As is well known, the Republic of Kazakhstan is a developing country.
Its GDP in 2024 (according to the Trading Economics website) was
\$288.41 billion overall, with a per capita GDP of \$11,850.17.

For a developing country, these are good figures. However, compared to
developed countries - leaders in the circular economy - this is modest.
For example, in Germany, one of the most developed countries in the
world and the largest in Europe by GDP, its overall GDP in 2024 was
\$4.66 trillion, with a per capita GDP of \$54,990.~

That is, in terms of these key economic indicators, Kazakhstan was
behind Germany by 16.2 and 4.6 times, respectively. Clearly, Kazakhstan
is currently unable to transition to a circular economy to the same
extent as Germany.

This circumstance is hindering the speed and scale of transformation in
Kazakhstan. Therefore, we need to select priorities that align with the
country' s economic potential and capabilities.

Secondly, the structure of industrial production influences the
"urgency" of the transition to a circular economy. As noted in the
literature {[}10{]}, such a transition occurs more quickly and
effectively in countries with a deficit of minerals and other natural
resources.

Kazakhstan possesses and is developing rich deposits of various metal
ores - zinc, lead, copper, chromium, iron, uranium, gold, aluminum,
titanium, molybdenum, beryllium, tantalum, and other rare earth elements
- as well as non-metallic minerals - oil, coal, phosphorites, etc. This
explains Kazakhstan' s position in the global raw
materials market - it is an exporting country.

The extractive industry, which creates relatively low added value,
significantly reduces the severity of the problem of resource renewal
and, more generally, the transition to a circular economy. Thus, the
structure of industrial production in Kazakhstan "restrains" this need.

Thirdly, the pace and scale of the transition to a circular economy in
Kazakhstan are also determined by territorial factors-population density
and the availability of free waste disposal sites.

Global experience shows that for countries with high population density
and a lack of free waste disposal sites, the transition to a circular
economy is especially important and inevitable. Short distances between
waste management facilities significantly reduce logistics costs and
make waste collection, processing, and disposal cheaper. This stimulates
the transition to a circular economy.~

{\bfseries Results and discussions.} As is well known, Kazakhstan has a low
population density - approximately 6.6 people/km² (as of 2024). For
comparison, in Germany, this figure is approximately 35 times higher -
approximately 233 people/km². Consequently, Kazakhstan' s
population is dispersed, with distances between settlements and waste
management facilities typically large. This negatively impacts on the
profitability of waste collection and recycling. As a result, the
development of a circular economy in Kazakhstan is being hindered from
an economic perspective.

There are also other conditions (factors) that determine the specifics
of the transition to a circular economy in various countries, including
non-economic ones. One such factor is the socio-cultural factor. This
circumstance is also noted in Russian literature {[}13{]}.

It can be noted, in particular, that Kazakhstanis typically do not
consider the separation of solid municipal waste by type; they are often
primarily concerned with adhering to the waste collection schedule.
It' s no coincidence that attempts to introduce separate
waste collection in 2018-2019 in Astana, Aktobe, Semey, Turkestan, and
Oskemen were virtually unsuccessful. This also speaks to
society' s relatively low readiness for the transition to
a circular economy.

In contrast, it' s common for any resident of a Western
European country to have several garbage bags in the kitchen-for paper,
plastic, glass, and food waste-which are then disposed of in the
appropriate bins according to established procedures. This is done
automatically, without any reminders.

Thus, the listed key conditions for the formation of a closed-loop
economy in Kazakhstan do not play an active stimulating role in the
transition to a circular economy.

This, in turn, determines significant characteristics of the process
itself and the key steps towards a circular economy.

As is known, the significant increase in environmental problems
triggered the transition to a circular economy in Kazakhstan.
International experience in the formation and development of a green
economy and a closed-loop economy also played an important role.

According to available data {[}14{]}, by 2024, more than 125 million
tons of waste had accumulated in Kazakhstan. Moreover, the volume of
municipal waste removed has been steadily increasing in recent years. In
2022, it was 3.8 million tons, in 2023 - 4.1 million tons, and in 2024 -
4.8 million tons. As a result, in these three years alone, this figure
has increased by 1 million tons (26.3\%).

Currently, there are 2,973 municipal solid waste (MSW) landfills in
Kazakhstan, of which only 608 (20.4\%) comply with sanitary and
environmental standards. Furthermore, in 2024 alone, 4,886 unauthorized
landfills were identified in Kazakhstan, 93\% of which have been
eliminated {[}14{]}.

In 2013, the Republic of Kazakhstan adopted the Concept for the
Transition to a Green Economy, which includes Section 3.5. «Waste
Management System» {[}15{]}. This section stipulates that Kazakhstan
must essentially re-establish its waste management system due to the
lack of the legal and organizational framework required for its optimal
functioning. This system must also incorporate elements of the 3R
(reduce-reuse-recycle) principles.

To implement the Concept, the Government successively approved two
Action Plans, which also defined the objectives for improving the waste
management system. Thus, the Plan for the period from 2013 to 2020 set
objectives, for example, for the separate collection of household waste
from consumers. The Plan for 2021-2030 highlights measures that follow
the principles of a circular economy: the development of special support
measures for the development of the waste management industry, including
its recycling; the development of organic waste processing to produce
biogas; and the construction of biogas plants at wastewater treatment
plants and poultry farms {[}10{]}.

In 2015-2016, the principle of extended producer responsibility (EPR)
was introduced in Kazakhstan. In this regard, corresponding amendments
were made to the Environmental Code of the Republic of Kazakhstan, and
government resolutions were adopted, «On Approving the Rules for the
Implementation of Extended Obligations of Producers (Importers)» and «On
Determining the EPR Operator (Importers)». The primary focus was on the
recycling fee that product manufacturers and importers were required to
pay to the private enterprise, EPR Operator LLP. Following well-known
corruption cases, the functions of the EPR operator were transferred to
the state-owned company, Zhasyl Damu JSC, in 2022.

In fact, as international experience shows, this is a broader concept,
encompassing the obligations of manufacturers and importers of goods
that have lost their consumer properties to collect, transport, process,
render harmless, use, and dispose of waste covered by EPR conditions.

In Kazakhstan, there are also specific measures within various
government programs that align with the principles of a circular
economy. Thus, in 2021, the government adopted the National Project
«Zhasyl Kazakhstan» («Green Kazakhstan»). It, among other things, sets
out objectives for improving the efficiency of waste management.

Kazakhstan is currently implementing the «Taza Kazakhstan» concept
(2024-2029), which aims to foster environmental awareness and increase
waste recycling. The Carbon Neutrality Strategy by 2060, which aims to
reduce carbon emissions and develop circular business models, and
national circular economy standards are also being implemented to
facilitate the transition to a circular economy.

In 2024, amendments and additions were made to the Concept for the
Transition of the Republic of Kazakhstan to a Green Economy, including
the section on waste management. These amendments call for the
widespread implementation of separate waste collection, increased
recycling, and reuse of both municipal and industrial waste.

The goal of implementing separate waste collection was already included
in the Government Action Plan for 2013-2020 but was not implemented in a
timely manner.

It should also be noted that all these state-adopted Concepts, Plans,
Resolutions, and other policy documents do not contain a comprehensive,
systemic, or integrated approach to implementing the transition to a
circular economy. A separate law on the circular economy - as in
Germany, for example - is absent.~

In essence, legislation and steps toward a circular economy in
Kazakhstan are «integrated» into the formation and development of a
green economy. This, in particular, distinguishes Kazakhstan from, for
example, developed Western European countries that are successfully
developing a circular economy.~

As noted, Kazakhstan places particular emphasis on its waste management
system. Table 1 presents statistical data characterizing it across
several key indicators for 2021-2024.
\end{multicols}

\tcap{Table 1 - Key Indicators Characterizing Kazakhstan' s Waste Management System, 2021--2024}
\begin{longtblr}[
  label = none,
  entry = none,
]{
  width = \linewidth,
  colspec = {Q[25]Q[596]Q[73]Q[73]Q[73]Q[73]},
  row{1} = {c},
  cell{2}{1} = {c},
  cell{2}{3} = {c},
  cell{2}{4} = {c},
  cell{2}{5} = {c},
  cell{2}{6} = {c},
  cell{3}{1} = {c},
  cell{3}{3} = {c},
  cell{3}{4} = {c},
  cell{3}{5} = {c},
  cell{3}{6} = {c},
  cell{4}{1} = {c},
  cell{4}{3} = {c},
  cell{4}{4} = {c},
  cell{4}{5} = {c},
  cell{4}{6} = {c},
  cell{5}{1} = {c},
  cell{5}{3} = {c},
  cell{5}{4} = {c},
  cell{5}{5} = {c},
  cell{5}{6} = {c},
  cell{6}{1} = {c},
  cell{6}{3} = {c},
  cell{6}{4} = {c},
  cell{6}{5} = {c},
  cell{6}{6} = {c},
  cell{7}{1} = {c},
  cell{7}{3} = {c},
  cell{7}{4} = {c},
  cell{7}{5} = {c},
  cell{7}{6} = {c},
  cells = {font = \small},
  hlines,
  vlines,
}
№   & \textbf{Indicators}                                                         & \textbf{2021} & \textbf{2022} & \textbf{2023} & \textbf{2024} \\
1   & Volume of received waste, thousand tons                                     & 3922.1        & 4003.4        & 4514.0        & 4541.8        \\
2   & Volume of sorted waste, thousand tons                                       & 1191.7        & 1223.8        & 1307.8        & 1320.1        \\
3   & Volume of waste sent to third-party organizations, thousand tons.Including: &               &               &               &               \\
3.1 & for recycling                                                               & 393.7         & 392.0         & 358.9         & 588.2         \\
3.2 & for disposal                                                                & 673.6         & 707.8         & 862.3         & 490.4         \\
4   & Processed to obtain products (secondary raw materials), thousand tons       & 18.1          & 23.3          & 31.2          & 49.2          
\end{longtblr}

\emph{Note: complied by authors based on data {[}16{]}}

\begin{multicols}{2}
An analysis of the presented data shows that the volume of waste
received has been steadily increasing in recent years: from 3,922,100
tons in 2021 to 4,541,800 tons in 2024, an increase of 619,700 tons, or
15.8\%.

The volume of sorted waste shows a similar trend. Between 2021 and 2024,
it increased from 1,191,700 tons to 1,320,100 tons, or an increase of
128,400 tons (10.8\%).

This is clearly illustrated by the diagrams in Figure 1.
\end{multicols}

\fig{e3/image13}[Fig.1 - Trends in Volume of Received and Sorted Waste in Kazakhstan, 2021-2024 (thousand tons)\\\normalfont{\emph{Note: complied by authors based on data {[}16{]}}}]

\begin{multicols}{2}
At the same time, a comparative analysis of these indicators shows that
the share of sorted waste in the total volume of received waste remains
virtually unchanged. As shown in Figure 2, during the period under
review, it was: 30.4\% in 2021, 30.6\% in 2022, 29.0\% in 2023, and
29.1\% in 2024. In other words, no positive dynamics are observed.
\end{multicols}

\fig{e3/image14}[Fig.2 - Percentage Share of Sorted Waste in Total Received Waste, 2021-2024\\\normalfont{\emph{Note: complied by authors based on data {[}16}{]}}]

As for the volume of waste sent to third-party organizations for
recycling and disposal, the available statistics show mixed and rather
abrupt changes. This prevents any consistent conclusions.

The indicator «processed to produce products (secondary raw materials)»
is growing from 18.1 thousand tons in 2021 to 49.2 thousand tons in
2024, an increase of approximately 2.7 times. Figure 3 illustrates this
substantial growth. However, in absolute terms, measured in tons, this
increase is relatively insignificant.

\fig{e3/image15}{}

{\bfseries Fig.3 - Volume of Waste Processed to Obtain Products (Secondary
Raw Materials) in Kazakhstan, 2021--2024}

\emph{Note: complied by authors based on data {[}16{]}}

To effectively manage waste, Kazakhstan has over 50 plastic recycling
plants, approximately 40 plants that recycle waste paper, 7 that recycle
glass, and approximately 20 that recycle rubber products {[}14{]}. In
2024, a plant that recycles batteries opened in Almaty. It should be
noted that waste recycling plants in the country are primarily
concentrated in Almaty, and transporting raw materials there from other
regions of Kazakhstan is usually unprofitable, primarily due to high
logistics costs.

Overall, an analysis of Kazakhstan' s solid waste sector
reveals several waste management issues:

- the unsystematic nature of the existing legislative framework and
regulatory legal acts in this area;

- annual increase in the generation and accumulation of household waste;

- waste collection and removal services are available only to residents
of large cities;

- the concentration of solid waste recycling facilities primarily in
Almaty and a shortage in other regions;

- underdeveloped waste recycling and disposal businesses;

- low levels of solid waste sorting and recycling;

- the failure of most landfills to meet requirements, with the presence
of open dumps;

- the lack of environmental awareness and waste management culture among
the majority of the population;

- insufficient training of government officials dealing with green
economy issues;

- ineffective attempts at separate waste collection in some cities.

To ensure a successful transition to a circular economy, studying and
implementing best international practices is crucial in Kazakhstan. Of
particular interest in this regard is the experience of Germany, where a
circular economy is the foundation of the country' s
sustainable development.

Germany is currently the leader among EU countries in terms of circular
economy development.100\% of waste is collected separately, 64\% of
household waste is recycled, 79\% of all waste is reused, 14\% of
necessary raw materials are recovered from waste, etc. {[}17{]}

The use of recycled materials significantly reduces energy consumption,
in particular: by more than 90\% in aluminum production, by
approximately 50\% in steel production, and by 35\% in glass production
{[}17{]}. For Germany, an industrialized country, this allows for a
significant reduction in the extraction and, most importantly, import of
natural resources.

Overall, the formation and development of a circular economy in Germany
has been systematic and gradual since the early 1970s. The following
steps were taken in establishing and improving the recycling and
disposal industry {[}17{]}.

First, political and strategic management was implemented. This entailed
creating the necessary legislative framework and regulations, as well as
government agencies responsible for implementing and monitoring the
process, and assigning responsibilities for waste collection,
processing, and disposal.

Second, social development was implemented. This includes fostering
environmental awareness among people, their positive attitude toward the
circular economy, and continuous training for personnel throughout the
industry.

Third, cost and financing were determined and analyzed. This entails
identifying and attracting funding sources, obtaining possible
subsidies, grants, and incentives, reimbursing costs for waste
processing and disposal companies, etc.

Fourth, stimulating the waste recycling market. This step involves
implementing programs to support investment in the waste processing and
recycling industry, reducing the tax burden, and identifying stable
markets for recovered secondary raw materials.

Fifth, technical capabilities. This includes separate waste collection,
the use of modern equipment for waste removal, the use of efficient
technologies for processing various types of waste, and the availability
of specialized waste storage facilities.

All of this requires studying this experience and identifying the
possibilities for its application in Kazakhstan, considering the
specifics of the country' s socioeconomic development and
mandatory adaptation to national circumstances.

In examining the transition to a circular economy in Kazakhstan, general
scientific methods of inquiry were used, such as scientific abstraction,
analysis and synthesis, induction and deduction. In addition,
comparison, tabulation, and analogy methods were applied. Data from
legislation and regulations, statistical and factual materials were also
used.

To improve the effectiveness of further steps toward developing a
circular economy in Kazakhstan, we believe it is advisable to first
examine the specific conditions for transitioning to it in our country.
Our previous analysis allows us to identify the following key factors:

- the underlying reason for the transition to a circular economy: the
worsening environmental challenges and the impact of global experience
in the transition to a circular economy;

- the main priorities for the transition to a circular economy at
present: along with environmental challenges, economic factors also
have a minor impact;

- the level of socio-economic development: Kazakhstan is a developing
country;

- the country' s economic potential to support the
transition to a circular economy: Limited compared to developed
countries;

- availability of minerals and other natural resources: Some of the
largest reserves in the world, with mining and processing of minerals
significantly exceeding domestic consumption;

- the structure of industrial production: The extractive industry plays
a key role, while the high-tech manufacturing industry is
comparatively less developed;

- position in the global raw materials market: An exporting country;

- population density: Low;

- availability of free waste disposal areas: Significant amounts
available;

- distances for transporting waste to processing facilities and the
associated transport costs: distances are typically long, and
logistics costs are also typically high;

- socio-cultural characteristics of the country' s
population: there is little inclination among the population to
separate waste; entrepreneurship in the circular economy is not
prestigious in Kazakhstani society and is in its infancy.

It should be noted that some of the conditions for the transition to a
circular economy are virtually impossible to change. These include,
first and foremost, the availability of mineral resources, its position
in the global raw materials market, and territorial factors. With
successful development, the country' s level of
socioeconomic development and, consequently, the ability to support
these transformations, the structure of industrial production, the
mentality of the population and businesses, and population density may
change over time.

Overall, the analysis showed that the key conditions for the formation
of a circular economy in Kazakhstan do not compel the transition to a
closed-loop economy and do not play an active role in this process. This
largely determines the relatively low pace, scale, and results of the
transition to a circular economy.

This circumstance, in turn, determined the characteristics of the key
steps in the transition to a circular economy in Kazakhstan. They are
presented in Table 2.

{\bfseries Table 2 - Characteristics of the Key Steps in the Transition to
a Circular Economy}

{\bfseries in Kazakhstan}

%% \begin{longtable}[]{@{}
%%   >{\raggedright\arraybackslash}p{(\linewidth - 2\tabcolsep) * \real{0.4999}}
%%   >{\raggedright\arraybackslash}p{(\linewidth - 2\tabcolsep) * \real{0.5001}}@{}}
%% \toprule\noalign{}
%% \begin{minipage}[b]{\linewidth}\centering
%% {\bfseries Main characteristics}
%% \end{minipage} & \begin{minipage}[b]{\linewidth}\centering
%% {\bfseries Features in Kazakhstan}
%% \end{minipage} \\
%% \midrule\noalign{}
%% \endhead
%% \bottomrule\noalign{}
%% \endlastfoot
%% The government' s stated start date for the transition to
%% a circular economy & Early 2010s \\
%% Adoption of the first legislative act on the circular economy & 2013:
%% Concept for the transition of the Republic of Kazakhstan to a green
%% economy (section on the waste management system) \\
%% The existence of a system of laws and regulations that ensure the
%% transition to a circular economy & The system has not been established.
%% There is no separate law on the circular economy. Provisions for the
%% transition to a circular economy are mainly integrated into green
%% economy legislation. There are regulations on waste sorting and
%% recycling \\
%% The presence of government bodies implementing and monitoring the
%% transition to a circular economy & Special government agencies have not
%% yet been fully formed. Civil servants are insufficiently trained \\
%% The existence and implementation of the Extended Producer Responsibility
%% (EPR) system & The EPR system has been formed and implemented with its
%% own characteristics (including the significant role of the recycling fee
%% on vehicles) \\
%% The presence of environmental awareness and a modern waste management
%% culture among the population and businesses & Some measures are being
%% taken to develop environmental awareness and a modern waste management
%% culture among the population \\
%% Compliance with legislative and regulatory requirements for the circular
%% economy & Incomplete implementation, low implementation and
%% effectiveness. The presence of corruption risks. \\
%% Compliance of municipal waste landfills with modern sanitary and
%% environmental requirements & A small proportion of municipal waste
%% landfills comply \\
%% The presence of unauthorized dumps & They are available in large
%% quantities and are being partially eliminated \\
%% Level of waste sorting and recycling & Low \\
%% The presence of a system of enterprises for sorting and processing all
%% major types of waste throughout the country & There are separate
%% enterprises for waste sorting and processing (mainly in Almaty), but the
%% system is not formed \\
%% The presence of a market for waste recycling & There are some elements
%% of this market \\
%% The result of implementing steps towards a circular economy & The
%% initial stage of the formation of the circular economy is underway, and
%% its individual elements exist \\
%% \end{longtable}

\emph{Note:compiled by authors}

Thus, the analysis shows that there are many unique aspects to the
transition to a circular economy in Kazakhstan. These are determined by
the country' s prevailing conditions, such as the level
of socioeconomic development, the availability of mineral resources,
territorial factors, and so on.

Moreover, Kazakhstan' s transition to a circular economy
began with a «lag» of approximately 35-40 years compared to developed
countries, and this circumstance offers a distinct advantage and
additional opportunities for Kazakhstan. Knowing and building on the
successful experience of transitioning to a circular economy-for
example, in Germany-may enable its implementation in Kazakhstan to a
certain extent. This will allow for more effective development of a
circular economy in Kazakhstan, naturally considering the specific
characteristics of our country.

In our opinion, based on this premise, Kazakhstan needs to prioritize,
including by drawing on global experience:~

- development of a comprehensive legislative and regulatory framework
for the circular economy, including the adoption of a law on the
circular economy;

- identification of government agencies responsible for organizing and
monitoring the implementation of these laws and regulations;

- raising the level of environmental awareness in Kazakhstani society
and its acceptance of the challenges and objectives of the circular
economy;

- training of personnel for government agencies and enterprises in the
circular economy, including entrepreneurs;

- development and implementation of government support measures for
waste recycling enterprises and the creation of products and energy
from the resulting raw materials;

- combating unauthorized dumping and creating a system of modern
landfills in all regions of the country;

- widespread implementation of separate waste collection;

- implementation of modern technologies and equipment for processing
various types of waste;

- creation of waste recycling enterprises and a material recycling
system in all regions of the country;

- promotion of businesses operating in the circular economy;

- support for scientific and applied research in the country on circular
economy issues with stimulation of subsequent commercialization of
their results.

Thus, an analysis of the development of a circular economy in Kazakhstan
revealed the specific conditions and, secondly, the process (steps) of
transitioning to such an economy. International experience in
transformation can be implemented in Kazakhstan, while considering the
specific characteristics of our country.

{\bfseries Conclusion.} The study shows that the initial reason for the
transition to a circular economy in Kazakhstan is the aggravation of
environmental problems, as well as the influence of similar experiences
in other countries.

The pace, scale, and results of the development of a circular economy in
Kazakhstan, which began in 2013, are primarily determined by the
specific conditions our country faces for transitioning to such an
economy. These include the level of socioeconomic development and the
corresponding opportunities for transformation, the availability of
minerals and other natural resources, the structure of industrial
production, the country' s position in the global raw
materials market, population density, the availability of free space for
waste disposal, the sociocultural characteristics of the population, and
so on.

As the analysis shows, the conditions for the development of a circular
economy in Kazakhstan are "restraining" the need to transition to this
economy.

The study also highlights the specifics of the process and key steps in
the transition to a circular economy in Kazakhstan. This is evident in
the existence and implementation of relevant legislation and
regulations, the creation of effective government bodies implementing
and overseeing the transition to a closed-loop economy, the
implementation of the "polluter pays" principle, the development of
environmental awareness and understanding of the challenges of
transitioning to a circular economy among the population and businesses,
and so on.

As a result of the ongoing reforms, Kazakhstan is currently in the
initial stages of developing a circular economy, with certain elements
already in place.

At the same time, the analysis demonstrates the universality and
universality of the same steps in transitioning to a circular economy in
Kazakhstan and abroad. Kazakhstan must continue to develop a circular
economy in accordance with its national interests and specific
characteristics. At the same time, it is advisable to implement best
international practices. This will allow for a more systematic and
effective transition to a closed-loop economy, considering the specific
characteristics of our country.~

{\bfseries Литература}

1. Рынок круговой экономики 2024--2031. URL:
\url{https://www.kingsresearch.com/ru/circular-economy-market-1046.-}
Дата обращения: 12.06.2025

2. Рязанова О.Е., Золотарева В.П. Циркулярная экономика: M.: KNORUS,
2025. -117 s. ISBN 978-5-406-01345-8

3. Жидебеккызы А., Темирбулатова Ж., Амангельдиева Б., Сахариева А. На
пути к экономике замкнутого цикла: анализ ситуации в Казахстане// Журнал
экономических исследований и делового администрирования. - 2023.- №
143(1).- С.16-32.

\href{https://doi.org/10.26577/be.2023.v143.i1.02}{DOI
10.26577/be.2023.v143.i1.02}.

4. Zhidebekkyzy, A., Temerbulatova, Z., Bilan, Y. The Improvement Of The
Waste Management System In Kazakhstan: Impact Evaluation//Polish Journal
of Management Studies, 25(2), 423--439.
\url{https://doi:10.17512/pjms.2022.25.2.27}.

5. Оразбеков Е. Отчет «Факторы, способствующие эффективному РОП в
Азиатско-Тихоокеанском регионе: уроки развитых и развивающих стран ЕС и
Азии. Казахстан. -- 2025- 53 с. URL

\href{http://www.switch-asia.eu/site/assets/files/4420/epr_kazakhstan_report_ru.pdf.-Дата}{www.switch-asia.eu/site/assets/files/4420/epr\_kazakhstan\_report\_ru.pdf.-Дата}
обращения:12.06.2025.

6. Аушарипова Д.Е., Кулумбетова Циркулярная экономика как инструмент
развития «зеленого» бизнеса в Казахстане// Вестник университета «Туран»-
2020. - № 3. - С.190-196,

\href{https://doi.org/10.46914/1562-2959-2020-1-3-190-196}{DOI
10.46914/1562-2959-2020-1-3-190-196}.

7. Аушарипова Д.Е., Кулумбетова Л.Б., Танкова Е. Инновационные
бизнес-модели для развития экологического предпринимательства в сфере
отходов Казахстана// Вестник университета «Туран»- 2023. - № 4. -
С.174-185.DOI 10.469/1562-2959-2023-1-4-174-187.

8. Zhaksybayeva, N., Serikkyzy, A., Baktymbet, A., Yousafzai, S.
Circular shifts: insights into Kazakhstan's circular business
ecosystem//Cogent Business \& Managementю-2024.-Vol.11(1) 11(1):2431652.
\href{https://doi.org/10.1080/23311975.2024.2431652}{DOI
10.1080/23311975.2024.2431652}.

9. World Bank. Circular Economy as an Opportunity for Central Asia -
Summary Report (English). Washington, D.C.: World Bank Group.
\url{http://documents.worldbank.org/curated/en/099052024074569900}.-
accessed on 23.08.2025.

10. Долгушин А.Б. Особенности государственного регулирования перехода на
экономику замкнутого цикла в отношении отходов в странах Азии// Вестник
российского университета кооперации//Central Asian Economic Review. -
2022. - № 4(50).- С.21-28.

11. Варавин Е.В., Козлова М.В., М.Ю. Маковецкий Развитие экологически
ответственного инвестирования: имплементация зарубежного опыта для
Казахстана// Central Asian Economic Review.-2021.-№ 4.- С.52-63.
\href{https://doi.org/10.52821/2789-4401-2021-4-52-63}{DOI
10.52821/2789-4401-2021-4-52-63}.

12. Zhidebekkyzy A., Moldabekova A., Amangeldiyeva B., Šanova P.
Transition to a circular economy: Exploring stakeholder perspectives in
Kazakhstan//Journal of International Studies. -2023.-Vol.16(3).-
P.144-158. DOI 10.14254/2071-8330.2023/16-3/8.

13. Zhidebekkyzy A., Moldabekova A., Amangeldiyeva B., Streimikis J.
Assessment of factors influencing pro-circular behavior of a
population//Economics \& Sociology. - 2022.-Vol.15(3)-P.202-215.
\href{https://doi:10.14254/2071-789x.2022/15-3/12}{DOI
10.14254/2071-789x.2022/15-3/12}.

14. Казинформ.125 млн тонн мусора скопилось на казахстанских свалках:
кто виноват, и что делать? URL:
\url{https://www.inform.kz/ru/125-mln-tonn-musora-skopilos-na-kazahstanskih-svalkah-kto-vinovat-i-chto-delat-d815d7-}
Дата обращения: 12.06.2025

15. О концепции по переходу Республики Казахстан к «зеленой экономике»
URL: \url{https://adilet.zan.kz/rus/docs/U1300000577-} Дата обращения:
12.06.2025

16. Окружающая среда -- Показатели «зеленой экономики.
https://stat.gov.kz/ru/

17. Таскин Ф.А. Реализация концепции экономики замкнутого цикла в
Германии// научно-Концепт. -2024. -№ 03. - С.193-199.DOI
10/24412/2304-120X-2024-13001.

{\bfseries References}

1. Rynok krugovoj jekonomiki 2024-2031. URL:
https://www.kingsresearch.com/ru/circular-economy-market-1046.- Data
obrashhenija: 12.06.2025. {[}in Russian{]}

2. Rjazanova O.E., Zolotareva V.P. Cirkuljarnaja jekonomika: M.: KNORUS,
2025. -117 s. ISBN 978-5-406-01345-8. {[}in Russian{]}

3. Zhidebekkyzy A., Temirbulatova Zh., Amangel' dieva B.,
Saharieva A. Na puti k jekonomike

zamknutogo cikla: analiz situacii v Kazahstane// Zhurnal jekonomicheskih
issledovanij i delovogo administrirovanija. - 2023.- № 143(1).- S.16-32.
DOI 10.26577/be.2023.v143.i1.02. {[}in Russian{]}

4. Zhidebekkyzy, A., Temerbulatova, Z., Bilan, Y. The Improvement Of The
Waste Management System In Kazakhstan: Impact Evaluation//Polish Journal
of Management Studies, 25(2), 423--439.
\url{https://doi:10.17512/pjms.2022.25.2.27}. {[}in Russian{]}

5. Orazbekov E. Otchet «Faktory, sposobstvujushhie jeffektivnomu ROP v
Aziatsko-Tihookeanskom regione: uroki razvityh i razvivajushhih stran ES
i Azii. Kazahstan. -- 2025- 53 s. URL
www.switch-asia.eu/site/assets/files/4420/epr\_kazakhstan\_report\_ru.pdf.-Data

obrashhenija:12.06.2025. {[}in Russian{]}

6. Ausharipova D.E., Kulumbetova Cirkuljarnaja jekonomika kak instrument
razvitija «zelenogo» biznesa v Kazahstane// Vestnik universiteta
«Turan»- 2020. - № 3. - S.190-196,

DOI 10.46914/1562-2959-2020-1-3-190-196. {[}in Russian{]}

7. Ausharipova D.E., Kulumbetova L.B., Tankova E. Innovacionnye
biznes-modeli dlja razvitija jekologicheskogo
predprinimatel' stva v sfere othodov Kazahstana// Vestnik
universiteta «Turan»- 2023. - № 4. - S.174-185.DOI
10.469/1562-2959-2023-1-4-174-187. {[}in Russian{]}

8. Zhaksybayeva, N., Serikkyzy, A., Baktymbet, A., Yousafzai, S.
Circular shifts: insights into Kazakhstan's circular business
ecosystem//Cogent Business \& Managementю-2024.-Vol.11(1) 11(1):2431652.
\href{https://doi.org/10.1080/23311975.2024.2431652}{DOI
10.1080/23311975.2024.2431652}.

9. World Bank. Circular Economy as an Opportunity for Central Asia -
Summary Report (English). Washington, D.C.: World Bank Group.
\url{http://documents.worldbank.org/curated/en/099052024074569900}.-
accessed on 23.08.2025.

10. Dolgushin A.B. Osobennosti gosudarstvennogo regulirovanija perehoda
na jekonomiku zamknutogo cikla v otnoshenii othodov v stranah Azii//
Vestnik rossijskogo universiteta kooperacii//Central Asian Economic
Review. - 2022.- № 4(50).- S.21-28. {[}in Russian{]}

11. Varavin E.V., Kozlova M.V., M.Ju. Makoveckij Razvitie jekologicheski
otvetstvennogo investirovanija: implementacija zarubezhnogo opyta dlja
Kazahstana// Central Asian Economic Review. - 2021.-№ 4.- S.52-63. DOI
10.52821/2789-4401-2021-4-52-63. {[}in Russian{]}

12. Zhidebekkyzy A., Moldabekova A., Amangeldiyeva B., Šanova P.
Transition to a circular economy: Exploring stakeholder perspectives in
Kazakhstan//Journal of International Studies. -2023.-Vol.16(3).-
P.144-158. DOI 10.14254/2071-8330.2023/16-3/8.

13. Zhidebekkyzy A., Moldabekova A., Amangeldiyeva B., Streimikis J.
Assessment of factors influencing pro-circular behavior of a
population//Economics \& Sociology. - 2022.-Vol.15(3)-P.202-215.
\href{https://doi:10.14254/2071-789x.2022/15-3/12}{DOI
10.14254/2071-789x.2022/15-3/12}.

14. Kazinform.125 mln tonn musora skopilos'{} na
kazahstanskih svalkah: kto vinovat, i chto delat'? URL:
https://www.inform.kz/ru/125-mln-tonn-musora-skopilos-na-kazahstanskih-svalkah-kto-vinovat-i-chto-delat-d815d7-
Data obrashhenija: 12.06.2025. {[}in Russian{]}

15. O koncepcii po perehodu Respubliki Kazahstan k «zelenoj jekonomike»
URL: https://adilet.zan.kz/rus/docs/U1300000577- Data obrashhenija:
12.06.2025. {[}in Russian{]}

16. Okruzhajushhaja sreda -- Pokazateli «zelenoj jekonomiki.
https://stat.gov.kz/ru/

17. Taskin F.A. Realizacija koncepcii jekonomiki zamknutogo cikla v
Germanii// nauchno-Koncept. -2024. -№ 03. - S.193-199.DOI
10/24412/2304-120X-2024-13001. {[}in Russian{]}

\emph{{\bfseries Information about the authors}}

Linok Y.N. - student, Kazakh-American Free University, Oskemen,
Kazakhstan, e-mail: linok\_yana@bk.ru;

Mubarakov Y.Y. - PhD student, Kazakh-American Free University, Oskemen,
Kazakhstan, e-mail:
mubarakovyeldar@gmail.com;

Zirkler B. - PhD, professor, Westsächsische Hochschule Zwickau
-- University of Applied Sciences, Zwickau, Germany; e-mail:
bernd.zirkler@fh-zwickau.de;

Sumareva Y.Y. - senior lecturer-researcher of the Business Department,
Kazakh-American Free University, Oskemen, Kazakhstan, e-mail:
xee\_89@mail.ru;

Bukhryakova A.S. -- master's student, Kazakh-American Free University,
Oskemen, Kazakhstan, e-mail: aminabukhryakova@gmail.com.

\emph{{\bfseries Сведения об авторах}}

Линок Я.Н.- cтудент, Казахстанско-Американский свободный университет,
Усть-Каменогорск, Казахстан, e-mail: linok\_yana@bk.ru;

Мубараков Е.Е. - PhD докторант, Казахстанско-Американский свободный
университет, Усть-Каменогорск, Казахстан, e-mail:
mubarakovyeldar@gmail.com;

Цирклер Б. - PhD, профессор, Westsächsische Hochschule Zwickau
-Университет прикладных наук, Цвиккау, Германия, е-mail:
bernd.zirkler@fh-zwickau.de;

Сумарева Е.Е. - старший преподаватель-исследователь кафедры
«Бизнеса», Казахстанско-Американский свободный университет,
Усть-Каменогорск, Казахстан, е-mail: xee\_89@mail.ru;

Бухрякова А.С. - магистрант, Казахстанско-Американский свободный
университет, Усть-Каменогорск, Казахстан, е-mail:
aminabukhryakova@gmail.com.
