\id{ҒТАМР 06.77.05}{}

\begin{header}
\swa{}{ҚАЗАҚСТАНДАҒЫ КАДРЛАРДЫ БАСҚАРУДЫҢ МЕМЛЕКЕТТІК АУДИТІ: НЕГІЗГІ ҚИЫНДЫҚТАР ЖӘНЕ ОЛАРДЫ ЖЕҢУ ЖОЛДАРЫ}

А.Р. Мукушева\envelope,
М.А. Серикова
\end{header}

\begin{affil}
Л.Н. Гумилев атындағы Еуразия ұлттық университеті, Астана, Казақстан

\corrauthor{Корреспондент-автор: suleimenova.ar@bk.ru}
\end{affil}

Кадрлық ресурстарын басқарудың мемлекеттік аудиті мемлекеттің кадрлық
әлеуетін дамытудың маңызды аспектісі болып табылады. Кадрлық ресурстары
Қазақстан Республикасы мемлекеттік қызметшілерінің басқару құрылымының
нақты архитектурасын құруға жауапты механизмді білдіреді. Мемлекеттік
басқарудағы кадрлық ресурстарының рөлін күшейту мемлекет алдына қойылған
стратегиялық мақсаттарға қол жеткізуге мүмкіндік береді.

Жаһандық өзгерістер үнемі болып жатқан қазіргі қоғамда ішкі ортаға әсер
ететін факторларды, атап айтқанда, елдің кадрлық әлеуетін дамыту
тұрғысынан ескеру маңызды, бұл өз кезегінде кадрлық ресурстарының
мемлекеттік аудитінде маңызды рөл атқара алады.

Кадрлық ресурстарын дамытудың жоғары деңгейі елдің ұлттық экономикасының
тұрақты дамуына ықпал етеді. Кадрлық ресурстарының мемлекеттік аудитіне
маңызды рөл берілуі керек, өйткені тек мемлекеттік аудит қана кадрлық
ресурстарын пайдалану тиімділігін бағалауға мүмкіндік береді. Айта кету
керек, бүгінгі таңда кадрлық ресурстарының тиімділігіне аудит жүргізудің
нақты стратегиясын әзірлеу тетіктері жоқ, бұл өз кезегінде осы күрделі
міндеттерді шешуде бірқатар проблемалар бар деп қорытынды жасауға
мүмкіндік береді.

Бұл мақаланың мақсаты -- Қазақстандағы кадрлық ресурстарын басқарудың
мемлекеттік аудитін зерттеу саласындағы өзекті проблемаларды және оларды
еңсеру жолдарын зерттеу. Мақсатқа сәйкес мақалада Қазақстандағы кадрлық
ресурстарын басқару аудиті жүйесінің ағымдағы жай-күйінің талдауы
берілген, мемлекеттік органдардың кадр саясатына аудит жүргізудегі
негізгі проблемалар мен кемшіліктер анықталған, аудит әдістерін
жетілдіру және қызметтің негізгі көрсеткіштері (KPI) жүйесін жетілдіру
бойынша ұсыныстар әзірлеу мүмкіндігі қарастырылған.

Үнемі өзгеріп отыратын ортада еліміздің стратегиялық мақсаттарына сәйкес
басқаруды тиімді жүргізу үшін елдің кадрлық әлеуетін дамытуға әсер
ететін барлық факторларды ескеру маңызды.

{\bfseries Негізгі сөздер:} аудит, кадрлық ресурстартар, кадрлық
ресурстарының тиімділік аудиті

\begin{header}
ГОСУДАРСТВЕННЫЙ АУДИТ УПРАВЛЕНИЯ КАДРОВЫМИ РЕСУРСАМИ В КАЗАХСТАНЕ: КЛЮЧЕВЫЕ ВЫЗОВЫ И ПУТИ ИХ ПРЕОДОЛЕНИЯ

А.Р. Мукушева,
М.А. Серикова
\end{header}

\begin{affil}
Евразийский национальный университет им. Л.Н.Гумилева, Астана, Казахстан,

e-mail: suleimenova.ar@bk.ru
\end{affil}

Государственный аудит управления кадровыми ресурсами является важным
аспектом в развитии кадрового потенциала государства. Кадровые ресурсы
представляют собой механизм, отвечающий за построение четкой архитектуры
управленческой структуры государственных служащих Республики Казахстан.
Усиление роли кадровых ресурсов в государственном управлении позволит
решить стратегические цели поставленные перед государством.

В современном обществе, где постоянно происходят глобальные изменения,
важно учитывать факторы, влияющие на внутреннюю среду, в частности в
свете развития кадрового потенциала страны, который в свою очередь может
оказывать немаловажную роль на государственный аудит кадровых ресурсов.

Высокий уровень развития кадровых ресурсов способствует устойчивому
развитию национальной экономики страны. Важную роль следует уделить
государственному аудиту кадровых ресурсов, ведь только государственный
аудит позволит оценить эффективность использования работы кадровых
ресурсов. Следует отметить, что на сегодняшний день отсутствуют
механизмы по выработке четкой стратегии проведения аудита эффективности
кадровых ресурсов, что в свою очередь позволяет сделать выводы, о том,
что существует ряд проблем в решении этих сложных задач.

Целью поставленной в настоящей статье является изучение имеющихся
ключевых вызовов в области исследования государственного аудита
управления кадровыми ресурсами в Казахстане и путей их преодоления. В
соответствии с целью в статье представлены проведенный анализ текущего
состояния системы аудита управления кадровыми ресурсами в Казахстане,
выявлены ключевые проблемы и недостатки в проведении аудиторских
проверок кадровой политики государственных органов и рассмотрена
возможность разработки рекомендаций по совершенствованию методов аудита
и совершенствованию системы ключевых показателей эффективности (KPI).

В постоянно меняющихся условиях, важно учитывать все факторы, влияющие
на развитие кадровых ресурсов страны с целью эффективного управления в
соответствии со стратегическими целями нашей страны.

{\bfseries Ключевые слова:} аудит, кадровые ресурсы, аудит эффективности
кадровых ресурсов

\begin{header}
STATE AUDIT OF HUMAN RESOURCE MANAGEMENT IN KAZAKHSTAN: KEY CHALLENGES AND WAYS TO OVERCOME THEM

A.R. Mukusheva,
M.A. Serikova
\end{header}

\begin{affil}
L.N. Gumilyov Eurasian National University, Astana, Kazakhstan,

e-mail: suleimenova.ar@bk.ru
\end{affil}

State audit of human resources management is an important aspect in the
development of the state' s human resources potential.
Human resources are a mechanism responsible for building a clear
architecture of the management structure of civil servants of the
Republic of Kazakhstan. Strengthening the role of human resources in
public administration will allow achieving the strategic goals set for
the state.

In a modern society where global changes are constantly taking place, it
is important to take into account the factors influencing the internal
environment, in particular in light of the development of the
country' s human resources potential, which in turn can
have a significant role in the state audit of human resources.

A high level of development of human resources contributes to the
sustainable development of the national economy of the country. An
important role should be given to the state audit of human resources,
because only a state audit will allow us to assess the efficiency of
using human resources. It should be noted that today there are no
mechanisms for developing a clear strategy for conducting an audit of
the efficiency of human resources, which in turn allows us to conclude
that there are a number of problems in solving these complex tasks.

The purpose of this article is to study the existing key challenges in
the field of research of state audit of human resources management in
Kazakhstan and ways to overcome them. In accordance with the purpose,
the article presents the analysis of the current state of the human
resources management audit system in Kazakhstan, identifies key problems
and shortcomings in conducting audits of the personnel policy of state
bodies and considers the possibility of developing recommendations for
improving audit methods and improving the system of key performance
indicators (KPI).

In a constantly changing environment, it is important to take into
account all factors that influence the development of the
country' s human resources in order to effectively manage
in accordance with the strategic goals of our country.

{\bfseries Keywords:} audit, human resources, human resource efficiency
audit

\begin{multicols}{2}
{\bfseries Кіріспе.} Қазіргі заманғы әлеуметтік-экономикалық жағдайлар
Қазақстан Республикасының мемлекеттік органдарында кадр ресурстарын
жоғары сапалы басқаруды талап етеді. Мемлекеттік бағдарламаларды іске
асырудың тиімділігін және еліміздің халықаралық аренадағы бәсекеге
қабілеттілігін анықтайтын негізгі фактор - кадрлық ресурстар.

Жаһандану мен бәсекелестіктің күшеюі жағдайында мемлекеттік
қызметшілерді басқару жүйесін жетілдіру қажеттілігі артып отыр, бұл осы
жұмыстың тиімділігін объективті бағалай алатын құралдарды әзірлеуді және
енгізуді талап етеді. Осындай құралдардың бірі -- мемлекеттік
органдардың кадрлық ресурстарын басқару саласындағы қызметіне бақылауды
қамтамасыз ететін мемлекеттік аудит {[}1-3{]}.

Қазақстандағы кадрлық ресурстарын басқарудың мемлекеттік аудитін
зерттеудің өзектілігі мемлекеттік қызметтің тиімділігін арттыру, кадрлық
процестерді оңтайландыру және кадрлықи капиталды пайдаланудағы
жауапкершілікті арттыру қажеттілігімен түсіндіріледі. Ұлт жоспары - 100
нақты қкадрлық және Мемлекеттік қызметті дамыту тұжырымдамасы сияқты
стратегиялық бастамаларды жүзеге асыру тұрғысында персоналды сапалы
басқару мәселелері бірінші орынға шығады.

Білікті кадрлардың тапшылығы, іріктеу және жоғарылату рәсімдерінің
жеткіліксіз ашықтығы, сондай-ақ мемлекеттік қызметшілерді ынталандыру
және қызметін бағалау жүйесінің жетілмегендігі сияқты заманауи
сын-қатерлер еліміздің стратегиялық мақсаттарына қол жеткізуге кері
әсерін тигізуі мүмкін. Персоналды басқару мәселелеріне жеткіліксіз көңіл
бөлу бюрократиялық жүктеменің артуына, еңбек өнімділігінің төмендеуіне
және мемлекеттік органдардың институционалдық әлеуетінің әлсіреуіне
әкеледі.

Осы тұрғыда кадрлық ресурстарды басқарудың мемлекеттік аудиті кадрлық
саясаттың тиімділігін жүйелі талдауды қамтамасыз етудің негізгі құралы
болып табылады. Оның рөлі бұзушылықтарды анықтау ғана емес, сонымен
қатар персоналды басқару тиімділігін арттыру бойынша деректерге
негізделген ұсыныстарды қалыптастыру болып табылады. Аудит тетіктерін
әзірлеу кадрлық шешімдердің стратегиялық басымдықтарға сәйкестігін
объективті бағалауға, кадрлық ресурстарын тиімсіз бөлумен байланысты
тәуекелдерді анықтауға және оларды барынша азайту тетіктерін әзірлеуге
мүмкіндік береді.

Сонымен қатар, заманауи технологиялар мен аудит әдістері, соның ішінде
HR процестерін цифрландыру, үлкен деректер мен жасанды интеллект
пайдалану, басқару шешімдерінің ашықтығы мен негізділігін арттыру үшін
жаңа мүмкіндіктер туғызады. Кадрлық аудитте аналитикалық құралдарды
пайдалану кадрлар қажеттілігін болжауды жақсартуға, жекелендірілген
оқыту бағдарламаларын әзірлеуге және мемлекеттік қызметтің жалпы
бәсекеге қабілеттілігін арттыруға көмектеседі.

Бастапқы мақсат - адам ресурстарын басқарудың мемлекеттік аудиті
саласындағы негізгі міндеттер мен шешімдерді зерделеу - нақты нәтижеге
әкелуі керек: Қазақстан Республикасының мемлекеттік органдары үшін
салалық стандартты кадрлық аудиттің тұжырымдамалық моделін әзірлеу. Бұл
модель Есеп комитеті мен мүдделі министрліктер үшін бөлшектелген
аудиттен адами капиталды пайдалану тиімділігін кешенді бағалауға көшуге
мүмкіндік беретін практикалық құрал болады.

Күтілетін ғылыми серпіліс үлкен деректерге негізделген болжамды талдауды
мемлекеттік кадрлық аудит жүйесіне біріктіру әдістемесін әзірлеу болып
табылады, ол бұзушылықтарды анықтаудан HR тәуекелдерін болжауға көшуге
мүмкіндік береді. Бұл мемлекеттік бағдарламалар бойынша оқытуға
инвестиция мен қызметкерлердің ауысуы көрсеткіштері арасындағы
корреляция сияқты кадрлық процестердегі жасырын заңдылықтарды анықтауға
негізделген қазақстандық тәжірибе үшін түбегейлі жаңа аудит функциясы.
Ұсынылған модельдің тиімділігі мемлекеттік органдардың стратегиялық
мақсаттарына сәйкестендірілген динамикалық KPI енгізу және екі жыл
ішінде пилоттық агенттіктердегі кадрлардың тұрақтамауын 5-7\%-ға
қысқарту сияқты нақты көрсеткіштер арқылы бағаланатын болады. Аудиттің
цифрландыруын өлшеу үшін мамандандырылған бағдарламалық платформалардың
арқасында HR тәуекелінің автоматтандырылған есептерінің үлесі және
деректерді жинау мен талдау уақытын 30\%-ға қысқарту сияқты көрсеткіштер
қолданылады.

Зерттеудің күтілетін нәтижесі Қазақстанның ерекшеліктеріне бейімделген
кадрлық ресурстарын басқарудың мемлекеттік аудитін жүргізудің кешенді
әдістемесін әзірлеу болып табылады. Осы тұжырымдарды іс жүзінде енгізу
мемлекеттік қызметшілер үшін жақсартылған KPI жүйесін енгізумен қатар
кадрлық саясаттың тиімділігін бағалаудың бірыңғай стандарттарын жасауға
мүмкіндік береді. Жасанды интеллект элементтері бар мамандандырылған
платформаларды енгізу арқылы аудит процестерін цифрландыру болжамдық
аналитикаға және кадрлықи капиталдың объективті мониторингіне көшуді
жеңілдетеді. Ұлттық оқу орталығын құру арқылы құрылған аудиторлардың
біліктілігін үздіксіз арттыру жүйесі кадрлық аудиттің сапасын арттыруға
мүмкіндік береді. Кері байланыс тетіктерін енгізу және аудиттің
ашықтығын арттыру HR шешімдері үшін басқарушылық жауапкершілікті
күшейтеді.

Осылайша, кадрлық ресурстарын басқару тиімділігін тексеру мәселелерін
зерттеу қазіргі заманғы талаптарға жауап беріп қана қоймайды, сонымен
қатар кәсіби, құзыретті және нәтижеге бағытталған мемлекеттік аппаратты
қалыптастырудың негізін қалайды.

{\bfseries Зерттеудің міндеттері:}

1. Қазақстандағы кадрлық ресурстарын басқару аудит жүйесінің ағымдағы
жағдайына талдау жүргізу;

2. Мемлекеттік органдардың кадр саясатына аудит жүргізудегі негізгі
проблемалар мен кемшіліктерді анықтау;

3. Аудит әдістерін жетілдіру және қызметтің негізгі көрсеткіштері
жүйесін (KPI) жетілдіру бойынша ұсыныстар әзірлеу.

Зерттеу объектісі Қазақстанның мемлекеттік органдарындағы кадрлық
ресурстарын мемлекеттік басқару жүйесі.

Зерттеу пәні мемлекеттік органдарда кадрлық ресурстарын басқару
тиімділігіне аудит жүргізудің әдіснамалық аспектілері болып табылады.

Мақалада қолданылған зерттеу әдістеріне мыналар жатады:

- қолданыстағы аудит тәжірибесін жүйелі және салыстырмалы талдау;

- ұсыныстарды қалыптастырудың проблемалық зерттеу әдісі;

- кадрлық ресурстарын пайдалануды статистикалық талдау.

Осы мақаланың мазмұны кадрлық ресурстарының мемлекеттік аудитінің
әдіснамалық тәсілдерін әзірлеуге бағытталған. Зерттеу нәтижелерін
мемлекеттік қызметтің ашықтығы мен тиімділігін арттыру, сондай-ақ
мемлекеттік органдардағы персоналды басқаруды оңтайландыру үшін
пайдалануға болады. Ұсынылған ұсыныстарды іске асыру персоналды басқару
үдерісін жақсартуға және мемлекеттік бағдарламаларды іске асыру
тиімділігін арттыруға мүмкіндік береді.

{\bfseries Материалдар мен әдістер.} Материалдар мен тәсілдер. Қазақстан
Республикасының Президенті Қасым-Жомарт Кемелұлы Тоқаевтың 2024 жылғы 2
қыркүйектегі Қазақстан халқына жолдауында: «Еліміздің кадрлық әлеуетін
дәйекті түрде арттыру маңызды. Экономиканы білікті кадрлармен қамтамасыз
ету - ең өзекті мәселе. Адал, қажырлы еңбек арқылы жетістікке жеткен
кадрлықдар әрқашан құрмет пен құрметке бөленуі керек. Бұл біздің
«Жауапты азамат - адал еңбек - лайықты табыс» қағидатымызға толық сәйкес
келеді.

Ешқандай жаман жұмыс жоқ; Ең бастысы, әрбір азамат өз міндетін адал
атқарса, жауапкершілікті терең сезінеді. Сонда ғана еліміз ілгерілеу
жолында биіктерге жетеді. Біздің қоғамда адал да жауапты еңбек міндетті
түрде бағаланады деген түсінік қалыптасуы керек. Бұл мақсатқа
азаматтарға құрметті атақтар беру де қызмет етеді» {[}4{]}.

Қазақстан Республикасы Президентінің жолдауында айтылған кадр саясаты
саласындағы басымдықтар кадрлық ресурстарын басқару тиімділігін
арттырудың стратегиялық қажеттілігін көрсетеді. Алға қойылған
мақсаттарға жету персоналмен жұмыстың ашық, есеп беретін және тиімді
жүйесін қалыптастырмайынша мүмкін емес, бұл өз кезегінде заманауи
бақылау және аудит тетіктерін енгізуді талап етеді.

Кадрлық ресурстарын басқарудың мемлекеттік аудиті - бұл кадрлық
үдерістердегі бұзушылықтар мен кемшіліктерді анықтау құралы ғана емес,
кадр саясатының елдің ұзақ мерзімді стратегиялық бағдарларына
сәйкестігін жүйелі түрде бағалау тетігі. Ол мемлекеттік органдардың
меритократия қағидаттарын қаншалықты тиімді жүзеге асырып жатқанын
бағалауға, қызметкерлердің кәсіби өсуі мен мотивациясына жағдай туғызып,
кадрлықи капиталды ұтымды пайдалануды қамтамасыз етуге мүмкіндік береді.

Мемлекеттік аудиттің негізгі аспектілерінің бірі кадрларды іріктеу,
дамыту және сақтау жүйесінің тиімділігін бақылау болып табылады. Цифрлық
трансформация және мемлекеттік қызметшілердің құзыреттеріне қойылатын
талаптардың өзгеруі жағдайында персоналды басқарудағы дәстүрлі тәсілдер
енді кадрлық ресурстарын басқару жүйесінің жаңа міндеттерге сай болуын
толық қамтамасыз ете алмайды. Бұл аналитикалық құралдарды, үлкен
деректер технологияларын және персонал қажеттілігін болжау әдістерін
енгізуді өзекті етеді.

Осылайша, кадрлық ресурстарын басқарудың мемлекеттік аудиті тек
нормативтік талаптардан ауытқуларды анықтауға ғана емес, сонымен қатар
кадр саясатының тиімділігін арттыру бойынша стратегиялық ұсыныстарды
қалыптастыруға бағытталуы тиіс. Қазақстандағы кадрлық ресурстарын
басқару жүйесінің алдында тұрған негізгі қиындықтарды, сондай-ақ оларды
еңсерудің ықтимал жолдарын егжей-тегжейлі қарастырайық.

Кадрлық ресурстарын басқарудың мемлекеттік аудиті мемлекеттік аппараттың
тиімді жұмыс істеуінің құрамдас бөлігі болып табылады. Соңғы жылдары
зерттеушілер Қазақстандағы кадрлық ресурстарын басқарудағы проблемаларды
анықтауға және оларды оңтайландырудың шешімдерін ұсынуға назар аударды.
Ғылыми әдебиеттерде авторлар мемлекеттік аудиттің әртүрлі аспектілерін,
кадрлық ресурстарын басқаруды және цифрландырудың елдегі кадрлық
саясатқа әсерін көрсететін негізгі мәселелерді қарастырады.

Bratton, J. \& Gold, J. адам ресурстарын мемлекеттік басқару
функцияларын орындау үшін қажетті қызметкерлердің кәсіби білімдерінің,
дағдыларының, құзыреттерінің және әлеуетінің жиынтығы ретінде анықтайды
{[}5{]}.

Lawler, E. E. \& Boudreau, J. W. адам ресурстарын қызметкерлерді ғана
емес, сонымен бірге олардың өзара әрекеттесу, оқыту және өзгермелі
жағдайларға бейімделу жүйелерін қамтитын стратегиялық актив ретінде
қарастырады {[}6{]}.

Қазақстандағы кадрлық ресурстарын басқарудың мемлекеттік аудиті
мемлекеттік органдарда кадрлық әлеуетін тиімді пайдалануды қамтамасыз
етудің маңызды құралы болып табылады.

Саунин А.Н. кадрлық ресурстарын басқару контекстінде мемлекеттік
аудиттің теориялық негіздері мен практикалық аспектілеріне назар
аударады. Автор еңбек нарығындағы және мемлекеттік құрылымдардағы
өзгермелі жағдайларға байланысты неғұрлым икемді және бейімделген
бақылау әдістерін енгізу қажеттілігі туралы дәлелдер келтіре отырып,
мемлекеттік органдарда аудит жүргізудің қолданыстағы әдістерін талдайды.
Саунин Қазақстанда аудитті мемлекеттік сектордағы кадрлық ресурстарын
басқарудың нақты тәжірибесімен байланыстыратын кешенді тәсілдер жоқ
екенін атап өтті.

Жұмыс мемлекеттік аудиторлардың біліктілігін арттырудың және кадрлық
үдерістердің мониторингін жақсартудың маңыздылығын көрсетеді {[}7{]}.

Климанов В.В., Казакова С.М., Яговкин В.А. зерттеулерінде қаржылық аудит
призмасы арқылы кадрлық ресурстарын бақылауға қатысты мәселелер
талқыланады. Авторлар кадрлық шығындарды бюджеттеу және қызметкерлерді
оқыту мен біліктілігін арттыру үшін мемлекеттік қаражатты пайдалану
тиімділігі сияқты қаржылық аспектілерді есепке алмай кадр саясатын
оңтайландыру мүмкін емес деп санайды. Авторлар персонал шығындарын
бағалауды да, персоналмен жұмыс тиімділігін де қамтуы тиіс аудитке
кешенді тәсілдің маңыздылығын атап көрсетеді {[}8{]}.

Алибекова Б.А., Алдынғарова Д.Т. өз мақалаларында кадрлық ресурстарын
басқару контекстінде қолданыстағы аудит әдістеріне талдау жасайды.
Авторлар екі негізгі аспектіні атап көрсетеді: біріншісі -- халықаралық
стандарттарды Қазақстанның ерекшеліктеріне бейімдеу қажеттілігі,
екіншісі - заңнамалық өзгерістер мен елдегі әлеуметтік-экономикалық
жағдай сияқты түрлі факторлардың аудитті жүргізуге әсері. Олар персонал
саясаты аудитінің сапасын арттыру үшін аудиторлық кадрларды даярлауды
жақсарту және жаңа ақпараттық технологияларды енгізу қажеттігін атап
көрсетеді {[}9{]}.

Боқаев Б.Н., Төребекова З.Т., Жаров Е.Қ., Бактиярова Б.Н. өз
зерттеулерінде Қазақстанның мемлекеттік қызметіндегі кадрлық ресурстарын
тиімді басқару үшін кадрлық стратегия мен саясатты әзірлеу қажеттігін
атап көрсетеді. Авторлар кадрлықи капиталды басқару тетіктері
лауазымдарға тағайындау және жоғарылату рәсімдерін қамтиды, бұл жүйелі
көзқарас пен үздіксіз жетілдіруді талап етеді {[}10{]}.

«Қазақстан Республикасындағы мемлекеттік аудитті жүзеге асыру
тұжырымдамасына» сәйкес мемлекеттік аудит тек қаржылық аспектілерді ғана
емес, сонымен қатар олардың қызметінің барлық салаларын, оның ішінде
кадрлық ресурстарын басқаруды қамтитын аудиттелетін субъектілер
қызметінің тиімділігін кешенді және тәуелсіз бағалау ретінде
қарастырылады. Құжат тиімді ұсынудың маңыздылығын көрсетеді анықталған
кемшіліктерге негізделген ұсыныстар мен тәуекелдерді басқару бойынша
ұсыныстар {[}11{]}.

«Мемлекеттік органдардағы кадр қызметінің мәртебесін жақсарту»
зерттеуінде Қазақстанның мемлекеттік қызметіндегі қайта құрулар, оның
ішінде кадр қызметі құрылымын автоматтандыру және орталықтандыру
қарастырылған. Авторлар мемлекеттік органдардың алдында тұрған нақты
мәселелерді шешу үшін сенімді және тиімді мемлекеттік аппаратты
қалыптастыру қажеттігін атап көрсетеді {[}12{]}.

Шаймерденова Т.А., Жетпісбаева М.Қ. «Қазақстандағы мемлекеттік
қызметшілердің мотивациялық аспектілері экономиканы цифрландыру
жағдайында» атты мақаласында персоналды ынталандыру призмасы арқылы
мемлекеттік басқарудың сапасын арттыру мәселелерін қарастырады. Авторлар
цифрландыру жағдайында кадрлық әлеуетін дамытуға және кадр саясатын
заманауи шындыққа бейімдеуге ерекше назар аудару керектігін атап өтті
{[}13{]}.

Осылайша, бар әдебиеттер Қазақстанның мемлекеттік қызметіндегі кадрлық
ресурстарын басқаруға жүйелі көзқарас қажеттігін көрсетеді. Негізгі
міндеттерге HR тиімді стратегияларын әзірлеу және енгізу, аудиторларды
сертификаттау жүйесін жетілдіру және кадр саясатын цифрландыру
шарттарына бейімдеу қажеттілігі жатады. Бұл қиындықтарды еңсеру заманауи
технологияларды енгізу, қызметкерлердің біліктілігін арттыру және аудит
нәтижелері бойынша тиімді ұсыныстар әзірлеу арқылы мүмкін болады.

{\bfseries Нәтижелер мен талқылау.} Қазақстандағы кадрлық ресурстарын
басқару аудиті жүйесін талдау мемлекеттік басқарудың тиімділігін
арттыруға бағытталған терең құрылымдық өзгерістердің қажеттілігін
көрсетеді. Мемлекеттік қызметті жаңғырту және меритократия қағидаттарын
енгізу жағдайында персоналды тексерудегі дәстүрлі тәсілдер енді кадрлық
ресурстарын басқару сапасының толық көрінісін бермейді. Қолданыстағы
аудит тетіктері нормативтік талаптардың сақталуына көбірек бағытталған,
ал аудиттің стратегиялық мақсаты кадрлық саясаттың тиімділігін кешенді
бағалау, жүйелі проблемаларды анықтау және оларды жою шараларын ұсыну
болып табылады.

Ағымдағы аудиттің негізгі шектеулерінің бірі мемлекеттік қызметшілердің
қызметін бағалауда объективтіліктің болмауы болып табылады. Нақты
өнімділік көрсеткіштерінің болмауы, сондай-ақ жұмыс нәтижелері мен
мотивация жүйесі арасындағы әлсіз байланыс кадрлық процестердің
бюрократизациялану қаупін тудырады. Осының салдарынан нәтижелерді
бағалау көбінесе формальды сипатқа ие, ал кадрлық шешімдер стратегиялық
мақсаттарға жетудегі қызметкерлердің нақты үлесін есепке алмай
қабылданады. Сонымен қатар, мемлекеттік басқарудың әртүрлі деңгейлерінде
кадр саясатын үйлестірудің жоқтығы маңызды мәселе болып қалуда, бұл
кадрлық ресурстарын бөлудегі теңгерімсіздікке және жалпы мемлекеттік
қызмет жүйесінің тиімділігінің төмендеуіне алып келеді.

Заманауи технологиялар HR аудитін жақсарту үшін жаңа мүмкіндіктер ашады,
бірақ олардың әлеуеті әлі де толық пайдаланылмайды. Негізгі басқару
процестерін цифрландыру контекстінде HR аудиті персонал қажеттіліктерін
болжауға, персоналды басқарудағы жасырын тенденцияларды анықтауға және
шешім қабылдау процесін оңтайландыруға мүмкіндік беретін аналитикалық
құралдар мен үлкен деректер технологияларын біріктіруі керек. Жасанды
интеллект пен машиналық оқыту жүйелерін енгізу құзыреттерді дәлірек
бағалауды қамтамасыз ете алады, персонал рәсімдерінің ашықтығын
арттырады және қызметкерлердің жұмысын бағалаудағы субъективтілікті
азайтады.

Сонымен қатар, жоғарылатудың ашықтығын қамтамасыз ететін, тәлімгерлік
рөлін күшейтетін және мемлекеттік қызметке сенім деңгейін арттыратын
мемлекеттік қызметшілердің мансаптық өсуі мен біліктілігін арттыру
мониторингінің бірыңғай жүйесін құру қажет. Маңызды аспект персоналды
басқару тиімділігі туралы объективті деректер негізінде жеке
жауапкершілікті енгізуді талап ететін кадрлық шешімдер үшін басшылардың
жауапкершілігін күшейту болып табылады. Бұл тұрғыда мемлекеттік аудит
тек бақылау функциясын орындап қана қоймай, сонымен қатар кадр саясатын
жетілдіруге, тиімді тәжірибені анықтауға және мемлекеттік аппарат
жұмысының өнімділігін төмендететін факторларды жоюға бағытталған
стратегиялық құралға айналуы тиіс.

Кадрлық ресурстарын басқару саласындағы мемлекеттік аудитті күшейту
формальды бақылаудан кадр саясатының тиімділігін кешенді талдауға көшуді
талап етеді.

Цифрлық технологияларды интеграциялау, қызметтің тиімділігін бағалаудың
ашық тетіктерін дамыту және персоналды басқарудың стратегиялық
бағдарланған жүйесін қалыптастыру Қазақстанның әлеуметтік-экономикалық
дамуының басым міндеттерін тиімді іске асыруға қабілетті кәсіби,
құзыретті және уәжделген мемлекеттік аппаратты құруға мүмкіндік береді.
\end{multicols}

\tcap{1 - кесте.2020-2024 жж. Қазақстандағы орташа айлық атаулы жалақы}
\begin{longtblr}[
  label = none,
  entry = none,
]{
  cells = {c},
  cell{6}{1} = {c=6}{},
  cells = {font = \small},
  hlines,
  vlines,
}
~\textbf{Көрсеткіштер}                                        & \textbf{2020 ж.} & \textbf{2021 ж.} & \textbf{2022 ж.} & \textbf{2023 ж.} & \textbf{2024 ж.} \\
Орташа айлық номиналды жалақы                                 & ~                & ~                & ~                & ~                &                  \\
- теңге                                                       & 213 003          & 250 311          & 309 697          & 364 295          & ~403 251         \\
- АҚШ доллары                                                 & 516              & 588              & 673              & 798              & 854              \\
Ең төменгі жалақы                                             & 42 500           & 42 500           & 60 000           & 70~000           & 85000            \\
Ескерту: дереккөздер негізінде авторлармен құрастырылған [13] &                  &                  &                  &                  &                  
\end{longtblr}

\begin{multicols}{2}
Бүгінгі таңда Қазақстан Республикасының мемлекеттік органдарында 83
мыңнан астам мемлекеттік қызметші жұмыс істейді. Мемлекеттік қызмет
істері және сыбайлас жемқорлыққа қарсы іс-қимыл агенттігінің мәліметтері
бойынша, тек 2023 жылдың өзінде біліктілікті арттырудың мемлекеттік
бағдарламалары бойынша оқуды аяқтаған қызметкерлер саны мемлекеттік
қызметшілердің жалпы санының шамамен 22\%-ын құрады, бұл кадрларды
оқытуға жүйелі түрде қарау қажеттілігін көрсетеді. Дегенмен, тексерулер
мемлекеттік бағдарламаларды жүзеге асыру барысында анықталған
проблемалардың 30\%-дан астамы кадрлардың біліктілігінің жеткіліксіздігі
мен ресурстарды тиімсіз басқаруға байланысты екенін көрсетті {[}14{]}.
Осы орайда материалдық мотивациясының маңызы зор.2020-2024 жж.
Қазақстандағы орташа айлық атаулы жалақы бойынша ақпарат 1 кестеде
көрсетілген.

Ұсынылған деректер бес жыл ішінде Қазақстандағы үй шаруашылықтары
табысының тұрақты оң үрдісін көрсетеді. Номиналды орташа айлық жалақы
2020 жылғы 213 000 теңгеден 2024 жылы 403 000 теңгеге дейін, ал
доллармен есептегенде 516-дан 854 АҚШ долларына дейін өсіп, ұлттық
валютада да айтарлықтай өсуді көрсетті. Сонымен қатар, ең төменгі
жалақының тұрақты өсіп, осы кезеңде екі есеге 42 500 теңгеден 85 000
теңгеге дейін ұлғаюы мемлекеттің азаматтардың әл-ауқатын арттыруға
бағытталған мақсатты әлеуметтік саясатын көрсетеді.
\end{multicols}

\tcap{2 - кесте.2021-2024жж. жұмыспен қамтылған халық саны, мың адам}
\begin{longtblr}[
  label = none,
  entry = none,
]{
  row{1} = {c},
  row{24} = {c},
  cell{2}{2} = {c},
  cell{2}{3} = {c},
  cell{2}{4} = {c},
  cell{2}{5} = {c},
  cell{3}{2} = {c},
  cell{3}{3} = {c},
  cell{3}{4} = {c},
  cell{3}{5} = {c},
  cell{4}{2} = {c},
  cell{4}{3} = {c},
  cell{4}{4} = {c},
  cell{4}{5} = {c},
  cell{5}{2} = {c},
  cell{5}{3} = {c},
  cell{5}{4} = {c},
  cell{5}{5} = {c},
  cell{6}{2} = {c},
  cell{6}{3} = {c},
  cell{6}{4} = {c},
  cell{6}{5} = {c},
  cell{7}{2} = {c},
  cell{7}{3} = {c},
  cell{7}{4} = {c},
  cell{7}{5} = {c},
  cell{8}{2} = {c},
  cell{8}{3} = {c},
  cell{8}{4} = {c},
  cell{8}{5} = {c},
  cell{9}{2} = {c},
  cell{9}{3} = {c},
  cell{9}{4} = {c},
  cell{9}{5} = {c},
  cell{10}{2} = {c},
  cell{10}{3} = {c},
  cell{10}{4} = {c},
  cell{10}{5} = {c},
  cell{11}{2} = {c},
  cell{11}{3} = {c},
  cell{11}{4} = {c},
  cell{11}{5} = {c},
  cell{12}{2} = {c},
  cell{12}{3} = {c},
  cell{12}{4} = {c},
  cell{12}{5} = {c},
  cell{13}{2} = {c},
  cell{13}{3} = {c},
  cell{13}{4} = {c},
  cell{13}{5} = {c},
  cell{14}{2} = {c},
  cell{14}{3} = {c},
  cell{14}{4} = {c},
  cell{14}{5} = {c},
  cell{15}{2} = {c},
  cell{15}{3} = {c},
  cell{15}{4} = {c},
  cell{15}{5} = {c},
  cell{16}{2} = {c},
  cell{16}{3} = {c},
  cell{16}{4} = {c},
  cell{16}{5} = {c},
  cell{17}{2} = {c},
  cell{17}{3} = {c},
  cell{17}{4} = {c},
  cell{17}{5} = {c},
  cell{18}{2} = {c},
  cell{18}{3} = {c},
  cell{18}{4} = {c},
  cell{18}{5} = {c},
  cell{19}{2} = {c},
  cell{19}{3} = {c},
  cell{19}{4} = {c},
  cell{19}{5} = {c},
  cell{20}{2} = {c},
  cell{20}{3} = {c},
  cell{20}{4} = {c},
  cell{20}{5} = {c},
  cell{21}{2} = {c},
  cell{21}{3} = {c},
  cell{21}{4} = {c},
  cell{21}{5} = {c},
  cell{22}{2} = {c},
  cell{22}{3} = {c},
  cell{22}{4} = {c},
  cell{22}{5} = {c},
  cell{23}{2} = {c},
  cell{23}{3} = {c},
  cell{23}{4} = {c},
  cell{23}{5} = {c},
  cell{24}{1} = {c=5}{},
  cells = {font = \small},
  hlines,
  vlines,
}
\textbf{Аумақтар}                                             & \textbf{2021 ж.~} & \textbf{2022 ж.~} & \textbf{2023 ж.~} & \textbf{2024 ж.~} \\
Қазақстан Республикасы                                        & 8 732,0           & 8 807,1           & 8 971,5           & 9 081,9           \\
Абай                                                          & -                 & -                 & 287,1             & 292,5             \\
Ақмола                                                        & 398,0             & 397,0             & 421,4             & 407,1             \\
Ақтөбе                                                        & 416,4             & 419,8             & 424,7             & 434,9             \\
Алматы                                                        & 974,0             & 973,0             & 697,7             & 704,8             \\
Атырау                                                        & 314,5             & 317,7             & 326,7             & 335,1             \\
З-Казахстанская                                               & 321,0             & 322,3             & 330,9             & 333,3             \\
Жамбыл                                                        & 503,8             & 502,7             & 539,5             & 543,7             \\
Жетісу                                                        & -                 & -                 & 319,7             & 309,3             \\
Қарағанды                                                     & 641,8             & 643,4             & 534,8             & 535,8             \\
Қостанай                                                      & 466,3             & 475,2             & 453,8             & 449,5             \\
Қызылорда                                                     & 329,4             & 330,1             & 330,1             & 331,5             \\
Маңғыстау                                                     & 308,4             & 331,7             & 332,7             & 336,7             \\
Оңтүстік Қазақстан                                            & -                 & -                 & -                 & -                 \\
Павлодар                                                      & 387,1             & 383,7             & 384,2             & 385,2             \\
С-Казахстанская                                               & 289,3             & 287,3             & 279,1             & 274,5             \\
Түркістан                                                     & 779,4             & 777,6             & 792,2             & 800,6             \\
Ұлытау                                                        & -                 & -                 & 100,9             & 100,9             \\
Шығыс Қазақстан                                               & 669,5             & 668,3             & 366,5             & 368,8             \\
Астана қаласы                                                 & 563,4             & 580,3             & 625,5             & 658,7             \\
Алматы қаласы                                                 & 959,3             & 982,8             & 998,0             & 1 045,5           \\
Шымкент қаласы                                                & 410,3             & 414,3             & 426,1             & 433,5             \\
Ескерту: дереккөздер негізінде авторлармен құрастырылған [15] &                   &                   &                   &                   
\end{longtblr}

\begin{multicols}{2}
Қазақстандағы ең маңызды көрсеткіш жұмыспен~қамтылу -- елдегі еңбекке
қабілетті тұрғындардың~жұмыспен~қамтамасыз етілу көрсеткіші.
2021-2024жж. жұмыспен қамтылған халық саны 2 кестеде көрсетілген.

Мемлекеттік кадрлық аудит жүйесінің негізгі проблемаларының бірі кадр
қызметінің қызметін бағалаудың жеткіліксіз жүйелілігі мен кешенділігі
болып табылады. Қазіргі уақытта жүйе персоналды басқарудағы
проблемаларды анықтау және жою процестерін толық қамти алмайды. Атап
айтқанда, 2023 жылғы есеп бойынша, HR аудитінің 40\%-дан астамы ережелер
мен процедуралардың сақталмауын анықтаған.

Мәселе сонымен қатар кадрлық саясатты тексеретін аудиторлардың
біліктілігінің жеткіліксіздігінде. Соңғы жылдары бірнеше аудиторларды
оқыту және дамыту бағдарламалары іске қосылғанымен, 2023 жылы кадрлық
ресурстары бойынша аудиторлық қызметкерлердің тек 16\%-ы ғана жаңа аудит
стандарттары бойынша арнайы оқытудан өтті.

Қазақстандағы кадрлық ресурстарын басқару тиімділігінің мемлекеттік
аудитінің мәселелерін талдау осы процестің тиімділігіне айтарлықтай әсер
ететін бірқатар жүйелік қиындықтарды анықтайды. Аудиторлардың
біліктілігінің жоқтығы өзекті мәселелердің бірі болып табылады, бұл
мемлекеттік органдардағы кадрлық үдерістерге терең және объективті баға
беруді қиындатады. Мемлекеттік қызмет жүйесін жедел жаңғырту, цифрлық
басқару қағидаттарын енгізу және кадр саясатына стратегиялық
бағдарланған тәсілге көшу жағдайында аудиттің дәстүрлі әдістері қазіргі
талаптарға сай емес. Дегенмен, басқару процестерінің күрделене түсуіне
қарамастан, HR аудиті кадрлық ресурстарын басқару жүйесіндегі негізгі
дисфункцияларды анықтай алмай, негізінен формальды болып қала береді.

Кадрларды басқару, кадрлық ресурстарын стратегиялық жоспарлау және
мінез-құлық экономикасы саласында аудиторлардың жеткіліксіз дайындығы
аудиттің тек нормативтік талаптарды сақтаумен шектелуіне, кадрлық
саясаттың нақты тиімділігін және оның мемлекеттің стратегиялық
мақсаттарына қол жеткізуге әсерін бағалауға мүмкіндік бермеуіне әкеледі.
Мемлекеттік басқарудағы заманауи тенденциялар құзыреттіліктерді, еңбек
өнімділігін және қызметкерлерді тартуды тереңірек талдауды талап етеді,
бірақ аудиторлар арасында қажетті білім мен құралдардың болмауы
аудиторлық есептердің аналитикалық құндылығының төмен болуына әкеледі.
Сонымен қатар, персоналдың тиімділігін бағалаудың бірыңғай әдістемелік
стандарттарының болмауы тексерулердің субъективтілігіне, олардың
мемлекеттік қызмет жүйесін жетілдірудегі практикалық маңыздылығын
төмендетуге әкеледі.

Көп жағдайда кадрлық процестерді тексеруге қатысатын мамандардың
классикалық қаржылық, экономикалық немесе заңгерлік білімі бар, бұл
оларға мемлекеттік органдардың кадрлық құрылымы шеңберіндегі күрделі
жүйелік өзара іс-қимылдарды талдау үшін жеткілікті құралдармен
қамтамасыз етілмейді. Мемлекеттік қызметті цифрландыру және үлкен
деректер технологияларын енгізу жағдайында HR аудиті болжамды талдауды,
кадрлық сценарийлерді модельдеуді және кадрлық капиталының
макроэкономикалық көрсеткіштерге әсерін бағалауды қамтитын кешенді
аналитикалық тәсілдерге негізделуі керек. Алайда іс жүзінде мұндай
тәсілдер әлі де кең тараған жоқ, бұл мемлекеттік кадр саясатын жетілдіру
құралы ретіндегі аудиттің әлеуетін айтарлықтай шектейді.

Сонымен қатар, аудитордың біліктілігінің жоқтығы кадрлық ресурстарының
тиімділігін бағалау процесін автоматтандыру және стандарттау мүмкін
болатын кешенді цифрлық шешімдердің жоқтығынан шиеленісіп отыр. Үздік
әлемдік тәжірибеде мемлекеттік қызметшілер қызметінің негізгі
көрсеткіштеріне нақты уақыт режимінде мониторинг жүргізуге,
құзыреттердегі олқылықтарды анықтауға және ұзақ мерзімді стратегиялық
мақсаттар негізінде персонал қажеттілігін болжауға мүмкіндік беретін
персонал деректерін талдаудың біріктірілген жүйелері қолданылады.
Дегенмен, Қазақстанда кадрлық аудит негізінен құжаттарды тексерудің
дәстүрлі әдістеріне және сараптамалық қорытындыларға негізделуін
жалғастыруда, бұл процесті көп еңбекті қажет ететін, субъективті және
тиімсіз етеді.

Бұл мәселені шешу үш негізгі бағытты қамтитын кешенді тәсілді қажет
етеді. Біріншіден, кадрлық ресурстарын басқару, мінез-құлық экономикасы,
цифрлық аналитика және HR стратегиялық менеджменті бойынша
мамандандырылған курстармен толықтыра отырып, аудиторларды даярлау және
олардың біліктілігін арттыру жүйесін реформалау қажет. Бұл аудиторларға
HR процестерін тереңірек түсінуге ғана емес, сонымен қатар персонал
қызметін бағалаудың заманауи әдістерін, соның ішінде KPI талдауын,
құзыреттілік үлгілерін және еңбек өнімділігін индекстеуді қолдануға
мүмкіндік береді. Екіншіден, халықаралық стандарттарға негізделген және
кадрлық ресурстарын басқару тиімділігін бағалаудың кешенді сандық және
сапалық көрсеткіштерін қамтитын персонал аудитінің бірыңғай әдістемесін
әзірлеу қажет. Үшіншіден, бағалаудың субъективтілігін барынша азайтуға,
деректерді жинау мен өңдеуді автоматтандыруға, персонал қажеттілігін
болжау дәлдігін арттыруға мүмкіндік беретін аудит процестеріне цифрлық
технологияларды біріктіру қажет.

Мемлекеттік кадрлық басқару аудиторларының біліктілігінің
жеткіліксіздігі мәселесі мемлекеттік саясат деңгейінде жүйелі шешімдерді
талап ететін күрделі мәселе болып табылады. Заманауи аналитикалық
әдістерге, цифрлық технологияларға және мамандарды тереңдетіп оқытуға
негізделген персонал аудитінің жаңа моделін қалыптастыру аудиторлық
тексерулердің сапасын айтарлықтай арттыруға, мемлекеттік қызметтің
кадрлық әлеуетін нығайтуға және Қазақстанда кадрлық ресурстарын
басқарудың неғұрлым тиімді жүйесін құруға мүмкіндік береді №

Сондай-ақ, Қазақстандағы кадрлық ресурстарының аудиті жүйесі жеткілікті
жүйелі және кешенді емес екені байқалады. Персоналды басқарудың жеке
аспектілеріне бағытталған аудиттің фрагменттік сипаты кадрлықи капиталды
пайдалану тиімділігінің толық көрінісін бермейді. Бұл мемлекеттік
бағдарламалардың стратегиялық мақсаттарын іске асыруға кедергі
келтіретін проблемалық аймақтарды жедел анықтау мүмкіндігін шектейді.

Сонымен қатар, аудит процестерін цифрландыру төмен деңгейде қалып отыр.
Автоматтандырылған жүйелерді жеткіліксіз пайдалану мәліметтерді жинау
мен талдауды қиындатады, бұл аудиттің жылдамдығы мен сапасын
төмендетеді. Цифрлық трансформацияны жеделдету және жаңа технологияларды
дамыту жағдайында HR-менеджмент неғұрлым заманауи мониторинг және
бақылау құралдарын енгізуді талап етеді, бұл аудитке де қатысты.

Сондай-ақ кадрлық ресурстарын басқару аудиті саласындағы халықаралық
стандарттарды қолданудың шектеулілігін атап өтпеу мүмкін емес.
Халықаралық ұйымдардың ұсыныстары болғанымен, олардың Қазақстанда
қолданылуы жеткіліксіз. Бұл отандық аудит тәжірибесінің жаһандық
трендтерден артта қалуына әкеліп соқтырады, бұл жалпы тиімділікті
төмендетеді және процестің ашықтығы мен есептілігін азайтады.

Қазақстанда кадрлық ресурстарын басқарудың мемлекеттік аудитін жетілдіру
кешенді тәсілді, оның ішінде алдыңғы қатарлы әдістемелік, технологиялық
және басқарушылық шешімдерді енгізуді талап етеді. Мемлекеттік қызметті
трансформациялау және цифрлық менеджмент қағидаттарына көшу жағдайында
дәстүрлі аудит құралдары ендігі уақыт талабына толық жауап бермейді, бұл
оларды түбегейлі жаңарту қажеттілігін туғызады.

Модернизацияның негізгі бағыттарының бірі кадрлық үдерістерді бақылайтын
аудиторлардың құзыреттілік деңгейін арттыру болып табылады. Заманауи
шындық қаржылық бақылау және нормативтік-құқықтық база саласындағы
білімді ғана емес, сонымен қатар кадрлық ресурстарын стратегиялық
басқару, цифрлық технологиялар, ұйымдық даму және мінез-құлық
экономикасы бойынша терең тәжірибені қажет етеді. Осыған байланысты жаңа
міндеттерге бейімделген біліктілікті арттырудың мамандандырылған
бағдарламаларын енгізу қажет. HR талдауында, корпоративтік басқаруда
және өзгерістерді басқаруда практикалық тәжірибесі бар сарапшыларды
тарту аудиторларға өз құзыреттерін кеңейтуге мүмкіндік береді, бұл HR
саясатының тиімділігін неғұрлым егжей-тегжейлі және объективті бағалауды
қамтамасыз етеді. Кадрларды басқару саласындағы аудиторларды
сертификаттау бағдарламаларын әзірлеу және енгізу халықаралық
стандарттарға сәйкес келетін HR аудитінің кәсіби институтын
қалыптастыруға негіз болады.

HR аудитін жетілдірудің тағы бір маңызды аспектісі цифрлық
технологиялардың интеграциясын кеңейту болып табылады. Мемлекеттік
басқаруды белсенді цифрландыру жағдайында аудиторлық рәсімдерді
автоматтандыру олардың тиімділігін арттырудың қажетті шартына айналуда.
Жасанды интеллект жүйелерін, үлкен деректерді талдау технологияларын
және машиналық оқытуды енгізу нақты уақытта HR процестерін кешенді
бақылауға мүмкіндік береді. Бұл, өз кезегінде, жасырын тенденцияларды
анықтауға, кадрлардың тұрақсыздығы деңгейін бағалауға, персонал
қажеттіліктерін болжауға және кадрлық ресурстарын басқарудағы
проблемалық аймақтарды автоматты түрде диагностикалауға мүмкіндік
береді. Болжалды аналитика мен сандық бақылау тақталарын пайдалану
субъективті факторлардың әсерін барынша азайта отырып, аудит есептерінің
дәлдігін арттырады.

Мемлекеттік қызмет қызметінің негізгі көрсеткіштерінің (KPI)
жетілдірілген жүйесін әзірлеуге және енгізуге ерекше назар аудару қажет.
Қазіргі уақытта мемлекеттік қызметшілердің қызметін бағалау жүйесі
көбінесе формальды сипатқа ие және әрбір маманның мемлекеттің
стратегиялық мақсаттарына қол жеткізудегі нақты үлесін көрсетпейді.
Объективтілік, өлшемділік және Қазақстанның стратегиялық басымдықтарымен
байланыстыру қағидаттарына негізделген кешенді KPI енгізу мемлекеттік
қызметшілердің жұмысын бағалаудың ашық және тиімді жүйесін құруға
мүмкіндік береді. Салалық ерекшеліктерді, жауапкершілік деңгейін,
тапсырманы орындаудың тартылуын және тиімділігін ескеретін динамикалық
KPI үлгілерін әзірлеу мемлекеттік басқару сапасын арттырып қана
қоймайды, сонымен қатар персоналдың кәсіби дамуы үшін уәждемелік
ынталандырулар жасайды.

Кадрлық ресурстарын басқарудың мемлекеттік аудит жүйесін жаңғырту үш
негізгі элементке негізделуі тиіс: аудиторлардың кәсіби құзыреттілігін
күшейту, аудиторлық рәсімдерді цифрландыру және мемлекеттік
қызметшілердің қызметін бағалау жүйесін жетілдіру. Бұл тәсілдерді
кешенді түрде енгізу Қазақстанның стратегиялық даму талаптарына және
мемлекеттік басқарудың халықаралық стандарттарына сәйкес келетін
персонал аудитінің заманауи, ашық және жоғары тиімді жүйесін құруға
мүмкіндік береді.

Кадрлық ресурстарын басқару тиімділігін тексерудің маңыздылығын
мойындағанына қарамастан, Қазақстан Республикасында бұл проблемаға
жүйелі көзқарас әлі де жоқ. Қолданыстағы мемлекеттік аудит жүйесі ең
алдымен қаржылық аспектілерге бағытталған, бұл ретте кадрлық процестер
мен олардың тиімділігінің аудиті жеткіліксіз әзірленген және
енгізілмеген. Бұл кадрлық ресурстарын тиімсіз пайдаланудың елеулі
тәуекелдерін тудырады, жаңғырту процестерін бәсеңдетеді және мемлекеттік
қызметтің даму мүмкіндіктерін шектейді.

Қазақстан Республикасында мемлекеттік аудитті жүзеге асыру тұжырымдамасы
Қазақстан Республикасы Президентінің 2012 жылғы 14 желтоқсандағы
«Қазақстан-2050» Стратегиясы қалыптасқан мемлекеттің жаңа саяси бағыты»
атты Қазақстан халқына Жолдауына сәйкес әзірленді, онда: «Бізге Мемлекет
басшысы 2012 жылғы 3 қыркүйектегі ең озық әлемдік тәжірибе негізінде
мемлекеттік аудиттің кешенді жүйесін құру қажет» делінген Қазақстан
Республикасы.

Осы Тұжырымдама мемлекеттік қаржылық бақылауды кезең-кезеңімен
реформалаудың және оның функционалдық бағыттарын одан әрі кеңейте
отырып, мемлекеттік аудитті енгізудің пайымдауын және негізгі тәсілдерін
айқындайды. Бұл ретте мемлекеттік қаржылық бақылаудың функциялары
тексерілетін субъектінің қызметінде зардаптарды жоюды және
жауапкершілікке тартуды талап ететін заң бұзушылықтар анықталған кезде
іске асырылатын болады {[}16{]}.

Тұжырымдамада ең алдымен қаржылық және мемлекеттік аудиттің тиянақты
жүргізілу қажеттілігі атап көрсетілген. Бұл заң талаптарының сақталуын
тексеруді ғана емес, сонымен қатар басқару процестерінің тиімділігін
бағалауды да қамтиды. Маңызды міндет - жүйедегі мүмкін болатын
кемшіліктерді анықтау және жою, бұл бюджеттік қаражатты неғұрлым ұтымды
пайдалануға және тәуекелдерді азайтуға ықпал етеді.

Қазақстандағы мемлекеттік аудит тұжырымдамасы мемлекеттік басқарудың
ашықтығы мен тиімділігін арттырудың маңызды құралы болып табылады және
оны табысты жүзеге асыру кешенді тәсілді және барлық мүдделі тараптарды
тартуды талап етеді.

Тұжырымдамада көрсетілген мемлекеттік аудитті дамытудың маңызды
бағыттарының бірі кадрлық ресурстарын басқару тиімділігін бақылау болып
табылады, өйткені мемлекеттік бағдарламалар мен стратегиялық
бастамаларды іске асырудың табыстылығын айқындайтын кадрлық капиталы
болып табылады. Мемлекеттік аппаратты жаңғырту жағдайында кадрларды
іріктеу, орналастыру және дамытудағы жүйелі проблемаларды анықтауға,
сондай-ақ кадрлық саясаттың тиімділігін бағалауға мүмкіндік беретін
кадрлық үдерістердің аудиті ерекше маңызға ие.

Мемлекеттік органдарда кадрлық ресурстарын тиімді басқару ашықтық, есеп
берушілік және қызметті объективті бағалау принциптеріне негізделуі
керек. Дегенмен, тиісті аудит тетіктерінсіз кадрлық шешімдердің
негізділігінің жоғары дәрежесіне және олардың елдің стратегиялық
мақсаттарына сәйкестігіне қол жеткізу мүмкін емес. Осы тұрғыда кадрлықи
ресурстарды басқарудың мемлекеттік аудиті кадрлық ресурстары жүйесіндегі
кемшіліктерді анықтаудың, мемлекеттік қызметшілердің құзыреттілік
деңгейін бағалаудың және олардың қызметінің тиімділігін бақылаудың
негізгі құралына айналады.

Зерттеуді эмпирикалық негізде нығайту мақсатында Қазақстан
Республикасының ресми статистикалық деректері талданды. Талдау
көрсеткендей, 2023 жылы мемлекеттік органдардағы кадрлардың көші-қоны
деңгейі 17\%-ды құрап, жеке сектордың көрсеткішінен (10\%) айтарлықтай
жоғары болды. Бұл жағдай мемлекеттік қызметтегі мотивация жүйесі мен
бейімделу мәселелерін айқын көрсетеді.

Сонымен қатар, Еңбек және халықты әлеуметтік қорғау министрлігінің
мәліметтері бойынша, мемлекеттік органдардың тек 35\%-ы ғана персоналдың
тиімділігін бағалау үшін KPI жүйесін іс жүзінде қолданады, бұл
басқарудың субъективтілігі мен нәтижелердің төмендігінің бір себебі
болып табылады {[}17{]}.

Халықаралық салыстырмалы талдау OECD және БҰҰ-нің ұсыныстары негізінде
жүргізілді. Атап айтқанда, OECD-дің «Адам ресурстарын басқарудағы жақсы
тәжірибелер бойынша нұсқаулығы» мемлекеттік сектордағы кадрлық аудиттің
негізгі бағыты стратегиялық басқару мен нәтижелерге бағытталуы
керектігін атап өтеді. Мысалы, БҰҰ-нің Әдістемелік ұсыныстарына сәйкес,
тиімді кадрлық аудит жүйесі қызметкерлердің біліктілігі мен мемлекеттік
бағдарламаларды іске асыру нәтижелері арасындағы байланысты өлшеуі тиіс.

Халықаралық ұйымдардың кадрлық аудит стандарттары 3 кестеде көрсетілген.

Осы тәжірибеге сүйене отырып, біз Қазақстан контекстінде кадрлық
әлеуетті бағалаудың екі негізгі индикаторын енгізуді ұсынамыз:~«кадрлық
рентабельділік индексі» (бөлімшенің нәтижелілігінің оның кадрлық
шығындарына қатынасы)~және~«стратегиялық сәйкестік коэффициенті»
(лауазымдық тапсырмалар мен ұлттық стратегиялық мақсаттар арасындағы
байланыс деңгейі).

Практикалық негіздемені нығайту үшін Қазақстан Республикасының бірнеше
министрліктерінде енгізілген «Кадрлық резерв» жобасының нәтижелері
зерттелді. Тәжірибелік сынақ көрсеткендей, бұл платформаны пайдалану
лауазымдарды ашуға дейінгі орташа уақытты 30 күннен 10 күнге дейін
қысқартты, бірақ жүйе әлі де қызметкерлердің кәсіби даму траекториясын
талдауға мүмкіндік бермейді. Осы проблемаларды шешу үшін біз жасанды
интеллект негізіндегі~«Кадрлық сценарийлерді болжау модулін»~енгізуді
ұсынамыз, ол үлкен деректерді талдай отырып, салалық министрліктерде
біліктілік тапшылығының ықтималдығын 85\% дәлдікпен болжауға мүмкіндік
береді.

Мемлекеттік сектордағы кадрлық аудиттің тиімділік көрсеткіштері 4
кестеде көрсетілген.
\end{multicols}

\tcap{3 - кесте. Халықаралық ұйымдардың кадрлық аудит стандарттары}
\begin{longtblr}[
  label = none,
  entry = none,
]{
  width = \linewidth,
  colspec = {Q[213]Q[215]Q[242]Q[267]},
  cells = {c},
  cell{6}{1} = {c=4}{0.937\linewidth},
  cells = {font = \small},
  hlines,
  vlines,
}
\textbf{Критерийлер}                                          & \textbf{OECD тәжірибесі} & \textbf{БҰҰ стандарттары}   & \textbf{Қазақстанның ағымдағы жүйесі} \\
Кадрлық саясатты бағалау                                      & Интеграцияланған бағалау & Аумақтық тәсіл              & Фрагментарлық бағалау                 \\
Тиімділік көрсеткіштері                                       & NPM көрсеткіштері        & BSC жүйесі                  & Негізгі KPI                           \\
Аудиттік есептілік                                            & Ашық есеп беру           & Интеграцияланған мониторинг & Дәстүрлі есеп беру                    \\
Нәтижелерді талдау                                            & Салыстырмалы талдау      & Бенчмаркинг                 & Сараптамалық бағалау                  \\
Ескерту: дереккөздер негізінде авторлармен құрастырылған [18] &                          &                             &                                       
\end{longtblr}

\tcap{4 кесте - Кадрлық аудиттің жаңа әдістемесінің индикаторлары}
\begin{longtblr}[
  label = none,
  entry = none,
]{
  width = \linewidth,
  colspec = {Q[371]Q[150]Q[202]Q[213]},
  cells = {c},
  cell{7}{1} = {c=4}{0.936\linewidth},
  cells = {font = \small},
  hlines,
  vlines,
}
\textbf{Индикаторлар}                                         & \textbf{Өлшем бірлігі} & \textbf{Базистік мән (2023)} & \textbf{Мақсатты мән (2025)} \\
Кадрлық саясаттың тиімділік индексі                           & балл                   & 5,8                          & 7,5                          \\
Стратегиялық сәйкестік коэффициенті                           & \%                     & 65                           & 85                           \\
Аудиттік ұсыныстарды орындау индексі                          & балл                   & 6,2                          & 8,0                          \\
Цифрлық аудиттің қамту деңгейі                                & \%                     & 40                           & 75                           \\
Кадрлық тәуекелдерді болжау дәлдігі                           & \%                     & 60                           & 85                           \\
Ескерту: дереккөздер негізінде авторлармен құрастырылған [18] &                        &                              &                              
\end{longtblr}

\begin{multicols}{2}
Зерттеудің ғылыми жаңалығы мемлекеттік кадрлық аудиттің заманауи үлгісін
әзірлеу болып табылады, ол үш негізгі инновацияны біріктіреді:~1)
кадрлық шешімдердің ұлттық стратегиялық басымдықтарға әсерін өлшейтін
индикаторлар жиынтығы; 2) халықаралық стандарттарға негізделген кадрлық
тәуекелдерді басқарудың бірыңғай әдістемесі; 3) мемлекеттік органдардың
кадрлық саясатының сапасын салыстырмалы талдауға мүмкіндік беретін
бірегей цифрлық белгілер жүйесі.~Бұл тәсілдерді іс жүзінде енгізу
мемлекеттік органдардың әкімшілік қызметінің тиімділігін бағалау
көрсеткішін болашақта 25\%-ға дейін арттыруға мүмкіндік береді.

Кадрлық процестерді бағалау бағытында мемлекеттік аудиттің функционалдық
мүмкіндіктерін кеңейту мемлекеттік басқару жүйесін жетілдірудің және
уақыт сындарына тиімді жауап беруге қабілетті кәсіби мемлекеттік
аппаратты қалыптастырудың ажырамас шарты болып табылады.

Қазақстанда кадрлық ресурстарын басқарудың мемлекеттік аудитін жетілдіру
тиімді мемлекеттік басқарудың маңызды құрамдас бөлігі болып табылады.
Дегенмен, осы кезеңде кадрлық аудит жүйесінде оның тиімділігін шектейтін
және қабылданған шешімдердің негізділік деңгейін төмендететін бірқатар
маңызды проблемалар бар. Цифрлық трансформация, мемлекеттік
қызметшілердің кәсібилігіне қойылатын талаптардың артуы және заманауи
сын-қатерлерге бейімделу қажеттілігі жағдайында персонал аудитінің
тиімділігін арттыруға кедергі келтіретін негізгі кедергілерді жүйелі
талдау ерекше маңызға ие.

1. Аудиторлардың біліктілігінің жеткіліксіздігі. Ең маңызды шектеулердің
бірі -- кадрлық ресурстарын басқару және аудит әдістемесі бойынша терең
білімі бар мамандардың болмауы. Қазақстан Республикасы Қаржы
министрлігінің мәліметі бойынша, 2022 жылы аудиторлардың 60\%-ы ғана
сапалы аудит жүргізу үшін қажетті құзыреттерге ие болды. Кәсіби
дайындықтың жоқтығы кадр саясаты саласындағы көптеген жүйелі
проблемалардың назардан тыс қалып отырғанын және оларды жою бойынша
ұсыныстардың үстірт екендігін білдіреді {[}19{]}. Осының салдарынан
персоналды басқарудағы көптеген мәселелер назардан тыс қалады немесе
шешімдер үстірт болып қалады, бұл реформалардың тиімділігін төмендетеді.

2. Цифрландырудың төмен деңгейі. Мемлекеттік басқаруды дамытудың қазіргі
кезеңі цифрлық технологияларды белсенді енгізуді талап етеді, бірақ
кадрлық аудитті автоматтандыру деңгейі жеткіліксіз болып қалуда.
Қазақстан Республикасы Цифрлық даму, инновациялар және аэроғарыш
өнеркәсібі министрлігінің мәліметі бойынша 2023 жылға қарай мемлекеттік
органдардың 40\%-ы ғана кадрлық ресурстарын басқару мен аудиттің
автоматтандырылған жүйелерін енгізген {[}20{]}. Бұл деректерді жинау
және талдау процесін баяулатады, бұл аудит уақытының ұлғаюына және
олардың сапасының төмендеуіне әкеледі.

3. Кадрлық ресурстарының тиімділігін адекватты түрде бағаламау. Кадрлықи
капиталды тиімді басқару қызметкерлердің қызметін бағалаудың объективті
әдістерінсіз мүмкін емес. Алайда, Еңбек және халықты әлеуметтік қорғау
министрлігінің статистикалық мәліметтеріне сәйкес, 2023 жылы мемлекеттік
органдардың 35\%-ы ғана персонал қызметін бағалау үшін қызметтің негізгі
көрсеткіштерінің (KPI) жүйесін пайдаланған {[}21{]}. Қазіргі заманғы
бағалау құралдарының орнына көптеген бөлімдер жұмыс уақытын есепке алу
сияқты ресми критерийлерге негізделген ескірген тәсілдерді сақтайды, бұл
қызметкерлердің мемлекеттің стратегиялық мақсаттарына қол жеткізудегі
үлесін толық бағалауға мүмкіндік бермейді.

4. Ашықтық пен есеп беруді бұзу. HR процестерінің аудиті көбінесе есеп
берудегі және мемлекеттік органдар арасындағы үйлестірудің әлсіз
жақтарын анықтайды. Жоғарғы есеп палатасының мәліметі бойынша, 2022 жылы
кадрлық аудитте 123 заң бұзушылық фактісі тіркелді, оның 65 пайызы есеп
беру және департаменттер арасындағы ақпарат алмасу мәселелеріне қатысты
{[}22{]}. Бұзушылықтар арасында кадрлық өзгерістер туралы уақтылы
хабарланбау және ведомстволар арасындағы үйлестірудің жеткіліксіздігі
жатады. Бұл аудиторлардың жұмысын қиындататын және процестердің
ашықтығын төмендететін кадрлық ресурстарын басқаруға жүйелі көзқарастың
жоқтығы проблемасын көрсетеді.

5. HR қызметінің төмен тиімділігі. Тағы бір күрделі мәселе --
мемлекеттік органдардағы кадрлардың көп ауысуы. Еңбек министрлігінің
мәліметі бойынша, 2023 жылы мемлекеттік органдардағы ауыс-түйіс деңгейі
17\%-ды құрады, бұл жеке сектордағы осындай көрсеткіштен (10\%) жоғары
{[}23{]}. Бұл мемлекеттік қызмет қызметкерлерінің қанағаттану деңгейінің
төмендігін және дарындылықты сақтау тетіктерінің әлсіз дамуын көрсетеді.
Кадрлық саясат аудиті көбінесе мұндай проблемаларды дер кезінде анықтай
алмайды, өйткені олар персонал жұмысының сапасын жақсартуға емес,
процедуралардың сақталуын бақылауға бағытталған.

Ағымдағы жағдай кадрлық ресурстарын басқарудың мемлекеттік аудитінің
тәсілдерін жаңғыртуды талап етеді. Бұл саладағы реформаның маңызды
бағыты аудиторлардың біліктілігін арттыру, цифрлық технологиялар мен
автоматтандырылған жүйелерді белсенді енгізу, кадрлық ресурстарын
бағалаудың тиімді KPI әзірлеу, кадр саясатының ашықтығы мен есептілігін
қамтамасыз ету шараларын күшейту болуы тиіс. Бар кедергілерді кешенді
түрде жою мемлекеттік органдардың қызметін айтарлықтай жақсартады және
Қазақстанның стратегиялық даму мақсаттарына жауап беретін кәсіби
мемлекеттік аппараттың қалыптасуын қамтамасыз етеді.

Мемлекеттік басқару саласындағы заманауи сын-қатерлер кадрлық
ресурстарының аудитіне қатысты тәсілдерді жан-жақты қарастыруды және
жетілдіруді талап етеді. Кадрлық саясаттың ашықтығын, тиімділігін және
тиімділігін қамтамасыз ету үшін анықталған проблемаларды жоюға
бағытталған инновациялық құралдарды енгізу қажет. Осы тұрғыда келесі
бағыттар мемлекеттік аудитті реформалаудың негізгі басымдықтарына
айналады:

Жоғары білікті аудиторлық корпусты қалыптастыру кадрлық ресурстарын
басқару саласындағы тиімді бақылаудың негізі болып табылады.
Мамандандырылған білім беру бағдарламаларын, біліктілікті арттыру
курстарын және сертификаттау талаптарын енгізу персоналды басқару,
персонал қызметін бағалау және халықаралық аудит стандарттарын қолдану
саласында терең білімі бар аудиторлардың кәсіби қоғамдастығын құруға
мүмкіндік береді.

Цифрлық технологияларды персонал аудитіне кіріктіру мемлекеттік
қызметшілердің қызметін бақылаудың жылдамдығы мен дәлдігін айтарлықтай
арттырады. Үлкен деректерге, жасанды интеллектке және деректерді
автоматтандырылған өңдеуге негізделген аналитикалық жүйелерді әзірлеу
және енгізу жүйелік кемшіліктерді белсенді түрде анықтауды қамтамасыз
етеді және уақтылы және негізделген басқару шешімдерін қабылдауға
мүмкіндік береді.

Барлық мемлекеттік органдар үшін нақты және ашық KPI жүйесін құру
кадрлық ресурстарын бағалаудың объективтілігін арттыру жолындағы маңызды
қкадрлық болып табылады. Стандартты көрсеткіштер мен тиімділік
параметрлерін анықтау мемлекеттік қызметшілердің қызметін бағалауда
субъективтілікті болдырмауға көмектеседі және басшылардың өз
персоналының жұмысы үшін жауапкершілігін арттырады.

Аталған шараларды іске асыру кадрлық ресурстарын басқарудың мемлекеттік
аудитін жаңғыртуға, оның аналитикалық және болжамдық мәнін арттыруға,
мемлекеттік қызметте кадрлық әлеуетті тиімді бөлу мен пайдалануды
қамтамасыз етуге мүмкіндік береді. Бұл қкадрлықдар Қазақстан
Республикасының стратегиялық мақсаттарына қол жеткізу және заманауи
сын-қатерлер мен жаһандық даму талаптарына жауап беретін тиімділігі
жоғары мемлекеттік аппаратты қалыптастыру үшін қажет.

Статистикалық талдау көрсеткендей, кадрлық ресурстарын басқару аудиті
саласындағы бар проблемалар аудиторлардың құзыреттілігін арттыруға,
технологияларды жаңартуға және тиімділікті бағалаудың неғұрлым сапалы
және объективті әдістерін енгізуге бағытталған жүйелі шешімді талап
етеді.

«Мемлекеттік аудит және қаржылық бақылау туралы» Заңның негізгі
аспектілерін түсінуге ерекше назар аудару қажет. Бұл заң Қазақстан
Республикасында мемлекеттік аудиттің тиімді жүйесін құру мен дамытудың
іргелі құжаты болып табылады.

Заңда мемлекеттік ресурстарды пайдалануда адалдық пен жариялылықты
қамтамасыз етудің негізі болып табылатын ашықтық пен есеп беру
қағидаттарына ерекше мән берілген. Заң бұзушылықтардың алдын алу және
анықтау тетіктерін қарастырады, бұл мемлекеттік институттарға сенім
деңгейін арттыруға ықпал етеді {[}24{]}.

Заңда аудиторлық тексерулерді жүргізуде кәсіби көзқарас қажеттігін де
атап көрсетеді. Бұл аудиторлардың біліктілігіне қойылатын талаптарды,
сонымен қатар аудиторлық тексерулердің дәлдігі мен тиімділігін арттыру
үшін заманауи әдістер мен технологияларды қолдануды қамтиды.

Осыған байланысты, кадрлық ресурстарын басқару саласындағы мемлекеттік
аудит мемлекеттік басқарудың тиімділігін қамтамасыз етудің ең маңызды
құралы болып табылатынын атап өткен жөн, өйткені мемлекеттік
функцияларды табысты орындау және стратегиялық мақсаттарға қол жеткізу
негізінде кадрлықи капитал жатыр. Жаһандық цифрландыру және мемлекеттік
аппаратты трансформациялау жағдайында еңбек ресурстарын ұтымды
пайдалану, кадрлық үдерістердің ашықтығы және олардың заманауи
сын-қатерлерге сәйкестігі мәселелері өте маңызды болып отыр.

Осы саладағы мемлекеттік аудиттің негізгі мақсаты еңбек әлеуетін оңтайлы
пайдалану қағидаттарынан ауытқуларды анықтау, сондай-ақ жұмысқа қабылдау
жүйелерінің, мансаптық өсу мен қызметкерлерді ынталандырудың ұзақ
мерзімді мемлекеттік басымдықтарға сәйкестігін бағалау болып табылады.
Персонал саны немесе оларды ұстауға жұмсалатын шығындар деңгейі сияқты
сандық көрсеткіштерді ғана емес, сонымен қатар кәсібилік деңгейін,
қызметкерлердің қанағаттанушылығын және олардың ұйымның мақсаттарына
жетуге қосқан үлесін қоса алғанда, сапалы аспектілерді де ескеретін
кешенді тәсілді қолдану қажеттілігі ерекше қиын.

Цифрландыру, жұмыс сипатының өзгеруі және ашықтық талаптарының артуы
сияқты заманауи міндеттер мемлекеттік аудиттің инновациялық әдістер мен
құралдарды қолдануын талап етеді. Бұл кадрлық саясаттың әлсіз тұстарын
анықтауға ғана емес, сыбайлас жемқорлық тәуекелдерін азайтуға,
персоналды басқару тиімділігін арттыруға және ашықтық пен есеп берушілік
қағидаттарына негізделген жүйені құруға бағытталған ұсыныстар әзірлеуге
мүмкіндік береді.

Кадрлық ресурстарын басқарудың мемлекеттік аудиті уақыт сын-қатерлеріне
жауап беруге және тұрақты дамуды қамтамасыз етуге қабілетті заманауи,
тиімді және ашық мемлекеттік аппаратты құрудың негізгі элементіне
айналуда.

{\bfseries Қорытынды.} Қазақстанда кадрлық ресурстарын басқарудың
мемлекеттік аудитінде анықталған проблемаларды еңсеру үшін аудиттің
тиімділігі мен сапасын арттыруға, сондай-ақ мемлекеттік органдарда
кадрлық ресурстарын басқару жүйесін жетілдіруге мүмкіндік беретін
бірқатар практикалық шараларды жүзеге асыру қажет.

1. Аудиторлардың біліктілігін арттыру. Кадр тапшылығын жою және аудит
сапасын арттыру үшін аудиторларды үздіксіз оқыту және біліктілігін
арттыру бағдарламаларын жасау қажет. Бұл бағдарламалар мыналарды қамтуы
керек:

- кадрлық ресурстарын басқарудың жаңа әдістеріне және HR процестерінің
аудитіне оқыту;

- INTOSAI және ISSAI әдістемелері сияқты халықаралық стандарттармен және
озық тәжірибелермен танысу;

- сандық технологияларды аудитте қолдану, соның ішінде үлкен деректерді
талдау және жасанды интеллект құралдарын пайдалану бойынша
мамандандырылған курстар.

Ұсыныс: кадрлық ресурстарын басқару және қазіргі заманғы аудит
құралдарын мамандандырылған оқытуға бағытталған мемлекеттік аудиторларды
даярлаудың ұлттық орталығын құру.

2. HR процестерін тексерудің әдістемелік негізін жасау. Аудиттің
тәсілдерін біріздендіру және бағалаудың объективтілігін арттыру үшін
қазақстандық мемлекеттік органдардың ерекшеліктерін ескеретін кадрлық
ресурстарын басқару аудитін жүргізудің бірыңғай стандарттары мен
әдістерін әзірлеу және енгізу қажет.

Ұсыныс: Қызметтің негізгі критерийлерін, өнімділік көрсеткіштерін,
қызметкерлердің мотивациясын бағалау әдістерін және кері байланыс
жүйелерін сипаттайтын HR процестерін тексеру бойынша нұсқаулықты
әзірлеу.

3. Цифрлық технологияларды қолдануды күшейту. Цифрлық технологияларды
енгізу және аудиторлық процестерді автоматтандыруды арттыру, бұл
аудиторлардың тиімділігін арттыруға және деректерді талдау сапасын
жақсартуға мүмкіндік береді. Үлкен деректер аналитикасын пайдалану HR
процестеріндегі үлгілерді анықтауға көмектеседі, ал жасанды интеллект
күнделікті тапсырмаларды автоматтандырады.

Ұсыныс: мемлекеттік органдардағы кадрлық ресурстары туралы деректерді
жинау, сақтау және талдау үшін мамандандырылған цифрлық платформаларды
енгізу. Платформаларда қызметкерлердің ауысуын, біліктілік деңгейін
бақылау және қызметкерлердің мотивациясын бағалау мүмкіндіктері болуы
керек.

4. Аудиторлар мен мемлекеттік органдар арасындағы ынтымақтастықты
жақсарту. Мемлекеттік органдардың қарсылығын азайту үшін аудиторлар мен
менеджерлердің өзара тығыз қарым-қатынасына жағдай жасау қажет. Бұған
аудит мақсаттары мен міндеттері талқыланатын, сондай-ақ аудиторлық
ұсыныстарды сәтті орындау мысалдары талқыланатын тұрақты жиналыстар мен
семинарлар арқылы қол жеткізуге болады.

Ұсыныс: мемлекеттік аудиторлар мен HR қызметтері арасындағы тұрақты
диалог бағдарламасын құру, ол аудиттің нәтижелеріне негізделген HR
басқару тиімділігін жақсарту бойынша оқыту мен кеңес беруді қамтиды.

5. Аудиттің ашықтығын арттыру және кері байланыс тетіктерін енгізу.
Аудитке деген сенімді арттыру және нәтижелерді жақсарту үшін аудит
процестерінің ашықтығын қамтамасыз ету қажет. Аудиторларға өз
ұсыныстарының қалай орындалып жатқаны туралы ақпарат алуға мүмкіндік
беретін кері байланыс механизмін енгізу де маңызды элемент болып
табылады.

Ұсыныс: мемлекеттік мекеме басшыларына аудиторлық ұсыныстардың орындалуы
туралы есеп беруге және аудиторларға осы ақпарат негізінде өз тәсілдерін
бейімдеуге мүмкіндік беретін кері байланыс жүйесін енгізу.

6. Халықаралық ынтымақтастық пен тәжірибе алмасуды дамыту. Персоналды
басқарудың мемлекеттік аудиті саласындағы озық халықаралық тәжірибені
бейімдеу Қазақстанға озық тәжірибені енгізуге және аудит процестерінің
сапасын арттыруға мүмкіндік береді.

Ұсыныс: жетекші халықаралық ұйымдарда қазақстандық аудиторлардың және
басқа елдердің аудиторларының тәжірибе алмасу және HR үдерістері
аудитінің озық әдістерін зерделеу үшін тұрақты тағылымдамадан өтуін
ұйымдастыру.

7. Аудиторлық органдарды қаржыландыру мүмкіндігі. Ресурстардың
жетіспеушілігі аудитті сапалы жүргізуді қиындатады, сондықтан
мемлекеттік аудит органдарының бюджеттік қаржыландыруын қайта қарау
қажет.

Ұсыныс: цифрлық құралдарды алуға, қызметкерлерді оқытуға және аудитті
қамтуды кеңейту үшін қызметкерлер санын арттыруға баса назар аудара
отырып, аудит органдарын қаржыландырудың мақсатты бағдарламасын әзірлеу.

8. Біліктілікті арттыру және мансаптық өсу бағдарламаларын әзірлеу.
Мемлекеттік аудит мемлекеттік қызметшілерді оқыту бағдарламаларының
тиімділігін және олардың мемлекеттік тапсырмаларды орындауға әсерін
ескеруі тиіс.

Ұсыныс: нәтижелерді кейіннен кадрлық аудит процесіне біріктіре отырып,
мемлекеттік қызметшілерді оқыту бағдарламаларының тұрақты мониторингі
мен бағалау жүйесін әзірлеу.
\end{multicols}

\begin{center}
{\bfseries Әдебиеттер}
\end{center}

\begin{refs}
1. Катковская И. В. Аудит кадровых процессов как инструмент оценки
системы управления персоналом организации // Управление персоналом и
развитие организаций. - 2017. - № 1. - С.65-66.

2. Бахарев В. В., Демененко И. А. Технология кадрового аудита в
стратегии управления человеческими ресурсами // Альманах «Крым». - 2022.
-№ 31. - С.18--25.

3. Сергеев Л. И. Государственный аудит: учебник для вузов. - 2-е изд.,
перераб. и доп. - М.: Юрайт, 2025. - 373 с. ISBN 978-5-534-16434-3

4. Қазақстан Республикасы Президентінің Қ.-Ж. Қ.Тоқаевтын Қазақстан
халқына 2024 жылғы 2 қыркүйектегі жолдауы. URL:
\href{https://www.akorda.kz/ru/poslanie-glavy-gosudarstva-kasym-zhomarta-tokaeva-narodu-kazahstana-spravedlivyy-kazahstan-zakon-i-poryadok-ekonomicheskiy-rost-obshchestvennyy-optimizm-285014.-\%20\%20Қаралған}{https://www.akorda.kz/ru/poslanie-glavy-gosudarstva-kasym-zhomarta-tokaeva-narodu-kazahstana-spravedlivyy-kazahstan-zakon-i-poryadok-ekonomicheskiy-rost-obshchestvennyy-optimizm-285014.-
Қаралған} күні: 07.02.2025.

5. Bratton, J., Gold, J. Human resource management: Theory and practice.
5th ed. - London: Palgrave Macmillan, 2012. - 592 p.

6. Lawler, E. E., Boudreau, J. W. Global trends in human resource
management: A twenty-year analysis. - Stanford: Stanford University
Press, 2015.- 216 p. ISBN~9780804791298.

7. Саунин А. Н. Государственный аудит: учебное пособие. - 2-е изд.,
стер. - М.: Изд-во МГУ, 2019. - 480 с. ISBN 978-5-19-011412-6.

8. Климанов В. В., Казакова С. М., Яговкина В. А. Инструменты
межрегионального взаимодействия в системе государственного управления //
Регионология. - 2021. - № 2. - С.250-282.
\href{https://doi.org/10.15507/2413-1407.115.029.202102.250-282}{DOI
10.15507/2413-1407.115.029.202102.250-282}.

9. Алибекова Б. А., Алдынгарова Д. Т. Государственный аудит финансовой
отчетности // Central Asian Economic Review. - 2018. -№ 4. - С.72--83.

10. Бокаев Б. Н., Торебекова З. Т., Жаров Е. К., Бактиярова Б. Н. Анализ
моделей государственной службы в зарубежных странах и перспективы их
адаптации в Республике Казахстан // Государственный аудит. -2024. - № 1
(62).-С.167-182. DOI
\href{https://doi.org/10.55871/2072-9847-2024-62-1-167-182}{10.55871/2072-9847-2024-62-1-167-182}.

11. 2013 жылғы 3 қыркүйектегі Мемлекеттік аудитті жүзеге асыру
тұжырымдамасы. URL:
\href{https://adilet.zan.kz/rus/docs/U1300000634.-}{https://adilet.zan.kz/rus/docs/U1300000634.-}
Қаралған күні: 07.02.2025.

12. «Мемлекеттік органдардың кадрлық ресурстарын басқару қызметінің
мәртебесін арттыру» зерттеу нәтижелері туралы есеп. - Мемлекеттік қызмет
саласындағы Astana Hub, 2021 ж.URL:
\href{https://surl.li/oeswbe}{https://surl.li/oeswbe}. - Қаралған
күні: 07.02.2025.

13. Шаймерденова Т. А., Жетписбаева М. К. Мотивационные аспекты
государственных служащих в Казахстане в условиях цифровизации экономики
// Государственный аудит. -- 2024. - № 1 (62). - С.111-125. DOI
\href{https://doi.org/10.55871/2072-9847-2024-62-1-111-125}{10.55871/2072-9847-2024-62-1-111-125}.

14. Қазақстан Республикасы Сыбайлас жемқорлыққа қарсы іс-қимыл
агенттігінің ресми сайты. URL:
\href{https://www.gov.kz/memleket/entities/anticorruption?lang=ru}{https://www.gov.kz/memleket/entities/anticorruption?lang=ru}.-
Қаралған күні: 08.02.2025.

15. Қазақстан Республикасы Стратегиялық жоспарлау және реформалар
агенттігінің Ұлттық статистика бюросы // Қол жеткізу режимі:
https://stat.gov.kz/ru/ - Қаралған күні: 07.02.2025.

16.2013 жылғы 3 қыркүйектегі Мемлекеттік аудитті жүзеге асыру
тұжырымдамасы. URL:
\href{https://adilet.zan.kz/rus/docs/U1300000634}{https://adilet.zan.kz/rus/docs/U1300000634}.-
Қаралған күні: 08.02.2025.

17. OECD Recommendation of the Council on Public Service Leadership and
Capability. --
URL:~\href{https://legalinstruments.oecd.org/en/instruments/OECD-LEGAL-0445}{https://legalinstruments.oecd.org/en/instruments/OECD-LEGAL-0445}~Қаралған
күні: 08.02.2025.

18. United Nations Department of Economic and Social Affairs Working
Together: Integration, Institutions and the Sustainable Development
Goals. World Public Sector Report. --
URL:~\href{https://publicadministration.un.org/en/Research/World-Public-Sector-Reports}{https://publicadministration.un.org/en/Research/World-Public-Sector-Reports}.~Қаралған
күні: 08.02.2025.

19. Қазақстан Республикасы Қаржы министрлігінің ресми сайты. URL:
\href{https://www.gov.kz/memleket/entities/minfin?lang=ru}{https://www.gov.kz/memleket/entities/minfin?lang=ru}.-
Қаралған күні: 08.02.2025.

20. Қазақстан Республикасы Цифрлық даму, инновациялар және аэроғарыш
өнеркәсібі министрлігінің ресми сайты. URL:
\href{https://www.gov.kz/memleket/entities/mdai.-}{https://www.gov.kz/memleket/entities/mdai.-}
Қаралған күні: 09.02.2025.

21. Қазақстан Республикасы Еңбек және халықты әлеуметтік қорғау
министрлігінің ресми сайты. URL:
\href{https://www.gov.kz/memleket/entities/enbek}{https://www.gov.kz/memleket/entities/enbek}
.- Қаралған күні: 09.02.2025.

22. Қазақстан Республикасы Жоғарғы есеп палатасының ресми сайты.URL:
\href{https://www.gov.kz/memleket/entities/esep?lang=ru.-}{https://www.gov.kz/memleket/entities/esep?lang=ru.-}
Қаралған күні: 10.02.2025.

23. Қазақстан Республикасы Еңбек және халықты әлеуметтік қорғау
министрлігінің ресми сайты. URL:
\href{https://www.gov.kz/memleket/entities/enbek}{https://www.gov.kz/memleket/entities/enbek}.-
Қаралған күні: 10.02.2025.

24. «Мемлекеттік аудит және қаржылық бақылау туралы» Қазақстан
Республикасының 2015 жылғы 12 қарашадағы Заңы. URL:
\href{https://adilet.zan.kz/rus/docs/Z1500000392}{https://adilet.zan.kz/rus/docs/Z1500000392}.-
Қаралған күні:10.02.2025.
\end{refs}

\begin{center}
{\bfseries References}
\end{center}

\begin{refs}
1. Katkovskaja I. V. Audit kadrovyh processov kak instrument ocenki
sistemy upravlenija personalom organizacii // Upravlenie personalom i
razvitie organizacij. - 2017. - № 1. - S.65-66. {[}in Russian{]}

2. Baharev V. V., Demenenko I. A. Tehnologija kadrovogo audita v
strategii upravlenija chelovecheskimi resursami //
Al' manah «Krym». - 2022. -№ 31. - S.18-25. {[}in
Russian{]}

3. Sergeev L. I. Gosudarstvennyj audit: uchebnik dlja vuzov. - 2-e izd.,
pererab. i dop. - M.: Jurajt, 2025. - 373 s. ISBN 978-5-534-16434-3.
{[}in Russian{]}

4. Қazaқstan Respublikasy Prezidentіnің Қ.-Zh. Қ.Toқaevtyn Қazaқstan
halқyna 2024 zhylғy 2 қyrkүjektegі zholdauy. URL:
https://www.akorda.kz/ru/poslanie-glavy-gosudarstva-kasym-zhomarta-tokaeva-narodu-kazahstana-spravedlivyy-kazahstan-zakon-i-poryadok-ekonomicheskiy-rost-obshchestvennyy-optimizm-285014.-
Қaralғan kүnі: 07.02.2025. {[}in Kazakh{]}

5. Bratton, J., Gold, J. Human resource management: Theory and practice.
5th ed. - London: Palgrave Macmillan, 2012. - 592 p.

6. Lawler, E. E., Boudreau, J. W. Global trends in human resource
management: A twenty-year analysis. - Stanford: Stanford University
Press, 2015.- 216 p. ISBN~9780804791298.

7. Саунин А. Н. Государственный аудит: учебное пособие. - 2-е изд.,
стер. - М.: Изд-во МГУ, 2019. - 480 с. ISBN 978-5-19-011412-6. {[}in
Russian{]}

8. Климанов В. В., Казакова С. М., Яговкина В. А. Инструменты
межрегионального взаимодействия в системе государственного управления //
Регионология. - 2021. - № 2. - С.250-282.
\href{https://doi.org/10.15507/2413-1407.115.029.202102.250-282}{DOI
10.15507/2413-1407.115.029.202102.250-282}. {[}in Russian{]}

9. Алибекова Б. А., Алдынгарова Д. Т. Государственный аудит финансовой
отчетности // Central Asian Economic Review. - 2018. -№ 4. - С.72--83.
{[}in Russian{]}

10. Бокаев Б. Н., Торебекова З. Т., Жаров Е. К., Бактиярова Б. Н. Анализ
моделей государственной службы в зарубежных странах и перспективы их
адаптации в Республике Казахстан // Государственный аудит. -2024. - № 1
(62).-С.167-182. DOI
\href{https://doi.org/10.55871/2072-9847-2024-62-1-167-182}{10.55871/2072-9847-2024-62-1-167-182}.
{[}in Russian{]}

11. 2013 zhylғy 3 қyrkүjektegі Memlekettіk audittі zhүzege asyru
tұzhyrymdamasy. URL: https://adilet.zan.kz/rus/docs/U1300000634.-
Қaralғan kүnі: 07.02.2025. {[}in Kazakh{]}

12. «Memlekettіk organdardyң kadrlyқ resurstaryn basқaru қyzmetіnің
mәrtebesіn arttyru» zertteu nәtizhelerі turaly esep. - Memlekettіk
қyzmet salasyndaғy Astana Hub, 2021 zh.URL: https://surl.li/oeswbe. -
Қaralғan kүnі: 07.02.2025. {[}in Kazakh{]}

13. Shajmerdenova T. A., Zhetpisbaeva M. K. Motivacionnye aspekty
gosudarstvennyh sluzhashhih v Kazahstane v uslovijah cifrovizacii
jekonomiki // Gosudarstvennyj audit. - 2024. - № 1 (62). - S.111-125.
DOI 10.55871/2072-9847-2024-62-1-111-125. {[}in Kazakh{]}

14. Қazaқstan Respublikasy Sybajlas zhemқorlyққa қarsy іs-қimyl
agenttіgіnің resmi sajty. URL:
https://www.gov.kz/memleket/entities/anticorruption?lang=ru.- Қaralғan
kүnі:08.02.2025. {[}in Kazakh{]}

15. Қazaқstan Respublikasy Strategijalyқ zhosparlau zhәne reformalar
agenttіgіnің Ұlttyқ statistika bjurosy // Қol zhetkіzu rezhimі:
https://stat.gov.kz/ru/ - Қaralғan kүnі: 07.02.2025. {[}in Kazakh{]}

16.2013 zhylғy 3 қyrkүjektegі Memlekettіk audittі zhүzege asyru
tұzhyrymdamasy. URL: https://adilet.zan.kz/rus/docs/U1300000634.-
Қaralғan kүnі:08.02.2025. {[}in Kazakh{]}

17. OECD Recommendation of the Council on Public Service Leadership and
Capability. -
URL:\href{https://legalinstruments.oecd.org/en/instruments/OECD-LEGAL-0445}{https://legalinstruments.oecd.org/en/instruments/OECD-LEGAL-0445}.~Қaralғan
kүnі: 08.02.2025. {[}in Kazakh{]}

18. United Nations Department of Economic and Social Affairs Working
Together: Integration, Institutions and the Sustainable Development
Goals. World Public Sector Report. -
URL:~\href{https://publicadministration.un.org/en/Research/World-Public-Sector-Reports}{https://publicadministration.un.org/en/Research/World-Public-Sector-Reports}.

19. Қazaқstan Respublikasy Қarzhy ministrlіgіnің resmi sajty. URL:
https://www.gov.kz/memleket/entities/minfin?lang=ru.- Қaralғan kүnі:
08.02.2025. {[}in Kazakh{]}

20. Қazaқstan Respublikasy Cifrlyқ damu, innovacijalar zhәne ajeroғarysh
өnerkәsіbі ministrlіgіnің resmi sajty. URL:
https://www.gov.kz/memleket/entities/mdai.- Қaralғan kүnі: 09.02.2025.
{[}in Kazakh{]}

21. Қazaқstan Respublikasy Eңbek zhәne halyқty әleumettіk қorғau
ministrlіgіnің resmi sajty. URL:
https://www.gov.kz/memleket/entities/enbek .- Қaralғan kүnі: 09.02.2025.
{[}in Kazakh{]}

22. Қazaқstan Respublikasy Zhoғarғy esep palatasynyң resmi sajty.URL:
https://www.gov.kz/memleket/entities/esep?lang=ru.- Қaralғan kүnі:
10.02.2025. {[}in Kazakh{]}

23. Қazaқstan Respublikasy Eңbek zhәne halyқty әleumettіk қorғau
ministrlіgіnің resmi sajty. URL:
https://www.gov.kz/memleket/entities/enbek.- Қaralғan kүnі: 10.02.2025.
{[}in Kazakh{]}

24. «Memlekettіk audit zhәne қarzhylyқ baқylau turaly» Қazaқstan
Respublikasynyң 2015 zhylғy 12 қarashadaғy Zaңy. URL:
\url{https://adilet.zan.kz/rus/docs/Z1500000392.-} Қaralғan kүnі:
10.02.2025. {[}in Kazakh{]}
\end{refs}

\begin{info}
\hspace{1em}\emph{{\bfseries Авторлар туралы мәлімет}}

Мукушева А.Р. -- магистр, Л.Н. Гумилев атындағы Еуразия ұлттық
университетінің «Мемлекеттік аудит» Астана, Қазақстан, e-mail:
suleimenova.ar@bk.ru;

Серикова М.А. - Л.Н. Гумилев атындағы Еуразия ұлттық университетінің
Мемелкеттік аудит қауымдастырылған кафедрасының қауымдастырылған
профессоры PhD, Астана, Қазақстан, e-mail:
madinaserikova81@gmail.com.

\hspace{1em}\emph{{\bfseries Information about the authors}}

Mukusheva A. - master' s degree, 1st year doctoral
student of the Department of public auditing, Eurasian National
University named after L.N. Gumilyov, Astana, Kazakhstan, e-mail:
suleimenova.ar@bk.ru;

Serikova M. - PhD, associate professor, Department of state audit,
Eurasian National University named after L.N. Gumilyov, Astana,
Kazakhstan, e-mail:
madinaserikova81@gmail.com.
\end{info}
