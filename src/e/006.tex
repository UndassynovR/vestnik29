\id{IRSTI 06.02.13}{}

\begin{header}
\swa{}{DIGITALIZATION OF SCIENCE AS A TOOL FOR INCREASING THE COMPETITIVENESS OF THE ECONOMY OF THE REPUBLIC OF KAZAKHSTAN}

D.A.Turganbayev\envelope,
B.K. Isayeva,
M.O. Keznhegul,
D.A. Seithozhina
\end{header}

\begin{affil}
L.N. Gumilyov Eurasian National University, Astana, Kazakhstan

\corrauthor{Correspondent-author: dos\_25.97@mail.ru}
\end{affil}

In modern conditions of global competition and rapid technological
development, the digitalization of science is becoming particularly
important as a factor in increasing economic competitiveness. The
purpose of this study is to determine and quantify the impact of the
digitalization of the scientific sphere on the growth of the digital
competitiveness of the economy of the Republic of Kazakhstan. The
relevance of the topic is due to the need for an accelerated transition
to a knowledge economy, where science and technology are the main
sources of added value growth, the development of high-tech industries
and investment attraction. Kazakhstan, despite the existence of digital
development and scientific moderniza\-tion strategies, faces a number of
limitations: fragmented digital infrastructure, weak integration of
science and business, insufficient level of digital competencies among
scientific personnel. In the course of the study, a comparative analysis
of the experience of the Republic of Korea was carried out, econometric
modeling methods were applied, and an integral index of the
digitalization of science was developed. The model revealed a stable
positive relationship between the level of digitalization of scientific
infrastructure and the country' s position in the global
ranking of digital competitiveness. It has been established that even
with a moderate increase in digitalization, it is possible to
significantly strengthen the innovation potential and increase the
efficiency of research and development (R\&D). The results confirm the
need to implement a holistic government policy on the digital
transformation of the scientific field, including process automation,
the development of digital platforms, cooperation with business, and
strengthening the digital skills of researchers. The findings can serve
as a basis for the formation of strategies for scientific, technological
and economic development of Kazakhstan in the context of global digital
trends.

{\bfseries Keywords:} digitalization, infrastructure, science, economics,
innovation, investment, artificial intelli\-gence.

\begin{header}
ҒЫЛЫМДЫ ЦИФРЛАНДЫРУ ҚАЗАҚСТАН РЕСПУБЛИКАСЫНЫҢ ЭКОНОМИКАСЫНЫҢ БӘСЕКЕГЕ ҚАБІЛЕТТІЛІГІН АРТТЫРУ ҚҰРАЛЫ РЕТІНДЕ

Д.Ә. Турғанбаев\envelope,
Б.К. Исаева,
М.О. Кенжеғұл,
Д.А. Сейтхожина
\end{header}

\begin{affil}
Л.Н. Гумилев атындағы Еуразия ұлттық университеті, Астана, Қазақстан,

e-mail: dos\_25.97@mail.ru
\end{affil}

Әлемдік бәсекелестік пен жедел дамып келе жатқан технологиялық дамудың
қазіргі жағдайында ғылым саласын цифрландырулар экономиканың бәсекеге
қабілеттілігін арттыру факторы ретінде маңызды болып табылады.
Зерттеудің мақсаты ғылым салаларын цифрландырудың Қазақстан
экономикасындағы цифрлық бәсекелестікке қабілеттілігінің артуына
әсерлерін анықтау және сандық негіздеу болып табылады. Тақырыптың
өзектілігі білім экономикасына жедел көшу қажеттілігімен байланысты,
мұнда ғылым мен технология қосылған құнның өсуінің, жоғары технологиялық
салалардың дамуының және инвестицияларды тартудың негізгі көзі болып
табылады. Қазақстан цифрлық даму және ғылыми жаңғырту стратегияларының
болуына қарамастан бірқатар шектеулерге тап болады: цифрлық
инфрақұрылымның бөлшектенуі, ғылым мен бизнестің әлсіз интеграциясы,
ғылыми кадрларда цифрлық құзыреттіліктің жеткіліксіз деңгейі. Зерттеу
барысында Корея Республикасының тәжірибесіне салыстырмалы талдау
жүргізілді, эконометрикалық модельдеу әдістері қолданылды, ғылымды
цифрландырудың интегралды индексі әзірленді. Модель ғылыми
инфрақұрылымды цифрландыру деңгейі мен цифрлық бәсекеге қабілеттіліктің
жаһандық рейтингіндегі елдің позициясы арасындағы тұрақты оң байланысты
анықтауға мүмкіндік берді. Цифрландырудың қалыпты өсуімен де
инновациялық әлеуетті едәуір күшейтуге және ғылыми-зерттеу және
тәжірибелік-конструкторлық жұмыстардың (ҒЗТКЖ) тиімділігін арттыруға
болатындығы анықталды. Нәтижелер процестерді автоматтандыруды, цифрлық
платформаларды дамытуды, бизнеспен кооперацияны және зерттеушілердің
цифрлық дағдыларын нығайтуды қоса алғанда, ғылыми саланы цифрлық
трансформациялау бойынша тұтас мемлекеттік саясатты іске асыру
қажеттігін растайды. Алынған тұжырымдар жаһандық цифрлық трендтер
контекстінде Қазақстанның ғылыми-технологиялық және экономикалық даму
стратегияларын қалыптастыру үшін негіз бола алады.

{\bfseries Түйін сөздер:} цифрландыру, инфрақұрылым, ғылым, экономика,
инновация, инвестиция, жасанды интеллект.

\begin{header}
ЦИФРОВИЗАЦИЯ НАУКИ КАК ИНСТРУМЕНТ ПОВЫШЕНИЯ КОНКУРЕНТОСПОСОБНОСТИ ЭКОНОМИКИ РЕСПУБЛИКИ КАЗАХСТАН

Д.Ә. Турғанбаев\envelope,
Б.К. Исаева,
М.О. Кенжеғұл,
Д.А. Сейтхожина
\end{header}

\begin{affil}
Евразийский национальный университе им.Л.Н.Гумилева, Астана, Казахстан,

e-mail: dos\_25.97@mail.ru
\end{affil}

В современных условиях глобальной конкуренции и стремительно
развивающегося технологического развития цифровизация науки является
важным фактором повышения конкурентоспособности экономики. Цель
исследования -- выявить и количественно обосновать влияние цифровизации
науки на повышение цифровой конкурентоспособности экономики Казахстана.
Актуальность темы связана с необходимостью ускоренного перехода к
экономике знаний, где наука и технологии являются основными источниками
роста добавленной стоимости, развития высокотехнологичных производств и
привлечения инвестиций. Несмотря на наличие стратегий цифрового развития
и научной модернизации, Казахстан сталкивается с рядом ограничений:
фрагментированность цифровой инфраструктуры, слабая интеграция науки и
бизнеса, недостаточный уровень цифровых компетенций научных кадров. В
ходе исследования проведен сравнительный анализ опыта Республики Корея,
использованы методы эконометрического моделирования и разработан
интегральный индекс цифровизации науки. Модель позволила выявить
устойчивую положительную связь между уровнем цифровизации научной
инфраструктуры и позицией страны в глобальном рейтинге цифровой
конкурентоспособности. Установлено, что даже при умеренном росте
цифровизации возможно существенное усиление инновационного потенциала и
повышение эффективности научно-исследовательских и
опытно-конструкторских работ (НИОКР). Результаты подтверждают
необходимость реализации комплексной государственной политики по
цифровой трансформации научной сферы, включая автоматизацию процессов,
развитие цифровых платформ, сотрудничество с бизнесом и усиление
цифровых навыков исследователей. Полученные выводы могут служить основой
для формирования стратегий научно-технологического и экономического
развития Казахстана в контексте глобальных цифровых трендов.

{\bfseries Ключевые слова:} цифровизация, инфраструктура, наука, экономика,
инновация, инвестиция, искусственный интеллект.

\begin{multicols}{2}
{\bfseries Introduction.} The rationale for the research topic is related
to the increasing role of science as a key resource for sustainable
development and competitiveness of the economy in the context of the
transition to the digital economy and the knowledge economy. Modern
scientific systems require a revision of traditional approaches to the
organization of research, financing, technology transfer and interaction
with the industrial sector. The digitalization of science is becoming
not just a technological modernization, but a fundamentally new paradigm
of scientific activity, involving openness, interconnectedness,
automation, analytical traceability and rapid commercialization of
results.

The relevance of the topic is determined by both internal and external
factors. First, digital transformation is becoming a priority of
Kazakhstan' s state policy («Digital Kazakhstan»,
«Science, Technology and Innovation - 2023-2027») {[}1{]}. Secondly,
competition for scientific personnel, patents and innovations is
intensifying in the international arena, where the digital maturity of
the scientific sphere provides countries with leading positions. Despite
the steps taken in Kazakhstan, the scientific system remains fragmented
by digital initiatives, weak integration with business, and limited
access to global data and digital ecosystems.

The object of the research is the scientific system of Kazakhstan in the
context of digital transformation. The subject of the study is the
impact of the digitalization of science on the competitiveness of the
economy.

The aim is to quantify and reveal the mechanisms of the impact of the
digitalization of science on macroeconomic and institutional indicators
of competitiveness.

Research objectives:

- analyze the current level of digitalization of science in Kazakhstan;

- Conduct a comparative analysis with the Republic of Korea;

- to develop an integral index of digitalization of science;

- build a model of the link between digitalization and the
competitiveness of the economy;

- Develop practical policy recommendations.

The hypothesis of the study is that the systemic digitalization of
scientific infrastructure has a significant positive impact on the
growth of digital competitiveness and innovation potential of the
country.

Research methods: econometric analysis, factor analysis, cross-country
comparison, expert assessment, index construction. The novelty of the
work is to build a quantitative model of the relationship between the
digitalization of science and the macroeconomic index of digital
competitiveness, as well as to identify barriers and directions of
digital transformation of science in Kazakhstan {[}2{]}.

The digitalization of the scientific sphere currently acts not only as a
tool for modernizing the research infrastructure, but also as an
important factor in the formation of sustainable competitive advantages
of the national economy. The transition to the digital economy requires
the government not only to increase investments in science, but also to
qualitatively update the entire scientific system, including digital
document management, platforms for knowledge and technology transfer,
digital analytics and artificial intelligence in scientific research.
Kazakhstan, as a state in an active phase of socio-economic
transformation, sets itself the task of modernizing the scientific
system through the introduction of digital technologies that meet
international standards. In recent years, digital grant management
platforms, online reporting systems, and individual elements of digital
bibliometry have been emerging in the country, but the overall level of
digital maturity of the scientific environment remains low. This creates
obstacles to the integration of Kazakh science into the international
scientific space and reduces the speed of innovative economic
development. International experience, especially in South Korea, shows
that the consistent digitalization of science can become a catalyst for
economic growth, accelerate scientific research, strengthen technology
transfer and increase the share of high-tech products in gross domestic
product (GDP) {[}3{]}. In the context of global competition, Kazakhstan
needs to rethink the strategic role of science in a digital society and
identify key areas for its digital transformation, which requires not
only institutional but also econometric foundations. The purpose of this
article is to identify and quantify the impact of the digitalization of
science on the competitiveness of the economy of Kazakhstan, to
determine the key parameters and conditions for the implementation of
digital scientific transformation based on comparable international
experience.

{\bfseries Materials and methods.} The research is based on a comprehensive
interdisciplinary approach to assessing the degree of digitalization of
the scientific sphere and its impact on the competitiveness of the
economy of the Republic of Kazakhstan in the context of the
transformation of the global scientific and innovation agenda. Works on
the digital transformation of science, knowledge economics, as well as
models for assessing digital competitiveness (IMD, OECD) are used as a
theoretical and methodological basis. The empirical database includes
data from the Ministry of Science and Higher Education of the Republic
of Kazakhstan, the Committee on Statistics, the National Center for
Scientific and Technical Information, international sources (IMD, World
Bank, OECD, ADB), as well as digital indicators from open scientific
repositories {[}4,5{]}. Special attention is paid to materials on the
scientific system of the Republic of Korea as a reference example of
successful digitalization and effective integration of science into the
digital economy.

Methodologically, the work is based on a combination of methods of
comparative analysis, factor and index modeling, as well as elements of
regression analysis. The central task was to build a sound quantitative
model that would allow us to determine the degree of influence of the
digitalization of scientific infrastructure on the digital
competitiveness of the economy. For this purpose, the integrated index
of digitalization of science (IDN) was proposed, which includes such
parameters as the level of digitalization of scientific administration
(including automation of grant processes, reporting and accounting of
R\&D results), the share of scientific publications in the public
domain, the prevalence of digital skills among researchers, the
introduction of advanced digital solutions (artificial intelligence, big
data analysis, blockchain) in research activities, as well as the degree
of integration of scientific organizations into international and
national digital ecosystems. All indicators are normalized and
aggregated into an integral indicator that allows for cross-country
comparison and time analysis.

The constructed model is based on the premise that the digitalization of
the scientific sphere has a multiplicative effect on such key components
of digital competitiveness as innovation activity, technology transfer,
efficiency of scientific expenditures and attraction of investments in
the high-tech sector. To empirically verify the hypothesis put forward,
a model of the dependence of the digital competitiveness index
(according to the IMD methodology) on the integral index of
digitalization of science, the share of R\&D in GDP and the share of
joint projects of science and business was formed. The calculations were
based on panel data for 2018-2024. The model has been tested for
statistical significance, stability, and interpretability. Standard
regression analysis tools were used: coefficient determination,
remainder analysis, multicollinearity and autocorrelation testing. In
addition, expert assessments (the Delphi method) were used to verify
qualitative parameters of digitalization, such as the digital
competencies of researchers and the digital connectivity of scientific
institutions.

The study also applied comparative logic: the Republic of Korea was
chosen not only because of its leadership in the digitalization of
science, but also because of the structural comparability of the
scientific system with Kazakhstan in terms of the number of scientific
institutions, the share of budget funding and the role of the state in
the strategic management of science. This allows us to draw conclusions
relevant to the Kazakh context. Information materials about the Korean
scientific ecosystem are obtained from open sources (KISTI, NTIS,
statistics Ministry of Science and ICT). The program documents of both
countries were also analyzed, including the Digital Kazakhstan strategy,
the state program of scientific and technological development of the
Republic of Kazakhstan for 2023-2027, and the digital strategy of South
Korea in the field of science and innovation. For quantitative
comparison, such indicators as the share of open access to scientific
results (in Kazakhstan - 28\%, in Korea - 82\%), digitalization of
scientific administration (40\% and 95\%, respectively), the share of
researchers with advanced digital skills (in Kazakhstan - 22\%, in Korea
- 74\%) were used. the number of R\&D projects using AI and Big Data (in
Kazakhstan - less than 10\%, in Korea - over 35\%).

To achieve the objectives of the study, a set of methods was used:
comparative cross-country analysis, factorial and econometric analysis,
as well as methods for constructing indexes. The research is based on
statistical data from the Ministry of Science and Higher Education of
the Republic of Kazakhstan, the World Bank, the IMD (Institute for
Management Development), as well as analytical reports on science and
digitalization of South Korea {[}6{]}. An integral indicator of the
digitalization of science was developed, which includes the following
components: the availability of digital scientific infrastructure, the
degree of digitalization of management processes, the use of IT
solutions in R\&D, the level of digital literacy of researchers and the
share of scientific publications in the public domain. Next, a linear
regression model was built to analyze the impact of the digitalization
of science on the digital competitiveness index of the economy:

\[C = \beta_0 + \beta_1 D + \beta_2 R + \beta_3 B + \varepsilon \]

where:

- $C$ - Digital competitiveness index (IMD Digital Competitiveness);

- $D$ - science digitalization index (0--1);

- $R$ - R\&D expenditures (\% of GDP);

- $B$ - share of joint science and business projects (\% of all R\&D);

- $\varepsilon$ - an accidental mistake.

The data for 2024, presented in the table below, was used for
quantitative assessment.
\end{multicols}

\tcap{Table 1 - Comparison of indicators of the Republic of Kazakhstan and the Republic of Korea (2024)}
\begin{longtblr}[
  label = none,
  entry = none,
]{
  cells = {c},
  cells = {font = \small},
  cell{6}{1} = {c=3}{},
  hlines,
  vlines,
}
\textbf{Indicator}                              & \textbf{Republic of Kazakhstan} & \textbf{Republic of Korea}  \\
Science Digitalization Index (DDD)              & 0,45                            & 0,78                        \\
RD expenses (\% of GDP) (RRR)                   & 0,80                            & 4,90                        \\
The share of science-business projects (BBB)    & 12 \%                           & 35 \%                       \\
Digital Competitiveness Index (C)               & 34\textsuperscript{th} place    & 8\textsuperscript{th} place \\
\textit{Note. Compiled based on the source [7]} &                                 &                             
\end{longtblr}

\begin{multicols}{2}
{\bfseries Results and discussion.} The econometric and comparative
analysis revealed a number of important relationships, confirming that
the level of digitalization of the scientific sphere directly affects
key macroeconomic parameters, in particular the index of digital
competitiveness, innovation activity and R\&D productivity. The main
regression model showed that the coefficient for the variable
characterizing the digitalization of science (DDD) is 12.1, which
indicates a high sensitivity of competitiveness to even minor changes in
the level of digitalization. This means that the strategic expansion of
digital infrastructure and tools in the scientific field can ensure the
rapid development of not only the scientific sector itself, but also the
economy as a whole due to multiplier effects. Additionally, calculations
show that with the simultaneous improvement of all three indicators -
the level of digitalization, the volume of R\&D and the share of joint
projects with businesses - it is possible to achieve an increase in the
country' s digital competitiveness index by 4-6 points
over 3-5 years. This is significant given the current international
competition in the field of science and technology, where even one
rating point can affect investment flows and international cooperation.

In addition to the basic model, an assessment was made of the
differences in the implementation of digital scientific initiatives
between Kazakhstan and the Republic of Korea. The results are shown in
the table:
\end{multicols}

\tcap{Table 2 - Comparison of key areas of digitalization of science: Kazakhstan and the Republic of Korea in 2024}
\begin{longtblr}[
  label = none,
  entry = none,
]{
  width = \linewidth,
  colspec = {Q[262]Q[267]Q[412]},
  cells = {c},
  cells = {font = \small},
  cell{10}{1} = {c=3}{},
  hlines,
  vlines,
}
\textbf{The direction of digitalization} & \textbf{Republic of Kazakhstan}      & \textbf{Republic of Korea}                                \\
Unified Digital Science Platform         & Missing, using disparate ICS         & KISTI, NTIS – integrated platforms                        \\
Open access to data and publications     & Partial, non-integrated              & Full, through the state platform                          \\
Automating grant allocation              & Partially implemented (Science Fund) & Fully automated via KIAT                                  \\
Integration of science and business      & Low, 12\% of projects are joint      & High, more 35 \%                                          \\
The use of AI and Big Data in science    & Local pilot projects                 & At the level of national scientific programs              \\
Digital certification of results         & Practically absent                   & Mandatory for all RD                                      \\
Digital skills of scientific staff       & Limited, fragmentary training        & Integrated into PhD programs and Master's degree programs \\
Standards of open science                & They are not legally fixed           & Implemented and mandatory since 2020                      \\
\textit{Note – Developed by the authors} &                                      &                                                           
\end{longtblr}

\begin{multicols}{2}
As can be seen from the table, Kazakhstan' s lag is not
so much due to the lack of basic digital solutions, but rather to the
lack of a systemic strategy and coordination between participants in the
scientific ecosystem. South Korea, on the contrary, was able to build a
vertically integrated model of digital science, where each link - from
universities to government agencies and businesses -- operates within
the framework of a common architecture {[}8{]}. In Kazakhstan, such
elements exist in isolation and do not produce a synergistic effect.
However, it is important to note that the potential for digitalization
in the country remains high: the presence of technical universities,
digital hubs, government support for the IT sector and the strategies
«Digital Kazakhstan» and «Science, Technology and Innovation -
2023-2027» creates the necessary conditions for the transition to a
digital model of science {[}9{]}.

Thus, the main conclusions from the analysis are as follows: (1) the
digitalization of science has a quantifiably significant impact on the
economy and its competitiveness; (2) Kazakhstan' s
current position in the ratings and integrated digitalization indices
remains below average, but demonstrates growth potential.; (3) The gap
with developed countries is due to institutional and organizational
barriers, not technical impossibility; (4) with the implementation of a
holistic policy of digitalization of science, a short-term increase in
the country' s international position and an increase in
the effectiveness of science as an industry are possible. The expected
effect of improving digitalization is expressed not only in increasing
ratings, but also in economic benefits: according to the calculations of
the model, even with a conditional increase in the digitalization index
of science by 0.2 (for example, from 0.45 to 0.65), an increase in
innovation productivity is expected by 15-18\%, a reduction in
transaction costs by 20\% and an increase in technology transfer by
25-30\%. In the long term, this translates into an increase in the share
of knowledge-intensive industries in GDP, the formation of sustainable
competitive advantages and a reduction in dependence on the raw
materials sector.

Thus, the digitalization of science in Kazakhstan is not only a matter
of modernizing the research environment, but also a strategic vector of
transformation of the entire economy. It is necessary to accelerate the
implementation of digital solutions, synchronize policies in science,
education and innovation, as well as enhance international scientific
cooperation through digital tools.

{\bfseries Conclusion.} At the present stage, the digitalization of science
in the Republic of Kazakhstan is becoming an integral part of the
formation of a competitive economy focused on innovative development and
integration into the global scientific and technological space. The
results of the study confirm that the level of digitalization of the
scientific sphere has a significant impact on macroeconomic indicators,
including the country' s digital competitiveness index,
R\&D productivity, innovation activity and investment attractiveness of
high-tech industries {[}10{]}. However, the current level of digital
maturity of Kazakh science is assessed as insufficient: there is no
single digital platform covering all levels of scientific activity,
there is a low share of digital services in the science management
system, digital analytics and automation of scientific knowledge
assessment and transfer processes are poorly developed. Additionally,
structural problems have been identified, such as weak integration of
science and business, limited access to international digital databases,
and the lack of digital mechanisms for certification and verification of
research results {[}11{]}.

These barriers reduce the effectiveness of public investment in science
and hinder the country' s innovation potential. To
overcome these problems, it is necessary to implement a comprehensive
strategy for the digitalization of science, including legislative
support for digital transformation, the development of a unified
scientific digital ecosystem, the introduction of open science
standards, digital identifiers, and ensuring the cybersecurity of
scientific data. A key focus should be the stimulation of intersectoral
cooperation -- between science, business and government - through
digital platforms for joint research and development, the use of
blockchain and AI technologies to manage scientific cycles and the
introduction of digital KPIs to assess scientific effectiveness
{[}12{]}. In addition, it is necessary to develop digital competencies
among all participants in the scientific process: researchers, teachers,
managers and experts, through targeted educational programs and digital
acceleration. Only in the case of a systematic approach, the
digitalization of science can become a powerful catalyst for the
transition to a knowledge economy and ensure
Kazakhstan' s sustainable competitive advantage in the
context of the fourth industrial revolution and global technological
competition.
\end{multicols}

\begin{center}
{\bfseries Литература}
\end{center}

\begin{refs}
1. Сембин А.Б. Управление проектами в условиях цифровой трансформации
Казахстана.-2021.- № 3(91). - С.229-234. DOI
10.46914/1562-2959-2021-1-3-229-234.

2. Dauliyeva G., Islamoglu M., Yeraliyeva A. Digital Economy as a Factor
of Sustainable Development Goals Progression in Kazakhstan. //Farabi
Journal of Social Sciences/ economics \& management-2023.- Vol 9(2). -
P.98-105. \href{https://DOI}{DOI} 10.26577/FJSS.2023.v9.i2.010.

3. Technology and Innovation Outlook 2023//OECD. -2023.- P.1-227. DOI
10.1787/0b55736e-en.

4. Sadyrova M., Yusupov K., Imanbekova B. Innovation processes in
Kazakhstan: development factors //Journal of Innovation and
Entrepreneurship. -2021.-Vol.10(1).-P.1-13. DOI
10.1186/s13731-021-00183-3.

5. Yeraliyeva A., Dauliyeva G., Andabayeva G., Nurmanova B.
Effectiveness of public administration of the digital economy in
Kazakhstan. Problems and Perspectives in Management. -2023.-Vol.
21(3).-P.125-137. DOI 10.21511/ppm.21(3).2023.10.

6. Wijaya J.R.T.,Manurung M. R.A.The Impact of Digitalization on
Financial Accounting Practices: A Literature Review in the Scopus
Database//
\href{https://www.researchgate.net/journal/Review-of-Applied-Accounting-Research-RAAR-2807-8969?_tp=eyJjb250ZXh0Ijp7ImZpcnN0UGFnZSI6InB1YmxpY2F0aW9uIiwicGFnZSI6InB1YmxpY2F0aW9uIn19}{Review
of Applied Accounting Research (RAAR)} -2025.-Vol.5(1).-P.53-70. DOI
10.30595/raar.v5i1.26345.

7. Turarova A.M. Dabyltayeva N.E., Ruziyeva E.A. et al. Unlocking
Intersectoral Integration in Kazakhstan's Agro-Industrial Complex:
Technological Innovations, Knowledge Transfer, and Value Chain
Governance as Predictors// Economies. -2023.-Vol.11(8):211
DOI 10.3390/economies11080211.

8. Gafu G., D.~Hernández‑Torrano, N.Terlikbayeva, A.~Zhanseitova.
Mapping the Landscape of SDG Research in Kazakhstan: A Machine
Learning--Based Approach// Journal of the Knowledge Economy.-2024.-
Vol.16(5).-P.15879-15904.DOI 10.1007/s13132-024-02543-2.

9. Shevyakova A., Petrenko Y., Daribekova A., Daribekova N. Features and
public financing of digitalization and E-Government: The case of
Kazakhstan // Journal of Infrastructure, Policy and Development.
-2024.-Vol.8(5): 3074. DOI 10.24294/jipd.v8i5.3074.

10. Хамраева Р.А., Сыздыкбаева К.Г., Омаров Г.Б. Влияние цифровых
технологий на экономику Казахстана//Международный журнал информационных
и коммуникационных технологий.- 2021.- Vol.1(1).-P.211-214. DOI
10.54309/IJICT.2020.1.1.068.

11. Karipbayev B.I., Zhakin S.M., Seifullina G.R. Kazakhstan's
Digitalization Format: Identity and Future.//RUDN Journal of
Philosophy.-2025.-Vol 29(2)-P.535-547. DOI
10.22363/2313-2302-2025-29-2-535-547.

12. Denissova O., Konurbayeva Z., Kulisz M., Yussubaliyeva M.,
Suieubayeva S. Measuring the Digital Economy in Kazakhstan: From Global
Indices to a Contextual Composite Index
(IDED)//Economies.-2025.-Vol.13(8):225 DOI 10.3390/economies13080225.
\end{refs}

\begin{center}
{\bfseries References}
\end{center}

\begin{refs}
1. Sembin A.B. Upravlenie proektami v uslovijah cifrovoj transformacii
Kazahstana.-2021.- № 3(91). - S.229-234. DOI
10.46914/1562-2959-2021-1-3-229-234. {[}in Russian{]}

2. Dauliyeva G., Islamoglu M., Yeraliyeva A. Digital Economy as a Factor
of Sustainable Development Goals Progression in Kazakhstan. //Farabi
Journal of Social Sciences/ economics \& management-2023.- Vol 9(2). -
P.98-105. \href{https://DOI}{DOI} 10.26577/FJSS.2023.v9.i2.010.

3. Technology and Innovation Outlook 2023//OECD. -2023.- P.1-227. DOI
10.1787/0b55736e-en.

4. Sadyrova M., Yusupov K., Imanbekova B. Innovation processes in
Kazakhstan: development factors //Journal of Innovation and
Entrepreneurship. -2021.-Vol.10(1).-P.1-13. DOI
10.1186/s13731-021-00183-3.

5. Yeraliyeva A., Dauliyeva G., Andabayeva G., Nurmanova B.
Effectiveness of public administration of the digital economy in
Kazakhstan. Problems and Perspectives in Management. -2023.-Vol.
21(3).-P.125-137. DOI 10.21511/ppm.21(3).2023.10.

6. Wijaya J.R.T.,Manurung M. R.A.The Impact of Digitalization on
Financial Accounting Practices: A Literature Review in the Scopus
Database//
\href{https://www.researchgate.net/journal/Review-of-Applied-Accounting-Research-RAAR-2807-8969?_tp=eyJjb250ZXh0Ijp7ImZpcnN0UGFnZSI6InB1YmxpY2F0aW9uIiwicGFnZSI6InB1YmxpY2F0aW9uIn19}{Review
of Applied Accounting Research (RAAR)} -2025.-Vol.5(1).-P.53-70. DOI
10.30595/raar.v5i1.26345.

7. Turarova A.M. Dabyltayeva N.E., Ruziyeva E.A. et al. Unlocking
Intersectoral Integration in Kazakhstan's Agro-Industrial Complex:
Technological Innovations, Knowledge Transfer, and Value Chain
Governance as Predictors// Economies. -2023.-Vol.11(8):211
DOI 10.3390/economies11080211.

8. Gafu G., D.~Hernández‑Torrano, N.Terlikbayeva, A.~Zhanseitova.
Mapping the Landscape of SDG Research in Kazakhstan: A Machine
Learning--Based Approach// Journal of the Knowledge Economy.-2024.-
Vol.16(5).-P.15879-15904.DOI 10.1007/s13132-024-02543-2.

9. Shevyakova A., Petrenko Y., Daribekova A., Daribekova N. Features and
public financing of digitalization and E-Government: The case of
Kazakhstan // Journal of Infrastructure, Policy and Development.
-2024.-Vol.8(5): 3074. DOI 10.24294/jipd.v8i5.3074.

10. Hamraeva R.A., Syzdykbaeva K.G., Omarov G.B. Vlijanie cifrovyh
tehnologij na jekonomiku Kazahstana//Mezhdunarodnyj zhurnal
informacionnyh i kommunikacionnyh tehnologij.- 2021.-
Vol.1(1).-P.211-214. DOI 10.54309/IJICT.2020.1.1.068. {[}in Russian{]}

11. Karipbayev B.I., Zhakin S.M., Seifullina G.R. Kazakhstan's
Digitalization Format: Identity and Future.//RUDN Journal of
Philosophy.-2025.-Vol 29(2)-P.535-547. DOI
10.22363/2313-2302-2025-29-2-535-547.

12. Denissova O., Konurbayeva Z., Kulisz M., Yussubaliyeva M.,
Suieubayeva S. Measuring the Digital Economy in Kazakhstan: From Global
Indices to a Contextual Composite Index
(IDED)//Economies.-2025.-Vol.13(8):225 DOI 10.3390/economies13080225
\end{refs}

\begin{info}
\hspace{1em}\emph{{\bfseries Information about the authors}}

Turganbayev D.A. - PhD student, L.N. Gumilyov Eurasian National
University, Astana, Kazakhstan, e-mail:
dos\_25.97@mail.ru;

Isayeva B.K. - associate professor, L.N. Gumilyov Eurasian National
University, Astana, Kazakhstan, e-mail:
issayeva\_bk@enu.kz;

Keznhegul M.O. - PhD student, L.N. Gumilyov Eurasian National
University, Astana, Kazakhstan, e-mail:
manshuk.kenzhegul@mail.ru;

Seithozhina D.A. - candidate of economic sciences, associate professor,
L.N. Gumilyov Eurasian National University, Astana, Kazakhstan, e-mail:
jaseit@mail.ru.

\hspace{1em}\emph{{\bfseries Сведения об авторах}}

Турганбаев Д.А.- докторант PhD, Евразийский национальный университет
имени Л.Н. Гумилева, Астана, Казахстан, e-mail:
dos\_25.97@mail.ru;

Исаева Б.К. - ассоциированный профессор, Евразийский национальный
университет имени Л.Н. Гумилева, Астана, Казахстан, e-mail:
issayeva\_bk@enu.kz;

Кенжегул М.О. - докторант PhD., Евразийский национальный университет
имени Л.Н. Гумилева, Астана, Казахстан, e-mail:
manshuk.kenzhegul@mail.ru;

Сейтхожина Д.А. - кандидат экономических наук, ассоциированный
профессор, Евразийский национальный университет имени Л.Н. Гумилева,
Астана, Казахстан, e-mail:
jaseit@mail.ru.
\end{info}
