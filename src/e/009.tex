\id{МРНТИ 06.52.13}{}

\begin{header}
\swa{}{ҚАЗАҚСТАНДАҒЫ БИЗНЕСТІҢ ӘЛЕУМЕТТІК ЖАУАПКЕРШІЛІГІ: ҚАЗІРГІ ЖАҒДАЙЫ ЖӘНЕ ДАМУ БАҒЫТТАРЫ}

С.Ж. Ибраимова,
С.Б. Касымова\envelope,
Г.Ж. Каримбаева,
Д.Б. Шарипова
\end{header}

\begin{affil}
Қ.Құлажанов атындағы Қазақ технология және бизнес университеті, Астана, Қазақстан

\corrauthor{Корреспондент-автор: Sanim\_81@list.ru}
\end{affil}

Бұл мақалада Қазақстан Республикасындағы корпоративтік әлеуметтік
жауапкершіліктің институционалдық негіздерінің қалыптасуы мен эволюциясы
қарастырылады. Зерттеу нысаны - әртүрлі көлемдегі кәсіпкерлік
құрылымдар-шағын, орта және ірі кәсіпорындар. Талдау пәні әлеуметтік
жауапкершілікті жүзеге асырудың ұйымдастырушылық-экономикалық
аспектілері мен практикалық құралдары болып табылады. Зерттеудің мақсаты
- корпоративтік әлеуметтік жауапкершілікті қалыптастырудың заманауи
тенденцияларын зерделеу, бар жүйелік проблемаларды анықтау, сондай-ақ
қазіргі экономика жағдайында тұрақты әлеуметтік жауапкершілік институтын
дамытудың стратегиялық бағыттарын анықтау. Әдістемелік негіз
аналитикалық, статистикалық, есептеу-аналитикалық, диалектикалық және
іздеу әдістерінен тұрды. Зерттеу нәтижелері бизнестің әлеуметтік
мәселелерді шешуге және қоғамның дамуына қатысуы қалыптасу сатысында
екенін көрсетеді. Әлеуметтік жауапкершіліктің ең белсенді тетіктерін ірі
компаниялар пайдаланады, ал мемлекет оларды жүзеге асыруда маңызды рөл
атқарады. Сонымен қатар, әлеуметтік жауапкершілік саласындағы
халықаралық стандарттарға сәйкестік деңгейі жеткіліксіз болып қалуда.
Менеджменттің халықаралық стандарттарын енгізуге қарамастан, олардағы
әлеуметтік компонент көбінесе кішігірім позицияларда болады. Әлеуметтік
жауапкершіліктің қолданыстағы тетіктері әлеуметтік саланы және адам
әлеуетін толыққанды дамыту үшін қажетті жағдайларды қамтамасыз етпейді

{\bfseries Түйін сөздер:} әлеуметтік жауапкершілік, бизнес ұйымдары,
әлеуметтік инвестициялар, нарық институттары, халықаралық стандарттар,
әлеуметтік сектор, бизнес-ұйымдар.

\begin{header}
СОЦИАЛЬНАЯ ОТВЕТСТВЕННОСТЬ БИЗНЕСА В КАЗАХСТАНЕ: СОВРЕМЕННОЕ СОСТОЯНИЕ И НАПРАВЛЕНИЯ РАЗВИТИЯ

С.Ж. Ибраимова,
С.Б. Касымова\envelope,
Г.Ж. Каримбаева,
Д.Б. Шарипова
\end{header}

\begin{affil}
Казахский университет технологии и бизнеса им. К.Кулажанова, Астана, Казахстан,

e-mail: Sanim\_81@list.ru
\end{affil}

В данной статье рассматривается формирование и эволюция
институциональных основ корпоративной социальной ответственности в
Республике Казахстан. Объектом исследования являются предпринимательские
структуры различных размеров - малые, средние и крупные предприятия.
Предметом анализа являются организационно-экономические аспекты и
практические инструменты реализации социальной ответственности. Цель
исследования - изучить современные тенденции формирования корпоративной
социальной ответственности, выявить существующие системные проблемы, а
также определить стратегические направления развития устойчивого
института социальной ответственности в условиях современной экономики.
Методологическую основу составили аналитический, статистический,
расчетно-аналитический, диалектический и поисковый методы. Результаты
исследования показывают, что участие бизнеса в решении социальных
проблем и развитии общества находится на стадии становления. Наиболее
активно механизмы социальной ответственности используют крупные
компании, при этом государство продолжает играть значительную роль в их
реализации. В то же время уровень соответствия международным стандартам
в области социальной ответственности остается недостаточным. Несмотря на
внедрение международных стандартов менеджмента, социальная составляющая
в них зачастую оказывается на второстепенных позициях. Существующие
механизмы социальной ответственности не обеспечивают необходимых условий
для полноценного развития социальной сферы и человеческого потенциала

{\bfseries Ключевые слова:} социальная ответственность, бизнес-организации,
социальные инвестиции, институты рынка, международные стандарты,
социальный сектор, бизнес-организации.

\begin{header}
SOCIAL RESPONSIBILITY OF BUSINESS IN KAZAKHSTAN: CURRENT STATE AND DIRECTIONS OF DEVELOPMENT

C.Zh. Ibraimova,
C.B. Kassymova\envelope,
G.Zh. Karimbayeva,
D.B. Sharipova
\end{header}

\begin{affil}
K.Kulazhanov Kazakh University of Technology and Business, Astana, Kazakhstan,

e-mail: Sanim\_81@list.ru
\end{affil}

This article examines the formation and evolution of the institutional
foundations of corporate social responsibility in the Republic of
Kazakhstan. The object of the study is business structures of various
sizes - small, medium and large enterprises. The subject of the analysis
is organizational and economic aspects and practical tools for
implementing social responsibility. The purpose of the study is to study
current trends in the formation of corporate social responsibility,
identify existing systemic problems, and identify strategic directions
for the development of a sustainable institution of social
responsibility in a modern economy. The methodological basis consists of
analytical, statistical, computational and analytical, dialectical and
search methods. The results of the study show that business
participation in solving social problems and developing society is in
its infancy. Large companies use social responsibility mechanisms most
actively, while the state continues to play a significant role in their
implementation. Large companies use social responsibility mechanisms
most actively, while the state continues to play a significant role in
their implementation. At the same time, the level of compliance with
international standards in the field of social responsibility remains
insufficient. Despite the introduction of international management
standards, the social component in them often finds itself in secondary
positions. The existing mechanisms of social responsibility do not
provide the necessary conditions for the full development of the social
sphere and human potential

{\bfseries Keywords:} social responsibility, business organizations, social
investments, market institutions, inter\-national standards, social
sector, business organizations

\begin{multicols}{2}
{\bfseries Кіріспе.} Нарықтық экономиканың жұмыс істеуі мен дамуы
жағдайындағы бизнес - ұйымдардың әлеуметтік жауапкершілігі өз кезегінде
экономиканың нақты секторының іргетасы болып табылатын әлеуметтік саланы
дамытудың маңызды құралы болып табылады. Әлеуметтік салада адами
капиталдың қалыптасуы мен дамуы қалыптасады, сондай-ақ білім беру,
денсаулық сақтау сияқты экономиканың маңызды салаларын дамыту үшін
жағдайлар жасалады. Индустриялық-инновациялық трендтерді ескере отырып,
қазіргі нарықтық жағдайларда экономиканың әлеуметтік секторын дамыту мен
нығайту тек мемлекеттің қатысуы шеңберінде ғана емес, сонымен қатар
бизнес-кәсіпкерліктің қатысуымен де жүргізілуі тиіс. Бизнес-ұйымдардың
әлеуметтік жауапкершілігінің тиімді тетіктерін құру мемлекеттің
экономикалық жүйесінің дамудың постиндустриалды кезеңіне қарқынды ауысуы
үшін кешенді жағдайлар жасайды.

Қазақстан Республикасында әлеуметтік саланы дамыту үшін мемлекет
тарапынан бизнес-ұйымдардың әлеуметтік жауапкершілігі қағидаттарында
бірқатар ұйымдастыру шаралары қабылданды: елдің Біріккен Ұлттар Ұйымына
мүшелігі, еңбек мәселелерін, жалақыны, жалпымемлекеттік табиғи
ресурстарды пайдалануды реттейтін заңнамалық-нормативтік құжаттардың кең
спектрін қабылдау.

Сонымен қатар, әлеуметтік жауапкершіліктің тұрақты нарықтық институтын
қалыптастыру саласында айтарлықтай проблемалар бар, атап айтқанда ірі
өнеркәсіптік кәсіпорындар тарапынан әлеуметтік салаға инвестициялардың
циклділігі мен тұрақсыздығы, әлеуметтік жауапкершіліктің халықаралық
стандарттарын игерудің төмен тенденциялары, шағын және орта бизнестің
әлеуметтік жауапкершілікке қатысу деңгейінің төмендігі.

Бизнес-ұйымдардың әлеуметтік жауапкершілігін дамыту жолдарын іздеу
мақсатында экономиканың нақты секторында жүргізілетін әлеуметтік
саясаттың барлық аспектілерін терең талдау талап етіледі. Келешекте
бизнес-ұйымдардың әлеуметтік жауапкершілігінің орнықты нарықтық
институтын қалыптастыру экономиканың нақты секторы мен әлеуметтік
саланың интеграцияланған өзара іс-қимылын, Қазақстан экономикасының
постиндустриалды жұмыс істеу кезеңіне көшуінің берік іргетасын құруға
мүмкіндік береді.

Осы зерттеудің мақсаты - бизнес-ұйымдардың әлеуметтік жауапкершілігін
қалыптастырудың заманауи үрдістерін талдау, әлеуметтік жауапкершіліктің
тұрақты нарықтық институтын құрудың жүйелі проблемалары мен стратегиялық
басымдықтарын анықтау.

Зерттеудің ғылыми жаңалығы - бизнестің әлеуметтік жауапкершілік
тенденцияларының сандық және сапалық жақтарын қамтитын талдаудың кешенді
әдіснамасын қолдану. Сандық талдау бизнес ұйымдарының динамикасы,
әлеуметтік жауапкершілікке салынған пайда мен инвестициялардың
динамикасы, халықаралық әлеуметтік жауапкершілік стандарттарын
тәжірибеге енгізудің индикативті көрсеткіштері сияқты деректерді
зерттеуге негізделген. Сапалы зерттеулер заңдар мен нормативтік -
құқықтық актілерге, әлеуметтік жауапкершілікке инвестициялар
басымдықтарының құрылымына, оны іс жүзінде жүзеге асырудың
ұйымдастырушылық-экономикалық механизміне бағытталған.

{\bfseries Материалдар мен әдістер.} Бизнестің әлеуметтік жауапкершілігі
тұжырымдамаларының тарихи дамуын талдау 1970 жылдардың басында негізгі
Тұжырымдаманың тек теориялық құндылығы бар екендігі белгілі болды,
өйткені оның ережелері іс жүзінде толық қолданыла алмайды. Тиісінше,
әлеуметтік жауапкершілік қағидаттарын жеке субъектілер деңгейінде
практикалық іске асыру мәселелері әлеуметтік жауапкершіліктің
философиялық сипатындағы мәселелермен алмастырылды, ал қоғамды
әлеуметтік ісшаралар мен бағдарламаларды өткізу нәтижелері туралы
ақпараттандыру мәселелері корпоративтік әлеуметтік сезімталдық
тұжырымдамасы аясында ерекше өзектілікке ие болды {[}1{]}. Корпоративтік
әлеуметтік жауапкершілік - бұл, ең алдымен, ұйымның қоршаған орта мен
қоғам үшін жауапкершілігі, ол бизнестің бәсекеге қабілеттілігімен
байланысты және жай ғана байланысты емес, компаниялардың бәсекеге
қабілеттілігіне айтарлықтай әсер етеді {[}2{]}.

Қазіргі ғылыми салада әлеуметтік жауапкершілік жеке адамдар мен
ұйымдардың шешім қабылдау процесінде қалыптасатын этикалық қағидат
(мәртебе) ретінде қарастырылады. Бұл этикалық қағидат шешім қабылдау
кезінде осы шешімдерді қабылдаған экономика субъектілерінің мүдделері
ғана емес, сонымен қатар кең әлеуметтік топтардың, жалпы қоғамның
мүдделері де ескерілетінін болжайды {[}3,4{]}.

Әлеуметтік жауапкершілік тұрақты даму шеңберіндегі кәсіпорындардың
экономикалық қызметінің нәтижелері мен мазмұнына мүдделі қоғамның
сұраныстарына жауап ретінде дамиды. Сондай-ақ, қазіргі қоғамда
әлеуметтік процестер меншік иелері үшін де, қарапайым тұтынушылар үшін
де инвестициялық шешімдер қабылдауда маңызды рөл атқаратынын ескерген
жөн; экономикалық қызметтен туындаған қоршаған ортаның ластануының
артуына алаңдаушылық артып келеді; бұқаралық ақпарат құралдары мен
ақпараттық-коммуникациялық технологиялардың көмегімен кәсіпкерлік
қызметтің ашықтығы үшін қажетті жағдайлар жасалады {[}5{]}.

Бизнес-ұйымдардың әлеуметтік жауапкершілігі - кәсіпорындар, фирмалар,
компаниялар, сондай-ақ басқа да бизнес-құрылымдар өздерінің
өндірістік-шаруашылық қызметі процесінде қоғамдық саланың тараптарына
әсер етудің әртүрлі нысандары үшін жауапкершілікті өзіне жүктей отырып,
қоғамның мүдделерін ескеретін тұжырымдама {[}6,7{]}.

Корпоративтік әлеуметтік жауапкершілік тұжырымдамасы корпорациялардың
қалыптасуы мен даму тенденцияларына параллель ұзақ кезеңнен өтті
{[}8{]}.

Корпоративтік әлеуметтік жауапкершілік теориясын қалыптастыру процесінде
қарама-қайшылықтар пайда болды, олар келесі атауларға ие болды:
«корпоративті эгоизм» (либералды тәсіл), «ақылға қонымды эгоизм»,
«корпоративті альтруизм».

Орталық Азия елдерінің ішінде Қазақстан Республикасы әлеуметтік
жауапкершілік саласындағы көшбасшы болып табылады. Елімізде әлеуметтік
жауапкершілікті дамытудың алғашқы алғышарттары 1990 жылдардың ортасында
өздерінің әлеуметтік жауапкершілігін көрсеткен шетелдік компаниялардың
нарыққа шығуымен пайда болды. Алайда, бұған қарамастан, осыдан бес жыл
бұрын корпоративтік әлеуметтік тұжырымдамасы Қазақстанда салыстырмалы
түрде жаңа болып саналды. Бүгінгі таңда отандық компаниялардың
әлеуметтік жауапкершіліктің негізгі принциптері туралы хабардарлығы тез
өсуде {[}9,10{]}.

Бизнестің әлеуметтік жауапкершілігі концепциясын интерпретациялауда,
алғашқы және анағұрлым дәстүрлі болып саналатын Милтон Фридманның
теориясы. Оның пікірінше, бизнестің жауапкершілігі негізінде акционерлер
өздерінің пайдасын өсіруі тиіс {[}11{]}.

Шетелдік ғалымдардың бизнес-ұйымдардың әлеуметтік жауапкершілік
теориясына ғылыми көзқарастарын зерттеу әлеуметтік жауапкершілік
эволюциялық-тарихи сипатта болатын және уақыт өте келе үнемі дамып
отыратын күрделі-жүйеленген құбылыс екенін көрсетеді. Уақыт өте келе
әлеуметтік жауапкершілікті экономист ғалымдар бизнестің логикалық
процестеріне қайшы келетін құбылыс ретінде емес, табыс пен пайда алуға
оң әсер ету құралы, кәсіпкерлік қызметтің қажетті және міндетті
компоненті ретінде қарастыра бастады.

Сондай-ақ, бизнес-ұйымдардың әлеуметтік жауапкершілігін заманауи
зерттеулер тұжырымдамалық аппаратқа және бұл құбылысты бизнес пен
кәсіпкерліктің даму тенденцияларына, шағын, орта және ірі
кәсіпорындардың, фирмалар мен компаниялардың қызметіне әсер ететін
фактор ретінде қарастыруға көп көңіл бөледі деген қорытындыға келу
маңызды. Бұл зерттеудің басқалардан басты айырмашылығы - бизнестің
әлеуметтік жауапкершілігі экономиканың нақты секторына әсер ететін
фактор ретінде ғана емес, сонымен қатар мемлекет, экономиканың
өнеркәсіптік секторы, шағын және орта бизнес қатысатын институционалдық
құбылыс ретінде қарастырылады. Бұл экономика субъектілері әлеуметтік
жауапкершілікті іс жүзінде іске асырудың нарықтық нормалары мен
ережелерін қалыптастырады, олар тек еркін нарықтық форматта ғана емес,
сонымен қатар басымдықтары, мақсаттары, міндеттері нақты құрылған
басқарылатын күйде болуы керек.

Әлеуметтік жауапкершілікке ғылыми көзқарастарды зерттеу қазақстандық
практика үшін бизнестің әлеуметтік жауапкершілігінің маңыздылығын
анықтауға мүмкіндік береді. Біздің ойымызша, Қазақстан Республикасының
экономикасында бизнес-ұйымдардың әлеуметтік жауапкершілігі - бұл
институционалдық құбылыс, ұлттық аумақтық шекаралар шегінде (өңірлер,
тұтастай алғанда ел) және олар үшін ішкі және сыртқы адам ресурстарына
шоғырланған әлеуметтік саясатты әзірлеуге бағдарланған ұйымдардың
топ-менеджментінің этикалық мінез-құлқын қалыптастырудың серпінді
процесі дәстүрлі қоғамдық қажеттіліктерге де, қоғамдық құндылықтарды
өзгертуге де бағытталған шектеулер.

Ғылыми зерттеу аясында бизнестің әлеуметтік жауапкершілік
тенденцияларының сандық және сапалық жақтарын қамтитын талдаудың кешенді
әдістемесі қолданылды. Сандық талдау бизнес ұйымдарының динамикасы,
әлеуметтік жауапкершілікке салынған пайда мен инвестициялардың
динамикасы, халықаралық әлеуметтік жауапкершілік стандарттарын
тәжірибеге енгізудің индикативті көрсеткіштері сияқты деректерді
зерттеуге негізделген.

Зерттеудің негізгі әдістері: аналитикалық және синтетикалық,
статистикалық, есептеу-аналитикалық зерттеу әдістері. Зерттеудің
аналитикалық әдісі шеңберінде бизнес - ұйымдардың әлеуметтік
жауапкершілігінің экономикалық және ұйымдастырушылық негіздерінің барлық
құрамдас элементтері, атап айтқанда: бизнес-ұйымдардың динамикасы және
олардың құрылымы; бизнес-ұйымдардың кірістерінің динамикасы; бизнестің
әлеуметтік жауапкершілігін реттейтін заңдар мен нормативтік-құқықтық
құжаттар жеке талданды; бизнес-ұйымдар тарапынан әлеуметтік
жауапкершілікке инвестициялардың бағыттары және олардың құрылымы;
өңірлік экономикалық жүйелер практикасында қолданылатын өнеркәсіптік жер
қойнауын пайдаланушы кәсіпорындардың әлеуметтік жауапкершілігінің
ұйымдастырушылық-экономикалық тетігі; менеджменттің халықаралық
стандарттарының жұмыс істеуі шеңберінде әлеуметтік жауапкершілікті
қалыптастыру критерийлері; әлеуметтік салық және әлеуметтік сақтандыру
жөніндегі кәсіпорындардың аударымдарының серпіні; атаулы және нақты
орташа жалақының серпіні. Қолданылатын әдіснама шеңберінде 2020 жылдан
2024 жылға дейінгі стратегиялық кезеңдегі деректер пайдаланылды.
Практикалық деректердің шектеулілігін ескере отырып, индикативті
көрсеткіштердің белгілі бір кешені таңдалған стратегиялық кезеңнен тыс
зерттелді.

Аналитикалық әдіс Қазақстан Республикасындағы әлеуметтік жауапкершілік
институтының жұмыс істеуінің ағымдағы жай-күйі туралы кешенді пайымдауды
қалыптастыруға мүмкіндік берді. Қолданылған әдіснаманың негізінде
бизнестің әлеуметтік жауапкершілік институтының қалыптасуы мен дамуының
қазіргі кезеңіне ғылыми негіздеме берілді.

{\bfseries Нәтижелер және талқылау.} Әлеуметтік жауапкершілік және оның
Қазақстан Республикасындағы тұжырымдамалық негіздері мемлекеттік
деңгейде де, шағын және орта бизнестің, ірі кәсіпорындардың,
фирмалардың, компаниялардың қызметінде де қаланды. Қазақстан
Республикасында «Мемлекет пен бизнестің әлеуметтік жауапкершілігін
қалыптастыру орнықты дамудың базалық шарттарының бірі болып
табылатынына» басты назар аударды. Осы қағида негізінде Қазақстан
Республикасы 2050 жылға қарай әлемнің дамыған 30 елінің қатарына кіруі
тиіс {[}12{]}.

Қазақстан Республикасының Біріккен Ұлттар Ұйымының (БҰҰ) халықаралық
ұйымына мүшелігі елдің бизнес-қоғамдастығын өз қызметінде әлеуметтік
жауапкершіліктің үш іргелі қағидатын басшылыққа алуға бағыттайды және
шақырады: адам құқықтарын сақтау саласындағы негізгі қағидаттарды іске
асыру; еңбек қатынастары нормаларын орындау; қоршаған ортаны қорғау.

Қазақстан Республикасында кәсіпкерлік бизнес-ортада шағын, орта және ірі
бизнес субъектілерінің қызметі ұйымдық негіздердің бірі болып табылады.

Шағын және орта бизнес саласында 1-кестеге сәйкес Қазақстан
Республикасында кәсіпкерлік субъектілерінің оң серпіні байқалады,
олардың арасында жеке кәсіпкерлер, шағын кәсіпорындар басым. Қазақстан
нарығында салыстырмалы түрде аз мөлшерде (1 және 2-кестелер) мөлшері
бойынша орташа кәсіпорындар, сондай-ақ ұйымдық-құқықтық мәртебесі
акционерлік қоғамдар болып табылатын кәсіпкерлік бизнес-құрылымдар
ұсынылған.
\end{multicols}

\tcap{1-кесте. Шағын және орта бизнес бөлінісіндегі Қазақстанның бизнес-ұйымдарының динамикасы}
\begin{longtblr}[
  label = none,
  entry = none,
]{
  width = \linewidth,
  colspec = {Q[600]Q[69]Q[63]Q[63]Q[63]Q[63]},
  cells = {c},
  cells = {font = \small},
  cell{1}{1} = {r=2}{},
  cell{1}{2} = {c=5}{0.321\linewidth},
  cell{8}{1} = {c=6}{0.928\linewidth},
  vlines,
  hline{1,3-9} = {-}{},
  hline{2} = {2-6}{},
}
\textbf{Көрсеткіштің атауы}                                                             & \textbf{Мәні, мың бірлік} &                                 &                                 &                                 &                                 \\
                                                                                      & \textbf{2020}             & \textbf{2021}                   & \textbf{2022}                   & \textbf{2023}                   & \textbf{2024}                   \\
Шағын және орта бизнес-кәсіпкерлік субъектілерінің жалпы саны, оның ішінде:             & 1610,4                    & 1694,7                          & 2026,5                          & 2178,9                          & 2262,4                          \\
- шағын бизнес кәсіпорындары                                                            & 398,8                     & 416,1                           & 438,4                           & 456,6                           & 464,8                           \\
- орта бизнес кәсіпорындары                                                             & 2,6                       & 2,9                             & 3,0                             & 3,0                             & 3,2                             \\
- жеке кәсіпкерлер                                                                      & 983,5                     & 1044,3                          & 1336,5                          & 1442,8                          & 1523,0                          \\
- шаруа қожалықтары                                                                     & 225,4                     & 231,4                           & 248,6                           & 276,4                           & 271,3                           \\
\textit{Ескерту - авторлар Bureau of National Statistics дереккөзі бойынша құрастырған} &                             &                                 &                                 &                                 &                                 \\
\end{longtblr}

\tcap{2-кесте.2024 жылға арналған меншіктің ұйымдық-құқықтық нысандары бөлінісінде Қазақстан Республикасы бизнес-ұйымдарының құрылымы}
\begin{longtblr}[
  label = none,
  entry = none,
]{
  width = \linewidth,
  colspec = {Q[356]Q[465]Q[119]},
  cells = {c},
  cells = {font = \small},
  cell{7}{1} = {c=3}{},
  hlines,
  vlines,
}
\textbf{Ұйымдық-құқықтық нысанның атауы}                                                & \textbf{Тіркелген шағын және орта бизнес субъектілерінің саны, мың бірлік} & \textbf{Құрылымы, \%} \\
Жеке кәсіпкерлік                                                                        & 1523,0                                                                     & 67,4                  \\
Серіктестіктер (жауапкершілігі шектеулі серіктестіктер)                                 & 467,9                                                                      & 20,68                 \\
Акционерлік қоғамдар                                                                    & 0,23                                                                       & 0,02                  \\
Шаруа (фермер) қожалықтары                                                              & 271,3                                                                      & 11,9                  \\
Барлығы                                                                                 & 2262,4                                                                     & 100,0                 \\
\textit{Ескерту - авторлар Bureau of National Statistics дереккөзі бойынша құрастырған} &                                                                            &                       
\end{longtblr}

\begin{multicols}{2}
Шағын және орта бизнеспен қатар әлеуметтік жауапкершілік динамикасы
тоқырау сипатына ие және 1-суретте көрсетілген ірі кәсіпорындардың
атрибуты болып табылады.

Бизнестің әлеуметтік жауапкершілігінің инвестициялық-экономикалық негізі
шаруашылық жүргізуші субъектілердің кірісі болып табылады. Жүргізілген
зерттеулер 3 және 4-кестелерге сәйкес шағын және орта, ірі
бизнес-кәсіпкерлік бөлінісіндегі кірістер мен пайда нарықтық
қатынастардың даму жылдарында іс жүзінде тепе-теңдік деңгейіне жеткенін
көрсетеді.

Бизнес-кәсіпкерлік субъектілерінің пайдасы Қазақстанда қалыптасқан
заңнама жүйесін ескере отырып, олардың қоғам алдындағы әлеуметтік
жауапкершілігін орындауы үшін басты инвестициялық негіз болып табылады.
Мәселен, 3-кестеде Қазақстанның шағын бизнес-ұйымдарының табысы мен
пайдасының динамикасын толығырақ қарастыруға болады.
\end{multicols}

{\bfseries 1 - сурет. Қазақстанда жұмыс істейтін ірі бизнес-ұйымдардың динамикасы, бірлік}

\emph{Ескерту - авторлар Bureau of National Statistics дереккөзі бойынша құрастырған}

\tcap{3 - кесте. Қазақстанның шағын бизнес-ұйымдарының табысы мен пайдасының серпіні}
\begin{longtblr}[
  label = none,
  entry = none,
]{
  width = \linewidth,
  colspec = {Q[433]Q[100]Q[100]Q[100]Q[100]Q[100]},
  cells = {c},
  cells = {font = \small},
  cell{1}{1} = {r=2}{},
  cell{1}{2} = {c=5}{0.5\linewidth},
  cell{7}{1} = {c=6}{0.932\linewidth},
  vlines,
  hline{1,3-8} = {-}{},
  hline{2} = {2-6}{},
}
\textbf{Көрсеткіштің атауы}                                                             & \textbf{Мәні} &               &               &               &               \\
                                                                                        & \textbf{2020} & \textbf{2021} & \textbf{2022} & \textbf{2023} & \textbf{2024} \\
Шағын бизнес-кәсіпкерлік субъектілерінің табысы, млн теңге                              & 23~164 042    & 28~025 374    & 36~539 684    & 45~577 129    & 54~197 184    \\
Қызметкерлердің орташа жылдық тізімдік саны, адам                                       & 1~409 882     & 1~421 017     & 1~645 995     & 1~756 265     & 1~745 668     \\
Қызметкерлердің жалақы қоры, млн теңге                                                  & 2~448 701     & 2~898 943     & 3~900 773     & 4~875 973     & 6~503 987     \\
Шағын бизнес-кәсіпкерлік субъектілерінің пайдасы, млн теңге                             & 4~742 647     & 8~649 502     & 11~275 290    & 14~560 620    & 17~604 983    \\
\textit{Ескерту - авторлар Bureau of National Statistics дереккөзі бойынша құрастырған} &               &               &               &               &               
\end{longtblr}

Бұдан әрі 4-кестеде Қазақстанда жұмыс істейтін ірі бизнес-ұйымдардың
табыс, пайда және рентабельділік динамикасын толығырақ қарастыруға
болады.

\tcap{4 - кесте. Қазақстанда жұмыс істейтін ірі және орта бизнес-ұйымдардың табыс, пайда және рентабельділік динамикасы}
\begin{longtblr}[
  label = none,
  entry = none,
]{
  width = \linewidth,
  colspec = {Q[440]Q[98]Q[98]Q[98]Q[98]Q[98]},
  cells = {c},
  cells = {font = \small},
  cell{1}{1} = {r=2}{},
  cell{1}{2} = {c=5}{0.49\linewidth},
  cell{6}{1} = {c=6}{0.929\linewidth},
  vlines,
  hline{1,3-7} = {-}{},
  hline{2} = {2-6}{},
}
\textbf{Көрсеткіштің атауы}                                                             & \textbf{Мәні} &               &               &               &               \\
                                                                                        & \textbf{2020} & \textbf{2021} & \textbf{2022} & \textbf{2023} & \textbf{2024} \\
Өнімді сатудан және қызмет көрсетуден түскен табыс, млн теңге                           & 46~675 618    & 60~323 895    & 78~132 450    & 81~956 087    & 91~236 611    \\
Рентабельділік, \%                                                                      & 11,6          & 28,0          & 23,6          & 18,5          & 15,8          \\
Ірі және орта кәсіпорындардың пайдасы, млн. теңге                                       & 6~679 821     & 14~796 652    & 17~995 441    & 14~764 670    & 14~653 528    \\
\textit{Ескерту - авторлар Bureau of National Statistics дереккөзі бойынша құрастырған} &               &               &               &               &               
\end{longtblr}

Қазақстан Республикасындағы бизнестің әлеуметтік жауапкершілігін
Конституция, Еңбек кодексі, Экологиялық Кодекс, «Жер қойнауы және жер
қойнауын пайдалану туралы» Заң, «Кәсіподақтар туралы» Заң сияқты заңдар
мен нормативтік-құқықтық құжаттар реттейді. Қазақстандағы бизнес-ұйымдар
тарапынан әлеуметтік жауапкершілікке инвестициялар 2-суретке сәйкес
ұсынылған бағыттар бойынша іске асырылады.

Бизнес-ұйымдардың пайдасын әлеуметтік жауапкершілікке инвестициялау
бағыттары

Жаңа жұмыс орындарын құру және еңбек нарығын дамыту

Салық міндеттемелерін кешенді және жан-жақты орындау

Әлеуметтік жауапкершілік қорларын қаржыландыру

Негізгі өндірістік жабдықтарды, кәсіпорындарды басқару жүйелерін
техникалық жаңғырту есебінен адам ресурстарының еңбек жағдайлары мен
қауіпсіздігін жақсарту

Тұтынушылардың проблемаларын шешуге бағдарлануын ескере отырып,
тауарлардың, көрсетілетін қызметтердің сапасы мен инновациясын жетілдіру

Өңірлердің әлеуметтік инфрақұрылымын дамыту

Білім беру саласын дамыту, адам ресурстарын даярлау, қайта даярлау және
олардың біліктілігін арттыру

Өнеркәсіптік кәсіпорындардың экологиялық қауіпсіздігін жетілдіру

{\bfseries 2 - сурет. Қазақстанның бизнес-ұйымдары тарапынан әлеуметтік жауапкершілікке инвестициялар бағыттары}

\emph{Ескерту - авторлар құрастырған}

Шағын және орта бизнес субъектілерінің әлеуметтік жауапкершілігі
өндірістік-шаруашылық қызметті жүргізудің құқықтық негіздерін сақтау
қағидаттарында іске асырылады, мысалы: еңбек заңнамасын сақтай отырып,
жаңа жұмыс орындарын құру; салық міндеттемелерін кешенді және жан-жақты
орындау; ең төменгі мөлшерден кем емес мөлшерде жалақы төлеу; ақшалай
қаражатты әлеуметтік сақтандыру қорына, міндетті медициналық сақтандыру
қорына аудару. Шағын және орта бизнес-кәсіпкерлік субъектілерінің
әлеуметтік жауапкершілік мүмкіндіктері олардың бір кәсіпорынға
шаққандағы табысы мен пайдасымен шектеледі.

Әлеуметтік жауапкершілікті іске асыру үшін жетекші ірі өңірлік
өнеркәсіптік кәсіпорындар облыс әкімдіктерімен әлеуметтік жобаларға
инвестициялар бөлу туралы меморандумдар жасайды. Өнеркәсіптік
кәсіпорындардың әлеуметтік жауапкершілік саясатына қатысты меморандумдар
динамикалық сипатқа ие, әр күнтізбелік жылға қаралады және жасалады. Бұл
ретте іс жүзінде 3 - суретте көрсетілген ұйымдастырушылық-экономикалық
тетік қолданылады.

Экономиканың жеке секторы

Экономиканың мемлекеттік секторы

Жетекші өңірлік өнеркәсіптік кәсіпорындар-жер қойнауын пайдаланушылар

Облыстардың әкімдіктері, өңірлік басқармалар

Өнеркәсіптік кәсіпорындардың пайдасын қалыптастыру

Әлеуметтік саланы дамыту басымдықтарын әзірлеу

Жер қойнауын пайдалану, табиғи ресурстарды пайдалану саласындағы
саясатты әзірлеу

Әлеуметтік саланы дамыту бойынша мемлекеттік-жекешелік әріптестік
саласындағы саясатты келісу

Әлеуметтік саланы қаржыландыру бойынша квоталарды келісу, меморандумдар
жасасу

Әлеуметтік инфрақұрылым объектілеріне басым бағыттар бойынша
инвестициялар салу

Өңірлік экономикалық жүйелердің әлеуметтік саласын дамыту

{\bfseries 3 - сурет. Өңірлік экономикалық жүйелер тәжірибесінде қолданылатын жер қойнауын пайдаланушы өнеркәсіп кәсіпорындарының әлеуметтік жауапкершілігінің ұйымдастырушылық-экономикалық тетігі}

\emph{Ескерту - авторлар құрастырған}

\begin{quote}
2022-2024 жылдардағы әлеуметтік жобаларды іске асыруға ірі өнеркәсіптік
кәсіпорындар тарапынан инвестициялардың серпіні 4-суретте келтірілген.

Ел өңірлерінің әлеуметтік саласын дамытуға инвестициялардың серпіні
уақытша кезеңдер бөлінісінде өзгермелі сипатқа ие. Әлеуметтік саланы
қолдау үшін ең жоғары инвестициялық шығындар 2024 жылға, ал ең төменгі
инвестициялық шығындар 2023 жылға келді.

Қазақстан Республикасының жекелеген өңірлеріндегі ірі өнеркәсіптік
кәсіпорындардың әлеуметтік жауапкершілігінің негізгі бағыттары 5-кестеде
келтірілген.
\end{quote}

{\bfseries 4 - сурет.2020 - 2024 жылдарға арналған әлеуметтік жобаларды іске асыруға ірі өнеркәсіптік кәсіпорындар тарапынан инвестициялар серпіні, млрд. теңге}

{\bfseries Ескерту - авторлар ERG Corporation Social projects дереккөзі бойынша құрастырған}

\tcap{5 - кесте. Қазақстанның жекелеген өңірлеріндегі ірі өнеркәсіптік кәсіпорындардың 2024 жылға арналған әлеуметтік жауапкершілігінің негізгі бағыттары}
\begin{longtblr}[
  label = none,
  entry = none,
]{
  row{1} = {c},
  row{6} = {c},
  cell{2}{1} = {c},
  cell{3}{1} = {c},
  cell{4}{1} = {c},
  cell{5}{1} = {c},
  cell{6}{1} = {c=2}{},
  cells = {font = \small},
  hlines,
  vlines,
}
\textbf{Аймақ}                                                                            & \textbf{Әлеуметтік жауапкершілік бағыты}                                                                                                                                                                                                                                           \\
Ақтөбе облысы                                                                             & {- Ақтөбе және Хромтау қалаларын абаттандыру;\\- балабақшаларды салу және жөндеу;\\- аумақтық кәсіптік-техникалық оқу орындарын қолдау;\\- қоғамдық мекемелерге көмек.}                                                                                                            \\
Қарағанды облысы                                                                          & {- жолдарды жөндеу;\\- балалар ойын алаңдарын салу;\\- аулаларды абаттандыру;\\- жол және коммуналдық техниканы сатып алу;\\- мектептерді жөндеу және жабдықтармен жарақтандыру;\\- мәдениет және спорт салаларын қолдау;\\- халықтың әлеуметтік осал топтарына көмек көрсету.}    \\
Қостанай облысы                                                                           & {- Қостанай, Рудный қалаларын абаттандыру;\\- мәдениет және спорт инфрақұрылымы объектілерін салу;\\- инженерлік-коммуникациялық желілерді жөндеу;\\- медициналық мекемелерді қолдау.}                                                                                             \\
Павлодар облысы                                                                           & {- Павлодар, Екібастұз, Ақсу қалаларын абаттандыру;\\- тұрғын үй құрылысы;\\- білім беру ұйымдары үшін жатақханалар салу;\\- мәдениет және спорт саласының инфрақұрылымдық объектілерін салу;\\- балалар алаңдарын салу;\\- білім беру ұйымдарын қолдау;\\- зейнеткерлерге көмек.} \\
\textit{Ескерту - авторлар ERG Corporation Social projects дереккөзі бойынша құрастырған} &                                                                                                                                                                                                                                                                                    
\end{longtblr}

Қазақстан Республикасында кәсіпорындардың, фирмалардың және басқа да
бизнес-кәсіпкерлік субъектілерінің әлеуметтік жауапкершілігі
қолданыстағы заңнама шеңберінде қамтамасыз етіледі және
кепілдендіріледі. Бұл бағыттағы маңызды құрал «Міндетті әлеуметтік
сақтандыру туралы» Заң болып табылады. Заң кәсіпорындарда,
бизнес-құрылымдарда еңбекақы төлеу жүйесіне параллель әлеуметтік пакетті
мәжбүрлеуді реттейді. Қолданыстағы заңнама шеңберінде кәсіпорындар
мемлекеттік әлеуметтік сақтандыру қорына ай сайынғы аударымдарды және
кәсіпорынның әрбір қызметкеріне әлеуметтік салық төлеуді жүзеге асырады.

{\bfseries 5 - сурет. Қазақстан Республикасындағы нақты орташа жалақының динамикасы, теңге}

\emph{Ескерту - авторлар Bureau of National Statistics дереккөзі бойынша құрастырған}

\begin{multicols}{2}
Қазақстан Республикасында соңғы жылдары суретке сәйкес орташа жалақының
номиналды өсу тенденциялары болды.

{\bfseries Қорытынды.} Қазақстан Республикасының экономикалық жүйесіндегі
бизнес-ұйымдардың әлеуметтік жауапкершілігін құру және дамыту үрдістерін
зерттеу шағын, орта, ірі кәсіпорындардың, фирмалардың, компаниялардың
қоғамды, әлеуметтік саланы дамытуға қатысуы қалыптасу сатысында тұрғанын
көрсетеді. Әлеуметтік жауапкершіліктің құралдары мен тетіктерін қолдану
ірі бизнес-кәсіпкерлік субъектілеріне тән, ал белгілі бір маңызды рөл
мемлекетке тиесілі. Әлеуметтік жауапкершілік халықаралық стандарттарға
толық сәйкес келмейді. Әлеуметтік жауапкершілікке бағытталған
халықаралық менеджмент стандарттарын бизнес-қауымдастық аз пайдаланады.

Қазақстан Республикасында бизнестің әлеуметтік жауапкершілігінің
қалыптасқан тетіктері мен құралдары әлеуметтік саланың серпінді дамуы,
елдің адами әлеуетінің сапалы дамуы үшін кешенді жағдайлар жасамайды,
бұл өз кезегінде экономикалық өсудің қарқындылығына ықпал етпейді.

Кешенде бизнес-ұйымдардың әлеуметтік жауапкершілігінің анықталған
тенденциялары экономиканың, өнеркәсіптің, шағын және орта бизнестің
нақты секторының қарқынды дамуының факторы емес, салдары болып табылады
деген қорытынды жасауға болады. Бизнестің әлеуметтік жауапкершілік
институты қалыптасады, бірақ экономиканы дамытудың локомотиві емес.
Нәтижесінде Қазақстан Республикасының қолданыстағы экономикалық жүйесі
таяудағы кемінде екі стратегиялық кезеңде индустриялық даму деңгейінде
қалатын болады және оның постиндустриалды кезеңге өтуінің берік іргетасы
ұзақ мерзімді кезеңде ғана қалыптаса алады.

Жоғарыда аталған проблемалар қолданыстағы заңнаманы және
нормативтік-құқықтық актілерді қоса алғанда, әлеуметтік жауапкершілік
саласындағы тиімсіз мемлекеттік саясатпен тікелей негізделеді. Ел
экономикасында бизнес-кәсіпкерлік субъектілеріне қатысты салық,
әлеуметтік, еңбек және экологиялық нормативтерді орындау бойынша
салыстырмалы түрде төмен талаптар қолданылады.

Жүргізілетін зерттеудің перспективалары микро, мезо және макро
деңгейлердегі әлеуметтік саланың дамуына тікелей әсерін ескере отырып,
ірі корпорациялардың да, шағын және орта бизнес кәсіпорындарының да
әлеуметтік жауапкершілігінің ұйымдастырушылық-экономикалық тетіктерін
жетілдіруге бағытталуы тиіс. Болашақ зерттеулерде экономикадағы
әлеуметтік саланың үлес салмағын анықтау, бизнес-ұйымдардың әлеуметтік
жауапкершілік тетіктерін іске асыру арқылы экономиканың дамудың
постиндустриалды кезеңіне өту өлшемдерін айқындау жоспарлануда.
\end{multicols}

\begin{center}
{\bfseries Әдебиеттер}
\end{center}

\begin{refs}
1. Мауленкулова Г.Е., Бигельдиева З.А., Исахметова А.Н. Тұрақты даму
шеңберіндегі бизнестің әлеуметтік жауапкершілігін зерттеу //
Статистика, есеп және аудит. - 2025. -№ 1(96). Б.- 244-254. DOI
10.51579/1563-2415.2025.-1.18.

2. Liagutov K.J., Zelezinskii A.L., Arhipova O.V.~ The impact of
corporate social responsibility on business competitiveness //
Economic Vector. -2023. -№ 4(35). -P.5-8. DOI
10.36807/2411-7269-2023-4-35-5-8.

3. Туреханова А.О., Устенова О.Ж. Маркетинговая оценка влияния
корпоративной социальной ответсвенности на развитие малого бизнеса//
Central Asian Economic Review. -2022. -Т.(4). - С.70-81. DOI
10.52821/2789-4401-2022-4-70-81.

4. Сайдулаева З. С. Место и роль субъектов малого и среднего
предпринимательства в развитии социальной ответственности бизнеса //
Central Asian Scientific Journal. - 2023. -T.1(16). - С.68-78.

5. Bulkhairova Z., Bermukhamedova G., Dzhanegizova A., Primbetova S.
Corporate Social Responsibility in Kazakhstan: Current State and Ways
of Development //~Eurasian Journal of Economic and Business Studies.
-2023. - 67(1). -Р.27--38.
\href{https://doi.org/10.47703/ejebs.v1i67.245}{DOI
10.47703/ejebs.v1i67.245}.

6. Carrera L. Corporate social responsibility. A strategy for social and
territorial sustainability // International Journal of Corporate
Social Responsibility. - 2022. - Vol.7(1). - P.1-11. DOI 10.1186/
s40991-022-00074-0.

7. Пустынникова Е.В., Тимохина А.В.Корпоративная социальная
ответственность бизнеса // Экономика и бизнес: теория и практика. -
2023. - №11-3 (105).-C.12-16. DOI 10.24412/2411-0450-2023-11-3-12-16.

8. Newman C., Rand J., Tarp F.,
\href{https://www.tandfonline.com/author/Trifkovic\%2C+Neda}{Trifkovic}
N. Corporate social responsibility in a competitive business
environment // The Journal of Development Studies. -2020. - Vol.
56(8). -Р.1455-1472.
\href{https://doi.org/10.1080/00220388.2019.1694144}{DOI
10.1080/00220388.2019.1694144}.

9. Yerniyazova A., Smailova Zh. Corporate social responsibility of
business in the republic of Kazakhstan //~Reports of the National
Academy of sciences of the Republic of Kazakhstan. - 2019. -Vol.4
(326). - Р.76--82. DOI 10.32014/2019.2518-1483.119.

10. Baltabayeva A.K., Altaibayeva Z.K., Shelomentseva V.P., Mutallyapova
S.E., Alimkhanova R.K. Promotion of the Principles of Corporate Social
Responsibility and their Implementation by Large Businesses in
Kazakhstan //~Australasian Accounting, Business and Finance Journal.
-2024. - №18(1). -Р.172-186. DOI 10.14453/aabfj.v18i1.11.

11. Тауекелова Т.Б., Абдикерова Г.О. Қазақстандық бизнестің әлеуметтік
жауапкершілігін қалыптастыру ерекшеліктері // ҚазҰУ Хабаршысы.
Психология және әлеуметтану сериясы. -2021. -№78(3). --Б.157--166.
\href{https://doi.org/10.26577/JPsS.2021.v78.i3.14}{DOI
10.26577/JPsS.2021.v78.i3.14}.

12. Социальная ответственность бизнеса. Бизнес бастама. -2018. URL:
\url{https://startinfo.kz/buisness/sozialno-v/}. -Қаралған күні:
30.05.2025.
\end{refs}

\begin{center}
{\bfseries References}
\end{center}

\begin{refs}
1. Maulenkulova G.E., Bigel' dieva Z.A., Isakhmetova A.N.
Turakty damu shenberіndegі biznestіn aleumettіk zhauapkershіlіgіn
zertteu // Statistika, esep zhane audit. - 2025. -№ 1(96). B.-
244-254. DOI 10.51579/1563-2415.2025.-1.18. {[}in Kazakh{]}

2. Liagutov K.J., Zelezinskii A.L., Arhipova O.V.~ The impact of
corporate social responsibility on business competitiveness //
Economic Vector. -2023. -№4(35). -P.5-8. DOI
10.36807/2411-7269-2023-4-35-5-8.

3. Turehanova A.O., Ustenova O.Zh. Marketingovaja ocenka vlijanija
korporativnoj social' noj otvetsvennosti na razvitie
malogo biznesa// Central Asian Economic Review. -2022. -T.(4).
-S.70-81. DOI 10.52821/2789-4401-2022-4-70-81. {[}in Russian{]}

4. Saidulaeva Z. S. Mesto i rol'{} sub"ektov malogo i
srednego predprinimatel' stva v razvitii
sotsial' noi otvetstvennosti biznesa // Central Asian
Scientific Journal. -2023. -T.1(16). -S.68-78. {[}in Russian{]}

5. Bulkhairova Z., Bermukhamedova G., Dzhanegizova A., Primbetova S.
Corporate Social Responsibility in Kazakhstan: Current State and Ways
of Development //~Eurasian Journal of Economic and Business Studies.
-2023. - 67(1). -Р.27--38.
\href{https://doi.org/10.47703/ejebs.v1i67.245}{DOI
10.47703/ejebs.v1i67.245}.

6. Carrera L. Corporate social responsibility. A strategy for social and
territorial sustainability // International Journal of Corporate
Social Responsibility. - 2022. - Vol.7(1). - P.1-11. DOI 10.1186/
s40991-022-00074-0.

7. Pustynnikova E.V. Korporativnaya sotsial' naya
otvetstvennost'{} biznesa // Ekonomika i biznes:
teoriya i praktika. -2023. -№11-3 (105). DOI
10.24412/2411-0450-2023-11-3-12-16. {[}in Russian{]}

8. Newman C., Rand J., Tarp F.,
\href{https://www.tandfonline.com/author/Trifkovic\%2C+Neda}{Trifkovic}
N. Corporate social responsibility in a competitive business
environment // The Journal of Development Studies. -2020. - Vol.
56(8). -Р.1455-1472.
\href{https://doi.org/10.1080/00220388.2019.1694144}{DOI
10.1080/00220388.2019.1694144}.

9. Yerniyazova A., Smailova Zh. Corporate social responsibility of
business in the republic of Kazakhstan //~Reports of the National
Academy of sciences of the Republic of Kazakhstan. - 2019. -Vol.4
(326). - Р.76-82. DOI 10.32014/2019.2518-1483.119.

10. Baltabayeva A.K., Altaibayeva Z.K., Shelomentseva V.P., Mutallyapova
S.E., Alimkhanova R.K. Promotion of the Principles of Corporate Social
Responsibility and their Implementation by Large Businesses in
Kazakhstan //~Australasian Accounting, Business and Finance Journal.
-2024. - №18(1). -Р.172-186. DOI 10.14453/aabfj.v18i1.11.

11. Tauekelova T.B., Abdikerova G.O. Kazakstandik biznestіn aleumettіk
zhauapkershіlіgіn kalyptastyru erekshelіkterі // ҚazUU Khabarshysy.
Psikhologiya zhane aleumettanu seriyasy. -2021. - №78(3). -B.157-166.
\href{https://doi.org/10.26577/JPsS.2021.v78.i3.14}{DOI
10.26577/JPsS.2021.v78.i3.14}. {[}in Kazakh{]}

12. Social' naya otvetstvennost'{} biznesa.
Biznes bastama. -2018. URL:
\url{https://startinfo.kz/buisness/sozialno-v/}. Date of access:
30.05.2025. {[}in Russian{]}
\end{refs}

\begin{info}
\hspace{1em}\emph{{\bfseries Авторлар туралы мәлімет}}

Ибраимова С.Ж. - э.ғ.к., профессор, Қ.Құлажанов атындағы Қазақ
технология және бизнес университеті, Астана, Қазақстан, е-mail:
saule\_ibraimova\_kz@mail.ru;

Касымова С.Б. - сеньор-лектор, магистр, Қ.Құлажанов атындағы Қазақ
технология және бизнес университеті, Астана, Қазақстан, е-mail:
Sanim\_81@list.ru;

Каримбаева Г.Ж. - э.ғ.к., асс.-профессор, Қ.Құлажанов атындағы Қазақ
технология және бизнес университеті, Астана, Қазақстан, е-mail:
karimbaevagulzan@gmail.com;

Шарипова Д.Б. - сеньор-лектор, магистр, Қ.Құлажанов атындағы Қазақ
технология және бизнес университеті, Астана, Қазақстан, е-mail:
Dinara.sharipova.73@mail.ru.

\hspace{1em}\emph{{\bfseries Information about the author}}

Ibraimova S.Zh. - Candidate of Economics, Professor, K.Kulazhanov
Kazakh University of Technology and Business, Astana, Kazakhstan,
e-mail: saule\_ibraimova\_kz@mail.ru;

Kassymova S.B. - Master' s degree, Senior Lecturer, K.Kulazhanov
Kazakh University of Technology and Business, Astana, Kazakhstan,
e-mail:Sanim\_81@list.ru;

Karimbayeva G.Zh. - Candidate of Economics, Ass.-Professor,
K.Kulazhanov Kazakh University of Technology and Business, Astana,
Kazakhstan, e-mail: karimbaevagulzan@gmail.com;

Sharipova D.B. - Master' s degree, Senior Lecturer, K.Kulazhanov
Kazakh University of Technology and Business, Astana, Kazakhstan,
e-mail: Dinara.sharipova.73@mail.ru.
\end{info}
