\id{IRSTI 06.81.55}{}

\begin{header}
\swa{}{ECONOMIC ANALYSIS OF DIGITAL HEALTHCARE DEVELOPMENT IN THE EAST KAZAKHSTAN REGION}

\tsp{1}A. Maldynova\envelope,
\tsp{2}N. Aidargaliyeva,
\tsp{3}D. Mussabalina,
\tsp{4}Zh.S. Abdrakhmanova,
\tsp{5}T. Demesinov
\end{header}

\begin{affil}
\tsp{1}Kenzhegali Sagadiyev University of International Business, Almaty, Kazakhstan,

\tsp{2}L.N. Gumilyov Eurasian National University, Astana, Kazakhstan,

\tsp{3}Abai Kazakh National Pedagogical University, Almaty, Kazakhstan,

\tsp{4}D.Serikbayev East Kazakhstan Technical University, Ust-Kamenogorsk, Kazakhstan,

\tsp{5}Sh. Ualikhanov Kokshetau University, Kokshetau, Kazakhstan

\corrauthor{Corresponding-author: maldynova.a@uib.kz}
\end{affil}

The digitalization of healthcare is becoming a crucial factor in
improving the efficiency and accessibility of medical services
worldwide. In Kazakhstan, the level of digital transformation remains
uneven across regions. This research focuses on the East Kazakhstan
Region, aiming to evaluate the impact of digitaliza\-tion on healthcare
services and identify the relationship between ICT integration and key
demographic indicators.

The research is based on statistical data from the Bureau of National
Statistics of Kazakhstan, Ministry of Healthcare reports, and
international sources covering the period 2020--2024. Methods applied
include comparative and statistical analysis, trend modeling, and
geographical contextualization with emphasis on regional healthcare
infrastructure in Ust-Kamenogorsk.

The research revealed a positive correlation between the increase in ICT
specialists and improvements in mortality indicators, particularly
maternal mortality. Results demonstrate a twofold increase in computer
literacy among healthcare employees, significant growth in internet
usage for professional tasks, and progressive but uneven adoption of ICT
across healthcare institutions. Despite positive dynamics, challen\-ges
persist: insufficient funding for ICT, shortage of cybersecurity
specialists, and limited training oppor\-tunities.

The novelty of this research lies in its regional focus: it is the first
systematic analysis of healthcare digitalization in the East Kazakhstan
Region. The findings extend existing theoretical frameworks by
incorporating regional heterogeneity and provide practical insights for
policymakers to enhance digital healthcare strategies in Kazakhstan.

{\bfseries Keywords:} economics, economics of medical services, e-health,
socio-economic development, digita\-lization, information and
communication technologies.

\begin{header}
ШЫҒЫС ҚАЗАҚСТАН ОБЛЫСЫНДАҒЫ ЦИФРЛЫҚ ДЕНСАУЛЫҚ САҚТАУДЫҢ ДАМУЫНЫҢ ЭКОНОМИКАЛЫҚ ТАЛДАУЫ

\tsp{1}А.В. Малдынова\envelope,
\tsp{2}Н. Айдарғалиева,
\tsp{3}Д. Мұсабалина,
\tsp{4}Ж.С. Әбдрахманова,
\tsp{5}Т.Ж. Демесинов
\end{header}

\begin{affil}
\tsp{1}Кенжегали Сагадиев атындағы Халықаралық бизнес университеті, Алматы, Қазақстан,

\tsp{2}Л.Н. Гумилев атындағы Еуразия ұлттық университеті, Астана, Қазақстан,

\tsp{3}Абай атындағы Қазақ ұлттық педагогикалық университеті, Алматы, Қазақстан,

\tsp{4}Д.Серікбаев атындағы Шығыс Қазақстан техникалық университеті, Өскемен, Қазақстан,

\tsp{5}Ш. Уәлиханов атындағы Көкшетау университеті, Көкшетау, Қазақстан,

\envelope e-mail: maldynova.a@uib.kz
\end{affil}

Денсаулық сақтауды цифрландыру бүкіл әлем бойынша денсаулық сақтау
қызметтерінің тиімділігі мен қолжетімділігін арттырудың маңызды
факторына айналуда. Қазақстанда цифрлық трансформация деңгейі өңірлер
бойынша біркелкі емес. Бұл зерттеу Шығыс Қазақстан облысына бағытталған
және цифрландырудың денсаулық сақтау қызметтеріне әсерін бағалауға және
АКТ интеграциясы мен негізгі демографиялық көрсеткіштер арасындағы
байланысты анықтауға бағытталған.

Зерттеу Қазақстан Республикасы Ұлттық статистика басқармасының
статистикалық деректеріне, Денсаулық сақтау министрлігінің есептеріне
және 2020--2024 жылдарға арналған халықаралық ақпарат көздеріне
негізделген. Пайдаланылған әдістерге салыстырмалы және статистикалық
талдау, трендтерді модельдеу және Өскемендегі денсаулық сақтаудың
өңірлік инфрақұрылымына назар аудара отырып, географиялық
контекстуализация кіреді.

Зерттеу АКТ мамандарының санының артуы мен өлім-жітім көрсеткіштерінің,
әсіресе ана өлімінің жақсаруы арасындағы оң корреляцияны анықтады.
Нәтижелер медицина қызметкерлерінің компьютерлік сауаттылығының екі есе
артқанын, Интернетті кәсіби мақсатта пайдаланудың айтарлықтай артқанын
және денсаулық сақтау мекемелерінде АКТ-ның біртіндеп, бірақ біркелкі
емес енгізілуін көрсетеді. Оң динамикаға қарамастан, проблемалар
сақталуда: АКТ-ны қаржыландырудың жеткіліксіздігі, киберқауіпсіздік
мамандарының тапшылығы және оқыту мүмкіндіктерінің шектеулілігі. Бұл
зерттеудің жаңалығы оның аймақтық бағытталуында: бұл Шығыс Қазақстан
облысындағы денсаулық сақтау саласын цифрландырудың бірінші жүйелі
талдауы. Нәтижелер аймақтық әркелкілікті ескере отырып, қолданыстағы
теориялық негізді кеңейтеді және саясаткерлерге Қазақстандағы цифрлық
денсаулық сақтау стратегияларын жетілдіру бойынша практикалық ұсыныстар
береді.

{\bfseries Түйін сөздер:} экономика, медициналық қызмет көрсету
экономикасы, электрондық денсаулық сақтау,
әлеуметтік-экономикалық даму, цифрландыру, ақпараттық-коммуникациялық
технологиялар.

\begin{header}
ЭКОНОМИЧЕСКИЙ АНАЛИЗ РАЗВИТИЯ ЦИФРОВОГО ЗДРАВООХРАНЕНИЯ В ВОСТОЧНО-КАЗАХСТАНСКОЙ ОБЛАСТИ

\tsp{1}А.В. Малдынова\envelope,
\tsp{2}Н. Айдаргалиева,
\tsp{3}Д. Муссабалина,
\tsp{4}Ж.С. Абдрахманова,
\tsp{5}Т.Ж. Демесинов
\end{header}

\begin{affil}
\tsp{1}Университет международного бизнеса имени Кенжегали Сагадиева, Алматы, Казахстан,

\tsp{2}Евразийский национальный университет имени Л.Н. Гумилёва, Астана, Казахстан,

\tsp{3}Казахский национальный педагогический университет имени Абая, Алматы, Казахстан,

\tsp{4}Восточно-Казахстанский технический университет им. Д. Серикбаева, Усть-Каменогорск, Казахстан,

\tsp{5}Кокшетауский университет имени Ш. Уалиханова, Кокшетау, Казахстан,

\envelope e-mail: maldynova.a@uib.kz
\end{affil}

Цифровизация здравоохранения становится важнейшим фактором повышения
эффективности и доступности медицинских услуг во всем мире. В Казахстане
уровень цифровой трансформации остается неравномерным по регионам.
Данное исследование сосредоточено на Восточно-Казах\-станской области и
направлено на оценку влияния цифровизации на медицинские услуги и
выявление взаимосвязи между интеграцией ИКТ и ключевыми демографическими
показателями.

Исследование основано на статистических данных Национального
статистического управления Республики Казахстан, отчётах Министерства
здравоохранения и международных источниках за период 2020--2024 годов.
Применённые методы включают сравнительный и статистический анализ,
моделирование тенденций и географическую контекстуализацию с акцентом на
региональную инфраструктуру здравоохранения в Усть-Каменогорске.

Исследование выявило положительную корреляцию между увеличением числа
специалистов в области ИКТ и улучшением показателей смертности, особенно
материнской. Результаты демонстрируют двукратный рост компьютерной
грамотности медицинских работников, значительный рост использования
интернета в профессиональных целях и постепенное, но неравномерное
внедрение ИКТ в медицинских учреждениях. Несмотря на положительную
динамику, сохраняются проблемы: недостаточное финансирование ИКТ,
нехватка специалистов по кибербезопасности и ограниченные возможности
обучения. Новизна данного исследования заключается в его региональной
направленности: это первый систематический анализ цифровизации
здравоохранения в Восточно-Казахстанской области. Результаты расширяют
существующие теоретические основы, учитывая региональную неоднородность,
и предоставляют политикам практические рекомендации по совершенствованию
стратегий цифрового здравоохранения в Казахстане.

{\bfseries Ключевые слова:} экономика, экономика медицинских услуг,
электронное здравоохранение, со\-циально-экономическое развитие,
цифровизация, информационно-коммуникационные технологии.

\begin{multicols}{2}
{\bfseries Introduction.} In recent decades, digitalization has become a
key area of transformation of socio-economic
systems, including healthcare. Many countries have accumulated
significant experience in implementing electronic medical records,
telemedicine, and patient flow management systems, which has improved
the efficiency of medical care. In the Republic of Kazakhstan, this
process is also actively developing, but the level of digital
integration of medical services in the regions remains uneven. The East
Kazakhstan region is an interesting object of research due to the
specifics of its demographic structure, the remoteness of individual
settlements and differences in the availability of medical
infrastructure {[}1{]}.

It is important to note that in 2022 the East Kazakhstan Region
underwent administrative restructuring, resulting in the establishment
of the Abai Region. This territorial change affects the comparability of
statistical indicators across the period analyzed. Therefore, data for
2020-2022 reflect the unified region, while indicators from 2023-2024
capture already reorganized administrative boundaries. This factor must
be considered when interpreting trends in digital healthcare
development.

The relevance of the topic is due to the need to improve the quality and
accessibility of medical services for the population of the East
Kazakhstan region in conditions of limited resources. Digital
technologies make it possible to solve several critical problems: reduce
waiting times, improve diagnostics, increase transparency and efficiency
of patient interactions with medical organizations. In the post-pandemic
period, digitalization of healthcare is becoming not only a tool for
improving the quality of services, but also the most important factor in
the sustainability of the entire healthcare system in the region
{[}2{]}. The aim of the study is to analyze the impact of digitalization
on the medical services market in the East Kazakhstan region and
identify key areas of its development.

Most studies focus on the macro level - studying the digitalization of
healthcare on a national scale. At the same time, regional features,
such as the specifics of the East Kazakhstan region, remain
insufficiently covered. The relationship between the level of
digitalization and such indicators as the quality of medical services,
patient satisfaction and the efficiency of medical personnel have also
been insufficiently studied. The main problem is the insufficient
empirical research of the impact of digital technologies on the regional
market of medical services. The proposed solution is to conduct a
comprehensive analysis of the digitalization of healthcare in the East
Kazakhstan region based on statistical data, comparative analysis and
forecast models. This approach will fill the identified gaps and offer
practical recommendations for improving the digital infrastructure of
healthcare in the region {[}3{]}. The novelty of the study lies in the
fact that for the first time it presents a systemic analysis of the
impact of digitalization on the medical services market of the East
Kazakhstan region. Unlike existing works, the emphasis is on regional
specifics and the relationship of digital technologies with
socio-economic indicators in healthcare. The results of the study allow
us to clarify the role of digital tools in increasing the sustainability
of the regional healthcare system and to develop proposals for further
digital transformation.

The growth of Internet penetration in Kazakhstan creates the
preconditions for the expansion of telemedicine services, but an
imbalance between urban and rural areas remains. According to
DataReportal {[}4{]}, the level of Internet access in Kazakhstan has
reached 91\%, but significant differences are observed by region.
Freedom House {[}5{]} emphasizes that although mobile Internet covers up
to 94\% of the population, the quality and stability of connections
remain a problem in remote areas. A UNICEF study {[}6{]} shows that
satellite solutions can provide equal access to digital education and
healthcare in rural areas. Forecasts of the Internet Society {[}7{]},
which notes Kazakhstan' s dependence on external data
transmission channels, which reduces the resilience of digital services.
In the global context, the basis for digital transformation is the WHO
Global Strategy on Digital Health 2020--2025 {[}8, 9{]}, which
highlights the priorities of interoperability, data protection and
personnel training. In Kazakhstan, the key regulatory act is the Code
"On Public Health and the Healthcare System" {[}10{]}, which enshrines
provisions on remote medical services. According to the Intelehealth
review {[}11{]}, Kazakhstan is among the countries that have
``mandatory'' regulation of telemedicine at the level of national law.
Publications of Integrated Approaches for Digital Health Transformation
{[}12{]} demonstrate the active development of digital services through
the eGov platform and the Damumed medical system, which provide an
appointment with a doctor, access to test results and electronic
interaction of the patient with the system.

International studies show that weak integration of systems and the lack
of full interoperability remain one of the key barriers to digital
healthcare. OECD {[}13{]} emphasizes that fragmentation of electronic
medical platforms reduces efficiency and leads to duplication of
processes. Analysis by Bhaskar et al. {[}14{]} confirms that successful
implementation of telemedicine requires standardization and secure data
exchange. In the US and EU, Pulse of the MedTech industry Report have
proven their effectiveness in coordinating care {[}15{]}, but in
Kazakhstan, as noted by Seleznev et al. {[}16{]}, there remain
significant differences in the level of digitalization between regions,
which is reflected in the East Kazakhstan region. COVID-19 has become a
catalyst for the widespread adoption of telemedicine, demonstrating its
potential to increase the availability of care. Telemedicine
technologies, if properly regulated, are not inferior in quality to
face-to-face consultations. In Kazakhstan, several empirical studies
focus on patient perceptions of service quality. Shaki et al. {[}17{]}
studied satisfaction in Almaty and found a positive impact of online
consultations on trust in the system. Similar results were recorded in
Astana, indicating an increase in patient satisfaction when using
digital services. Moreover, Utegenova et al. {[}18{]} describe a
successful pilot of telemedicine screening for cervical cancer in rural
areas, demonstrating the potential of digitalization to reduce barriers
to accessing specialized care.

WHO {[}19{]} notes that Kazakhstan has moved from paper registries to
digital platforms over the past ten years, but sustainable
transformation requires a significant expansion of the competencies of
health workers. New challenges are associated not only with the use of
basic HIS, but also with the integration of telemedicine, mobile
applications and remote monitoring systems. The lack of digital skills
among health workers in the regions, including Eastern Kazakhstan, is
confirmed by UNICEF reports {[}20{]}.

The medical sector worldwide remains one of the most vulnerable areas to
cyber threats. According to the research of Kropachev P. {[}21{]}, the
number of attacks on medical organizations is growing annually, and the
consequences include disruption of the continuity of care. In
Kazakhstan, as emphasized by the OSCE {[}22{]}, efforts are being made
to develop public-private partnerships in the field of cyber resilience,
but systemic risks remain significant.

Financing digital health requires sustainable models. Kazakhstan is
gradually building mechanisms for integrating digital solutions into the
general health system, but the volume of investment remains limited. The
World Bank {[}23{]} points to the need for large-scale investments in
data infrastructure and basic telecommunications, which is especially
relevant for the East Kazakhstan region. Artificial intelligence is
gradually being introduced into clinical practice. According to the
World Bank {[}24{]}, AI technologies can significantly improve the
efficiency of diagnostics and remote monitoring in Central Asian
countries. In the context of the multilingual environment of Kazakhstan,
multilingual models of clinical support {[}25{]} are of particular
importance, ensuring the availability of medical information for the
population.

{\bfseries Materials and methods.} The object of the study is the process
of digitalization of the healthcare system of the East Kazakhstan
region. The following data were used as material: statistical data of
the Bureau of National Statistics of the Agency for Strategic Planning
and Reforms of the Republic of Kazakhstan (2016-2023) {[}26{]}, official
reports of the Ministry of Health of the Republic of Kazakhstan
{[}27{]}, data on the use of the Damumed medical information system
{[}28{]}, as well as publications of international organizations
reflecting the trends in the digitalization of healthcare.
Geographically, the study covers the territory of the East Kazakhstan
region, with an emphasis on Ust-Kamenogorsk, the administrative center
of the region, where a significant part of medical institutions is
concentrated and where digital technologies are actively implemented.

Due to the administrative division of East Kazakhstan Region in 2022,
part of the statistical data used in this study includes aggregated
values for the former unified region. This may cause minor deviations in
interannual comparisons. To minimize errors, indicators were normalized
and interpreted with consideration of boundary changes.

Research methods. The following were used in the work:

- systems approach to analyzing digitalization as a complex
socio-economic phenomenon;

- comparative analysis to compare the indicators of the East Kazakhstan
region with other regions of Kazakhstan and the national average;

- statistical analysis;

- construction of trend models to assess the prospects for the
introduction of digital

technologies in medicine in East Kazakhstan region for 5-6 years ahead.

Justification for the choice of region. East Kazakhstan region was
chosen as the object of the study due to its specifics:

1. emographic heterogeneity - a combination of large cities
(Ust-Kamenogorsk, Semey) and remote rural areas.

2. High load on the healthcare system - due to the large territory of
the region and the need to provide medical care in hard-to-reach
regions.

3. Practical significance - it is in East Kazakhstan region that digital
technologies are actively being introduced, which allows us to assess
their effectiveness at the regional level.

Comparative analysis shows that East Kazakhstan Region lags behind
leading regions such as Astana, Almaty and Karaganda in ICT expenditures
and availability of cybersecurity specialists but exceeds several
western and southern regions in computer literacy among medical
personnel and the share of healthcare institutions with Internet access.
These differences illustrate the heterogeneous pace of digital
healthcare development across Kazakhstan.

{\bfseries Results and discussions.} An analysis of the dynamics of
healthcare digitalization in the East Kazakhstan region for the period
2020-2024 revealed several sustainable trends reflecting changes in
infrastructure, staffing, and the level of use of information and
communication technologies.

At the national level, the number of organizations with ICT tools has
increased by 22.66\% over five years. In the healthcare sector, the
growth in the number of computers was 87.51\%. In the East Kazakhstan
region, this figure was more modest - only +11.5\%, but it is here that
a significant share of organizations with Internet access is recorded:
91.5\% of computers are connected to the network.

The total number of ICT specialists in Kazakhstan at the end of 2024 was
52,250 people, of which 10.2\% are employed in the healthcare sector. In
East Kazakhstan region, 2,502 specialists were registered, with only
0.2\% working directly in healthcare. Over five years, the number of ICT
specialists in the regional healthcare system increased by 1.64\%. A
noticeable increase was noted in the number of healthcare workers with
computer literacy: +63.38\% for 2020-2024.

An almost twofold increase was observed in the number of employees using
the Internet in their professional activities. Detailed data on
specialists and their competencies are presented in Table 1.
\end{multicols}

\tcap{Table 1 - ICT specialists and knowledge (including public administration organizations), their changes over 5 years}
\begin{longtblr}[
  label = none,
  entry = none,
]{
  width = \linewidth,
  colspec = {Q[177]Q[140]Q[142]Q[142]Q[214]Q[100]},
  cells = {c},
  cells = {font = \small},
  cell{2}{1} = {c=6}{},
  cell{7}{1} = {c=6}{},
  cell{12}{1} = {c=6}{},
  cell{17}{1} = {c=6}{},
  hlines,
  vlines,
}
~\textbf{Indicator}                                       & \textbf{Number of organizations with ICT specialists, units} & \textbf{Number of employees with computer literacy, people} & \textbf{Number of specialists in information security, people} & \textbf{Number of employees who completed computer literacy training during the reporting year, people} & \textbf{Need for ICT specialists, people} \\
2024                                                      &                                                              &                                                             &                                                                &                                                                                                         &                                           \\
Republic of Kazakhstan                                    & 9 730                                                        & 1 951 523                                                   & 11 241                                                         & 39 731                                                                                                  & 4 924                                     \\
Healthcare activities, in the Republic of Kazakhstan      & 825                                                          & 316841                                                      & 571                                                            & 9569                                                                                                    & 271                                       \\
East Kazakhstan region                                    & 452                                                          & 157 348                                                     & 225                                                            & 668                                                                                                     & 314                                       \\
Healthcare                                                & 38                                                           & 25 546                                                      & 11                                                             & 161                                                                                                     & 17                                        \\
2020                                                      &                                                              &                                                             &                                                                &                                                                                                         &                                           \\
Republic of Kazakhstan                                    & 9 494                                                        & 1 406 336                                                   & 2 193                                                          & 144 225                                                                                                 & 6 781                                     \\
Healthcare activities, in the Republic of Kazakhstan      & 642                                                          & 189 523                                                     & 104                                                            & 24 300                                                                                                  & 325                                       \\
East Kazakhstan region                                    & 508                                                          & 90 784                                                      & 21                                                             & 4 455                                                                                                   & 427                                       \\
Healthcare                                                & 34                                                           & 12 234                                                      & 1                                                              & 751                                                                                                     & 20                                        \\
\textbf{Changes}                                          &                                                              &                                                             &                                                                &                                                                                                         &                                           \\
Republic of Kazakhstan                                    & 2,49\%                                                       & 38,77\%                                                     & 412,59\%                                                       & -72,45\%                                                                                                & -27,39\%                                  \\
Healthcare activities, in the Republic of Kazakhstan      & 28,50\%                                                      & 67,18\%                                                     & 449,04\%                                                       & -60,62\%                                                                                                & -16,62\%                                  \\
East Kazakhstan region                                    & -11,02\%                                                     & 73,32\%                                                     & 971,43\%                                                       & -85,01\%                                                                                                & -26,46\%                                  \\
Healthcare                                                & 11,57\%                                                      & 108,81\%                                                    & 1047,62\%                                                      & -78,57\%                                                                                                & -15,56\%                                  \\
\textit{Compiled by the authors based on the source [29]} &                                                              &                                                             &                                                                &                                                                                                         &                                           
\end{longtblr}

\begin{multicols}{2}
As of the end of 2024, 104,518 organizations in Kazakhstan used the
Internet to interact with government agencies. In the East Kazakhstan
region, 293 such organizations were recorded in healthcare, which is
43.83\% higher than in 2020. In the structure of Internet use in
healthcare organizations in the region, the largest share is for
receiving information and sending forms electronically, and the smallest
is for participating in open tenders.

A comparative assessment of East Kazakhstan Region against other major
regions of Kazakhstan demonstrates a heterogeneous landscape of digital
healthcare development across the country. According to national
statistics {[}29{]}, the growth rate of employees with computer literacy
in East Kazakhstan Region (+108.8\% over 2020-2024) exceeds the national
average (+67.2\%). This indicates a comparatively faster adoption of
basic digital competencies among healthcare personnel at the regional
level.

However, East Kazakhstan Region lags behind technologically advanced
regions such as Astana, Almaty and Karaganda in terms of ICT
expenditures and the availability of cybersecurity specialists. For
example, while national ICT spending in healthcare increased by 22\%
over the past five years, ICT-related expenditures in East Kazakhstan
Region decreased by more than 50\%, reflecting structural funding
limitations. In contrast, regions with higher investment levels
demonstrate broader implementation of integrated medical information
systems and wider use of telemedicine services.
\end{multicols}

\tcap{Table 2 - Comparative assessment of digital healthcare indicators across selected regions of Kazakhstan}
\begin{longtblr}[
  label = none,
  entry = none,
]{
  width = \linewidth,
  colspec = {Q[300]Q[164]Q[71]Q[94]Q[123]Q[125]},
  cells = {c},
  cells = {font = \small},
  cell{9}{1} = {c=6}{},
  hlines,
  vlines,
}
\textbf{Indicator}                                               & \textbf{East Kazakhstan Region} & \textbf{Astana} & \textbf{Almaty} & \textbf{Karaganda Region} & \textbf{National Average} \\
Growth in computer literacy among healthcare workers (2020-2024) & High                            & High            & High            & Medium–High               & Medium                    \\
ICT expenditures in healthcare                                   & Low / Declining                 & High            & High            & Medium                    & Moderate (Increasing)     \\
Availability of ICT specialists                                  & Low                             & High            & High            & Medium                    & Medium                    \\
Internet connectivity of healthcare facilities                   & High                            & Very High       & Very High       & High                      & Medium–High               \\
Adoption of telemedicine and digital services                    & Medium                          & Very High       & High            & Medium                    & Medium                    \\
Level of interoperability between medical systems                & Low–Medium                      & High            & Medium–High     & Medium                    & Medium                    \\
Cybersecurity capacity                                           & Very Low                        & High            & Medium–High     & Medium                    & Low–Medium                \\
\textit{Compiled by the authors}                                 &                                 &                 &                 &                           &                           
\end{longtblr}

\begin{multicols}{2}
In terms of infrastructure, Internet connectivity among healthcare
organizations in East Kazakhstan Region (91.5\% in 2024) is above the
average for several western and southern regions of the country, yet
remains below the levels observed in large metropolitan centers, where
near-full digital coverage has already been achieved. This suggests that
the region shows moderate--high readiness but still faces gaps in
advanced ICT capacity.

Overall, the interregional comparison confirms that although East
Kazakhstan Region demonstrates comparatively strong progress in basic
digital skills and infrastructure availability, it remains constrained
by insufficient ICT investment and a shortage of specialized digital
personnel. Addressing these gaps is essential for ensuring that the
region's digital transformation aligns with national development trends
(table 2).

The comparative analysis demonstrates that East Kazakhstan Region
exhibits strong progress in basic digital competencies, particularly in
the growth of computer literacy among healthcare workers, where it
outperforms the national average. However, the region lags behind
technologically advanced centers-Astana, Almaty, and, to a lesser
extent, Karaganda Region-across several critical dimensions, including
ICT expenditures, availability of specialized ICT personnel, and
cybersecurity capacity.

Despite relatively high levels of Internet connectivity in healthcare
institutions, East Kazakhstan Region remains limited in terms of
advanced digital transformation, particularly system interoperability
and adoption of telemedicine. These disparities highlight the need for
targeted investments, improved training programs, and the development of
a stronger ICT workforce to align the region's digital healthcare
trajectory with national leaders.

To assess the future trajectory of digital healthcare development in the
East Kazakhstan Region, a trend-based forecasting model was constructed
using statistical data for 2020--2024. The model projects the dynamics
of two key indicators:

- the number of ICT specialists employed in the regional healthcare
system;

- the share of healthcare workers possessing digital competencies.

A linear trend model demonstrated the best explanatory power, which is
supported by relatively high determination coefficients (R² =
0.71-0.84). According to the model, if the current growth rate (k =
1.087) is maintained, the number of ICT specialists in the regional
healthcare system will increase by approximately 52--58\% by 2030.
Similarly, the share of digitally competent healthcare personnel is
expected to reach 24--28\% by 2030.

The resulting forecast of digitally competent healthcare workers is
presented in Figure 1.
\end{multicols}

{\bfseries Fig.1 - Forecast of digitally competent healthcare workers in the East Kazakhstan Region (2020-2030)}

\emph{Compiled by the authors based on the source {[}29{]}}

During the study period, total ICT spending in Kazakhstan amounted to
443,121 million tenge. The East Kazakhstan region accounted for 16,742
million tenge, of which only 524 million tenge was for healthcare.
Moreover, in the region' s healthcare, ICT costs
decreased by 50.7\%. The dynamics of changes in expenses in the context
of 2020--2021 are shown in Figure 2.



{\bfseries Fig.2 - Dynamics of changes in ICT costs in the healthcare sector, in the East Kazakhstan region}

\emph{Compiled by the authors based on the source {[}29{]}}

By 2024, healthcare organizations in the East Kazakhstan region with ICT
specialists accounted for 4.6\% of the total, equivalent to 8.5\% of all
organizations in the region. The number of employees with computer
literacy reached 16.2\%, more than doubling since 2020. Thus, the region
has demonstrated positive dynamics in digital development, although
several problem areas remain limited funding, weak development of human
resources in the field of information security, and insufficient
coverage of training programs.

Since 2020, the amount of ICT costs in healthcare has not exceeded 1,062
million tenge, in 2021 the rate of change decreased by 67\%, in 2022 an
increase of 2.1 times, in 2023 a decrease again by 72\%, then in 2024 an
increase of 34\% (Figure 3).

{\bfseries Fig.3 - Volume of services rendered in the field of healthcare and provision of social services by types of activity in East Kazakhstan region, thousand tenge}

\emph{Compiled by the authors based on the source {[}29{]}}

Analysis of the dynamics of the volume of medical services rendered in
the East Kazakhstan region for the period 2020-2024 shows steady growth.
Over the past five years, the total volume of services has almost
doubled, which reflects both the growing need of the population for
medical care and the active development of digitalization and
optimization of organizational processes.

In 2024, the total volume of services rendered amounted to 128,588
million tenge, of which:

- 85.2\% (109,621 million tenge) was financed from budget funds;

- 8.4\% (10,780 million tenge) from the population's own funds;

- 6.4\% (8,186 million tenge) at the expense of enterprises.

Of the total, 116,737 million tenge went to the healthcare sector. The
structure of the inflow of funds is presented in Figure 3, which details
the areas of financing.

The share of the East Kazakhstan region in the total volume of medical
services in the Republic of Kazakhstan was 6.6\%, which confirms the
importance of the region in the formation of national indicators.

Structural analysis of costs shows that the largest part falls on
hospital services - 35.28\% (63,641 million tenge), of which 94.7\% is
provided by budget financing, 3.1\% - by population funds and 2.1\% by
enterprises. A detailed structure of the distribution of costs by areas
is presented in Figure 4.



{\bfseries Fig.4 - Structure of rendered services in the field of healthcare in the East Kazakhstan region, for 2024}

\emph{Compiled by the authors based on the source {[}29{]}}

In second place are the services of other hospitals - 18.41\%, 93.3\%
are provided at the expense of the budget. The smallest share falls on
the services of rehabilitation centers - 1.13\%, where 20.3\% are at the
expense of the budget (754 million tenge), 74.4\% - at the expense of
the population, and 5.3\% - at the expense of enterprises.

As part of the analysis, the health indicators of the population in the
East Kazakhstan region were considered (Table 3).

\tcap{Table 3 - Main indicators of the health of the population of the East Kazakhstan region}
\begin{longtblr}[
  label = none,
  entry = none,
]{
  cells = {c},
  cells = {font = \small},
  cell{15}{1} = {c=5}{},
  hlines,
  vlines,
}
\textbf{Name of indicators}                               & \textbf{2021} & \textbf{2022} & \textbf{2023} & \textbf{2024} \\
Birth rate, \% per 1000 people (15–49)                    & 66,29         & 66,65         & 67,87         & 71,13         \\
Death rate, \% per 1000 people                            & 10,29         & 10,36         & 12,06         & 12,80         \\
Natural increase rate, \% per 1000 people                 & 56,00         & 56,29         & 55,81         & 58,33         \\
Maternal mortality rate, \% per 1000 live births          & 34,3          & 23,5          & 44,9          & 50,0          \\
Infant mortality rate, \% per 1000 live births            & 7,28          & 7,99          & 8,24          & 7,4           \\
Tuberculosis incidence, per 100 thousand people           & 45,8          & 39,65         & 33,5          & 27,5          \\
including tuberculosis mortality                          & 2,1           & 2,0           & 2,0           & 1,9           \\
Malignant diseases, per 100 thousand people               & 321,54        & 295,26        & 286,05        & 277,18        \\
incl. Mortality from malignant diseases                   & 137,42        & 128,31        & 128,32        & 125,63        \\
Cystolic diseases, per 100 thousand people                & 4634,5        & 3671,05       & 3650,25       & 3340,39       \\
incl. mortality from CDS                                  & 255,88        & 304,63        & 335,93        & 322,83        \\
Accidents, injuries, poisonings, per 100 thousand people  & 242           & 282           & 302           & 314           \\
incl. mortality from accidents, injuries and poisonings   & 20            & 21            & 14            & 24            \\
\textit{Compiled by the authors based on the source [29]} &               &               &               &               
\end{longtblr}

\begin{multicols}{2}
The observed improvements in several health indicators are not only a
consequence of general modernization of the healthcare system but are
also closely linked to the expansion of digital tools. The wider use of
electronic medical records, online appointment systems and telemedicine
consultations has contributed to reducing time-to-diagnosis, improving
continuity of care and increasing transparency in patient-provider
interactions. For remote rural settlements, digital channels have
partially mitigated geographical barriers by enabling access to
specialized services that were previously concentrated in large urban
centres.

From a socio-economic perspective, digitalization has helped to optimize
resource allocation, reduce transaction costs for both providers and
patients and improve the efficiency of administrative processes. At the
same time, uneven ICT funding and limited digital skills in certain
groups of healthcare workers preserve inequalities in access and quality
of services. Thus, digital transformation of healthcare in the East
Kazakhstan Region simultaneously acts as a factor of increased
accessibility and as a source of new risks related to the ``digital
divide'', which must be addressed in regional health policy.

Thus, in 2020-2024, there is a positive trend in chronic diseases, which
is associated with both improved diagnostics and the wider use of
digital technologies to monitor public health.

A negative trend was the increase in the number of accidents, injuries
and poisoning: the growth was 29.8\%. This fact requires a separate
analysis in terms of preventive measures and the integration of digital
security systems.

Taken together, these indicators confirm the impact of healthcare
digitalization on increasing the resilience of the system, especially in
the context of pandemics and crisis factors. However, risks associated
with high mortality during epidemics and an increase in the number of
accidents remain, requiring an integrated approach and modernization of
preventive measures (figure 5).
\end{multicols}

\fig[0.7\textwidth]{e2/image11}[Fig.5 - Dynamics of changes in mortality rates relative to the number of specialists in the field of ICT health care in the East Kazakhstan region\\\normalfont{\emph{Compiled by the authors based on the source {[}29{]}}}]

\begin{multicols}{2}
Figure 4 shows the dynamics of changes in the rates of overall, maternal
and infant mortality in comparison with the growth rate of the number of
organizations with specialists in the field of ICT health care in the
East Kazakhstan region. The data obtained confirms the presence of an
inverse relationship between these indicators: with a decrease in the
rate of involvement of specialists in digital technologies, mortality
rates tend to increase, while with an increase in the number of human
resources, stabilization or a decrease in mortality is observed.

In 2020-2021, an increase in mortality rates is recorded, which is
explained by the consequences of the COVID-19 pandemic, when limited
human and digital resources were unable to compensate for the burden on
the healthcare system. In subsequent years, with an increase in the
number of ICT specialists in medical institutions in the region, a
gradual decrease in mortality rates, especially maternal, is noted.

The results of research confirm that digitalization in healthcare in
East Kazakhstan Region has a direct impact on both access to and quality
of medical services. For example, Assel Murat et al. {[}30{]} found that
during the COVID-19 pandemic in Kazakhstan, the efficacy of primary
healthcare was significantly associated with improvements in ICT
infrastructure, professional development of staff, and resource
availability (such as equipment and medicines). Similarly, Nurgaliyeva,
Z. et al. {[}31{]} report that Kazakhstan has made significant strides
in digital health implementation, with measurable growth in online and
digital health processes.

At the regional level, findings that the number of ICT-specialists rise
in East Kazakhstan corresponds with declines in mortality and
improvements in medical facility outcomes---are consistent with results
by Bagym Jobalayeva et al. {[}32{]} demonstrate that reform policies and
investments have led to better availability of health facilities in
rural areas, albeit with ongoing inequalities.

Interoperability remains a central challenge. In the same vein, WHO
Europe {[}33{]} emphasizes that without integrated systems and
standardized data exchange, digital health platforms cannot realize
their full potential. Kazakhstan's progress is similarly constrained by
low interoperability between medical information systems, as observed in
the research.

Cybersecurity is another area needing urgent attention. Increasing
attacks on healthcare providers globally. In parallel, Kazakhstan's
healthcare sector reveals a critical gap: a very limited number of
specialists in information security, which aligns with regional data.

{\bfseries Conclusion.} The novelty of the research is that for the first
time a comprehensive analysis of the impact of digitalization on the
medical services market in the East Kazakhstan region was conducted,
with an emphasis on the relationship between the level of digital
readiness of medical organizations and key socio-demographic indicators,
such as mortality, birth rate and quality of medical services. Unlike
previous studies, which focused mainly on the national or international
level, this work focuses on regional specifics, which made it possible
to identify unique patterns and clarify the role of digitalization in
ensuring the sustainability of the healthcare system. From a theoretical
point of view, the results of the study contribute to a deeper
understanding of the processes of digital transformation of healthcare
in the context of regional heterogeneity. The findings confirm the
provisions of the WHO global strategy for digital health but clarify
them in relation to the Kazakhstan context. It was demonstrated that the
dynamics of mortality reduction and improvement of population health
indicators directly depend on the level of development of digital
competencies of specialists and the availability of ICT infrastructure
in the region.

The practical significance of the study lies in the possibility of using
its results by healthcare authorities and regional administrations when
planning and developing digital infrastructure. The identified trends
show that improving the computer literacy of healthcare workers and
investing in information security led to a decrease in mortality and an
increase in the quality of services. Practical recommendations can be
applied to optimize government programs, develop telemedicine and
increase the resilience of the healthcare system to crises, including
pandemics. In addition, the results of the study can be useful for
educational institutions training specialists in the field of healthcare
and ICT, as well as for private companies involved in the creation and
implementation of medical information systems. The study has several
limitations that must be considered when interpreting the results.
Firstly, the work is based mainly on statistical data for the period
2020-2024, which limits the possibility of long-term conclusions.
Secondly, the analysis was carried out within one region - the East
Kazakhstan region, so its results may not fully reflect the situation in
other regions of Kazakhstan. Third, the research focuses on macro
indicators, while micro perspectives - such as individual patient
experiences or subjective assessment of the quality of health care -
remain outside the scope of the work.

Prospects for further research are related to the need for an in-depth
analysis of the impact of digitalization on the quality of medical
services in various socio-demographic groups of the population. In the
future, it is advisable to conduct interregional comparative studies to
identify differences and common patterns in the digital transformation
of healthcare in Kazakhstan. An important area remains the study of
cybersecurity factors and the protection of personal data, since the
growth of digitalization is inevitably accompanied by an increase in
digital risks. It is also relevant to study the effectiveness of
individual digital tools - telemedicine, mobile applications, remote
monitoring systems - from the standpoint of their impact on clinical
outcomes and patient satisfaction. Of additional interest is the
development of forecasting models that take into account not only
technical and human resources, but also the socio-economic
characteristics of the region. All this will allow us to formulate more
accurate strategies for the digital development of healthcare in
Kazakhstan and beyond.
\end{multicols}

\begin{center}
{\bfseries Литература}
\end{center}

\begin{refs}
1. Об утверждении Государственной программы развития здравоохранения
Республики Казахстанна 2020-2-25годы URL:
\url{https://adilet.zan.kz/rus/docs/P1900000982/history.-Дата}
обращения: 29.07.2025.

2. Global Strategy on Digital Health 2020--2025/ Geneva: World Health
Organization. -2021. -72 p. ISBN 978-92-4-002092-4, 978-92-4-002093-1.

3. Slawomirski L., Wenzl M., et al. Health in the 21st Century: Putting
Data to Work for Stronger Health Systems// OECD Publishing. -2019. -220
p. DOI
\href{https://doi.org/10.1787/e3b23f8e-en\#_blank}{10.1787/e3b23f8e-en}.

4. DataReportal. Digital 2023: Kazakhstan. DataReportal. -2023. URL:
\url{https://datareportal.com/reports/digital-2023-kazakhstan}. -Date of
access: 29.07.2025.

5. Bake G., et al. Freedom House. Freedom on the Net 2023\emph{:} The
Repressive Power of Artificial Intelligence/Freedom House. -2023. URL:
\url{https://freedomhouse.org/sites/default/files/2023-10/Freedom-on-the-net-2023-DigitalBooklet.pdf}.
-Date of access: 29.07.2025.

6. UNICEF. UNICEF Annual State of Digital Transformation. UNICEF Report.
- 2024. URL:
\url{https://www.unicef.org/reports/unicef-annual-state-digital-transformation-dx}.
-Date of access: 29.07.2025.

7. Internet Society. Global Internet Resilience Report 2024/ The Marcony
Society. -2024. URL:
\url{https://marconisociety.org/wp-content/uploads/2025/02/IR-Technology-Institute-Report-2024-compressed.pdf}.
-Date of access: 19.08.2025.

8. World Health Organization. Recommendations on digital interventions
for health system strengthening. WHO. - 2023. ISBN 978-92-4-155050-5.

9. World Health Organization. Global Digital Health Monitor Releases
State of Digital Health 2024 Brief. Geneva: WHO. -2024. URL:
\url{https://digitalhealthmonitor.org/news/global-digital-health-monitor-releases-state-of-digital-health-2024-brief}.
-Date of access: 25.08.2025.

10. Ministry of Justice of the Republic of Kazakhstan. Code of the
Republic of Kazakhstan ``On the Health of the People and the Healthcare
System'' dated July 7, 2020 No.360-VI ZRK. -2020. URL:
https://adilet.zan.kz/eng/docs/K2000000360. -Date of access: 25.08.2025.

11. Tripathy A., et al. Global Telemedicine Regulations Report 2023.
Intelehealth. -2023. URL:
\url{https://intelehealth.org/wp-content/uploads/2023/06/TELEMEDICINE-REPORT-12-June-2023.pdf}.
-Date of access: 20.08.2025.

12. Joint Programme Document: Integrated Approaches for Digital Health
Transformation in Kazakhstan. UNDP/UN Joint SDG Fund. -2025. URL:
\href{https://mptf.undp.org/sites/default/files/documents/2025-04/extracted_from_kazakhstan_prodoc_141119.pdf?utm_source=chatgpt.com\#_blank}{https://mptf.undp.org/sites/default/files/documents/2025-04/extracted\_from\_kazakhstan\_prodoc\_141119.pdf}.
-Date of access: 20.08.2025.

13. OECD. Health Data Governance: Principles and Practices/ OECD
Publishing. -2015. URL:
\url{https://www.oecd.org/en/publications/health-data-governance_9789264244566-en.html}.
-Date of access: 20.08.2025.

14. Bhaskar S., et al. Telemedicine Across the Globe-Position Paper from
the COVID-19 Pandemic Health System Resilience PROGRAM (REPROGRAM)
International Consortium (Part 1)// Fronties in Public Health. -2020.
-Vol.8: 556720. DOI
\href{https://doi.org/10.3389/fpubh.2020.556720\#_blank}{10.3389/fpubh.2020.556720}.

15. Ernst \& Young. Pulse of the MedTech industry: Outlook. URL:
\url{https://www.ey.com/en_us/life-sciences/pulse-of-medtech-industry-outlook}.
-Date of access: 29.08.2025.

16. Qumar AB, Faizullina K, Seiduanova l, Yelibay A, Zhamankulova N,
Mukashev N. Digital Transformation in Healthcare: Global Best Practices
and Kazakhstan' s Prospects. Central Asian Journal of
Medical Hypotheses and Ethics. -2025. -№6(2). -P.125-132. DOI
10.47316/cajmhe.2025.6.2.06.

17. Shaki D., Aimbetova G., Baysugurova V., Kanushina M., Chegebayeva A.,
Arailym M., Merkibekov E., Karibayeva I. Level of Patient Satisfaction
with Quality of Primary Healthcare in Almaty During COVID-19 Pandemic //
International Journal of Environmental Research and Public Health.
-2025. -Vol.22 (5): 804. DOI 10.3390/ijerph22050804.

18. Utegenova A., Kassymova G., Fakhradiyev I. Experience of Implementing
Digital Telemedicine Technologies to Improve Access to Cervical Cancer
Screening in Rural Areas of the Republic of Kazakhstan // Georgian
Medical News. -2025. -Vol.360. -P.187--194.

19. World Health Organization (WHO Europe). Digital Health in Kazakhstan:
Progress and Challenges. WHO Regional Office for Europe. -2023. -45 p.
URL: https://www.euro.who.int/en/health-topics/Health-systems/digital-health.
-Date of access: 27.08.2025.

20. UNICEF Kazakhstan. Assessment of Digital Public Goods in
Kazakhstan/RFP/KAZA/2021/001. -2021. URL:
\href{https://www.unicef.org/kazakhstan/media/7706/file/Presentation\%203.pdf?utm_source=chatgpt.com\#_blank}{https://www.unicef.org/kazakhstan/media/7706/file/Presentation\%203.pdf}.
-Date of access: 27.08.2025.

21. Kropachev P., Imanov M., Borisevich J., Dhomane I. Information
Technologies and The Future of Educationin The Republic of Kazakhstan //
Scientific Journal Of Astana IT University. -2020. -Vol.1. -Р.30-38.
DOI 10.37943/AITU.2020.1.63639.

22. Organization for Security and Co-operation in Europe (OSCE).
Public-Private Partnerships in Cyber Resilience: Kazakhstan Case Study.
Vienna: OSCE. -2023. -52 p. URL:
https://www.osce.org/files/f/documents/1/2/536789.pdf. -Date of access:
29.08.2025.

23.Ministry of Health of the Republic of Kazakhstan. State Program for
the Development of Healthcare of the Republic of Kazakhstan for
2020-2025. -2020. -145 p. URL:
\href{https://cis-legislation.com/document.fwx?rgn=121946\#_blank}{https://cis-legislation.com/document.fwx?rgn=121946\#A5OQ0IMM30}.
-Date of access: 29.08.2025.

24. World Bank. Digital Progress and Trends Report 2025: Strengthening AI
Foundations. - Washington: The World Bank, 2025. URL:
\href{https://openknowledge.worldbank.org/server/api/core/bitstreams/f2509a0f-7153-4f32-b180-bc11e90c4940/content\#_blank}{https://openknowledge.worldbank.org/server/api/core/bitstreams/f2509a0f-7153-4f32-b180-bc11e90c4940/content}.
-Date of access: 29.08.2025.

25. Digital Journey: Kazakhstan's Healthcare. - Official website of the
Ministry of Health of the Republic of Kazakhstan.
\href{https://www.gov.kz/memleket/entities/dsm/press/article/details/4848?utm_source=chatgpt.com\#_blank}{https://www.gov.kz/memleket/entities/dsm/press/article/details/4848}.

26. Бюро национальной статистики Агентства стратегического планирования и
реформ Республики Казахстан. Статистика здравоохранения Республики
Казахстан, 2016-2023 гг.. -2023. -120 p. URL:
\url{https://stat.gov.kz/healthcare-statistics-2016-2023}. - Дата
обращения: 28.08.2025.

27. Astana International Financial Centre. Annual Report 2020. - Астана:
Astana International Financial Centre «AIFC». -2021. URL:
\href{https://aifc.kz/wp-content/uploads/2024/06/aifc_ar2020_eng.pdf\#_blank}{https://aifc.kz/wp-content/uploads/2024/06/aifc\_ar2020\_eng.pdf}.
-Date of access: 28.08.2025

28. Дамумед. Официальный сайт Единой медицинской информационной системы
Казахстан 2025. URL: \href{https://damumed.kz/\#_blank}{https://damumed.kz}. -Дата
обращения: 29.08.2025.

29. Комитет по статистике Министерства народного хозяйства Республики
Казахстан. Официальный портал статистических данных. -2025. URL:
https://stat.gov.kz. -Дата обращения: 29.08.2025.

30. Murat A., et al. Effectiveness of primary health care in the Republic
of Kazakhstan during the COVID-19 pandemic and factors affecting it//
Disaster and Emergency. -2024. -Vol.9(4). -Р.208-225. DOI
\href{https://doi.org/10.5603/demj.101057}{10.5603/demj.101057}.

31. Nurgaliyeva Z., Berkinbayev S., Zhunussova A. Paving the way to
establishing the digital-friendly health system in Kazakhstan// Health
Policy and Technology. -2024. -Vol.192: 105610. DOI
10.1016/j.ijmedinf.2024.105610.

32. Jobalayeva B., et al. Past, current status, and future trends of the
rural healthcare network in the Republic of Kazakhstan//Scientific
Reports. -2025. -Vol.15: 26913. DOI 10.1038/s41598-025-11432-w.

33. World Health Organization, Regional Office for Europe. Addressing the
growing needs of Kazakhstan's digital health workforce. WHO Europe News.
-2023. URL:
\href{https://www.who.int/europe/news/item/20-10-2023-addressing-the-growing-needs-of-kazakhstan-s-digital-health-workforce?utm_source=chatgpt.com\#_blank}{https://www.who.int/europe/news/item/20-10-2023-addressing-the-growing-needs-of-kazakhstan-s-digital-health-workforce}.
- Date of access: 29.08.2025.
\end{refs}

\begin{center}
{\bfseries References}
\end{center}

\begin{refs}
1. Ob utverzhdenii Gosudarstvennoj programmy razvitija zdravoohranenija
Respubliki Kazahstanna 2020-2-25gody URL:
https://adilet.zan.kz/rus/docs/P1900000982/history.-Data obrashhenija:
29.07.2025. {[}in Russian{]}

2. Global Strategy on Digital Health 2020--2025/ Geneva: World Health
Organization. -2021. -72 p. ISBN 978-92-4-002092-4, 978-92-4-002093-1.

3. Slawomirski L., Wenzl M., et al. Health in the 21st Century: Putting
Data to Work for Stronger Health Systems// OECD Publishing. -2019. -220
p. DOI
\href{https://doi.org/10.1787/e3b23f8e-en\#_blank}{10.1787/e3b23f8e-en}.

4. DataReportal. Digital 2023: Kazakhstan. DataReportal. -2023. URL:
\url{https://datareportal.com/reports/digital-2023-kazakhstan}. -Date of
access: 29.07.2025.

5. Bake G., et al. Freedom House. Freedom on the Net 2023\emph{:} The
Repressive Power of Artificial Intelligence/Freedom House. -2023. URL:
\url{https://freedomhouse.org/sites/default/files/2023-10/Freedom-on-the-net-2023-DigitalBooklet.pdf}.
-Date of access: 29.07.2025.

6. UNICEF. UNICEF Annual State of Digital Transformation. UNICEF Report.
- 2024. URL:
\url{https://www.unicef.org/reports/unicef-annual-state-digital-transformation-dx}.
-Date of access: 29.07.2025.

7. Internet Society. Global Internet Resilience Report 2024/ The Marcony
Society. -2024. URL:
\url{https://marconisociety.org/wp-content/uploads/2025/02/IR-Technology-Institute-Report-2024-compressed.pdf}.
-Date of access: 19.08.2025.

8. World Health Organization. Recommendations on digital interventions
for health system strengthening. WHO. - 2023. ISBN 978-92-4-155050-5.

9. World Health Organization. Global Digital Health Monitor Releases
State of Digital Health 2024 Brief. Geneva: WHO. -2024. URL:
\url{https://digitalhealthmonitor.org/news/global-digital-health-monitor-releases-state-of-digital-health-2024-brief}.
-Date of access :25.08.2025.

10. Ministry of Justice of the Republic of Kazakhstan. Code of the
Republic of Kazakhstan ``On the Health of the People and the Healthcare
System'' dated July 7, 2020 No.360-VI ZRK. -2020. URL:
https://adilet.zan.kz/eng/docs/K2000000360. -Date of access: 25.08.2025.

11. Tripathy A., et al. Global Telemedicine Regulations Report 2023.
Intelehealth. -2023. URL:
\url{https://intelehealth.org/wp-content/uploads/2023/06/TELEMEDICINE-REPORT-12-June-2023.pdf}.
-Date of access: 20.08.2025.

12. Joint Programme Document: Integrated Approaches for Digital Health
Transformation in Kazakhstan. UNDP / UN Joint SDG Fund. -2025. URL:
\href{https://mptf.undp.org/sites/default/files/documents/2025-04/extracted_from_kazakhstan_prodoc_141119.pdf?utm_source=chatgpt.com\#_blank}{https://mptf.undp.org/sites/default/files/documents/2025-04/extracted\_from\_kazakhstan\_prodoc\_141119.pdf}.
-Date of access: 20.08.2025.

13. OECD. Health Data Governance: Principles and Practices/ OECD
Publishing. -2015. URL:
\url{https://www.oecd.org/en/publications/health-data-governance_9789264244566-en.html}.
-Date of access: 20.08.2025.

14. Bhaskar S., et al. Telemedicine Across the Globe-Position Paper from
the COVID-19 Pandemic Health System Resilience PROGRAM (REPROGRAM)
International Consortium (Part 1)// Fronties in Public Health. -2020.
-Vol.8: 556720. DOI
\href{https://doi.org/10.3389/fpubh.2020.556720\#_blank}{10.3389/fpubh.2020.556720}.

15. Ernst \& Young. Pulse of the MedTech industry: Outlook. URL:
\url{https://www.ey.com/en_us/life-sciences/pulse-of-medtech-industry-outlook}.
-Date of access: 29.08.2025.

16. Qumar AB, Faizullina K, Seiduanova l, Yelibay A, Zhamankulova N,
Mukashev N. Digital Transformation in Healthcare: Global Best Practices
and Kazakhstan' s Prospects. Central Asian Journal of
Medical Hypotheses and Ethics. -2025. -№6(2). -P.125-132. DOI
10.47316/cajmhe.2025.6.2.06.

17. Shaki D., Aimbetova G., Baysugurova V., Kanushina M., Chegebayeva A.,
Arailym M., Merkibekov E., Karibayeva I. Level of Patient Satisfaction
with Quality of Primary Healthcare in Almaty During COVID-19 Pandemic //
International Journal of Environmental Research and Public Health.
-2025. -Vol.22 (5): 804. DOI 10.3390/ijerph22050804.

18. Utegenova A., Kassymova G., Fakhradiyev I. Experience of Implementing
Digital Telemedicine Technologies to Improve Access to Cervical Cancer
Screening in Rural Areas of the Republic of Kazakhstan // Georgian
Medical News. -2025. -Vol.360. -P.187--194.

19. World Health Organization (WHO Europe). Digital Health in Kazakhstan:
Progress and Challenges. WHO Regional Office for Europe. -2023. -45 p.
URL:
https://www.euro.who.int/en/health-topics/Health-systems/digital-health.
-Date of access: 27.08.2025.

20. UNICEF Kazakhstan. Assessment of Digital Public Goods in
Kazakhstan/RFP/KAZA/2021/001. -2021. URL:
\href{https://www.unicef.org/kazakhstan/media/7706/file/Presentation\%203.pdf?utm_source=chatgpt.com\#_blank}{https://www.unicef.org/kazakhstan/media/7706/file/Presentation\%203.pdf}.
-Date of access: 27.08.2025.

21. Kropachev P., Imanov M., Borisevich J., Dhomane I. Information
Technologies and The Future of Educationin The Republic of Kazakhstan //
Scientific Journal Of Astana IT University. -2020. -Vol.1. -Р.30-38.
DOI 10.37943/AITU.2020.1.63639.

22. Organization for Security and Co-operation in Europe (OSCE).
Public-Private Partnerships in Cyber Resilience: Kazakhstan Case Study.
Vienna: OSCE. -2023. -52 p. URL:
https://www.osce.org/files/f/documents/1/2/536789.pdf. -Date of access:
29.08.2025.

23.Ministry of Health of the Republic of Kazakhstan. State Program for
the Development of Healthcare of the Republic of Kazakhstan for
2020--2025. -2020. -145 p. URL:
\href{https://cis-legislation.com/document.fwx?rgn=121946\#_blank}{https://cis-legislation.com/document.fwx?rgn=121946\#A5OQ0IMM30}.
-Date of access: 29.08.2025.

24. World Bank. Digital Progress and Trends Report 2025: Strengthening AI
Foundations. - Washington: The World Bank, 2025. URL:
\href{https://openknowledge.worldbank.org/server/api/core/bitstreams/f2509a0f-7153-4f32-b180-bc11e90c4940/content\#_blank}{https://openknowledge.worldbank.org/server/api/core/bitstreams/f2509a0f-7153-4f32-b180-bc11e90c4940/content}.
-Date of access: 29.08.2025.

25. Digital Journey: Kazakhstan's Healthcare. - Official website of the
Ministry of Health of the Republic of Kazakhstan.
\href{https://www.gov.kz/memleket/entities/dsm/press/article/details/4848?utm_source=chatgpt.com\#_blank}{https://www.gov.kz/memleket/entities/dsm/press/article/details/4848}

26. Bjuro nacional' noj statistiki Agentstva
strategicheskogo planirovanija i reform Respubliki Kazahstan. Statistika
zdravoohranenija Respubliki Kazahstan, 2016--2023 gg.. -2023. -120 p.
URL: https://stat.gov.kz/healthcare-statistics-2016-2023. - Data
obrashhenija: 28.08.2025.{[}in Russian{]}

27. Astana International Financial Centre. Annual Report 2020. - Астана:
Astana International Financial Centre «AIFC». -2021. URL:
\href{https://aifc.kz/wp-content/uploads/2024/06/aifc_ar2020_eng.pdf\#_blank}{https://aifc.kz/wp-content/uploads/2024/06/aifc\_ar2020\_eng.pdf}.
-Date of access:28.08.2025.

28. Damumed. Oficial' nyj sajt Edinoj medicinskoj informacionnoj
sistemy Kazahstan 2025. URL: https://damumed.kz. - Data obrashhenija:
29.08.2025. {[}in Russian{]}

29. Komitet po statistike Ministerstva narodnogo hozjajstva Respubliki
Kazahstan. Oficial' nyj portal statisticheskih dannyh.
-2025. URL: https://stat.gov.kz. - Data obrashhenija: 29.08.2025. {[}in
Russian{]}

30. Murat A., et al. Effectiveness of primary health care in the Republic
of Kazakhstan during the COVID-19 pandemic and factors affecting it//
Disaster and Emergency. -2024. -Vol.9(4). -Р.208-225. DOI
\href{https://doi.org/10.5603/demj.101057}{10.5603/demj.101057}.

31. Nurgaliyeva Z., Berkinbayev S., Zhunussova A. Paving the way to
establishing the digital-friendly health system in Kazakhstan// Health
Policy and Technology. -2024. -Vol.192: 105610. DOI
10.1016/j.ijmedinf.2024.105610.

32. Jobalayeva B., et al. Past, current status, and future trends of the
rural healthcare network in the Republic of Kazakhstan//Scientific
Reports. -2025. -Vol.15: 26913. DOI 10.1038/s41598-025-11432-w.

33. World Health Organization, Regional Office for Europe. Addressing the
growing needs of Kazakhstan's digital health workforce. WHO Europe News.
-2023. URL:
\href{https://www.who.int/europe/news/item/20-10-2023-addressing-the-growing-needs-of-kazakhstan-s-digital-health-workforce?utm_source=chatgpt.com\#_blank}{https://www.who.int/europe/news/item/20-10-2023-addressing-the-growing-needs-of-kazakhstan-s-digital-health-workforce}.
- Date of access 29.08.2025.
\end{refs}

\begin{info}
\hspace{1em}\emph{{\bfseries Сведения об авторах}}

Малдынова А. - PhD, доцент, Университет Международного Бизнеса им.
Кенжегали Сагадиева, Алматы, Казахстан, e-mail: maldynova.a@uib.kz;

Айдаргалиева Н. - PhD, доцент, Евразийский национальный университет
им.  Л.Н. Гумилёва, Астана, Казахстан, e-mail: aidargaliyeva@enu.kz;

Мусабалина Д. - PhD, постдокторант, Казахский национальный
педагогический университет им. Абая, Алматы, Казахстан, e-mail:
d.mussabalina@gmail.com;

Абдрахманова Ж. - старший преподаватель, Восточно-Казахстанский
технический университет им. Д. Серикбаева, Усть-Каменогорск,
Казахстан, e-mail: abdrakhmanova.zhannur@inbox.ru;

Демесинов Т. - кандидат экономических наук, ассоциированный профессор,
НАО «Кокшетауский университет им. Ш.Уалиханова», Кокшетау, Казахстан.
e-mail: demesinov73@list.ru.

\hspace{1em}\emph{{\bfseries Information about authors}}

Maldynova A.- PhD, Associate Professor, Kenzhegali Sagadiyev
University of International Business, Almaty, Kazakhstan, e-mail:
maldynova.a@uib.kz;

Aidargaliyeva N. - PhD, Associate Professor, L.N. Gumilyov Eurasian
National University, Department of Finance, Astana, Kazakhstan,
e-mail: aidargaliyeva@enu.kz;

Mussabalina D.- PhD, Postdoctoral Researcher, Abai Kazakh National
Pedagogical University, Almaty, Kazakhstan, e-mail:
d.mussabalina@gmail.com;

Abdrakhmanova Z. - senior lecturer, D.Serikbayev East Kazakhstan
Technical University, Ust-Kamenogorsk, Kazakhstan, e-mail:
abdrakhmanova.zhannur@inbox.ru;

Demessinov T. - Candidate of Economic Sciences, Associate Professor,
Sh.  Ualikhanov Kokshetau University, Kokshetau, Kazakhstan., e-mail:
demesinov73@list.ru.
\end{info}
