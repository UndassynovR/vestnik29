\id{IRSTI 06.71.45}{}

\begin{header}
\swa{}{ANALYSIS OF THE EFFECT OF CULTURAL ADAPTATION ON ACADEMIC PERFORMANCE}

\tsp{1}D. Satybaldiyeva\envelope,
\tsp{2}B. Keneshbayev,
\tsp{3}R. Tazhibayeva,
\tsp{3}S. Kaltayeva,
\tsp{4}A. Mutaliyeva
\end{header}

\begin{affil}
\tsp{1}Satbayev University, Almaty, Kazakhstan,

\tsp{2}Akhmet Yassawi University, Turkestan, Kazakhstan,

\tsp{3}International University of Tourism and Hospitality, Turkestan, Kazakhstan,

\tsp{4}Regional Innovation University, Shymkent, Kazakhstan

\corrauthor{Correspondent-author: d.satybaldiyeva@satbayev.university}
\end{affil}

Changes and advances in areas such as communication, technology,
transportation, and education have provided many opportunities for
people from different cultures to live together in the same place. There
are many organizational and individual strategies that are used
differently to manage such intercultural interactions and differences. A
new concept at the top of these strategies is cultural intelligence.
This is because people with high levels of cultural intelligence can
adapt quickly to the systems and social life of the country they visit,
which also creates significant advantages in achieving academic success
(being excellent in school, exemplary). In this context, adaptation
processes and levels of achievement affect individuals to a certain
extent. For this reason, this study provides an opportunity to analyze
whether cultural intelligence and cultural adaptation have an impact on
academic performance. Therefore, this research aims to determine the
relationship between cultural intelligence and cultural adaptation and
their impact on academic performance. For this reason, the sample group
of the study consisted only of foreign students studying at the Khoja
Ahmet Yasawi International Kazakh-Turkish University. The study surveyed
a total of 275 international students, with 253 providing complete and
correct responses. The results showed that
participants'{} cultural intelligence levels were higher
than their cultural adaptation levels, and that higher levels of
cultural intelligence were associated with higher academic performance.

{\bfseries Keywords:} academic performance, cultural intelligence,
intercultural interactions, communication, cultural adaptation,
international students.

\begin{header}
АНАЛИЗ ВЛИЯНИЯ КУЛЬТУРНОЙ АДАПТАЦИИ НА АКАДЕМИЧЕСКУЮ УСПЕВАЕМОСТЬ

\tsp{1}Д. Сатыбалдиева\envelope,
\tsp{2}Б. Кенешбаев,
\tsp{3}Р. Тажибаева,
\tsp{3}С. Калтаева,
\tsp{4}А. Муталиева
\end{header}

\begin{affil}
\tsp{1}Университет Сатпаева, Алматы, Казахстан,

\tsp{2}Университет Ахмеда Ясави, Туркестан, Казахстан,

\tsp{3}Международный университет туризма и гостеприимства, Туркестан, Казахстан,

\tsp{4}Региональный инновационный университет, Шымкент, Казахстан,

e-mail: d.satybaldiyeva@satbayev.university
\end{affil}

Изменения и достижения в таких сферах, как коммуникация, технологии,
транспорт и образование, создали множество возможностей для людей из
разных культур жить вместе в одном месте. Существует множество
организационных и индивидуальных стратегий, которые по-разному
применяются для управления такими межкультурными взаимодействиями и
различиями. Одним из новых и ключевых понятий среди этих стратегий
является культурный интеллект. Это объясняется тем, что люди с высоким
уровнем культурного интеллекта способны быстро адаптироваться к системам
и социальной жизни страны, которую они посещают, что создаёт
значительные преимущества в достижении академического успеха (высокая
успеваемость, примерное обучение). В этом контексте процессы адаптации и
уровень достижений в определённой степени влияют на личность. Поэтому
данное исследование предоставляет возможность проанализировать,
оказывают ли культурный интеллект и культурная адаптация влияние на
академическую успеваемость. Таким образом, цель данного исследования --
определить взаимосвязь между культурным интеллектом и культурной
адаптацией и их влияние на академическую успеваемость. В связи с этим
выборка исследования включала только иностранных студентов, обучающихся
в Международном казахско-турецком университете имени Ходжи Ахмета Ясави.
В исследовании приняли участие 275 иностранных студентов, из которых 253
предоставили полные и корректные ответы. Результаты показали, что уровни
культурного интеллекта участников были выше уровней их культурной
адаптации и что более высокий уровень культурного интеллекта был связан
с более высокой академической успеваемостью.

{\bfseries Ключевые слова:} академическая успеваемость, культурный
интеллект, межкультурные взаимодействия, коммуникация, культурная
адаптация, иностранные студенты.

\begin{header}
МӘДЕНИ БЕЙІМДЕЛУДІҢ АКАДЕМИЯЛЫҚ ҮЛГЕРІМГЕ ӘСЕРІН ТАЛДАУ

\tsp{1}Д. Сатыбалдиева\envelope,
\tsp{2}Б. Кенешбаев,
\tsp{3}Р. Тажибаева,
\tsp{3}С. Калтаева,
\tsp{4}А. Муталиева
\end{header}

\begin{affil}
\tsp{1}Сәтбаев университеті, Алматы, Қазақстан,

\tsp{2} Ахмет Ясауи университеті, Түркістан, Қазақстан,

\tsp{3} Халықаралық туризм және меймандостық университеті, Түркістан, Қазақстан,

\tsp{4} Аймақтық инновациялық университеті, Шымкент, Қазақстан,

e-mail: d.satybaldiyeva@satbayev.university
\end{affil}

Коммуникация, технология, көлік және білім беру сияқты салалардағы
өзгерістер мен жетістіктер әртүрлі мәдениеттерден шыққан адамдардың бір
жерде бірге өмір сүруіне көптеген мүмкіндіктер берді. Мұндай
мәдениетаралық өзара әрекеттестіктерді және айырмашылықтарды басқару
үшін түрлі ұйымдық және жеке стратегиялар қолданылады. Бұл
стратегиялардың ішінде жаңа әрі маңызды ұғым - мәдени интеллект. Себебі
мәдени интеллект деңгейі жоғары адамдар барған елдің жүйелері мен
әлеуметтік өміріне тез бейімделе алады, бұл академиялық табысқа (үздік
оқу, үлгілі студент болу) қол жеткізуде елеулі артықшылықтар береді. Осы
тұрғыда бейімделу үдерістері мен жетістік деңгейі жеке тұлғаларға
белгілі бір дәрежеде әсер етеді. Сондықтан бұл зерттеу мәдени интеллект
пен мәдени бейімделудің академиялық үлгерімге әсерін талдауға мүмкіндік
береді. Осыған байланысты зерттеудің мақсаты - мәдени интеллект пен
мәдени бейімделудің арасындағы өзара байланысты және олардың академиялық
үлгерімге ықпалын анықтау. Зерттеу үлгісі Ходжа Ахмет Ясауи атындағы
Халықаралық қазақ-түрік университетінде оқитын шетелдік студенттермен
шектелді. Зерттеуге барлығы 275 шетелдік студент қатысты, олардың 253-і
толық және дұрыс жауаптар берді. Нәтижелер көрсеткендей, қатысушылардың
мәдени интеллект деңгейі олардың мәдени бейімделу деңгейінен жоғары
болды және мәдени интеллект деңгейі жоғары студенттердің академиялық
үлгерімі де жоғары болды.

{\bfseries Түйін сөздер:} академиялық үлгерім, мәдени интеллект,
мәдениетаралық өзара әрекеттестік, коммуникация, мәдени бейімделу,
шетелдік студенттер.

\begin{multicols}{2}
{\bfseries Introduction}. The communication factor, which is the basis of
human life, connects individuals, enterprises and states in
today' s global world. However, this connection also
creates many problems. When people from different cultures meet, work or
study in the same place, and live together, some communication problems
and misunderstandings arise. In order to minimize the adversarial
relationship, that is, misunderstandings, between people or
organizations with different cultures, the problems of adaptation and
interaction that cause them must be solved.

Changes and advances in areas such as communication, technology,
transportation and education provide many opportunities for people from
different cultures to live together. There are many organizational and
individual strategies that are used differently in intercultural
interaction and managing differences. At the top of these strategies is
a new concept - cultural intelligence.

Cultural intelligence is the ability to perceive, understand, interpret,
and assimilate the characteristics of different cultures in order to
manage cultural differences in a multicultural environment with many
cultures. People with a high level of cultural intelligence can quickly
adapt to the systems and social life of the country they visit, which
also creates significant advantages in achieving academic success (being
excellent in school, exemplary). Therefore, this research study examines
the impact of cultural intelligence and cultural adaptation on academic
performance {[}1{]}.

In this regard, the origin of the concept of culture comes from the
Latin word "cultura" and is used in the sense of sowing, harvesting. The
first definition of the concept of culture was given in 1871 by the
British anthropologist Taylor. According to him, "culture is a complex
whole consisting of knowledge, belief, art, morals, customs, and any
other capabilities and habits acquired by people as members of society"
{[}2{]}. Although there is no single opinion regarding the concept of
intelligence, different definitions have been made in each century.
Wesher defines intelligence as ``the general ability of an individual to
think logically and to influence his environment.'' Storddad defines
intelligence as ``the ability of an individual to exhibit mental
behavior that is difficult, complex, abstract, economically relevant,
socially valuable, and has special qualities and to resist excitement''
{[}3{]}.

From the past to the present, cultural values
have had a great influence on the concept of
intelligence, and the factors that constitute intelligence vary from
culture to culture. Considering the cultural structure of Western
societies, the use of intelligence is seen as a sound decision. In the
cultural structure of many African societies, intelligence varies
according to needs and desires. Attitudes and behaviors that are defined
as intelligent at the cultural level of Western societies can be judged
as useless and absurd in different cultures {[}4{]}.

Research on intelligence speaks of three different intellectual
directions that appear in the emergence of the concept of intelligence.
These are social intelligence, emotional intelligence, and cognitive
intelligence. Cultural intelligence, on the other hand, is seen as
different from other areas of intelligence, accepts the differences that
globalization brings to life, and manifests itself in diverse,
multicultural cultural environments and societies {[}5{]}.

The presence of different people and societies, organizations in
intercultural environments for the purpose of education, travel, work or
improvement of living conditions leads to the coexistence of cultural
diversity. In this context, adaptation processes and levels of
achievement affect individuals to a certain extent. For this reason,
this study allows us to analyze whether cultural intelligence and
cultural adaptation have an impact on academic performance. It also
contributes to achieving Sustainable Development Goal 4 (SDG 4) by
examining the impact of cultural intelligence and cultural adaptation on
academic performance. This goal also includes promoting education for
sustainable development. Therefore, this research aims to determine the
relationship between cultural intelligence and cultural adaptation and
their impact on academic performance. It should be noted that in similar
studies, methods such as the Analytic Hierarchy Process (AHP), fuzzy set
theory, key indicator selection methods, and other modern approaches to
multicriteria evaluation are widely used.

However, in this study, a questionnaire survey method was applied, which
made it possible to obtain direct data from respondents The study
examines in detail the impact of sub-dimensions of cultural intelligence
and cultural adaptation on academic performance {[}6-8{]}. The data
collection process during the research lasted 3 months. The sample group
of the study consisted of foreign students studying at the Khoja Ahmet
Yasawi International Kazakh-Turkish University. As part of the study, a
total of 275 foreign students were distributed questionnaires, and 253
people provided complete and correct responses, which were used in the
full research work. One of the reasons why only the Khoja Ahmet Yasawi
International Kazakh-Turkish University was selected in the study is
that this university addresses the need to achieve SDG 4 (ensure
inclusive and equitable quality education and promote lifelong learning
opportunities) by providing access to university-level education,
vocational training, and a variety of necessary skills, and they pay
special attention to equity issues.

{\bfseries Literature review.} People and societies interact with many
different cultures around the world, and the way to succeed in this
interaction is through cultural intelligence.

This section of the research paper reviews the literature on how to
acquire and develop cultural intelligence. Triandis argued that in order
for individuals to have cultural intelligence, they must have certain
qualities. The first is that they must postpone thinking about others
until they have the right information. A lot of information is needed to
make good decisions. Since culture affects behavior and understanding,
communication between cultures and the behavior of people differ in this
regard. Cultures differ from each other in their characteristics. In
this regard, two types of culture can be distinguished. One is
collective culture, and the other is individual culture. Triandis argued
that an individual increases his cultural intelligence by analyzing the
positive and negative features of the culture in which he lives and of
different cultures {[}9{]}.

Earley et al. {[}10{]} argued that cultural intelligence is a concept
that requires training, education, and training in terms of
intelligence, emotion, and behavior. Intellectual education is based on
categorizing individuals using various methods and giving them different
names.

How we classify people around us varies from culture to culture.
Behavioral training allows people to freely engage in their desired
behaviors and actions, minimizing behaviors that are not generally
accepted by the public. Earley and Mosakowski {[}11{]} stated that
people who are able to develop their cultural intelligence are
psychologically healthy and professionally successful. Earley and
Mosakowski {[}12{]} also stated that cultural intelligence can be
developed through a multi-stage approach (Figure 1). The first of these
approaches is to analyze the strengths and weaknesses of individuals in
order to determine how they can develop cultural intelligence. If we
look at the relevant literature on this topic, we can see that various
tools have been developed for the results of this analysis. In this
context, questions have been developed that measure cultural
intelligence differently, and these questions have revealed the
strengths and weaknesses of individuals and formed the basis for the
support and training that the individual should receive in view of this
outcome. In the second stage, training, education, and training methods
are selected to improve the weaknesses of the individual. In the third
stage, education and training aimed at a specific individual or group
begins. In the fourth stage, individuals or groups must organize their
resources and opportunities around themselves to improve their
weaknesses. In the fifth stage, the individual enters the cultural
environment in which he must, must be. In this stage, he puts into
practice what he has learned, demonstrating his strengths in terms of
cultural intelligence and continuing his weaknesses. The final sixth
stage is the stage of evaluation of the new skills acquired or mastered
by the individual. This stage is carried out through personal feedback
from the environment. After these steps, a decision is made whether the
individual needs additional training or not.

\fig[0.3\textwidth]{e2/image5}[Fig.1 - The process of developing cultural intelligence\\\normalfont{\emph{Source: Compiled by the authors based on references {[}11,12{]}}}]
\vspace{-1em}
Thomas and Inkson {[}13{]} stated that experience is more effective than
learning or education in developing cultural intelligence. Thomas and
Inkson {[}13{]} also stated that there are many factors that contribute
to the development of cultural intelligence. They stated {[}14, 15,
16{]}: The first of these factors is that individuals have little
exposure to or interest in other cultures because they are accustomed to
using their own culture. The second factor is that one must be open to
this situation in order to recognize and accept different cultures. The
third stage, they try to accept the behaviors and rules of other
cultures. Here, a detailed understanding of cultural differences begins
to develop. The fourth stage, the rules of different cultures are
adapted to alternative behavior. In the final fifth stage, these people
are able to adapt to changes and differences more easily than other
people in a multicultural environment.

According to Plum, cultural intelligence also encompasses the broader
concept of culture. Culture is relevant to every organization, society,
and individual. Therefore, the development of cultural intelligence is
also important for the following situations {[}17{]}:

- when there is a structural change or merger in an organization, it is
necessary to

achieve harmony between different cultures and create a new
organizational culture;

- structures with different departments should turn working conditions
and methods into advantages and solve problems;

- organizations with cross-cultural diversity should respect differences
and integrate these differences into the business environment;

- in a multicultural environment, organizations with different
structures (age, gender, education, race, etc.) should create effective
cooperation and awareness in their workplaces.

From this perspective, cultural intelligence is the ability to absorb,
perceive, interpret and accept different cultural characteristics in
order to manage cultural differences in an environment where many
different cultures live.

Cultural intelligence is a new phenomenon that has emerged with
globalization. Therefore, we can say that cultural intelligence is a key
ability that managers and employees in the 21st century must have.
Cultural intelligence allows both managers and employees to adapt to
different cultures, make informed and correct decisions, and establish
good relationships.

In this regard, if we dwell in detail on the research conducted in the
field of cultural intelligence, their brief descriptions are presented
in Table 1.
\end{multicols}

\tcap{Table 1 - Research conducted in the field of cultural intelligence}
\begin{longtblr}[
  label = none,
  entry = none,
]{
  width = \linewidth,
  colspec = {Q[10]Q[120]Q[600]},
  column{2} = {c},
  cells = {font = \small},
  cell{1}{1} = {r=3}{c},
  cell{4}{1} = {c=3}{0.94\linewidth},
  cell{5}{1} = {r=3}{c},
  cell{8}{1} = {c=3}{0.94\linewidth},
  cell{9}{1} = {r=3}{c},
  cell{12}{1} = {c=3}{0.94\linewidth},
  cell{13}{1} = {r=3}{c},
  cell{16}{1} = {c=3}{0.94\linewidth},
  vlines,
  hline{1,4-5,8-9,12-13,16-17} = {-}{},
  hline{2-3,6-7,10-11,14-15} = {2-3}{},
}
\textbf{1}                             & \textbf{Purpose of the research}         & Identify the relationship between the five major personality factors and dimensions of cultural intelligence                                                                                                                                                                                                                                                            \\
                                       & \textbf{The result of the research}      & Openness to experience was found to be an important personality factor, related to an individual's ability to interact effectively with people from different cultures. There was a positive relationship between mindfulness and high metacognitive cultural intelligence, and high adaptability was also found to be associated with behavioral cultural intelligence \\
                                       & \textbf{Characteristics of participants} & 338 management undergraduate students participated                                                                                                                                                                                                                                                                                                                      \\
                                       &                                          &                                                                                                                                                                                                                                                                                                                                                                         \\
\textbf{2}                             & \textbf{Purpose of the research}         & Identifying the role of cultural intelligence in intercultural leadership activities and practices                                                                                                                                                                                                                                                                      \\
                                       & \textbf{The result of the research}      & It has been found that a leader's cultural intelligence is important in managing employees who have grown up in different cultural environments, and that the development of cultural intelligence increases in direct proportion to the time they spend living in another culture                                                                                      \\
                                       & \textbf{Characteristics of participants} & 32 Western European managers (including 26 Australian managers) and 19 Chinese managers participated                                                                                                                                                                                                                                                                    \\
                                       &                                          &                                                                                                                                                                                                                                                                                                                                                                         \\
\textbf{3}                             & \textbf{Purpose of the research}         & Determining the relationship between employees' academic performance and cultural intelligence                                                                                                                                                                                                                                                                          \\
                                       & \textbf{The result of the research}      & A positive relationship was found between employees' academic performance and the motivational dimension of cultural intelligence. However, a negative relationship was found between the behavioral dimension of cultural intelligence                                                                                                                                 \\
                                       & \textbf{Characteristics of participants} & Research conducted on declining industrial sectors                                                                                                                                                                                                                                                                                                                      \\
                                       &                                          &                                                                                                                                                                                                                                                                                                                                                                         \\
\textbf{4}                             & \textbf{Purpose of the research}         & Identifying the relationship between cultural intelligence and organizational commitment (dependence)                                                                                                                                                                                                                                                                   \\
                                       & \textbf{The result of the research}      & A positive relationship was found between organizational commitment (dependence) and informational, motivational, and behavioral dimensions of cultural intelligence. It was found that the greater the knowledge about culture and the nature of cultural differences, the higher the commitment (dependence) to the organization                                      \\
                                       & \textbf{Characteristics of participants} & 100 university employees in Iran participated                                                                                                                                                                                                                                                                                                                           \\
\textit{Note:} Compiled by the authors &                                          &                                                                                                                                                                                                                                                                                                                                                                         
\end{longtblr}

\begin{multicols}{2}

Thus, a review of some relevant literature published on the topic of the
study was conducted, and although it is known that many studies have
been conducted on culture, organizations, organizational culture and
organizational structure, it is clearly seen that there has been no
adequate research on cultural intelligence and cultural adaptation and
their impact on academic performance during the screening {[}18{]}.

In this regard, although it is known that scientific research is being
conducted on the concepts and concepts of cultural intelligence and
cultural adaptation in many countries, this does not apply to
Kazakhstan. There are very few local scientific studies on the concepts
of cultural intelligence and cultural adaptation, they only learn and
try to study it from foreign sources. This is because the concepts and
concepts of cultural intelligence and cultural adaptation are very new
and are just beginning to be learned.

{\bfseries Materials and methods.} The presence of people and societies in
multicultural or intercultural environments for learning or teaching,
traveling, working, and other living conditions leads to the coexistence
of cultural diversity. In this context, adaptation, assimilation
processes, and achievement levels affect individuals to different
degrees. For this reason, this research study provides an opportunity to
analyze the impact of cultural intelligence and cultural adaptation on
academic performance based on research findings.

This research study is an important resource or scientific literature,
as it expands the concept of cultural intelligence, which is a new
concept in the literature, and broadly reveals the impact of the
concepts of academic performance and cultural adaptation on cultural
intelligence.

The main hypotheses of the research work were also developed in
accordance with the research objectives, that is, cultural intelligence
as an independent variable and cultural adaptation as a dependent
variable were formed between academic performance (GPA-grade point
average). Therefore, there are two main hypotheses in this research
work. They are:

H\tsb{1}: Cultural intelligence has a significant and
significant effect on academic performance;

H\tsb{2}: Cultural adaptation has a significant and
significant effect on academic performance.

In this regard, there are some limitations of this research work. The
first limitation of the study arises from the selection of samples. The
study participants (sample) were selected from the Khoja Ahmet Yasawi
International Kazakh-Turkish University using the convenience (easy)
sampling method. Although the convenience (easy) sampling method is a
frequently used sampling method, its reliability is lower than other
samples. Therefore, it can be considered a limitation of this study.

Another limitation of the study is the sample size. As part of the
study, a total of 275 foreign students were surveyed, and 253 of them
completed and returned complete and correct responses. Although this
number was sufficient for statistical analysis, more reliable results
could have been achieved with a larger sample size. Another limitation
of the study is time.

The data collection process during the study lasted 3 months. Extending
this period would have undoubtedly increased the sample size, but due to
time constraints, this expansion was not possible.

The final limitation of the study stems from the location of the sample.
The sample consisted of foreign students studying only in the cities of
Turkestan and Kentau. Since a significant proportion of foreign students
are students from different countries, we can say that the reliability
of the results is high.

As mentioned above, the main population and sample of the study
consisted of the foreign students studying at the departments of the
Khoja Ahmet Yasawi International Kazakh-Turkish University in Turkestan
and Kentau (preparatory, bachelor' s,
master' s, doctoral studies). When selecting the research
sample, a convenient (easy) sampling method was used.

The survey method was used in the study. The survey used in the study
consists of three parts. The first part contains 5 questions to
determine the demographic characteristics of the study participants. The
second part of the survey contains a 20-question "Cultural Intelligence"
scale aimed at determining the cultural intelligence of the
participants. The last part of the study contains a 14-item "Cultural
Adaptation" scale used to measure the cultural adaptation of the
participants.

The Cultural Intelligence Scale included in the questionnaire was
developed by Ang and colleagues {[}5{]}. This scale consists of 20 items
(questions) (KZ1...-...KZ20). The scale consists of four dimensions:
``Higher Cognitive (Metacognitive)'', ``Cognitive (Cognitive)'',
``Motivational'' and ``Behavioral'', which allow individuals to
effectively conduct their activities in different cultural environments
or multicultural environments and be successful in that environment. The
scale aims to objectively reveal individuals' perceptions of culture.
The scale used a 5-point Likert method, coded as ``strongly disagree'',
``disagree'', ``unsure'', ``agree'' and ``strongly agree''. Items 1, 2,
3 and 4 of the scale indicate ``Higher Cognitive (Metacognitive)
Intelligence''; Items 5, 6, 7, 8, 9, and 10 measure ``Cognitive
Intelligence''; items 11, 12, 13, 14, and 15 measure ``Motivational
Intelligence''; and items 16, 17, 18, 19, and 20 measure ``Behavioral
Intelligence.'' Levels of cultural intelligence were determined
according to the total score obtained on the scale. An individual with a
high score was considered to have high cultural intelligence. A similar
interpretation is valid for the smaller dimensions of the scale.

The Cultural Adaptation Scale included in the questionnaire was
developed by Black {[}19{]} and Black \& Stephen {[}20{]}. This scale
consists of 14 items (questions) (KA1...-...KA14). The scale consists of
three dimensions: "Physical Adaptation", "Social Adaptation", and "Work
Environment Adaptation". The scale uses a 5-point Likert scale coded as
"completely incompatible", "somewhat compatible", "compatible", and
"completely compatible". Items 1, 2, 3, 5, 6, and 7 of the scale measure
"Physical Adaptation"; items 8, 9, 10, and 11 measure "Social
Adaptation"; and items 11, 12, 13, and 14 measure "Work Environment
Adaptation".

The data obtained from the survey were analyzed using SPSS 23 and AMOS
23 data analysis software. In addition, ``Descriptive Statistics'',
``Confirmatory Factor Analysis'' and ``Structural Equation Modeling''
were used in the analysis of the data and data.

As mentioned above, the main population and sample of the study which
foreign students studying at the International Kazakh-Turkish University
named after Khoja Akhmet Yassawi (preparatory, undergraduate, graduate,
doctoral). The demographic characteristics and characteristics of the
sample are summarized in Table 2.
\end{multicols}

\tcap{Table 2 - Demographic features and characteristics of participants}
\begin{longtblr}[
  label = none,
  entry = none,
]{
  cells = {c},
  cells = {font = \small},
  cell{1}{1} = {c=2}{},
  cell{2}{1} = {r=3}{},
  cell{5}{1} = {r=6}{},
  cell{11}{1} = {r=5}{},
  cell{16}{1} = {r=6}{},
  cell{22}{1} = {c=6}{},
  vlines,
  hline{1-2,5,11,16,22-23} = {-}{},
  hline{3-4,6-10,12-15,17-21} = {2-6}{},
}
                                      &                                  & \textbf{Frequency} & {\textbf{Share}\\\textbf{(\%)}} & {\textbf{Eliglibe Share}\\\textbf{(\%)}} & {\textbf{Total Share}\\\textbf{(\%)}} \\
\textbf{Gender}                       & \textbf{Female}                  & 145                & 57,3                            & 57,3                                     & 57,3                                  \\
                                      & \textbf{Male}                    & 108                & 42,7                            & 42,7                                     & 100,0                                 \\
                                      & \textbf{All}                     & 253                & 100,0                           & 100,0                                    &                                       \\
\textbf{Age}                          & \textbf{18-24}                   & 118                & 46,6                            & 46,6                                     & 46,6                                  \\
                                      & \textbf{25-31}                   & 94                 & 37,2                            & 37,2                                     & 83,8                                  \\
                                      & \textbf{32-38}                   & 31                 & 12,3                            & 12,3                                     & 96,0                                  \\
                                      & \textbf{39-45}                   & 4                  & 1,6                             & 1,6                                      & 97,6                                  \\
                                      & \textbf{46 ≤}                    & 6                  & 2,4                             & 2,4                                      & 100,0                                 \\
                                      & \textbf{All}                     & 253                & 100,0                           & 100,0                                    &                                       \\
\textbf{Country}                      & \textbf{Turkey}                  & 33                 & 13,0                            & 13,0                                     & 13,0                                  \\
                                      & \textbf{Russia, China, Mongolia} & 100                & 39,5                            & 39,5                                     & 52,6                                  \\
                                      & \textbf{Central Asia}            & 106                & 41,9                            & 41,9                                     & 94,5                                  \\
                                      & \textbf{Other}                   & 14                 & 5,5                             & 5,5                                      & 100,0                                 \\
                                      & \textbf{All}                     & 253                & 100,0                           & 100,0                                    &                                       \\
\textbf{GPA}                          & \textbf{1.50 - 1.99}             & 2                  & 0,8                             & 0,8                                      & 0,8                                   \\
                                      & \textbf{2.00 - 2.49}             & 16                 & 6,3                             & 6,3                                      & 7,1                                   \\
                                      & \textbf{2.50 - 2.99}             & 33                 & 13,0                            & 13,0                                     & 20,2                                  \\
                                      & \textbf{3.00 - 3.49}             & 107                & 42,3                            & 42,3                                     & 62,5                                  \\
                                      & \textbf{3.50 - 4.00}             & 95                 & 37,5                            & 37,5                                     & 100,0                                 \\
                                      & \textbf{All}                     & 253                & 100,0                           & 100,0                                    &                                       \\
\textit{Note:} Compiled by the author &                                  &                    &                                 &                                          &                                       
\end{longtblr}

\begin{multicols}{2}
As shown in Table 2, of the foreign students who participated in the
study, 145 (57.3\%) were female and 108 (42.7\%) were male. The majority
of participants (46.6\%) were under 24 years of age.41.9\% of
participants came from Central Asian countries, 39.5\% from Russia,
China, Mongolia, 13.0\% from Turkey, and the remaining 5.5\% were
students from other countries. The study assessed students' academic
performance using the average score. The analysis revealed that a
significant proportion of foreign students (approximately 80\%) who
participated in the study had an average GPA of at least 3. In other
words, the study was conducted with students who were good and very good
students, that is, participants.

As mentioned in the previous section, the ``Cultural Intelligence
Scale'' developed by Ang and colleagues {[}21{]} was introduced and used
in the survey. In the following part of the study, basic statistical
information on cultural intelligence, cultural adaptation and their
subscales: arithmetic means and standard deviations were calculated, and
the results obtained are summarized in Table 3.
\end{multicols}

\tcap{Table 3 - Basic statistical information regarding cultural intelligence, cultural adaptation and their subscales}
\begin{longtblr}[
  label = none,
  entry = none,
]{
  cells = {c},
  cells = {font = \small},
  cell{11}{1} = {c=4}{},
  hlines,
  vlines,
}
                                      & \textbf{N} & \textbf{Arithmetic mean} & \textbf{Standard deviation} \\
Cultural Intelligence                 & 253        & 3,8731                   & 0,49210                     \\
Cultural Adaptation                   & 253        & 3,4593                   & 0,57254                     \\
Metacognitive Intelligence            & 253        & 4,2069                   & 0,69566                     \\
Cognitive Intelligence                & 253        & 3,3887                   & 0,68880                     \\
Motivational Intelligence             & 253        & 4,0731                   & 0,66225                     \\
Behavioral Intelligence               & 253        & 3,8237                   & 0,71052                     \\
Physical Adaptation                   & 253        & 3,4608                   & 0,61407                     \\
Social Adaptation                     & 253        & 3,4585                   & 0,71044                     \\
Work Environment Adaptation           & 253        & 3,4585                   & 0,74752                     \\
\textit{Note:} Compiled by the author &            &                          &                             
\end{longtblr}

\begin{multicols}{2}
When examining Table 3, it was found that the participants had above
average cultural intelligence and cultural adaptation. In general, it
was also observed that the participants' cultural intelligence levels
(3.87) were higher than their cultural adaptation levels. It was also
found that the participants had high levels of metacognitive
intelligence, while their cognitive intelligence levels were relatively
low. It was also observed that all sub-dimensions of cultural adaptation
were at the same level.

In the next part of the study, the relationship between cultural
intelligence, which was used as one of the main variables in the study,
and metacognitive, cognitive, motivational, and behavioral
intelligences, which are sub-dimensions of cultural intelligence, and
academic performance, was examined through correlation analysis. The
results obtained are presented in Table 4 below.

As can be seen from Table 4, positive and significant relationships were
found between the cultural intelligence of the foreign students
participating in the study and their academic performance (grade point
average) in all subscales except the subscale called cognitive
intelligence (p<0.05). In other words, the increase in the
cultural intelligence of the foreign students participating in the study
also increases their grade point average (grade point average).

Another key variable used in the study was cultural adaptation and the
relationship between the subscales of cultural adaptation, physical
adaptation, social adaptation, and work environment adaptation, and
academic performance, which were also examined through correlation
analysis, and the results are summarized in Table 5 below.

As can be seen from Table 5, there were positive and significant
relationships (p<0.05) between the cultural adaptation of the
foreign students participating in the study and all subscales of
cultural adaptation and their academic performance (average grades). In
other words, the increase in cultural adaptation of the foreign students
participating in the study also increases their average grades
(progress).

In this section of the study, the relationships between cultural
intelligence and its subscales and cultural adaptation and its subscales
were analyzed. The results obtained as a result of the correlation
analysis are summarized in Table 6.
\end{multicols}

\tcap{Table 4 - Results of correlation analysis on cultural intelligence, its subscales, and academic performance}
\begin{longtblr}[
  label = none,
  entry = none,
]{
  cells = {c},
  cells = {font = \small},
  cell{8}{1} = {c=8}{},
  hlines,
  vlines,
}
№                                     & \textbf{Name}              & \textbf{1} & \textbf{2} & \textbf{3} & \textbf{4} & \textbf{5} & \textbf{6} \\
1                                     & Average score              & 1          &            &            &            &            &            \\
2                                     & Cultural intelligence      & ,216**     & 1          &            &            &            &            \\
3                                     & Metacognitive intelligence & ,178**     & ,700**     & 1          &            &            &            \\
4                                     & Cognitive intelligence     & ,035       & ,684**     & ,310**     & 1          &            &            \\
5                                     & Motivational intelligence  & ,185**     & ,752**     & ,451**     & ,278**     & 1          &            \\
6                                     & Behavioral intelligence    & ,217**     & ,721**     & ,241**     & ,362**     & ,441**     & 1          \\
\textit{Note:} Compiled by the author &                            &            &            &            &            &            &            
\end{longtblr}
\vspace{-1em}
\hspace{1em}{\small *p<0,05 **p<0,01}


\tcap{Tablе 5 - Results of correlation analysis regarding cultural adaptation, its subscales, and academic performance}
\begin{longtblr}[
  label = none,
  entry = none,
]{
  cells = {c},
  cells = {font = \small},
  cell{8}{1} = {c=8}{},
  hlines,
  vlines,
}
№                                     & \textbf{Name}              & \textbf{1} & \textbf{2} & \textbf{3} & \textbf{4} & \textbf{5} & \textbf{6} \\
1                                     & Average score              & 1          &            &            &            &            &            \\
2                                     & Cultural intelligence      & ,216**     & 1          &            &            &            &            \\
3                                     & Metacognitive intelligence & ,178**     & ,700**     & 1          &            &            &            \\
4                                     & Cognitive intelligence     & ,035       & ,684**     & ,310**     & 1          &            &            \\
5                                     & Motivational intelligence  & ,185**     & ,752**     & ,451**     & ,278**     & 1          &            \\
6                                     & Behavioral intelligence    & ,217**     & ,721**     & ,241**     & ,362**     & ,441**     & 1          \\
\textit{Note:} Compiled by the author &                            &            &            &            &            &            &            
\end{longtblr}
\vspace{-1em}
\hspace{1em}{\small *p<0,05 **p<0,01}

\tcap{Table 6 - Correlation analysis between cultural intelligence and its subscales and cultural adaptation and its subscales}
\begin{longtblr}[
  label = none,
  entry = none,
]{
  cells = {c},
  cells = {font = \small},
  cell{11}{1} = {c=11}{},
  hlines,
  vlines,
}
№                                     & \textbf{Name}               & \textbf{1} & \textbf{2} & \textbf{3} & \textbf{4} & \textbf{5} & \textbf{6} & \textbf{7} & \textbf{8} & \textbf{9} \\
1                                     & Cultural Intelligence       & 1          &            &            &            &            &            &            &            &            \\
2                                     & Cultural Adaptation         & ,339**     & 1          &            &            &            &            &            &            &            \\
3                                     & Metacognitive Intelligence  & ,700**     & ,173**     & 1          &            &            &            &            &            &            \\
4                                     & Cognitive Intelligence      & ,684**     & ,311**     & ,310**     & 1          &            &            &            &            &            \\
5                                     & Motivational Intelligence   & ,752**     & ,219**     & ,451**     & ,278**     & 1          &            &            &            &            \\
6                                     & Behavioral Intelligence     & ,721**     & ,266**     & ,241**     & ,362**     & ,441**     & 1          &            &            &            \\
7                                     & Physical Adaptation         & ,298**     & ,816**     & ,173**     & ,238**     & ,174**     & {,262*\\*} & 1          &            &            \\
8                                     & Social Adaptation           & ,271**     & ,817**     & ,111*      & ,260**     & ,180**     & {,222*\\*} & {,507*\\*} & 1          &            \\
9                                     & {Work Environment \\Adaptation} & ,278**     & ,851**     & ,149*      & ,272**     & ,188**     & {,185*\\*} & {,572*\\*} & {,510\\**} & 1          \\
\textit{Note:} Compiled by the author &                             &            &            &            &            &            &            &            &            &            
\end{longtblr}
\vspace{-1em}
\hspace{1em}{\small *p<0,05 **p<0,01}

\begin{multicols}{2}
As can be seen from Table 6, positive and significant relationships were
also found between cultural intelligence and all its subscales and
cultural adaptation and all its subscales (p<0.05).

The information obtained after the structural model analysis performed
with the AMOS program to test the main hypotheses of the study is shown
in Table 7.
\end{multicols}

\tcap{Table 7 - Relationships and coefficients in a structural model}
\begin{longtblr}[
  label = none,
  entry = none,
]{
  cells = {c},
  cells = {font = \small},
  cell{4}{1} = {c=7}{},
  hlines,
  vlines,
}
                                      &      &                       & \textbf{Prediction coefficient} & \textbf{Standard error} & \textbf{C.R.} & \textbf{P} \\
Average grade                         & <--- & Cultural Intelligence & ,306                            & ,119                    & 2,561         & ,010       \\
Average grade                         & <--- & Cultural Adaptation   & ,233                            & ,103                    & 2,266         & ,023       \\
\textit{Note:} Compiled by the author &      &                       &                                 &                         &               &            
\end{longtblr}

\begin{multicols}{2}
As shown in Table 7, the analysis revealed a significant relationship
(effect) between all variables at the p<0.05 significance level.
The analysis revealed a positive and significant relationship (ß=0.306)
between cultural intelligence and academic performance (average grade)
and a positive and significant relationship (ß=0.233) between cultural
adaptation and academic performance (average grade). In other words, the
main hypotheses of the study, H\tsb{1} and H\tsb{2},
were accepted.

{\bfseries Results and discussion.} In general, this research paper
examined in detail the concepts of cultural intelligence and cultural
adaptation, which can be used as a new strategy for managing differences
in intercultural communication. In other words, this work was conducted
to identify and explain the relationship between cultural intelligence,
which is a new concept in the field of intercultural communication and
adaptation, and cultural adaptation on academic performance. The results
of this study showed how important the concepts of cultural intelligence
and cultural adaptation are. Also, as a result of the study, we can say
that in order to successfully establish and successfully continue
intercultural communication, it is necessary to develop and effectively
use the phenomenon of cultural intelligence. People with high cultural
intelligence can easily solve problems that may arise from different
cultures and successfully conduct the processes arising from such
interactions. In the future, people who will succeed in different
cultures will be those who increase their levels of cultural
intelligence.

For this reason, people should effectively master and use the phenomenon
of cultural intelligence. Therefore, this research paper is an important
resource or scientific literature, because it expands the concept of
cultural intelligence, which is a new concept in the literature, and
widely reveals the impact of the concepts of academic performance and
cultural adaptation on cultural intelligence.

If we focus on the results of the study individually, it was found that
the cultural intelligence and cultural adaptation of the foreign
students who participated in the study were above average. In general,
it was also observed that the cultural intelligence levels of the
participants were higher than their cultural adaptation levels. In
addition, it was found that the participants had high levels of
metacognitive intelligence, while their cognitive intelligence levels
were relatively low. It was also observed that all subscales of cultural
adaptation were at the same level.

However, positive and significant relationships were found between the
cultural intelligence of the foreign students who participated in the
study and their academic performance (average grades) in all subscales
except the subscale called cognitive intelligence (p<0.05). In
other words, an increase in the cultural intelligence of the foreign
students who participated in the study also increases their average
grades (progress). However, a positive and significant relationship was
found between the cultural adaptation of the foreign students
participating in the study and all subscales of cultural adaptation and
their academic performance (grade point average) (p<0.05). In
other words, the increase in the cultural adaptation of the foreign
students participating in the study also increases their grade point
average (progress).

The overall analysis revealed a positive and significant relationship
(ß=0.306) between cultural intelligence and academic performance
(average grade) and a positive and significant relationship (ß=0.233)
between cultural adaptation and academic performance (average grade). In
other words, the main hypotheses of the study, H1 and H2, were accepted.

It was also noted that the results obtained were consistent with the
results of previous similar studies. Therefore, the results of this
study will guide future research in the field of cultural intelligence
and provide the following recommendations:

- individuals will be more effective in adapting to cultural differences
outside their own culture when interacting with people from that
culture.

- an important element in intercultural interaction and an important
factor in ensuring successful integration is the language knowledge of
that culture. For this reason, participants should be fluent in the
language of the culture they live in.

- future studies can investigate the impact of cultural intelligence on
an individual' s personality and how
people' s presence in different cultural environments can
affect their personality. Therefore, in future studies, the dimensions
of cultural intelligence can be considered separately.

{\bfseries Conclusion.} This study empirically examined the effect of
cultural intelligence and cultural adaptation on the academic
performance of international students studying at Khoja Akhmet Yassawi
International Kazakh-Turkish University. The findings demonstrate that
cultural intelligence - particularly its metacognitive and motivational
dimensions - serves as a strong predictor of students' academic success,
whereas the cognitive dimension shows a comparatively weaker influence.
Similarly, adaptation to the physical, social, and academic/work
environment emerged as a significant determinant of academic outcomes,
indicating that adjustment challenges extend beyond cultural knowledge
and encompass students' everyday interactions, emotional comfort, and
integration into the learning ecosystem.

These results are broadly consistent with previous studies emphasizing
the critical role of cultural adaptation in educational attainment. For
instance, research by Earley \& Mosakowski (2004) {[}12{]}, Templer et
al. (2006) {[}14{]}, and Berry (2005) {[}4{]} similarly report that
higher cultural adjustment is associated with lower acculturative stress
and better academic outcomes among international students. Furthermore,
our findings regarding the metacognitive and motivational components of
cultural intelligence align with Ang et al. (2007) {[}5{]}, who argue
that students capable of regulating cultural knowledge and maintaining
motivation in diverse environments are more likely to succeed
academically. The weaker influence of the cognitive dimension also
echoes earlier empirical studies demonstrating that mere cultural
knowledge is insufficient without the ability to apply it effectively in
real social contexts.

Beyond theoretical alignment, the results have important practical
implications for higher education management. Universities hosting
international students must recognize that cultural adaptation is not an
automatic process but a developmental trajectory requiring systematic
support. Intercultural communication training, language assistance,
mentorship programs, and socially inclusive campus environments can
substantially improve students' adjustment experiences. Academic
advisors and faculty should also be trained to recognize cultural
barriers that may hinder performance and provide culturally responsive
pedagogical strategies.

Moreover, strengthening institutional support systems contributes not
only to individual student performance but also to broader policy goals.
Enhancing cultural adaptation aligns with Sustainable Development Goal 4
by ensuring equitable, inclusive, and quality education for diverse
student groups. As higher education becomes increasingly
internationalized, universities that invest in cultural intelligence
development and structured adaptation programs will be better positioned
to attract, retain, and empower global learners.

In conclusion, the study reaffirms that cultural intelligence and
cultural adaptation are foundational to the academic success of
international students. Future research could further explore
longitudinal effects, mediating variables such as acculturative stress
or social support, and comparative analyses across different cultural or
institutional contexts. Such extensions would deepen our understanding
of how multicultural learning environments can be designed to unlock the
full academic potential of international student populations.
\end{multicols}

\begin{center}
{\bfseries References}
\end{center}

\begin{refs}
1. . Tang L., Zhang Ch., Cui Y. Beyond borders: The effects of perceived
cultural distance, cultural intelligence, cross-cultural adaptation on
academic performance among international students of higher education
// International Journal of Intercultural Relations. -2024. -Vol.103:
102083. DOI 10.1016/j.ijintrel.2024.102083.

2. . Robert B. A Formal Concept of Culture in the Classification of
Alfred L. Kroeber and Clyde Kluckhohn // Analecta. -2016. -Vol. XXV.
-P.61--101.

3. Fiori M., Vesely-Maillefer A.K. Emotional Intelligence as an Ability:
Theory, Challenges, and New Directions // Emotional Intelligence in
Education. -2018. -P.23-47. DOI 10.1007/978-3-319-90633-1\_2.

4. Berry J.W., Colleen W. Commentary on Redefining Interactions Across
Cultures and Organizations // Group \& Organization Management. -2006.
-Vol.31(1). -P.64-78. DOI 10.1177/1059601105275264.

5. Ang S., Van Dyne L., Koh C., Ng K.Y., Templer K.J., Tay C.,
Chandrasekar N.A. Cultural Intelligence: Its Measurement and Effects
on Cultural Judgment and Decision Making, Cultural Adaptation and Task
Performance // Management and Organization Review. -2007. -Vol.3 (3).
-P.335-371. DOI 10.1111/j.1740-8784.2007.00082.x.

6. Khoja Ahmet Yassawi International Kazakh--Turkish University.
University website. -- URL:
\url{https://ayu.edu.kz/en/2795/surdurulebilir-kalkinma-hedefleri}.
--Date of access: 02.08.2025.

7. Satybaldiyeva D., Mukhanova G., Tymbayeva Zh., Tyshkanbayeva M.,
Bolatkyzy S. Applying the expert method to determine a company's
strategic goals // Transport Problems. -2023. -Vol.18 (2). -P.
123--132. DOI 10.20858/tp.2023.18.2.11

8. Tymbayeva Zh., Satybaldiyeva D., Kakitayeva Zh., Kazhmuratova A.
Evaluation of the Feasibility of Maintaining Sustainable Consumer
Behavior Among Generation Z Youth in Almaty// Eurasian Journal of
Economic and Business Studies. -2024. -Vol.68 (4). -P.5--17. DOI
10.47703/ejebs.v68i4.428.

9. Triandis H.C. Cultural Intelligence in Organisations // Group \&
Organisation Management. -2006. -Vol.31 (1). -P.20--26. DOI
10.1177/1059601105275253.

10. Earley P.Ch., Ang S., Tan J.S. CQ: Developing Cultural Intelligence at
Work //California: Stanford University Press. -2006. DOI
\href{http://dx.doi.org/10.1515/9781503619715}{10.1515/9781503619715}.

11. Earley P.Ch., Mosakowski E. Cultural intelligence // Harvard Business
Review. -2004. -Vol.82 (10). -P.139--146: 158. DOI
10.1007/978-3-8349-8724-2\_4.

12. Earley P.Ch., Mosakowski E. Toward culture intelligence: turning
cultural differences into a workplace advantage //Academy of
Management Perspectives. -2004. -Vol.18 (3). -P.151--157. DOI
10.5465/AME.2004.28561784.

13. Thomas D.C., Inkson K. Cultural Intelligence: People skills for a
global workplace/ Berrett-Koehler Publishers. -2004. -240 p. ISBN-13
978-1576752562.

14. Templer K.J., Cheryl T., Anand N.Ch. Motivational Cultural
Intelligence, Realistic Job Preview, Realistic Living Conditions
Preview, and Cross-cultural Adjustment // Group \& Organization
Management. -2006. -Vol.31 (1). -P.154--173. DOI
10.1177/1059601105275293.

15. Burhanuddin B., Ben F., Supriyanto A., Sunandar A., Sunarni S.,
Sumarsono R.B. Effects of university organizational culture on student
academic behavior in Indonesia // International Journal of Educational
Management. -2024. -Vol.38 (2). -P.549--567. DOI
10.1108/IJEM-11-2023-0553.

16. Yeşil S. Kültürel Farklılıkların Yönetimi ve Alternatif Bir Strateji:
Kültürel Zeka// Karamanoğlu Mehmetbey Üniversitesi Sosyal Ve Ekonomik
Araştırmalar Dergi. -2009. -Vol.11 (16). -P.100--131. {[}in
Turkish{]}

17. Moran R.T., Harris P.R., Moran S.V. Managing Cultural Differences:
Global Leadership Skills and Knowledge for the 21st Century, 8th
ed./Routledge. -2011. -586 p. DOI
\url{https://doi.org/10.4324/9781856179249}.

18. Sheng L., Dai J., Lei J. The impacts of academic adaptation on
psychological and sociocultural adaptation among international
students in China: The moderating role of friendship // International
Journal of Intercultural Relations. -2022. -Vol.8(6). -P.79--89. DOI
10.1016/j.ijintrel.2022.06.001.

19. Black J. Work Role Transitions: A Study of American Expatriate
Managers in Japan // Journal of International Business Studies. -1988.
-Vol.19. -P.277--294. DOI 10.1057/palgrave.jibs.8490383.

20. Black J.S., Stephens G.K. The influence of the spouse on American
expatriate adjustment and intent to stay in Pacific Rim overseas
assignments // Journal of Management. -1989. -Vol.15 (4). -P.
529--544. DOI 10.1177/014920638901500403.

21. Ang S., Van Dyne L., Koh C. Personality Correlates of the Four-Factor
Model of Cultural Intelligence // Group \& Organization Management.
-2006. -Vol.31 (1). -P.100--123. DOI 10.1177/1059601105275267.
\end{refs}

\begin{info}
\hspace{1em}\emph{{\bfseries Сведения об авторах}}

Сатыбалдиева Д.О. - PhD, ассоциированный профессор, Казахский
национальный исследовательский технический университет им.К.И.Сатпаева,
Алматы, Казахстан, e-mail:d.satybaldiyeva@satbayev.university;

Кенешбаев Б.Ж. - Ph.D, Старший преподаватель, Международный
казахско-турецкий университет им. Ходжи Ахмеда Ясави, Туркестан,
Казахстан, e-mail:
keneshbayev\_bektur@ayu.edu.kz;

Тажибаева Р.М. - к.э.н., ассоциированный профессор (доцент) Школа
гостеприимства, Международный университет туризма и гостеприимства,
Туркестан, Казахстан, e-mail raihan.tazhibaeva@iuth.edu.kz;

Калтаева С.А. - к.э.н., ассоциированный профессор (доцент) Школа
гостеприимства, Международный университет туризма и гостеприимства,
Туркестан, Казахстан, e-mail
saule.kaltaeva@iuth.edu.kz;

Муталиева А.А. - PhD, Старший преподаватель, Региональный инновационный
университет, Шымкент, Казахстан, е-mail:
Alua012@mail.ru.

\hspace{1em}\emph{{\bfseries Information about the authors}}

Satybaldiyeva D.O. - PhD, Associate Professor, K.I. Satpayev Kazakh
National Research Technical University, Almaty, Kazakhstan, e-mail:
d.satybaldiyeva@satbayev.university;

Keneshbayev B.Zh. - Ph.D., Senior Lecturer, Khoja Akhmet Yassawi
International Kazakh--Turkish University, Turkestan, Kazakhstan, e-mail:
keneshbayev\_bektur@ayu.edu.kz;

Tazhibayeva R.M. - Candidate of Economic Sciences, Associate Professor
(Docent), School of Hospitality, International University of Tourism and
Hospitality, Turkestan, Kazakhstan, e-mail:
raihan.tazhibaeva@iuth.edu.kz;

Kaltaeva S.A. - Candidate of Economic Sciences, Associate Professor
(Docent), School of Hospitality, International University of Tourism and
Hospitality, Turkestan, Kazakhstan, e-mail: saule.kaltaeva@iuth.edu.kz;

Mutaliyeva A.A. - PhD, Senior Lecturer, Regional Innovation University,
Shymkent, Kazakhstan, е-mail: Alua012@mail.ru.
\end{info}
